\documentclass[12pt]{article} 
\usepackage{amsmath} 
\usepackage[dvips]{graphicx}
\usepackage{multirow} 
\usepackage{geometry} 
\usepackage{pdflscape}
\usepackage[labelfont=bf]{caption} 
\usepackage{setspace}
\usepackage[running]{lineno} 
% \usepackage[numbers,sort]{natbib}
\usepackage[round]{natbib} 
\usepackage{array}
\usepackage[table]{xcolor}

\topmargin -1.5cm % 0.0cm 
\oddsidemargin 0.0cm % 0.2cm 
\textwidth 6.5in
\textheight 9.0in % 21cmhttps://www.overleaf.com/project/6048001a3ac22e3aff2ced4a
\footskip 1.0cm % 1.0cm

\newenvironment{refquote}{\bigskip \begin{it}}{\end{it}\medskip}

\usepackage{authblk}

\begin{document}

% Due June 18.

\section*{Reply to Editor}

    \begin{refquote}

    Dear Dr Alyssa Cirtwill

    Re: Species motif participation provides unique information about species risk of extinction.  Cirtwill, Alyssa; Åkesson, Anna; Wootton, Katherine; Eklöf, Anna.

    Thank you for submitting your manuscript to Journal of Animal Ecology.

    I have now received reviewers' reports and the Associate Editor's comments on your manuscript and looked at it myself. As you can see from their comments below, a few final revisions are required.

    I suspect you can deal with these minor revisions fairly easily, and I look forward to seeing your revised manuscript by 17-Mar-2024. If you foresee any problems with meeting this deadline, please contact the editorial office on: admin@journalofanimalecology.org

    * Please respond to all the comments of the reviewers and Associate Editor, which are set out below, by changes in the manuscript and point by point in your response.  Please ensure you insert your response in the correct 'Response to decision letter' section when you submit your revision online. Please ensure that you complete this thoroughly so that it is clear to the editors and reviewers how you have improved your manuscript. Please insert your point by point response in the correct section of the submission process in the Authors Response to Comments text box.  Please bear in mind that in the Author Response to Comments text box has limited formatting (no italics/ bold), so please distinguish your answers clearly.

    *Please upload a Word document for use by our Publisher. PDF files are not acceptable.


    Yours sincerely
    Professor Darren Evans
    Editor, Journal of Animal Ecology

    \end{refquote}

    \textbf{R:}

\clearpage


\section*{Reply to Associate Editor}

    \begin{refquote}
    Dear Alyssa Cirtwill et al,

    The revised manuscript has been seen by a reviewer and myself and we agree that all the comments have been addressed correctly. However, there a few minor points left to address.

    Best wishes,
    Daniel
    \end{refquote}

    - The Supplementary material is not uploaded, maybe due to some problems or errors during the submission process?
    \textbf{R:} We have now uploaded a corrected Supplementary material file and apologise for the error in the previous submission. We are not sure how that occurred.
    - Line 68: ‘when’ is written twice
    \textbf{R:} Corrected.
    - Line 358-360: This sentence is not clear. Please rewrite
    \textbf{R:} We have revised the sentence (as well as the preceding and following sentences) and hope that the end of this paragraph is now clearer.

    Revised sentence:
    \begin{quotation}
    This is consistent with other studies showing context-dependent effects of omnivory~\citep{Bascompte2005,Monteiro2016}.
    In highly-connected networks, species in omnivory motifs are very likely to interact with other species in the network (outside the omnivory motif).
    These `external' interactions can stabilise omnivory motifs that would be unstable in isolation~\citep{Kratina2012}, perhaps mitigating the risk of participating in omnivory rather than motifs that are more stable in isolation~\citep{Borelli2015}.
    Moreover, omnivory tends to be particularly beneficial when interaction strengths are weak, which is generally the case in highly-connected networks~\citep{Emmerson2004}.
    These earlier findings suggest that high connectance reduces the risk and increases the benefit of participating in omnivory motifs, consistent with our simulation results.

    \end{quotation}

    Original lines 354-362:

    \begin{quotation}
    This is consistent with other studies showing context-dependent effects of omnivory~\citep{Bascompte2005,Monteiro2016} and particularly beneficial effects when interaction strengths are weak (as will generally be the case at high connectance)~\citep{Emmerson2004}. Although omnivory motifs tend to be less stable in isolation the the other motifs we consider here~\citep{Borrelli2015}, interactions with other species can stabilise even intrinsically unstable omnivory when embedded in a network~\citep{Kratina2012}. As well as leading to weaker average interaction strengths, a higher connectance offers many opportunities for additional interactions that could stabilise unstable omnivory.
    \end{quotation}


\clearpage


\section*{Reply to reviewer 1}

    \begin{refquote}

        This revision is very well done, and the authors responses to reviewer comments are comprehensive. The changes made by the authors have greatly improved the clarity of the manuscript. The new Fig 4 (now Fig 3) offers a much better means of comparing the distribution of slopes across disturbance and connectance levels within each motif type.

        In the uploaded documents it seems like the supplement "motif_participation_SI.pdf" contains  another version of the paper rather than the Appendix, and I did not see the supplementary materials in any of the files.

    \end{refquote}

    \textbf{R:} We thank the Reviewer for their careful comments on the previous submission which greatly improved the manuscript. We have double-checked the Appendix and uploaded the correct version and apologise for the error.


\clearpage

\clearpage

    \bibliographystyle{jae} 
    \bibliography{anna_bib_new} % Do not abbreviate journal titles, papers with >

\end{document}