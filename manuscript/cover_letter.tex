
\documentclass[10.5pt]{letter}

\usepackage[bitstream-charter]{mathdesign}
\usepackage{microtype}
\usepackage{graphicx}
\usepackage[letterpaper, margin=0.9in, top=0.4in, bottom=0.4in]{geometry}

\longindentation=30pt

\address{Alyssa Cirtwill \\ University of Helsinki \\ Spatial Foodweb Ecology Group \\ Helsinki, Finland}

\date{October XX, 2022}


\begin{document}

\begin{letter}{
\vspace{-2.5cm}

Editorial Board\\
Journal of Animal Ecology} % Name/title of the addressee


\opening{Dear Editor,}

We invite you to consider the enclosed manuscript ``Species motif participation provides unique information about species risk of extinction'' for publication in the \emph{Journal of Animal Ecology} as an Article. 

Ecological effects of disturbances in food webs are known to depend on global (e.g. food web connectance) and local (e.g. trophic level) network properties. Disturbances to the basal level of a food web have a particularly strong impact on the persistence of all species. 
However, the locations for secondary extinctions are difficult to identify as both direct and indirect interactions among species are important. As global network properties are too coarse-grained to include interactions and local network properties disregard any indirect effects, we must consider other structural properties in order to fully understand secondary extinctions. Meso-scale network properties, such as smaller sub-graphs within a large network (motifs), offer a potentially valuable middle-ground approach. Motifs have been shown to be important for community-level stability, but how motifs relate to risk of secondary extinctions is so far unexplored.   

Here, we use a Bayesian network approach to simulate different levels of disturbance in complex food webs.
In this framework, we analyse the relationships between consumers' participation in different three-species motifs and extinction risks, under the influence of different levels of disturbance to the basal species. 
Moreover, we analyze whether the motif profile of the full food web affects the extinction risk of consumer species.
This extra layer of analysis places motifs as a bridge between species-level and network-level properties affecting extinction risk.
Several novel key results emerge.

First, we show that a species' participation in different motifs is clearly related to its probability of extinction. Interestingly, this relationship is strongly influenced by the level of the disturbance, which indicates important 'tipping points' where increased level of disturbance can turn a previously beneficial characteristic into a risk factor. 
Second, we demonstrate that relationships between motif participation and extinction risk synthesize the information provided by simpler local properties, such as trophic level and in-degree. 
Third, we show that the overall motif profile of a network is related to the average persistence of consumer species, capturing the same information as network size and connectance. 

As we take an important step towards understanding of how secondary effects propagate in complex food webs, we believe our method and results will interest a wide audience of ecologists including both theoreticians and empiricists. We therefore think the \emph{Journal of Animal Ecology} would be an ideal platform for our research.

We thank you for your time and attention dedicated to our work.


\vspace{2\parskip}

\hspace{4\parskip} Sincerely,

\hspace{4\parskip} Alyssa R. Cirtwill, Anna {\AA}kesson,  Kate Wootton, and Anna Ekl\"of

\vspace{4\parskip}
\centering
\includegraphics[width=60mm]{HY_logo_Eng.eps}


\end{letter} 
\end{document}

