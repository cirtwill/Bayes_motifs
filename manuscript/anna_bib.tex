@article {Ceballosetal2017,
	author = {Ceballos, Gerardo and Ehrlich, Paul R. and Dirzo, Rodolfo},
	title = {Biological annihilation via the ongoing sixth mass extinction signaled by vertebrate population losses and declines},
	volume = {114},
	number = {30},
	pages = {E6089--E6096},
	year = {2017},
	doi = {10.1073/pnas.1704949114},
	publisher = {National Academy of Sciences},
	abstract = {The strong focus on species extinctions, a critical aspect of the contemporary pulse of biological extinction, leads to a common misimpression that Earth{\textquoteright}s biota is not immediately threatened, just slowly entering an episode of major biodiversity loss. This view overlooks the current trends of population declines and extinctions. Using a sample of 27,600 terrestrial vertebrate species, and a more detailed analysis of 177 mammal species, we show the extremely high degree of population decay in vertebrates, even in common {\textquotedblleft}species of low concern.{\textquotedblright} Dwindling population sizes and range shrinkages amount to a massive anthropogenic erosion of biodiversity and of the ecosystem services essential to civilization. This {\textquotedblleft}biological annihilation{\textquotedblright} underlines the seriousness for humanity of Earth{\textquoteright}s ongoing sixth mass extinction event.The population extinction pulse we describe here shows, from a quantitative viewpoint, that Earth{\textquoteright}s sixth mass extinction is more severe than perceived when looking exclusively at species extinctions. Therefore, humanity needs to address anthropogenic population extirpation and decimation immediately. That conclusion is based on analyses of the numbers and degrees of range contraction (indicative of population shrinkage and/or population extinctions according to the International Union for Conservation of Nature) using a sample of 27,600 vertebrate species, and on a more detailed analysis documenting the population extinctions between 1900 and 2015 in 177 mammal species. We find that the rate of population loss in terrestrial vertebrates is extremely high{\textemdash}even in {\textquotedblleft}species of low concern.{\textquotedblright} In our sample, comprising nearly half of known vertebrate species, 32\% (8,851/27,600) are decreasing; that is, they have decreased in population size and range. In the 177 mammals for which we have detailed data, all have lost 30\% or more of their geographic ranges and more than 40\% of the species have experienced severe population declines (\&gt;80\% range shrinkage). Our data indicate that beyond global species extinctions Earth is experiencing a huge episode of population declines and extirpations, which will have negative cascading consequences on ecosystem functioning and services vital to sustaining civilization. We describe this as a {\textquotedblleft}biological annihilation{\textquotedblright} to highlight the current magnitude of Earth{\textquoteright}s ongoing sixth major extinction event.},
	issn = {0027-8424},
	URL = {https://www.pnas.org/content/114/30/E6089},
	eprint = {https://www.pnas.org/content/114/30/E6089.full.pdf},
	journal = {Proceedings of the National Academy of Sciences}
}


@article {Diazetal2019,
	author = {D{\'\i}az, Sandra and Settele, Josef and Brond{\'\i}zio, Eduardo S. and Ngo, Hien T. and Agard, John and Arneth, Almut and Balvanera, Patricia and Brauman, Kate A. and Butchart, Stuart H. M. and Chan, Kai M. A. and Garibaldi, Lucas A. and Ichii, Kazuhito and Liu, Jianguo and Subramanian, Suneetha M. and Midgley, Guy F. and Miloslavich, Patricia and Moln{\'a}r, Zsolt and Obura, David and Pfaff, Alexander and Polasky, Stephen and Purvis, Andy and Razzaque, Jona and Reyers, Belinda and Chowdhury, Rinku Roy and Shin, Yunne-Jai and Visseren-Hamakers, Ingrid and Willis, Katherine J. and Zayas, Cynthia N.},
	title = {Pervasive human-driven decline of life on Earth points to the need for transformative change},
	volume = {366},
	number = {6471},
	elocation-id = {eaax3100},
	year = {2019},
	doi = {10.1126/science.aax3100},
	publisher = {American Association for the Advancement of Science},
	abstract = {For decades, scientists have been raising calls for societal changes that will reduce our impacts on nature. Though much conservation has occurred, our natural environment continues to decline under the weight of our consumption. Humanity depends directly on the output of nature; thus, this decline will affect us, just as it does the other species with which we share this world. D{\'\i}az et al. review the findings of the largest assessment of the state of nature conducted as of yet. They report that the state of nature, and the state of the equitable distribution of nature{\textquoteright}s support, is in serious decline. Only immediate transformation of global business-as-usual economies and operations will sustain nature as we know it, and us, into the future.},
	issn = {0036-8075},
	URL = {https://science.sciencemag.org/content/366/6471/eaax3100},
	eprint = {https://science.sciencemag.org/content/366/6471/eaax3100.full.pdf},
	journal = {Science}
}

@article{binzer2011susceptibility,
  title={The susceptibility of species to extinctions in model communities},
  author={Binzer, Amrei and Brose, Ulrich and Curtsdotter, Alva and Ekl{\"o}f, Anna and Rall, Bj{\"o}rn C and Riede, Jens O and de Castro, Francisco},
  journal={Basic and Applied Ecology},
  volume={12},
  number={7},
  pages={590--599},
  year={2011},
  publisher={Elsevier}
}

@article{giling2019plant,
  title={Plant diversity alters the representation of motifs in food webs},
  author={Giling, Darren P and Ebeling, Anne and Eisenhauer, Nico and Meyer, Sebastian T and Roscher, Christiane and Rzanny, Michael and Voigt, Winfried and Weisser, Wolfgang W and Hines, Jes},
  journal={Nature communications},
  volume={10},
  number={1},
  pages={1--7},
  year={2019},
  publisher={Nature Publishing Group}
}

@article{scherber2010bottom,
  title={Bottom-up effects of plant diversity on multitrophic interactions in a biodiversity experiment},
  author={Scherber, Christoph and Eisenhauer, Nico and Weisser, Wolfgang W and Schmid, Bernhard and Voigt, Winfried and Fischer, Markus and Schulze, Ernst-Detlef and Roscher, Christiane and Weigelt, Alexandra and Allan, Eric and others},
  journal={Nature},
  volume={468},
  number={7323},
  pages={553--556},
  year={2010},
  publisher={Nature Publishing Group}
}

@article{proulx2010diversity,
  title={Diversity promotes temporal stability across levels of ecosystem organization in experimental grasslands},
  author={Proulx, Rapha{\"e}l and Wirth, Christian and Voigt, Winfried and Weigelt, Alexandra and Roscher, Christiane and Attinger, Sabine and Baade, Jussi and Barnard, Romain L and Buchmann, Nina and Buscot, Fran{\c{c}}ois and others},
  journal={PLoS one},
  volume={5},
  number={10},
  pages={e13382},
  year={2010},
  publisher={Public Library of Science}
}

@article{prill2005dynamic,
  title={Dynamic properties of network motifs contribute to biological network organization},
  author={Prill, Robert J and Iglesias, Pablo A and Levchenko, Andre},
  journal={PLoS Biol},
  volume={3},
  number={11},
  pages={e343},
  year={2005},
  publisher={Public Library of Science}
}

@article{bascompte2005simple,
  title={Simple trophic modules for complex food webs},
  author={Bascompte, Jordi and Meli{\'a}n, Carlos J},
  journal={Ecology},
  volume={86},
  number={11},
  pages={2868--2873},
  year={2005},
  publisher={Wiley Online Library}
}

@article{curtsdotter2011robustness,
  title={Robustness to secondary extinctions: comparing trait-based sequential deletions in static and dynamic food webs},
  author={Curtsdotter, Alva and Binzer, Amrei and Brose, Ulrich and de Castro, Francisco and Ebenman, Bo and Ekl{\"o}f, Anna and Riede, Jens O and Thierry, Aaron and Rall, Bj{\"o}rn C},
  journal={Basic and Applied Ecology},
  volume={12},
  number={7},
  pages={571--580},
  year={2011},
  publisher={Elsevier}
}

@article{dunne2009cascading,
  title={Cascading extinctions and community collapse in model food webs},
  author={Dunne, Jennifer A and Williams, Richard J},
  journal={Philosophical Transactions of the Royal Society B: Biological Sciences},
  volume={364},
  number={1524},
  pages={1711--1723},
  year={2009},
  publisher={The Royal Society London}
}

@article{cirtwill2018feeding,
  title={Feeding environment and other traits shape species’ roles in marine food webs},
  author={Cirtwill, Alyssa R and Ekl{\"o}f, Anna},
  journal={Ecology letters},
  volume={21},
  number={6},
  pages={875--884},
  year={2018},
  publisher={Wiley Online Library}
}

@article{Rosenblatt2016,
title = {Climate Change, Nutrition, and Bottom-Up and Top-Down Food Web Processes},
journal = {Trends in Ecology & Evolution},
volume = {31},
number = {12},
pages = {965-975},
year = {2016},
issn = {0169-5347},
doi = {https://doi.org/10.1016/j.tree.2016.09.009},
url = {https://www.sciencedirect.com/science/article/pii/S0169534716301744},
author = {Adam E. Rosenblatt and Oswald J. Schmitz},
keywords = {carbon dioxide, herbivore, plant, predator, temperature, water},
abstract = {Climate change ecology has focused on climate effects on trophic interactions through the lenses of temperature effects on organismal physiology and phenological asynchronies. Trophic interactions are also affected by the nutrient content of resources, but this topic has received less attention. Using concepts from nutritional ecology, we propose a conceptual framework for understanding how climate affects food webs through top-down and bottom-up processes impacted by co-occurring environmental drivers. The framework integrates climate effects on consumer physiology and feeding behavior with effects on resource nutrient content. It illustrates how studying responses of simplified food webs to simplified climate change might produce erroneous predictions. We encourage greater integrative complexity of climate change research on trophic interactions to resolve patterns and enhance predictive capacities.}
}
@article{Eklofetal2013,
author = {Eklöf, Anna and Tang, Si and Allesina, Stefano},
title = {{Secondary extinctions in food webs: a Bayesian network approach}},
journal = {Methods in Ecology and Evolution},
volume = {4},
number = {8},
issn = {2041-210X},
url = {http://dx.doi.org/10.1111/2041-210X.12062},
doi = {10.1111/2041-210X.12062},
pages = {760--770},
keywords = {Bayesian networks, biodiversity loss, cascading extinctions, dynamical model, food webs},
year = {2013},
}

@book{Jensen_Nielsen,
Author = {Jensen, Finn V. and Nielsen, Thomas Dyhre},
ISBN = {9780387682822},
Publisher = {New York : Springer, c2007.},
Series = {Information science and statistics},
Title = {Bayesian networks and decision graphs. [Elektronisk resurs].},
URL = {https://login.e.bibl.liu.se/login?url=https://search.ebscohost.com/login.aspx?direct=true&AuthType=ip,uid&db=cat00115a&AN=lkp.627114&lang=sv&site=eds-live&scope=site},
Year = {2007},
}

@article{Tarjan1972,
author = {Tarjan, Robert},
title = {Depth-First Search and Linear Graph Algorithms},
journal = {SIAM Journal on Computing},
volume = {1},
number = {2},
pages = {146-160},
year = {1972},
doi = {10.1137/0201010},
}

@article{Olesen_origPD,
author = {Kristian G. Olesen and Uffe Kjaerulff and Frank Jensen and Finn V. Jensen and Bjørn Falck and Steen Andreassen and Stig K. Andersen},
title = {A Munin Network for the Median Nerve - A Case Study on Loops},
journal = {Applied Artificial Intelligence},
volume = {3},
number = {2-3},
pages = {385-403},
year = {1989},
doi = {10.1080/08839518908949933},
URL = {http://dx.doi.org/10.1080/08839518908949933    
},
eprint = { 
        http://dx.doi.org/10.1080/08839518908949933    
},
}

@article{Allesinaetal2005,
author = {Allesina, Stefano and Bodini, Antonio and Bondavalli, Cristina},
title = {Ecological subsystems via graph theory: the role of strongly connected components},
journal = {Oikos},
volume = {110},
number = {1},
pages = {164-176},
abstract = {In this paper we investigate ecological flow networks via graph theory in search of the real sequential chains through which energy passes from producers to consumers in complex food webs. We obtain such fundamental pathways by identifying strongly connected components (SCCs), subsystems that groups species that take part in cycling, and performing topological sorting on the acyclic graphs that are obtained. Topological sorting identifies preferential directions for energy to flow from sources to sinks, while recycling remains confined within each SCC. Resolving food web networks for SCC highlights the possibility that compartments can be found in ecosystems, but this does not seem a general rule. The four aquatic food webs described in detail show a rather clear subdivision between benthic and pelagic subcommunities, a result that is discussed in the light of other studies. Should further research confirm these results, new insight into the way ecosystems use energy will be provided, with implications on cycling, reciprocal dependency of variables and indirect effects.},
year = {2005}
}



@INPROCEEDINGS{Rohrbein_CfPD, 
author={F. Rohrbein and J. Eggert and E. Korner}, 
booktitle={2009 International Joint Conference on Neural Networks}, 
title={Child-friendly divorcing: Incremental hierarchy learning in Bayesian networks}, 
year={2009}, 
pages={2711-2716}, 
doi={10.1109/IJCNN.2009.5178995}, 
ISSN={2161-4393}, 
month={June},}

@INPROCEEDINGS{Rohrbein_Waldow,
author={U. von Waldow and F. Rohrbein},
title={{Structure learning in Bayesian networks with parent divorcing}},
year={2015},
journal={EAPCogSci 2015},
}

@article{Mathieu2020,
abstract = {Environmental DNA (eDNA) is becoming a standard tool in environmental monitoring that aims to quantify spatiotemporal variation for the measurement and prediction of ecosystem change. eDNA surveys have complex workflows encompassing multiple decision-making steps in which uncertainties can accumulate due to field sampling design, molecular biology lab work, and bioinformatics analyses. We conducted a quantitative review of studies published prior to December 2017 (n = 431) that had sampled eDNA from a variety of ecosystems and that had explicitly accounted for variability and uncertainty associated with eDNA workflows, either in their study design (e.g., replication) or data analysis (e.g., statistically modeling the spatiotemporal variation). We recorded differences among research studies in their spatial and temporal study design, the detected scales of natural variation in the study taxa, and how researchers measured and addressed the multiple sources of variability and uncertainty associated with the eDNA workflow. We show that relatively few studies used eDNA to understand temporal variation in biodiversity compared to spatial variation, and fewer described how uncertainties were addressed. We recommend increasing the number of temporal studies and to account for both natural variation and sources of uncertainty, such as imperfect detection, when undertaking eDNA surveys. Of studies that quantified spatiotemporal variation, this review identified gaps in the scales over which researchers have observed these patterns. Increasing the number of long-term and broad-scale eDNA studies will improve understanding of how useful eDNA is at scales relevant for monitoring the effects of environmental changes such as climatic shifts or land use change. Even where sources of spatiotemporal variation and uncertainty were accounted for, the effort in quantifying this variation differed among the different steps in the eDNA process, from field, to laboratory and bioinformatics procedures, depending on the type of community studied (micro- vs. macro-organism communities). We recommend more consistent experimental and modeling methods, accounting for spatiotemporal variation, and uncertainty in eDNA collection, and analysis, and incorporation of prior knowledge of sources of variability via Bayesian modeling approaches to account for uncertainties such as imperfect detection, to generate robust diversity estimates and increase the comparability of eDNA datasets for environmental monitoring across space and time.},
author = {Mathieu, Chlo{\'{e}} and Hermans, Syrie M. and Lear, Gavin and Buckley, Thomas R. and Lee, Kevin C. and Buckley, Hannah L.},
doi = {10.3389/fevo.2020.00135},
file = {:Users/alyssacirtwill/Downloads/fevo-08-00135.pdf:pdf},
issn = {2296701X},
journal = {Frontiers in Ecology and Evolution},
keywords = {Bioinformatics,Community,EDNA,Experimental design,Single taxon,Spatiotemporal scale,Uncertainty,Variability},
number = {May},
pages = {1--14},
title = {{A systematic review of sources of variability and uncertainty in eDNA data for environmental monitoring}},
volume = {8},
year = {2020}
}
@article{Matesanz2019,
abstract = {Most work on plant community ecology has been performed above ground, neglecting the processes that occur in the soil. DNA metabarcoding, in which multiple species are computationally identified in bulk samples, can help to overcome the logistical limitations involved in sampling plant communities belowground. However, a major limitation of this methodology is the quantification of species' abundances based on the percentage of sequences assigned to each taxon. Using root tissues of five dominant species in a semi-arid Mediterranean shrubland (Bupleurum fruticescens, Helianthemum cinereum, Linum suffruticosum, Stipa pennata and Thymus vulgaris), we built pairwise mixtures of relative abundance (20{\%}, 50{\%} and 80{\%} biomass), and implemented two methods (linear model fits and correction indices) to improve estimates of root biomass. We validated both methods with multispecies mixtures that simulate field-collected samples. For all species, we found a positive and highly significant relationship between the percentage of sequences and biomass in the mixtures (R2 =.44–.66), but the equations for each species (slope and intercept) differed among them, and two species were consistently over- and under-estimated. The correction indices greatly improved the estimates of biomass percentage for all five species in the multispecies mixtures, and reduced the overall error from 17{\%} to 6{\%}. Our results show that, through the use of post-sequencing quantification methods on mock communities, DNA metabarcoding can be effectively used to determine not only species' presence but also their relative abundance in field samples of root mixtures. Importantly, knowledge of these aspects will allow us to study key, yet poorly understood, belowground processes.},
author = {Matesanz, Silvia and Pescador, David S. and P{\'{i}}as, Beatriz and S{\'{a}}nchez, Ana M. and Chac{\'{o}}n-Labella, Julia and Illuminati, Angela and de la Cruz, Marcelino and L{\'{o}}pez-Angulo, Jes{\'{u}}s and Mar{\'{i}}-Mena, Neus and Vizca{\'{i}}no, Ant{\'{o}}n and Escudero, Adri{\'{a}}n},
doi = {10.1111/1755-0998.13049},
file = {:Users/alyssacirtwill/Downloads/matesanz2019.pdf:pdf},
issn = {17550998},
journal = {Molecular Ecology Resources},
keywords = {DNA metabarcoding,Mediterranean shrubland,coexistence,mock communities,plant abundance,rbcL region,root biomass,sequence},
number = {5},
pages = {1265--1277},
pmid = {31232514},
title = {{Estimating belowground plant abundance with DNA metabarcoding}},
volume = {19},
year = {2019}
}
@article{Makiola2019,
abstract = {Plant pathogens such as rust fungi (Pucciniales) are of global economic and ecological importance. This means there is a critical need to reliably and cost-effectively detect, identify, and monitor these fungi at large scales. We investigated and analyzed the causes of differences between next-generation sequencing (NGS) metabarcoding approaches and traditional DNA cloning in the detection and quantification of recognized species of rust fungi from environmental samples. We found significant differences between observed and expected numbers of shared rust fungal operational taxonomic units (OTUs) among different methods. However, there was no significant difference in relative abundance of OTUs that all methods were capable of detecting. Differences among the methods were mainly driven by the method's ability to detect specific OTUs, likely caused by mismatches with the NGS metabarcoding primers to some Puccinia species. Furthermore, detection ability did not seem to be influenced by differences in sequence lengths among methods, the most appropriate bioinformatic pipeline used for each method, or the ability to detect rare species. Our findings are important to future metabarcoding studies, because they highlight the main sources of difference among methods, and rule out several mechanisms that could drive these differences. Furthermore, strong congruity among three fundamentally different and independent methods demonstrates the promising potential of NGS metabarcoding for tracking important taxa such as rust fungi from within larger NGS metabarcoding communities. Our results support the use of NGS metabarcoding for the large-scale detection and quantification of rust fungi, but not for confirming the absence of species.},
author = {Makiola, Andreas and Dickie, Ian A. and Holdaway, Robert J. and Wood, Jamie R. and Orwin, Kate H. and Lee, Charles K. and Glare, Travis R.},
doi = {10.1002/mbo3.780},
file = {:Users/alyssacirtwill/Downloads/mbo3.780.pdf:pdf},
issn = {20458827},
journal = {MicrobiologyOpen},
keywords = {Illumina,Ion Torrent,Pucciniales,cloning,next-generation sequencing,plant pathogens},
number = {7},
pages = {e780},
pmid = {30585441},
title = {{Biases in the metabarcoding of plant pathogens using rust fungi as a model system}},
volume = {8},
year = {2019}
}
@article{Deiner2018,
abstract = { The analysis of environmental DNA (eDNA) using metabarcoding has increased in use as a method for tracking biodiversity of ecosystems. Little is known about eDNA in marine human-modified environments, such as commercial ports, which are key sites to monitor for anthropogenic impacts on coastal ecosystems. To optimise an eDNA metabarcoding protocol in these environments, seawater samples were collected in a commercial port and methodologies for concentrating and purifying eDNA were tested for their effect on eukaryotic DNA yield and subsequent richness of Operational Taxonomic Units (OTUs). Different filter materials [Cellulose Nitrate (CN) and Glass Fibre (GF)], with different pore sizes (0.5 µm, 0.7 µm and 1.2 µm) and three previously published liquid phase extraction methods were tested. The number of eukaryotic OTUs detected differed by a factor of three amongst the method combinations. The combination of CN filters with phenol-chloroform-isoamyl alcohol extractions recovered a higher amount of eukaryotic DNA and OTUs compared to GF filters and the chloroform-isoamyl alcohol extraction method. Pore size was not independent of filter material but did affect the yield of eukaryotic DNA. For the OTUs assigned to a highly successful non-indigenous species, Styelaclava , the two extraction methods with phenol significantly outperformed the extraction method without phenol; other experimental treatments did not contribute significantly to detection. These results highlight that careful consideration of methods is warranted because choice of filter material and extraction method create false negative detections of marine eukaryotic OTUs and underestimate taxonomic richness from environmental samples. },
author = {Deiner, Kristy and Lopez, Jacqueline and Bourne, Steve and Holman, Luke and Seymour, Mathew and Grey, Erin K. and Lacoursi{\`{e}}re, Ana{\"{i}}s and Li, Yiyuan and Renshaw, Mark A. and Pfrender, Michael E. and Rius, Marc and Bernatchez, Louis and Lodge, David M.},
doi = {10.3897/mbmg.2.28963},
file = {:Users/alyssacirtwill/Downloads/MBMG{\_}article{\_}28963{\_}en{\_}1.pdf:pdf},
issn = {2534-9708},
journal = {Metabarcoding and Metagenomics},
keywords = {0,18s ribosomal,access article distributed under,al,and reproduction in any,and source,cc by 4,copyright kristy deiner et,creative commons attribution license,distribution,edna,high-throughput-sequencing,medium,metazoan eukaryotes,non-indigenous species,provided the original author,seawater,the terms of the,this is an open,which permits unrestricted use},
pages = {1--15},
title = {{Optimising the detection of marine taxonomic richness using environmental DNA metabarcoding: the effects of filter material, pore size and extraction method}},
volume = {2},
year = {2018}
}
@article{Gauzens2019,
abstract = {Understanding how changes in biodiversity will impact the stability and functioning of ecosystems is a central challenge in ecology. Food web approaches have been advocated to link community composition with ecosystem functioning by describing the fluxes of energy among species or trophic groups. However, estimating such fluxes remain problematic because current methods become unmanageable as network complexity increases. We developed a generalization of previous indirect estimation methods assuming a steady-state system (Hunt et al.,); the model estimates energy fluxes in a top-down manner assuming system equilibrium; each node's losses (consumption and physiological) balances its consumptive gains. Jointly, we provide theoretical and practical guidelines to use the fluxweb R package (available on CRAN at https://cran.rproject.org/web/packages/fluxweb/index.html). We also present how the framework can merge with the allometric theory of ecology (Brown, Gillooly, Allen, Savage, {\&} West,; to calculate fluxes based on easily obtainable organism-level data (i.e., body masses and species groups—e.g., plants, animals), opening its use to food webs of all complexities. Physiological losses (metabolic losses or losses due to death other than from predation within the food web) may be directly measured or estimated using allometric relationships based on the metabolic theory of ecology, and losses and gains due to predation are a function of ecological efficiencies that describe the proportion of energy that is used for biomass production. The primary output is a matrix of fluxes among the nodes of the food web. These fluxes can be used to describe the role of a species, a function of interest (e.g., predation; total fluxes to predators), multiple functions, or total energy flux (system throughflow or multitrophic functioning). Additionally, the package includes functions to calculate network stability based on the Jacobian matrix, providing insight into how resilient the network is to small perturbations at steady state. Overall, fluxweb provides a flexible set of functions that greatly increase the feasibility of implementing food web energetic approaches to more complex systems. As such, the package facilitates novel opportunities for mechanistically linking quantitative food webs and ecosystem functioning in real and dynamic natural landscapes.},
author = {Gauzens, Benoit and Barnes, Andrew and Giling, Darren P. and Hines, Jes and Jochum, Malte and Lefcheck, Jonathan S. and Rosenbaum, Benjamin and Wang, Shaopeng and Brose, Ulrich},
doi = {10.1111/2041-210X.13109},
file = {:Users/alyssacirtwill/Downloads/gauzens2018.pdf:pdf},
issn = {2041210X},
journal = {Methods in Ecology and Evolution},
keywords = {ecosystem function,energy fluxes,food web,interaction strength,stability},
number = {2},
pages = {270--279},
title = {{fluxweb: An R package to easily estimate energy fluxes in food webs}},
volume = {10},
year = {2019}
}
@article{Geng,
author = {Geng, Xiaojun and Id, Orcid and Id, Orcid and Prof, Shilong and Piao, Josep and Id, Orcid and Id, Orcid},
doi = {10.1111/gcb.15301},
file = {:Users/alyssacirtwill/Downloads/10.1111@gcb.15301.pdf:pdf},
isbn = {0000000297615},
title = {{Running Title:}}
}
@article{Korner2010,
abstract = {In most temperate tree species, phenological events such as flowering and autumnal cessation of growth are not primarily controlled by temperature.},
author = {K{\"{o}}rner, Christian and Basler, David},
doi = {10.1126/science.1186473},
file = {:Users/alyssacirtwill/Downloads/korner2010.pdf:pdf},
issn = {00368075},
journal = {Science},
number = {5972},
pages = {1461--1462},
title = {{Phenology under global warming}},
volume = {327},
year = {2010}
}
@article{Schnell2015,
abstract = {Metabarcoding of environmental samples on second-generation sequencing platforms has rapidly become a valuable tool for ecological studies. A fundamental assumption of this approach is the reliance on being able to track tagged amplicons back to the samples from which they originated. In this study, we address the problem of sequences in metabarcoding sequencing outputs with false combinations of used tags (tag jumps). Unless these sequences can be identified and excluded from downstream analyses, tag jumps creating sequences with false, but already used tag combinations, can cause incorrect assignment of sequences to samples and artificially inflate diversity. In this study, we document and investigate tag jumping in metabarcoding studies on Illumina sequencing platforms by amplifying mixed-template extracts obtained from bat droppings and leech gut contents with tagged generic arthropod and mammal primers, respectively. We found that an average of 2.6{\%} and 2.1{\%} of sequences had tag combinations, which could be explained by tag jumping in the leech and bat diet study, respectively. We suggest that tag jumping can happen during blunt-ending of pools of tagged amplicons during library build and as a consequence of chimera formation during bulk amplification of tagged amplicons during library index PCR. We argue that tag jumping and contamination between libraries represents a considerable challenge for Illumina-based metabarcoding studies, and suggest measures to avoid false assignment of tag jumping-derived sequences to samples.},
author = {Schnell, Ida B{\ae}rholm and Bohmann, Kristine and Gilbert, M. Thomas P.},
doi = {10.1111/1755-0998.12402},
file = {:Users/alyssacirtwill/Downloads/1755-0998.12402.pdf:pdf},
issn = {17550998},
journal = {Molecular Ecology Resources},
keywords = {Chimeras,Diversity assessment,Environmental DNA,Metabarcoding,Second-generation sequencing,Tag jumping},
number = {6},
pages = {1289--1303},
title = {{Tag jumps illuminated - reducing sequence-to-sample misidentifications in metabarcoding studies}},
volume = {15},
year = {2015}
}
@article{Tikhonov2020,
abstract = {Joint Species Distribution Modelling (JSDM) is becoming an increasingly popular statistical method for analysing data in community ecology. Hierarchical Modelling of Species Communities (HMSC) is a general and flexible framework for fitting JSDMs. HMSC allows the integration of community ecology data with data on environmental covariates, species traits, phylogenetic relationships and the spatio-temporal context of the study, providing predictive insights into community assembly processes from non-manipulative observational data of species communities. The full range of functionality of HMSC has remained restricted to Matlab users only. To make HMSC accessible to the wider community of ecologists, we introduce Hmsc 3.0, a user-friendly r implementation. We illustrate the use of the package by applying Hmsc 3.0 to a range of case studies on real and simulated data. The real data consist of bird counts in a spatio-temporally structured dataset, environmental covariates, species traits and phylogenetic relationships. Vignettes on simulated data involve single-species models, models of small communities, models of large species communities and models for large spatial data. We demonstrate the estimation of species responses to environmental covariates and how these depend on species traits, as well as the estimation of residual species associations. We demonstrate how to construct and fit models with different types of random effects, how to examine MCMC convergence, how to examine the explanatory and predictive powers of the models, how to assess parameter estimates and how to make predictions. We further demonstrate how Hmsc 3.0 can be applied to normally distributed data, count data and presence–absence data. The package, along with the extended vignettes, makes JSDM fitting and post-processing easily accessible to ecologists familiar with r.},
author = {Tikhonov, Gleb and Opedal, {\O}ystein H. and Abrego, Nerea and Lehikoinen, Aleksi and de Jonge, Melinda M.J. and Oksanen, Jari and Ovaskainen, Otso},
doi = {10.1111/2041-210X.13345},
file = {:Users/alyssacirtwill/Downloads/2041-210X.13345.pdf:pdf},
issn = {2041210X},
journal = {Methods in Ecology and Evolution},
keywords = {community ecology,community modelling,community similarity,hierarchical modelling of species communities,joint species distribution modelling,multivariate data,species distribution modelling},
number = {3},
pages = {442--447},
title = {{Joint species distribution modelling with the r-package Hmsc}},
volume = {11},
year = {2020}
}
@article{Ovaskainen2017,
abstract = {Community ecology aims to understand what factors determine the assembly and dynamics of species assemblages at different spatiotemporal scales. To facilitate the integration between conceptual and statistical approaches in community ecology, we propose Hierarchical Modelling of Species Communities (HMSC) as a general, flexible framework for modern analysis of community data. While non-manipulative data allow for only correlative and not causal inference, this framework facilitates the formulation of data-driven hypotheses regarding the processes that structure communities. We model environmental filtering by variation and covariation in the responses of individual species to the characteristics of their environment, with potential contingencies on species traits and phylogenetic relationships. We capture biotic assembly rules by species-to-species association matrices, which may be estimated at multiple spatial or temporal scales. We operationalise the HMSC framework as a hierarchical Bayesian joint species distribution model, and implement it as R- and Matlab-packages which enable computationally efficient analyses of large data sets. Armed with this tool, community ecologists can make sense of many types of data, including spatially explicit data and time-series data. We illustrate the use of this framework through a series of diverse ecological examples.},
author = {Ovaskainen, Otso and Tikhonov, Gleb and Norberg, Anna and {Guillaume Blanchet}, F. and Duan, Leo and Dunson, David and Roslin, Tomas and Abrego, Nerea},
doi = {10.1111/ele.12757},
file = {:Users/alyssacirtwill/Downloads/ele.12757.pdf:pdf},
issn = {14610248},
journal = {Ecology Letters},
keywords = {Assembly process,biotic filtering,community distribution,community modelling,community similarity,environmental filtering,functional trait,joint species distribution model,metacommunity,phylogenetic signal},
number = {5},
pages = {561--576},
pmid = {28317296},
title = {{How to make more out of community data? A conceptual framework and its implementation as models and software}},
volume = {20},
year = {2017}
}
@article{Radicchi2011,
abstract = {Many systems in nature, society, and technology can be described as networks, where the vertices are the system's elements, and edges between vertices indicate the interactions between the corresponding elements. Edges may be weighted if the interaction strength is measurable. However, the full network information is often redundant because tools and techniques from network analysis do not work or become very inefficient if the network is too dense, and some weights may just reflect measurement errors and need to be be discarded. Moreover, since weight distributions in many complex weighted networks are broad, most of the weight is concentrated among a small fraction of all edges. It is then crucial to properly detect relevant edges. Simple thresholding would leave only the largest weights, disrupting the multiscale structure of the system, which is at the basis of the structure of complex networks and ought to be kept. In this paper we propose a weight-filtering technique based on a global null model [Global Statistical Significance (GloSS) filter], keeping both the weight distribution and the full topological structure of the network. The method correctly quantifies the statistical significance of weights assigned independently to the edges from a given distribution. Applications to real networks reveal that the GloSS filter is indeed able to identify relevant connections between vertices. {\textcopyright} 2011 American Physical Society.},
archivePrefix = {arXiv},
arxivId = {1009.2913},
author = {Radicchi, Filippo and Ramasco, Jos{\'{e}} J. and Fortunato, Santo},
doi = {10.1103/PhysRevE.83.046101},
eprint = {1009.2913},
file = {:Users/alyssacirtwill/Downloads/PhysRevE.83.046101.pdf:pdf},
issn = {15393755},
journal = {Physical Review E - Statistical, Nonlinear, and Soft Matter Physics},
number = {4},
pages = {046101},
title = {{Information filtering in complex weighted networks}},
volume = {83},
year = {2011}
}
@article{Jirku2012,
abstract = {Comparison of diagnostic methods for Plasmodium spp. in humans from Uganda and the Central African Republic showed that parasites can be efficiently detected by PCR in fecal samples. These results, which rely solely on PCR-based examination of feces, validate numerous estimates of the prevalence of malaria in great apes.},
author = {Jirků, Milan and Pomajb{\'{i}}kov{\'{a}}, Kateřina and Petr{\v{z}}elkov{\'{a}}, Kl{\'{a}}ra J. and Hůzov{\'{a}}, Zuzana and Modr{\'{y}}, David and Luke{\v{s}}, Julius},
doi = {10.3201/eid1804.110984},
file = {:Users/alyssacirtwill/Downloads/11-0984{\_}finalD.pdf:pdf},
issn = {10806040},
journal = {Emerging Infectious Diseases},
number = {4},
pages = {634--636},
pmid = {22469389},
title = {{Detection of Plasmodium spp. in human feces}},
volume = {18},
year = {2012}
}
@article{Bell2017a,
abstract = {Premise of the study: To study pollination networks in a changing environment, we need accurate, high-throughput methods. Previous studies have shown that more highly resolved networks can be constructed by studying pollen loads taken from bees, relative to field observations. DNA metabarcoding potentially allows for faster and finer-scale taxonomic resolution of pollen compared to traditional approaches (e.g., light microscopy), but has not been applied to pollination networks. Methods: We sampled pollen from 38 bee species collected in Florida from sites differing in forest management. We isolated DNA from pollen mixtures and sequenced rbcL and ITS2 gene regions from all mixtures in a single run on the Illumina MiSeq platform. We identified species from sequence data using comprehensive rbcL and ITS2 databases. Results: We successfully built a proof-of-concept quantitative pollination network using pollen metabarcoding. Discussion: Our work underscores that pollen metabarcoding is not quantitative but that quantitative networks can be constructed based on the number of interacting individuals. Due to the frequency of contamination and false positive reads, isolation and PCR negative controls should be used in every reaction. DNA metabarcoding has advantages in efficiency and resolution over microscopic identification of pollen, and we expect that it will have broad utility for future studies of plant–pollinator interactions.},
author = {Bell, Karen L. and Fowler, Julie and Burgess, Kevin S. and Dobbs, Emily K. and Gruenewald, David and Lawley, Brice and Morozumi, Connor and Brosi, Berry J.},
doi = {10.3732/apps.1600124},
file = {:Users/alyssacirtwill/Downloads/apps.1600124 (1).pdf:pdf},
isbn = {2012670092009},
issn = {1537-2197},
journal = {Applications in Plant Sciences},
keywords = {an ecosystem,as a key driver,dna metabarcoding,have been recognized,interactions between mu-,its,maintenance of biodiversity,mutualistic networks,of the creation and,palynology,plant,pollination networks,pollinator interactions,rbcl,the full set of,tually beneficial species in},
number = {6},
pages = {1600124},
title = {{Applying Pollen DNA Metabarcoding to the Study of Plant–Pollinator Interactions}},
volume = {5},
year = {2017}
}
@article{Dianati2016,
abstract = {Empirical networks of weighted dyadic relations often contain "noisy" edges that alter the global characteristics of the network and obfuscate the most important structures therein. Graph pruning is the process of identifying the most significant edges according to a generative null model and extracting the subgraph consisting of those edges. Here, we focus on integer-weighted graphs commonly arising when weights count the occurrences of an "event" relating the nodes. We introduce a simple and intuitive null model related to the configuration model of network generation and derive two significance filters from it: the marginal likelihood filter (MLF) and the global likelihood filter (GLF). The former is a fast algorithm assigning a significance score to each edge based on the marginal distribution of edge weights, whereas the latter is an ensemble approach which takes into account the correlations among edges. We apply these filters to the network of air traffic volume between US airports and recover a geographically faithful representation of the graph. Furthermore, compared with thresholding based on edge weight, we show that our filters extract a larger and significantly sparser giant component.},
archivePrefix = {arXiv},
arxivId = {1503.04085},
author = {Dianati, Navid},
doi = {10.1103/PhysRevE.93.012304},
eprint = {1503.04085},
file = {:Users/alyssacirtwill/Downloads/1503.04085.pdf:pdf},
issn = {24700053},
journal = {Physical Review E},
keywords = {ergm,exponential random graph model,statistical significance,weighted networks},
number = {1},
title = {{Unwinding the hairball graph: Pruning algorithms for weighted complex networks}},
volume = {93},
year = {2016}
}
@incollection{Zhou2012,
abstract = {We propose a novel problem to simplify weighted graphs by pruning least important edges from them. Simplified graphs can be used to improve visualization of a network, to extract its main structure, or as a pre-processing step for other data mining algorithms. We define a graph connectivity function based on the best paths between all pairs of nodes. Given the number of edges to be pruned, the problem is then to select a subset of edges that best maintains the overall graph connectivity. Our model is applicable to a wide range of settings, including probabilistic graphs, flow graphs and distance graphs, since the path quality function that is used to find best paths can be defined by the user. We analyze the problem, and give lower bounds for the effect of individual edge removal in the case where the path quality function has a natural recursive property. We then propose a range of algorithms and report on experimental results on real networks derived from public biological databases. The results show that a large fraction of edges can be removed quite fast and with minimal effect on the overall graph connectivity. A rough semantic analysis of the removed edges indicates that few important edges were removed, and that the proposed approach could be a valuable tool in aiding users to view or explore weighted graphs. {\textcopyright} 2012 Springer-Verlag Berlin Heidelberg.},
address = {Berlin},
author = {Zhou, Fang and Mahler, S{\'{e}}bastien and Toivonen, Hannu},
booktitle = {Bisociative Knowledge Discovery},
doi = {10.1007/978-3-642-31830-6_13},
file = {:Users/alyssacirtwill/Downloads/Zhou2012{\_}Chapter{\_}SimplificationOfNetworksByEdge.pdf:pdf},
isbn = {9783642318290},
issn = {03029743},
pages = {179--198},
publisher = {Springer},
title = {{Simplification of networks by edge pruning}},
year = {2012}
}
@article{Bell2017,
abstract = {Premise of the study: To study pollination networks in a changing environment, we need accurate, high-throughput methods. Previous studies have shown that more highly resolved networks can be constructed by studying pollen loads taken from bees, relative to field observations. DNA metabarcoding potentially allows for faster and finer-scale taxonomic resolution of pollen compared to traditional approaches (e.g., light microscopy), but has not been applied to pollination networks. Methods: We sampled pollen from 38 bee species collected in Florida from sites differing in forest management. We isolated DNA from pollen mixtures and sequenced rbcL and ITS2 gene regions from all mixtures in a single run on the Illumina MiSeq platform. We identified species from sequence data using comprehensive rbcL and ITS2 databases. Results: We successfully built a proof-of-concept quantitative pollination network using pollen metabarcoding. Discussion: Our work underscores that pollen metabarcoding is not quantitative but that quantitative networks can be constructed based on the number of interacting individuals. Due to the frequency of contamination and false positive reads, isolation and PCR negative controls should be used in every reaction. DNA metabarcoding has advantages in efficiency and resolution over microscopic identification of pollen, and we expect that it will have broad utility for future studies of plant–pollinator interactions.},
author = {Bell, Karen L. and Fowler, Julie and Burgess, Kevin S. and Dobbs, Emily K. and Gruenewald, David and Lawley, Brice and Morozumi, Connor and Brosi, Berry J.},
doi = {10.3732/apps.1600124},
file = {:Users/alyssacirtwill/Downloads/apps.1600124.pdf:pdf},
isbn = {2012670092009},
issn = {1537-2197},
journal = {Applications in Plant Sciences},
keywords = {an ecosystem,as a key driver,dna metabarcoding,have been recognized,interactions between mu-,its,maintenance of biodiversity,mutualistic networks,of the creation and,palynology,plant,pollination networks,pollinator interactions,rbcl,the full set of,tually beneficial species in},
number = {6},
pages = {1600124},
title = {{Applying Pollen DNA Metabarcoding to the Study of Plant–Pollinator Interactions}},
volume = {5},
year = {2017}
}
@article{Pafco2018,
abstract = {Strongylid nematodes in large terrestrial herbivores such as great apes, equids, elephants, and humans tend to occur in complex communities. However, identification of all species within strongylid communities using traditional methods based on coproscopy or single nematode amplification and sequencing is virtually impossible. High-throughput sequencing (HTS) technologies provide opportunities to generate large amounts of sequence data and enable analyses of samples containing a mixture of DNA from multiple species/genotypes. We designed and tested an HTS approach for strain-level identification of gastrointestinal strongylids using ITS-2 metabarcoding at the MiSeq Illumina platform in samples from two free-ranging non-human primate species inhabiting the same environment, but differing significantly in their host traits and ecology. Although we observed overlapping of particular haplotypes, overall the studied primate species differed in their strongylid nematode community composition. Using HTS, we revealed hidden diversity in the strongylid nematode communities in non-human primates, more than one haplotype was found in more than 90{\%} of samples and coinfections of more than one putative species occurred in 80{\%} of samples. In conclusion, the HTS approach on strongylid nematodes, preferably using fecal samples, represents a time and cost-efficient way of studying strongylid communities and provides a resolution superior to traditional approaches.},
author = {Paf{\v{c}}o, Barbora and {\v{C}}{\'{i}}{\v{z}}kov{\'{a}}, Dagmar and Kreisinger, Jakub and Hasegawa, Hideo and Vallo, Peter and Shutt, Kathryn and Todd, Angelique and Petr{\v{z}}elkov{\'{a}}, Kl{\'{a}}ra J. and Modr{\'{y}}, David},
doi = {10.1038/s41598-018-24126-3},
file = {:Users/alyssacirtwill/Downloads/s41598-018-24126-3.pdf:pdf},
issn = {20452322},
journal = {Scientific Reports},
number = {1},
pages = {5933},
pmid = {29651122},
title = {{Metabarcoding analysis of strongylid nematode diversity in two sympatric primate species}},
volume = {8},
year = {2018}
}
@article{Liu2010,
author = {Liu, Weimin and Li, Yingying and Learn, Gerald H and Rudicell, Rebecca S and Robertson, Joel D and Keele, Brandon F and Ndjango, Jean-bosco N and Sanz, Crickette M and Morgan, David B and Locatelli, Sabrina and Gonder, Mary K and Kranzusch, Philip J and Walsh, Peter D and Mpoudi-ngole, Eitel and Georgiev, Alexander V and Muller, Martin N and Shaw, M and Peeters, Martine and Sharp, Paul M and Rayner, Julian C and Beatrice, H},
doi = {10.1038/nature09442.Origin},
file = {:Users/alyssacirtwill/Downloads/nihms-253525.pdf:pdf},
journal = {Nature},
number = {7314},
pages = {420--425},
title = {{Liu et al. 2010 Origin of human malaria prasite P. falciparum in gorillas}},
volume = {467},
year = {2010}
}
@article{Boyer2013,
abstract = {Predation is often difficult to observe or quantify for species that are rare, very small, aquatic or nocturnal. The assessment of such species' diet can be conducted using molecular methods that target prey DNA remaining in predators' guts and faeces. These techniques do not require high taxonomic expertise, are applicable to soft-bodied prey and allow for identification at the species level. However, for generalist predators, the presence of mixed prey DNA in guts and faeces can be a major impediment as it requires development of specific primers for each potential prey species for standard (Sanger) sequencing. Therefore, next generation sequencing methods have recently been applied to such situations. In this study, we used 454-pyrosequencing to analyse the diet of Powelliphanta augusta, a carnivorous landsnail endemic to New Zealand and critically endangered after most of its natural habitat has been lost to opencast mining. This species was suspected to feed mainly on earthworms. Although earthworm tissue was not detectable in snail faeces, earthworm DNA was still present in sufficient quantity to conduct molecular analyses. Based on faecal samples collected from 46 landsnails, our analysis provided a complete map of the earthworm-based diet of P. augusta. Predated species appear to be earthworms that live in the leaf litter or earthworms that come to the soil surface at night to feed on the leaf litter. This indicates that P. augusta may not be selective and probably predates any earthworm encountered in the leaf litter. These findings are crucial for selecting future translocation areas for this highly endangered species. The molecular diet analysis protocol used here is particularly appropriate to study the diet of generalist predators that feed on liquid or soft-bodied prey. Because it is non-harmful and non-disturbing for the studied animals, it is also applicable to any species of conservation interest. {\textcopyright} 2013 Boyer et al.},
author = {Boyer, St{\'{e}}phane and Wratten, Stephen D. and Holyoake, Andrew and Abdelkrim, Jawad and Cruickshank, Robert H.},
doi = {10.1371/journal.pone.0075962},
file = {:Users/alyssacirtwill/Downloads/journal.pone.0075962.PDF:PDF},
issn = {19326203},
journal = {PLoS ONE},
number = {9},
pages = {1--8},
pmid = {24086671},
title = {{Using Next-Generation Sequencing to Analyse the Diet of a Highly Endangered Land Snail (Powelliphanta augusta) Feeding on Endemic Earthworms}},
volume = {8},
year = {2013}
}
@article{Du2015,
abstract = {Aim: The phylogenetic constraint hypothesis of flowering phenology states that closely related species flower at similar times of the year. We test this hypothesis for the Chinese angiosperm flora and assess additional effects of growth form, deciduousness, pollination mode and fruit type. We further examine whether the phylogenetic conservatism of flowering phenology tends to increase from tropical to temperate latitudes. Location: China. Methods: The midpoint of flowering time for 19,631 angiosperm species present in China was compiled. The phylogenetic signal for flowering time was evaluated for the whole country using the Blomberg K-value (adjusted for circular data). We then regressed the phylogenetic signal for 28 provinces as a function of their latitude. An analysis of variance for circular data was conducted to test the differences among growth forms. Watson-Williams tests for circular flowering data were used to compare flowering dates between deciduous and evergreen species, animal-pollinated and wind-pollinated species, and fleshy and non-fleshy fruits. Results: The results support the phylogenetic constraint hypothesis. The phylogenetic signal at the whole country scale was lower than that at the province scale. Phylogenetic signal was also lower at tropical latitudes than at temperate latitudes. Flowering dates were associated with biological traits, with growth form having the largest effect. Main conclusions: Flowering phenology was constrained by phylogeny, and so one should account for phylogeny when studying the underlying drivers of phenology. The strength of phylogenetic conservatism appears weaker at larger scales and becomes stronger towards temperate regions. Flowering phenology also varies predictably according to biological traits such as growth form, suggesting that both phylogeny and traits could be used to inform the flowering times of species for which no phenology data are available. It remains to be tested whether the phylogenetic signal for other functional traits putatively related with flowering time also increases with latitude.},
author = {Du, Yanjun and Mao, Lingfeng and Queenborough, Simon A. and Freckleton, Robert P. and Chen, Bin and Ma, Keping},
doi = {10.1111/geb.12303},
file = {:Users/alyssacirtwill/Downloads/Du{\_}et{\_}al-2015{\_}Phylogenetic constraints and trait correlates of flowering phenology in China.pdf:pdf},
issn = {14668238},
journal = {Global Ecology and Biogeography},
keywords = {Chinese flora,Deciduousness,Fruit type,Growth form,Phylogenetic conservatism,Pollination type},
number = {8},
pages = {928--938},
title = {{Phylogenetic constraints and trait correlates of flowering phenology in the angiosperm flora of China}},
volume = {24},
year = {2015}
}
@article{Ficetola2016,
abstract = {Environmental DNA (eDNA) and metabarcoding are boosting our ability to acquire data on species distribution in a variety of ecosystems. Nevertheless, as most of sampling approaches, eDNA is not perfect. It can fail to detect species that are actually present, and even false positives are possible: a species may be apparently detected in areas where it is actually absent. Controlling false positives remains a main challenge for eDNA analyses: in this issue of Molecular Ecology Resources, Lahoz-Monfort et al. test the performance of multiple statistical modelling approaches to estimate the rate of detection and false positives from eDNA data. Here, we discuss the importance of controlling for false detection from early steps of eDNA analyses (laboratory, bioinformatics), to improve the quality of results and allow an efficient use of the site occupancy-detection modelling (SODM) framework for limiting false presences in eDNA analysis.},
author = {Ficetola, Gentile Francesco and Taberlet, Pierre and Coissac, Eric},
doi = {10.1111/1755-0998.12508},
file = {:Users/alyssacirtwill/Downloads/1755-0998.12508.pdf:pdf},
issn = {17550998},
journal = {Molecular Ecology Resources},
keywords = {Bioinformatics,Controls,Laboratory conditions,Occupancy,Sampling error,eDNA},
number = {3},
pages = {604--607},
pmid = {27062589},
title = {{How to limit false positives in environmental DNA and metabarcoding?}},
volume = {16},
year = {2016}
}
@article{Memmott2010,
abstract = {Climate change is expected to drive species extinct by reducing their survival, reproduction and habitat. Less well appreciated is the possibility that climate change could cause extinction by changing the ecological interactions between species. If ecologists, land managers and policy makers are to manage farmland biodiversity sustainably under global climate change, they need to understand the ways in which species interact with each other as this will affect the way they respond to climate change. Here, we consider the ability of nectar flower mixtures used in field margins to provide sufficient forage for bumble-bees under future climate change. We simulated the effect of global warming on the network of plant-pollinator interactions in two types of field margin: a four-species pollen and nectar mix and a six-species wildflower mix. While periods without flowering resources and periods with no food were rare, curtailment of the field season was very common for the bumble-bees in both mixtures. The effect of this, however, could be ameliorated by adding extra species at the start and end of the flowering season. The plant species that could be used to future-proof margins against global warming are discussed. {\textcopyright} 2010 The Royal Society.},
author = {Memmott, Jane and Carvell, Claire and Pywell, Richard F. and Craze, Paul G.},
doi = {10.1098/rstb.2010.0015},
file = {:Users/alyssacirtwill/Desktop/Pollination opinion papers/Memmott, Carvell, Pyewell {\&} Craze (2010).pdf:pdf},
issn = {14712970},
journal = {Philosophical Transactions of the Royal Society B: Biological Sciences},
keywords = {Bumble-bee,Climate change,Field margin},
number = {1549},
pages = {2071--2079},
title = {{The potential impact of global warming on the efficacy of field margins sown for the conservation of bumble-bees}},
volume = {365},
year = {2010}
}
@article{Ryser2020,
abstract = {The objective of this paper is to promote the use of solar energy in powering traffic signal systems for rural areas in Qatar with no power grid. A photovoltaic system is needed in order to use this energy continuously. The results of the investigation of components, design, and market availability are shown in the paper. Solar cells, which are used for absorbing sunlight and generating electric current, are the main source for the system's operation. A charge controller is used to control the flow of charge through the battery and to protect the battery from overcharging and deep discharging. A dc-dc converter is used to regulate the output voltage which depends on the type of dc to dc converter. Lead acid batteries are used as the electric energy storage for the PV system to use electrical energy in the absence of sunlight. The principle operation of the system and the feasibility of using it for rural area with no power grid have been studied. For this project, a mount tracker was constructed that enabled the solar panel to be placed at 0, 15, 30, 45, 60, 75 and 90 degree angles in order to determine which angle and what time provides the optimum voltage. Experimental results for different angles of radiation at different times of the day and different days of the year are shown in the paper.},
archivePrefix = {arXiv},
arxivId = {تم},
author = {Ryser, Remo and Hirt, Myriam R. and H{\"{a}}ussler, Johanna and Gravel, Dominique and Brose, Ulrich},
doi = {10.1016/j.solener.2019.02.027},
eprint = {تم},
file = {:Users/alyssacirtwill/Downloads/2020.06.03.131425v1.full.pdf:pdf},
isbn = {9789896540821},
issn = {0038092X},
journal = {bioRxiv},
keywords = {CST,Concentrated solar radiation,Concentrated solar thermal,Heat transfer,Process heat,Renewable energy,Solar energy,Solar particle receiver,Solar receiver,Solar thermal,Solar vortex receiver,Working fluid,نتم},
pages = {1--9},
title = {{Landscape heterogeneity buffers biodiversity of meta-foood-webs under global change through rescue and drainage effects}},
url = {https://doi.org/10.1016/j.solener.2019.02.027{\%}0Ahttps://www.golder.com/insights/block-caving-a-viable-alternative/{\%}0A???},
year = {2020}
}
@article{Williams2004a,
abstract = {While trophic levels have found broad application throughout ecology, they are also in much contention on analytical and empirical grounds. Here, we use a new generation of data and theory to examine long-standing questions about trophic-level limits and degrees of omnivory. The data include food webs of the Chesapeake Bay, U.S.A., the island of Saint Martin, a U.K. grassland, and a Florida seagrass community, which appear to be the most trophically complete food webs available in the primary literature due to their inclusion of autotrophs and empirically derived estimates of the relative energetic contributions of each trophic link. We show that most (54{\%}) of the 212 species in the four food webs can be unambiguously assigned to a discrete trophic level. Omnivory among the remaining species appears to be quite limited, as judged by the standard deviation of omnivores' energy-weighted food-chain lengths. This allows simple algorithms based on binary food webs without energetic details to yield surprisingly accurate estimates of species' trophic and omnivory levels. While maximum trophic levels may plausibly exceed historically asserted limits, our analyses contradict both recent empirical claims that these limits are exceeded and recent theoretical claims that rampant omnivory eliminates the scientific utility of the trophic-level concept.},
author = {Williams, Richard J. and Martinez, Neo D.},
doi = {10.1086/381964},
file = {:Users/alyssacirtwill/Downloads/williams2004.pdf:pdf},
issn = {00030147},
journal = {American Naturalist},
keywords = {Food chains,Omnivory,Trophic level},
number = {3},
pages = {458--468},
title = {{Limits to trophic levels and omnivory in complex food webs: Theory and data}},
volume = {163},
year = {2004}
}
@article{Tylianakis2007,
abstract = {Global conversion of natural habitats to agriculture has led to marked changes in species diversity and composition. However, it is less clear how habitat modification affects interactions among species. Networks of feeding interactions (food webs) describe the underlying structure of ecological communities, and might be crucially linked to their stability and function. Here, we analyse 48 quantitative food webs for cavity-nesting bees, wasps and their parasitoids across five tropical habitat types. We found marked changes in food-web structure across the modification gradient, despite little variation in species richness. The evenness of interaction frequencies declined with habitat modification, with most energy flowing along one or a few pathways in intensively managed agricultural habitats. In modified habitats there was a higher ratio of parasitoid to host species and increased parasitism rates, with implications for the important ecosystem services, such as pollination and biological control, that are performed by host bees and wasps. The most abundant parasitoid species was more specialized in modified habitats, with reduced attack rates on alternative hosts. Conventional community descriptors failed to discriminate adequately among habitats, indicating that perturbation of the structure and function of ecological communities might be overlooked in studies that do not document and quantify species interactions. Altered interaction structure therefore represents an insidious and functionally important hidden effect of habitat modification by humans. {\textcopyright}2007 Nature Publishing Group.},
author = {Tylianakis, Jason M. and Tscharntke, Teja and Lewis, Owen T.},
doi = {10.1038/nature05429},
file = {:Users/alyssacirtwill/Downloads/tylianakis2007.pdf:pdf},
issn = {14764687},
journal = {Nature},
number = {7124},
pages = {202--205},
pmid = {17215842},
title = {{Habitat modification alters the structure of tropical host-parasitoid food webs}},
volume = {445},
year = {2007}
}
@article{Cagua2019,
abstract = {An important dimension of a species' role is its ability to alter the state and maintain the diversity of its community. Centrality metrics have often been used to identify these species, which are sometimes referred as “keystone” species. However, the relationship between centrality and keystoneness is largely phenomenological and based mostly on our intuition regarding what constitutes an important species. While centrality is useful when predicting which species' extinctions could cause the largest change in a community, it says little about how these species could be used to attain or preserve a particular community state. Here we introduce structural controllability, an approach that allows us to quantify the extent to which network topology can be harnessed to achieve a desired state. It also allows us to quantify a species' control capacity—its relative importance—and identify the set of species that are critical in this context because they have the largest possible control capacity. We illustrate the application of structural controllability with ten pairs of uninvaded and invaded plant-pollinator communities. We found that the controllability of a community is not dependent on its invasion status, but on the asymmetric nature of its mutual dependences. While central species were also likely to have a large control capacity, centrality fails to identify species that, despite being less connected, were critical in their communities. Interestingly, this set of critical species was mostly composed of plants and included every invasive species in our dataset. We also found that species with high control capacity, and in particular critical species, contribute the most to the stable coexistence of their community. This result was true, even when controlling for the species' degree, abundance/interaction strength, and the relative dependence of their partners. Synthesis. Structural controllability is strongly related to the stability of a network and measures the difficulty of managing an ecological community. It also identifies species that are critical to sustain biodiversity and to change or maintain the state of their community and are therefore likely to be very relevant for management and conservation.},
author = {Cagua, Edgar Fernando and Wootton, Kate L. and Stouffer, Daniel B.},
doi = {10.1111/1365-2745.13147},
file = {:Users/alyssacirtwill/Downloads/1365-2745.13147.pdf:pdf},
issn = {13652745},
journal = {Journal of Ecology},
keywords = {control capacity,invasive species,management interventions,mutualism,network control theory,plant population and community dynamics,species' importance,structural stability},
number = {4},
pages = {1779--1790},
title = {{Keystoneness, centrality, and the structural controllability of ecological networks}},
volume = {107},
year = {2019}
}
@article{Dunne2016,
abstract = {There is a nearly 10,000-year history of human presence in the western Gulf of Alaska, but little understanding of how human foragers integrated into and impacted ecosystems through their roles as hunter-gatherers. We present two highly resolved intertidal and nearshore food webs for the Sanak Archipelago in the eastern Aleutian Islands and use them to compare trophic roles of prehistoric humans to other species. We find that the native Aleut people played distinctive roles as super-generalist and highly-omnivorous consumers closely connected to other species. Although the human population was positioned to have strong effects, arrival and presence of Aleut people in the Sanak Archipelago does not appear associated with long-term extinctions. We simulated food web dynamics to explore to what degree introducing a species with trophic roles like those of an Aleut forager, and allowing for variable strong feeding to reflect use of hunting technology, is likely to trigger extinctions. Potential extinctions decreased when an invading omnivorous super-generalist consumer focused strong feeding on decreasing fractions of its possible resources. This study presents the first assessment of the structural roles of humans as consumers within complex ecological networks, and potential impacts of those roles and feeding behavior on associated extinctions.},
author = {Dunne, Jennifer A. and Maschner, Herbert and Betts, Matthew W. and Huntly, Nancy and Russell, Roly and Williams, Richard J. and Wood, Spencer A.},
doi = {10.1038/srep21179},
file = {:Users/alyssacirtwill/Downloads/srep21179.pdf:pdf},
issn = {20452322},
journal = {Scientific Reports},
number = {July 2015},
pages = {1--9},
pmid = {26884149},
publisher = {Nature Publishing Group},
title = {{The roles and impacts of human hunter-gatherers in North Pacific marine food webs}},
url = {http://dx.doi.org/10.1038/srep21179},
volume = {6},
year = {2016}
}
@article{Molnar2012,
author = {{Moln{\'{a}}r V.}, Attila and T{\"{o}}k{\"{o}}lyi, J{\'{a}}cint and V{\'{e}}gv{\'{a}}ri, Zsolt and Sramk{\'{o}}, G{\'{a}}bor and Sulyok, J{\'{o}}zsef and Barta, Zolt{\'{a}}n},
doi = {10.1111/j.1365-2745.2012.02003.x},
file = {:Users/alyssacirtwill/Downloads/Molnar-2012-Pollination-mode-predicts-phenology in orchids.pdf:pdf},
journal = {Journal of Ecology},
pages = {1141--1152},
title = {{Pollination mode predicts phenological response to climate change in terrestrial orchids : a case study from central Europe}},
volume = {100},
year = {2012}
}
@article{Bohan2013,
author = {Bohan, David A. and Woodward, Guy},
doi = {10.1016/B978-0-12-420002-9.10000-9},
file = {:Users/alyssacirtwill/Downloads/Advances{\_}in{\_}Ecological{\_}Research{\_}Ecological{\_}Network.pdf:pdf},
isbn = {9780124200029},
issn = {00652504},
journal = {Advances in Ecological Research},
number = {October 2018},
pages = {13--18},
title = {{Preface. Editorial Commentary: The potential for network approaches to improve knowledge, understanding, and prediction of the structure and functioning of agricultural systems}},
volume = {49},
year = {2013}
}
@article{Davis2010,
abstract = {Climate change has resulted in major changes in the phenology-i.e. the timing of seasonal activities, such as flowering and bird migration-of some species but not others. These differential responses have been shown to result in ecological mismatches that can have negative fitness consequences. However, the ways in which climate change has shaped changes in biodiversity within and across communities are not well understood. Here, we build on our previous results that established a link between plant species' phenological response to climate change and a phylogenetic bias in species' decline in the eastern United States. We extend a similar approach to plant and bird communities in the United States and the UK that further demonstrates that climate change has differentially impacted species based on their phylogenetic relatedness and shared phenological responses. In plants, phenological responses to climate change are often shared among closely related species (i.e. clades), even between geographically disjunct communities. And in some cases, this has resulted in a phylogenetically biased pattern of non-native species success. In birds, the pattern of decline is phylogenetically biased but is not solely explained by phenological response, which suggests that other traits may better explain this pattern. These results illustrate the ways in which phylogenetic thinking can aid in making generalizations of practical importance and enhance efforts to predict species' responses to future climate change. {\textcopyright} 2010 The Royal Society.},
author = {Davis, Charles C. and Willis, Charles G. and Primack, Richard B. and Miller-Rushing, Abraham J.},
doi = {10.1098/rstb.2010.0130},
file = {:Users/alyssacirtwill/Downloads/rstb.2010.0130.pdf:pdf},
issn = {14712970},
journal = {Philosophical Transactions of the Royal Society B: Biological Sciences},
keywords = {Climate change,Community ecology,Extinction,Invasive species,Phenology,Phylogeny},
number = {1555},
pages = {3202--3213},
title = {{The importance of phylogeny to the study of phenological response to global climate change}},
volume = {365},
year = {2010}
}
@article{Willis2008,
abstract = {Climate change has led to major changes in the phenology (the timing of seasonal activities, such as flowering) of some species but not others. The extent to which flowering-time response to temperature is shared among closely related species might have important consequences for community-wide patterns of species loss under rapid climate change. Henry David Thoreau initiated a dataset of the Concord, Massachusetts, flora that spans ≈150 years and provides information on changes in species abundance and flowering time. When these data are analyzed in a phylogenetic context, they indicate that change in abundance is strongly correlated with flowering-time response. Species that do not respond to temperature have decreased greatly in abundance, and include among others anemones and buttercups [Ranunculaceae pro parte (p.p.)], asters and campanulas (Asterales), bluets (Rubiaceae p.p.), bladderworts (Lentibulariaceae), dogwoods (Cornaceae), lilies (Liliales), mints (Lamiaceae p.p.), orchids (Orchidaceae), roses (Rosaceae p.p.), saxifrages (Saxifragales), and violets (Malpighiales). Because flowering-time response traits are shared among closely related species, our findings suggest that climate change has affected and will likely continue to shape the phylogenetically biased pattern of species loss in Thoreau's woods. {\textcopyright} 2008 by The National Academy of Sciences of the USA.},
author = {Willis, Charles G. and Ruhfel, Brad and Primack, Richard B. and Miller-Rushing, Abraham J. and Davis, Charles C.},
doi = {10.1073/pnas.0806446105},
file = {:Users/alyssacirtwill/Downloads/17029.full.pdf:pdf},
issn = {00278424},
journal = {Proceedings of the National Academy of Sciences of the United States of America},
keywords = {Conservation,Extinction,Phenology,Phylogenetic conservatism,Phylogeny},
number = {44},
pages = {17029--17033},
title = {{Phylogenetic patterns of species loss in Thoreau's woods are driven by climate change}},
volume = {105},
year = {2008}
}
@article{Bodin2017,
abstract = {Managing ecosystems is challenging because of the high number of stakeholders, the permeability of man-made political and jurisdictional demarcations in relation to the temporal and spatial extent of biophysical processes, and a limited understanding of complex ecosystem and societal dynamics. Given these conditions, collaborative governance is commonly put forward as the preferred means of addressing environmental problems. Under this paradigm, a deeper understanding of if, when, and how collaboration is effective, and when other means of addressing environmental problems are better suited, is needed. Interdisciplinary research on collaborative networks demonstrates that which actors get involved, with whom they collaborate, and in what ways they are tied to the structures of the ecosystems have profound implications on actors' abilities to address different types of environmental problems.},
author = {Bodin, {\"{O}}rjan},
doi = {10.1126/science.aan1114},
file = {:Users/alyssacirtwill/Downloads/Bodin - 2017 - Collaborative environmental governance Achieving collective action in social-ecological systems (1).pdf:pdf},
issn = {10959203},
journal = {Science},
number = {6352},
pmid = {28818915},
title = {{Collaborative environmental governance: Achieving collective action in social-ecological systems}},
volume = {357},
year = {2017}
}
@article{Granot2020,
abstract = {Aim: It is often assumed that species in richer sites are more specialized, but empirical studies show conflicting results. In the present study, we quantify the correlation between community-level niche breadth and richness. We contrast three mechanisms for gradients in niche breadth: climate, community assembly and nested interactions. First, the climatic stability within the tropics enables species to specialize, resulting in high richness. Under this scenario, we predict stronger richness–niche breadth correlations over larger latitudinal extents and when using environmental niche breadth measures (e.g., habitat). Second, in species-rich areas, biotic interactions drive species to specialize. This may yield richness–niche breadth correlations regardless of the latitudinal extent and the type of niche breadth measure examined, whether environmental or functional (e.g., diet). Third, increased richness intensifies interactions between extreme specialists and generalists. Here, we predict stronger richness–niche breadth correlations when using functional niche breadth measures. Location: Global. Time period: 1973–2018. Major taxa studied: Many taxa. Methods: We conducted a meta-analysis, with the effect size estimated as the correlation between richness and community-averaged niche breadth extracted from each study. We also examined how these correlations depend on the niche breadth measure used (environmental or functional), scale (grain and latitudinal extent), ecosystem and taxa. Results: We found a strong negative correlation between richness and niche breadth, and overall, a non-significant correlation between latitude and niche breadth. The richness–niche breadth correlation was independent of the niche breadth measure used (environmental or functional). Scale, ecosystem and taxa had little effect on the strength of the correlation. Main conclusions: We confirm that species in richer sites, but not necessarily in the tropics, are more specialized. This finding is not dependent on scale or on the type of niche breadth measure used. These results suggest that high richness drives community-level specialization, and thus community assembly is likely to be the major driver of niche breadth rather than climatic gradients shaping both niche breadth and richness.},
author = {Granot, Itai and Belmaker, Jonathan},
doi = {10.1111/geb.13011},
file = {:Users/alyssacirtwill/Downloads/Granot{\_}et{\_}al-2019-Global{\_}Ecology{\_}and{\_}Biogeography.pdf:pdf},
issn = {14668238},
journal = {Global Ecology and Biogeography},
keywords = {latitude,latitudinal diversity gradient,meta-analysis,niche breadth,niche width,specialization,species diversity,species richness},
number = {1},
pages = {159--170},
title = {{Niche breadth and species richness: Correlation strength, scale and mechanisms}},
volume = {29},
year = {2020}
}
@article{Rollin2013,
abstract = {Bees provide an essential pollination service for crops and wild plants. However, substantial declines in bee populations and diversity have been observed in Europe and North America for the past 50 years, partly due to the loss of natural habitats and reduction of plant diversity resulting from agricultural intensification. To mitigate the negative effects of agricultural intensification, agri-environmental schemes (AES) have been proposed to sustain bees and others pollinators in agrosystems. AES include the preservation of semi-natural habitats such as grasslands, fallows, woodlots, hedgerows or set-aside field margins. However, empirical evidence suggest that the use of those semi-natural habitats by bees may vary greatly among bee functional groups and may further be influenced by the presence of alternative foraging habitats such as mass-flowering crops. The present study sets out to investigate whether the three bee groups typically targeted by AES (honey bees, bumble bees and other wild bees) differ in the way they use those semi-natural habitats relative to common mass-flowering crops (oilseed rape, sunflower, alfalfa) in an intensive agricultural farming system. A clear segregation pattern in the use of floral resources appeared between honey bees and wild bees, with the former being tightly associated with mass-flowering crops and the latter with semi-natural habitats. Bumble bees had an intermediate strategy and behaved as habitat generalists. Therefore, it would be sensible to treat the three bee groups with distinct AES management strategies, and to further consider potential effects on AES efficiency of alternative foraging habitats in the surrounding. This study also stresses the importance of native floral resources, particularly in semi-natural herbaceous habitats, for sustaining wild bee populations. {\textcopyright} 2013 Elsevier B.V.},
author = {Rollin, Orianne and Bretagnolle, Vincent and Decourtye, Axel and Aptel, Jean and Michel, Nadia and Vaissi{\`{e}}re, Bernard E. and Henry, Micka{\"{e}}l},
doi = {10.1016/j.agee.2013.07.007},
file = {:Users/alyssacirtwill/Downloads/Rollin2013AgrEcosystEnv179{\_}78 (1).pdf:pdf},
issn = {01678809},
journal = {Agriculture, Ecosystems and Environment},
keywords = {Agri-environmental schemes,Agrosystem,Apoidea,Generalized linear mixed model,Mass-flowering crop,Semi-natural habitat},
number = {October},
pages = {78--86},
publisher = {Elsevier B.V.},
title = {{Differences Of floral resource use between honey bees and wild bees in an intensive farming system}},
url = {http://dx.doi.org/10.1016/j.agee.2013.07.007},
volume = {179},
year = {2013}
}
@article{Baude2016,
abstract = {There is considerable concern over declines in insect pollinator communities and potential impacts on the pollination of crops and wildflowers. Among the multiple pressures facing pollinators, decreasing floral resources due to habitat loss and degradation has been suggested as a key contributing factor. However, a lack of quantitative data has hampered testing for historical changes in floral resources. Here we show that overall floral rewards can be estimated at a national scale by combining vegetation surveys and direct nectar measurements. We find evidence for substantial losses in nectar resources in England and Wales between the 1930s and 1970s; however, total nectar provision in Great Britain as a whole had stabilized by 1978, and increased from 1998 to 2007. These findings concur with trends in pollinator diversity, which declined in the mid-twentieth century but stabilized more recently. The diversity of nectar sources declined from 1978 to 1990 and thereafter in some habitats, with four plant species accounting for over 50{\%} of national nectar provision in 2007. Calcareous grassland, broadleaved woodland and neutral grassland are the habitats that produce the greatest amount of nectar per unit area from the most diverse sources, whereas arable land is the poorest with respect to amount of nectar per unit area and diversity of nectar sources. Although agri-environment schemes add resources to arable landscapes, their national contribution is low. Owing to their large area, improved grasslands could add substantially to national nectar provision if they were managed to increase floral resource provision. This national-scale assessment of floral resource provision affords new insights into the links between plant and pollinator declines, and offers considerable opportunities for conservation.},
author = {Baude, Mathilde and Kunin, William E. and Boatman, Nigel D. and Conyers, Simon and Davies, Nancy and Gillespie, Mark A.K. and Morton, R. Daniel and Smart, Simon M. and Memmott, Jane},
doi = {10.1038/nature16532},
file = {:Users/alyssacirtwill/Downloads/baude2016.pdf:pdf},
issn = {14764687},
journal = {Nature},
number = {7588},
pages = {85--88},
publisher = {Nature Publishing Group},
title = {{Historical nectar assessment reveals the fall and rise of floral resources in Britain}},
url = {http://dx.doi.org/10.1038/nature16532},
volume = {530},
year = {2016}
}
@article{Guimaraes2011a,
abstract = {A major current challenge in evolutionary biology is to understand how networks of interacting species shape the coevolutionary process. We combined a model for trait evolution with data for twenty plant-animal assemblages to explore coevolution in mutualistic networks. The results revealed three fundamental aspects of coevolution in species-rich mutualisms. First, coevolution shapes species traits throughout mutualistic networks by speeding up the overall rate of evolution. Second, coevolution results in higher trait complementarity in interacting partners and trait convergence in species in the same trophic level. Third, convergence is higher in the presence of super-generalists, which are species that interact with multiple groups of species. We predict that worldwide shifts in the occurrence of super-generalists will alter how coevolution shapes webs of interacting species. Introduced species such as honeybees will favour trait convergence in invaded communities, whereas the loss of large frugivores will lead to increased trait dissimilarity in tropical ecosystems. {\textcopyright} 2011 Blackwell Publishing Ltd/CNRS.},
author = {Guimar{\~{a}}es, Paulo R. and Jordano, Pedro and Thompson, John N.},
doi = {10.1111/j.1461-0248.2011.01649.x},
file = {:Users/alyssacirtwill/Downloads/j.1461-0248.2011.01649.x.pdf:pdf},
issn = {14610248},
journal = {Ecology Letters},
keywords = {Coevolution,Complementarity,Convergence,Ecological networks,Evolutionary cascades,Generalists,Mutualisms,Pollination,Seed dispersal,Small-world networks},
number = {9},
pages = {877--885},
title = {{Evolution and coevolution in mutualistic networks}},
volume = {14},
year = {2011}
}
@article{Neto2019,
author = {Neto, J{\'{u}}lio M and Nilsson, Jan-{\aa}ke and Nilsson, Johan and Hegemann, Arne},
doi = {10.1111/oik.07280},
file = {:Users/alyssacirtwill/Downloads/fornoff2016.pdf:pdf},
title = {{Accepted Ar tic le learning Accepted Ar tic le}},
year = {2019}
}
@article{Lopezaraiza-Mikel2007,
abstract = {Studies of pairwise interactions have shown that an alien plant can affect the pollination of a native plant, this effect being mediated by shared pollinators. Here we use a manipulative field experiment, to investigate the impact of the alien plant Impatiens glandulifera on an entire community of coflowering native plants. Visitation and pollen transport networks were constructed to compare replicated I. glandulifera invaded and I. glandulifera removal plots. Invaded plots had significantly higher visitor species richness, visitor abundance and flower visitation. However, the pollen transport networks were dominated by alien pollen grains in the invaded plots and consequently higher visitation may not translate in facilitation for pollination. The more generalized insects were more likely to visit the alien plant, and Hymenoptera and Hemiptera were more likely to visit the alien than Coleoptera. Our data indicate that generalized native pollinators can provide a pathway of integration for alien plants into native visitation systems. {\textcopyright} 2007 Blackwell Publishing Ltd/CNRS.},
author = {Lopezaraiza-Mikel, Martha E. and Hayes, Richard B. and Whalley, Martin R. and Memmott, Jane},
doi = {10.1111/j.1461-0248.2007.01055.x},
file = {:Users/alyssacirtwill/Downloads/Lopezaraiza et al. Ecology Letters 2007.pdf:pdf},
issn = {1461023X},
journal = {Ecology Letters},
keywords = {Competition,Ecological networks,Facilitation,Food webs,Generalization,Impatiens glandulifera,Invasive species,Pollen transport webs,Pollination,Visitation webs},
number = {7},
pages = {539--550},
title = {{The impact of an alien plant on a native plant-pollinator network: An experimental approach}},
volume = {10},
year = {2007}
}
@article{Peralta2020,
abstract = {Abstract Morphology and phenology influence plant?pollinator network structure, but whether they generate more stable pairwise interactions with higher pollination success remains unknown. Here we evaluate the importance of morphological trait matching, phenological overlap and specialisation for the spatio-temporal stability (measured as variability) of plant?pollinator interactions and for pollination success, while controlling for species' abundance. To this end, we combined a 6-year plant?pollinator interaction dataset, with information on species traits, phenologies, specialisation, abundance and pollination success, into structural equation models. Interactions among abundant plants and pollinators with well-matched traits and phenologies formed the stable and functional backbone of the pollination network, whereas poorly matched interactions were variable in time and had lower pollination success. We conclude that phenological overlap could be more useful for predicting changes in species interactions than species abundances, and that non-random extinction of species with well-matched traits could decrease the stability of interactions within communities and reduce their functioning.},
author = {Peralta, Guadalupe and V{\'{a}}zquez, Diego P. and Chacoff, Natacha P. and Lom{\'{a}}scolo, Silvia B. and Perry, George L. W. and Tylianakis, Jason M.},
doi = {10.1111/ele.13510},
file = {:Users/alyssacirtwill/Downloads/10.1111@ele.13510.pdf:pdf},
issn = {1461-023X},
journal = {Ecology Letters},
keywords = {2020,abundance,cialisation,ecology letters,interaction frequency,pollination success,pollinator impact,spatial variability,spe-,temporal variability},
title = {{Trait matching and phenological overlap increase the spatio‐temporal stability and functionality of plant–pollinator interactions}},
year = {2020}
}
@article{Pocock2012a,
abstract = {Understanding species' interactions and the robustness of interaction networks to species loss is essential to understand the effects of species' declines and extinctions. In most studies, different types of networks (such as food webs, parasitoid webs, seed dispersal networks, and pollination networks) have been studied separately. We sampled such multiple networks simultaneously in an agroecosystem. We show that the networks varied in their robustness; networks including pollinators appeared to be particularly fragile. We show that, overall, networks did not strongly covary in their robustness, which suggests that ecological restoration (for example, through agri-environment schemes) benefitting one functional group will not inevitably benefit others. Some individual plant species were disproportionately well linked to many other species. This type of information can be used in restoration management, because it identifies the plant taxa that can potentially lead to disproportionate gains in biodiversity.},
author = {Pocock, Michael J.O. and Evans, Darren M. and Memmott, Jane},
doi = {10.1126/science.1214915},
file = {:Users/alyssacirtwill/Desktop/Pollination opinion papers/Pocock, Evans {\&} Memmott Science 2012 - Copy.pdf:pdf},
issn = {10959203},
journal = {Science},
number = {6071},
pages = {973--977},
pmid = {22363009},
title = {{The robustness and restoration of a network of ecological networks}},
volume = {335},
year = {2012}
}
@article{Herrera2019,
abstract = {Pollinator service is essential for successful sexual reproduction and long-term population persistence of animal-pollinated plants, and innumerable studies have shown that insufficient service by pollinators results in impaired sexual reproduction (“pollen limitation”). Studies directly addressing the predictors of variation in pollinator service across species or habitats remain comparatively scarce, which limits our understanding of the primary causes of natural variation in pollen limitation. This paper evaluates the importance of pollination-related features, evolutionary history, and environment as predictors of pollinator service in a large sample of plant species from undisturbed montane habitats in southeastern Spain. Quantitative data on pollinator visitation were obtained for 191 insect-pollinated species belonging to 142 genera in 43 families, and the predictive values of simple floral traits (perianth type, class of pollinator visitation unit, and visitation unit dry mass), phylogeny, and habitat type were assessed. A total of 24,866 pollinator censuses accounting for 5,414,856 flower-minutes of observation were conducted on 510 different dates. Flowering patch and single flower visitation probabilities by all pollinators combined were significantly predicted by the combined effects of perianth type (open vs. restricted), class of visitation unit (single flower vs. flower packet), mass of visitation unit, phylogenetic relationships, and habitat type. Pollinator composition at insect order level varied extensively among plant species, largely reflecting the contrasting visitation responses of Coleoptera, Diptera, Hymenoptera, and Lepidoptera to variation in floral traits. Pollinator composition had a strong phylogenetic component, and the distribution of phylogenetic autocorrelation hotspots of visitation rates across the plant phylogeny differed widely among insect orders. Habitat type was a key predictor of pollinator composition, as major insect orders exhibited decoupled variation across habitat types in visitation rates. Comprehensive pollinator sampling of a regional plant community has shown that pollinator visitation and composition can be parsimoniously predicted by a combination of simple floral features, habitat type, and evolutionary history. Ambitious community-level studies can help to formulate novel hypotheses and questions, shed fresh light on long-standing controversies in pollination research (e.g., “pollination syndromes”), and identify methodological cautions that should be considered in pollination community studies dealing with small, phylogenetically biased plant species samples.},
author = {Herrera, Carlos M.},
doi = {10.1002/ecm.1402},
file = {:Users/alyssacirtwill/Downloads/10.1002@ecm.1402.pdf:pdf},
issn = {15577015},
journal = {Ecological Monographs},
keywords = {Mediterranean mountain habitats,floral traits,phylogenetic niche conservatism,phylogenetic signal,plant community,pollinator composition,pollinator functional abundance,pollinator service},
pages = {0--3},
title = {{Flower traits, habitat, and phylogeny as predictors of pollinator service: a plant community perspective}},
year = {2019}
}
@article{Blanchet2020,
abstract = {There is a rich amount of information in co-occurrence (presence-absence) data that could be used to understand community assembly. This proposition first envisioned by Forbes (1907) and then Diamond (1975) prompted the development of numerous modelling approaches (e.g. null model analysis, co-occurrence networks and, more recently, joint species distribution models). Both theory and experimental evidence support the idea that ecological interactions may affect co-occurrence, but it remains unclear to what extent the signal of interaction can be captured in observational data. It is now time to step back from the statistical developments and critically assess whether co-occurrence data are really a proxy for ecological interactions. In this paper, we present a series of arguments based on probability, sampling, food web and coexistence theories supporting that significant spatial associations between species (or lack thereof) is a poor proxy for ecological interactions. We discuss appropriate interpretations of co-occurrence, along with potential avenues to extract as much information as possible from such data.},
author = {Blanchet, F Guillaume and Cazelles, Kevin and Gravel, Dominique},
doi = {10.1111/ele.13525},
file = {:Users/alyssacirtwill/Downloads/10.1111@ele.13525.pdf:pdf},
issn = {1461-023X},
journal = {Ecology Letters},
keywords = {2020,absence data,co-occurrence analysis,co-occurrence networks,ecological interactions,ecology letters,presence,statistical inference},
pages = {ele.13525},
title = {{Co‐occurrence is not evidence of ecological interactions}},
url = {https://onlinelibrary.wiley.com/doi/abs/10.1111/ele.13525},
year = {2020}
}
@article{CaraDonna2020,
author = {CaraDonna, Paul J. and Waser, Nickolas M.},
doi = {10.1111/oik.07526},
file = {:Users/alyssacirtwill/Downloads/10.1111@oik.07526.pdf:pdf},
journal = {Oikos},
pages = {1--12},
title = {{Temporal flexibility in the structure of plant-pollinator networks}},
year = {2020}
}
@article{Ovaskainen2013,
abstract = {Climate change may disrupt interspecies phenological synchrony, with adverse consequences to ecosystem functioning. We present here a 40-y-long time series on 10,425 dates that were systematically collected in a single Russian locality for 97 plant, 78 bird, 10 herptile, 19 insect, and 9 fungal phenological events, as well as for 77 climatic events related to temperature, precipitation, snow, ice, and frost. We show that species are shifting their phenologies at dissimilar rates, partly because they respond to different climatic factors, which in turn are shi{\S}fting at dissimilar rates. Plants have advanced their spring phenology even faster than average temperature has increased, whereas migratory birds have shown more divergent responses and shifted, on average, less than plants. Phenological events of birds and insects were mainly triggered by climate cues (variation in temperature and snow and ice cover) occurring over the course of short periods, whereas many plants, herptiles, and fungi were affected by long-term climatic averages. Year-to-year variation in plants, herptiles, and insects showed a high degree of synchrony, whereas the phenological timing of fungi did not correlate with any other taxonomic group. In many cases, species that are synchronous in their year-to-year dynamics have also shifted in congruence, suggesting that climate change may have disrupted phenological synchrony less than has been previously assumed. Our results illustrate how a multidimensional change in the physical environment has translated into a community-level change in phenology.},
author = {Ovaskainen, Otso and Skorokhodova, Svetlana and Yakovleva, Marina and Sukhov, Alexander and Kutenkov, Anatoliy and Kutenkova, Nadezhda and Shcherbakov, Anatoliy and Meyke, Evegeniy and {Del Mar Delgado}, Maria},
doi = {10.1073/pnas.1305533110},
file = {:Users/alyssacirtwill/Downloads/13434.full.pdf:pdf},
issn = {00278424},
journal = {Proceedings of the National Academy of Sciences of the United States of America},
keywords = {Boreal forest,Global warming,Mismatch,Trophic interactions},
number = {33},
pages = {13434--13439},
title = {{Community-level phenological response to climate change}},
volume = {110},
year = {2013}
}
@article{Saunders2018,
abstract = {Current research, management and outreach programmes relevant to insect pollinator conservation are strongly focused on relationships between pollinators and insect-pollinated crops and wild plants. Pollinators also visit wind-pollinated plants to collect pollen, or for nest sites and materials, but these interactions are largely overlooked. I review documented records of bee and syrphid fly species collecting pollen from wind-pollinated plant taxa, including economically important crops, and provide the most comprehensive collation of peer-reviewed records of pollinators visiting wind-pollinated plants to date. I argue for more basic research into functional relationships between insect pollinators and wind-pollinated plants. I found over 200 visitation records for 101 wind-pollinated plant genera in 25 families, including 4 of the 12 gymnosperm families. Almost half the records (49{\%}) were for grasses and sedges (Poales). I also identified records of bees and/or syrphid flies visiting 10 economically important wind-pollinated crop plant species, including three major grain crops (rice, corn, and sorghum). Most records (70{\%}) were from indirect pollen analysis from hives, nest cells or insect bodies, highlighting the need for more direct observational studies of plant–pollinator interactions. Insect pollinator communities require resource diversity to persist in a landscape. Hence, researchers and land managers aiming to identify links between pollinators and ecosystem function should also consider broader interactions beyond the standard traits of the entomophily syndrome.},
author = {Saunders, Manu E.},
doi = {10.1111/icad.12243},
file = {:Users/alyssacirtwill/Downloads/icad.12243.pdf:pdf},
issn = {17524598},
journal = {Insect Conservation and Diversity},
keywords = {Ambophily,anemophily,ecosystem services,functional traits,plant–pollinator interactions,trait linkage},
number = {1},
pages = {13--31},
title = {{Insect pollinators collect pollen from wind-pollinated plants: implications for pollination ecology and sustainable agriculture}},
volume = {11},
year = {2018}
}
@article{Aldridge2011a,
abstract = {1.Shifts in the spatial and temporal patterns of flowering could affect the resources available to pollinators, and such shifts might become more common as climate change progresses. 2.As mid-summer temperatures have warmed, we found that a montane meadow ecosystem in the southern Rocky Mountains of the United States exhibits a trend toward a bimodal distribution of flower abundance, characterized by a mid-season reduction in total flower number, instead of a broad, unimodal flowering peak lasting most of the summer season. 3.We examined the shapes of community-level flowering curves in this system and found that the typical unimodal peak results from a pattern of complementary peaks in flowering among three distinct meadow types (dry, mesic and wet) within the larger ecosystem. However, high mid-summer temperatures were associated with divergent shifts in the flowering curves of these individual meadow types. Specifically, warmer summers appeared to cause increasing bimodality in mesic habitats, and a longer interval between early and late flowering peaks in wet and dry habitats. 4.Together, these habitat-specific shifts produced a longer mid-season valley in floral abundance across the larger ecosystem in warmer years. Because of these warming-induced changes in flowering patterns, and the significant increase in summer temperatures in our study area, there has been a trend toward non-normality of flowering curves over the period 1974-2009. This trend reflects increasing bimodality in total community-wide flowering. 5.The resulting longer periods of low flowering abundance in the middle of the summer season could negatively affect pollinators that are active throughout the season, and shifts in flowering peaks within habitats might create mismatches between floral resources and demand by pollinators with limited foraging ranges. 6.Synthesis. Early-season climate conditions are getting warmer and drier in the high altitudes of the southern Rocky Mountains. We present evidence that this climate change is disrupting flowering phenology within and among different moisture habitats in a sub-alpine meadow ecosystem, causing a mid-season decline in floral resources that might negatively affect mutualists, especially pollinators. Our findings suggest that climate change can have complex effects on phenology at small spatial scales, depending on patch-level habitat differences. {\textcopyright} 2011 The Authors. Journal of Ecology {\textcopyright} 2011 British Ecological Society.},
author = {Aldridge, George and Inouye, David W. and Forrest, Jessica R.K. and Barr, William A. and Miller-Rushing, Abraham J.},
doi = {10.1111/j.1365-2745.2011.01826.x},
file = {:Users/alyssacirtwill/Downloads/j.1365-2745.2011.01826.x.pdf:pdf},
issn = {00220477},
journal = {Journal of Ecology},
keywords = {Climate change,Cumulative flowering density,Flower abundance,Flowering phenology,Plant-climate interactions,Pollinators,Resource availability,Rocky Mountain Biological Laboratory},
number = {4},
pages = {905--913},
title = {{Emergence of a mid-season period of low floral resources in a montane meadow ecosystem associated with climate change}},
volume = {99},
year = {2011}
}
@article{Pettersson2019,
abstract = {Collapse, stability, and dynamical shifts between these states are hallmarks of ecological systems. A major goal in ecosystem research is to identify how limits of these states change with diversity, complexity, interaction-topology, and hierarchies. The primary focus has been on identifying conditions for a system to shift from strict stability to complete collapse. While this boundary is indeed of central importance, it is possible that real ecosystems can show a larger variety of responses to environmental changes. Here, rather than focusing solely on limits of stability or collapse, we quantify and map the full phase space and the boundaries between regions with different response characteristics. We explore this phase space as biodiversity and complexity are varied for interaction webs in which consumer-resource interactions are chosen randomly and driven by Generalized-Lotka-Volterra dynamics. The ability to pinpoint the location of a system within this phase space and quantify the system's proximity to collapse is made possible via a novel mathematical analysis that we develop. Previous work that focused only on collapse lacks the context within the overall phase space to be able to predict when systems are nearing or are poised to collapse. Moreover, in contrast to previous collapse predictions, we account for the fact that dynamics often lead to single species extinctions. Allowing and accounting for these single-species extinctions reveals more detailed structure of the complexity-stability phase space and introduces an intermediate phase between stability and collapse-Extinction Continuum-that give a more nuanced view of how an ecosystem can respond to internal and external changes. With this extended phase space and our construction of predictive measures based strictly on observable quantities, real systems can be better mapped-than using canonical measures by May or critical slowdown-for proximity to collapse and path through phase-space to collapse. S ystem stability and collapse are core concepts for ecology and complex systems and have been studied both theoretically and empirically. Over the years many features of ecosystems have been posited as stabilising factors such as species interaction modules 1 , long weakly-interacting trophic loops 2 , nestedness 3 , species body-size ratios 4 , correlations in species interaction strengths 5 , and trophic coherence 6. The picture of stability that has emerged is multi-faceted. For example, some features are not purely stabilising or destabilising and can interact in non-trivial ways 7. In addition, there are multiple perspectives on stability including resilience and resistance 8,9 as well as many ways to represent the web of interactions between species in an ecosystem (trophic, mutualistic/competitive, mixed). A generic way of investigating ecosystem stability relies on dynamical models. For instance, analysing stabilising effects of Lotka-Volterra-type interaction modules with few species 10,11 or modelling species-rich systems with versions of Generalized-Lotka-Volterra (GLV) models where the species interaction-strengths are typically sampled from statistical distributions 12-14. Pioneering work on stability was done by Robert May who used Random Matrix Theory to show the importance of system complexity in terms of number of species, number of interactions, and variability of interaction strength 15. The conclusions of this work ran counter to the previous paradigm that complexity begets stability through functional re},
author = {Pettersson, Susanne and Savage, Van M and Nilsson-Jacobi, Martin},
doi = {10.1101/713578},
file = {:Users/alyssacirtwill/Desktop/cirtwill.github.io/papers/rsif.2019.0391.pdf:pdf},
isbn = {0000000171403},
keywords = {biocomplexity,biomathematics,computational},
title = {{Predicting Collapse of Complex Ecological Systems: Quantifying the Stability-Complexity Continuum}},
url = {http://dx.doi.org/10.1101/713578},
year = {2019}
}
@article{Roslin_inprep,
author = {Roslin, Tomas and Antao, Laura and Meyke, Evgeniy and Lo, Coong and Tikhonov, Gleb and Gurarie, Eliezer and Abadonova, Marina and Abduraimov, Ozodbek and Akimova, Tatiana and Akkiev, Muzhigit and Ananin, Aleksandr and Andreeva, Elena and Antipin, Maxim and Arzamascev, Konstantin and Babina, Svetlana and Bakin, Oleg and Barabancova, Anna and Basilskaja, Inna and Belova, Nina and Bespalova, Tatjana and Bisikalova, Evgeniya and Bobretsov, Anatoly and Bobrov, Vladimir and Bobrovskyi, Vadim and Bochkareva, Elena and Bogdanov, Gennady and Bolshakov, Vladimir and Bukharova, Evgeniya and Butunina, Alena and Buyvolov, Yuri and Buyvolova, Anna and Chakhireva, Elena and Chashchina, Olga and Cherenkova, Nadezhda and Chistjakov, Sergej and Chuhontseva, Svetlana and Davydov, Evgeniy A and Demchenko, Viktor and Diadicheva, Elena and Dobrolyubov, Aleksandr and Dostoyevskaya, Ludmila and Drovnina, Svetlana and Drozdova, Zoya and Dubanaev, Akynaly and Dubrovsky, Yuriy and Elsukov, Sergey and Epova, Lidia and Ermakova, Olga S and Ershkova, Elena and Ermakova, Olga and Esengeldenova, Aleksandra and Evstigneev, Oleg and Fedotova, Violetta and Filatova, Tatiana and Gashev, Sergey and Gavrilov, Anatoliy and Golovcov, Dmitrij and Goncharova, Nadezhda and Gorbunova, Elena and Gordeeva, Tatyana and Grishchenko, Vitaly and Gromyko, Ludmila and Hohryakov, Vladimir and Hritankov, Alexander and Igosheva, Svetlana and Ivanova, Uliya and Ivanova, Natalya and Kalinkin, Yury and Kazansky, Fedor and Kiseleva, Darya and Knorre, Anastasia and Kolpashikov, Leonid and Korobov, Evgenii and Korolyova, Helen and Korotkikh, Natalia and Kosenkov, Gennadiy and Kotlugalyamova, Elvira and Kozlovsky, Evgeny and Kozsheechkin, Vladimir and Kozyr, Irina and Krasnopevtseva, Aleksandra and Kruglikov, Sergey and Kuberskaya, Olga},
file = {:Users/alyssacirtwill/Downloads/Roslin et al NCC in review.pdf:pdf},
journal = {TBD},
title = {{Large-scale variation in the strength and direction of phenological change}}
}
@article{Baldock2019,
abstract = {Urban areas are often perceived to have lower biodiversity than the wider countryside, but a few small-scale studies suggest that some urban land uses can support substantial pollinator populations. We present a large-scale, well-replicated study of floral resources and pollinators in 360 sites incorporating all major land uses in four British cities. Using a systems approach, we developed Bayesian network models integrating pollinator dispersal and resource switching to estimate city-scale effects of management interventions on plant–pollinator community robustness to species loss. We show that residential gardens and allotments (community gardens) are pollinator ‘hotspots': gardens due to their extensive area, and allotments due to their high pollinator diversity and leverage on city-scale plant–pollinator community robustness. Household income was positively associated with pollinator abundance in gardens, highlighting the influence of socioeconomic factors. Our results underpin urban planning recommendations to enhance pollinator conservation, using increasing city-scale community robustness as our measure of success.},
author = {Baldock, Katherine C.R. and Goddard, Mark A. and Hicks, Damien M. and Kunin, William E. and Mitschunas, Nadine and Morse, Helen and Osgathorpe, Lynne M. and Potts, Simon G. and Robertson, Kirsty M. and Scott, Anna V. and Staniczenko, Phillip P.A. and Stone, Graham N. and Vaughan, Ian P. and Memmott, Jane},
doi = {10.1038/s41559-018-0769-y},
file = {:Users/alyssacirtwill/Desktop/Pollination opinion papers/Baldock et al 2019 Nature Ecol {\&} Evol (1).pdf:pdf},
issn = {2397334X},
journal = {Nature Ecology and Evolution},
number = {3},
pages = {363--373},
pmid = {30643247},
publisher = {Springer US},
title = {{A systems approach reveals urban pollinator hotspots and conservation opportunities}},
url = {http://dx.doi.org/10.1038/s41559-018-0769-y},
volume = {3},
year = {2019}
}
@article{Pichler2020,
abstract = {Ecologists have long suspected that species are more likely to interact if their traits match in a particular way. For example, a pollination interaction may be more likely if the proportions of a bee's tongue fit a plant's flower shape. Empirical estimates of the importance of trait-matching for determining species interactions, however, vary significantly among different types of ecological networks. Here, we show that ambiguity among empirical trait-matching studies may have arisen at least in parts from using overly simple statistical models. Using simulated and real data, we contrast conventional generalized linear models (GLM) with more flexible Machine Learning (ML) models (Random Forest, Boosted Regression Trees, Deep Neural Networks, Convolutional Neural Networks, Support Vector Machines, na{\"{i}}ve Bayes, and k-Nearest-Neighbor), testing their ability to predict species interactions based on traits, and infer trait combinations causally responsible for species interactions. We found that the best ML models can successfully predict species interactions in plant–pollinator networks, outperforming GLMs by a substantial margin. Our results also demonstrate that ML models can better identify the causally responsible trait-matching combinations than GLMs. In two case studies, the best ML models successfully predicted species interactions in a global plant–pollinator database and inferred ecologically plausible trait-matching rules for a plant–hummingbird network from Costa Rica, without any prior assumptions about the system. We conclude that flexible ML models offer many advantages over traditional regression models for understanding interaction networks. We anticipate that these results extrapolate to other ecological network types. More generally, our results highlight the potential of machine learning and artificial intelligence for inference in ecology, beyond standard tasks such as image or pattern recognition.},
archivePrefix = {arXiv},
arxivId = {1908.09853},
author = {Pichler, Maximilian and Boreux, Virginie and Klein, Alexandra Maria and Schleuning, Matthias and Hartig, Florian},
doi = {10.1111/2041-210X.13329},
eprint = {1908.09853},
file = {:Users/alyssacirtwill/Downloads/2041-210X.13329.pdf:pdf},
issn = {2041210X},
journal = {Methods in Ecology and Evolution},
keywords = {bipartite networks,causal inference,deep learning,hummingbirds,insect pollinators,machine learning,pollination syndromes,predictive modelling},
number = {2},
pages = {281--293},
title = {{Machine learning algorithms to infer trait-matching and predict species interactions in ecological networks}},
volume = {11},
year = {2020}
}
@article{Ma2019,
abstract = {Sustainable management of ecosystems and growth in agricultural productivity is at the heart of the United Nations' Sustainable Development Goals for 2030. New management regimes could revolutionize agricultural production, but require an evaluation of the risks and opportunities. Replacing existing conventional weed management with genetically modified, herbicide-tolerant (GMHT) crops, for example, might reduce herbicide applications and increase crop yields, but remains controversial owing to concerns about potential impacts on biodiversity. Until now, such new regimes have been assessed at the species or assemblage level, whereas higher-level ecological network effects remain largely unconsidered. Here, we conduct a large-scale network analysis of invertebrate communities across 502 UK farm sites to GMHT management in different crop types. We find that network-level properties were overwhelmingly shaped by crop type, whereas network structure and robustness were apparently unaltered by GMHT management. This suggests that taxon-specific effects reported previously did not escalate into higher-level systemic structural change in the wider agricultural ecosystem. Our study highlights current limitations of autecological assessments of effect in agriculture in which species interactions and potential compensatory effects are overlooked. We advocate adopting the more holistic system-level evaluations that we explore here, which complement existing assessments for meeting our future agricultural needs.},
author = {Ma, Athen and Lu, Xueke and Gray, Clare and Raybould, Alan and Tamaddoni-Nezhad, Alireza and Woodward, Guy and Bohan, David A.},
doi = {10.1038/s41559-018-0757-2},
file = {:Users/alyssacirtwill/Downloads/ma2018.pdf:pdf},
isbn = {4155901807572},
issn = {2397334X},
journal = {Nature Ecology and Evolution},
number = {2},
pages = {260--264},
pmid = {30598528},
publisher = {Springer US},
title = {{Ecological networks reveal resilience of agro-ecosystems to changes in farming management}},
url = {http://dx.doi.org/10.1038/s41559-018-0757-2},
volume = {3},
year = {2019}
}
@article{Bohan2020,
author = {Bohan, David A},
file = {:Users/alyssacirtwill/Downloads/les{\_}mardis{\_}ecoserv{\_}CHAP3{\_}hdef{\_}2020.pdf:pdf},
title = {needs for ecological functions provided by biodiversity},
year = {2020}
}
@book{Ferre2008,
abstract = {Key notes Abstract : In 2006 , 102 million hectares of GM crops were produced globally . GM cotton and maize with insect resistance were grown on 12 . 1 and 20 . 1 million hectares in 9 and 13 countries , respectively with these crops collectively representing about 32 {\%} of all GM crops grown in 2006 . These insect resistant GM crops produce various Cry toxins from Bacillus thuringiensis ( Bt ) and provide for highly selective and effective control of lepidopteran and coleopteran pests , primarily bollworms , borers and root- worms , which are the most damaging pests of cotton and maize worldwide . It is estimated that between 1996 and 2005 the deployment of Bt cotton and maize has reduced the volume of insecticide active ingredient used for pest control by 94 . 5 and 7 . 0 million kg and increased farm income through reduced costs and improved yields by US {\$} 7 . 51 and 2 . 37 billion , respectively . For cotton and maize pests susceptible to Bt toxins , these GM crops are an extremely successful form of host plant resistance , one of many pest management tactics that can be integrated in pest management systems . Reductions in insecticide use through adoption of Bt crops have broadened opportunities for biological control of all cotton and maize pests but most other pest management tactics have remained largely unchanged or modified only slightly in Bt crops . Many studies have clearly demonstrated enhanced natural enemy abundance in Bt crops compared with conventional crops subject to broad-spectrum chemical insecticides . A few studies also have focused on understanding the functional contribution of this natural enemy conservation . In both systems , several non-target pests have become more problematic in Bt crop fields in some countries largely due to reductions in insecticide use for target pests . Changes in IPM practice , enhanced biological control and the emergence of nontarget pests are further illustrated by examples from the Bt cotton system .},
author = {Ferr{\'{e}}, Juan and Gonz{\'{a}}lez-cabrera, Joel and Bel, Yolanda and Escriche, Baltasar and Haughton, Alison J and Bohan, David A},
file = {:Users/alyssacirtwill/Downloads/iobc-wprs{\_}bulletin{\_}2008{\_}33.pdf:pdf},
isbn = {9789290672074},
pages = {23--25},
title = {{Working Group “ GMOs in Integrated Plant Production ”, Proceedings of the third Working Group meeting “ Ecological Impact of Genetically Modified Organisms ” at Warsaw ( Poland ), 23-25 May , 2007 . Editors : J{\"{o}}rg Romeis , Michael Meissle {\&} Olivier Sanvid}},
volume = {33},
year = {2008}
}
@article{Rollin2013a,
abstract = {Bees provide an essential pollination service for crops and wild plants. However, substantial declines in bee populations and diversity have been observed in Europe and North America for the past 50 years, partly due to the loss of natural habitats and reduction of plant diversity resulting from agricultural intensification. To mitigate the negative effects of agricultural intensification, agri-environmental schemes (AES) have been proposed to sustain bees and others pollinators in agrosystems. AES include the preservation of semi-natural habitats such as grasslands, fallows, woodlots, hedgerows or set-aside field margins. However, empirical evidence suggest that the use of those semi-natural habitats by bees may vary greatly among bee functional groups and may further be influenced by the presence of alternative foraging habitats such as mass-flowering crops. The present study sets out to investigate whether the three bee groups typically targeted by AES (honey bees, bumble bees and other wild bees) differ in the way they use those semi-natural habitats relative to common mass-flowering crops (oilseed rape, sunflower, alfalfa) in an intensive agricultural farming system. A clear segregation pattern in the use of floral resources appeared between honey bees and wild bees, with the former being tightly associated with mass-flowering crops and the latter with semi-natural habitats. Bumble bees had an intermediate strategy and behaved as habitat generalists. Therefore, it would be sensible to treat the three bee groups with distinct AES management strategies, and to further consider potential effects on AES efficiency of alternative foraging habitats in the surrounding. This study also stresses the importance of native floral resources, particularly in semi-natural herbaceous habitats, for sustaining wild bee populations. {\textcopyright} 2013 Elsevier B.V.},
author = {Rollin, Orianne and Bretagnolle, Vincent and Decourtye, Axel and Aptel, Jean and Michel, Nadia and Vaissi{\`{e}}re, Bernard E. and Henry, Micka{\"{e}}l},
doi = {10.1016/j.agee.2013.07.007},
file = {:Users/alyssacirtwill/Downloads/Rollin2013AgrEcosystEnv179{\_}78.pdf:pdf},
issn = {01678809},
journal = {Agriculture, Ecosystems and Environment},
keywords = {Agri-environmental schemes,Agrosystem,Apoidea,Generalized linear mixed model,Mass-flowering crop,Semi-natural habitat},
number = {October},
pages = {78--86},
publisher = {Elsevier B.V.},
title = {{Differences Of floral resource use between honey bees and wild bees in an intensive farming system}},
url = {http://dx.doi.org/10.1016/j.agee.2013.07.007},
volume = {179},
year = {2013}
}
@article{Henry2020,
abstract = {There is an emerging controversy among bee biologists, land managers and beekeepers about the legitimacy of high-density beekeeping in natural protected areas due to the risks of detrimental interactions with local wild bees. The conflicting needs of wild bee conservation and productive beekeeping requires the adoption of inclusive conservation measures. The distance-based beekeeping regulation is a relevant candidate approach in that respect. It consists in increasing spacings among neighbouring apiaries so as to reduce the proportion of land cover under detrimental competition for floral resources. This approach stems from the concept of Apiary Influence Range (AIR), i.e. the distance range around apiaries within which measurements of native plant-pollinators interactions are significantly altered. The seminal study on this topic reported AIRs spanning distances of 0.6–1.1 km around apiaries. The objective of this study is to provide conservation biologists and practitioners with a roadmap to manage the coexistence between productive beekeeping and wild bee conservation, along with a formalised terminology. We first introduce the key theoretical ideas linked with the AIR. Then, we develop the associated calculation rationale to help land managers achieve their wild bee protection goals. Finally, we further provide original AIR values complementary to those available in recent literature. We believe the distance-based beekeeping regulation is in practice more tractable than setting maximal honey bee colony density rules. It may contribute to guide bee biologists and conservation practitioners towards successful inclusive bee conservation, providing the approach can be supported by a broader range of trials in various environmental contexts and using standardised terminology.},
author = {Henry, Micka{\"{e}}l and Rodet, Guy},
doi = {10.1016/j.actao.2020.103555},
file = {:Users/alyssacirtwill/Downloads/10.1016@j.actao.2020.103555.pdf:pdf},
issn = {1146609X},
journal = {Acta Oecologica},
number = {November 2019},
pages = {103555},
publisher = {Elsevier},
title = {{The apiary influence range: A new paradigm for managing the cohabitation of honey bees and wild bee communities}},
url = {https://doi.org/10.1016/j.actao.2020.103555},
volume = {105},
year = {2020}
}
@article{Hoiss2015,
abstract = {Plant-pollinator interactions are essential for the functioning of terrestrial ecosystems, but are increasingly affected by global change. The risks to such mutualistic interactions from increasing temperature and more frequent extreme climatic events such as drought or advanced snow melt are assumed to depend on network specialization, species richness, local climate and associated parameters such as the amplitude of extreme events. Even though elevational gradients provide valuable model systems for climate change and are accompanied by changes in species richness, responses of plant-pollinator networks to climatic extreme events under different environmental and biotic conditions are currently unknown. Here, we show that elevational climatic gradients, species richness and experimentally simulated extreme events interactively change the structure of mutualistic networks in alpine grasslands. We found that the degree of specialization in plant-pollinator networks (H2′) decreased with elevation. Nonetheless, network specialization increased after advanced snow melt at high elevations, whereas changes in network specialization after drought were most pronounced at sites with low species richness. Thus, changes in network specialization after extreme climatic events depended on climatic context and were buffered by high species richness. In our experiment, only generalized plant-pollinator networks changed in their degree of specialization after climatic extreme events. This indicates that contrary to our assumptions, network generalization may not always foster stability of mutualistic interaction networks.},
author = {Hoiss, Bernhard and Krauss, Jochen and Steffan-Dewenter, Ingolf},
doi = {10.1111/gcb.12968},
file = {:Users/alyssacirtwill/Downloads/hoiss2015.pdf:pdf},
issn = {13652486},
journal = {Global Change Biology},
keywords = {Climate change,Ecosystem stability,Elevational gradients,Field experiment,Generalization,Mutualistic networks,Specialization},
number = {11},
pages = {4086--4097},
title = {{Interactive effects of elevation, species richness and extreme climatic events on plant-pollinator networks}},
volume = {21},
year = {2015}
}
@article{Simmons2020,
author = {Simmons, Benno I and Beckerman, Andrew P and Hansen, Katrine and Maruyama, Pietro K},
file = {:Users/alyssacirtwill/Desktop/2020.04.30.070391v1.full.pdf:pdf},
title = {{Niche-based processes and neutrality influence patterns of indirect interactions in mutualistic networks}},
year = {2020}
}
@article{Divertidos,
author = {Divertidos, Memes and Atractivas, Especies Poco},
doi = {10.1111/cobi.13523.This},
file = {:Users/alyssacirtwill/Desktop/Pollination opinion papers/memes{\_}for{\_}conservation.pdf:pdf},
keywords = {conservaci{\'{o}}n,de especies poco populares,el conocimiento existente sobre,especie amenazada,humor,incentivar el inter{\'{e}}s y,la participaci{\'{o}}n de las,los mecanismos que pueden,memes de internet,mercadotecnia de la,mono n{\'{a}}sico,personas en la protecci{\'{o}}n,pol{\'{i}}ticas de conservaci{\'{o}}n,redes sociales,resumen,tendencias de google,y poco atractivas},
title = {{Lenda et al. 19-295 Palabras clave: especie amenazada, humor, memes de internet, mercadotecnia de la conservaci{\'{o}}n, mono n{\'{a}}sico, pol{\'{i}}ticas de conservaci{\'{o}}n, redes sociales, Tendencias de Google}}
}
@article{Sauve2016,
abstract = {Pollination and herbivory networks have mainly been studied separately, highlighting 19 their distinct structural characteristics and the related processes and dynamics. However, 20 most plants interact with both pollinators and herbivores, and there is evidence that both 21 types of interaction affect each other. Here we investigated the way plants connect these 22 mutualistic and antagonistic networks together, and the consequences for community 23 stability. Using an empirical dataset, we show that the way plants connect pollination and 24 herbivory networks is not random and promotes community stability. Analyses of the 25 structure of binary and quantitative networks show different results: the plants' generalism 26 with regards to pollinators is positively correlated to their generalism with regards to 27 herbivores when considering binary interactions, but not when considering quantitative 28 interactions. We also show that plants that share the same pollinators do not share the same 29 herbivores. However, the way plants connect pollination and herbivory networks promotes 30 stability for both binary and quantitative networks. Our results highlight the relevance of 31 considering the diversity of interaction types in ecological communities, and stress the need 32 to better quantify the costs and benefits of interactions, as well as to develop new metrics 33 characterising the way different interaction types are combined within ecological networks.},
author = {Sauve, Alix M. C. and Th{\'{e}}bault, Elisa and Pocock, Michael J. O. and Fontaine, Colin},
doi = {https://doi.org/10.1890/15-0132.1},
file = {:Users/alyssacirtwill/Documents/Papers/Sauve et al.{\_}2016{\_}Ecology.pdf:pdf},
issn = {0012-9658},
journal = {Ecology},
keywords = {Antagonism,Community stability,Herbivory network,Multiple interaction types,Mutualism,Network structure,Pollination network},
number = {4},
pages = {908--917},
title = {{How plants connect pollination and herbivory networks and their contribution to community stability}},
url = {https://esajournals.onlinelibrary.wiley.com/doi/10.1890/15-0132.1},
volume = {97},
year = {2016}
}
@article{Montoya2002,
abstract = {The analysis of some species-rich, well-defined food webs shows that they display the so-called small world behavior shared by a number of disparate complex systems. The three systems analysed (Ythan estuary web, Silwood web and the Little Rock lake web) have different levels of taxonomic resolution, but all of them involve high clustering and short path lengths (near two degrees of separation) between species. Additionally, the distribution of connections P (k) which is skewed in all the webs analysed shows long tails indicative of power-law scaling. These features suggest that communities might be self-organized in a non-random fashion that might have important consequences in their resistance to perturbations (such as species removal). The consequences for ecological theory are outlined. {\textcopyright} 2002 Elsevier Science Ltd.},
archivePrefix = {arXiv},
arxivId = {cond-mat/0011195},
author = {Montoya, Jose M. and Sol{\'{e}}, Ricard V.},
doi = {10.1006/jtbi.2001.2460},
eprint = {0011195},
file = {:Users/alyssacirtwill/Downloads/montoya2002.pdf:pdf},
issn = {00225193},
journal = {Journal of Theoretical Biology},
number = {3},
pages = {405--412},
primaryClass = {cond-mat},
title = {{Small world patterns in food webs}},
volume = {214},
year = {2002}
}
@article{DeManincor2020,
abstract = {For plant–pollinator interactions to occur, the flowering of plants and the flying period of pollinators (i.e. their phenologies) have to overlap. Yet, few models make use of this principle to predict interactions and fewer still are able to compare interaction networks of different sizes. Here, we tackled both challenges using Bayesian structural equation models (SEM), incorporating the effect of phenological overlap in six plant–hoverfly networks. Insect and plant abundances were strong determinants of the number of visits, while phenology overlap alone was not sufficient, but significantly improved model fit. Phenology overlap was a stronger determinant of plant–pollinator interactions in sites where the average overlap was longer and network compartmentalization was weaker, i.e. at higher latitudes. Our approach highlights the advantages of using Bayesian SEMs to compare interaction networks of different sizes along environmental gradients and articulates the various steps needed to do so.},
author = {de Manincor, Natasha and Hautekeete, Nina and Piquot, Yves and Schatz, Bertrand and Vanappelghem, C{\'{e}}dric and Massol, Fran{\c{c}}ois},
doi = {10.1111/oik.07259},
file = {:Users/alyssacirtwill/Desktop/Pollination opinion papers/deManincor{\_}2020{\_}Oikos.pdf:pdf},
issn = {16000706},
journal = {Oikos},
keywords = {Bayesian model,interaction probability,latent block model,latitudinal gradient,mutualistic network,phenology overlap,species abundance,structural equation model},
title = {{Does phenology explain plant–pollinator interactions at different latitudes? An assessment of its explanatory power in plant–hoverfly networks in French calcareous grasslands}},
year = {2020}
}
@article{Palareti2016b,
abstract = {Introduction: D-dimer assay, generally evaluated according to cutoff points calibrated for VTE exclusion, is used to estimate the individual risk of recurrence after a first idiopathic event of venous thromboembolism (VTE). Methods: Commercial D-dimer assays, evaluated according to predetermined cutoff levels for each assay, specific for age (lower in subjects {\textless}70 years) and gender (lower in males), were used in the recent DULCIS study. The present analysis compared the results obtained in the DULCIS with those that might have been had using the following different cutoff criteria: traditional cutoff for VTE exclusion, higher levels in subjects aged ≥60 years, or age multiplied by 10. Results: In young subjects, the DULCIS low cutoff levels resulted in half the recurrent events that would have occurred using the other criteria. In elderly patients, the DULCIS results were similar to those calculated for the two age-adjusted criteria. The adoption of traditional VTE exclusion criteria would have led to positive results in the large majority of elderly subjects, without a significant reduction in the rate of recurrent event. Conclusion: The results confirm the usefulness of the cutoff levels used in DULCIS.},
author = {Palareti, G. and Legnani, C. and Cosmi, B. and Antonucci, E. and Erba, N. and Poli, D. and Testa, S. and Tosetto, A.},
doi = {10.1111/ijlh.12426},
file = {:Users/alyssacirtwill/Documents/Papers/Volf et al.{\_}2017{\_}Journal of Animal Ecology.pdf:pdf},
issn = {1751553X},
journal = {International Journal of Laboratory Hematology},
keywords = {Cutoff criteria,D-dimer,Recurrence,Venous thromboembolism},
number = {1},
pages = {42--49},
title = {{Comparison between different D-Dimer cutoff values to assess the individual risk of recurrent venous thromboembolism: Analysis of results obtained in the DULCIS study}},
volume = {38},
year = {2016}
}
@article{Wernberg2015a,
abstract = {Ecosystem reconfigurations arising from climate-driven changes in species distributions are expected to have profound ecological, social, and economic implications. Here we reveal a rapid climate-driven regime shift of Australian temperate reef communities, which lost their defining kelp forests and became dominated by persistent seaweed turfs. After decades of ocean warming, extreme marine heat waves forced a 100-kilometer range contraction of extensive kelp forests and saw temperate species replaced by seaweeds, invertebrates, corals, and fishes characteristic of subtropical and tropical waters.This community-wide tropicalization fundamentally altered key ecological processes, suppressing the recovery of kelp forests.},
author = {Wernberg, Thomas and Bennett, Scott and Babcock, Russell C and Bettignies, Thibaut De and Cure, Katherine and Depczynski, Martial and Dufois, Francois and Fromont, Jane and Fulton, Christopher J and Hovey, Renae K and Harvey, Euan S and Holmes, Thomas H and Kendrick, Gary a and Radford, Ben and Santana-garcon, Julia and Saunders, Benjamin J and Smale, Dan a and Thomsen, Mads S and Tuckett, Chenae a and Tuya, Fernando},
doi = {10.1126/science.aad8745},
file = {:Users/alyssacirtwill/Documents/Papers/Wernberg et al.{\_}2015{\_}Science.pdf:pdf},
issn = {0036-8075},
journal = {Science},
number = {6295},
pages = {169--172},
title = {{Temperate Marine Ecosystem}},
volume = {353},
year = {2015}
}
@article{Wikander1981a,
abstract = {The investigation is based on material from populations in the depths of Korsfjorden, western Norway. The faecal pellets are described and classified. The process of gut clearance of individuals placed in glass vessels without any sediment was much retarded compared to individuals dwelling in native sediment. Gut capacity increases with increased shell length in A. nitida and remains principally constant in A. longicallus. Gut capacity per unit weight of the animal gives curves of opposite slope for the two species. The relative amount of deposit ingested per 24 hours is principally constant in A. nitida, but decreases with increasing size in A. longicallus. Estimations concerning the degree of reworking of the deposit by a hypothetical population of A. nitida indicate that one individual takes the same deposit into the mantle cavity approximately 50 times during one year. It is assumed that only A. nitida, with regard to sediment reworking, can play an important role in the biotope, because of its occasional high abundance. A time of adaptation to the aquarium situation is necessary before any experimental results can be considered reliable. {\textcopyright} 1981 Taylor {\&} Francis Group, LLC.},
author = {Wikander, Per Bie},
doi = {10.1080/00364827.1981.10414518},
file = {:Users/alyssacirtwill/Documents/Papers/Wikander{\_}1981{\_}Sarsia.pdf:pdf},
issn = {00364827},
journal = {Sarsia},
number = {1},
pages = {35--48},
title = {{Quantitative aspects of deposit feeding in abra nitida (m{\"{u}}ller) and a. longicallus (scacchi) (bivalvia, tellinacea)}},
volume = {66},
year = {1981}
}
@article{Wilson2007a,
abstract = {• In the clade of Penstemon and segregate genera, pollination syndromes are well defined among the 284 species. Most display combinations of floral characters associated with pollination by Hymenoptera, the ancestral mode of pollination for this clade. Forty-one species present characters associated with hummingbird pollination, although some of these ornithophiles are also visited by insects. • The ornithophiles are scattered throughout the traditional taxonomy and across phylogenies estimated from nuclear (internal transcribed spacer (ITS)) and chloroplast DNA (trnCD/TL) sequence data. Here, the number of separate origins of ornithophily is estimated, using bootstrap phylogenies and constrained parsimony searches. • Analyses suggest 21 separate origins, with overwhelming support for 10 of these. Because species sampling was incomplete, this is probably an underestimate. • Penstemons therefore show great evolutionary lability with respect to acquiring hummingbird pollination; this syndrome acts as an attractor to which species with large sympetalous nectar-rich flowers have frequently been drawn. By contrast, penstemons have not undergone evolutionary shifts backwards or to other pollination syndromes. Thus, they are an example of both striking evolutionary lability and constrained evolution. {\textcopyright} The Authors (2007).},
author = {Wilson, Paul and Wolfe, Andrea D. and Armbruster, W. Scott and Thomson, James D.},
doi = {10.1111/j.1469-8137.2007.02219.x},
file = {:Users/alyssacirtwill/Documents/Papers/Wilson et al.{\_}2007{\_}New Phytologist.pdf:pdf},
issn = {0028646X},
journal = {New Phytologist},
keywords = {Conservatism,Constraint,Homoplasy,Lability,Parallelism,Penstemon,Pollination,Speciational drive},
number = {4},
pages = {883--890},
title = {{Constrained lability in floral evolution: Counting convergent origins of hummingbird pollination in Penstemon and Keckiella}},
volume = {176},
year = {2007}
}
@article{Wirta2015c,
abstract = {In the High Arctic, the species richness of spiders is typically low, but abundances can be very high. Thus, how the few spider species occurring in the region choose their prey, and what prey taxa they focus on, may significantly affect the community structure of arctic arthropods. Here we estimate the ecological imprint of adult spiders of three large-bodied species coexisting in Northeast Greenland: the morphologically similar crab spiders Xysticus deichmanni and X. labradorensis (Thomisidae) and the wolf spider Pardosa glacialis (Lycosidae). To describe an important part of these spiders' diet in detail, we amplified and sequenced DNA from prey remains in their guts, selectively focusing on two of the most abundant prey orders in the area (Diptera and Lepidoptera). By comparing the resultant sequences to a reference library including most taxa encountered in the region, we assigned the prey to species. Among the spider taxa occurring in the region, the wolf spider Pardosa glacialis is dominant in terms of both biomass and density. All three spider species proved to be wide generalists, with no detectable differences in prey choice among either the two crab spiders, or among these crab spiders and the wolf spider. This lack of dietary differentiation among species may be caused by the limited prey availability in the Arctic, forcing the predators to both generalism and opportunism. Given the substantial abundance of spiders and the lack of other predatory arthropods in the region, the opportunistic prey choice observed implies that these High-Arctic spider species have the potential for inflicting a strong influence on their prey community.},
author = {Wirta, Helena K. and Weingartner, Elisabeth and Hamb{\"{a}}ck, Peter A. and Roslin, Tomas},
doi = {10.1016/j.baae.2014.11.003},
file = {:Users/alyssacirtwill/Documents/Papers/Wirta et al.{\_}2015{\_}Basic and Applied Ecology.pdf:pdf},
issn = {16180089},
journal = {Basic and Applied Ecology},
keywords = {CO1,DNA barcode,Diet width,Molecular diet analysis,Niche overlap},
number = {1},
pages = {86--92},
publisher = {Elsevier GmbH},
title = {{Extensive niche overlap among the dominant arthropod predators of the High Arctic}},
url = {http://dx.doi.org/10.1016/j.baae.2014.11.003},
volume = {16},
year = {2015}
}
@article{Yguel2011b,
abstract = {Hosts belonging to the same species suffer dramatically different impacts from their natural enemies. This has been explained by host neighbourhood, that is, by surrounding host-species diversity or spatial separation between hosts. However, even spatially neighbouring hosts may be separated by many million years of evolutionary history, potentially reducing the establishment of natural enemies and their impact. We tested whether phylogenetic isolation of oak hosts from neighbouring trees within a forest canopy reduces phytophagy. We found that an increase in phylogenetic isolation by 100 million years corresponded to a 10-fold decline in phytophagy. This was not due to poorer living conditions for phytophages on phylogenetically isolated oaks. Neither species diversity of neighbouring trees nor spatial distance to the closest oak affected phytophagy. We suggest that reduced pressure by natural enemies is a major advantage for individuals within a host species that leave their ancestral niche and grow among distantly related species. {\textcopyright} 2011 Blackwell Publishing Ltd/CNRS.},
author = {Yguel, Benjamin and Bailey, Richard and Tosh, N. Denise and Vialatte, Aude and Vasseur, Chlo{\'{e}} and Vitrac, Xavier and Jean, Frederic and Prinzing, Andreas},
doi = {10.1111/j.1461-0248.2011.01680.x},
file = {:Users/alyssacirtwill/Documents/Papers/Yguel et al.{\_}2011{\_}Ecology Letters.pdf:pdf},
issn = {1461023X},
journal = {Ecology Letters},
keywords = {Community phylogeny,Forest canopy,Insect herbivory,Intraspecific variation,Lepidoptera,Macroevolution,Plant-insect interactions,Quercus,Temperate forest},
number = {11},
pages = {1117--1124},
title = {{Phytophagy on phylogenetically isolated trees: Why hosts should escape their relatives}},
volume = {14},
year = {2011}
}
@article{Ødegaard2005a,
abstract = {Current methods for measuring similarity among phytophagous insect communities fail to consider the phylogenetic relationship between host plants. We analysed this relation based on 3580 host observations of 1174 beetle species associated with 100 species of angiosperms in two different forest types in Panama. We quantified the significance of genetic distance as well as taxonomic rank among angiosperms in relation to species overlap in beetle assemblages. A logarithmic model describing the decrease in beetle species similarity between host-plant species of increasing phylogenetic distance explains 35{\%} of the variation. Applied to taxonomic rank categories the results imply that except for the ancient branching of monocots from dicots, only adaptive radiations of plants on the family and genus level are important for host utilization among phytophagous beetles. These findings enable improvements in estimating host specificity and species richness through correction for phylogenetic relatedness between hosts and consideration of the host-specific fauna associated with monocots. {\textcopyright}2005 Blackwell Publishing Ltd/CNRS.},
author = {{\O}degaard, Frode and Diserud, Ola H. and {\O}stbye, Kjartan},
doi = {10.1111/j.1461-0248.2005.00758.x},
file = {:Users/alyssacirtwill/Documents/Papers/{\O}degaard, Diserud, {\O}stbye{\_}2005{\_}Ecology Letters.pdf:pdf},
issn = {1461023X},
journal = {Ecology Letters},
keywords = {Canopy crane,Coleoptera,Evolution of host range,Herbivore communities,Host specificity,Insect-plant interactions,Panama,Plant taxonomy,Species richness,Tropical forests},
number = {6},
pages = {612--617},
title = {{The importance of plant relatedness for host utilization among phytophagous insects}},
volume = {8},
year = {2005}
}
@article{Canete2008a,
author = {Ca{\~{n}}ete, JI and C{\'{a}}rdenas, CA and Oyarz{\'{u}}n, Sylvia and Plana, Jordi and Palacios, Mauricio and Santana, Mario},
file = {:Users/alyssacirtwill/Documents/Papers/Ca{\~{n}}ete et al.{\_}2008{\_}Revista de Biolog{\'{i}}a Marina y Oceanograf{\'{i}}a.pdf:pdf},
journal = {Revista de Biolog{\'{i}}a Marina y Oceanograf{\'{i}}a},
keywords = {bentos subant{\'{a}}rtico,bop{\'{i}}ridos,lit{\'{o}}didos,parasitismo,regi{\'{o}}n magall{\'{a}}nica},
number = {2},
pages = {265--274},
title = {{Tuberculata Richardson, 1904 (Isopoda: Bopyridae): a Parasite of Juveniles of the King Crab Lithodes Santolla (Molina, 1782)(}},
url = {http://redalyc.uaemex.mx/redalyc/html/479/47943204/47943204.html},
volume = {43},
year = {2008}
}
@article{Chamberlain2014b,
abstract = {Summary: Species community composition is known to alter the network of interactions between two trophic levels, potentially affecting its functioning (e.g. plant pollination success) and the stability of communities. Phylogenies vary in shape with regard to the rate of evolutionary change across a tree (influencing tree balance) and variation in the timing of branching events (affecting the distribution of node ages in trees), both of which may influence the structure of species interaction networks. Because related species are likely to share many of the traits that regulate interactions, the shape of phylogenetic trees may provide some insights into the distribution of traits within communities, and hence the likelihood of interaction among species. However, little attention has been paid to the potential effects of changes in phylogenetic diversity (PD) on interaction networks. Phylogenetic diversity is influenced by species diversity within a community, but also how distantly-related the constituent species are from one another. Here, we evaluate the relationship between two important measures of phylogenetic diversity (tree shape and age of nodes) and the structure of plant-pollinator interaction networks using empirical and simulated data. Whereas the former allows us to evaluate patterns in real communities, the latter allows us to evaluate more systematically the relationship between tree shape and network structure under three different models of trait evolution. In empirical networks, less balanced plant phylogenies were associated with lower connectance in interaction networks indicating that communities with the descendants of recent radiations are more diverged and specialized in their partnerships. In simulations, tree balance and the distribution of nodes through time were included in the best models for modularity, and the second best models for connectance and nestedness. In models assuming random evolutionary change through time (i.e. Brownian motion), less balanced trees and trees with nodes near the tips exhibited greater modularity, whereas in models with an early burst of radiation followed by relative stasis (i.e. early-burst models) more balanced trees and trees with nodes near roots had greater modularity. Synthesis. Overall, these results suggest that the shape of phylogenies can influence the structure of plant-pollinator interaction networks. However, the mismatch between simulations and empirical data indicate that no simple model of trait evolution mimics that observed in real communities. {\textcopyright} 2014 British Ecological Society.},
author = {Chamberlain, Scott and V{\'{a}}zquez, Diego P. and Carvalheiro, Luisa and Elle, Elizabeth and Vamosi, Jana C.},
doi = {10.1111/1365-2745.12293},
file = {:Users/alyssacirtwill/Documents/Papers/Chamberlain et al.{\_}2014{\_}Journal of Ecology.pdf:pdf},
issn = {13652745},
journal = {Journal of Ecology},
keywords = {Connectance,Diversity,Modularity,Nestedness,Network structure,Phylogeny imbalance,Plant population and community dynamics,Plant-pollinator interactions},
number = {5},
pages = {1234--1243},
title = {{Phylogenetic tree shape and the structure of mutualistic networks}},
volume = {102},
year = {2014}
}
@article{Cirtwill2018a,
abstract = {Many different concepts have been used to describe species' roles in food webs (i.e., the ways in which species participate in their communities as consumers and resources). As each concept focuses on a different aspect of food-web structure, it can be difficult to relate these concepts to each other and to other aspects of ecology. Here we use the Eltonian niche as an overarching framework, within which we summarize several commonly-used role concepts (degree, trophic level, motif roles, and centrality). We focus mainly on the topological versions of these concepts but, where dynamical versions of a role concept exist, we acknowledge these as well. Our aim is to highlight areas of overlap and ambiguity between different role concepts and to describe how these roles can be used to group species according to different strategies (i.e., equivalence and functional roles). The existence of “gray areas” between role concepts make it essential for authors to carefully consider both which role concept(s) are most appropriate for the analyses they wish to conduct and what aspect of species' niches (if any) they wish to address. The ecological meaning of differences between species' roles can change dramatically depending on which role concept(s) are used.},
author = {Cirtwill, Alyssa R. and {Dalla Riva}, Giulio Valentino and Gaiarsa, Marilia P. and Bimler, Malyon D. and Cagua, E. Fernando and Coux, Camille and Dehling, D. Matthias},
doi = {10.1016/j.fooweb.2018.e00093},
file = {:Users/alyssacirtwill/Documents/Papers/Cirtwill et al.{\_}2018{\_}Food Webs.pdf:pdf},
issn = {23522496},
journal = {Food Webs},
keywords = {Eltonian niche,Network structure},
pages = {e00093},
publisher = {Elsevier Inc},
title = {{A review of species role concepts in food webs}},
url = {https://doi.org/10.1016/j.fooweb.2018.e00093},
volume = {16},
year = {2018}
}
@article{Cirtwill2018,
abstract = {Inter-annual turnover in community composition can affect the richness and functioning of ecological communities. If incoming and outgoing species do not interact with the same partners, ecological functions such as pollination may be disrupted. Here, we explore the extent to which turnover affects species' roles – as defined based on their participation in different motifs positions – in a series of temporally replicated plant–pollinator networks from high-Arctic Zackenberg, Greenland. We observed substantial turnover in the plant and pollinator assemblages, combined with significant variation in species' roles between networks. Variation in the roles of plants and pollinators tended to increase with the amount of community turnover, although a negative interaction between turnover in the plant and pollinator assemblages complicated this trend for the roles of pollinators. This suggests that increasing turnover in the future will result in changes to the roles of plants and likely those of pollinators. These changing roles may in turn affect the functioning or stability of this pollination network.},
author = {Cirtwill, Alyssa R. and Roslin, Tomas and Rasmussen, Claus and Olesen, Jens Mogens and Stouffer, Daniel B.},
doi = {10.1111/oik.05074},
file = {:Users/alyssacirtwill/Documents/Papers/Cirtwill et al.{\_}2018{\_}Oikos.pdf:pdf},
issn = {16000706},
journal = {Oikos},
keywords = {inter-annual variation,intra-annual variation,network structure,pollination,turnover},
number = {8},
pages = {1163--1176},
title = {{Between-year changes in community composition shape species' roles in an Arctic plant–pollinator network}},
volume = {127},
year = {2018}
}
@article{Cirtwill2019a,
abstract = {Descriptions of ecological networks typically assume that the same interspecific interactions occur each time a community is observed. This contrasts with the known stochasticity of ecological communities: community composition, species abundances and link structure all vary in space and time. Moreover, finite sampling generates variation in the set of interactions actually observed. For interactions that have not been observed, most datasets will not contain enough information for the ecologist to be confident that unobserved interactions truly did not occur. Here, we develop the conceptual and analytical tools needed to capture uncertainty in the estimation of pairwise interactions. To define the problem, we identify the different contributions to the uncertainty of an interaction. We then outline a framework to quantify the uncertainty around each interaction by combining data on observed co-occurrences with prior knowledge. We illustrate this framework using perhaps the most extensively sampled network to date. We found significant uncertainty in estimates for the probability of most pairwise interactions. This uncertainty can, however, be constrained with informative priors. This uncertainty scaled up to summary measures of network structure such as connectance and nestedness. Even with informative priors, we are likely to miss many interactions that may occur rarely or under different local conditions. Overall, we demonstrate the importance of acknowledging the uncertainty inherent in network studies, and the utility of treating interactions as probabilities in pinpointing areas where more study is needed. Most importantly, we stress that networks are best thought of as systems constructed from random variables, the stochastic nature of which must be acknowledged for an accurate representation. Doing so will fundamentally change network analyses and yield greater realism.},
author = {Cirtwill, Alyssa R. and Ekl{\"{o}}f, Anna and Roslin, Tomas and Wootton, Kate and Gravel, Dominique},
doi = {10.1111/2041-210X.13180},
file = {:Users/alyssacirtwill/Documents/Papers/Cirtwill et al.{\_}2019{\_}Methods in Ecology and Evolution(2).pdf:pdf},
issn = {2041210X},
journal = {Methods in Ecology and Evolution},
keywords = {Bayesian networks,ecological networks,probabilistic interactions,sampling error,spatial variability,temporal variability,uncertainty},
number = {6},
pages = {902--911},
title = {{A quantitative framework for investigating the reliability of empirical network construction}},
volume = {10},
year = {2019}
}
@article{Cirtwill2019,
abstract = {Descriptions of ecological networks typically assume that the same interspecific interactions occur each time a community is observed. This contrasts with the known stochasticity of ecological communities: community composition, species abundances and link structure all vary in space and time. Moreover, finite sampling generates variation in the set of interactions actually observed. For interactions that have not been observed, most datasets will not contain enough information for the ecologist to be confident that unobserved interactions truly did not occur. Here, we develop the conceptual and analytical tools needed to capture uncertainty in the estimation of pairwise interactions. To define the problem, we identify the different contributions to the uncertainty of an interaction. We then outline a framework to quantify the uncertainty around each interaction by combining data on observed co-occurrences with prior knowledge. We illustrate this framework using perhaps the most extensively sampled network to date. We found significant uncertainty in estimates for the probability of most pairwise interactions. This uncertainty can, however, be constrained with informative priors. This uncertainty scaled up to summary measures of network structure such as connectance and nestedness. Even with informative priors, we are likely to miss many interactions that may occur rarely or under different local conditions. Overall, we demonstrate the importance of acknowledging the uncertainty inherent in network studies, and the utility of treating interactions as probabilities in pinpointing areas where more study is needed. Most importantly, we stress that networks are best thought of as systems constructed from random variables, the stochastic nature of which must be acknowledged for an accurate representation. Doing so will fundamentally change network analyses and yield greater realism.},
author = {Cirtwill, Alyssa R. and Ekl{\"{o}}f, Anna and Roslin, Tomas and Wootton, Kate and Gravel, Dominique},
doi = {10.1111/2041-210X.13180},
file = {:Users/alyssacirtwill/Documents/Papers/Cirtwill et al.{\_}2019{\_}Methods in Ecology and Evolution.pdf:pdf},
issn = {2041210X},
journal = {Methods in Ecology and Evolution},
keywords = {Bayesian networks,ecological networks,probabilistic interactions,sampling error,spatial variability,temporal variability,uncertainty},
number = {6},
pages = {902--911},
title = {{A quantitative framework for investigating the reliability of empirical network construction}},
volume = {10},
year = {2019}
}
@article{Cirtwill,
author = {Cirtwill, Alyssa R and Stouffer, Daniel B and Romanuk, Tamara N},
file = {:Users/alyssacirtwill/Documents/Papers/Cirtwill, Stouffer, Romanuk{\_}2015{\_}Proceedings of the Royal Society B Biological Sciences.pdf:pdf},
title = {{Specialisation in food webs scales with species richness but not with latitude}}
}
@article{Cirtwill2015a,
abstract = {Previous analyses of empirical food webs (the networks of who eats whom in a community) have revealed that parasites exert a strong influence over observed food web structure and alter many network properties such as connectance and degree distributions. It remains unclear, however, whether these community-level effects are fully explained by differences in the ways that parasites and free-living species interact within a food web. To rigorously quantify the interrelationship between food web structure, the types of species in a web and the distinct types of feeding links between them, we introduce a shared methodology to quantify the structural roles of both species and feeding links. Roles are quantified based on the frequencies with which a species (or link) appears in different food web motifs - the building blocks of networks. We hypothesized that different types of species (e.g. top predators, basal resources, parasites) and different types of links between species (e.g. classic predation, parasitism, concomitant predation on parasites along with their hosts) will show characteristic differences in their food web roles. We found that parasites do indeed have unique structural roles in food webs. Moreover, we demonstrate that different types of feeding links (e.g. parasitism, predation or concomitant predation) are distributed differently in a food web context. More than any other interaction type, concomitant predation appears to constrain the roles of parasites. In contrast, concomitant predation links themselves have more variable roles than any other type of interaction. Together, our results provide a novel perspective on how both species and feeding link composition shape the structure of an ecological community and vice versa.},
author = {Cirtwill, Alyssa R. and Stouffer, Daniel B.},
doi = {10.1111/1365-2656.12323},
file = {:Users/alyssacirtwill/Documents/Papers/Cirtwill, Stouffer{\_}2015{\_}Journal of Animal Ecology(2).pdf:pdf},
issn = {13652656},
journal = {Journal of Animal Ecology},
keywords = {Interaction roles,Network motifs,Role dispersion,Role diversity,Species roles},
number = {3},
pages = {734--744},
title = {{Concomitant predation on parasites is highly variable but constrains the ways in which parasites contribute to food web structure}},
volume = {84},
year = {2015}
}
@article{Cirtwill2016b,
abstract = {Aim: MacArthur and Wilson's original formulation of the theory of island biogeography (TIB) included the corollary hypothesis that species richness might affect immigration and extinction rates. Building on this, other researchers have suggested additional top-down and bottom-up effects. We compare these hypotheses to identify the strongest candidates for inclusion in a ‘trophic TIB'. Location: Six mangrove islands in the Florida Keys, USA. Methods: We studied a classic island biogeography time series featuring lists of species observed on six mangrove islands during roughly 16 censuses each across 700 days. We first used this time series to determine the number of opportunities for species to immigrate to an island for the first time (n = 18,420), to go locally extinct (n = 1943) or to re-immigrate to an island after having previously gone extinct (n = 1813). We then leveraged information on the predators and prey of those species to estimate the potential for top-down and bottom-up interactions during each census period. Finally, we constructed statistical models to test for species richness, top-down, and bottom-up effects on per-species immigration and extinction probabilities and validated them by comparing each model with a similar model based on the classic TIB. Results: We found that models including bottom-up effects gave the greatest improvement over the classic TIB models. Extinction probability in particular decreased sharply for species with both basal resources and animal prey available. Species richness and top-down effects had far weaker impacts on per-species probabilities of immigration and extinction. Main conclusions: Our findings suggest that incorporating information on the trophic structure of island communities – particularly the species-specific availability of resources – can substantially alter predictions of extinction probabilities. Immigration probability, on the contrary, appeared largely stochastic. Incorporating trophic information into predictions of extinction rates therefore represents the most promising and best-supported way to extend the TIB.},
author = {Cirtwill, Alyssa R. and Stouffer, Daniel B.},
doi = {10.1111/geb.12332},
file = {:Users/alyssacirtwill/Documents/Papers/Cirtwill, Stouffer{\_}2016{\_}Global Ecology and Biogeography.pdf:pdf},
issn = {14668238},
journal = {Global Ecology and Biogeography},
keywords = {Bottom-up effects,community assembly,food web,predator–prey interactions,species richness,theory of island biogeography,top-down effects},
number = {7},
pages = {900--911},
title = {{Knowledge of predator–prey interactions improves predictions of immigration and extinction in island biogeography}},
volume = {25},
year = {2016}
}
@article{May1972a,
abstract = {Gardner and Ashby1 have suggested that large complex systems which are assembled (connected) at random may be expected to be stable up to a certain critical level of connectance, and then, as this increases, to suddenly become unstable. Their conclusions were based on the trend of computer studies of systems with 4, 7 and 10 variables. {\textcopyright} 1972 Nature Publishing Group.},
author = {May, Robert M.},
doi = {10.1038/238413a0},
file = {:Users/alyssacirtwill/Documents/Papers/Cohen, Newman{\_}1985{\_}Journal of Theoretical Biology.pdf:pdf},
issn = {00280836},
journal = {Nature},
number = {5364},
pages = {413--414},
pmid = {4559589},
title = {{Will a large complex system be stable?}},
volume = {238},
year = {1972}
}
@article{Cole2020,
abstract = {Agricultural intensification and associated loss of high-quality habitats are key drivers of insect pollinator declines. With the aim of decreasing the environmental impact of agriculture, the 2014 EU Common Agricultural Policy (CAP) defined a set of habitat and landscape features (Ecological Focus Areas: EFAs) farmers could select from as a requirement to receive basic farm payments. To inform the post-2020 CAP, we performed a European-scale evaluation to determine how different EFA options vary in their potential to support insect pollinators under standard and pollinator-friendly management, as well as the extent of farmer uptake. A structured Delphi elicitation process engaged 22 experts from 18 European countries to evaluate EFAs options. By considering life cycle requirements of key pollinating taxa (i.e. bumble bees, solitary bees and hoverflies), each option was evaluated for its potential to provide forage, bee nesting sites and hoverfly larval resources. EFA options varied substantially in the resources they were perceived to provide and their effectiveness varied geographically and temporally. For example, field margins provide relatively good forage throughout the season in Southern and Eastern Europe but lacked early-season forage in Northern and Western Europe. Under standard management, no single EFA option achieved high scores across resource categories and a scarcity of late season forage was perceived. Experts identified substantial opportunities to improve habitat quality by adopting pollinator-friendly management. Improving management alone was, however, unlikely to ensure that all pollinator resource requirements were met. Our analyses suggest that a combination of poor management, differences in the inherent pollinator habitat quality and uptake bias towards catch crops and nitrogen-fixing crops severely limit the potential of EFAs to support pollinators in European agricultural landscapes. Policy Implications. To conserve pollinators and help protect pollination services, our expert elicitation highlights the need to create a variety of interconnected, well-managed habitats that complement each other in the resources they offer. To achieve this the Common Agricultural Policy post-2020 should take a holistic view to implementation that integrates the different delivery vehicles aimed at protecting biodiversity (e.g. enhanced conditionality, eco-schemes and agri-environment and climate measures). To improve habitat quality we recommend an effective monitoring framework with target-orientated indicators and to facilitate the spatial targeting of options collaboration between land managers should be incentivised.},
author = {Cole, Lorna J. and Kleijn, David and Dicks, Lynn V. and Stout, Jane C. and Potts, Simon G. and Albrecht, Matthias and Balzan, Mario V. and Bartomeus, Ignasi and Bebeli, Penelope J. and Bevk, Danilo and Biesmeijer, Jacobus C. and Chlebo, R{\'{o}}bert and Dautartė, An{\v{z}}elika and Emmanouil, Nikolaos and Hartfield, Chris and Holland, John M. and Holzschuh, Andrea and Knoben, Nieke T.J. and Kov{\'{a}}cs-Hosty{\'{a}}nszki, Anik{\'{o}} and Mandelik, Yael and Panou, Heleni and Paxton, Robert J. and Petanidou, Theodora and {Pinheiro de Carvalho}, Miguel A.A. and Rundl{\"{o}}f, Maj and Sarthou, Jean Pierre and Stavrinides, Menelaos C. and Suso, Maria Jose and Szentgy{\"{o}}rgyi, Hajnalka and Vaissi{\`{e}}re, Bernard E. and Varnava, Androulla and Vil{\`{a}}, Montserrat and Zemeckis, Romualdas and Scheper, Jeroen},
doi = {10.1111/1365-2664.13572},
file = {:Users/alyssacirtwill/Documents/Papers/Cole et al.{\_}2019{\_}Journal of Applied Ecology.pdf:pdf},
issn = {13652664},
journal = {Journal of Applied Ecology},
keywords = {CAP Green Architecture,Common Agricultural Policy,Ecological Focus Areas,agri-environment schemes,bees,habitat complementarity,pollination services,pollinator conservation},
number = {March 2019},
pages = {1--14},
title = {{A critical analysis of the potential for EU Common Agricultural Policy measures to support wild pollinators on farmland}},
year = {2020}
}
@article{Lindgreen2004,
abstract = {Corruption is defined as private individuals or enterprises who misuse public resources for private power and/or political gains. They do so through abusing public officials whose behavior deviates from the formal government rules of conduct. Ethical behavior is defined as individuals or enterprises adhering to a non-corrupt work or business practice. A review of the academic literature is conducted drawing on perspectives from the political, economic, and anthropological sciences. Insights from a Danish research program are reported on. This program identifies five different actions for dealing with corruption: (1) no action; (2) withdrawals from markets; (3) decentralized decision-making process; (4) establishment of an anti-corruption code; and (5) mutual commitment through integrity pact. The following aspects of ethical behavior should be regulated through an anticorruption code: the company vis-a-vis political parties; gifts and entertainment expenses; political campaign contributions; and policy against small-scale corruption. Directions for future research are considered including the role of international organizations and multinational companies in fighting corruption and fostering ethical behavior; the role of countries and their governments; and the management systems. [PUBLICATION ABSTRACT]},
author = {Lindgreen, A and Lindgreen, A.},
doi = {10.1023/B},
file = {:Users/alyssacirtwill/Documents/Papers/Dal{\'{e}}n, G{\"{o}}therstr{\"{o}}m, Angerbj{\"{o}}rn{\_}2004{\_}Conservation Genetics.pdf:pdf},
isbn = {0000014060},
issn = {0167-4544},
journal = {Journal of Business Ethics},
keywords = {Behavior,Corrupt practices,Corruption},
number = {1},
pages = {31--39},
title = {{Corruption and unethical behavior: report on a set of Danish guidelines}},
volume = {51},
year = {2004}
}
@article{Coux2016a,
abstract = {Species roles in ecological networks combine to generate their architecture, which contributes to their stability. Species trait diversity also affects ecosystem functioning and resilience, yet it remains unknown whether species' contributions to functional diversity relate to their network roles. Here, we use 21 empirical pollen transport networks to characterise this relationship. We found that, apart from a few abundant species, pollinators with original traits either had few interaction partners or interacted most frequently with a subset of these partners. This suggests that narrowing of interactions to a subset of the plant community accompanies pollinator niche specialisation, congruent with our hypothesised trade-off between having unique traits vs. being able to interact with many mutualist partners. Conversely, these effects were not detected in plants, potentially because key aspects of their flowering traits are conserved at a family level. Relating functional and network roles can provide further insight into mechanisms underlying ecosystem functioning.},
author = {Coux, Camille and Rader, Romina and Bartomeus, Ignasi and Tylianakis, Jason M.},
doi = {10.1111/ele.12612},
file = {:Users/alyssacirtwill/Documents/Papers/Coux et al.{\_}2016{\_}Ecology Letters.pdf:pdf},
issn = {14610248},
journal = {Ecology letters},
keywords = {Biodiversity,ecosystem functioning,interaction,mutualistic network,resilience,stability,web},
number = {7},
pages = {762--770},
title = {{Linking species functional roles to their network roles}},
volume = {19},
year = {2016}
}
@article{Palareti2016c,
abstract = {Introduction: D-dimer assay, generally evaluated according to cutoff points calibrated for VTE exclusion, is used to estimate the individual risk of recurrence after a first idiopathic event of venous thromboembolism (VTE). Methods: Commercial D-dimer assays, evaluated according to predetermined cutoff levels for each assay, specific for age (lower in subjects {\textless}70 years) and gender (lower in males), were used in the recent DULCIS study. The present analysis compared the results obtained in the DULCIS with those that might have been had using the following different cutoff criteria: traditional cutoff for VTE exclusion, higher levels in subjects aged ≥60 years, or age multiplied by 10. Results: In young subjects, the DULCIS low cutoff levels resulted in half the recurrent events that would have occurred using the other criteria. In elderly patients, the DULCIS results were similar to those calculated for the two age-adjusted criteria. The adoption of traditional VTE exclusion criteria would have led to positive results in the large majority of elderly subjects, without a significant reduction in the rate of recurrent event. Conclusion: The results confirm the usefulness of the cutoff levels used in DULCIS.},
author = {Palareti, G. and Legnani, C. and Cosmi, B. and Antonucci, E. and Erba, N. and Poli, D. and Testa, S. and Tosetto, A.},
doi = {10.1111/ijlh.12426},
file = {:Users/alyssacirtwill/Documents/Papers/Delmas et al.{\_}2017{\_}Methods in Ecology and Evolution.pdf:pdf},
issn = {1751553X},
journal = {International Journal of Laboratory Hematology},
keywords = {Cutoff criteria,D-dimer,Recurrence,Venous thromboembolism},
number = {1},
pages = {42--49},
title = {{Comparison between different D-Dimer cutoff values to assess the individual risk of recurrent venous thromboembolism: Analysis of results obtained in the DULCIS study}},
volume = {38},
year = {2016}
}
@article{Devoto2012a,
abstract = {Theory developed from studying changes in the structure and function of communities during natural or managed succession can guide the restoration of particular communities. We constructed 30 quantitative plant-flower visitor networks along a managed successional gradient to identify the main drivers of change in network structure. We then applied two alternative restoration strategies in silico (restoring for functional complementarity or redundancy) to data from our early successional plots to examine whether different strategies affected the restoration trajectories. Changes in network structure were explained by a combination of age, tree density and variation in tree diameter, even when variance explained by undergrowth structure was accounted for first. A combination of field data, a network approach and numerical simulations helped to identify which species should be given restoration priority in the context of different restoration targets. This combined approach provides a powerful tool for directing management decisions, particularly when management seeks to restore or conserve ecosystem function. {\textcopyright} 2012 Blackwell Publishing Ltd/CNRS.},
author = {Devoto, Mariano and Bailey, Sallie and Craze, Paul and Memmott, Jane},
doi = {10.1111/j.1461-0248.2012.01740.x},
file = {:Users/alyssacirtwill/Documents/Papers/Devoto et al.{\_}2012{\_}Ecology Letters.pdf:pdf},
issn = {14610248},
journal = {Ecology Letters},
keywords = {Ecosystem function,Functional complementarity,Functional redundancy,Pine forest,Plant-animal interaction,Plant-pollinator network,Redundancy analysis,Restoration,Restoration strategy,Succession},
number = {4},
pages = {319--328},
title = {{Understanding and planning ecological restoration of plant-pollinator networks}},
volume = {15},
year = {2012}
}
@article{Jajszczyk2009,
author = {Jajszczyk, Andrzej},
doi = {10.1002/bes2.1214},
file = {:Users/alyssacirtwill/Documents/Papers/Drezner, Drezner{\_}2016{\_}Bulletin of the Ecological Society of America.pdf:pdf},
isbn = {2103481003},
issn = {23276096},
journal = {IEEE Communications Magazine},
keywords = {abstract,analysis,and cataloging biodiversity,camera trap,camera trapping,camera-trapping,data,imagery from camera traps,in the,inventory and monitoring,method,networks,picture,statistics,such use of camera,supports ecological investigations,tool,traps continues to expand},
number = {January 2016},
pages = {18},
title = {{E Merging T Echnologies in}},
year = {2009}
}
@article{Zilber-Rosenberg2008,
abstract = {We present here the hologenome theory of evolution, which considers the holobiont (the animal or plant with all of its associated microorganisms) as a unit of selection in evolution. The hologenome is defined as the sum of the genetic information of the host and its microbiota. The theory is based on four generalizations: (1) All animals and plants establish symbiotic relationships with microorganisms. (2) Symbiotic microorganisms are transmitted between generations. (3) The association between host and symbionts affects the fitness of the holobiont within its environment. (4) Variation in the hologenome can be brought about by changes in either the host or the microbiota genomes; under environmental stress, the symbiotic microbial community can change rapidly. These points taken together suggest that the genetic wealth of diverse microbial symbionts can play an important role both in adaptation and in evolution of higher organisms. During periods of rapid changes in the environment, the diverse microbial symbiont community can aid the holobiont in surviving, multiplying and buying the time necessary for the host genome to evolve. The distinguishing feature of the hologenome theory is that it considers all of the diverse microbiota associated with the animal or the plant as part of the evolving holobiont. Thus, the hologenome theory fits within the framework of the 'superorganism' proposed by Wilson and Sober. {\textcopyright} 2008 Federation of European Microbiological Societies. Published by Blackwell Publishing Ltd. All rights reserved.},
author = {Zilber-Rosenberg, Ilana and Rosenberg, Eugene},
doi = {10.1111/j.1574-6976.2008.00123.x},
file = {:Users/alyssacirtwill/Documents/Papers/32-5-723.pdf:pdf},
issn = {01686445},
journal = {FEMS Microbiology Reviews},
keywords = {Evolution,Holobiont,Hologenome theory,Microbial symbiont,Superorganism,Symbiosis},
number = {5},
pages = {723--735},
pmid = {18549407},
title = {{Role of microorganisms in the evolution of animals and plants: The hologenome theory of evolution}},
volume = {32},
year = {2008}
}
@article{Weiss2017,
abstract = {Background: Data from 16S ribosomal RNA (rRNA) amplicon sequencing present challenges to ecological and statistical interpretation. In particular, library sizes often vary over several ranges of magnitude, and the data contains many zeros. Although we are typically interested in comparing relative abundance of taxa in the ecosystem of two or more groups, we can only measure the taxon relative abundance in specimens obtained from the ecosystems. Because the comparison of taxon relative abundance in the specimen is not equivalent to the comparison of taxon relative abundance in the ecosystems, this presents a special challenge. Second, because the relative abundance of taxa in the specimen (as well as in the ecosystem) sum to 1, these are compositional data. Because the compositional data are constrained by the simplex (sum to 1) and are not unconstrained in the Euclidean space, many standard methods of analysis are not applicable. Here, we evaluate how these challenges impact the performance of existing normalization methods and differential abundance analyses. Results: Effects on normalization: Most normalization methods enable successful clustering of samples according to biological origin when the groups differ substantially in their overall microbial composition. Rarefying more clearly clusters samples according to biological origin than other normalization techniques do for ordination metrics based on presence or absence. Alternate normalization measures are potentially vulnerable to artifacts due to library size. Effects on differential abundance testing: We build on a previous work to evaluate seven proposed statistical methods using rarefied as well as raw data. Our simulation studies suggest that the false discovery rates of many differential abundance-testing methods are not increased by rarefying itself, although of course rarefying results in a loss of sensitivity due to elimination of a portion of available data. For groups with large ({\~{}}10×) differences in the average library size, rarefying lowers the false discovery rate. DESeq2, without addition of a constant, increased sensitivity on smaller datasets ({\textless} 20 samples per group) but tends towards a higher false discovery rate with more samples, very uneven ({\~{}}10×) library sizes, and/or compositional effects. For drawing inferences regarding taxon abundance in the ecosystem, analysis of composition of microbiomes (ANCOM) is not only very sensitive (for {\textgreater} 20 samples per group) but also critically the only method tested that has a good control of false discovery rate. Conclusions: These findings guide which normalization and differential abundance techniques to use based on the data characteristics of a given study.},
author = {Weiss, Sophie and Xu, Zhenjiang Zech and Peddada, Shyamal and Amir, Amnon and Bittinger, Kyle and Gonzalez, Antonio and Lozupone, Catherine and Zaneveld, Jesse R. and V{\'{a}}zquez-Baeza, Yoshiki and Birmingham, Amanda and Hyde, Embriette R. and Knight, Rob},
doi = {10.1186/s40168-017-0237-y},
file = {:Users/alyssacirtwill/Documents/Papers/40168{\_}2017{\_}Article{\_}237.pdf:pdf},
isbn = {4016801702},
issn = {20492618},
journal = {Microbiome},
keywords = {Differential abundance,Microbiome,Normalization,Statistics},
number = {1},
pages = {1--18},
pmid = {28253908},
publisher = {Microbiome},
title = {{Normalization and microbial differential abundance strategies depend upon data characteristics}},
volume = {5},
year = {2017}
}
@article{Allesina2011b,
abstract = {Few food web theory hypotheses/predictions can be readily tested using likelihoods of reproducing the data. Simple probabilistic models for food web structure, however, are an exception as their likelihoods were recently derived. Here I test the performance of a more complex model for food web structure that is grounded in the allometric scaling of interactions with body size and the theory of optimal foraging (Allometric Diet Breadth Model-ADBM). This deterministic model has been evaluated by measuring the fraction of trophic relations it correctly predicts. I contrasted this value with that produced by simpler models based on body sizes and found that the quantitative information on allometric scaling and optimal foraging does not significantly increase model fit. Also, I present a method to compute the p-value for the fraction of trophic interactions correctly predicted by the ADBM, or any other model, with respect to three probabilistic models. I find that the ADBM predicts significantly more links than random graphs, but other models can outperform it. Although optimal foraging and allometric scaling may improve our understanding of food webs, the ADBM needs to be modified or replaced to find support in the data. {\textcopyright} 2010 Elsevier Ltd.},
archivePrefix = {arXiv},
arxivId = {0911.2021},
author = {Allesina, Stefano},
doi = {10.1016/j.jtbi.2010.06.040},
eprint = {0911.2021},
file = {:Users/alyssacirtwill/Documents/Papers/Allesina{\_}2011{\_}Journal of Theoretical Biology.pdf:pdf},
issn = {00225193},
journal = {Journal of Theoretical Biology},
keywords = {Allometric relation,Food webs,Likelihoods,Model selection,Optimal foraging},
number = {1},
pages = {161--168},
publisher = {Elsevier},
title = {{Predicting trophic relations in ecological networks: A test of the Allometric Diet Breadth Model}},
url = {http://dx.doi.org/10.1016/j.jtbi.2010.06.040},
volume = {279},
year = {2011}
}
@article{Xiao2016,
author = {Xiao, Hongtao and Hu, Yigang and Lang, Zedong and Fang, Bohao and Guo, Weibin and Zhang, Qi and Pan, Xuan and Lu, Xin},
file = {:Users/alyssacirtwill/Documents/Papers/Bergamini et al.{\_}2017{\_}Oikos.pdf:pdf},
journal = {Journal of Avian Biology},
number = {Accepted Author Manuscript. doi:10.1111/jav.00934},
pages = {0--2},
title = {{Accepted Ar tic le dynamics Accepted Ar tic le}},
year = {2016}
}
@article{Bowen2013a,
abstract = {Diet estimation in marine mammals relies on indirect methods including recovery of prey hard parts from stomachs and feces, quantitative fatty acid signature analysis (QFASA), stable isotope mixing models, and identification of prey DNA in stomach contents and feces. Experimental evidence (9 species/13 studies) shows that digestion strongly influences the proportion and size of otoliths that can be recovered in feces. Number correction factors (NCF) and digestion coefficients have been experimentally determined to reduce the biases in fecal analysis. Correction factors and coefficients have not been determined for diet estimated from stomach contents. QFASA estimates which prey species and amounts must have been eaten to account for the fatty acid composition of the predator. Experimental studies on mammals and seabirds (9 species/10 studies) indicate that accurate estimates of diet can be determined using QFASA. Stable isotope mixing models provide rather coarse taxonomic resolution of diet composition. Prey DNA analysis shows promise as a method to estimate the species composition of diet, but further development and testing is needed to validate its use. Obtaining a representative sample from marine mammal populations is a significant challenge. Therefore, the use of complementary methods is recommended to obtain the most informative results. {\textcopyright} 2012 by the Society for Marine Mammalogy.},
author = {Bowen, W. D. and Iverson, S. J.},
doi = {10.1111/j.1748-7692.2012.00604.x},
file = {:Users/alyssacirtwill/Documents/Papers/Arrizabalaga-Escudero et al.{\_}2018{\_}Molecular Ecology(2).pdf:pdf},
issn = {08240469},
journal = {Marine Mammal Science},
keywords = {DNA,Fatty acids,Feces,Otoliths,Stable isotopes},
number = {4},
pages = {719--754},
title = {{Methods of estimating marine mammal diets: A review of validation experiments and sources of bias and uncertainty}},
volume = {29},
year = {2013}
}
@article{Bluthgen2008,
abstract = {The structure of ecological interaction networks is often interpreted as a product of meaningful ecological and evolutionary mechanisms that shape the degree of specialization in community associations. However, here we show that both unweighted network metrics (connectance, nestedness, and degree distribution) and weighted network metrics (interaction evenness, interaction strength asymmetry) are strongly constrained and biased by the number of observations. Rarely observed species are inevitably regarded as "specialists," irrespective of their actual associations, leading to biased estimates of specialization. Consequently, a skewed distribution of species observation records (such as the lognormal), combined with a relatively low sampling density typical for ecological data, already generates a "nested" and poorly "connected" network with "asymmetric interaction strengths" when interactions are neutral. This is confirmed by null model simulations of bipartite networks, assuming that partners associate randomly in the absence of any specialization and any variation in the correspondence of biological traits between associated species (trait matching). Variation in the skewness of the frequency distribution fundamentally changes the outcome of network metrics. Therefore, interpretation of network metrics in terms of fundamental specialization and trait matching requires an appropriate control for such severe constraints imposed by information deficits. When using an alternative approach that controls for these effects, most natural networks of mutualistic or antagonistic systems show a significantly higher degree of reciprocal specialization (exclusiveness) than expected under neutral conditions. A higher exclusiveness is coherent with a tighter coevolution and suggests a lower ecological redundancy than implied by nested networks. {\textcopyright} 2008 by the Ecological Society of America.},
author = {Bl{\"{u}}thgen, Nico and Fr{\"{u}}nd, Jochen and Vazquez, Diego P. and Menzel, Florian},
doi = {10.1890/07-2121.1},
file = {:Users/alyssacirtwill/Documents/Papers/Bl{\"{u}}thgen et al.{\_}2008{\_}Ecology.pdf:pdf},
issn = {00129658},
journal = {Ecology},
keywords = {Abundance distribution,Biological traits,Connectance,Degree distribution,Ecological networks,Interaction diversity,Interaction strength,Nestedness,Null models,Specialization},
number = {12},
pages = {3387--3399},
pmid = {19137945},
title = {{What do interaction network metrics tell us about specialization and biological traits?}},
volume = {89},
year = {2008}
}
@article{Bowen2013c,
abstract = {Diet estimation in marine mammals relies on indirect methods including recovery of prey hard parts from stomachs and feces, quantitative fatty acid signature analysis (QFASA), stable isotope mixing models, and identification of prey DNA in stomach contents and feces. Experimental evidence (9 species/13 studies) shows that digestion strongly influences the proportion and size of otoliths that can be recovered in feces. Number correction factors (NCF) and digestion coefficients have been experimentally determined to reduce the biases in fecal analysis. Correction factors and coefficients have not been determined for diet estimated from stomach contents. QFASA estimates which prey species and amounts must have been eaten to account for the fatty acid composition of the predator. Experimental studies on mammals and seabirds (9 species/10 studies) indicate that accurate estimates of diet can be determined using QFASA. Stable isotope mixing models provide rather coarse taxonomic resolution of diet composition. Prey DNA analysis shows promise as a method to estimate the species composition of diet, but further development and testing is needed to validate its use. Obtaining a representative sample from marine mammal populations is a significant challenge. Therefore, the use of complementary methods is recommended to obtain the most informative results. {\textcopyright} 2012 by the Society for Marine Mammalogy.},
author = {Bowen, W. D. and Iverson, S. J.},
doi = {10.1111/j.1748-7692.2012.00604.x},
file = {:Users/alyssacirtwill/Documents/Papers/Bowen, Iverson{\_}2013{\_}Marine Mammal Science(2).pdf:pdf},
issn = {08240469},
journal = {Marine Mammal Science},
keywords = {DNA,Fatty acids,Feces,Otoliths,Stable isotopes},
number = {4},
pages = {719--754},
title = {{Methods of estimating marine mammal diets: A review of validation experiments and sources of bias and uncertainty}},
volume = {29},
year = {2013}
}
@article{Bowen2013b,
abstract = {Diet estimation in marine mammals relies on indirect methods including recovery of prey hard parts from stomachs and feces, quantitative fatty acid signature analysis (QFASA), stable isotope mixing models, and identification of prey DNA in stomach contents and feces. Experimental evidence (9 species/13 studies) shows that digestion strongly influences the proportion and size of otoliths that can be recovered in feces. Number correction factors (NCF) and digestion coefficients have been experimentally determined to reduce the biases in fecal analysis. Correction factors and coefficients have not been determined for diet estimated from stomach contents. QFASA estimates which prey species and amounts must have been eaten to account for the fatty acid composition of the predator. Experimental studies on mammals and seabirds (9 species/10 studies) indicate that accurate estimates of diet can be determined using QFASA. Stable isotope mixing models provide rather coarse taxonomic resolution of diet composition. Prey DNA analysis shows promise as a method to estimate the species composition of diet, but further development and testing is needed to validate its use. Obtaining a representative sample from marine mammal populations is a significant challenge. Therefore, the use of complementary methods is recommended to obtain the most informative results. {\textcopyright} 2012 by the Society for Marine Mammalogy.},
author = {Bowen, W. D. and Iverson, S. J.},
doi = {10.1111/j.1748-7692.2012.00604.x},
file = {:Users/alyssacirtwill/Documents/Papers/Bowen, Iverson{\_}2013{\_}Marine Mammal Science.pdf:pdf},
issn = {08240469},
journal = {Marine Mammal Science},
keywords = {DNA,Fatty acids,Feces,Otoliths,Stable isotopes},
number = {4},
pages = {719--754},
title = {{Methods of estimating marine mammal diets: A review of validation experiments and sources of bias and uncertainty}},
volume = {29},
year = {2013}
}
@article{Brandle2006a,
abstract = {We explore the relationship between the pairwise similarity of assemblages of exploiters (phytophagous insects and parasitic fungi) and pairwise genetic distance, range overlap, niche overlap as well as habitat overlap of host trees. Presence of exploiters was extracted from published literature for 23 tree genera occurring in central Europe (6164 host records of phytophagous insects and 860 host records of parasitic fungi). Across all pairs of tree genera, we found a strong negative correlation between the pairwise similarity of assemblages and genetic distances of hosts. This close correlation is due to deep differences in the composition of assemblages on coniferous and deciduous tree genera. Range, niche and habitat overlap were always of much less importance than genetic distance to explain the variation of pairwise similarity of assemblages of exploiters, although some correlations were significant. Therefore in general host switches of exploiters between related hosts are more important that host switches between hosts co-occurring in the same habitat. We found a robust relationship of the pairwise similarity of assemblages of insects and the pairwise similarity of assemblages of fungi which points to the possibility that insects are vectors for parasitic fungi which promotes correlated switches of insects and fungi. Copyright {\textcopyright} Oikos 2006.},
author = {Br{\"{a}}ndle, Martin and Brandl, Roland},
doi = {10.1111/j.2006.0030-1299.14418.x},
file = {:Users/alyssacirtwill/Documents/Papers/Br{\"{a}}ndle, Brandl{\_}2006{\_}Oikos.pdf:pdf},
issn = {00301299},
journal = {Oikos},
number = {2},
pages = {296--304},
title = {{Is the composition of phytophagous insects and parasitic fungi among trees predictable?}},
volume = {113},
year = {2006}
}
@article{Davy1911,
author = {Davy, A. J.},
doi = {10.1093/nq/s11-IV.101.449-k},
file = {:Users/alyssacirtwill/Documents/Papers/Brose et al.{\_}2006{\_}Ecology.pdf:pdf},
issn = {00293970},
journal = {Notes and Queries},
keywords = {allometry,body length,body mass,body-size ratio,food webs,host,parasitoid,predation},
number = {101},
pages = {449},
title = {{North Devon words C. 1600}},
volume = {s11-IV},
year = {1911}
}
@article{Capdevila2019a,
abstract = {Understanding the combined effects of global and local stressors is crucial for conservation and management, yet challenging due to the different scales at which these stressors operate. Here, we examine the effects of one of the most pervasive threats to marine biodiversity, ocean warming, on the early life stages of the habitat-forming macroalga Cystoseira zosteroides, its long-term consequences for population resilience, and its combined effect with physical stressors. First, we performed a controlled laboratory experiment exploring the impacts of warming on early life stages. Settlement and survival of germlings were measured at 16°C (control), 20°C, and 24°C, and both processes were affected by increased temperatures. Then, we integrated this information into stochastic, density-dependent integral projection models. Recovery time after a major disturbance significantly increased in warmer scenarios. The stochastic population growth rate ($\lambda$s) was not strongly affected by warming alone, as high adult survival compensated for thermal-induced recruitment failure. Nevertheless, warming coupled with recurrent physical disturbances had a strong impact on $\lambda$s and population viability. Synthesis. The impact of warming effects on early stages may significantly decrease the natural ability of habitat-forming algae to rebound after major disturbances. These findings highlight that, in a global warming context, populations of deep-water macroalgae will become more vulnerable to further disturbances, and stress the need to incorporate abiotic interactions into demographic models.},
author = {Capdevila, Pol and Hereu, Bernat and Salguero-G{\'{o}}mez, Roberto and Rovira, Graciella and Medrano, Alba and Cebrian, Emma and Garrabou, Joaquim and Kersting, Diego K. and Linares, Cristina},
doi = {10.1111/1365-2745.13090},
file = {:Users/alyssacirtwill/Documents/Papers/Capdevila et al.{\_}2019{\_}Journal of Ecology.pdf:pdf},
issn = {13652745},
journal = {Journal of Ecology},
keywords = {climate change,demography,human impacts,population ecology,quasi-extinction,recovery,seaweeds,stress interactions},
number = {3},
pages = {1129--1140},
title = {{Warming impacts on early life stages increase the vulnerability and delay the population recovery of a long-lived habitat-forming macroalga}},
volume = {107},
year = {2019}
}
@article{Kirkegaard2006a,
abstract = {A population of Theodoxus fluviatilis L in the littoral zone of Lake Esrom was investigated from November 1977 to February 1979. The population was sampled every month in the winter period and twice during the rest of the year. Biomass was estimated as ash-free dry weight (AFDW) of the organic matter both of the soft parts of the animal and the shell itself. The relation between AFDW (c) and shell length (l) was log c = 2.9509 × log (l)-1.7120. The population comprised more than 1 year-class, which could be separated by shell length, by a narrow band on the shells and the growth of algae on the shell. The life cycle lasted 21/2 - 3 years. The oldest animals had a shell length of 7.0-7.5 mm. A few individuals who were estimated to be 31/2 years had a shell length up to 8.6 mm. Population density varied between 575 and 2115 individuals m-2 on the stony substratum. The average was 1160 individuals m-2. Mortality was low during the summer period. In winter many animals died due to the effect of ice and stormy weather on the stony substratum. Growth of the animals was estimated from the shell length. Maximum growth was observed from May to August with no growth during the winter. Egg capsules were found on the stones all year round. New capsules were found from late May to the middle of November. Most freshly laid capsules were observed in May-June and August-September. Capsules from the late summer hatched in spring and capsules laid in the spring hatched in August-September. The average annual net production for the whole population was estimated by three methods. The Allen curve method gave 1.895 AFDW m-2, the growth-increment method gave 1.784 mg AFDW m-2 and the Hynes method 2.284 mg AFDW m-2. Corresponding estimated P/B ratios were 1.29, 1.30 and 1.57. Annual net-production of the four investigated year-classes was 16 mg AFDW m-2 year-1 for 1975, 224 mg AFDW m-2 year-1 for 1976, 1.258 mg AFDW m-2 year-1 for 1977 and 287 mg AFDW m-2 year-1 for 1978. P/B ratios for the three oldest year-classes were, respectively, 0.32, 0.50 and 1.67. A comparison with other investigations on gastropod life cycles, reproduction and P/B ratios is made and differences discussed. Variations are correlated to temperature, and food quality and quantity. {\textcopyright} 2005 Elsevier GmbH. All rights reserved.},
author = {Kirkegaard, J{\o}rn},
doi = {10.1016/j.limno.2005.11.002},
file = {:Users/alyssacirtwill/Documents/Papers/Kirkegaard{\_}2006{\_}Limnologica.pdf:pdf},
issn = {00759511},
journal = {Limnologica},
keywords = {Gastropoda,Lake Esrom,Life cycle,Neritidae,P/B ratio,Production,Theodoxus},
number = {1},
pages = {26--41},
title = {{Life history, growth and production of Theodoxus fluviatilis in Lake Esrom, Denmark}},
volume = {36},
year = {2006}
}
@article{Klug1980a,
abstract = {Scanning electron microscopy, light microscopy, and direct isolations were used to examine the distribution and diversity of bacteria in the gut tracts of larval stages of Tipula abdominalis. The animal had an enlarged hindgut which housed a diverse bacterial community in the lumen and directly attached to the gut wall. Distinct localization was noted, with the most dense and most diverse community anterior to the rectum. A distinct architecture of bacteria occurred in this region, characterized by a layering or a "weblike" array of filamentous bacteria overlying mats of bacteria closely associated with the gut wall. Although morphological diversity was high in the hindgut, filamentous bacteria were the dominant morphology observed. The attached microbiota, sloughed during ecdysis, recolonized to the same density and diversity observed before the molt. The majority of the isolatable bacterial types were facultatively anaerobic. The distinct localization and attached nature of the hindgut bacteria and the recolonization after each molt suggest they are indigenous to this region of the gut tract.},
author = {Klug, M. J. and Kotarski, S.},
doi = {10.1128/aem.40.2.408-416.1980},
file = {:Users/alyssacirtwill/Documents/Papers/Klug, Kotarski{\_}1980{\_}Applied and environmental microbiology.pdf:pdf},
issn = {0099-2240},
journal = {Applied and Environmental Microbiology},
number = {2},
pages = {408--416},
pmid = {16345618},
title = {{Bacteria Associated with the Gut Tract of Larval Stages of the Aquatic Cranefly Tipula abdominalis (Diptera; Tipulidae) †}},
volume = {40},
year = {1980}
}
@article{Kruczynski1972,
abstract = {Bay scallops containing adult female pea crabs are slightly smaller than noninfected scallops. Infected scallops tend to weigh less than noninfected of the same size. The growth of 3 size groups of infected and noninfected scallops was measured over a 3 month period and infected scallops grew less than noninfected. {\textcopyright} 1972 Estuarine Research Federation.},
author = {Kruczynski, William L.},
doi = {10.2307/1351068},
file = {:Users/alyssacirtwill/Documents/Papers/Kruczynski{\_}2006{\_}Chesapeake Science.pdf:pdf},
issn = {00093262},
journal = {Chesapeake Science},
number = {3},
pages = {218--220},
title = {{The effect of the pea crab, Pinnotheres maculatus Say, on growth of the bay scallop, Argopecten irradians concentricus (Say)}},
volume = {13},
year = {1972}
}
@article{Levin2014,
author = {Levin, Donald a and Anderson, Wyatt W and The, Source and Naturalist, American and Oct, No Sep},
file = {:Users/alyssacirtwill/Documents/Papers/Levin, Anderson{\_}1970{\_}The American Naturalist.pdf:pdf},
number = {939},
pages = {455--467},
title = {{The University of Chicago Competition for Pollinators between Simultaneously Flowering Species}},
volume = {104},
year = {2014}
}
@article{Lundgren2005a,
abstract = {A pollination network of 26 pollinator species interacting with 17 plant species from the small Greenlandic island Uummannaq was analyzed for multiple parameters values. Of the insects collected, 77{\%} of all individuals and 77{\%} of all species belonged to Diptera. The ratio of pollinators to plant species was 1.5, which is lower than in other Arctic pollination networks. This might be a double-island effect as Uummannaq is a small island next to Greenland. Connectance was 14.3{\%}, and linkage level of pollinator and plant species averaged 2.4 and 3.7 species links, respectively. The characteristic path length and average clustering coefficient of the 1-mode networks were 1.4 and 0.83, respectively, for the pollinator species and 1.3 and 0.79, respectively, for the plant species. For both pollinator species and plant species, the tail of the degree distribution had the best fit to an exponential model, indicating that the most connected species was constrained in their linking. However, the extremely short path length and high clustering indicated that the networks had small-world behavior, meaning that any disturbance is spread very fast to the entire network and that the networks are error tolerant but vulnerable to attack on the most linked species. {\textcopyright} 2005 Regents of the University of Colorado.},
author = {Lundgren, Rebekka and Olesen, Jens M.},
doi = {10.1657/1523-0430(2005)037[0514:TDAHCW]2.0.CO;2},
file = {:Users/alyssacirtwill/Documents/Papers/Lundgren, Olesen{\_}2005{\_}Arctic, Antarctic, and Alpine Research.pdf:pdf},
issn = {15230430},
journal = {Arctic, Antarctic, and Alpine Research},
number = {4},
pages = {514--520},
title = {{The dense and highly connected world of greenland's plants and their pollinators}},
volume = {37},
year = {2005}
}
@article{Gabriella2015,
abstract = {Using data from the 1984-85 Consumer Expenditure Survey (CES), we examined how expenditure patterns of elderly persons (aged 65-74 and 75 and over) at different income-to-needs levels differ from those of younger mature adults (aged 45-54 and 55-64) at similar income-to-needs levels. Patterns of spending are examined in a variety of areas within three domains - giving, recreation, and essentials. Several important differences exist in the ways that households headed by persons of different ages allocate their expenditures in the domains of essentials and recreation, but few differences exist in the domain of giving. Consistent income group differences exist in expenditure patterns across virtually every area within the three domains. Both age and income differences remain significant when multivariate analyses are performed; thus, the story told is one in which both age and income play important roles.},
author = {Gabriella, Josefsson and Gabriella, Passagerare Josefsson},
file = {:Users/alyssacirtwill/Documents/Papers/LH{\_}WEBCKI.LT.STANDALONE.LExUaRfljcZMKzPzS7J443.pdf:pdf},
isbn = {0016-9013},
number = {965},
pages = {23617257},
title = {{Boarding Pass - Boarding Pass}},
year = {2015}
}
@article{Ludsin2001a,
abstract = {We explored the recent (1969-1996) dynamics of fish communities within Lake Erie, a system formerly degraded by eutrophication and now undergoing oligotrophication owing to phosphorus abatement programs. By merging bottom trawl data from two lake basins of contrasting productivity with life-history information (i.e., tolerances to environmental degradation, diet and temperature preferences), we examined (1) the relationship between system productivity and species richness, (2) whether fish communities are resilient to eutrophication, and (3) whether oligotrophication necessarily leads to reduced sport and commercial fish production. Reduced phosphorus loading has led to fish community rehabilitation. In the productive west basin, six species tolerant of eutrophy (i.e., anoxia, turbidity) declined in abundance, whereas the abundance of three intolerant species increased through time. In the less productive central basin, although only one tolerant species declined, four species intolerant of eutrophic conditions recovered with oligotrophication. These differential responses appear to derive from dissimilar mechanisms by which reduced productivity alters habitat and resource availability for fishes. Specifically, enhanced bottom oxygen, combined with reduced biogenic turbidity and sedimentation, likely drove the loss of tolerant species in the west basin by reducing detrital mass or the ability of these species to compete with intolerant species under conditions of improved water clarity. In contrast, reduced bottom anoxia, which enhanced availability of cool- and cold-water habitat and benthic macroinvertebrate communities, appears important to the recovery of intolerant species in the central basin. Ultimately, these productivity-induced shifts caused species richness to decline in Lake Erie's west basin and to increase in its central basin. Beyond confirming that unimodal models of productivity and species diversity can describe fish community change in a recovering system, our results provide optimism in an otherwise dismal state of affairs in fisheries management (e.g., overexploitation), given that many recovering intolerant species are desired sport or commercial fishes.},
author = {Ludsin, Stuart A. and Kershner, Mark W. and Blocksom, Karen A. and Knight, Roger L. and Stein, Roy A.},
doi = {10.1890/1051-0761(2001)011[0731:LADILE]2.0.CO;2},
file = {:Users/alyssacirtwill/Documents/Papers/Ludsin et al.{\_}2001{\_}Ecological Applications.pdf:pdf},
issn = {10510761},
journal = {Ecological Applications},
keywords = {Detrended correspondence analysis,Eutrophication,Great Lakes,Lake Erie,Oligotrophication,Phosphorus abatement,Productivity,Resilience,Species diversity,Species richness,Species turnover,Succession},
number = {3},
pages = {731--746},
title = {{Life after death in Lake Erie: Nutrient controls drive fish species richness, rehabilitation}},
volume = {11},
year = {2001}
}
@article{Macpherson2002a,
abstract = {The increase in species richness from the poles to the Equator has been observed in numerous terrestrial and aquatic taxa. A number of different hypotheses have been put forward as explanations for this trend, e.g. area and energy availability. However, whether these hypotheses apply to large spatial scales in marine environments remains unclear. The present study shows a clear latitudinal gradient from high to low latitude (from 80°N to 70°S) in marine species richness for 6643 species (fishes and invertebrates) in 10 different taxa dwelling in benthic and pelagic habitats on both sides of the Atlantic. The patterns in benthic taxa are strongly influenced by coastal hydrographic processes, with marked peaks and troughs, and consequently the gradients are not symmetric along both Atlantic sides. Pelagic taxa show a plateau-shaped distribution and the influence from coastal events on gradients could not be demonstrated. The relationships between species richness and different environmental factors indicate that area size does not explain the latitudinal pattern in benthic species richness on a large spatial scale. Sea-surface temperature (positive relationship) is the best predictor of this pattern for benthic species, and nitrate concentration (negative relationship) is the best predictor for pelagic species. The results call into question the existence of a single primary cause that would explain the pattern in marine species richness on a large spatial scale.},
author = {Macpherson, E.},
doi = {10.1098/rspb.2002.2091},
file = {:Users/alyssacirtwill/Documents/Papers/Macpherson{\_}2002{\_}Proceedings. Biological sciences The Royal Society.pdf:pdf},
issn = {14712970},
journal = {Proceedings of the Royal Society B: Biological Sciences},
keywords = {Atlantic Ocean,Diversity gradients,Latitudinal species richness,Marine diversity},
number = {1501},
pages = {1715--1720},
title = {{Large-scale species-richness gradients in the Atlantic Ocean}},
volume = {269},
year = {2002}
}
@article{Maron2019a,
abstract = {More than five decades ago, Ehrlich and Raven proposed a revolutionary idea–that the evolution of novel plant defense could spur adaptive radiation in plants. Despite motivating much work on plant–herbivore coevolution and defense theory, Ehrlich and Raven never proposed a mechanism for their "escape and radiate" model. Recent intriguing mechanisms proposed by Marquis et al. include sympatric divergence, pleiotropic effects of plant defense traits on reproductive isolation, and strong postzygotic isolation, but these may not be general features of herbivore-mediated speciation. An alternate view is that herbivores may impose strong divergent selection on defenses in allopatric plant populations, with plant–herbivore coevolution driving local adaptation resulting in plant speciation. Building on these ideas, we propose three scenarios that consider the role of herbivores in plant speciation. These include (1) vicariance, subsequent coevolution within populations and adaptive divergence between geographically isolated populations, (2) colonization of a new habitat lacking effective herbivores followed by loss of defense and then re-evolution and coevolution of defense in response to novel herbivores, and (3) evolution of a new defense followed by range expansion, vicariance, and coevolution. We discuss the general role of coevolution in plant speciation and consider outstanding issues related to understanding: (1) the mechanisms behind cospeciation of plants and insects, (2) geographic variation in defense phenotypes, (3) how defensive traits and geography map on plant phylogenies, and (4) the role of herbivores in driving character displacement in defense phenotypes of related species in sympatry.},
author = {Maron, John L. and Agrawal, Anurag A. and Schemske, Douglas W.},
doi = {10.1002/ecy.2704},
file = {:Users/alyssacirtwill/Documents/Papers/Maron, Agrawal, Schemske{\_}2019{\_}Ecology.pdf:pdf},
issn = {00129658},
journal = {Ecology},
keywords = {plant defense,plant–herbivore coevolution,plant–insect cospeciation,speciation,vicariance},
number = {7},
title = {{Plant–herbivore coevolution and plant speciation}},
volume = {100},
year = {2019}
}
@article{Fujishima1972,
author = {Fujishima, Akira and Honda, Kenichi},
doi = {10.1038/239137a0},
file = {:Users/alyssacirtwill/Documents/Papers/May{\_}1972{\_}Nature.pdf:pdf},
isbn = {0028-0836},
issn = {0028-0836},
journal = {Nature new biology},
pages = {37--38},
pmid = {4561957},
title = {{{\textcopyright} 1972 Nature Publishing Group}},
volume = {238},
year = {1972}
}
@article{May2013,
author = {May, Robert M},
file = {:Users/alyssacirtwill/Documents/Papers/May{\_}1973{\_}Ecology.pdf:pdf},
number = {3},
pages = {638--641},
title = {{Qualitative Stability in Model Ecosystems Author ( s ): Robert M . May QUALITATIVE STABILITY IN MODEL ECOSYSTEMS '}},
volume = {54},
year = {2013}
}
@article{Mata2019a,
abstract = {DNA metabarcoding is increasingly used in dietary studies to estimate diversity, composition and frequency of occurrence of prey items. However, few studies have assessed how technical and biological replication affect the accuracy of diet estimates. This study addresses these issues using the European free-tailed bat Tadarida teniotis, involving high-throughput sequencing of a small fragment of the COI gene in 15 separate faecal pellets and a 15-pellet pool per each of 20 bats. We investigated how diet descriptors were affected by variability among (a) individuals, (b) pellets of each individual and (c) PCRs of each pellet. In addition, we investigated the impact of (d) analysing separate pellets vs. pellet pools. We found that diet diversity estimates increased steadily with the number of pellets analysed per individual, with seven pellets required to detect {\~{}}80{\%} of prey species. Most variation in diet composition was associated with differences among individual bats, followed by pellets per individual and PCRs per pellet. The accuracy of frequency of occurrence estimates increased with the number of pellets analysed per bat, with the highest error rates recorded for prey consumed infrequently by many individuals. Pools provided poor estimates of diet diversity and frequency of occurrence, which were comparable to analysing a single pellet per individual, and consistently missed the less common prey items. Overall, our results stress that maximizing biological replication is critical in dietary metabarcoding studies and emphasize that analysing several samples per individual rather than pooled samples produce more accurate results.},
author = {Mata, Vanessa A. and Rebelo, Hugo and Amorim, Francisco and McCracken, Gary F. and Jarman, Simon and Beja, Pedro},
doi = {10.1111/mec.14779},
file = {:Users/alyssacirtwill/Documents/Papers/Mata et al.{\_}2018{\_}Molecular Ecology(2).pdf:pdf},
issn = {1365294X},
journal = {Molecular Ecology},
keywords = {bat ecology,metabarcoding,molecular diet analyses,replication,sampling design,trophic ecology},
number = {2},
pages = {165--175},
title = {{How much is enough? Effects of technical and biological replication on metabarcoding dietary analysis}},
volume = {28},
year = {2019}
}
@article{Maynard2018a,
abstract = {Ecological networks that exhibit stable dynamics should theoretically persist longer than those that fluctuate wildly. Thus, network structures which are over-represented in natural systems are often hypothesised to be either a cause or consequence of ecological stability. Rarely considered, however, is that these network structures can also be by-products of the processes that determine how new species attempt to join the community. Using a simulation approach in tandem with key results from random matrix theory, we illustrate how historical assembly mechanisms alter the structure of ecological networks. We demonstrate that different community assembly scenarios can lead to the emergence of structures that are often interpreted as evidence of ‘selection for stability'. However, by controlling for the underlying selection pressures, we show that these assembly artefacts—or spandrels—are completely unrelated to stability or selection, and are instead by-products of how new species are introduced into the system. We propose that these network-assembly spandrels are critically overlooked aspects of network theory and stability analysis, and we illustrate how a failure to adequately account for historical assembly can lead to incorrect inference about the causes and consequences of ecological stability.},
author = {Maynard, Daniel S. and Serv{\'{a}}n, Carlos A. and Allesina, Stefano},
doi = {10.1111/ele.12912},
file = {:Users/alyssacirtwill/Documents/Papers/Maynard, Serv{\'{a}}n, Allesina{\_}2018{\_}Ecology Letters.pdf:pdf},
issn = {14610248},
journal = {Ecology Letters},
keywords = {Coexistence,community assembly,interspecific competition,network structure,stability},
number = {3},
pages = {324--334},
title = {{Network spandrels reflect ecological assembly}},
volume = {21},
year = {2018}
}
@article{McCann2005a,
abstract = {The dynamics of ecological systems include a bewildering number of biotic interactions that unfold over a vast range of spatial scales. Here, employing simple and general empirical arguments concerning the nature of movement, trophic position and behaviour we outline a general theory concerning the role of space and food web structure on food web stability. We argue that consumers link food webs in space and that this spatial structure combined with relatively rapid behavioural responses by consumers can strongly influence the dynamics of food webs. Employing simple spatially implicit food web models, we show that large mobile consumers are inordinately important in determining the stability, or lack of it, in ecosystems. More specifically, this theory suggests that mobile higher order organisms are potent stabilizers when embedded in a variable, and expansive spatial structure. However, when space is compressed and higher order consumers strongly couple local habitats then mobile consumers can have an inordinate destabilizing effect. Preliminary empirical arguments show consistency with this general theory. {\textcopyright}2005 Blackwell Publishing Ltd/CNRS.},
author = {McCann, K. S. and Rasmussen, J. B. and Umbanhowar, J.},
doi = {10.1111/j.1461-0248.2005.00742.x},
file = {:Users/alyssacirtwill/Documents/Papers/McCann, Rasmussen, Umbanhowar{\_}2005{\_}Ecology Letters.pdf:pdf},
issn = {1461023X},
journal = {Ecology Letters},
keywords = {Compartment,Consumer-resource interaction,Food web,Foraging,Scale,Space,Stability},
number = {5},
pages = {513--523},
pmid = {21352455},
title = {{The dynamics of spatially coupled food webs}},
volume = {8},
year = {2005}
}
@article{Melian2009a,
abstract = {Most studies on ecological networks consider only a single interaction type (e.g. competitive, predatory or mutualistic), and try to developrules for system stability based exclusively on properties of this interaction type. However, the stability of ecological networks may be more dependent on the way different interaction types are combined in real communities. To address this issue, we start by compiling an ecological network in the Do{\~{n}}ana Biological Reserve, southern Spain, with 390 species and 798 mu-tualistic and antagonistic interactions. We characterize network structure by looking at how mutualistic and antagonistic interactions are combined across all plant species. Both the ratio of mutualistic to antagonistic interactions per plant, and the number of basic modules with an antagonistic and a mutualistic interaction are very heterogeneous across plant species, with a few plant species showing very high values for these parameters. To assess the implications of these network patterns on species diversity, we study analytically and by simulation a model of this ecological network. We find that the observed correlation between strong interaction strengths and high mutualistic to antagonistic ratios in a few plant species significantly increases community diversity. Thus, to predict the persistence of biodiversity we need to understand how interaction strength and the architecture of ecological networks with different interaction types are combined. {\textcopyright} 2008 The Authors.},
author = {Meli{\'{a}}n, Carlos J. and Bascompte, Jordi and Jordano, Pedro and Křivan, Vlastimil},
doi = {10.1111/j.1600-0706.2008.16751.x},
file = {:Users/alyssacirtwill/Documents/Papers/Meli{\'{a}}n et al.{\_}2009{\_}Oikos.pdf:pdf},
issn = {00301299},
journal = {Oikos},
number = {1},
pages = {122--130},
title = {{Diversity in a complex ecological network with two interaction types}},
volume = {118},
year = {2009}
}
@article{Moran2003a,
author = {Moran, Matthew D.},
doi = {10.1034/j.1600-0706.2003.12010.x},
file = {:Users/alyssacirtwill/Documents/Papers/Moran{\_}2003{\_}Oikos.pdf:pdf},
issn = {00301299},
journal = {Oikos},
number = {2},
pages = {403--405},
title = {{Arguments for rejecting the sequential bonferroni in ecological studies}},
volume = {100},
year = {2003}
}
@article{Dunne2002d,
abstract = {Food-web structure mediates dramatic effects of biodiversity loss including secondary and 'cascading' extinctions. We studied these effects by simulating primary species loss in 16 food webs from terrestrial and aquatic ecosystems and measuring robustness in terms of the secondary extinctions that followed. As observed in other networks, food webs are more robust to random removal of species than to selective removal of species with the most trophic links to other species. More surprisingly, robustness increases with food-web connectance but appears independent of species richness and omnivory. In particular, food webs experience 'rivet-like' thresholds past which they display extreme sensitivity to removal of highly connected species. Higher connectance delays the onset of this threshold. Removing species with few trophic connections generally has little effect though there are several striking exceptions. These findings emphasize how the number of species removed affects ecosystems differently depending on the trophic functions of species removed.},
author = {Dunne, Jennifer A. and Williams, Richard J. and Martinez, Neo D.},
doi = {10.1046/j.1461-0248.2002.00354.x},
file = {:Users/alyssacirtwill/Documents/Papers/Dunne, Williams, Martinez{\_}2002{\_}Ecology Letters(2).pdf:pdf},
issn = {1461023X},
journal = {Ecology Letters},
keywords = {Biodiversity,Connectance,Ecosystem function,Food web,Network structure,Robustness,Secondary extinctions,Species loss,Species richness,Topology},
number = {4},
pages = {558--567},
title = {{Network structure and biodiversity loss in food webs: Robustness increases with connectance}},
volume = {5},
year = {2002}
}
@article{Vargas1998,
abstract = {Treatment of carbonyl substrates with 2-mercaptoethanol in the presence of Tonsil Actisil FF (TAAF) as the catalyst, affords the corresponding oxathiolanes. Reaction yields are good and the work-up is very simple.},
author = {Vargas, M. and Arroyo, G. A. and Miranda, R. and Aceves, J. M. and Velasco, B. and Delgado, F.},
doi = {10.1515/HC.1998.4.1.21},
file = {:Users/alyssacirtwill/Documents/Papers/Dunne, Williams, Martinez{\_}2002{\_}Ecology Letters.pdf:pdf},
issn = {07930283},
journal = {Heterocyclic Communications},
keywords = {2002,5,558,567,biodiversity,connectance,ecology letters,ecosystem function,food web,network structure,robustness,secondary extinctions,species loss,species richness,topology},
number = {1},
pages = {21--24},
title = {{A facile method to obtain oxathiolanes by bentonitic earth (TAAF) catalyst}},
volume = {4},
year = {1998}
}
@article{Eklof2013b,
abstract = {Summary: Ecological communities are composed of populations connected in tangled networks of ecological interactions. Therefore, the extinction of a species can reverberate through the network and cause other (possibly distantly connected) species to go extinct as well. The study of these secondary extinctions is a fertile area of research in ecological network theory. However, to facilitate practical applications, several improvements to the current analytical approaches are needed. In particular, we need to consider that (i) species have different 'a priori' probabilities of extinction, (ii) disturbances can simultaneously affect several species, and (iii) extinction risk of consumers likely grows with resource loss. All these points can be included in dynamical models, which are, however, difficult to parameterize. Here we advance the study of secondary extinctions with Bayesian networks. We show how this approach can account for different extinction responses using binary - where each resource has the same importance - and quantitative data - where resources are weighted by their importance. We simulate ecological networks using a popular dynamical model (the Allometric Trophic Network model) and use it to test our method. We find that the Bayesian network model captures the majority of the secondary extinctions produced by the dynamical model and that consumers' responses to species loss are best modelled using a nonlinear sigmoid function. We also show that an approach based exclusively on food web structure loses power when species at higher trophic levels are preferentially lost. Because the loss of apex predators is unfortunately widespread, the results highlight a serious limitation of studies on network robustness. {\textcopyright} 2013 British Ecological Society.},
author = {Ekl{\"{o}}f, Anna and Tang, Si and Allesina, Stefano},
doi = {10.1111/2041-210X.12062},
file = {:Users/alyssacirtwill/Documents/Papers/Ekl{\"{o}}f, Tang, Allesina{\_}2013{\_}Methods in Ecology and Evolution.pdf:pdf},
issn = {2041210X},
journal = {Methods in Ecology and Evolution},
keywords = {Bayesian networks,Biodiversity loss,Cascading extinctions,Dynamical model,Food webs},
number = {8},
pages = {760--770},
title = {{Secondary extinctions in food webs: A Bayesian network approach}},
volume = {4},
year = {2013}
}
@article{Frost2016a,
abstract = {Species have strong indirect effects on others, and predicting these effects is a central challenge in ecology. Prey species sharing an enemy (predator or parasitoid) can be linked by apparent competition, but it is unknown whether this process is strong enough to be a community-wide structuring mechanism that could be used to predict future states of diverse food webs. Whether species abundances are spatially coupled by enemy movement across different habitats is also untested. Here, using a field experiment, we show that predicted apparent competitive effects between species, mediated via shared parasitoids, can significantly explain future parasitism rates and herbivore abundances. These predictions are successful even across edges between natural and managed forests, following experimental reduction of herbivore densities by aerial spraying of insecticide over 20 hectares. This result shows that trophic indirect effects propagate across networks and habitats in important, predictable ways, with implications for landscape planning, invasion biology and biological control.},
author = {Frost, Carol M. and Peralta, Guadalupe and Rand, Tatyana A. and Didham, Raphael K. and Varsani, Arvind and Tylianakis, Jason M.},
doi = {10.1038/ncomms12644},
file = {:Users/alyssacirtwill/Documents/Papers/Frost et al.{\_}2016{\_}Nature Communications.pdf:pdf},
issn = {20411723},
journal = {Nature Communications},
pages = {1--12},
publisher = {Nature Publishing Group},
title = {{Apparent competition drives community-wide parasitism rates and changes in host abundance across ecosystem boundaries}},
url = {http://dx.doi.org/10.1038/ncomms12644},
volume = {7},
year = {2016}
}
@book{Gaba2018a,
abstract = {Due to increasing compost use in agriculture, there is an urgent need to evaluate compost bene fi ts and impacts versus other fertilizers. Here we review the recent progress made in the quanti fi cation of positive effects associated with compost use on land using life cycle assessment (LCA), an internationally recognised environmental tool. Nine environmental bene fi ts were identi fi ed in an extensive literature review: nutrient supply, carbon sequestration, weed pest and disease suppression, increase in crop yield, decreased soil erosion, retention of soil moisture, increased soil workability, enhanced soil biological properties and biodiversity, and gain in crop nutritional quality. Quantitative fi gures for each bene fi t were drawn from the literature and classi fi ed into short-term: less than 1 year; mid-term: less than 10 years and long-term: less than 100 years.},
author = {Gaba, Sabrina and Alignier, Audrey and Aviron, St{\'{e}}phanie and Barot, S{\'{e}}bastien and Blouin, Manuel and Hedde, Micka{\"{e}}l and Jabot, Franck and Vergnes, Alan and Bonis, Anne and Bonthoux, S{\'{e}}bastien and Bourgeois, B{\'{e}}renger and Bretagnolle, Vincent and Catarino, Rui and Coux, Camille and Gardarin, Antoine and Giffard, Brice and {Le Gal}, Antoine and Lecomte, Jane and Miguet, Paul and Piutti, S{\'{e}}verine and Rusch, Adrien and Zwicke, Marine and Couvet, Denis},
doi = {10.1007/978-3-319-90309-5_1},
file = {:Users/alyssacirtwill/Documents/Papers/Gaba et al.{\_}2018{\_}Unknown.pdf:pdf},
isbn = {9783319903095},
number = {January},
pages = {1--46},
title = {{Ecology for Sustainable and Multifunctional Agriculture}},
year = {2018}
}
@article{Giling2019a,
abstract = {Changes in the diversity of plant communities may undermine the economically and environmentally important consumer species they support. The structure of trophic interactions determines the sensitivity of food webs to perturbations, but rigorous assessments of plant diversity effects on network topology are lacking. Here, we use highly resolved networks from a grassland biodiversity experiment to test how plant diversity affects the prevalence of different food web motifs, the smaller recurrent sub-networks that form the building blocks of complex networks. We find that the representation of tri-trophic chain, apparent competition and exploitative competition motifs increases with plant species richness, while the representation of omnivory motifs decreases. Moreover, plant species richness is associated with altered patterns of local interactions among arthropod consumers in which plants are not directly involved. These findings reveal novel structuring forces that plant diversity exerts on food webs with potential implications for the persistence and functioning of multitrophic communities.},
author = {Giling, Darren P. and Ebeling, Anne and Eisenhauer, Nico and Meyer, Sebastian T. and Roscher, Christiane and Rzanny, Michael and Voigt, Winfried and Weisser, Wolfgang W. and Hines, Jes},
doi = {10.1038/s41467-019-08856-0},
file = {:Users/alyssacirtwill/Documents/Papers/Giling et al.{\_}2019{\_}Nature Communications.pdf:pdf},
issn = {20411723},
journal = {Nature Communications},
number = {1},
pages = {1--7},
publisher = {Springer US},
title = {{Plant diversity alters the representation of motifs in food webs}},
url = {http://dx.doi.org/10.1038/s41467-019-08856-0},
volume = {10},
year = {2019}
}
@article{Guimaraes2007a,
abstract = {The structure of mutualistic networks provides clues to processes shaping biodiversity [1-10]. Among them, interaction intimacy, the degree of biological association between partners, leads to differences in specialization patterns [4, 11] and might affect network organization [12]. Here, we investigated potential consequences of interaction intimacy for the structure and coevolution of mutualistic networks. From observed processes of selection on mutualistic interactions, it is expected that symbiotic interactions (high-interaction intimacy) will form species-poor networks characterized by compartmentalization [12, 13], whereas nonsymbiotic interactions (low intimacy) will lead to species-rich, nested networks in which there is a core of generalists and specialists often interact with generalists [3, 5, 7, 12, 14]. We demonstrated an association between interaction intimacy and structure in 19 ant-plant mutualistic networks. Through numerical simulations, we found that network structure of different forms of mutualism affects evolutionary change in distinct ways. Change in one species affects primarily one mutualistic partner in symbiotic interactions but might affect multiple partners in nonsymbiotic interactions. We hypothesize that coevolution in symbiotic interactions is characterized by frequent reciprocal changes between few partners, but coevolution in nonsymbiotic networks might show rare bursts of changes in which many species respond to evolutionary changes in a single species. {\textcopyright} 2007 Elsevier Ltd. All rights reserved.},
author = {Guimar{\~{a}}es, Paulo R. and Rico-Gray, Victor and Oliveira, Paulo S S. and Izzo, Thiago J. and dos Reis, S{\'{e}}rgio F. and Thompson, John N.},
doi = {10.1016/j.cub.2007.09.059},
file = {:Users/alyssacirtwill/Documents/Papers/Guimar{\~{a}}es et al.{\_}2007{\_}Current Biology.pdf:pdf},
issn = {09609822},
journal = {Current Biology},
keywords = {EVO{\_}ECOL},
number = {20},
pages = {1797--1803},
title = {{Interaction Intimacy Affects Structure and Coevolutionary Dynamics in Mutualistic Networks}},
volume = {17},
year = {2007}
}
@article{Hembry2018a,
abstract = {Biological intimacy—the degree of physical proximity or integration of partner taxa during their life cycles—is thought to promote the evolution of reciprocal specialization and modularity in the networks formed by co-occurring mutualistic species, but this hypothesis has rarely been tested. Here, we test this “biological intimacy hypothesis” by comparing the network architecture of brood pollination mutualisms, in which specialized insects are simultaneously parasites (as larvae) and pollinators (as adults) of their host plants to that of other mutualisms which vary in their biological intimacy (including ant-myrmecophyte, ant-extrafloral nectary, plant-pollinator and plant-seed disperser assemblages). We use a novel dataset sampled from leafflower trees (Phyllanthaceae: Phyllanthus s. l. [Glochidion]) and their pollinating leafflower moths (Lepidoptera: Epicephala) on three oceanic islands (French Polynesia) and compare it to equivalent published data from congeners on continental islands (Japan). We infer taxonomic diversity of leafflower moths using multilocus molecular phylogenetic analysis and examine several network structural properties: modularity (compartmentalization), reciprocality (symmetry) of specialization and algebraic connectivity. We find that most leafflower-moth networks are reciprocally specialized and modular, as hypothesized. However, we also find that two oceanic island networks differ in their modularity and reciprocal specialization from the others, as a result of a supergeneralist moth taxon which interacts with nine of 10 available hosts. Our results generally support the biological intimacy hypothesis, finding that leafflower-moth networks (usually) share a reciprocally specialized and modular structure with other intimate mutualisms such as ant-myrmecophyte symbioses, but unlike nonintimate mutualisms such as seed dispersal and nonintimate pollination. Additionally, we show that generalists—common in nonintimate mutualisms—can also evolve in intimate mutualisms, and that their effect is similar in both types of assemblages: once generalists emerge they reshape the network organization by connecting otherwise isolated modules.},
author = {Hembry, David H. and Raimundo, Rafael L.G. and Newman, Erica A. and Atkinson, Lesje and Guo, Chang and Guimar{\~{a}}es, Paulo R. and Gillespie, Rosemary G.},
doi = {10.1111/1365-2656.12841},
file = {:Users/alyssacirtwill/Documents/Papers/Hembry et al.{\_}2018{\_}Journal of Animal Ecology.PDF:PDF},
issn = {13652656},
journal = {Journal of Animal Ecology},
keywords = {Epicephala,Glochidion,Phyllanthus,biological intimacy hypothesis,co-evolution,modularity,network evolution,reciprocal specialization},
number = {4},
pages = {1160--1171},
title = {{Does biological intimacy shape ecological network structure? A test using a brood pollination mutualism on continental and oceanic islands}},
volume = {87},
year = {2018}
}
@article{Holt2009,
abstract = {Indirect interactions are almost inevitable in any multi-species community. Understanding the implications of such interactions is a challenging task, in light of the very large number of ways species can be tied together in complex food webs. One approach to this complexity is to focus on strong interactions among a relatively small number (e.g. 3-6) of species interacting in defined configurations: community modules. In recent years, the discipline of community ecology has developed a substantial body of theory focused on such modules. Modules often clearly describe the basic features of empirical systems, particularly in simplified anthropogenic landscapes, and also help to isolate and characterize key processes driving the dynamics of more complex communities. In this chapter, we draw out a number of insights from ecological studies of modules which we believe are relevant to biological control. We emphasize in particular the module of 'shared predation', where a natural enemy attacks two or more species of prey. Theoretical studies suggest a number of 'rules of thumb', including: (i) the greatest risk to non-targets may occur from control agents that are only moderately effective on the target; (ii) targets with a high reproductive capacity can indirectly endanger non-targets; (iii) there can be transient phases of extinction risk for non-targets during the establishment phase of control agents, particularly for species with high attack rates; (iv) at a landscape scale, mobile agents can endanger the fate of non-targets at sites other than the area of control; (v) using specialist natural enemies can pose risks to non-targets, if there are generalist resident predators/ parasitoids which can exploit these introduced agents. The theoretical models help to highlight circumstances when these effects should be particularly strong.},
author = {Holt, R. D. and Hochberg, M. E.},
doi = {10.1079/9780851994536.0013},
file = {:Users/alyssacirtwill/Documents/Papers/Holt et al.{\_}2001{\_}Evaluating Indirect Ecological Effects of Biological Control.pdf:pdf},
journal = {Evaluating indirect ecological effects of biological control. Key papers from the symposium "Indirect ecological effects in biological control", Montpellier, France, 17-20 October 1999},
number = {January 2001},
pages = {13--37},
title = {{Indirect interactions, community modules and biological control: a theoretical perspective.}},
year = {2009}
}
@article{Ives1987,
abstract = {Derives a model that has one predator and one prey species; the prey species can display antipredator behavior, thereby decreasing their chance of being captured by the predator, but they must pay for this protection by a cost exacted through decreased fecundity or increased mortality caused by factors other than predation. The population-dynamic consequences of antipredator behaviors are explored by comparing systems in which the efficiencies of the antipredator behavior differ; as the antipredator behavior becomes more efficient, the prey need to invest less in order to achieve the same level of protection from the predator. For any degree of efficiency, the prey choose their level of investment in antipredator behavior in order to optimize their expected reproductive fitness. By assuming only that the predators and prey coexist and that there is a stable equilibrium, increased efficiency of antipredator behaviors increases prey densities and decreases the ratio of predator-to-prey densities. This is true even though the prey's level of investment in antipredator behavior initially increases and then decreases with increasing efficiency of the antipredator behavior. Consequently, the effect of antipredator behaviors on population densities cannot be inferred from the level of prey investment in these behaviors. Antipredator behaviors also tend to decrease the oscillatory dynamics inherent in model predator-prey systems.-from Authors},
author = {Ives, A. R. and Dobson, A. P.},
doi = {10.1086/284719},
file = {:Users/alyssacirtwill/Documents/Papers/Ives, Dobson{\_}2002{\_}The American Naturalist.pdf:pdf},
issn = {00030147},
journal = {American Naturalist},
number = {3},
pages = {431--447},
title = {{Antipredator behavior and the population dynamics of simple predator-prey systems.}},
volume = {130},
year = {1987}
}
@article{Jinks2019a,
abstract = {Structural habitat complexity is a fundamental attribute influencing ecological food webs. Simplification of complex habitats occurs due to both natural and anthropogenic pressures that can alter productivity of food webs. Relationships between food web structure and habitat complexity may be influenced by multiple mechanisms, and untangling these can be challenging. We investigated whether (1) size spectra vary across a gradient of habitat complexity in seagrass meadows and (2) structural complexity changes the importance of different primary producers supporting the food web (determined using stable isotope analysis) in the Great Barrier Reef World Heritage Area. We found that moderately complex meadows had much steeper size spectra slopes, caused by a higher abundance of smaller animals and fewer larger animals, while meadows on either end of the complexity scale (low and a single meadow with very high complexity) had shallower slopes, indicative of a more balanced distribution of animal sizes across the spectrum. We also found that the importance of epiphytic algae as a food source was high in most meadows, despite the increase in seagrass surface area on which epiphytes could grow. The consistent importance of epiphytic algae suggests that the changes in the availability of different potential food sources did not affect food web structure. Our findings indicate that food web structure may change with variations in structural complexity because of changes in the abundance of smaller and/or larger animals. Food web structure and food sources are important determinants of the dynamic stability of food webs. Size spectra analysis is already used as a monitoring tool for assessing populations of key fisheries species in commercial fishing operations, and thus, we recommend using size spectra as a proxy for assessing the structure of the food webs in different types of seagrass meadows. Size spectra may be a useful indicator of how different meadows provide for ecosystem services such as fisheries.},
author = {Jinks, Kristin I. and Brown, Christopher J. and Rasheed, Michael A. and Scott, Abigail L. and Sheaves, Marcus and York, Paul H. and Connolly, Rod M.},
doi = {10.1002/ecs2.2928},
file = {:Users/alyssacirtwill/Documents/Papers/Jinks et al.{\_}2019{\_}Ecosphere.pdf:pdf},
issn = {21508925},
journal = {Ecosphere},
keywords = {Great Barrier Reef,abundance–biomass size spectra,habitat complexity,predator–prey interactions,seagrass,size spectra,stable isotope analysis,structural complexity},
number = {11},
title = {{Habitat complexity influences the structure of food webs in Great Barrier Reef seagrass meadows}},
volume = {10},
year = {2019}
}
@article{Sciences2015,
author = {Sciences, Plant},
file = {:Users/alyssacirtwill/Documents/Papers/Sciences{\_}2015{\_}Unknown.pdf:pdf},
pages = {1--3},
title = {{Department of Animal and Plant Sciences}},
year = {2015}
}
@article{Zool1968,
author = {Zool, J},
file = {:Users/alyssacirtwill/Documents/Papers/Jones{\_}1968{\_}Journal of the Zoological Society of London.pdf:pdf},
pages = {363--376},
title = {{carnivorous intertidal isopod Euvydice}},
year = {1968}
}
@article{Moran2015,
abstract = {With the increasing appreciation for the crucial roles that microbial symbionts play in the development and fitness of plant and animal hosts, there has been a recent push to interpret evolution through the lens of the “hologenome”—the collective genomic content of a host and its microbiome. But how symbionts evolve and, particularly, whether they undergo natural selection to benefit hosts are complex issues that are associated with several misconceptions about evolutionary processes in host-associated microbial communities. Microorganisms can have intimate, ancient, and/or mutualistic associations with hosts without having undergone natural selection to benefit hosts. Likewise, observing host-specific microbial community composition or greater community similarity among more closely related hosts does not imply that symbionts have coevolved with hosts, let alone that they have evolved for the benefit of the host. Although selection at the level of the symbiotic community, or hologenome, occurs in some cases, it should not be accepted as the null hypothesis for explaining features of host–symbiont associations.},
author = {Moran, Nancy A. and Sloan, Daniel B.},
doi = {10.1371/journal.pbio.1002311},
file = {:Users/alyssacirtwill/Documents/Papers/journal.pbio.1002311.PDF:PDF},
issn = {15457885},
journal = {PLoS Biology},
number = {12},
pages = {1--10},
title = {{The Hologenome Concept: Helpful or Hollow?}},
volume = {13},
year = {2015}
}
@article{Sellman2018,
abstract = {Numerical models for simulating outbreaks of infectious diseases are powerful tools for informing surveillance and control strategy decisions. However, large-scale spatially explicit models can be limited by the amount of computational resources they require, which poses a problem when multiple scenarios need to be explored to provide policy recommendations. We introduce an easily implemented method that can reduce computation time in a standard Susceptible-Exposed-Infectious-Removed (SEIR) model without introducing any further approximations or truncations. It is based on a hierarchical infection process that operates on entire groups of spatially related nodes (cells in a grid) in order to efficiently filter out large volumes of susceptible nodes that would otherwise have required expensive calculations. After the filtering of the cells, only a subset of the nodes that were originally at risk are then evaluated for actual infection. The increase in efficiency is sensitive to the exact configuration of the grid, and we describe a simple method to find an estimate of the optimal configuration of a given landscape as well as a method to partition the landscape into a grid configuration. To investigate its efficiency, we compare the introduced methods to other algorithms and evaluate computation time, focusing on simulated outbreaks of foot-and-mouth disease (FMD) on the farm population of the USA, the UK and Sweden, as well as on three randomly generated populations with varying degree of clustering. The introduced method provided up to 500 times faster calculations than pairwise computation, and consistently performed as well or better than other available methods. This enables large scale, spatially explicit simulations such as for the entire continental USA without sacrificing realism or predictive power.},
author = {Sellman, Stefan and Tsao, Kimberly and Tildesley, Michael J. and Brommesson, Peter and Webb, Colleen T. and Wennergren, Uno and Keeling, Matt J. and Lindstr{\"{o}}m, Tom},
doi = {10.1371/journal.pcbi.1006086},
file = {:Users/alyssacirtwill/Documents/Papers/journal.pcbi.1006086.pdf:pdf},
isbn = {1111111111},
issn = {15537358},
journal = {PLoS Computational Biology},
number = {4},
pages = {1--27},
title = {{Need for speed: An optimized gridding approach for spatially explicit disease simulations}},
volume = {14},
year = {2018}
}
@article{Karinho-Betancourt2015a,
abstract = {Summary: Theory predicts patterns of defense across taxa based on notions of tradeoffs and synergism among defensive traits when plants and herbivores coevolve. Because the expression of characters changes ontogenetically, the evolution of plant strategies may be best understood by considering multiple traits along a trajectory of plant development. Here we addressed the ontogenetic expression of chemical and physical defenses in 12 Datura species, and tested for macroevolutionary correlations between defensive traits using phylogenetic analyses. We used liquid chromatography coupled to mass spectrometry to identify the toxic tropane alkaloids of Datura, and also estimated leaf trichome density. We report three major patterns. First, we found different ontogenetic trajectories of alkaloids and leaf trichomes, with alkaloids increasing in concentration at the reproductive stage, whereas trichomes were much more variable across species. Second, the dominant alkaloids and leaf trichomes showed correlated evolution, with positive and negative associations. Third, the correlations between defensive traits changed across ontogeny, with significant relationships only occurring during the juvenile phase. The patterns in expression of defensive traits in the genus Datura are suggestive of adaptation to complex selective environments varying in space and time.},
author = {Kari{\~{n}}ho-Betancourt, Eunice and Agrawal, Anurag A. and Halitschke, Rayko and N{\'{u}}{\~{n}}ez-Farf{\'{a}}n, Juan},
doi = {10.1111/nph.13300},
file = {:Users/alyssacirtwill/Documents/Papers/Kari{\~{n}}ho-Betancourt et al.{\_}2015{\_}New Phytologist.pdf:pdf},
issn = {14698137},
journal = {New Phytologist},
keywords = {Chemical ecology,Comparative method,Datura,Leaf trichome,Ontogeny,Plant defense,Tradeoffs,Tropane alkaloids},
number = {2},
pages = {796--806},
title = {{Phylogenetic correlations among chemical and physical plant defenses change with ontogeny}},
volume = {206},
year = {2015}
}
@article{Kestemont1989a,
abstract = {The suitability and nutritional value of the euryhaline rotifer Brachionus plicatilis, fed on baker's yeast, for the larval stage of a small European cyprinid, the gudgeon Gobio gobio L., were investigated. Several parameters such as level and frequency of feeding, maximal food intake and feed conversion rate (FCR) were determined during the first month of larval rearing. The survival rate of the larvae was very high, generally more than 90{\%}. The best growth was attained with the highest daily ration (from 2500 rotifers per larva during the first week of rearing to 5500 rotifers per larva during the fourth) and a 4 times-a-day feeding frequency, with a FCR of 0.86. From a mean initial body weight of 0.5 mg at hatching, the larvae reached 17.5 mg in 3 weeks. However, a reduction in the growth rate was observed after the second week, indicating a nutritional deficiency of the yeast-fed rotifers or a too small size of this live food. When the fishes are larger than 10-12 mm (after about 10-15 days of feeding), they would probably grow better on a larger prey such as Artemia salina or on an appropriate dry food. {\textcopyright} 1989.},
author = {Kestemont, Patrick and Awa{\"{i}}ss, Aboubacar},
doi = {10.1016/0044-8486(89)90042-2},
file = {:Users/alyssacirtwill/Documents/Papers/Kestemont, Awa{\"{i}}ss{\_}1989{\_}Aquaculture.pdf:pdf},
issn = {00448486},
journal = {Aquaculture},
number = {3-4},
pages = {305--318},
title = {{Larval rearing of the gudgeon, Gobio gobio L., under optimal conditions of feeding with the rotifer, Brachionus plicatilis O.F. M{\"{u}}ller}},
volume = {83},
year = {1989}
}
@article{Schleuning2012a,
abstract = {Species-rich tropical communities are expected to be more specialized than their temperate counterparts [1-3]. Several studies have reported increasing biotic specialization toward the tropics [4-7], whereas others have not found latitudinal trends once accounting for sampling bias [8, 9] or differences in plant diversity [10, 11]. Thus, the direction of the latitudinal specialization gradient remains contentious. With an unprecedented global data set, we investigated how biotic specialization between plants and animal pollinators or seed dispersers is associated with latitude, past and contemporary climate, and plant diversity. We show that in contrast to expectation, biotic specialization of mutualistic networks is significantly lower at tropical than at temperate latitudes. Specialization was more closely related to contemporary climate than to past climate stability, suggesting that current conditions have a stronger effect on biotic specialization than historical community stability. Biotic specialization decreased with increasing local and regional plant diversity. This suggests that high specialization of mutualistic interactions is a response of pollinators and seed dispersers to low plant diversity. This could explain why the latitudinal specialization gradient is reversed relative to the latitudinal diversity gradient. Low mutualistic network specialization in the tropics suggests higher tolerance against extinctions in tropical than in temperate communities. {\textcopyright} 2012 Elsevier Ltd.},
author = {Schleuning, Matthias and Fr{\"{u}}nd, Jochen and Klein, Alexandra Maria and Abrahamczyk, Stefan and Alarc{\'{o}}n, Ruben and Albrecht, Matthias and Andersson, Georg K.S. and Bazarian, Simone and B{\"{o}}hning-Gaese, Katrin and Bommarco, Riccardo and Dalsgaard, Bo and Dehling, D. Matthias and Gotlieb, Ariella and Hagen, Melanie and Hickler, Thomas and Holzschuh, Andrea and Kaiser-Bunbury, Christopher N. and Kreft, Holger and Morris, Rebecca J. and Sandel, Brody and Sutherland, William J. and Svenning, Jens Christian and Tscharntke, Teja and Watts, Stella and Weiner, Christiane N. and Werner, Michael and Williams, Neal M. and Winqvist, Camilla and Dormann, Carsten F. and Bl{\"{u}}thgen, Nico},
doi = {10.1016/j.cub.2012.08.015},
file = {:Users/alyssacirtwill/Documents/Papers/Schleuning et al.{\_}2012{\_}Current Biology.pdf:pdf},
issn = {09609822},
journal = {Current Biology},
number = {20},
pages = {1925--1931},
title = {{Specialization of mutualistic interaction networks decreases toward tropical latitudes}},
volume = {22},
year = {2012}
}
@article{McLeod2020a,
abstract = {Understanding drivers of antagonistic interactions across temporal and spatial scales is important for predicting community structure. In particular, studies examining spatial variation in ecological networks are critical for anticipating community responses to anthropogenic change. Most studies examining spatial interaction turnover focus on bipartite networks begging the question of whether the results are also reflected in unipartite, multi-trophic networks. To examine the spatial turnover in food web interactions, the environmental and ecological drivers of this, and the influence of interaction turnover on the preservation of individual species' roles, we used a spatially expansive multi-trophic antagonistic ecological network data set of 129 lakes spanning over 1000 kms. We used $\beta$-diversity metrics to quantify spatial turnover in interactions and calculated the relative contributions of interaction rewiring and turnover in top, intermediate, and basal species to network turnover. We then investigated the relative and combined role of multiple ecological drivers (e.g., abundance, thermal tolerance) and environmental drivers (e.g., latitude, total phosphorus) on internal network structure. Finally, we used a motif analysis to measure the effect of spatial interaction turnover on the variation in individual species' roles. We observed high interaction turnover across lakes, driven primarily by turnover in basal species but also the rewiring of interactions among shared species, driven, in part by underlying environmental gradients (e.g., species richness). Contrary to previous food web models applied to single sites, none of the ecological drivers we considered were effective predictors of lake-specific interactions perhaps indicating an important distinction between network model accuracy at regional and local extents. Finally, despite high spatial turnover in interactions, species' roles were highly conserved across the study lakes demonstrating the potential of species' roles for predicting community structure. These findings demonstrate how integrating species' fundamental roles into trait-based approaches may improve our predictions of ecological networks at local scales.},
author = {McLeod, Anne M. and Leroux, Shawn J. and Chu, Cindy},
doi = {10.1002/ecs2.3018},
file = {:Users/alyssacirtwill/Documents/Papers/Section, Leroux{\_}2020{\_}Unknown.pdf:pdf},
issn = {21508925},
journal = {Ecosphere},
keywords = {abundance,beta diversity,competition,food webs,interaction turnover,morphology,phylogeny,thermal tolerance},
number = {2},
title = {{Effects of species traits, motif profiles, and environment on spatial variation in multi-trophic antagonistic networks}},
volume = {11},
year = {2020}
}
@article{Seed1969a,
abstract = {Significant differences in the infection of M. edulis and the “Padstow type” mussel with P. pisum are recorded, and some possible explanations for these differences are discussed. Both types of Mytilus from the mid and lower regions of the mussel bed showed heavier infections than mussels higher on the shore. Even so, the differences between the two types were still maintained. A relationship exists between crab and mussel size, larger crabs being found only in larger hosts. The smallest mussel found to be infected with Pinnotheres measured 3{\textperiodcentered}35 cm in length. Infection in M. edulis was found to increase with increased size of host, the largest occurring mussels having from 80 to 100{\%} infection. Larger mussels occurred in greater numbers in the low shore. It is assumed that infection in the “Padstow type” would show a similar relationship if sufficient recordings had been available. The presence of the crab causes gill damage, and infected mussels show considerably lower tissue weights and slightly greater shell weights than uninfected mussels of similar size. The presence of the crab does not appear to influence the reproductive capacity of the mussel. Copyright {\textcopyright} 1969, Wiley Blackwell. All rights reserved},
author = {Seed, R.},
doi = {10.1111/j.1469-7998.1969.tb02158.x},
file = {:Users/alyssacirtwill/Documents/Papers/Seed{\_}1969{\_}Journal of Zoology.pdf:pdf},
issn = {14697998},
journal = {Journal of Zoology},
number = {4},
pages = {413--420},
title = {{The incidence of the Pea crab, Pinnotheres pisum in the two types of Mytilus (Mollusca: Bivalvia) from Padstow, south‐west England}},
volume = {158},
year = {1969}
}
@article{Simmons2019a,
abstract = {Indirect interactions play an essential role in governing population, community and coevolutionary dynamics across a diverse range of ecological communities. Such communities are widely represented as bipartite networks: graphs depicting interactions between two groups of species, such as plants and pollinators or hosts and parasites. For over thirty years, studies have used indices, such as connectance and species degree, to characterise the structure of these networks and the roles of their constituent species. However, compressing a complex network into a single metric necessarily discards large amounts of information about indirect interactions. Given the large literature demonstrating the importance and ubiquity of indirect effects, many studies of network structure are likely missing a substantial piece of the ecological puzzle. Here we use the emerging concept of bipartite motifs to outline a new framework for bipartite networks that incorporates indirect interactions. While this framework is a significant departure from the current way of thinking about bipartite ecological networks, we show that this shift is supported by analyses of simulated and empirical data. We use simulations to show how consideration of indirect interactions can highlight differences missed by the current index paradigm that may be ecologically important. We extend this finding to empirical plant–pollinator communities, showing how two bee species, with similar direct interactions, differ in how specialised their competitors are. These examples underscore the need to not rely solely on network- and species-level indices for characterising the structure of bipartite ecological networks.},
author = {Simmons, Benno I. and Cirtwill, Alyssa R. and Baker, Nick J. and Wauchope, Hannah S. and Dicks, Lynn V. and Stouffer, Daniel B. and Sutherland, William J.},
doi = {10.1111/oik.05670},
file = {:Users/alyssacirtwill/Documents/Papers/Simmons et al.{\_}2019{\_}Oikos.pdf:pdf},
issn = {16000706},
journal = {Oikos},
keywords = {ecological networks,food webs,herbivory,indirect interactions,motifs,mutualistic networks,parasitism,pollination,seed dispersal},
number = {2},
pages = {154--170},
title = {{Motifs in bipartite ecological networks: uncovering indirect interactions}},
url = {https://onlinelibrary.wiley.com/doi/10.1111/oik.05670},
volume = {128},
year = {2019}
}
@article{Slove2010b,
abstract = {One possible explanation for the latitudinal gradient in species richness often demonstrated is a related gradient in niche breadth, which may allow for denser species packing in the more stable environments at low latitudes.The evidence for such a gradient is, however, ambiguous, and the results have varied as much as the methods. Several studies have considered the non-independence of species, but few have performed explicit phylogenetic analyses.In the present study, we tested for a correlation between diet breadth and latitude of distribution in Nymphalinae butterflies using generalised estimating equations (GEE) and accounting for phylogenetic independence.Using a simple model with only latitude of distribution as a predictor variable revealed a significant positive relationship with diet breadth. Previous studies, however, have shown that diet breadth is also correlated with butterfly range size, and in turn, that range size may be correlated with latitude of distribution. Including geographical range size in the model also turned out to have a profound effect on the results - to the extent that the relationship between latitude of distribution and diet breadth was effectively reversed.We conclude that, at least for this group of butterflies, there is no evidence for a positive correlation between latitude of species distribution and diet breadth when controlling for range size, and that the effect may actually even be reversed. {\textcopyright} 2010 The Authors. Ecological Entomology {\textcopyright} 2010 The Royal Entomological Society.},
author = {Slove, Jessica and Janz, Niklas},
doi = {10.1111/j.1365-2311.2010.01238.x},
file = {:Users/alyssacirtwill/Documents/Papers/Slove, Janz{\_}2010{\_}Ecological Entomology.pdf:pdf},
issn = {03076946},
journal = {Ecological Entomology},
keywords = {Diversification,Generalisation,Host range,Latitude,Polyphagy,Specialisation},
number = {6},
pages = {768--774},
title = {{Phylogenetic analysis of the latitude-niche breadth hypothesis in the butterfly subfamily Nymphalinae}},
volume = {35},
year = {2010}
}
@article{Stouffer2012a,
abstract = {Studies of ecological networks (the web of interactions between species in a community) demonstrate an intricate link between a community's structure and its long-term viability. It remains unclear, however, how much a community's persistence depends on the identities of the species present, or how much the role played by each species varies as a function of the community in which it is found. We measured species' roles by studying how species are embedded within the overall network and the subsequent dynamic implications. Using data from 32 empirical food webs, we find that species' roles and dynamic importance are inherent species attributes and can be extrapolated across communities on the basis of taxonomic classification alone. Our results illustrate the variability of roles across species and communities and the relative importance of distinct species groups when attempting to conserve ecological communities.},
author = {Stouffer, Daniel B. and Sales-Pardo, Marta and Sirer, M. Irmak and Bascompte, Jordi},
doi = {10.1126/science.1216556},
file = {:Users/alyssacirtwill/Documents/Papers/Stouffer et al.{\_}2012{\_}Science(2).pdf:pdf},
issn = {10959203},
journal = {Science},
number = {6075},
pages = {1489--1492},
title = {{Evolutionary conservation of species' roles in food webs}},
volume = {335},
year = {2012}
}
@article{Stouffer2010a,
abstract = {Understanding food-web persistence is an important long-term objective of ecology because of its relevance in maintaining biodiversity. To date, many dynamic studies of food-web behaviour - both empirical and theoretical - have focused on smaller sub-webs, called trophic modules, because these modules are more tractable experimentally and analytically than whole food webs. The question remains to what degree studies of trophic modules are relevant to infer the persistence of entire food webs. Four trophic modules have received particular attention in the literature: tri-trophic food chains, omnivory, exploitative competition, and apparent competition. Here, we integrate analysis of these modules' dynamics in isolation with those of whole food webs to directly assess the appropriateness of scaling from modules to food webs. We find that there is not a direct, one-to-one, relationship between the relative persistence of modules in isolation and their effect on persistence of an entire food web. Nevertheless, we observe that those modules which are most commonly found in empirical food webs are those that confer the greatest community persistence. As a consequence, we demonstrate that there may be significant dynamic justifications for empirically-observed food-web structure. {\textcopyright} 2009 Blackwell Publishing Ltd/CNRS.},
author = {Stouffer, Daniel B. and Bascompte, Jordi},
doi = {10.1111/j.1461-0248.2009.01407.x},
file = {:Users/alyssacirtwill/Documents/Papers/Stouffer, Bascompte{\_}2010{\_}Ecology Letters.pdf:pdf},
issn = {1461023X},
journal = {Ecology Letters},
keywords = {Apparent competition,Dynamics,Ecological networks,Exploitative competition,Food chain,Network motif,Omnivory,Trophic module},
number = {2},
pages = {154--161},
title = {{Understanding food-web persistence from local to global scales}},
volume = {13},
year = {2010}
}
@article{Stouffer2012b,
abstract = {Studies of ecological networks (the web of interactions between species in a community) demonstrate an intricate link between a community's structure and its long-term viability. It remains unclear, however, how much a community's persistence depends on the identities of the species present, or how much the role played by each species varies as a function of the community in which it is found. We measured species' roles by studying how species are embedded within the overall network and the subsequent dynamic implications. Using data from 32 empirical food webs, we find that species' roles and dynamic importance are inherent species attributes and can be extrapolated across communities on the basis of taxonomic classification alone. Our results illustrate the variability of roles across species and communities and the relative importance of distinct species groups when attempting to conserve ecological communities.},
author = {Stouffer, Daniel B. and Sales-Pardo, Marta and Sirer, M. Irmak and Bascompte, Jordi},
doi = {10.1126/science.1216556},
file = {:Users/alyssacirtwill/Documents/Papers/Stouffer et al.{\_}2012{\_}Science.pdf:pdf},
issn = {10959203},
journal = {Science},
number = {6075},
pages = {1489--1492},
title = {{Evolutionary conservation of species' roles in food webs}},
volume = {335},
year = {2012}
}
@article{Strauss2006a,
abstract = {Despite the dominating role of pollinators in floral evolution, mounting evidence reveals significant additional, often antagonistic, influences of abiotic and biotic non-pollinator agents. Even when pollinators and other agents impose selection on floral traits in the ...$\backslash$n},
author = {Strauss, Sharon Y and Whittall, Justen B},
file = {:Users/alyssacirtwill/Documents/Papers/Strauss, Whittall{\_}2006{\_}Ecology and evolution of flowers.pdf:pdf},
isbn = {0198570864},
journal = {Ecology and evolution of flowers},
number = {October},
pages = {120--138},
title = {{Non-pollinator agents of selection on floral traits}},
url = {http://www.researchgate.net/profile/Sharon{\_}Strauss/publication/254469747{\_}Non-pollinator{\_}agents{\_}of{\_}selection{\_}on{\_}floral{\_}traits/links/53d275b60cf220632f3c9c39.pdf{\%}5Cnpapers2://publication/uuid/F1BC2A23-C948-4B44-A97C-C6D1A238C113},
year = {2006}
}
@article{Sydenham2018a,
abstract = {Aim: Because the ecological similarity between species is expected to increase with relatedness and that speciation is a local process, phylogeny may provide a common measure for the influence of ecological and biogeographic processes on community assembly. We tested if similarities in floral visitation patterns within communities and the phylogenetic beta-diversity among communities were related to the position of bees within the bee phylogeny. Location: Global. Methods: We combined a genus level phylogeny with within-genera phylogenies for the bee species occurring within 18 globally distributed bee-flower networks. Networks consisted of a matrix of bee and plant species and information on whether or not a bee species had been observed visiting flowers of a given plant species. For each network, we used Abouheif's Cmean to test if the similarity in floral associations (niche similarity) between bees and the number of plant species visited displayed a significant phylogenetic signal. To test if biogeography influenced the relatedness among species within networks we tested if the phylogenetic beta-diversity increased with geographical distance and dissimilarity in climatic conditions among networks. Results: We found a phylogenetic signal for niche similarity in only 50{\%} of the bee-flower networks. However, network size influenced the likelihood of observing a phylogenetic signal and for seven of the eight bee-flower networks with {\textgreater}20 species it was statistically significant. On a global scale, the phylogenetic beta-diversity increased with geographical distances and with climatic dissimilarity between sites. Main conclusions: Bee communities are structured by processes of speciation and migration so that regional species pools are dominated by a subset of the global phylogenetic clades, resulting in increasing phylogenetic beta-diversity with geographical distance. Moreover, ecological filtering processes operating at both local (floral resource use) and continental (climatic constraints) scale determine the distribution of species among resources and geographical regions. The assembly of bee communities should therefore be understood as a product of both biogeographic and community ecological processes.},
author = {Sydenham, Markus Arne Kj{\ae}r and Eldegard, Katrine and Hegland, Stein Joar and Nielsen, Anders and Totland, {\O}rjan and Fjellheim, Siri and Moe, Stein R.},
doi = {10.1111/jbi.13103},
file = {:Users/alyssacirtwill/Documents/Papers/Sydenham et al.{\_}2018{\_}Journal of Biogeography.pdf:pdf},
issn = {13652699},
journal = {Journal of Biogeography},
keywords = {beta-diversity,biogeography,community ecology,phylogenetic signal,plant–pollinator networks,wild bees},
number = {2},
pages = {461--472},
title = {{Community level niche overlap and broad scale biogeographic patterns of bee communities are driven by phylogenetic history}},
volume = {45},
year = {2018}
}
@article{Thebault2010a,
abstract = {Research on the relationship between the architecture of ecological networks and community stability has mainly focused on one type of interaction at a time, making difficult any comparison between different network types. We used a theoretical approach to show that the network architecture favoring stability fundamentally differs between trophic and mutualistic networks. A highly connected and nested architecture promotes community stability in mutualistic networks, whereas the stability of trophic networks is enhanced in compartmented and weakly connected architectures. These theoretical predictions are supported by a meta-analysis on the architecture of a large series of real pollination (mutualistic) and herbivory (trophic) networks. We conclude that strong variations in the stability of architectural patterns constrain ecological networks toward different architectures, depending on the type of interaction.},
author = {Th{\'{e}}bault, Elisa and Fontaine, Colin},
doi = {10.1126/science.1188321},
file = {:Users/alyssacirtwill/Documents/Papers/Th{\'{e}}bault, Fontaine{\_}2010{\_}Science.pdf:pdf},
issn = {00368075},
journal = {Science},
number = {5993},
pages = {853--856},
pmid = {20705861},
title = {{Stability of ecological communities and the architecture of mutualistic and trophic networks}},
volume = {329},
year = {2010}
}
@book{Akerlof1970,
abstract = {Predicting the binding mode of flexible polypeptides to proteins is an important task that falls outside the domain of applicability of most small molecule and protein−protein docking tools. Here, we test the small molecule flexible ligand docking program Glide on a set of 19 non-$\alpha$-helical peptides and systematically improve pose prediction accuracy by enhancing Glide sampling for flexible polypeptides. In addition, scoring of the poses was improved by post-processing with physics-based implicit solvent MM- GBSA calculations. Using the best RMSD among the top 10 scoring poses as a metric, the success rate (RMSD ≤ 2.0 {\AA} for the interface backbone atoms) increased from 21{\%} with default Glide SP settings to 58{\%} with the enhanced peptide sampling and scoring protocol in the case of redocking to the native protein structure. This approaches the accuracy of the recently developed Rosetta FlexPepDock method (63{\%} success for these 19 peptides) while being over 100 times faster. Cross-docking was performed for a subset of cases where an unbound receptor structure was available, and in that case, 40{\%} of peptides were docked successfully. We analyze the results and find that the optimized polypeptide protocol is most accurate for extended peptides of limited size and number of formal charges, defining a domain of applicability for this approach.},
archivePrefix = {arXiv},
arxivId = {arXiv:1011.1669v3},
author = {Akerlof},
booktitle = {Journal of Chemical Information and Modeling},
doi = {10.1017/CBO9781107415324.004},
eprint = {arXiv:1011.1669v3},
file = {:Users/alyssacirtwill/Documents/Papers/Unknown{\_}1991{\_}Unknown.pdf:pdf},
isbn = {9788578110796},
issn = {1098-6596},
keywords = {icle},
number = {9},
pages = {1689--1699},
pmid = {25246403},
title = {{済無No Title No Title}},
volume = {53},
year = {1970}
}
@article{VanderZanden1999a,
abstract = {Food web structure is paramount in regulating a variety of ecologic patterns and processes, although food web studies are limited by poor empirical descriptions of inherently complex systems. In this study, stable isotope ratios ($\delta$15N and $\delta$13C) were used to quantify trophic relationships and food chain length (measured as a continuous variable) in 14 Ontario and Quebec lakes. All lakes contained lake trout as the top predator, although lakes differed in the presumed number of trophic levels leading to this species. The presumed number of trophic levels was correlated with food chain length and explained 40{\%} of the among-lake variation. Food chain length was most closely related to fish species richness (r2 = 0.69) and lake area (r2= 0.50). However, the two largest study lakes had shorter food chains than lakes of intermediate size and species richness, producing hump-shaped relationships with food chain length. Lake productivity was not a powerful predictor of food chain length (r2 = 0.36), and we argue that productive space (productivity multiplied by area) is a more accurate measure of available energy. This study addresses the need for improved food web descriptions that incorporate information about energy flow and the relative importance of trophic pathways.},
author = {{Vander Zanden}, M. Jake and Shuter, Brian J. and Lester, Nigel and Rasmussen, Joseph B.},
doi = {10.1086/303250},
file = {:Users/alyssacirtwill/Documents/Papers/Vander Zanden et al.{\_}1999{\_}The American Naturalist.pdf:pdf},
issn = {00030147},
journal = {American Naturalist},
keywords = {Food chains,Food webs,Productive space,Productivity,Trophic position,Trophic structure},
number = {4},
pages = {406--416},
title = {{Patterns of food chain length in lakes: A stable isotope study}},
volume = {154},
year = {1999}
}
@article{Vermaat2009a,
abstract = {The covariance among a range of 20 network structural properties of food webs plus net primary productivity was assessed for 14 published food webs using principal components analysis. Three primary components explained 84{\%} of the variability in the data sets, suggesting substantial covariance among the properties employed in the literature. The first dimension explained 48{\%} of the variance and could be ascribed to connectance, covarying significantly with the proportion of intermediate species and characteristic path length. The second dimension explained 19{\%} and was related to trophic species richness. The third axis explained 17{\%} and was related to ecosystem net primary productivity. A distinct opposite clustering of connectance, the proportion of intermediate species, and mean trophic level vs. the proportion of top and basal species and path length suggests a dichotomy in food-web structure. Food webs appear either clustered and highly interconnected or elongated with fewer links. {\textcopyright} 2009 by the Ecological Society of America.},
author = {Vermaat, Jan E. and Dunne, Jennifer A. and Gilbert, Alison J.},
doi = {10.1890/07-0978.1},
file = {:Users/alyssacirtwill/Documents/Papers/Vermaat, Dunne, Gilbert{\_}2009{\_}Ecology.pdf:pdf},
issn = {00129658},
journal = {Ecology},
keywords = {Connectance,Food-web properties,Network topology,Principal components,Trophic species},
number = {1},
pages = {278--282},
pmid = {19294932},
title = {{Major dimensions in food-web structure properties}},
volume = {90},
year = {2009}
}
@article{Palareti2016,
abstract = {Introduction: D-dimer assay, generally evaluated according to cutoff points calibrated for VTE exclusion, is used to estimate the individual risk of recurrence after a first idiopathic event of venous thromboembolism (VTE). Methods: Commercial D-dimer assays, evaluated according to predetermined cutoff levels for each assay, specific for age (lower in subjects {\textless}70 years) and gender (lower in males), were used in the recent DULCIS study. The present analysis compared the results obtained in the DULCIS with those that might have been had using the following different cutoff criteria: traditional cutoff for VTE exclusion, higher levels in subjects aged ≥60 years, or age multiplied by 10. Results: In young subjects, the DULCIS low cutoff levels resulted in half the recurrent events that would have occurred using the other criteria. In elderly patients, the DULCIS results were similar to those calculated for the two age-adjusted criteria. The adoption of traditional VTE exclusion criteria would have led to positive results in the large majority of elderly subjects, without a significant reduction in the rate of recurrent event. Conclusion: The results confirm the usefulness of the cutoff levels used in DULCIS.},
author = {Palareti, G. and Legnani, C. and Cosmi, B. and Antonucci, E. and Erba, N. and Poli, D. and Testa, S. and Tosetto, A.},
doi = {10.1111/ijlh.12426},
file = {:Users/alyssacirtwill/Documents/Papers/Volf et al.{\_}2017{\_}Journal of Animal Ecology(2).pdf:pdf},
issn = {1751553X},
journal = {International Journal of Laboratory Hematology},
keywords = {Cutoff criteria,D-dimer,Recurrence,Venous thromboembolism},
number = {1},
pages = {42--49},
title = {{Comparison between different D-Dimer cutoff values to assess the individual risk of recurrent venous thromboembolism: Analysis of results obtained in the DULCIS study}},
volume = {38},
year = {2016}
}
@article{Morris2014b,
author = {Morris, Rebecca J and Lewis, Owen T and Godfray, H Charles J and Annales, Source and Fennici, Zoologici and Ecology, Spatial and Herbivorous, O F and Morris, J and Lewis, Owen Т and Godfray, H Charles J},
file = {:Users/alyssacirtwill/Documents/Papers/Morris et al.{\_}2014{\_}Unknown.pdf:pdf},
number = {4},
pages = {449--462},
title = {{Finnish Zoological and Botanical Publishing Board Apparent competition and insect community structure : towards a spatial perspective Apparent competition and insect community structure : towards a spatial perspective}},
volume = {42},
year = {2014}
}
@article{Morte1999a,
abstract = {The stomach contents of 344 four-spotted megrim, (Lepidorhombus boscii) and 159 megrim (Lepidorhombus whiffiagonis), off the eastern coast of the Gulf of Valencia (Spain), were analysed. The two species examined do not appear to have very similar diets, based on the species composition of prey. The vacuity coefficient is not high for any of the species, the main food being Crustacea (Decapoda and Mysidacea). Also Amphipoda and Teleostei are components of the diet. Variations in the food of both fish related to their length show few small crustaceans as prey of the major specimens. Finally, there was evidence for seasonal variation of the quality and quantity of the food consumed. There was no great dietary overlap between these two species.},
author = {Morte, Salom{\'{e}} and Red{\'{o}}n, Manuel J. and Sanz-Brau, Antonio},
doi = {10.1017/S0025315497000180},
file = {:Users/alyssacirtwill/Documents/Papers/Morte, Red{\'{o}}n, Sanz-Brau{\_}1999{\_}Journal of the Marine Biological Association of the United Kingdom.pdf:pdf},
issn = {00253154},
journal = {Journal of the Marine Biological Association of the United Kingdom},
number = {1},
pages = {161--169},
title = {{Feeding ecology of two megrims Lepidorhombus boscii and Lepidorhombus whiffiagonis in the western Mediterranean (Gulf of Valencia, Spain)}},
volume = {79},
year = {1999}
}
@article{Mougi2012a,
abstract = {Ecological theory predicts that a complex community formed by a number of species is inherently unstable, guiding ecologists to identify what maintains species diversity in nature. Earlier studies often assumed a community with only one interaction type, either an antagonistic, competitive, or mutualistic interaction, leaving open the question of what the diversity of interaction types contributes to the community maintenance. We show theoretically that the multiple interaction types might hold the key to understanding community dynamics. A moderate mixture of antagonistic and mutualistic interactions can stabilize population dynamics. Furthermore, increasing complexity leads to increased stability in a "hybrid" community. We hypothesize that the diversity of species and interaction types may be the essential element of biodiversity that maintains ecological communities.},
author = {Mougi, A. and Kondoh, M.},
doi = {10.1126/science.1220529},
file = {:Users/alyssacirtwill/Documents/Papers/Mougi, Kondoh{\_}2012{\_}Science.pdf:pdf},
issn = {10959203},
journal = {Science},
number = {6092},
pages = {349--351},
title = {{Diversity of interaction types and ecological community stability}},
volume = {337},
year = {2012}
}
@article{Peralta2014a,
abstract = {Complementary resource use and redundancy of species that fulfill the same ecological role are two mechanisms that can respectively increase and stabilize process rates in ecosystems. For example, predator complementarity and redundancy can determine prey consumption rates and their stability, yet few studies take into account the multiple predator species attacking multiple prey at different rates in natural communities. Thus, it remains unclear whether these biodiversity mechanisms are important determinants of consumption in entire predator-prey assemblages, such that food-web interaction structure determines community-wide consumption and stability. Here, we use empirical quantitative food webs to study the community-wide effects of functional complementarity and redundancy of consumers (parasitoids) on herbivore control in temperate forests. We find that complementarity in host resource use by parasitoids was a strong predictor of absolute parasitism rates at the community level and that redundancy in host-use patterns stabilized community-wide parasitism rates in space, but not through time. These effects can potentially explain previous contradictory results from predator diversity research. Phylogenetic diversity (measured using taxonomic distance) did not explain functional complementarity or parasitism rates, so could not serve as a surrogate measure for functional complementarity. Our study shows that known mechanisms underpinning predator diversity effects on both functioning and stability can easily be extended to link food webs to ecosystem functioning. {\textcopyright} 2014 by the Ecological Society of America.},
author = {Peralta, Guadalupe and Frost, Carol M. and Rand, Tatyana A. and Didham, Raphael K. and Tylianakis, Jason M.},
doi = {10.1890/13-1569.1},
file = {:Users/alyssacirtwill/Documents/Papers/Peralta et al.{\_}2014{\_}Ecology.pdf:pdf},
issn = {00129658},
journal = {Ecology},
keywords = {Biodiversity,Ecosystem Functioning,Food-Web Structure,Insurance Hypothesis,Niche Partitioning,Parasitism,Phylogenetic Species Variability,Spatial Variability,Temporal Variability},
number = {7},
pages = {1888--1896},
title = {{Complementarity and redundancy of interactions enhance attack rates and spatial stability in host-parasitoid food webs}},
volume = {95},
year = {2014}
}
@article{Paine1980a,
author = {Paine, Robert T},
file = {:Users/alyssacirtwill/Documents/Papers/Paine{\_}1980{\_}The Journal of Animal Ecology.pdf:pdf},
journal = {Journal of Animal Ecology},
number = {3},
pages = {666--685},
title = {{Food Webs : Linkage , Interaction Strength and Community Infrastructure Author ( s ): R . T . Paine Source : Journal of Animal Ecology , Vol . 49 , No . 3 ( Oct ., 1980 ), pp . 666-685 Published by : British Ecological Society Stable URL : http://www.jsto}},
volume = {49},
year = {1980}
}
@article{Peralta2019a,
abstract = {Different modelling approaches have been used to relate the structure of mutualistic interactions with the stability of communities. However, inconsistencies arise when we compare modelling outcomes with the patterns of interactions observed in empirical studies. To shed light on these inconsistencies, we explored the network structure–stability relationship by incorporating the cost of mutualistic interactions, a long ignored feature of mutualisms, into population dynamics models. We assessed the changes in the relationship between network structure (species richness, connectance, modularity) and community stability (species persistence, resilience), and between network structure and community structural attributes (average abundance), using models with increasing levels of cost for mutualistic communities. We found that adding the potential cost of mutualistic interactions affected the strength of the network structure–stability relationship. Our results revive the question of whether the structure of mutualistic networks determines community stability.},
author = {Peralta, Guadalupe and Stouffer, Daniel B. and Bringa, Eduardo M. and V{\'{a}}zquez, Diego P.},
doi = {10.1111/oik.06503},
file = {:Users/alyssacirtwill/Documents/Papers/Peralta et al.{\_}2019{\_}Oikos.pdf:pdf},
issn = {16000706},
journal = {Oikos},
keywords = {abundance,network structure,persistence,plant–pollinator interactions,population dynamics,resilience},
title = {{No such thing as a free lunch: interaction costs and the structure and stability of mutualistic networks}},
year = {2019}
}
@article{Pompanon2012a,
abstract = {The analysis of food webs and their dynamics facilitates understanding of the mechanistic processes behind community ecology and ecosystem functions. Having accurate techniques for determining dietary ranges and components is critical for this endeavour. While visual analyses and early molecular approaches are highly labour intensive and often lack resolution, recent DNA-based approaches potentially provide more accurate methods for dietary studies. A suite of approaches have been used based on the identification of consumed species by characterization of DNA present in gut or faecal samples. In one approach, a standardized DNA region (DNA barcode) is PCR amplified, amplicons are sequenced and then compared to a reference database for identification. Initially, this involved sequencing clones from PCR products, and studies were limited in scale because of the costs and effort required. The recent development of next generation sequencing (NGS) has made this approach much more powerful, by allowing the direct characterization of dozens of samples with several thousand sequences per PCR product, and has the potential to reveal many consumed species simultaneously (DNA metabarcoding). Continual improvement of NGS technologies, on-going decreases in costs and current massive expansion of reference databases make this approach promising. Here we review the power and pitfalls of NGS diet methods. We present the critical factors to take into account when choosing or designing a suitable barcode. Then, we consider both technical and analytical aspects of NGS diet studies. Finally, we discuss the validation of data accuracy including the viability of producing quantitative data. {\textcopyright} 2011 Blackwell Publishing Ltd.},
author = {Pompanon, Francois and Deagle, Bruce E. and Symondson, William O.C. and Brown, David S. and Jarman, Simon N. and Taberlet, Pierre},
doi = {10.1111/j.1365-294X.2011.05403.x},
file = {:Users/alyssacirtwill/Documents/Papers/Pompanon et al.{\_}2012{\_}Molecular Ecology.pdf:pdf},
issn = {09621083},
journal = {Molecular Ecology},
keywords = {DNA barcoding,DNA metabarcoding,faeces,food webs,herbivory,predation},
number = {8},
pages = {1931--1950},
pmid = {22171763},
title = {{Who is eating what: Diet assessment using next generation sequencing}},
volume = {21},
year = {2012}
}
@article{Ponisio2019a,
abstract = {Disconnected habitat fragments are poor at supporting population and community persistence; restoration ecologists, therefore, advocate for the establishment of habitat networks across landscapes. Few empirical studies, however, have considered how networks of restored habitat patches affect metacommunity dynamics. Here, using a 10-year study on restored hedgerows and unrestored field margins within an intensive agricultural landscape, we integrate occupancy modelling with network theory to examine the interaction between local and landscape characteristics, habitat selection and dispersal in shaping pollinator metacommunity dynamics. We show that surrounding hedgerows and remnant habitat patches interact with the local floral diversity, bee diet breadth and bee body size to influence site occupancy, via colonisation and persistence dynamics. Florally diverse sites and generalist, small-bodied species are most important for maintaining metacommunity connectivity. By providing the first in-depth assessment of how a network of restored habitat influences long-term population dynamics, we confirm the conservation benefit of hedgerows for pollinator populations and demonstrate the importance of restoring and maintaining habitat networks within an inhospitable matrix.},
author = {Ponisio, Lauren C. and de Valpine, Perry and M'Gonigle, Leithen K. and Kremen, Claire},
doi = {10.1111/ele.13257},
file = {:Users/alyssacirtwill/Documents/Papers/Ponisio et al.{\_}2019{\_}Ecology Letters.pdf:pdf},
issn = {14610248},
journal = {Ecology Letters},
keywords = {Agriculture,graph,hedgerow,metapopulation,network,restoration,wild bee},
number = {7},
pages = {1048--1060},
title = {{Proximity of restored hedgerows interacts with local floral diversity and species' traits to shape long-term pollinator metacommunity dynamics}},
volume = {22},
year = {2019}
}
@article{Prosser2010a,
author = {Prosser, James I.},
doi = {10.1111/j.1462-2920.2010.02201.x},
file = {:Users/alyssacirtwill/Documents/Papers/Prosser{\_}2010{\_}Environmental Microbiology.pdf:pdf},
issn = {14622912},
journal = {Environmental Microbiology},
number = {7},
pages = {1806--1810},
title = {{Replicate or lie}},
volume = {12},
year = {2010}
}
@article{Ponisio2017b,
abstract = {One of the major challenges in evolutionary ecology is to understand how coevolution shapes species interaction networks. Important topological properties of networks such as nestedness and modularity are thought to be affected by coevolution. However, there has been no test whether coevolution does, in fact, lead to predictable network structure. Here, we investigate the structure of simulated bipartite networks generated under different modes of coevolution. We ask whether evolutionary processes influence network structure and, furthermore, whether any emergent trends are influenced by the strength or "intimacy" of the species interactions. We find that coevolution leaves a weak and variable signal on network topology, particularly nestedness and modularity, which was not strongly affected by the intimacy of interactions. Our findings indicate that network metrics, on their own, should not be used to make inferences about processes underlying the evolutionary history of communities. Instead, a more holistic approach that combines network approaches with traditional phylogenetic and biogeographic reconstructions is needed.},
author = {Ponisio, Lauren C. and M'Gonigle, Leithen K.},
doi = {10.1002/ecs2.1798},
file = {:Users/alyssacirtwill/Documents/Papers/Ponisio, M'Gonigle{\_}2017{\_}Ecosphere.pdf:pdf},
issn = {21508925},
journal = {Ecosphere},
keywords = {Bipartite,Evolution,Interaction intimacy,Modularity,Nestedness,Phylogenetic interaction structure},
number = {4},
title = {{Coevolution leaves a weak signal on ecological networks}},
volume = {8},
year = {2017}
}
@article{Portalier2019a,
abstract = {Robust predictions of predator–prey interactions are fundamental for the understanding of food webs, their structure, dynamics, resistance to species loss, response to invasions and ecosystem function. Most current food web models measure parameters at the food web level to predict patterns at the same level. Thus, they are sensitive to the quality of the data and may be ineffective in predicting non-observed interactions and disturbed food webs. There is a need for mechanistic models that predict the occurrence of a predator–prey interaction based on lower levels of organization (i.e. the traits of organisms) and the properties of their environment. Here, we present such a model that focuses on the predation act itself. We built a Newtonian, mechanical model for the processes of searching, capturing and handling of a prey item by a predator. Associated with general metabolic laws, we predict the net energy gain from predation for pairs of pelagic or flying predator species and their prey depending on their body sizes. Predicted interactions match well with data from the most extensive predator–prey database, and overall model accuracy is greater than the allometric niche model. Our model shows that it is possible to accurately predict the structure of food webs using only a few mechanical traits. It underlines the importance of physical constraints in structuring food webs. A plain language summary is available for this article.},
author = {Portalier, S{\'{e}}bastien M.J. and Fussmann, Gregor F. and Loreau, Michel and Cherif, Mehdi},
doi = {10.1111/1365-2435.13254},
file = {:Users/alyssacirtwill/Documents/Papers/Portalier et al.{\_}2019{\_}Functional Ecology.pdf:pdf},
issn = {13652435},
journal = {Functional Ecology},
keywords = {body size,energy,mechanics,predation,trophic link},
number = {2},
pages = {323--334},
title = {{The mechanics of predator–prey interactions: First principles of physics predict predator–prey size ratios}},
volume = {33},
year = {2019}
}
@article{Ratnasingham2007,
abstract = {The Barcode of Life Data System ( BOLD ) is an informatics workbench aiding the acquisition, storage, analysis and publication of DNA barcode records. By assembling molecular, morphological and distributional data, it bridges a traditional bioinformatics chasm. BOLD is freely available to any researcher with interests in DNA barcoding. By providing specialized services, it aids the assembly of records that meet the standards needed to gain BARCODE designation in the global sequence databases. Because of its web-based delivery and flexible data security model, it is also well positioned to support projects that involve broad research alliances. This paper provides a brief introduction to the key elements of BOLD , dis- cusses their functional capabilities, and concludes by examining computational resources and future prospects.},
author = {Ratnasingham, Sujeevan and Hebert, Paul D N},
doi = {10.1111/j.1471-8286.2006.01678.x},
file = {:Users/alyssacirtwill/Documents/Papers/Ratnasingham, Hebert{\_}2007{\_}Molecular Ecology Notes.pdf:pdf},
isbn = {1471-8286},
issn = {08927553},
journal = {Molecular Ecology Notes},
keywords = {COI,DNA barcoding,gene sequence,informatics,revision accepted 17 November 2006,species identification,taxonomy Received 30 July 2006},
number = {April 2016},
pages = {355--364},
pmid = {18784790},
title = {{The Barcode of Life Data System}},
volume = {7},
year = {2007}
}
@article{Robertson1983a,
abstract = {The amphipod Allorchestes compressa Dana inhabits large accumulations of detached macrophytes in the surf-zone of sandy beaches in southern Western Australia. A. compressa is most abundant on branching red algae and least abundant on intact thalli of the kelp Ecklonia radiata (Turn.) J. Agardh., yet the major component of the gut contents is brown algae (probably E. radiata) and decomposing E. radiata ranked first in laboratory food preference experiments. Observations on the feeding behaviour of Allorchestes compressa indicated that the amphipods obtain their food by feeding on small pieces ({\textless} 3 cm) of macrophyte tissue trapped within the highly branched algae, or amphipods may move with and feed on the plant particles as they are swept around in the surf. In a particle selection experiment, using plant particles 1-3 mm sieved from the surf, A. compressa selected particles of Ecklonia radiata, leached Ulva sp., Sargassum spp., and seagrass leaves but avoided branching red algae. The influence of potential foods on the darwinian fitness of Allorchestes compressa was assessed on the basis of adult survival, the percentage of females which carried eggs, growth rates, and time to maturity measured in laboratory rearing experiments. Fitness increased in the order red algae → intact seagrass leaves → mixed particles (1-3 mm) sieved from the surf → Ecklonia radiata tissue. Given the constraints of fish predation and the fluctuating supply of E. radiata, amphipods in the surf consume close to their theoretically optimum diet by feeding mainly on E. radiata from amongst the available particles of different macrophytes. Estimates of the significance of the Allorchestes compressa population in the turnover of Ecklonia radiata biomass in the surf-zone (estimated as g Ecklonia consumed per g Ecklonia per day) showed that amphipods could turnover E. radiata biomass twice per month in summer and once every 1 to 2 months during spring and autumn. These rates are comparable with those measured for the physical breakdown and microbial decomposition of E. radiata and, except during winter, grazing by Allorchestes compressa must, therefore, be considered an important process during the remineralization of nutrients tied up in kelp biomass in the surf-zone. {\textcopyright} 1983.},
author = {Robertson, A. I. and Lucas, J. S.},
doi = {10.1016/0022-0981(83)90138-7},
file = {:Users/alyssacirtwill/Documents/Papers/Robertson, Lucas{\_}1983{\_}Journal of Experimental Marine Biology and Ecology.pdf:pdf},
issn = {00220981},
journal = {Journal of Experimental Marine Biology and Ecology},
number = {2},
pages = {99--124},
title = {{Food choice, feeding rates, and the turnover of macrophyte biomass by a surf-zone inhabiting amphipod}},
volume = {72},
year = {1983}
}
@article{Rezende2007c,
abstract = {The interactions between plants and their animal pollinators and seed dispersers have moulded much of Earth's biodiversity. Recently, it has been shown that these mutually beneficial interactions form complex networks with a well-defined architecture that may contribute to biodiversity persistence. Little is known, however, about which ecological and evolutionary processes generate these network patterns. Here we use phylogenetic methods to show that the phylogenetic relationships of species predict the number of interactions they exhibit in more than one-third of the networks, and the identity of the species with which they interact in about half of the networks. As a consequence of the phylogenetic effects on interaction patterns, simulated extinction events tend to trigger coextinction cascades of related species. This results in a non-random pruning of the evolutionary tree and a more pronounced loss of taxonomic diversity than expected in the absence of a phylogenetic signal. Our results emphasize how the simultaneous consideration of phylogenetic information and network architecture can contribute to our understanding of the structure and fate of species-rich communities. {\textcopyright}2007 Nature Publishing Group.},
author = {Rezende, Enrico L. and Lavabre, Jessica E. and Guimar{\~{a}}es, Paulo R. and Jordano, Pedro and Bascompte, Jordi},
doi = {10.1038/nature05956},
file = {:Users/alyssacirtwill/Documents/Papers/Rezende et al.{\_}2007{\_}Nature.pdf:pdf},
issn = {14764687},
journal = {Nature},
number = {7156},
pages = {925--928},
title = {{Non-random coextinctions in phylogenetically structured mutualistic networks}},
volume = {448},
year = {2007}
}
@article{Palareti2016a,
abstract = {Introduction: D-dimer assay, generally evaluated according to cutoff points calibrated for VTE exclusion, is used to estimate the individual risk of recurrence after a first idiopathic event of venous thromboembolism (VTE). Methods: Commercial D-dimer assays, evaluated according to predetermined cutoff levels for each assay, specific for age (lower in subjects {\textless}70 years) and gender (lower in males), were used in the recent DULCIS study. The present analysis compared the results obtained in the DULCIS with those that might have been had using the following different cutoff criteria: traditional cutoff for VTE exclusion, higher levels in subjects aged ≥60 years, or age multiplied by 10. Results: In young subjects, the DULCIS low cutoff levels resulted in half the recurrent events that would have occurred using the other criteria. In elderly patients, the DULCIS results were similar to those calculated for the two age-adjusted criteria. The adoption of traditional VTE exclusion criteria would have led to positive results in the large majority of elderly subjects, without a significant reduction in the rate of recurrent event. Conclusion: The results confirm the usefulness of the cutoff levels used in DULCIS.},
author = {Palareti, G. and Legnani, C. and Cosmi, B. and Antonucci, E. and Erba, N. and Poli, D. and Testa, S. and Tosetto, A.},
doi = {10.1111/ijlh.12426},
file = {:Users/alyssacirtwill/Documents/Papers/Roubinet et al.{\_}2017{\_}Ecological Applications.pdf:pdf},
issn = {1751553X},
journal = {International Journal of Laboratory Hematology},
keywords = {Cutoff criteria,D-dimer,Recurrence,Venous thromboembolism},
number = {1},
pages = {42--49},
title = {{Comparison between different D-Dimer cutoff values to assess the individual risk of recurrent venous thromboembolism: Analysis of results obtained in the DULCIS study}},
volume = {38},
year = {2016}
}
@article{Hamback2016,
abstract = {Inflow of matter and organisms may strongly affect the local density and diversity of organisms. This effect is particularly evident on shores where organisms with aquatic larval stages enter the terrestrial food web. The identities of such trophic links are not easily estimated as spiders, a dominant group of shoreline predator, have external digestion. We compared trophic links and the prey diversity of spiders on different shore types along the Baltic Sea: on open shores and on shores with a reed belt bordering the water. A priori, we hypothesized that the physical structure of the shoreline reduces the flow between ecosystem and the subsidies across the sea-land interface. To circumvent the lack of morphologically detectable remains of spider prey, we used a combination of stable isotope and molecular gut content analyses. The two tools used for diet analysis revealed complementary information on spider diets. The stable isotope analysis indicated that spiders on open shores had a marine signal of carbon isotopes, while spiders on reedy shores had a terrestrial signal. The molecular analysis revealed a diverse array of dipteran and lepidopteran prey, where spiders on open and reedy shores shared a similar diet with a comparable proportion of chironomids, the larvae of which live in the marine system. Comparing the methods suggests that differences in isotope composition of the two spider groups occurred because of differences in the chironomid diets: as larvae, chironomids of reedy shores likely fed on terrestrial detritus and acquired a terrestrial isotope signature, while chironomids of open shores utilized an algal diet and acquired a marine isotope signature. Our results illustrate how different methods of diet reconstruction may shed light on complementary aspects of nutrient transfer. Overall, they reveal that reed belts can reduce connectivity between habitats, but also function as a source of food for predators. K E Y W O R D S Baltic Sea, chironomids, DNA barcoding, Pardosa, stable isotope analysis},
author = {Hamb{\"{a}}ck, Peter A. and Weingartner, Elisabeth and Dal{\'{e}}n, Love and Wirta, Helena and Roslin, Tomas},
doi = {10.1002/ece3.2536},
isbn = {2045-7758 (Linking)},
issn = {20457758},
journal = {Ecology and Evolution},
keywords = {Baltic Sea,DNA barcoding,Pardosa,chironomids,stable isotope analysis},
number = {23},
pages = {8431--8439},
pmid = {28031795},
title = {{Spatial subsidies in spider diets vary with shoreline structure: Complementary evidence from molecular diet analysis and stable isotopes}},
volume = {6},
year = {2016}
}
@misc{bioenergeticfw,
author = {Delmas, Eva and Brose, Ulrich and Gravel, Dominique and Stouffer, Daniel B. and Poisot, Timoth{\'{e}}e},
booktitle = {Github},
doi = {10.5281/zenodo.3597556},
title = {{Bio-energetic food web model v1.1.2}},
url = {https://github.com/PoisotLab/BioEnergeticFoodWebs.jl},
urldate = {2020-03-01},
year = {2019}
}
@article{McLeod2020,
author = {McLeod, Anne M. and Leroux, Shawn J. and Chu, Cindy},
doi = {10.1002/ecs2.3018},
issn = {2150-8925},
journal = {Ecosphere},
keywords = {abundance,beta diversity,competition,food webs,interaction turnover,morphology,phylogeny,thermal},
number = {2},
title = {{Effects of species traits, motif profiles, and environment on spatial variation in multi‐trophic antagonistic networks}},
volume = {11},
year = {2020}
}
@book{Gaba2018,
abstract = {Due to increasing compost use in agriculture, there is an urgent need to evaluate compost bene fi ts and impacts versus other fertilizers. Here we review the recent progress made in the quanti fi cation of positive effects associated with compost use on land using life cycle assessment (LCA), an internationally recognised environmental tool. Nine environmental bene fi ts were identi fi ed in an extensive literature review: nutrient supply, carbon sequestration, weed pest and disease suppression, increase in crop yield, decreased soil erosion, retention of soil moisture, increased soil workability, enhanced soil biological properties and biodiversity, and gain in crop nutritional quality. Quantitative fi gures for each bene fi t were drawn from the literature and classi fi ed into short-term: less than 1 year; mid-term: less than 10 years and long-term: less than 100 years.},
author = {Gaba, Sabrina and Alignier, Audrey and Aviron, St{\'{e}}phanie and Barot, S{\'{e}}bastien and Blouin, Manuel and Hedde, Micka{\"{e}}l and Jabot, Franck and Vergnes, Alan and Bonis, Anne and Bonthoux, S{\'{e}}bastien and Bourgeois, B{\'{e}}renger and Bretagnolle, Vincent and Catarino, Rui and Coux, Camille and Gardarin, Antoine and Giffard, Brice and {Le Gal}, Antoine and Lecomte, Jane and Miguet, Paul and Piutti, S{\'{e}}verine and Rusch, Adrien and Zwicke, Marine and Couvet, Denis},
doi = {10.1007/978-3-319-90309-5_1},
isbn = {9783319903095},
number = {January},
pages = {1--46},
title = {{Ecology for Sustainable and Multifunctional Agriculture}},
year = {2018}
}
@article{Tylianakis2014,
abstract = {Network studies have described the complex interactions among species. Concomitantly, researchers have searched for signals of ecosystem tipping points and attributes of systems that resist them. A recent study combines these areas, showing that attributes of pollination network structure delay critical transitions, and generating a wealth of new research questions. {\textcopyright} 2014 Elsevier Ltd.},
author = {Tylianakis, Jason M. and Coux, Camille},
doi = {10.1016/j.tplants.2014.03.006},
file = {:Users/alyssacirtwill/Documents/Papers/Tylianakis, Coux{\_}2014{\_}Trends in Plant Science.pdf:pdf},
issn = {13601385},
journal = {Trends in Plant Science},
number = {5},
pages = {281--283},
publisher = {Elsevier Ltd},
title = {{Tipping points in ecological networks}},
url = {http://dx.doi.org/10.1016/j.tplants.2014.03.006},
volume = {19},
year = {2014}
}
@article{Enquist2019,
abstract = {A key feature of life's diversity is that some species are common but many more are rare. Nonetheless, at global scales, we do not know what fraction of biodiversity consists of rare species. Here, we present the largest compilation of global plant diversity to quantify the fraction of Earth's plant biodiversity that are rare. A large fraction, {\~{}}36.5{\%} of Earth's {\~{}}435,000 plant species, are exceedingly rare. Sampling biases and prominent models, such as neutral theory and the k-niche model, cannot account for the observed prevalence of rarity. Our results indicate that (i) climatically more stable regions have harbored rare species and hence a large fraction of Earth's plant species via reduced extinction risk but that (ii) climate change and human land use are now disproportionately impacting rare species. Estimates of global species abundance distributions have important implications for risk assessments and conservation planning in this era of rapid global change.},
author = {Enquist, Brian J. and Feng, Xiao and Boyle, Brad and Maitner, Brian and Newman, Erica A. and J{\o}rgensen, Peter M{\o}ller and Roehrdanz, Patrick R. and Thiers, Barbara M. and Burger, Joseph R. and Corlett, Richard T. and Couvreur, Thomas L.P. and Dauby, Gilles and Donoghue, John C. and Foden, Wendy and Lovett, Jon C. and Marquet, Pablo A. and Merow, Cory and Midgley, Guy and Morueta-Holme, Naia and Neves, Danilo M. and Oliveira-Filho, Ary T. and Kraft, Nathan J.B. and Park, Daniel S. and Peet, Robert K. and Pillet, Michiel and Serra-Diaz, Josep M. and Sandel, Brody and Schildhauer, Mark and {\v{S}}{\'{i}}mov{\'{a}}, Irena and Violle, Cyrille and Wieringa, Jan J. and Wiser, Susan K. and Hannah, Lee and Svenning, Jens Christian and McGill, Brian J.},
doi = {10.1126/sciadv.aaz0414},
file = {:Users/alyssacirtwill/Documents/Papers/Enquist et al.{\_}2019{\_}Unknown.pdf:pdf},
issn = {23752548},
journal = {Science Advances},
number = {11},
pages = {1--14},
title = {{The commonness of rarity: Global and future distribution of rarity across land plants}},
volume = {5},
year = {2019}
}
@article{Mora2019,
abstract = {Null models have become a crucial tool for understanding structure within incidence matrices across multiple biological contexts. For example, they have been widely used for the study of ecological and biogeographic questions, testing hypotheses regarding patterns of community assembly, species co-occurrence and biodiversity. However, to our knowledge we remain without a general and flexible approach to study the mechanisms explaining such structures. Here, we provide a method for generating 'correlation-informed' null models, which combine the classic concept of null models and tools from community ecology, like joint statistical modelling. Generally, this model allows us to assess whether the information encoded within any given correlation matrix is predictive for explaining structural patterns observed within an incidence matrix. To demonstrate its utility, we apply our approach to two different case studies that represent examples of common scenarios encountered in community ecology. First, we use a phylogenetically informed null model to detect a strong evolutionary fingerprint within empirically observed food webs, reflecting key differences in the impact of shared evolutionary history when shaping the interactions of predators or prey. Second, we use multiple informed null models to identify which factors determine structural patterns of species assemblages, focusing in on the study of nestedness and the influence of site size, isolation, species range and species richness. In addition to offering a versatile way to study the mechanisms shaping the structure of any incidence matrix, including those describing ecological communities, our approach can also be adapted further to test even more sophisticated hypotheses.},
author = {Mora, Bernat Bramon and {Dalla Riva}, Giulio V. and Stouffer, Daniel B.},
doi = {10.1098/rsif.2018.0747},
file = {:Users/alyssacirtwill/Documents/Papers/Mora, Dalla Riva, Stouffer{\_}2019{\_}Unknown.pdf:pdf},
issn = {17425662},
journal = {Journal of the Royal Society Interface},
keywords = {Ecological networks,Nestedness,Network motifs,Null models,Species assemblages,Structural patterns},
number = {151},
title = {{Unmasking structural patterns in incidence matrices: An application to ecological data}},
volume = {16},
year = {2019}
}
@article{Popovic2019,
abstract = {Ecologists often investigate co-occurrence patterns in multi-species data in order to gain insight into the ecological causes of observed co-occurrences. Apart from direct associations between the two species of interest, they may co-occur because of indirect effects, where both species respond to another variable, whether environmental or biotic (e.g. a mediator species). A wide variety of methods are now available for modelling how environmental filtering drives species distributions. In contrast, methods for studying other causes of co-occurence are much more limited. “Graphical” methods, which can be used to study how mediator species impact co-occurrence patterns, have recently been proposed for use in ecology. However, available methods are limited to presence/absence data or methods assuming multivariate normality, which is problematic when analysing abundances. We propose Gaussian copula graphical models (GCGMs) for studying the effect of mediator species on co-occurence patterns. GCGMs are a flexible type of graphical model which naturally accommodates all data types, for example binary (presence/absence), counts, as well as ordinal data and biomass, in a unified framework. Simulations demonstrate that GCGMs can be applied to a much broader range of data types than the methods currently used in ecology, and perform as well as or better than existing methods in many settings. We apply GCGMs to counts of hunting spiders, in order to visualise associations between species. We also analyse abundance data of New Zealand native forest cover (on an ordinal scale) to show how GCGMs can be used analyse large and complex datasets. In these data, we were able to reproduce known species relationships as well as generate new ecological hypotheses about species associations.},
author = {Popovic, Gordana C. and Warton, David I. and Thomson, Fiona J. and Hui, Francis K.C. and Moles, Angela T.},
doi = {10.1111/2041-210X.13247},
file = {:Users/alyssacirtwill/Documents/Papers/Popovic et al.{\_}2019{\_}Unknown.pdf:pdf},
issn = {2041210X},
journal = {Methods in Ecology and Evolution},
keywords = {Gaussian copula,co-occurence data,graphical models,null model,ordinal data,species associations},
number = {9},
pages = {1571--1583},
title = {{Untangling direct species associations from indirect mediator species effects with graphical models}},
url = {https://besjournals.onlinelibrary.wiley.com/doi/full/10.1111/2041-210X.13247},
volume = {10},
year = {2019}
}
@article{Sebastian-Gonzalez2015,
abstract = {Aim: We aim to characterize the macroecological patterns in the structure of mutualistic seed-dispersal networks. Tropical areas hold more species than temperate ones. This difference in species number may favour ecological processes that minimize interspecific competition in species-rich areas. There is theoretical evidence that both modularity (i.e. the presence of semi-independent groups of highly interacting species) and nestedness (i.e. specialists interact with a subset of the species interacting with generalists) reduce the effects of competition. Thus, we expect high degrees of modularity or nestedness at low latitudes in seed-dispersal networks. Moreover, we test whether climate, topography and human impact influence network structure. Location: Thirty-four qualitative and 21 weighted seed-dispersal interaction networks located world-wide. Methods: We related the degree of modularity and nestedness of seed-dispersal interaction networks with latitude. To disentangle the macroecological drivers of network structure, we also associated modularity/nestedness with species richness, altitudinal range, human impact and an array of climate predictors: precipitation, temperature, precipitation/temperature seasonality and historical climate-change velocity and anomaly. Results: Binary networks showed stronger macroecological patterns than weighted networks. Latitude was unrelated to the structure of seed-dispersal networks, but more nested assemblages were species rich and were located in areas with a high degree of human impact, high temperature seasonality, low precipitation, and, especially on the mainland, high stability in precipitation. Modular networks were species rich and found in areas with low human impact. For both nestedness and modularity, the effects of species richness and human impact were especially strong and consistent. Main conclusions: As for previous macroecological studies of mutualistic networks, we found that the structure of seed-dispersal assemblages was related to current and historical climate. The largest influences on nestedness and modularity, however, were the number of competing species and the degree of human impact. This suggests that human disturbance, not just climate, is an important factor determining the structure of a seed-dispersal network.},
author = {Sebasti{\'{a}}n-Gonz{\'{a}}lez, Esther and Dalsgaard, Bo and Sandel, Brody and Guimar{\~{a}}es, Paulo R.},
doi = {10.1111/geb.12270},
file = {:Users/alyssacirtwill/Documents/Papers/Sebasti{\'{a}}n-Gonz{\'{a}}lez et al.{\_}2015{\_}Unknown.pdf:pdf},
issn = {14668238},
journal = {Global Ecology and Biogeography},
keywords = {Climate,Conservation,Ecological networks,Frugivory,Human impact,Latitude,Mutualism,Species interactions},
number = {3},
pages = {293--303},
title = {{Macroecological trends in nestedness and modularity of seed-dispersal networks: Human impact matters}},
volume = {24},
year = {2015}
}
@article{Cole2019,
author = {Cole, Lorna J. and Kleijn, David and Dicks, Lynn V. and Potts, Simon G. and Albrecht, Matthias and Balzan, Mario V. and Bartomeous, Ignasi and Bebeli, Penelope J. and Bevk, Danilo and Biesmeijer, Jacobus C. and Chlebo, R{\'{o}}bert and D{\`{a}}utart{\"{e}}, Anzelika and Emmanouil, Nikolaos and Hartfield, Chris and Holland, John M. and Holzdchuh, Andrea and Knober, Nieke TJ. and Kov{\'{a}}cs-Hosty{\^{a}}nszki, Anik{\'{o}} and Mandelik, Yael and Panou, Heleni and Paxton, Robert J. and Petanidou, Theodora and {Pinheriro de Varvalho}, Miguel A.A. and Rundl{\"{o}}f, Maj and Sarthou, Jean-Pierre and Stavrinides, Menelaos C. and Suso, Maria Jose and Szentgy{\"{o}}rgyi, Hajnalka and Vaissi{\`{e}}re, Bernard E. and Varanva, Androulla and Vil{\'{a}}, Montserrat and Zemeckis, Romualdas and Scheper, Jeroen},
doi = {10.1111/1365-2664.13572},
journal = {Journal of Applied Ecology},
number = {March 2019},
pages = {1--14},
title = {{A critical analysis of the potential for EU Common Agricultural Policy measures to support wild pollinators on farmland}},
year = {2019}
}
@article{Cirtwill2020,
abstract = {Abstract Related plants are often hypothesised to interact with similar sets of pollinators and herbivores, but this idea has only mixed empirical support. This may be because plant families vary in their tendency to share interaction partners. We quantify overlap of interaction partners for all pairs of plants in 59 pollination and 11 herbivory networks based on the numbers of shared and unshared interaction partners (thereby capturing both proportional and absolute overlap). We test 1) for relationships between phylogenetic distance and partner overlap within each network, 2) whether these relationships varied with the composition of the plant community, and 3) whether wellrepresented plant families showed different relationships. Across all networks, more closely-related plants tended to have greater overlap. The strength of this relationship within a network was unrelated to the composition of the network?s plant component, but, when considered separately, different plant families showed different relationships between phylogenetic distance and overlap of interaction partners. The variety of relationships between phylogenetic distance and partner overlap in different plant families likely reflects a comparable variety of ecological and evolutionary processes. Considering factors affecting particular species-rich groups within a community may be the key to understanding the distribution of interactions at the network level.},
author = {Cirtwill, Alyssa R. and {Dalla Riva}, Giulio V. and Baker, Nick J. and Ohlsson, Mikael and Norstr{\"{o}}m, Isabelle and Wohlfarth, Inger‐Marie and Thia, Joshua A. and Stouffer, Daniel B.},
doi = {10.1111/nph.16420},
file = {:Users/alyssacirtwill/Documents/Papers/Cirtwill et al.{\_}2020{\_}New Phytologist.pdf:pdf},
issn = {0028-646X},
journal = {New Phytologist},
keywords = {and herbi-,ecological networks,herbivory,hypothesized to interact with,niche overlap,phylogenetic signal,pollination,related plants are often,similar sets of pollinators,specialization},
title = {{Related plants tend to share pollinators and herbivores, but strength of phylogenetic signal varies among plant families}},
year = {2020}
}
@article{Devoto2012,
abstract = {Theory developed from studying changes in the structure and function of communities during natural or managed succession can guide the restoration of particular communities. We constructed 30 quantitative plant-flower visitor networks along a managed successional gradient to identify the main drivers of change in network structure. We then applied two alternative restoration strategies in silico (restoring for functional complementarity or redundancy) to data from our early successional plots to examine whether different strategies affected the restoration trajectories. Changes in network structure were explained by a combination of age, tree density and variation in tree diameter, even when variance explained by undergrowth structure was accounted for first. A combination of field data, a network approach and numerical simulations helped to identify which species should be given restoration priority in the context of different restoration targets. This combined approach provides a powerful tool for directing management decisions, particularly when management seeks to restore or conserve ecosystem function. {\textcopyright} 2012 Blackwell Publishing Ltd/CNRS.},
author = {Devoto, Mariano and Bailey, Sallie and Craze, Paul and Memmott, Jane},
doi = {10.1111/j.1461-0248.2012.01740.x},
issn = {14610248},
journal = {Ecology Letters},
keywords = {Ecosystem function,Functional complementarity,Functional redundancy,Pine forest,Plant-animal interaction,Plant-pollinator network,Redundancy analysis,Restoration,Restoration strategy,Succession},
number = {4},
pages = {319--328},
title = {{Understanding and planning ecological restoration of plant-pollinator networks}},
volume = {15},
year = {2012}
}
@article{Foord2016,
abstract = {Aim: Our aim is to test if long-term patterns in $\alpha$ alpha and $\beta$ beta diversity along an elevational transect on two aspects of a mountain are consistent through time using spiders as model organisms, quantify the role of elevation and time (seasonal and inter-annual) in explaining these patterns and partition the relative contribution of nestedness, species turnover and species loss in explaining these diversity patterns. Location: The transect is across the Cederberg mountains in the Cape Floristic Kingdom, Western Cape, South Africa and is constituted by 17 sites with an elevational range of 1900 m on two aspects of the mountain (east and west). Methods: Spider assemblages were sampled biannually (wet and dry seasons) over 6 years. Four replicates per site, each consisting of a 5 × 2 pitfall grid, were sampled for 5 days sessions. Generalized linear mixed models with Poisson error structure for species richness (alpha diversity), binomial error structure for beta diversity (Jaccard dissimilarity $\beta$cc and its partitioned components, $\beta$-3 and $\beta$rich), and Gaussian error structure for beta diversity values standardized by a null model (SES) were used to model the effects of elevation and season on these two indices respectively. Results: Although varying considerably between years, spider alpha diversity had a hump-shaped pattern on the western aspect and U-shaped on the eastern aspect. However, season interacted with elevation to produce more complex patterns during the dry season. There was no significant nestedness except for two instances on the western aspect. Replacement accounted for 60–70{\%} of beta diversity between sites, and elevational distance decay in beta diversity was the result of increased turnover on the western aspect and increased species loss on the eastern aspect. Standardized patterns suggest that there are no effects of season on beta diversity except for a decreased distance decay during the dry season on the western aspect. Main conclusions: Large-scale predictors of spider alpha diversity explained small amounts variation in spider diversity, pointing to the importance of local and stochastic processes. Regional turnover of spider diversity is mainly the result of niche processes, suggesting localized adaptation of taxa, further supported by the lack of nestedness in assemblages.},
author = {Foord, Stefan H. and Dippenaar-Schoeman, Anna S.},
doi = {10.1111/jbi.12817},
file = {:Users/alyssacirtwill/Documents/Papers/Foord, Dippenaar-Schoeman{\_}2016{\_}Journal of Biogeography.pdf:pdf},
issn = {13652699},
journal = {Journal of Biogeography},
keywords = {Araneae,Cape Floristic Region,alpha diversity,beta diversity,epigeal,long term},
number = {12},
pages = {2354--2365},
title = {{The effect of elevation and time on mountain spider diversity: a view of two aspects in the Cederberg mountains of South Africa}},
volume = {43},
year = {2016}
}
@article{Robertson1983,
abstract = {The amphipod Allorchestes compressa Dana inhabits large accumulations of detached macrophytes in the surf-zone of sandy beaches in southern Western Australia. A. compressa is most abundant on branching red algae and least abundant on intact thalli of the kelp Ecklonia radiata (Turn.) J. Agardh., yet the major component of the gut contents is brown algae (probably E. radiata) and decomposing E. radiata ranked first in laboratory food preference experiments. Observations on the feeding behaviour of Allorchestes compressa indicated that the amphipods obtain their food by feeding on small pieces ({\textless} 3 cm) of macrophyte tissue trapped within the highly branched algae, or amphipods may move with and feed on the plant particles as they are swept around in the surf. In a particle selection experiment, using plant particles 1-3 mm sieved from the surf, A. compressa selected particles of Ecklonia radiata, leached Ulva sp., Sargassum spp., and seagrass leaves but avoided branching red algae. The influence of potential foods on the darwinian fitness of Allorchestes compressa was assessed on the basis of adult survival, the percentage of females which carried eggs, growth rates, and time to maturity measured in laboratory rearing experiments. Fitness increased in the order red algae → intact seagrass leaves → mixed particles (1-3 mm) sieved from the surf → Ecklonia radiata tissue. Given the constraints of fish predation and the fluctuating supply of E. radiata, amphipods in the surf consume close to their theoretically optimum diet by feeding mainly on E. radiata from amongst the available particles of different macrophytes. Estimates of the significance of the Allorchestes compressa population in the turnover of Ecklonia radiata biomass in the surf-zone (estimated as g Ecklonia consumed per g Ecklonia per day) showed that amphipods could turnover E. radiata biomass twice per month in summer and once every 1 to 2 months during spring and autumn. These rates are comparable with those measured for the physical breakdown and microbial decomposition of E. radiata and, except during winter, grazing by Allorchestes compressa must, therefore, be considered an important process during the remineralization of nutrients tied up in kelp biomass in the surf-zone. {\textcopyright} 1983.},
author = {Robertson, A. I. and Lucas, J. S.},
doi = {10.1016/0022-0981(83)90138-7},
issn = {00220981},
journal = {Journal of Experimental Marine Biology and Ecology},
number = {2},
pages = {99--124},
title = {{Food choice, feeding rates, and the turnover of macrophyte biomass by a surf-zone inhabiting amphipod}},
volume = {72},
year = {1983}
}
@article{Samnegard2019,
abstract = {Apple is considered the most important fruit crop in temperate areas and profitable production depends on multiple ecosystem services, including the reduction of pest damage and the provision of sufficient pollination levels. Management approaches present an inherent trade-off as each affects species differently. We quantified the direct and indirect effects of management (organic vs. integrated pest management, IPM) on species richness, ecosystem services, and fruit production in 85 apple orchards in three European countries. We also quantified how habit composition influenced these effects at three spatial scales: within orchards, adjacent to orchards, and in the surrounding landscape. Organic management resulted in 48{\%} lower yield than IPM, and also that the variation between orchards was large with some organic orchards having a higher yield than the average yield of IPM orchards. The lower yield in organic orchards resulted directly from management practices, and from higher pest damage in organic orchards. These negative yield effects were partly offset by indirect positive effects from more natural enemies and higher flower visitation rates in organic orchards. Two factors other than management affected species richness and ecosystem services. Higher cover of flowering plants within and adjacent to the apple trees increased flower visitation rates by pollinating insects and a higher cover of apple orchards in the landscape decreased species richness of beneficial arthropods. The species richness of beneficial arthropods in orchards was uncorrelated with fruit production, suggesting that diversity can be increased without large yield loss. At the same time, organic orchards had 38{\%} higher species richness than IPM orchards, an effect that is likely due to differences in pest management. Synthesis and applications. Our results indicate that organic management is more efficient than integrated pest management in developing environmentally friendly apple orchards with higher species richness. We also demonstrate that there is no inherent trade-off between species richness and yield. Development of more environmentally friendly means for pest control, which do not negatively affect pollination services, needs to be a priority for sustainable apple production.},
author = {Samneg{\aa}rd, Ulrika and Alins, Georgina and Boreux, Virginie and Bosch, Jordi and Garc{\'{i}}a, Daniel and Happe, Anne Kathrin and Klein, Alexandra Maria and Mi{\~{n}}arro, Marcos and Mody, Karsten and Porcel, Mario and Rodrigo, Anselm and Roquer-Beni, Laura and Tasin, Marco and Hamb{\"{a}}ck, Peter A.},
doi = {10.1111/1365-2664.13292},
file = {:Users/alyssacirtwill/Documents/Papers/Samneg{\aa}rd et al.{\_}2019{\_}Journal of Applied Ecology.pdf:pdf},
issn = {13652664},
journal = {Journal of Applied Ecology},
keywords = {apple production,biological control,integrated pest management,natural enemies,organic management,pollination services,species richness,structural equation model},
number = {4},
pages = {802--811},
title = {{Management trade-offs on ecosystem services in apple orchards across Europe: Direct and indirect effects of organic production}},
volume = {56},
year = {2019}
}
@article{McInnes2017a,
abstract = {DNA metabarcoding of food in animal scats provides a non-invasive dietary analysis method for vertebrates. A variety of molecular approaches can be used to recover dietary DNA from scats; however, many of these also recover non-food DNA. Blocking primers can be used to inhibit amplification of some non-target DNA, but this may not always be feasible, especially when multiple distinct non-target groups are present. We have developed scat collection protocols to optimise the detection of food DNA in vertebrate scat samples. Using shy albatross Thalassarche cauta as a case study, we investigated how DNA amplification success and the proportion of food DNA detected are influenced by both environmental and physiological parameters. We show that both the amount and type of non-target DNA vary with sample freshness, the collection substrate, fasting period and developmental stage of the consumer. Fresh scat samples yielded the highest proportion of food sequences. Collecting scats from dirt substrates reduced the proportion of food DNA and increased the proportion of contaminating DNA. Food DNA detection rates changed throughout the albatross breeding season and related to the time since feeding and the developmental stage of the animal. Fasting albatross produced scats dominated by parasite amplicons in universal PCR analysis, with little food DNA recovered. Samples from very young animals also produced reduced food DNA proportions. Based on our observations, we recommend the following procedures for field scat collections to ensure high-quality samples for dietary DNA metabarcoding studies. Ideally, (i) collect fresh scats; (ii) from surfaces with minimal contamination (e.g. rock or ice); (iii) collect scats from animals with minimum time since feeding and avoid fasting animals; (iv) avoid young animals that are not feeding directly (e.g. not weaned or fledged) or target larger/older individuals. The optimised field sampling protocols that we describe will improve the quality of dietary data from vertebrates by focusing on samples most likely to contain food DNA. They will also help minimise contamination issues from non-target DNA and provide standardised field methods in this rapidly expanding area of research.},
author = {McInnes, Julie C. and Alderman, Rachael and Deagle, Bruce E. and Lea, Mary Anne and Raymond, Ben and Jarman, Simon N.},
doi = {10.1111/2041-210X.12677},
file = {:Users/alyssacirtwill/Documents/Papers/McInnes et al.{\_}2017{\_}Methods in Ecology and Evolution.pdf:pdf},
issn = {2041210X},
journal = {Methods in Ecology and Evolution},
keywords = {albatross,cestoda,faeces,food,molecular,non-invasive,parasites,prey,procellariiformes,seabird},
number = {2},
pages = {192--202},
title = {{Optimised scat collection protocols for dietary DNA metabarcoding in vertebrates}},
volume = {8},
year = {2017}
}
@article{Liu2019,
author = {Liu, Mingxin and Clarke, Laurence J. and Baker, Susan C and Jordan, Gregory J and Burridge, Christopher P},
doi = {10.1111/een.12831},
file = {:Users/alyssacirtwill/Documents/Papers/Liu et al.{\_}2019{\_}Ecological Entomology.pdf:pdf},
journal = {Ecological Entomology},
number = {December},
title = {{A practical guide to DNA metabarcoding for}},
year = {2019}
}
@article{Eitzinger2018,
abstract = {In mammalian cells, mismatch recognition has been attributed to two partially redundant heterodimeric protein complexes of MutS homologues, MSH2-MSH3 and MSH2-MSH6. We have conducted a comparative analysis of Msh3 and Msh6 deficiency in mouse intestinal tumorigenesis by generating Apc1638N mice deficient in Msh3, Msh6 or both. We have found that Apc1638N mice defective in Msh6 show reduced survival and a 6-7- fold increase in intestinal tumor multiplicity. In contrast, Msh3- deficient Apc1638N mice showed no difference in survival and intestinal tumor multiplicity as compared with Apc1638N mice. However, when Msh3 deficiency is combined with Msh6 deficiency (Msh3(-/-)Msh6(-/- )Apc1638N), the survival rate of the mice was further reduced compared to Msh6(-/-)Apc(1638N) mice because of a high multiplicity of intestinal tumors at a younger age. Almost 90{\%} of the intestinal tumors from both Msh6(-/-)Apc1638N and Msh3(-/-)Msh6(-/-)Apc1638N mice contained truncation mutations in the wild-type Apc allele. Apc mutations in Msh6(-/-)Apc1638N mice consisted predominantly of base substitutions (93{\%}) creating stop codons, consistent with a major role for Msh6 in the repair of base-base mismatches. However, in Msh3(-/- )Msh6(-/-)Apc1638N tumors, we observed a mixture of base substitutions (46{\%}) and frameshifts (54{\%}), indicating that in Msh6(-/-)Apc1638N mice frameshift mutations in the Apc gene were suppressed by Msh3. Interestingly, all except one of the Apc mutations detected in mismatch repair-deficient intestinal tumors were located upstream of the third 20-amino acid beta-catenin binding repeat and before all of the Ser-Ala- Met-Pro repeats, suggesting that there is selection for loss of multiple domains involved in beta-catenin regulation. Our analysis therefore has revealed distinct mutational spectra and clarified the roles of Msh3 and Msh6 in DNA repair and intestinal tumorigenesis.},
author = {Eitzinger, Bernhard and Abrego, Nerea and Gravel, Dominique and Huotari, Tea and Vesterinen, Eero J. and Roslin, Tomas},
doi = {10.1111/mec.14872},
file = {:Users/alyssacirtwill/Documents/Papers/Eitzinger et al.{\_}2019{\_}Molecular Ecology(2).pdf:pdf},
journal = {Molecular Ecology},
keywords = {altitudinal gradient,body mass,interaction probability,lycosidae,metabarcoding,predator–prey interaction},
number = {2},
pages = {266--280},
title = {{Assessing changes in arthropod predator–prey interactions through DNA-based gut content analysis—variable environment, stable diet}},
volume = {28},
year = {2019}
}
@article{Bowen2013,
abstract = {Diet estimation in marine mammals relies on indirect methods including recovery of prey hard parts from stomachs and feces, quantitative fatty acid signature analysis (QFASA), stable isotope mixing models, and identification of prey DNA in stomach contents and feces. Experimental evidence (9 species/13 studies) shows that digestion strongly influences the proportion and size of otoliths that can be recovered in feces. Number correction factors (NCF) and digestion coefficients have been experimentally determined to reduce the biases in fecal analysis. Correction factors and coefficients have not been determined for diet estimated from stomach contents. QFASA estimates which prey species and amounts must have been eaten to account for the fatty acid composition of the predator. Experimental studies on mammals and seabirds (9 species/10 studies) indicate that accurate estimates of diet can be determined using QFASA. Stable isotope mixing models provide rather coarse taxonomic resolution of diet composition. Prey DNA analysis shows promise as a method to estimate the species composition of diet, but further development and testing is needed to validate its use. Obtaining a representative sample from marine mammal populations is a significant challenge. Therefore, the use of complementary methods is recommended to obtain the most informative results. {\textcopyright} 2012 by the Society for Marine Mammalogy.},
author = {Bowen, W. D. and Iverson, S. J.},
doi = {10.1111/j.1748-7692.2012.00604.x},
issn = {08240469},
journal = {Marine Mammal Science},
keywords = {DNA,Fatty acids,Feces,Otoliths,Stable isotopes},
number = {4},
pages = {719--754},
title = {{Methods of estimating marine mammal diets: A review of validation experiments and sources of bias and uncertainty}},
volume = {29},
year = {2013}
}
@article{McInnes2017,
abstract = {Almost all of the world's fisheries overlap spatially and temporally with foraging seabirds, with impacts that range from food supplementation (through scavenging behind vessels), to resource competition and incidental mortality. The nature and extent of interactions between seabirds and fisheries vary, as does the level and efficacy of management and mitigation. Seabird dietary studies provide information on prey diversity and often identify species that are also caught in fisheries, providing evidence of linkages which can be used to improve ecosystem based management of fisheries. However, species identification of fish can be difficult with conventional dietary techniques. The black-browed albatross (Thalassarche melanophris) has a circumpolar distribution and has suffered major population declines due primarily to incidental mortality in fisheries. We use DNA metabarcoding of black-browed albatross scats to investigate their fish prey during the breeding season at six sites across their range, over two seasons. We identify the spatial and temporal diversity of fish in their diets and overlaps with fisheries operating in adjacent waters. Across all sites, 51 fish species from 33 families were identified, with 23 species contributing {\textgreater}10{\%} of the proportion of samples or sequences at any site. There was extensive geographic variation but little inter-annual variability in fish species consumed. Several fish species that are not easily accessible to albatross, but are commercially harvested or by-caught, were detected in the albatross diet during the breeding season. This was particularly evident at the Falkland Islands and Iles Kerguelen where higher fishery catch amounts (or discard amounts where known) corresponded to higher occurrence of these species in diet samples. This study indicates ongoing interactions with fisheries through consumption of fishery discards, increasing the risk of seabird mortality. Breeding success was higher at sites where fisheries discards were detected in the diet, highlighting the need to minimize discarding to reduce impacts on the ecosystem. DNA metabarcoding provides a valuable non-invasive tool for assessing the fish prey of seabirds across broad geographic ranges. This provides an avenue for fishery resource managers to assess compliance of fisheries with discard policies and the level of interaction with scavenging seabirds.},
author = {McInnes, Julie C. and Jarman, Simon N. and Lea, Mary Anne and Raymond, Ben and Deagle, Bruce E. and Phillips, Richard A. and Catry, Paulo and Stanworth, Andrew and Weimerskirch, Henri and Kusch, Alejandro and Gras, Micha{\"{e}}l and Cherel, Yves and Maschette, Dale and Alderman, Rachael},
doi = {10.3389/fmars.2017.00277},
file = {:Users/alyssacirtwill/Documents/Papers/McInnes et al.{\_}2017{\_}Frontiers in Marine Science.pdf:pdf},
issn = {22967745},
journal = {Frontiers in Marine Science},
keywords = {Fish diversity,Fisheries resource management,Scat,Seabird-fishery interaction,Seabirds,Southern ocean,Thalassarche melanophris,Trawl fishery},
pages = {277},
title = {{DNA metabarcoding as a marine conservation and management tool: a circumpolar examination of fishery discards in the diet of threatened albatrosses}},
volume = {4},
year = {2017}
}
@article{McInnes2017b,
abstract = {Gelatinous zooplankton are a large component of the animal biomass in all marine environments, but are considered to be uncommon in the diet of most marine top predators. However, the diets of key predator groups like seabirds have conventionally been assessed from stomach content analyses, which cannot detect most gelatinous prey. As marine top predators are used to identify changes in the overall species composition of marine ecosystems, such biases in dietary assessment may impact our detection of important ecosystem regime shifts. We investigated albatross diet using DNA metabarcoding of scats to assess the prevalence of gelatinous zooplankton consumption by two albatross species, one of which is used as an indicator species for ecosystem monitoring. Black-browed and Campbell albatross scats were collected from eight breeding colonies covering the circumpolar range of these birds over two consecutive breeding seasons. Fish was the main dietary item at most sites; however, cnidarian DNA, primarily from scyphozoan jellyfish, was present in 42{\%} of samples overall and up to 80{\%} of samples at some sites. Jellyfish was detected during all breeding stages and consumed by adults and chicks. Trawl fishery catches of jellyfish near the Falkland Islands indicate a similar frequency of jellyfish occurrence in albatross diets in years of high and low jellyfish availability, suggesting jellyfish consumption may be selective rather than opportunistic. Warmer oceans and overfishing of finfish are predicted to favour jellyfish population increases, and we demonstrate here that dietary DNA metabarcoding enables measurements of the contribution of gelatinous zooplankton to the diet of marine predators.},
author = {McInnes, Julie C. and Alderman, Rachael and Lea, Mary Anne and Raymond, Ben and Deagle, Bruce E. and Phillips, Richard A. and Stanworth, Andrew and Thompson, David R. and Catry, Paulo and Weimerskirch, Henri and Suazo, Cristi{\'{a}}n G. and Gras, Micha{\"{e}}l and Jarman, Simon N.},
doi = {10.1111/mec.14245},
file = {:Users/alyssacirtwill/Documents/Papers/McInnes et al.{\_}2017{\_}Molecular Ecology.pdf:pdf},
issn = {1365294X},
journal = {Molecular Ecology},
keywords = {climate change,cnidarians,faeces,food,indicator species,scats,seabird},
number = {18},
pages = {4831--4845},
title = {{High occurrence of jellyfish predation by black-browed and Campbell albatross identified by DNA metabarcoding}},
volume = {26},
year = {2017}
}
@article{Leyland2005,
abstract = {This paper reviews empirical Bayes methods for disease mapping. A distinction is made between spatial models (which take into account the geographical distribution of disease) and nonspatial models. Several estimators are presented, and methods of estimation are described. Empirical Bayes methods are compared with full Bayes methods, and we argue that both have their place.},
author = {Leyland, Alastair H. and Davies, Carolyn A.},
doi = {10.1191/0962280205sm387oa},
file = {:Users/alyssacirtwill/Documents/Papers/Leyland, Davies{\_}2005{\_}Statistical Methods in Medical Research.pdf:pdf},
isbn = {0962-2802 (Print)},
issn = {09622802},
journal = {Statistical Methods in Medical Research},
number = {1},
pages = {17--34},
pmid = {15690998},
title = {{Empirical Bayes methods for disease mapping}},
volume = {14},
year = {2005}
}
@article{Pitt2009,
abstract = {Studies of the trophic ecology of gelatinous zooplankton have predominantly employed gut content analyses and grazing experiments. These approaches record only what is consumed rather than what is assimilated by the jellyfish, only provide evidence of recent feeding, and unless digestion rates of different prey are known, may provide biased estimates of the relative importance of different prey to jellyfish diets. Biochemical tracers, such as stable isotopes and fatty acids, offer several advantages because they differentiate between what is assimilated and what is simply ingested, they provide an analysis of diet that is integrated over time, and may be useful for identifying contributions from sources (e.g., bacteria) that cannot be achieved using gut content approaches. Stable isotope analysis has become more rigorous through recent advances that provide: (1) signature determination of microscopic organisms such as microalgae, (2) analysis of dissolved organic carbon, and (3) improved quantification of relative source contributions. The limitation that natural tracer techniques require different dietary sources to have unique signatures can potentially be overcome using pulse-chase isotope enrichment experiments. Trophic studies of gelatinous zooplankton would benefit by integrating several approaches. For example, gut content analyses may be used to identify potential dietary sources. Stable isotopes could then be used to determine which sources are assimilated and modeling could be used to quantify the contribution of different sources to the diet. Analysis of fatty acid profiles could be used to identify contributions of bacterioplankton to the diet and, potentially, to provide an alternative means of identifying dietary sources in situations where the isotopic signatures of different potential dietary sources overlap. In this review, we outline the application, advantages, and limitations of gut content analyses and stable isotope and fatty acid tracer techniques and discuss the benefits of using an integrated approach toward studies of the trophic ecology of gelatinous zooplankton. {\textcopyright} 2008 Springer Science+Business Media B.V.},
author = {Pitt, K. A. and Connolly, R. M. and Meziane, T.},
doi = {10.1007/s10750-008-9581-z},
file = {:Users/alyssacirtwill/Documents/Papers/Pitt, Connolly, Meziane{\_}2009{\_}Hydrobiologia.pdf:pdf},
issn = {00188158},
journal = {Hydrobiologia},
keywords = {Diet,Gelatinous zooplankton,Gut contents,Trophic ecology},
number = {1},
pages = {119--132},
title = {{Stable isotope and fatty acid tracers in energy and nutrient studies of jellyfish: a review}},
volume = {616},
year = {2009}
}
@article{Peralta2019,
author = {Peralta, Guadalupe and Stouffer, Daniel B. and Bringa, Eduardo M. and V{\'{a}}zquez, Diego P.},
doi = {10.1111/oik.06503},
journal = {Oikos},
title = {{No such thing as a free lunch: interaction costs and the structure and stability of mutualistic networks}},
year = {2019}
}
@article{Drezner2016,
author = {Drezner, Zvi and Drezner, Taly Dawn},
doi = {10.1002/bes2.1214},
isbn = {2103481003},
issn = {23276096},
journal = {Bulletin of the Ecological Society of America},
keywords = {abstract,analysis,and cataloging biodiversity,camera trap,camera trapping,camera-trapping,data,imagery from camera traps,in the,inventory and monitoring,method,networks,picture,statistics,such use of camera,supports ecological investigations,tool,traps continues to expand},
number = {1},
pages = {91--98},
title = {{A remedy for the overzealous Bonferroni technique for multiple statistical tests}},
volume = {97},
year = {2016}
}
@article{Moran2003,
author = {Moran, Matthew D.},
doi = {10.1034/j.1600-0706.2003.12010.x},
issn = {00301299},
journal = {Oikos},
number = {2},
pages = {403--405},
title = {{Arguments for rejecting the sequential bonferroni in ecological studies}},
volume = {100},
year = {2003}
}
@article{Brandle2006,
abstract = {We explore the relationship between the pairwise similarity of assemblages of exploiters (phytophagous insects and parasitic fungi) and pairwise genetic distance, range overlap, niche overlap as well as habitat overlap of host trees. Presence of exploiters was extracted from published literature for 23 tree genera occurring in central Europe (6164 host records of phytophagous insects and 860 host records of parasitic fungi). Across all pairs of tree genera, we found a strong negative correlation between the pairwise similarity of assemblages and genetic distances of hosts. This close correlation is due to deep differences in the composition of assemblages on coniferous and deciduous tree genera. Range, niche and habitat overlap were always of much less importance than genetic distance to explain the variation of pairwise similarity of assemblages of exploiters, although some correlations were significant. Therefore in general host switches of exploiters between related hosts are more important that host switches between hosts co-occurring in the same habitat. We found a robust relationship of the pairwise similarity of assemblages of insects and the pairwise similarity of assemblages of fungi which points to the possibility that insects are vectors for parasitic fungi which promotes correlated switches of insects and fungi. Copyright {\textcopyright} Oikos 2006.},
author = {Br{\"{a}}ndle, Martin and Brandl, Roland},
doi = {10.1111/j.2006.0030-1299.14418.x},
issn = {00301299},
journal = {Oikos},
number = {2},
pages = {296--304},
title = {{Is the composition of phytophagous insects and parasitic fungi among trees predictable?}},
volume = {113},
year = {2006}
}
@article{Maynard2018,
abstract = {Ecological networks that exhibit stable dynamics should theoretically persist longer than those that fluctuate wildly. Thus, network structures which are over-represented in natural systems are often hypothesised to be either a cause or consequence of ecological stability. Rarely considered, however, is that these network structures can also be by-products of the processes that determine how new species attempt to join the community. Using a simulation approach in tandem with key results from random matrix theory, we illustrate how historical assembly mechanisms alter the structure of ecological networks. We demonstrate that different community assembly scenarios can lead to the emergence of structures that are often interpreted as evidence of ‘selection for stability'. However, by controlling for the underlying selection pressures, we show that these assembly artefacts—or spandrels—are completely unrelated to stability or selection, and are instead by-products of how new species are introduced into the system. We propose that these network-assembly spandrels are critically overlooked aspects of network theory and stability analysis, and we illustrate how a failure to adequately account for historical assembly can lead to incorrect inference about the causes and consequences of ecological stability.},
author = {Maynard, Daniel S. and Serv{\'{a}}n, Carlos A. and Allesina, Stefano},
doi = {10.1111/ele.12912},
issn = {14610248},
journal = {Ecology Letters},
keywords = {Coexistence,community assembly,interspecific competition,network structure,stability},
number = {3},
pages = {324--334},
title = {{Network spandrels reflect ecological assembly}},
volume = {21},
year = {2018}
}
@article{Altermatt2011,
abstract = {Many herbivorous insects feed on plant tissues as larvae but use other resources as adults. Adult nectar feeding is an important component of the diet of many adult herbivores, but few studies have compared adult and larval feeding for broad groups of insects. We compiled a data set of larval host use and adult nectar sources for 995 butterfly and moth species (Lepidoptera) in central Europe. Using a phylogenetic generalized least squares approach, we found that those Lepidoptera that fed on a wide range of plant species as larvae were also nectar feeding on a wide range of plant species as adults. Lepidoptera that lack functional mouthparts as adults used more plant species as larval hosts, on average, than did Lepidoptera with adult mouthparts. We found that 54{\%} of Lepidoptera include their larval host as a nectar source. By creating null models that described the similarity between larval and adult nectar sources, we furthermore showed that Lepidoptera nectar feed on their larval host more than would be expected if they fed at random on available nectar sources. Despite nutritional differences between plant tissue and nectar, we show that there are similarities between adult and larval feeding in Lepidoptera. This suggests that either behavioral or digestive constraints are retained throughout the life cycle of holometabolous herbivores, which affects host breadth and identity. {\textcopyright} 2011 by The University of Chicago.},
author = {Altermatt, Florian and Pearse, Ian S.},
doi = {10.1086/661248},
file = {:Users/alyssacirtwill/Documents/Papers/Altermatt, Pearse{\_}2011{\_}American Naturalist.pdf:pdf},
issn = {00030147},
journal = {American Naturalist},
keywords = {Food plants,Herbivorous insects,Larval diet,Lepidoptera,Nectar-producing plants,Plant-insect interactions},
number = {3},
pages = {372--382},
title = {{Similarity and specialization of the larval versus adult diet of European butterflies and moths}},
volume = {178},
year = {2011}
}
@incollection{Strauss2006,
abstract = {Despite the dominating role of pollinators in floral evolution, mounting evidence reveals significant additional, often antagonistic, influences of abiotic and biotic non-pollinator agents. Even when pollinators and other agents impose selection on floral traits in the ...$\backslash$n},
address = {Oxford, UK},
author = {Strauss, Sharon Y and Whittall, Justen B},
booktitle = {Ecology and evolution of flowers},
chapter = {7},
edition = {1},
editor = {Harder, Lawrence D. and Barrett, Spencer C. H.},
isbn = {0198570864},
pages = {120--138},
publisher = {Oxford University Press},
title = {{Non-pollinator agents of selection on floral traits}},
url = {http://www.researchgate.net/profile/Sharon{\_}Strauss/publication/254469747{\_}Non-pollinator{\_}agents{\_}of{\_}selection{\_}on{\_}floral{\_}traits/links/53d275b60cf220632f3c9c39.pdf{\%}5Cnpapers2://publication/uuid/F1BC2A23-C948-4B44-A97C-C6D1A238C113},
year = {2006}
}
@article{Astegiano2017,
abstract = {Species establish different interactions (e.g. antagonistic, mutualistic) with multiple species, forming multilayer ecological networks. Disentangling network co-structure in multilayer networks is crucial to predict how biodiversity loss may affect the persistence of multispecies assemblages. Existing methods to analyse multilayer networks often fail to consider network co-structure. We present a new method to evaluate the modular co-structure of multilayer networks through the assessment of species degree co-distribution and network module composition. We focus on modular structure because of its high prevalence among ecological networks. We apply our method to two Lepidoptera-plant networks, one describing caterpillar-plant herbivory interactions and one representing adult Lepidoptera nectaring on flowers, thereby possibly pollinating them. More than 50{\%} of the species established either herbivory or visitation interactions, but not both. These species were over-represented among plants and lepidopterans, and were present in most modules in both networks. Similarity in module composition between networks was high but not different from random expectations. Our method clearly delineates the importance of interpreting multilayer module composition similarity in the light of the constraints imposed by network structure to predict the potential indirect effects of species loss through interconnected modular networks.},
author = {Astegiano, Julia and Altermatt, Florian and Massol, Fran{\c{c}}ois},
doi = {10.1038/s41598-017-15811-w},
file = {:Users/alyssacirtwill/Documents/Papers/Astegiano, Altermatt, Massol{\_}2017{\_}Scientific Reports.pdf:pdf},
issn = {20452322},
journal = {Scientific Reports},
number = {1},
pages = {1--11},
title = {{Disentangling the co-structure of multilayer interaction networks: Degree distribution and module composition in two-layer bipartite networks}},
volume = {7},
year = {2017}
}
@article{Melian2009,
abstract = {Most studies on ecological networks consider only a single interaction type (e.g. competitive, predatory or mutualistic), and try to developrules for system stability based exclusively on properties of this interaction type. However, the stability of ecological networks may be more dependent on the way different interaction types are combined in real communities. To address this issue, we start by compiling an ecological network in the Do{\~{n}}ana Biological Reserve, southern Spain, with 390 species and 798 mu-tualistic and antagonistic interactions. We characterize network structure by looking at how mutualistic and antagonistic interactions are combined across all plant species. Both the ratio of mutualistic to antagonistic interactions per plant, and the number of basic modules with an antagonistic and a mutualistic interaction are very heterogeneous across plant species, with a few plant species showing very high values for these parameters. To assess the implications of these network patterns on species diversity, we study analytically and by simulation a model of this ecological network. We find that the observed correlation between strong interaction strengths and high mutualistic to antagonistic ratios in a few plant species significantly increases community diversity. Thus, to predict the persistence of biodiversity we need to understand how interaction strength and the architecture of ecological networks with different interaction types are combined. {\textcopyright} 2008 The Authors.},
author = {Meli{\'{a}}n, Carlos J. and Bascompte, Jordi and Jordano, Pedro and Křivan, Vlastimil},
doi = {10.1111/j.1600-0706.2008.16751.x},
issn = {00301299},
journal = {Oikos},
number = {1},
pages = {122--130},
title = {{Diversity in a complex ecological network with two interaction types}},
volume = {118},
year = {2009}
}
@article{Hembry2018,
abstract = {Biological intimacy—the degree of physical proximity or integration of partner taxa during their life cycles—is thought to promote the evolution of reciprocal specialization and modularity in the networks formed by co-occurring mutualistic species, but this hypothesis has rarely been tested. Here, we test this “biological intimacy hypothesis” by comparing the network architecture of brood pollination mutualisms, in which specialized insects are simultaneously parasites (as larvae) and pollinators (as adults) of their host plants to that of other mutualisms which vary in their biological intimacy (including ant-myrmecophyte, ant-extrafloral nectary, plant-pollinator and plant-seed disperser assemblages). We use a novel dataset sampled from leafflower trees (Phyllanthaceae: Phyllanthus s. l. [Glochidion]) and their pollinating leafflower moths (Lepidoptera: Epicephala) on three oceanic islands (French Polynesia) and compare it to equivalent published data from congeners on continental islands (Japan). We infer taxonomic diversity of leafflower moths using multilocus molecular phylogenetic analysis and examine several network structural properties: modularity (compartmentalization), reciprocality (symmetry) of specialization and algebraic connectivity. We find that most leafflower-moth networks are reciprocally specialized and modular, as hypothesized. However, we also find that two oceanic island networks differ in their modularity and reciprocal specialization from the others, as a result of a supergeneralist moth taxon which interacts with nine of 10 available hosts. Our results generally support the biological intimacy hypothesis, finding that leafflower-moth networks (usually) share a reciprocally specialized and modular structure with other intimate mutualisms such as ant-myrmecophyte symbioses, but unlike nonintimate mutualisms such as seed dispersal and nonintimate pollination. Additionally, we show that generalists—common in nonintimate mutualisms—can also evolve in intimate mutualisms, and that their effect is similar in both types of assemblages: once generalists emerge they reshape the network organization by connecting otherwise isolated modules.},
author = {Hembry, David H. and Raimundo, Rafael L.G. and Newman, Erica A. and Atkinson, Lesje and Guo, Chang and Guimar{\~{a}}es, Paulo R. and Gillespie, Rosemary G.},
doi = {10.1111/1365-2656.12841},
issn = {13652656},
journal = {Journal of Animal Ecology},
keywords = {Epicephala,Glochidion,Phyllanthus,biological intimacy hypothesis,co-evolution,modularity,network evolution,reciprocal specialization},
number = {4},
pages = {1160--1171},
title = {{Does biological intimacy shape ecological network structure? A test using a brood pollination mutualism on continental and oceanic islands}},
volume = {87},
year = {2018}
}
@article{Simmons2019,
abstract = {Indirect interactions play an essential role in governing population, community and coevolutionary dynamics across a diverse range of ecological communities. Such communities are widely represented as bipartite networks: graphs depicting interactions between two groups of species, such as plants and pollinators or hosts and parasites. For over thirty years, studies have used indices, such as connectance and species degree, to characterise the structure of these networks and the roles of their constituent species. However, compressing a complex network into a single metric necessarily discards large amounts of information about indirect interactions. Given the large literature demonstrating the importance and ubiquity of indirect effects, many studies of network structure are likely missing a substantial piece of the ecological puzzle. Here we use the emerging concept of bipartite motifs to outline a new framework for bipartite networks that incorporates indirect interactions. While this framework is a significant departure from the current way of thinking about bipartite ecological networks, we show that this shift is supported by analyses of simulated and empirical data. We use simulations to show how consideration of indirect interactions can highlight differences missed by the current index paradigm that may be ecologically important. We extend this finding to empirical plant-pollinator communities, showing how two bee species, with similar direct interactions, differ in how specialised their competitors are. These examples underscore the need to not rely solely on network-and species-level indices for characterising the structure of bipartite ecological networks. Forum Although bipartite ecological networks are a widely-used representation of ecological communities, current methods for describing them-using network-and species-level indices-discard a lot of information about network structure. Here we argue for increased use of motifs (small subnetworks depicting interactions between a given number of species) to characterise bipartite networks. We show that motifs capture significantly more information about network structure than indices in both simulated and empirical data, and find that they are remarkably robust to sampling effects. We discuss how these findings can be used to advance the study of ecological networks. Synthesis},
author = {Simmons, Benno I. and Cirtwill, Alyssa R. and Baker, Nick J. and Wauchope, Hannah S. and Dicks, Lynn V. and Stouffer, Daniel B. and Sutherland, William J.},
doi = {10.1111/oik.05670},
issn = {16000706},
journal = {Oikos},
keywords = {ecological networks,food webs,herbivory,indirect interactions,motifs,mutualistic networks,parasitism,pollination,seed dispersal},
number = {2},
pages = {154--170},
title = {{Motifs in bipartite ecological networks: uncovering indirect interactions}},
url = {https://www.biorxiv.org/content/early/2018/05/04/315010},
volume = {128},
year = {2019}
}
@article{Eklof2013a,
abstract = {Summary: Ecological communities are composed of populations connected in tangled networks of ecological interactions. Therefore, the extinction of a species can reverberate through the network and cause other (possibly distantly connected) species to go extinct as well. The study of these secondary extinctions is a fertile area of research in ecological network theory. However, to facilitate practical applications, several improvements to the current analytical approaches are needed. In particular, we need to consider that (i) species have different 'a priori' probabilities of extinction, (ii) disturbances can simultaneously affect several species, and (iii) extinction risk of consumers likely grows with resource loss. All these points can be included in dynamical models, which are, however, difficult to parameterize. Here we advance the study of secondary extinctions with Bayesian networks. We show how this approach can account for different extinction responses using binary - where each resource has the same importance - and quantitative data - where resources are weighted by their importance. We simulate ecological networks using a popular dynamical model (the Allometric Trophic Network model) and use it to test our method. We find that the Bayesian network model captures the majority of the secondary extinctions produced by the dynamical model and that consumers' responses to species loss are best modelled using a nonlinear sigmoid function. We also show that an approach based exclusively on food web structure loses power when species at higher trophic levels are preferentially lost. Because the loss of apex predators is unfortunately widespread, the results highlight a serious limitation of studies on network robustness. {\textcopyright} 2013 British Ecological Society.},
author = {Ekl{\"{o}}f, Anna and Tang, Si and Allesina, Stefano},
doi = {10.1111/2041-210X.12062},
issn = {2041210X},
journal = {Methods in Ecology and Evolution},
keywords = {Bayesian networks,Biodiversity loss,Cascading extinctions,Dynamical model,Food webs},
number = {8},
pages = {760--770},
title = {{Secondary extinctions in food webs: A Bayesian network approach}},
volume = {4},
year = {2013}
}
@article{Guimaraes2007,
abstract = {The structure of mutualistic networks provides clues to processes shaping biodiversity [1-10]. Among them, interaction intimacy, the degree of biological association between partners, leads to differences in specialization patterns [4, 11] and might affect network organization [12]. Here, we investigated potential consequences of interaction intimacy for the structure and coevolution of mutualistic networks. From observed processes of selection on mutualistic interactions, it is expected that symbiotic interactions (high-interaction intimacy) will form species-poor networks characterized by compartmentalization [12, 13], whereas nonsymbiotic interactions (low intimacy) will lead to species-rich, nested networks in which there is a core of generalists and specialists often interact with generalists [3, 5, 7, 12, 14]. We demonstrated an association between interaction intimacy and structure in 19 ant-plant mutualistic networks. Through numerical simulations, we found that network structure of different forms of mutualism affects evolutionary change in distinct ways. Change in one species affects primarily one mutualistic partner in symbiotic interactions but might affect multiple partners in nonsymbiotic interactions. We hypothesize that coevolution in symbiotic interactions is characterized by frequent reciprocal changes between few partners, but coevolution in nonsymbiotic networks might show rare bursts of changes in which many species respond to evolutionary changes in a single species. {\textcopyright} 2007 Elsevier Ltd. All rights reserved.},
author = {Guimar{\~{a}}es, Paulo R. and Rico-Gray, Victor and Oliveira, Paulo S S. and Izzo, Thiago J. and dos Reis, S{\'{e}}rgio F. and Thompson, John N.},
doi = {10.1016/j.cub.2007.09.059},
issn = {09609822},
journal = {Current Biology},
keywords = {EVO{\_}ECOL},
number = {20},
pages = {1797--1803},
title = {{Interaction Intimacy Affects Structure and Coevolutionary Dynamics in Mutualistic Networks}},
volume = {17},
year = {2007}
}
@article{May1973,
author = {May, Robert M},
journal = {Ecology},
number = {3},
pages = {638--641},
title = {{Qualitative stability in model ecosystems}},
volume = {54},
year = {1973}
}
@article{Ryser2019,
abstract = {Habitat fragmentation is threatening global biodiversity. To date, there is only limited understanding of how habitat fragmentation or any alteration to the spatial structure of a landscape in general, affects species diversity within complex ecological networks such as food webs. Here, we present a dynamic and spatially-explicit food web model which integrates complex food web dynamics at the local scale and species-specific dispersal dynamics at the landscape scale, allowing us to study the interplay of local and spatial processes in metacommunities. We explore how habitat fragmentation, defined as a decrease of habitat availability and an increase of habitat isolation, affects the species diversity patterns of complex food webs ($\alpha$-, $\beta$-, $\gamma$-diversity), and specifically test whether there is a trophic dependency in the effect of habitat fragmentation on species diversity. In our model, habitat isolation is the main driver causing species loss and diversity decline. Our results emphasise that large-bodied consumer species at high trophic positions go extinct faster than smaller species at lower trophic levels, despite being superior dispersers that connect fragmented landscapes better. We attribute the loss of top species to a combined effect of higher biomass loss during dispersal with increasing habitat isolation in general, and the associated energy limitation in highly fragmented landscapes, preventing higher trophic levels to persist. To maintain trophic-complex and species-rich communities calls for effective conservation planning which considers the interdependence of trophic and spatial dynamics as well as the spatial context of a landscape and its energy availability.},
author = {Ryser, Remo and H{\"{a}}ussler, Johanna and Stark, Markus and Brose, Ulrich and Rall, Bj{\"{o}}rn C. and Guill, Christian},
doi = {10.1098/rspb.2019.1177},
file = {:Users/alyssacirtwill/Documents/Papers/Ryser et al.{\_}2019{\_}Proceedings of the Royal Society B Biological Sciences.pdf:pdf},
issn = {0962-8452},
journal = {Proceedings of the Royal Society B: Biological Sciences},
keywords = {computational,ecology,theoretical biology},
number = {1908},
pages = {20191177},
title = {{The biggest losers: habitat isolation deconstructs complex food webs from top to bottom}},
volume = {286},
year = {2019}
}
@article{Galiana2018,
abstract = {Species-area relationships (SARs) are pivotal to understand the distribution of biodiversity across spatial scales. We know little, however, about how the network of biotic interactions in which biodiversity is embedded changes with spatial extent. Here we develop a new theoretical framework that enables us to explore how different assembly mechanisms and theoretical models affect multiple properties of ecological networks across space. We present a number of testable predictions on network-area relationships (NARs) for multi-trophic communities. Network structure changes as area increases because of the existence of different SARs across trophic levels, the preferential selection of generalist species at small spatial extents and the effect of dispersal limitation promoting beta-diversity. Developing an understanding of NARs will complement the growing body of knowledge on SARs with potential applications in conservation ecology. Specifically, combined with further empirical evidence, NARs can generate predictions of potential effects on ecological communities of habitat loss and fragmentation in a changing world.},
author = {Galiana, Nuria and Lurgi, Miguel and Claramunt-L{\'{o}}pez, Bernat and Fortin, Marie Jos{\'{e}}e and Leroux, Shawn and Cazelles, Kevin and Gravel, Dominique and Montoya, Jos{\'{e}} M.},
doi = {10.1038/s41559-018-0517-3},
file = {:Users/alyssacirtwill/Documents/Papers/Galiana et al.{\_}2018{\_}Nature Ecology and Evolution.pdf:pdf},
issn = {2397334X},
journal = {Nature Ecology and Evolution},
number = {5},
pages = {782--790},
title = {{The spatial scaling of species interaction networks: SI}},
volume = {2},
year = {2018}
}
@article{Cohen1985,
abstract = {Gardner and Ashby1 have suggested that large complex systems which are assembled (connected) at random may be expected to be stable up to a certain critical level of connectance, and then, as this increases, to suddenly become unstable. Their conclusions were based on the trend of computer studies of systems with 4, 7 and 10 variables. {\textcopyright} 1972 Nature Publishing Group.},
author = {Cohen, Joel E. and Newman, Charles M.},
doi = {10.1038/238413a0},
issn = {00280836},
journal = {Journal of Theoretical Biology},
pages = {153--156},
pmid = {4559589},
title = {{When will a large complex system be stable?}},
volume = {113},
year = {1985}
}
@article{Portalier2019,
abstract = {Robust predictions of predator–prey interactions are fundamental for the understanding of food webs, their structure, dynamics, resistance to species loss, response to invasions and ecosystem function. Most current food web models measure parameters at the food web level to predict patterns at the same level. Thus, they are sensitive to the quality of the data and may be ineffective in predicting non-observed interactions and disturbed food webs. There is a need for mechanistic models that predict the occurrence of a predator–prey interaction based on lower levels of organization (i.e. the traits of organisms) and the properties of their environment. Here, we present such a model that focuses on the predation act itself. We built a Newtonian, mechanical model for the processes of searching, capturing and handling of a prey item by a predator. Associated with general metabolic laws, we predict the net energy gain from predation for pairs of pelagic or flying predator species and their prey depending on their body sizes. Predicted interactions match well with data from the most extensive predator–prey database, and overall model accuracy is greater than the allometric niche model. Our model shows that it is possible to accurately predict the structure of food webs using only a few mechanical traits. It underlines the importance of physical constraints in structuring food webs. A plain language summary is available for this article.},
author = {Portalier, S{\'{e}}bastien M.J. and Fussmann, Gregor F. and Loreau, Michel and Cherif, Mehdi},
doi = {10.1111/1365-2435.13254},
issn = {13652435},
journal = {Functional Ecology},
keywords = {body size,energy,mechanics,predation,trophic link},
number = {2},
pages = {323--334},
title = {{The mechanics of predator–prey interactions: First principles of physics predict predator–prey size ratios}},
volume = {33},
year = {2019}
}
@article{Mougi2012,
abstract = {Ecological theory predicts that a complex community formed by a number of species is inherently unstable, guiding ecologists to identify what maintains species diversity in nature. Earlier studies often assumed a community with only one interaction type, either an antagonistic, competitive, or mutualistic interaction, leaving open the question of what the diversity of interaction types contributes to the community maintenance. We show theoretically that the multiple interaction types might hold the key to understanding community dynamics. A moderate mixture of antagonistic and mutualistic interactions can stabilize population dynamics. Furthermore, increasing complexity leads to increased stability in a "hybrid" community. We hypothesize that the diversity of species and interaction types may be the essential element of biodiversity that maintains ecological communities.},
author = {Mougi, A. and Kondoh, M.},
doi = {10.1126/science.1220529},
issn = {10959203},
journal = {Science},
number = {6092},
pages = {349--351},
title = {{Diversity of interaction types and ecological community stability}},
volume = {337},
year = {2012}
}
@article{Dunne2002c,
abstract = {Food-web structure mediates dramatic effects of biodiversity loss including secondary and ‘cascading' extinctions. We studied these effects by simulating primary species loss in 16 food webs from terrestrial and aquatic ecosystems and measuring robustness in terms of the secondary extinctions that followed. As observed in other networks, food webs are more robust to random removal of species than to selective removal of species with the most trophic links to other species. More surprisingly, robustness increases with food-web connectance but appears independent of species richness and omnivory. In particular, food webs experience ‘rivet-like' thresholds past which they display extreme sensitivity to removal of highly connected species. Higher connectance delays the onset of this threshold. Removing species with few trophic connections generally has little effect though there are several striking exceptions. These findings emphasize how the number of species removed affects ecosystems differently depending on the trophic functions},
author = {Dunne, Jennifer A. and Williams, Richard J. and Martinez, Neo D.},
doi = {10.1515/HC.1998.4.1.21},
issn = {07930283},
journal = {Ecology Letters},
keywords = {2002,5,558,567,biodiversity,connectance,ecology letters,ecosystem function,food web,network structure,robustness,secondary extinctions,species loss,species richness,topology},
pages = {558--567},
title = {{Network structure and biodiversity loss in food webs: robustness increases with connectance}},
volume = {5},
year = {2002}
}
@article{Allesina2011a,
abstract = {Few food web theory hypotheses/predictions can be readily tested using likelihoods of reproducing the data. Simple probabilistic models for food web structure, however, are an exception as their likelihoods were recently derived. Here I test the performance of a more complex model for food web structure that is grounded in the allometric scaling of interactions with body size and the theory of optimal foraging (Allometric Diet Breadth Model-ADBM). This deterministic model has been evaluated by measuring the fraction of trophic relations it correctly predicts. I contrasted this value with that produced by simpler models based on body sizes and found that the quantitative information on allometric scaling and optimal foraging does not significantly increase model fit. Also, I present a method to compute the p-value for the fraction of trophic interactions correctly predicted by the ADBM, or any other model, with respect to three probabilistic models. I find that the ADBM predicts significantly more links than random graphs, but other models can outperform it. Although optimal foraging and allometric scaling may improve our understanding of food webs, the ADBM needs to be modified or replaced to find support in the data. {\textcopyright} 2010 Elsevier Ltd.},
archivePrefix = {arXiv},
arxivId = {0911.2021},
author = {Allesina, Stefano},
doi = {10.1016/j.jtbi.2010.06.040},
eprint = {0911.2021},
issn = {00225193},
journal = {Journal of Theoretical Biology},
keywords = {Allometric relation,Food webs,Likelihoods,Model selection,Optimal foraging},
number = {1},
pages = {161--168},
publisher = {Elsevier},
title = {{Predicting trophic relations in ecological networks: A test of the Allometric Diet Breadth Model}},
url = {http://dx.doi.org/10.1016/j.jtbi.2010.06.040},
volume = {279},
year = {2011}
}
@article{Jinks2019,
author = {Jinks, Kristin I and Brown, Christopher J and Rasheed, Michael A and Scott, Abigail L and Sheaves, Marcus},
journal = {Ecosphere},
keywords = {2019 the authors,abundance,accepted 18 september 2019,access article under the,biomass size spectra,c,commons attribution,copyright,corresponding editor,debra p,great barrier reef,habitat complexity,peters,predator,prey interactions,received 1 september 2019,seagrass,size spectra,stable isotope analysis,structural complexity,terms of the creative,this is an open},
number = {November},
pages = {e02928},
title = {{Habitat complexity influences the structure of food webs in Great Barrier Reef seagrass meadows}},
volume = {10},
year = {2019}
}
@article{Staudacher2016,
abstract = {Successful biological control of agricultural pests is dependent on a thorough understanding of the underlying trophic interactions between predators and prey. Studying trophic interactions can be challenging, particularly when generalist predators that frequently use multiple prey and interact with both pest and alternative prey are considered. In this context, diagnostic PCR proved to be a suitable approach, however at present, prey-specific PCR primers necessary for assessing such interactions across trophic levels are missing. Here we present a new set of 45 primers designed to target a wide range of invertebrate taxa common to temperate cereal crops: cereal aphids, their natural enemies such as carabid beetles, ladybeetles, lacewings, and spiders, and potential alternative prey groups (earthworms, springtails, and dipterans). These primers were combined in three ‘ready to use' multiplex PCR assays for quick and cost-effective analyses of large numbers of predator samples. The assays were tested on 560 carabids collected in barley fields in Sweden. Results from this screening suggest that aphids constitute a major food source for carabids in cereal crops (overall DNA detection rate: 51 {\%}), whereas alternative extraguild and intraguild prey appear to be less frequently preyed upon when aphids are present (11 {\%} for springtails and 12 {\%} for earthworms; 1 {\%} for spiders and 4 {\%} for carabids). In summary, the newly developed molecular assays proved reliable and effective in assessing previously cryptic predator–prey trophic interactions, specifically with focus on biological control of aphids. The diagnostic PCR assays will be applicable manifold as the targeted invertebrates are common to many agricultural systems of the temperate region.},
author = {Staudacher, Karin and Jonsson, Mattias and Traugott, Michael},
doi = {10.1007/s10340-015-0685-8},
file = {:Users/alyssacirtwill/Documents/Papers/Staudacher, Jonsson, Traugott{\_}2016{\_}Journal of Pest Science.pdf:pdf},
issn = {16124758},
journal = {Journal of Pest Science},
keywords = {Carabid beetles,Generalist predators,Group-specific primer,Molecular gut content analysis,Multiplex PCR},
pages = {281--293},
publisher = {Springer Berlin Heidelberg},
title = {{Diagnostic PCR assays to unravel food web interactions in cereal crops with focus on biological control of aphids}},
volume = {89},
year = {2016}
}
@article{Fawcett2012,
abstract = {Most research in biology is empirical, yet empirical studies rely fundamentally on theoretical work for generating testable predictions and interpreting observations. Despite this interdependence, many empirical studies build largely on other empirical studies with little direct reference to relevant theory, suggesting a failure of communication that may hinder scientific progress. To investigate the extent of this problem, we analyzed how the use of mathematical equations affects the scientific impact of studies in ecology and evolution. The density of equations in an article has a significant negative impact on citation rates, with papers receiving 28{\%} fewer citations overall for each additional equation per page in the main text. Long, equation-dense papers tend to be more frequently cited by other theoretical papers, but this increase is outweighed by a sharp drop in citations from nontheoretical papers (35{\%} fewer citations for each additional equation per page in the main text). In contrast, equations presented in an accompanying appendix do not lessen a paper's impact. Our analysis suggests possible strategies for enhancing the presentation of mathematical models to facilitate progress in disciplines that rely on the tight integration of theoretical and empirical work.},
author = {Fawcett, Tim W. and Higginson, Andrew D.},
doi = {10.1073/pnas.1205259109},
file = {:Users/alyssacirtwill/Documents/Papers/Fawcett, Higginson{\_}2012{\_}Proceedings of the National Academy of Sciences of the United States of America.pdf:pdf},
issn = {00278424},
journal = {Proceedings of the National Academy of Sciences of the United States of America},
keywords = {Impact factor,Mathematical formula,Mathematical literacy,Theoretical biology},
number = {29},
pages = {11735--11739},
title = {{Heavy use of equations impedes communication among biologists}},
volume = {109},
year = {2012}
}
@article{Cirtwill2019MEE,
author = {Cirtwill, Alyssa R. and Ekl{\"{o}}f, Anna and Roslin, Tomas and Wootton, Kate and Gravel, Dominique},
doi = {10.1111/2041-210X.13180},
journal = {Methods in Ecology and Evolution},
keywords = {department of ecology,of agricultural sciences,sweden,swedish university,uppsala},
number = {6},
pages = {902--911},
title = {{A quantitative framework for investigating the reliability of empirical network construction}},
volume = {10},
year = {2019}
}
@article{Roubinet2017,
abstract = {Introduction: D-dimer assay, generally evaluated according to cutoff points calibrated for VTE exclusion, is used to estimate the individual risk of recurrence after a first idiopathic event of venous thromboembolism (VTE). Methods: Commercial D-dimer assays, evaluated according to predetermined cutoff levels for each assay, specific for age (lower in subjects {\textless}70 years) and gender (lower in males), were used in the recent DULCIS study. The present analysis compared the results obtained in the DULCIS with those that might have been had using the following different cutoff criteria: traditional cutoff for VTE exclusion, higher levels in subjects aged ≥60 years, or age multiplied by 10. Results: In young subjects, the DULCIS low cutoff levels resulted in half the recurrent events that would have occurred using the other criteria. In elderly patients, the DULCIS results were similar to those calculated for the two age-adjusted criteria. The adoption of traditional VTE exclusion criteria would have led to positive results in the large majority of elderly subjects, without a significant reduction in the rate of recurrent event. Conclusion: The results confirm the usefulness of the cutoff levels used in DULCIS.},
author = {Roubinet, Eve and Birkhofer, Klaus and Malsher, Gerard and Staudacher, Karin and Ekboom, Barbara and Traugott, Michael and Jonsson, Mattias},
doi = {https://doi.org/10.1002/eap.1510},
issn = {1751553X},
journal = {Ecological Applications},
keywords = {Cutoff criteria,D-dimer,Recurrence,Venous thromboembolism},
number = {4},
pages = {1167--1177},
title = {{Diet of generalist predators reflects effects of cropping period and farming system on extra- and intraguild prey}},
volume = {27},
year = {2017}
}
@article{Pompanon2012,
abstract = {The analysis of food webs and their dynamics facilitates understanding of the mechanistic processes behind community ecology and ecosystem functions. Having accurate techniques for determining dietary ranges and components is critical for this endeavour. While visual analyses and early molecular approaches are highly labour intensive and often lack resolution, recent DNA-based approaches potentially provide more accurate methods for dietary studies. A suite of approaches have been used based on the identification of consumed species by characterization of DNA present in gut or faecal samples. In one approach, a standardized DNA region (DNA barcode) is PCR amplified, amplicons are sequenced and then compared to a reference database for identification. Initially, this involved sequencing clones from PCR products, and studies were limited in scale because of the costs and effort required. The recent development of next generation sequencing (NGS) has made this approach much more powerful, by allowing the direct characterization of dozens of samples with several thousand sequences per PCR product, and has the potential to reveal many consumed species simultaneously (DNA metabarcoding). Continual improvement of NGS technologies, on-going decreases in costs and current massive expansion of reference databases make this approach promising. Here we review the power and pitfalls of NGS diet methods. We present the critical factors to take into account when choosing or designing a suitable barcode. Then, we consider both technical and analytical aspects of NGS diet studies. Finally, we discuss the validation of data accuracy including the viability of producing quantitative data.},
author = {Pompanon, Francois and Deagle, Bruce E. and Symondson, William O.C. and Brown, David S. and Jarman, Simon N. and Taberlet, Pierre},
doi = {10.1111/j.1365-294X.2011.05403.x},
issn = {09621083},
journal = {Molecular Ecology},
keywords = {DNA barcoding,DNA metabarcoding,faeces,food webs,herbivory,predation},
number = {8},
pages = {1931--1950},
title = {{Who is eating what: diet assessment using next generation sequencing}},
volume = {21},
year = {2012}
}
@article{Waldner2012,
abstract = {DNA-based gut content analysis has become an important tool for unravelling feeding interactions in invertebrate communities under natural conditions. It usually implies killing of the consumer and extracting the DNA from its food, using either the whole animal or its dissected gut. This post-mortem approach, however, is not suitable for investigating the diet of rare or protected species and also prohibits tracking individual dietary preferences as each consumer can provide trophic information only once. Moreover, removing large numbers of consumers from a habitat for analysis might critically change population densities and affect species interactions. Here, we present DNA-based analysis of invertebrate regurgitates, a novel approach to overcome these limitations. Conducting feeding experiments where adult Poecilus cupreus (Coleoptera: Carabidae) were fed with larvae of Amphimallon solstitiale (Coleoptera: Scarabaeidae), we show that detection success in regurgitates compared to samples prepared from whole beetles was similar or significantly enhanced for small/medium and large prey DNA fragments, respectively. Prey DNA detection success remained high in regurgitates stored in ethanol for 21 months at room temperature prior to DNA extraction. We conclude that in those invertebrates where regurgitates can be obtained, examination of food DNA in regurgitates offers many advantages over conventional post-mortem gut content analysis.},
author = {Waldner, Thomas and Traugott, Michael},
doi = {10.1111/j.1755-0998.2012.03135.x},
file = {:Users/alyssacirtwill/Documents/Papers/Waldner, Traugott{\_}2012{\_}Molecular Ecology Resources.pdf:pdf},
issn = {1755098X},
journal = {Molecular Ecology Resources},
keywords = {Amphimallon solstitiale,Carabidae,Diagnostic PCR,Food web,Molecular gut content analysis,Poecilus cupreus,Prey detection,Scarabaeidae,Trophic interactions},
number = {4},
pages = {669--675},
title = {{DNA-based analysis of regurgitates: a noninvasive approach to examine the diet of invertebrate consumers}},
volume = {12},
year = {2012}
}
@article{ArrizabalagaEscudero2018,
author = {Arrizabalaga-Escudero, Aitor and Clare, Elizabeth L. and Salsamendi, Egoitz and Alberdi, Antton and Garin, Inazio and Aihartza, Joxerra and Goiti, Urtzi},
doi = {https://doi-org.ezp.sub.su.se/10.1111/mec.14508},
file = {:Users/alyssacirtwill/Documents/Papers/Arrizabalaga-Escudero et al.{\_}2018{\_}Molecular Ecology.pdf:pdf},
journal = {Molecular Ecology},
number = {5},
pages = {1273--1283},
title = {{Assessing niche partitioning of co-occurring sibling bat species by DNA metabarcoding}},
volume = {27},
year = {2018}
}
@article{Dalen2004,
author = {Dal{\'{e}}n, Love and G{\"{o}}therstr{\"{o}}m, Anders and Angerbj{\"{o}}rn, Anders},
doi = {10.1023/B:COGE.0000014060.54070.45},
journal = {Conservation Genetics},
keywords = {alopex lagopus,as a source of,confirm,dna,endangered populations are complicated,faeces are frequently used,gulo gulo,however,information in conservation to,pcr,primers,to low densities,to study due,vulpes vulpes},
number = {1},
pages = {109--111},
title = {{Identifying species from pieces of feces}},
volume = {5},
year = {2004}
}
@article{Wirta2015b,
abstract = {In the High Arctic, the species richness of spiders is typically low, but abundances can be very high. Thus, how the few spider species occurring in the region choose their prey, and what prey taxa they focus on, may significantly affect the community structure of arctic arthropods. Here we estimate the ecological imprint of adult spiders of three large-bodied species coexisting in Northeast Greenland: the morphologically similar crab spiders Xysticus deichmanni and X. labradorensis (Thomisidae) and the wolf spider Pardosa glacialis (Lycosidae). To describe an important part of these spiders' diet in detail, we amplified and sequenced DNA from prey remains in their guts, selectively focusing on two of the most abundant prey orders in the area (Diptera and Lepidoptera). By comparing the resultant sequences to a reference library including most taxa encountered in the region, we assigned the prey to species. Among the spider taxa occurring in the region, the wolf spider Pardosa glacialis is dominant in terms of both biomass and density. All three spider species proved to be wide generalists, with no detectable differences in prey choice among either the two crab spiders, or among these crab spiders and the wolf spider. This lack of dietary differentiation among species may be caused by the limited prey availability in the Arctic, forcing the predators to both generalism and opportunism. Given the substantial abundance of spiders and the lack of other predatory arthropods in the region, the opportunistic prey choice observed implies that these High-Arctic spider species have the potential for inflicting a strong influence on their prey community.},
author = {Wirta, Helena K. and Weingartner, Elisabeth and Hamb{\"{a}}ck, Peter A. and Roslin, Tomas},
doi = {10.1016/j.baae.2014.11.003},
issn = {16180089},
journal = {Basic and Applied Ecology},
keywords = {CO1,DNA barcode,Diet width,Molecular diet analysis,Niche overlap},
number = {1},
pages = {86--92},
publisher = {Elsevier GmbH},
title = {{Extensive niche overlap among the dominant arthropod predators of the High Arctic}},
url = {http://dx.doi.org/10.1016/j.baae.2014.11.003},
volume = {16},
year = {2015}
}
@article{Wernberg2015,
abstract = {Ecosystem reconfigurations arising from climate-driven changes in species distributions are expected to have profound ecological, social, and economic implications. Here we reveal a rapid climate-driven regime shift of Australian temperate reef communities, which lost their defining kelp forests and became dominated by persistent seaweed turfs. After decades of ocean warming, extreme marine heat waves forced a 100-kilometer range contraction of extensive kelp forests and saw temperate species replaced by seaweeds, invertebrates, corals, and fishes characteristic of subtropical and tropical waters.This community-wide tropicalization fundamentally altered key ecological processes, suppressing the recovery of kelp forests.},
author = {Wernberg, Thomas and Bennett, Scott and Babcock, Russell C and Bettignies, Thibaut De and Cure, Katherine and Depczynski, Martial and Dufois, Francois and Fromont, Jane and Fulton, Christopher J and Hovey, Renae K and Harvey, Euan S and Holmes, Thomas H and Kendrick, Gary a and Radford, Ben and Santana-garcon, Julia and Saunders, Benjamin J and Smale, Dan a and Thomsen, Mads S and Tuckett, Chenae a and Tuya, Fernando},
doi = {10.1126/science.aad8745},
issn = {0036-8075},
journal = {Science},
number = {6295},
pages = {169--172},
title = {{Temperate Marine Ecosystem}},
volume = {353},
year = {2015}
}
@article{BoLD,
author = {Ratnasingham, Sujeevan and Hebert, Paul D. N.},
doi = {10.1111/j.1471-8286.2006.01678.x},
journal = {Molecular Ecology Notes},
keywords = {2006,coi,dna barcoding,gene sequence,informatics,received 30 july 2006,revision accepted 17 november,species identification,taxonomy},
number = {3},
pages = {355--364},
title = {bold: the barcode of life data system},
volume = {7},
year = {2007}
}
@article{Novotny2005,
abstract = {Studies of host specificity in tropical insect herbivores are evolving from a focus on insect distribution data obtained by canopy fogging and other mass collecting methods, to a focus on obtaining data on insect rearing and experimentally verified feeding patterns. We review this transition and identify persisting methodological problems. Replicated quantitative surveys of plant-herbivore food webs, based on sampling efforts of an order of magnitude greater than is customary at present, may be cost-effectively achieved by small research teams supported by local assistants. Survey designs that separate historical and ecological determinants of host specificity by studying herbivores feeding on the same plant species exposed to different environmental or experimental conditions are rare. Further, we advocate the use of host-specificity measures based on plant phylogeny. Existing data suggest that a minority of species in herbivore communities feed on a single plant species when alternative congeneric hosts are available. Thus, host plant range limits tend to coincide with those of plant genera, rather than species or suprageneric taxa. Host specificity among tropical herbivore guilds decreases in the sequence: granivores 〉 leaf-miners 〉 fructivores 〉 leaf-chewers = sap-suckers 〉 xylophages 〉 root-feeders, thus paralleling patterns observed in temperate forests. Differences in host specificity between temperate and tropical forests are difficult to assess since data on tropical herbivores originate from recent field studies, whereas their temperate counterparts derive from regional host species lists, assembled over many years. No major increase in host specificity from temperate to tropical communities is evident. This conclusion, together with the recent downward revisions of extremely high estimates of tropical species richness, suggest that tropical ecosystems may not be as biodiverse as previously thought. {\textcopyright} 2005 The Royal Society.},
author = {Novotny, Vojtech and Basset, Yves},
doi = {10.1098/rspb.2004.3023},
file = {:Users/alyssacirtwill/Documents/Papers/Novotny, Basset{\_}2005{\_}Proceedings of the Royal Society B Biological Sciences.pdf:pdf},
issn = {14712970},
journal = {Proceedings of the Royal Society B: Biological Sciences},
keywords = {Food web,Herbivore guild,Host plant range,Insect sampling,Rainforest,Species richness},
number = {1568},
pages = {1083--1090},
title = {{Host specificity of insect herbivores in tropical forests}},
volume = {272},
year = {2005}
}
@article{Yguel2011a,
abstract = {Hosts belonging to the same species suffer dramatically different impacts from their natural enemies. This has been explained by host neighbourhood, that is, by surrounding host-species diversity or spatial separation between hosts. However, even spatially neighbouring hosts may be separated by many million years of evolutionary history, potentially reducing the establishment of natural enemies and their impact. We tested whether phylogenetic isolation of oak hosts from neighbouring trees within a forest canopy reduces phytophagy. We found that an increase in phylogenetic isolation by 100 million years corresponded to a 10-fold decline in phytophagy. This was not due to poorer living conditions for phytophages on phylogenetically isolated oaks. Neither species diversity of neighbouring trees nor spatial distance to the closest oak affected phytophagy. We suggest that reduced pressure by natural enemies is a major advantage for individuals within a host species that leave their ancestral niche and grow among distantly related species.},
author = {Yguel, Benjamin and Bailey, Richard and Tosh, N. Denise and Vialatte, Aude and Vasseur, Chlo{\'{e}} and Vitrac, Xavier and Jean, Frederic and Prinzing, Andreas},
doi = {10.1111/j.1461-0248.2011.01680.x},
issn = {1461023X},
journal = {Ecology Letters},
keywords = {Community phylogeny,Forest canopy,Insect herbivory,Intraspecific variation,Lepidoptera,Macroevolution,Plant-insect interactions,Quercus,Temperate forest},
number = {11},
pages = {1117--1124},
title = {{Phytophagy on phylogenetically isolated trees: Why hosts should escape their relatives}},
volume = {14},
year = {2011}
}
@article{Schoener2001,
abstract = {There has been considerable research on both top-down effects1,2 and on disturbances3,4,5 in ecological communities; however, the interaction between the two, when the disturbance is catastrophic, has rarely been examined6. Predators may increase the probability of prey extinction resulting from a catastrophic disturbance both by reducing prey population size7,8 and by changing ecological traits of prey individuals such as habitat characteristics8,9 in a way that increases the vulnerability of prey species to extinction. We show that a major hurricane in the Bahamas led to the extinction of lizard populations on most islands onto which a predator had been experimentally introduced, whereas no populations became extinct on control islands. Before the hurricane, the predator had reduced prey populations to about half of those on control islands. Two months after the hurricane, we found only recently hatched individuals—apparently lizards survived the inundating storm surge only as eggs. On predator-introduction islands, those hatchling populations were a smaller fraction of pre-hurricane populations than on control islands. Egg survival allowed rapid recovery of prey populations to pre-hurricane levels on all control islands but on only a third of predator-introduction islands—the other two-thirds lost their prey populations. Thus climatic disturbance compounded by predation brought prey populations to extinction.},
author = {Schoener, Thomas W and Spiller, David A and Losos, Jonathan B},
doi = {10.1038/35084071},
file = {:Users/alyssacirtwill/Documents/Papers/Schoener T. W., Spiller D. A., Losos J. B.{\_}2000{\_}Nature.pdf:pdf},
issn = {1476-4687},
journal = {Nature},
number = {6843},
pages = {183--186},
title = {{Predators increase the risk of catastrophic extinction of prey populations}},
url = {https://doi.org/10.1038/35084071},
volume = {412},
year = {2001}
}
@article{SchoenerT.W.2000,
abstract = {There has been considerable research on both top-down effects1,2 and on disturbances3±5 in ecological communities; however, the interaction between the two, when the disturbance is catastrophic, has rarely been examined6. Predators may increase the probability of prey extinction resulting from a catastrophic disturbance both by reducing prey population size7,8 and by changing ecological traits of prey individuals such as habitat characteristics8,9 in a way that increases the vulnerability of prey species to extinction. We show that a major hurricane in the Bahamas led to the extinction of lizard populations on most islands onto which a predator had been experimentally introduced, whereas no populations became extinct on control islands. Before the hurricane, the predator had reduced prey populations to about half of those on control islands. Two months after the hurricane, we found only recently hatched individuals–apparently lizards survived the inundating stormsurge only as eggs. On predator-introduction islands, those hatchling populations were a smaller fraction of pre-hurricane populations than on control islands. Egg survival allowed rapid recovery of prey populations to pre-hurricane levels on all control islands but on only a third of predator-introduction islands–the other two-thirds lost their prey populations. Thus climatic disturbance compounded by predation brought prey populations to extinction.},
author = {{Schoener T. W.} and {Spiller D. A.} and {Losos J. B.}},
journal = {Nature},
number = {6843},
pages = {183--186},
title = {{Predators increase the risk of catastrophic extinction of prey populations}},
volume = {412},
year = {2000}
}
@article{Thebault2010,
abstract = {Research on the relationship between the architecture of ecological networks and community stability has mainly focused on one type of interaction at a time, making difficult any comparison between different network types. We used a theoretical approach to show that the network architecture favoring stability fundamentally differs between trophic and mutualistic networks. A highly connected and nested architecture promotes community stability in mutualistic networks, whereas the stability of trophic networks is enhanced in compartmented and weakly connected architectures. These theoretical predictions are supported by a meta-analysis on the architecture of a large series of real pollination (mutualistic) and herbivory (trophic) networks. We conclude that strong variations in the stability of architectural patterns constrain ecological networks toward different architectures, depending on the type of interaction.},
author = {Th{\'{e}}bault, Elisa and Fontaine, Colin},
doi = {10.1126/science.1188321},
issn = {00368075},
journal = {Science},
number = {5993},
pages = {853--856},
title = {{Stability of ecological communities and the architecture of mutualistic and trophic networks}},
volume = {329},
year = {2010}
}
@article{Wilson2007,
abstract = {In the clade of Penstemon and segregate genera, pollination syndromes are well defined among the 284 species. Most display combinations of floral characters associated with pollination by Hymenoptera, the ancestral mode of pollination for this clade. Forty-one species present characters associated with hummingbird pollination, although some of these ornithophiles are also visited by insects. The ornithophiles are scattered throughout the traditional taxonomy and across phylogenies estimated from nuclear (internal transcribed spacer (ITS)) and chloroplast DNA (trnCD/TL) sequence data. Here, the number of separate origins of ornithophily is estimated, using bootstrap phylogenies and constrained parsimony searches. Analyses suggest 21 separate origins, with overwhelming support for 10 of these. Because species sampling was incomplete, this is probably an underestimate. Penstemons therefore show great evolutionary lability with respect to acquiring hummingbird pollination; this syndrome acts as an attractor to which species with large sympetalous nectar-rich flowers have frequently been drawn. By contrast, penstemons have not undergone evolutionary shifts backwards or to other pollination syndromes. Thus, they are an example of both striking evolutionary lability and constrained evolution.},
author = {Wilson, Paul and Wolfe, Andrea D. and Armbruster, W. Scott and Thomson, James D.},
doi = {10.1111/j.1469-8137.2007.02219.x},
issn = {0028646X},
journal = {New Phytologist},
keywords = {Conservatism,Constraint,Homoplasy,Lability,Parallelism,Penstemon,Pollination,Speciational drive},
number = {4},
pages = {883--890},
title = {{Constrained lability in floral evolution: Counting convergent origins of hummingbird pollination in Penstemon and Keckiella}},
volume = {176},
year = {2007}
}
@article{Ives2002,
abstract = {We derive a model that has one predator and one prey species and includes antipredator behavior. The prey are allowed to increase their investment in antipredator behavior, thereby decreasing their chance of being captured by the predator. However, they must pay for this protection by a cost exacted through decreased fecundity or increased mortality caused by factors other than predation. The population-dynamic consequences of antipredator behaviors are explored by comparing systems in which the efficiencies of the antipredator behavior differ; as the antipredator behavior becomes more efficient, the prey need to invest less in order to achieve the same level of protection from the predator. We assume that for any degree of efficiency, the prey choose their level of investment in antipredator behavior in order to optimize their expected reproductive fitness. By assuming only that the predators and prey coexist and that there is a stable equilibrium, we show that increased efficiency of antipredator behaviors increases prey densities and decreases the ratio of predator-to-prey densities. This is true even though the prey's level of investment in antipredator behavior initially increases and then decreases with increasing efficiency of the antipredator behavior. Consequently, the effect of antipredator behaviors on population densities cannot be inferred from the level of prey investment in these behaviors. A specific model is used to illustrate the second result of the paper, that antipredator behaviors tend to decrease the oscillatory dynamics inherent in model predatorprey systems.},
author = {Ives, Anthony R. and Dobson, Andrew P.},
doi = {10.1086/284719},
issn = {0003-0147},
journal = {The American Naturalist},
number = {3},
pages = {431--447},
title = {{Antipredator behavior and the population dynamics of simple predator-prey systems}},
volume = {130},
year = {2002}
}
@article{GonzilezTato2013,
archivePrefix = {arXiv},
arxivId = {arXiv:1011.1669v3},
author = {Ives, Anthony R. and Godfray, H. C. J.},
doi = {10.1086/521238},
eprint = {arXiv:1011.1669v3},
isbn = {VO -},
issn = {1443458X},
journal = {The American Naturalist},
pages = {E1--E14},
pmid = {17891731},
title = {{Phylogenetic analysis of trophic associations}},
url = {http://www.journals.uchicago.edu/doi/10.1086/521238},
volume = {168},
year = {2006}
}
@article{Ludsin2001,
author = {Ludsin, Stuart A and Kershner, Mark W and Blocksom, Karen A and Knight, Roger L and Stein, Roy A.},
journal = {Ecological Applications},
keywords = {detrended correspondence analysis,eutrophication,gotrophication,great lakes,lake erie,oli-,phosphorus abatement,productivity,resilience,species diversity,species richness,species turnover,succession},
number = {3},
pages = {731--746},
title = {{Life after death in Lake Erie: nutrient controls drive fish species richness, rehabilitation}},
volume = {11},
year = {2001}
}
@article{Aberhan2003,
author = {Aberhan, Martin and Baumiller, Tomasz K.},
doi = {10.1130/G19938.1},
file = {:Users/alyssacirtwill/Documents/Papers/Aberhan, Baumiller{\_}2003{\_}Geology.pdf:pdf},
journal = {Geology},
keywords = {bivalvia,diversity,jurassic,mass extinction,paleoecology,selectivity},
number = {12},
pages = {1077--1080},
title = {{Selective extinction among Early Jurassic bivalves: a consequence of anoxia}},
volume = {31},
year = {2003}
}
@article{VanderZanden1999,
abstract = {Food web structure is paramount in regulating a variety of ecologic patterns and processes, although food web studies are limited by poor empirical descriptions of inherently complex systems. In this study, stable isotope ratios (delta15N and delta13C) were used to quantify trophic relationships and food chain length (measured as a continuous variable) in 14 Ontario and Quebec lakes. All lakes contained lake trout as the top predator, although lakes differed in the presumed number of trophic levels leading to this species. The presumed number of trophic levels was correlated with food chain length and explained 40{\%} of the among-lake variation. Food chain length was most closely related to fish species richness (r2=0.69) and lake area (r2=0.50). However, the two largest study lakes had shorter food chains than lakes of intermediate size and species richness, producing hump-shaped relationships with food chain length. Lake productivity was not a powerful predictor of food chain length (r2=0.36), and we argue that productive space (productivity multiplied by area) is a more accurate measure of available energy. This study addresses the need for improved food web descriptions that incorporate information about energy flow and the relative importance of trophic pathways.},
author = {{Vander Zanden}, M. Jake and Shuter, B. J. and Lester, N. and Rasmussen, J. B.},
doi = {10.2307/2463588},
issn = {00030147},
journal = {The American Naturalist},
keywords = {1960,been increasing recognition of,chains,food,food webs,hairston et al,productive space,productivity,since the publication of,the importance of food,there has,trophic position,trophic structure,web},
number = {4},
pages = {406--416},
title = {{Patterns of food chain length in lakes: a stable isotope study}},
volume = {154},
year = {1999}
}
@article{Vazquez2009,
abstract = {BACKGROUND: Ecologists and evolutionary biologists are becoming increasingly interested in networks as a framework to study plant-animal mutualisms within their ecological context. Although such focus on networks has brought about important insights into the structure of these interactions, relatively little is still known about the mechanisms behind these patterns. SCOPE: The aim in this paper is to offer an overview of the mechanisms influencing the structure of plant-animal mutualistic networks. A brief summary is presented of the salient network patterns, the potential mechanisms are discussed and the studies that have evaluated them are reviewed. This review shows that researchers of plant-animal mutualisms have made substantial progress in the understanding of the processes behind the patterns observed in mutualistic networks. At the same time, we are still far from a thorough, integrative mechanistic understanding. We close with specific suggestions for directions of future research, which include developing methods to evaluate the relative importance of mechanisms influencing network patterns and focusing research efforts on selected representative study systems throughout the world.},
author = {V{\'{a}}zquez, Diego P. and Bluthgen, Nico and Cagnolo, Luciano and Chacoff, Natacha P.},
doi = {10.1093/aob/mcp057},
isbn = {0305-7364},
issn = {10958290},
journal = {Annals of Botany},
keywords = {Antplant interactions,forbidden links,mutualism,neutrality,plantanimal interactions,pollination,seed dispersal,trait matching},
number = {9},
pages = {1445--1457},
pmid = {19304996},
title = {{Uniting pattern and process in plant-animal mutualistic networks: A review}},
volume = {103},
year = {2009}
}
@article{Cagnolo2011,
abstract = {1. Biological communities are organized in complex interaction networks such as food webs, which topology appears to be non-random. Gradients, compartments, nested subsets and even combinations of these structures have been shown in bipartite networks. However, in most studies only one pattern is tested against randomness and mechanistic hypotheses are generally lacking. 2. Here we examined the topology of regional, coexisting plant-herbivore and host-parasitoid food webs to discriminate between the mentioned network patterns. We also evaluated the role of species body size, local abundance, regional frequency and phylogeny as determinants of network topology. 3. We found both food webs to be compartmented, with interaction range boundaries imposed by host phylogeny. Species degree within compartments was mostly related to their regional frequency and local abundance. Only one compartment showed an internal nested structure in the distribution of interactions between species, but species position within this compartment was unrelated to species size or abundance. 4. These results suggest that compartmentalization may be more common than previously considered, and that network structure is a result of multiple, hierarchical, non-exclusive processes.},
author = {Cagnolo, Luciano and Salvo, Adriana and Valladares, Graciela},
doi = {10.1111/j.1365-2656.2010.01778.x},
isbn = {0021-8790},
issn = {00218790},
journal = {Journal of Animal Ecology},
keywords = {Abundance,Body size,Compartmentalization,Food webs,Leafminers,Nestedness,Network topology,Parasitoids,Phylogeny},
number = {2},
pages = {342--351},
pmid = {21143226},
title = {{Network topology: Patterns and mechanisms in plant-herbivore and host-parasitoid food webs}},
volume = {80},
year = {2011}
}
@book{Winfield1991,
address = {Bury St. Edmonds, Suffolk},
editor = {Winfield, Ian J. and Nelson, Joseph S.},
isbn = {9789401053693},
publisher = {Springer},
title = {{Cyprinid fishes: Systematics, biology and exploitation}},
year = {1991}
}
@article{May1980,
author = {May, Linda},
journal = {Hydrobiologia},
pages = {177--180},
title = {{On the ecology of Notholca squamula M$\backslash$"{\{}u{\}}ller in Loch Leven, Kinross, Scotland}},
volume = {73},
year = {1980}
}
@article{Jones1968,
author = {Jones, D. A.},
journal = {Journal of the Zoological Society of London},
pages = {363--376},
title = {{The functional morphology of the digestive system in the carnivorous intertidal isopod $\backslash$emph{\{}Eurydice{\}}}},
volume = {156},
year = {1968}
}
@article{Kruczynski2006,
abstract = {Bay scallops containing adult female pea crabs are slightly smaller than noninfected scallops. Infected scallops tend to weigh less than noninfected of the same size. The growth of 3 size groups of infected and noninfected scallops was measured over a 3 month period and infected scallops grew less than noninfected. -� 1972 Estuarine Research Federation},
author = {Kruczynski, William L.},
doi = {10.2307/1351068},
issn = {00093262},
journal = {Chesapeake Science},
number = {3},
pages = {218},
title = {{The Effect of the Pea Crab, Pinnotheres maculatus Say, on Growth of the Bay Scallop, Argopecten irradians concentricus (Say)}},
volume = {13},
year = {2006}
}
@article{Vasconcelos2004,
abstract = {Summary The feeding ecology of the lesser weever, Echiichthys vipera, from the adjacent coastal areas of the Douro and Tagus estuaries (Portugal) was studied between October 2000 and July 2002. The stomach contents of 246 individuals were analysed and diet was characterized by the numerical, gravimetric, occurrence and vacuity indices. Variation of feeding habits with fish length ({\textless}95 and {\textgreater}95{\^{A}} mm) and geographical area was considered. Diet of the lesser weever comprised a large variety of prey (28 species), the most important of which were crustaceans (numerical index, NI{\^{A}} ={\^{A}} 93.5{\%}; occurrence index, OI{\^{A}} ={\^{A}} 75.6{\%}), namely Mysidacea (especially Schistomysis sp.), Amphipoda (mainly Gammarus subtypicus) and Isopoda (Idotea spp.), and also Teleostei (mostly larval stages that posted a gravimetric index, GI{\^{A}} ={\^{A}} 53.0{\%}). Diet varied with fish length, with large individuals showing a larger diversity of prey items. Furthermore, specimens from Douro also showed a higher diversity of prey items than those from Tagus. More than 50{\%} of the stomachs were empty, being the highest vacuity values relative to smaller fishes as well as to individuals from the Tagus estuary adjacent coastal area},
author = {Vasconcelos, R. and Prista, N. and Cabral, H. and Costa, M. J.},
doi = {10.1111/j.1439-0426.2004.00547.x},
file = {:Users/alyssacirtwill/Documents/Papers/Vasconcelos et al.{\_}2004{\_}Journal of Applied Ichthyology.pdf:pdf},
issn = {01758659},
journal = {Journal of Applied Ichthyology},
number = {3},
pages = {211--216},
title = {{Feeding ecology of the lesser weever, Echiichthys vipera (Cuvier, 1829), on the western coast of Portugal}},
volume = {20},
year = {2004}
}
@article{Canete2008,
author = {Ca{\~{n}}ete, JI and C{\'{a}}rdenas, CA and Oyarz{\'{u}}n, Sylvia and Plana, Jordi and Palacios, Mauricio and Santana, Mario},
journal = {Revista de Biolog{\'{i}}a Marina y Oceanograf{\'{i}}a},
keywords = {bentos subant{\'{a}}rtico,bop{\'{i}}ridos,lit{\'{o}}didos,parasitismo,regi{\'{o}}n magall{\'{a}}nica},
number = {2},
pages = {265--274},
title = {{Tuberculata Richardson, 1904 (Isopoda: Bopyridae): a Parasite of Juveniles of the King Crab Lithodes Santolla (Molina, 1782)(}},
url = {http://redalyc.uaemex.mx/redalyc/html/479/47943204/47943204.html},
volume = {43},
year = {2008}
}
@article{Morte1999,
abstract = { The stomach contents of 344 four-spotted megrim, ( Lepidorhombus boscii ) and 159 megrim ( Lepidorhombus whiffiagonis ), off the eastern coast of the Gulf of Valencia (Spain), were analysed. The two species examined do not appear to have very similar diets, based on the species composition of prey. The vacuity coefficient is not high for any of the species, the main food being Crustacea (Decapoda and Mysidacea). Also Amphipoda and Teleostei are components of the diet. Variations in the food of both fish related to their length show few small crustaceans as prey of the major specimens. Finally, there was evidence for seasonal variation of the quality and quantity of the food consumed. There was no great dietary overlap between these two species. },
author = {Morte, Salom{\'{e}} and Red{\'{o}}n, Manuel J. and Sanz-Brau, Antonio},
doi = {10.1017/s0025315497000180},
issn = {0025-3154},
journal = {Journal of the Marine Biological Association of the United Kingdom},
number = {1},
pages = {161--169},
title = {{ Feeding ecology of two megrims Lepidorhombus boscii and Lepidorhombus whiffiagonis in the western Mediterranean (Gulf of Valencia, Spain) }},
volume = {79},
year = {1999}
}
@article{Kestemont1989,
abstract = {The suitability and nutritional value of the euryhaline rotifer Brachionus plicatilis, fed on baker's yeast, for the larval stage of a small European cyprinid, the gudgeon Gobio gobio L., were investigated. Several parameters such as level and frequency of feeding, maximal food intake and feed conversion rate (FCR) were determined during the first month of larval rearing. The survival rate of the larvae was very high, generally more than 90{\%}. The best growth was attained with the highest daily ration (from 2500 rotifers per larva during the first week of rearing to 5500 rotifers per larva during the fourth) and a 4 times-a-day feeding frequency, with a FCR of 0.86. From a mean initial body weight of 0.5 mg at hatching, the larvae reached 17.5 mg in 3 weeks. However, a reduction in the growth rate was observed after the second week, indicating a nutritional deficiency of the yeast-fed rotifers or a too small size of this live food. When the fishes are larger than 10-12 mm (after about 10-15 days of feeding), they would probably grow better on a larger prey such as Artemia salina or on an appropriate dry food. {\textcopyright} 1989.},
author = {Kestemont, Patrick and Awa{\"{i}}ss, Aboubacar},
doi = {10.1016/0044-8486(89)90042-2},
issn = {00448486},
journal = {Aquaculture},
number = {3-4},
pages = {305--318},
title = {{Larval rearing of the gudgeon, Gobio gobio L., under optimal conditions of feeding with the rotifer, Brachionus plicatilis O.F. M{\"{u}}ller}},
volume = {83},
year = {1989}
}
@article{Seed1969,
abstract = {Significant differences in the infection of M. edulis and the “Padstow type” mussel with P. pisum are recorded, and some possible explanations for these differences are discussed. Both types of Mytilus from the mid and lower regions of the mussel bed showed heavier infections than mussels higher on the shore. Even so, the differences between the two types were still maintained. A relationship exists between crab and mussel size, larger crabs being found only in larger hosts. The smallest mussel found to be infected with Pinnotheres measured 3{\textperiodcentered}35 cm in length. Infection in M. edulis was found to increase with increased size of host, the largest occurring mussels having from 80 to 100{\%} infection. Larger mussels occurred in greater numbers in the low shore. It is assumed that infection in the “Padstow type” would show a similar relationship if sufficient recordings had been available. The presence of the crab causes gill damage, and infected mussels show considerably lower tissue weights and slightly greater shell weights than uninfected mussels of similar size. The presence of the crab does not appear to influence the reproductive capacity of the mussel.},
author = {Seed, R.},
doi = {10.1111/j.1469-7998.1969.tb02158.x},
issn = {14697998},
journal = {Journal of Zoology},
number = {4},
pages = {413--420},
title = {{The incidence of the Pea crab, Pinnotheres pisum in the two types of Mytilus (Mollusca: Bivalvia) from Padstow, south‐west England}},
volume = {158},
year = {1969}
}
@article{Morris2014a,
author = {Morris, Rebecca J and Lewis, Owen T and Godfray, H Charles J and Annales, Source and Fennici, Zoologici and Ecology, Spatial and Herbivorous, O F and Morris, J and Lewis, Owen Т and Godfray, H Charles J},
number = {4},
pages = {449--462},
title = {{Finnish Zoological and Botanical Publishing Board Apparent competition and insect community structure : towards a spatial perspective Apparent competition and insect community structure : towards a spatial perspective}},
volume = {42},
year = {2014}
}
@article{Kirkegaard2006,
abstract = {A population of Theodoxus fluviatilis L in the littoral zone of Lake Esrom was investigated from November 1977 to February 1979. The population was sampled every month in the winter period and twice during the rest of the year. Biomass was estimated as ash-free dry weight (AFDW) of the organic matter both of the soft parts of the animal and the shell itself. The relation between AFDW (c) and shell length (l) was log c = 2.9509 × log (l)-1.7120. The population comprised more than 1 year-class, which could be separated by shell length, by a narrow band on the shells and the growth of algae on the shell. The life cycle lasted 21/2 - 3 years. The oldest animals had a shell length of 7.0-7.5 mm. A few individuals who were estimated to be 31/2 years had a shell length up to 8.6 mm. Population density varied between 575 and 2115 individuals m-2 on the stony substratum. The average was 1160 individuals m-2. Mortality was low during the summer period. In winter many animals died due to the effect of ice and stormy weather on the stony substratum. Growth of the animals was estimated from the shell length. Maximum growth was observed from May to August with no growth during the winter. Egg capsules were found on the stones all year round. New capsules were found from late May to the middle of November. Most freshly laid capsules were observed in May-June and August-September. Capsules from the late summer hatched in spring and capsules laid in the spring hatched in August-September. The average annual net production for the whole population was estimated by three methods. The Allen curve method gave 1.895 AFDW m-2, the growth-increment method gave 1.784 mg AFDW m-2 and the Hynes method 2.284 mg AFDW m-2. Corresponding estimated P/B ratios were 1.29, 1.30 and 1.57. Annual net-production of the four investigated year-classes was 16 mg AFDW m-2 year-1 for 1975, 224 mg AFDW m-2 year-1 for 1976, 1.258 mg AFDW m-2 year-1 for 1977 and 287 mg AFDW m-2 year-1 for 1978. P/B ratios for the three oldest year-classes were, respectively, 0.32, 0.50 and 1.67. A comparison with other investigations on gastropod life cycles, reproduction and P/B ratios is made and differences discussed. Variations are correlated to temperature, and food quality and quantity. {\textcopyright} 2005 Elsevier GmbH. All rights reserved.},
author = {Kirkegaard, J{\o}rn},
doi = {10.1016/j.limno.2005.11.002},
issn = {00759511},
journal = {Limnologica},
keywords = {Gastropoda,Lake Esrom,Life cycle,Neritidae,P/B ratio,Production,Theodoxus},
number = {1},
pages = {26--41},
title = {{Life history, growth and production of Theodoxus fluviatilis in Lake Esrom, Denmark}},
volume = {36},
year = {2006}
}
@article{Bonsall1999,
abstract = {Indirect effects such as apparent competition (in which two hosts that do not compete for resources interact via a shared natural enemy) are increasingly being shown to be prevalent in the structure and function of ecological assemblages. Here, we review the empirical and theoretical evidence for these enemy-mediated effects in host-parasitoid assemblages. We first address questions about the design of experiments to test for apparent competition. Second, we consider factors likely to affect the coexistence of host species that share a parasitoid and are involved in apparent competition. We show that parasitoid aggregation, and the switching effect that this can generate when hosts occur in separate patches, not only promotes persistence but is also strongly stabilizing. The broader consequences of these effects are discussed.},
author = {Bonsall, Michael B. and Hassell, Michael P.},
doi = {10.1007/PL00011983},
file = {:Users/alyssacirtwill/Documents/Papers/Bonsall, Hassell{\_}1999{\_}Researches on Population Ecology.pdf:pdf},
issn = {00345466},
journal = {Researches on Population Ecology},
keywords = {Aggregation,Coexistence,Heterogeneity of risk,Indirect interactions,Switching},
number = {1},
pages = {59--68},
title = {{Parasitoid-mediated effects: Apparent competition and the persistence of host-parasitoid assemblages}},
volume = {41},
year = {1999}
}
@article{Wikander1981,
abstract = {The investigation is based on material from populations in the depths of Kors- fjorden, western Norway. The faecal pellets are described and classified. The process of gut clearance of individuals placed in glass vessels without any sediment was much retarded compared to individuals dwelling in native sediment. Gut capacity increases with increased shell length in A. nitida and remains principally constant in A. longicallus. Gut capacity per unit weight of the animal gives curves of opposite slope for the two species. The relative amount of deposit ingested per 24 hours is principally constant in A. nitida, but decreases with increasing size in A. longicallus. Estimations concerning the degree of reworking of the deposit by a hypothetical population of A. nitida indicate that one individual takes the same deposit into the mantle cavity approximately 50 times during one year. It is assumed that only A. nitida, with regard to sediment reworking, can play an important role in the bio- tope, because of its occasional high abundance. A time of adaptation to the aquarium situation is necessary before any experimental results can be considered reliable},
author = {Wikander, Per Bie},
doi = {10.1080/00364827.1981.10414518},
issn = {00364827},
journal = {Sarsia},
number = {1},
pages = {35--48},
title = {{Quantitative aspects of deposit feeding in abra nitida (m{\"{u}}ller) and a. longicallus (scacchi) (bivalvia, tellinacea)}},
volume = {66},
year = {1981}
}
@article{Gittenberger2000,
author = {Gittenberger, Adriaan and Goud, Jeroen and Gittenberger, Edmund},
file = {:Users/alyssacirtwill/Documents/Papers/Gittenberger, Goud, Gittenberger{\_}2000{\_}Unknown.pdf:pdf},
keywords = {coral,coral reefs,egg-capsules,indo-paci fi c,mollusc associations,parasitic snails},
number = {1},
pages = {1--13},
title = {{(Gastropoda: Epitoniidae) associated with mushroom corals (Scleractinia: Fungiidae) from Sulawesi, Indonesia, with the description of four new species Adriaan Gittenberger, Jeroen Goud and Edmund Gittenberger}},
volume = {114},
year = {2000}
}
@article{Klug1980,
abstract = {Scanning electron microscopy, light microscopy, and direct isolations were used to examine the distribution and diversity of bacteria in the gut tracts of larval stages of Tipula abdominalis. The animal had an enlarged hindgut which housed a diverse bacterial community in the lumen and directly attached to the gut wall. Distinct localization was noted, with the most dense and most diverse community anterior to the rectum. A distinct architecture of bacteria occurred in this region, characterized by a layering or a "weblike" array of filamentous bacteria overlying mats of bacteria closely associated with the gut wall. Although morphological diversity was high in the hindgut, filamentous bacteria were the dominant morphology observed. The attached microbiota, sloughed during ecdysis, recolonized to the same density and diversity observed before the molt. The majority of the isolatable bacterial types were facultatively anaerobic. The distinct localization and attached nature of the hindgut bacteria and the recolonization after each molt suggest they are indigenous to this region of the gut tract.},
author = {Klug, M J and Kotarski, S},
issn = {0099-2240},
journal = {Applied and environmental microbiology},
number = {2},
pages = {408--16},
pmid = {16345618},
title = {{Bacteria Associated with the Gut Tract of Larval Stages of the Aquatic Cranefly Tipula abdominalis (Diptera; Tipulidae).}},
url = {http://www.ncbi.nlm.nih.gov/pubmed/16345618{\%}0Ahttp://www.pubmedcentral.nih.gov/articlerender.fcgi?artid=PMC291589},
volume = {40},
year = {1980}
}
@article{Slove2010a,
abstract = {2. The evidence for such a gradient is, however, ambiguous, and the results have varied as much as the methods. Several studies have considered the non-independence of species, but few have performed explicit phylogenetic analyses. 3. In the present study, we tested for a correlation between diet breadth and latitude of distribution in Nymphalinae butterflies using generalised estimating equations (GEE) and accounting for phylogenetic independence. 4. Using a simple model with only latitude of distribution as a predictor variable revealed a significant positive relationship with diet breadth. Previous studies, however, have shown that diet breadth is also correlated with butterfly range size, and in turn, that range size may be correlated with latitude of distribution. Including geographical range size in the model also turned out to have a profound effect on the results - to the extent that the relationship between latitude of distribution and diet breadth was effectively reversed. 5. We conclude that, at least for this group of butterflies, there is no evidence for a positive correlation between latitude of species distribution and diet breadth when controlling for range size, and that the effect may actually even be reversed.},
author = {Slove, Jessica and Janz, Niklas},
doi = {10.1111/j.1365-2311.2010.01238.x},
issn = {03076946},
journal = {Ecological Entomology},
keywords = {Diversification,Generalisation,Host range,Latitude,Polyphagy,Specialisation},
number = {6},
pages = {768--774},
title = {{Phylogenetic analysis of the latitude-niche breadth hypothesis in the butterfly subfamily Nymphalinae}},
volume = {35},
year = {2010}
}
@article{Post2000,
abstract = {Food-chain length is an important characteristic of ecological communities1: it influences community structure2, ecosystem functions1,2,3,4 and contaminant concentrations in top predators5,6. Since Elton7 first noted that food-chain length was variable among natural systems, ecologists have considered many explanatory hypotheses1,4,8,9, but few are supported by empirical evidence4,10,11. Here we test three hypotheses that predict food-chain length to be determined by productivity alone (productivity hypothesis)4,10,12,13, ecosystem size alone (ecosystem-size hypothesis)14,15 or a combination of productivity and ecosystem size (productive-space hypothesis)7,16,17,18. The productivity and productive-space hypotheses propose that food-chain length should increase with increasing resource availability; however, the productivity hypothesis does not include ecosystem size as a determinant of resource availability. The ecosystem-size hypothesis is based on the relationship between ecosystem size and species diversity, habitat availability and habitat heterogeneity14,15. We find that food-chain length increases with ecosystem size, but that the length of the food chain is not related to productivity. Our results support the hypothesis that ecosystem size, and not resource availability, determines food-chain length in these natural ecosystems.},
author = {Post, David M and Pace, Michael L and Hairston, Nelson G},
doi = {10.1038/35016565},
issn = {1476-4687},
journal = {Nature},
number = {6790},
pages = {1047--1049},
title = {{Ecosystem size determines food-chain length in lakes}},
url = {https://doi.org/10.1038/35016565},
volume = {405},
year = {2000}
}
@article{Trøjelsgaard2013,
abstract = {Aim: Interacting communities of species are organized into complex networks, and network analysis is reckoned to be a strong tool for describing their architecture. Many species assemblies show strong macroecological patterns, e.g. increasing species richness with decreasing latitude, but whether this latitudinal diversity gradient scales up to entities as complex as networks is unknown. We investigated this using a dataset of 54 community-wide pollination networks and hypothesized that pollination networks would display a latitudinal and altitudinal species richness gradient, increasing specialization towards the tropics, and that network topology would be affected by current climate. Location: Global. Methods: Each network was organized as a presence/absence matrix, consisting of P plant species, A pollinator species and their links. From these matrices, network parameters were estimated. Additionally, data about geography (latitude, elevation), climate at the network site (temperature, precipitation) and sampling effort (observation days) and extent (study-plot size) were gathered. Analyses were done using simultaneous autoregressive modelling (SAR). Results: Species richness did not vary strongly with either latitude or elevation. However, network modularity decreased significantly with latitude whereas mean number of links per plant species (L p) and A/P ratio peaked at mid-latitude. Above 500m a.s.l., A/P ratio decreased and mean number of links per pollinator species (L a) increased with elevation. L p displayed mid-ambient peaks with temperature and nestedness and modularity displayed linear relationships with precipitation. Main conclusion: Pollination networks showed macroecological patterns. No strong latitudinal or altitudinal gradient in species richness was observed. L p and the A/P ratio peaked at mid-latitude whereas modularity decreased linearly. Both patterns are suggestive of a more specialized interaction structure towards the tropics. In particular, mean annual precipitation appeared influential on network topology as both nestedness and modularity varied significantly. Importantly, corrected regressions suggest that neither sampling effort nor extent affected the observed patterns. {\textcopyright} 2012 Blackwell Publishing Ltd.},
author = {Tr{\o}jelsgaard, Kristian and Olesen, Jens M.},
doi = {10.1111/j.1466-8238.2012.00777.x},
file = {:Users/alyssacirtwill/Documents/Papers/Tr{\o}jelsgaard, Olesen{\_}2013{\_}Global Ecology and Biogeography.pdf:pdf},
issn = {1466822X},
journal = {Global Ecology and Biogeography},
keywords = {Climate change,Ecological networks,Geographical gradients,Macroecology,Pollination,Sampling effort,Species interactions},
number = {2},
pages = {149--162},
title = {{Macroecology of pollination networks}},
volume = {22},
year = {2013}
}
@article{Capdevila2019,
abstract = {Understanding the combined effects of global and local stressors is crucial for conservation and management, yet challenging due to the different scales at which these stressors operate. Here, we examine the effects of one of the most pervasive threats to marine biodiversity, ocean warming, on the early life stages of the habitat-forming macroalga Cystoseira zosteroides, its long-term consequences for population resilience, and its combined effect with physical stressors. First, we performed a controlled laboratory experiment exploring the impacts of warming on early life stages. Settlement and survival of germlings were measured at 16°C (control), 20°C, and 24°C, and both processes were affected by increased temperatures. Then, we integrated this information into stochastic, density-dependent integral projection models. Recovery time after a major disturbance significantly increased in warmer scenarios. The stochastic population growth rate ($\lambda$s) was not strongly affected by warming alone, as high adult survival compensated for thermal-induced recruitment failure. Nevertheless, warming coupled with recurrent physical disturbances had a strong impact on $\lambda$s and population viability. Synthesis. The impact of warming effects on early stages may significantly decrease the natural ability of habitat-forming algae to rebound after major disturbances. These findings highlight that, in a global warming context, populations of deep-water macroalgae will become more vulnerable to further disturbances, and stress the need to incorporate abiotic interactions into demographic models. {\textcopyright} 2018 The Authors. Journal of Ecology {\textcopyright} 2018 British Ecological Society},
author = {Capdevila, Pol and Hereu, Bernat and Salguero-G{\'{o}}mez, Roberto and Rovira, Graciella and Medrano, Alba and Cebrian, Emma and Garrabou, Joaquim and Kersting, Diego K. and Linares, Cristina},
doi = {10.1111/1365-2745.13090},
issn = {13652745},
journal = {Journal of Ecology},
keywords = {climate change,demography,human impacts,population ecology,quasi-extinction,recovery,seaweeds,stress interactions},
number = {3},
pages = {1129--1140},
title = {{Warming impacts on early life stages increase the vulnerability and delay the population recovery of a long-lived habitat-forming macroalga}},
volume = {107},
year = {2019}
}
@article{Alberdi2018,
abstract = {{\textcopyright} 2017 British Ecological Society. Metabarcoding of environmental samples has many challenges and limitations that require carefully considered laboratory and analysis workflows to ensure reliable results. We explore how decisions regarding study design, laboratory set-up, and bioinformatic processing affect the final results, and provide guidelines for reliable study of environmental samples. We evaluate the performance of four primer sets targeting COI and 16S regions characterizing arthropod diversity in bat faecal samples, and investigate how metabarcoding results are affected by parameters including: (1) number of PCR replicates per sample, (2) sequencing depth, (3) PCR replicate processing strategy (i.e. either additively, by combining the sequences obtained from the PCR replicates, or restrictively, by only retaining sequences that occur in multiple PCR replicates for each sample), (4) minimum copy number for sequences to be retained, (5) chimera removal, and (6) similarity thresholds for Operational Taxonomic Unit (OTU) clustering. Lastly, we measure within- and between-taxa dissimilarities when using sequences from public databases to determine the most appropriate thresholds for OTU clustering and taxonomy assignment. Our results show that the use of multiple primer sets reduces taxonomic biases and increases taxonomic coverage. Taxonomic profiles resulting from each primer set are principally affected by how many PCR replicates are carried out per sample and how sequences are filtered across them, the sequence copy number threshold and the OTU clustering threshold. We also report considerable diversity differences between PCR replicates from each sample. Sequencing depth increases the dissimilarity between PCR replicates unless the bioinformatic strategies to remove allegedly artefactual sequences are adjusted according to the number of analysed sequences. Finally, we show that the appropriate identity thresholds for OTU clustering and taxonomy assignment differ between markers. Metabarcoding of complex environmental samples ideally requires (1) investigation of whether more than one primer sets targeting the same taxonomic group is needed to offset primer biases, (2) more than one PCR replicate per sample, (3) bioinformatic processing of sequences that balance diversity detection with removal of artefactual sequences, and (4) empirical selection of OTU clustering and taxonomy assignment thresholds tailored to each marker and the obtained taxa.},
author = {Alberdi, Antton and Aizpurua, Ostaizka and Gilbert, M. Thomas P. and Bohmann, Kristine},
doi = {10.1111/2041-210X.12849},
file = {:Users/alyssacirtwill/Documents/Papers/Alberdi et al.{\_}2018{\_}Methods in Ecology and Evolution.pdf:pdf},
issn = {2041210X},
journal = {Methods in Ecology and Evolution},
keywords = {PCR replicates,biodiversity assessment,environmental DNA,faecal samples,high throughput sequencing,metabarcoding primers,molecular diet analyses,operational taxonomic unit,primer bias,taxonomic assignment},
number = {1},
pages = {134--147},
title = {{Scrutinizing key steps for reliable metabarcoding of environmental samples}},
volume = {9},
year = {2018}
}
@article{Novotny2004,
abstract = {We characterized a plant-caterpillar food web from secondary vegetation in a New Guinean rain forest that included 63 plant species (87.5{\%} of the total basal area), 546 Lepidoptera species and 1679 trophic links between them. The strongest 14 associations involved 50{\%} of all individual caterpillars while some links were extremely rare. A caterpillar randomly picked from the vegetation will, with ≥ 50{\%} probability, (1) feed on one to three host plants (of the 63 studied), (2) feed on {\textless} 20{\%} of local plant biomass and (3) have ≥ 90{\%} of population concentrated on a single host plant species. Generalist species were quantitatively unimportant. Caterpillar assemblages on locally monotypic plant genera were distinct, while sympatric congeneric hosts shared many caterpillar species. The partitioning of the plant-caterpillar food web thus depends on the composition of the vegetation. In secondary forest the predominant plant genera were locally monotypic and supported locally isolated caterpillar assemblages.},
author = {Novotny, Vojtech and Miller, Scott E. and Leps, Jan and Basset, Yves and Bito, Darren and Janda, Milan and Hulcr, Jiri and Damas, Kipiro and Weiblen, George D.},
doi = {10.1111/j.1461-0248.2004.00666.x},
file = {:Users/alyssacirtwill/Documents/Papers/Novotny et al.{\_}2004{\_}Ecology Letters.pdf:pdf},
issn = {1461023X},
journal = {Ecology Letters},
keywords = {Ecological succession,Herbivore communities,Host specificity,Insect-plant interactions,Invasive species,Lepidoptera,Malesia,Papua New Guinea,Species richness,Tropical forests},
number = {11},
pages = {1090--1100},
title = {{No tree an island: The plant-caterpillar food web of a secondary rain forest in New Guinea}},
volume = {7},
year = {2004}
}
@article{Novotny2002,
author = {Novotny, Vojtech and Basset, Yves and Miller, Scott E and Weiblen, George D. and Bremer, Birgitta and Cizek, Lukas and Drozd, Pavel},
file = {:Users/alyssacirtwill/Documents/Papers/Novotny et al.{\_}2002{\_}Nature.pdf:pdf},
journal = {Nature},
number = {6883},
pages = {841--844},
title = {{Low host specificity of herbivorous insects in a tropical forest}},
url = {http://www.nature.com/doifinder/10.1038/416841a{\%}5Cnpapers2://publication/doi/10.1038/416841a},
volume = {416},
year = {2002}
}
@article{Ødegaard2005,
abstract = {Current methods for measuring similarity among phytophagous insect communities fail to consider the phylogenetic relationship between host plants. We analysed this relation based on 3580 host observations of 1174 beetle species associated with 100 species of angiosperms in two different forest types in Panama. We quantified the significance of genetic distance as well as taxonomic rank among angiosperms in relation to species overlap in beetle assemblages. A logarithmic model describing the decrease in beetle species similarity between host-plant species of increasing phylogenetic distance explains 35{\%} of the variation. Applied to taxonomic rank categories the results imply that except for the ancient branching of monocots from dicots, only adaptive radiations of plants on the family and genus level are important for host utilization among phytophagous beetles. These findings enable improvements in estimating host specificity and species richness through correction for phylogenetic relatedness between hosts and consideration of the host-specific fauna associated with monocots.},
author = {{\O}degaard, Frode and Diserud, Ola H. and {\O}stbye, Kjartan},
doi = {10.1111/j.1461-0248.2005.00758.x},
issn = {1461023X},
journal = {Ecology Letters},
keywords = {Canopy crane,Coleoptera,Evolution of host range,Herbivore communities,Host specificity,Insect-plant interactions,Panama,Plant taxonomy,Species richness,Tropical forests},
number = {6},
pages = {612--617},
title = {{The importance of plant relatedness for host utilization among phytophagous insects}},
volume = {8},
year = {2005}
}
@article{Prosser2010,
abstract = {Andr{\'{e}}n and colleagues (2008) recently published a paper reminding us of good scientific practice, taught and learnt during our early years of research, but fre- quently forgotten. This article has a similar objective. It is highly likely that the majority of microbial ecologists have taken a basic course in statistics. Unfortunately, the basic principles of statistical analysis and its significance and necessity are frequently ignored. My own exposure to this problem, and my suffering of the consequences, arise through reading, reviewing and editing articles on microbial diversity. The suffering is greatest with articles utilizing ‘expensive' or cutting edge techniques, such as analysis of clone library sequences, microarray analysis and (increasingly) high-throughput sequencing. The problems, however, are more widespread and exist beyond these techniques and beyond studies of micro- bial diversity.},
author = {Prosser, James I.},
doi = {10.1111/j.1462-2920.2010.02201.x},
journal = {Environmental Microbiology},
number = {7},
pages = {1806--1810},
title = {{Replicate or lie}},
volume = {12},
year = {2010}
}
@article{Volf2017,
author = {Volf, M and Pyszko, P and Abe, T and Libra, M and Kot{\'{a}}skov{\'{a}}, N and {\v{S}}igut, M and Kumar, R and Kaman, O and Butterill, P. T. and {\v{S}}ipo{\v{s}}, J. and Abe, H and Fukushima, H and Drozd, P and Kamata, N and Murakami, M and Novotny, V},
doi = {10.1111/ijlh.12426},
issn = {1751553X},
journal = {Journal of Animal Ecology},
number = {3},
pages = {556--565},
title = {{Phylogenetic composition of host plant communities drives plant-herbivore food web structure}},
volume = {86},
year = {2017}
}
@article{econullnetr,
author = {Vaughan, Ian P and Gotelli, Nicholas J. and Memmott, Jane and Pearson, Caitlin E and Woodward, Guy and Symondson, William O.C.},
journal = {Methods in Ecology and Evolution},
number = {3},
pages = {728----733},
title = {{econullnetr: an R package using null models to analyse the structure of ecological networks and identify resource selection}},
volume = {9},
year = {2018}
}
@book{changepoint2,
address = {R package version 2.2.2},
author = {Killick, R. and Haynes, K and Eckley, I. A.},
title = {{changepoint: an R package for changepoint analysis}},
year = {2016}
}
@book{nlme,
address = {R package version 3.1-137},
author = {Pinheiro, Jose and Bates, Douglas and DebRoy, Saikat and Sarkar, Deepayan},
title = {{nlme: Linear and nonlinear mixed effects models}},
year = {2018}
}
@article{changepoint,
author = {Killick, R and Eckley, I. A.},
journal = {Journal of Statistical Software},
number = {3},
pages = {1--19},
title = {{changepoint: an R package for changepoint analysis}},
volume = {58},
year = {2014}
}
@article{Bergamini2017,
abstract = {Ecologists are increasingly aware of the interplay between evolutionary history and ecological processes in shaping current species interaction patterns. The inclusion of phylogenetic relationships in studies of species interaction networks has shown that closely related species commonly interact with sets of similar species. Notably, the degree of phylogenetic conservatism in antagonistic ecological interactions is frequently stronger among species at lower trophic levels than among those at higher trophic levels. One hypothesis that accounts for this asymmetry is that competition among consumer species promotes resource partitioning and offsets the maintenance of dietary similarity by phylogenetic inertia. Here, we used a regional plant–herbivore network comprised of Asteraceae species and flower-head endophagous insects to evaluate how the strength of phylogenetic conservatism in species interactions differs between the two trophic levels. We also addressed whether the asymmetry in the strength of the phylogenetic signal between plants and animals depends on the overall degree of relatedness among the herbivores. We show that, beyond the previously reported compositional similarity, closely related species also share a greater proportion of counterpart phylogenetic history, both for resource and consumer species. Comparison of the patterns found in the entire network with those found in subnetworks composed of more phylogenetically restricted groups of herbivores provides evidence that resource partitioning occurs mostly at deeper phylogenetic levels, so that a positive phylogenetic signal in antagonist similarity is detectable even between closely related consumers in monophyletic subnetworks. The asymmetry in signal strength between trophic levels is most apparent in the way network modules reflect resource phylogeny, both for the entire network and for subnetworks. Taken together, these results suggest that evolutionary processes, such as phylogenetic conservatism and independent colonization history of the insect groups may be the main forces generating the phylogenetic structure observed in this particular plant–herbivore network system.},
author = {Bergamini, Leonardo Lima and Lewinsohn, Thomas M. and Jorge, Leonardo R. and Almeida-Neto, M{\'{a}}rio},
doi = {10.1111/oik.03567},
issn = {16000706},
journal = {Oikos},
number = {5},
pages = {703--712},
title = {{Manifold influences of phylogenetic structure on a plant–herbivore network}},
volume = {126},
year = {2017}
}
@article{Ponisio2017a,
abstract = {One of the major challenges in evolutionary ecology is to understand how coevolution shapes species interaction networks. Important topological properties of networks such as nestedness and modularity are thought to be affected by coevolution. However, there has been no test whether coevolution does, in fact, lead to predictable network structure. Here, we investigate the structure of simulated bipartite networks generated under different modes of coevolution. We ask whether evolutionary processes influence network structure and, furthermore, whether any emergent trends are influenced by the strength or “intimacy” of the species interactions. We find that coevolution leaves a weak and variable signal on network topology, particularly nestedness and modularity, whichwas not strongly affected by the intimacy of interactions. Our findings indicate that network metrics, on their own, should not be used to make inferences about processes underlying the evolutionary history of communities. Instead, a more holistic approach that combines network approaches with traditional phylogenetic and biogeographic reconstructions is needed. Key},
author = {Ponisio, Lauren C. and M'Gonigle, Leithen K.},
doi = {10.1002/ecs2.1798},
issn = {21508925},
journal = {Ecosphere},
keywords = {Bipartite,Evolution,Interaction intimacy,Modularity,Nestedness,Phylogenetic interaction structure},
number = {4},
title = {{Coevolution leaves a weak signal on ecological networks}},
volume = {8},
year = {2017}
}
@article{Aizen2016,
abstract = {Similarity among species in traits related to ecological interactions is frequently associated with common ancestry. Thus, closely related species usually interact with ecologically similar partners, which can be reinforced by diverse co-evolutionary processes. The effect of habitat fragmentation on the phylogenetic signal in interspecific interactions and correspondence between plant and ani- mal phylogenies is, however, unknown. Here, we address to what extent phylogenetic signal and co-phylogenetic congruence of plant–animal interactions depend on habitat size and isolation by analysing the phylogenetic structure of 12 pollination webs from isolated Pampean hills. Phyloge- netic signal in interspecific interactions differed among webs, being stronger for flower-visiting insects than plants. Phylogenetic signal and overall co-phylogenetic congruence increased indepen- dently with hill size and isolation. We propose that habitat fragmentation would erode the phylo- genetic structure of interaction webs. A decrease in phylogenetic signal and co-phylogenetic correspondence in plant–pollinator interactions could be associated with less reliable mutualism and erratic co-evolutionary change.},
author = {Aizen, Marcelo A. and Gleiser, Gabriela and Sabatino, Malena and Gilarranz, Luis J. and Bascompte, Jordi and Verd{\'{u}}, Miguel},
doi = {10.1111/ele.12539},
file = {:Users/alyssacirtwill/Documents/Papers/Aizen et al.{\_}2016{\_}Ecology Letters.pdf:pdf},
issn = {14610248},
journal = {Ecology Letters},
keywords = {Area effect,Co-phylogenetic correspondence,Habitat islands,Isolation,Mutualistic networks,Pampas,Phylogenetic structure,Pollination webs},
number = {1},
pages = {29--36},
title = {{The phylogenetic structure of plant-pollinator networks increases with habitat size and isolation}},
volume = {19},
year = {2016}
}
@article{Maron2019,
author = {Maron, John L. and Agrawal, Anurag A. and Schemske, Douglas W.},
doi = {10.1002/ecy.2704},
issn = {0012-9658},
journal = {Ecology},
pages = {e02704},
title = {{Plant‐herbivore coevolution and plant speciation}},
url = {https://onlinelibrary.wiley.com/doi/abs/10.1002/ecy.2704},
year = {2019}
}
@article{Gilbert2015,
abstract = {The host ranges of plant pathogens and herbivores are phylogenetically constrained, so that closely related plant species are more likely to share pests and pathogens. Here we conducted a reanalysis of data from published experimental studies to test whether the severity of host-enemy interactions follows a similar phylogenetic signal. The impact of herbivores and pathogens on their host plants declined steadily with phylogenetic distance from the most severely affected focal hosts. The steepness of this phylogenetic signal was similar to that previously measured for binary-response host ranges. Enemy behavior and development showed similar, but weaker phylogenetic signal, with oviposition and growth rates declining with evolutionary distance from optimal hosts. Phylogenetic distance is an informative surrogate for estimating the likely impacts of a pest or pathogen on potential plant hosts, and may be particularly useful in early assessing risk from emergent plant pests, where critical decisions must be made with incomplete host records.},
author = {Gilbert, Gregory S. and Briggs, Heather M. and Magarey, Roger},
doi = {10.1371/journal.pone.0123758},
file = {:Users/alyssacirtwill/Documents/Papers/Gilbert, Briggs, Magarey{\_}2015{\_}PLoS ONE.PDF:PDF},
issn = {19326203},
journal = {PLoS ONE},
number = {4},
pages = {e0123758},
title = {{The impact of plant enemies shows a phylogenetic signal}},
volume = {10},
year = {2015}
}
@article{Junker2015,
abstract = {Community ecology has moved from descriptive studies to more mechanistic approaches asking questions about causes and consequences of community composition and interactions between species. Many ecological processes are shaped by the presense or absence of functional groups, not necessarily species. Thus, the diversity of functional traits, i.e. their interspecific variation, is a key feature of plant communities with consequences on other trophic levels. In a simulation study based on a quantitative flower-visitor network and quantitative measurements of flower traits, we tested how the functional FDiv and phylogenetic PDiv of plant communities affect animal species richness and diversity as well as network properties. Within the limitations of the assumption that plants maintain the qualitative and quantitative interactions with animals in subsampled communities, we found that functionally diverse plant communities support a large number of animal species (not necessarily animal diversity). Additionally, the network structure was more complementarily specialized (higher H'2-values) and comprised a larger number of unrealized links (low connectance) and thus a higher partitioning of resources among consumers in functionally diverse plant communities than in communities with a lower FDiv. For the phylogenetic diversity PDic of plant communities we found contrasting effects, which may be explained by divergences or convergences of functional traits. Our results support the notion that functionally diverse plant communities offer a large number of niches that can be occupied by a larger number of flower visiting species specialized to a specific set of flower traits. Thus, functional flower triats serve as barriers that exclude some flower visitors but also as attractive features that facilitate interactions with other animal species. Our study fosters a trait-based definition of niches and functional groups and may stimulate field studies testing the predictions of this simulation.},
author = {Junker, Robert R. and Bl{\"{u}}thgen, Nico and Keller, Alexander},
doi = {10.1007/s10682-014-9747-2},
file = {:Users/alyssacirtwill/Documents/Papers/Junker, Bl{\"{u}}thgen, Keller{\_}2015{\_}Evolutionary Ecology.pdf:pdf},
issn = {02697653},
journal = {Evolutionary Ecology},
keywords = {Antagonistic flower visitors,Barriers,Divergences,Evolution of flower traits,Networks,Quantitative functional traits},
number = {3},
pages = {437--450},
title = {{Functional and phylogenetic diversity of plant communities differently affect the structure of flower-visitor interactions and reveal convergences in floral traits}},
volume = {29},
year = {2015}
}
@article{Endara2017,
abstract = {Coevolutionary models suggest that herbivores drive diversification and community composition in plants. For herbivores, many questions remain regarding how plant defenses shape host choice and community structure. We addressed these questions using the tree genus Inga and its lepidopteran herbivores in the Amazon. We constructed phylogenies for both plants and insects and quantified host associations and plant defenses. We found that similarity in herbivore assemblages between Inga species was correlated with similarity in defenses. There was no correlation with phylogeny, a result consistent with our observations that the expression of defenses in Inga is independent of phylogeny. Furthermore, host defensive traits explained 40{\%} of herbivore community similarity. Analyses at finer taxonomic scales showed that different lepidopteran clades select hosts based on different defenses, suggesting taxon-specific histories of herbivore-host plant interactions. Finally, we compared the phylogeny and defenses of Inga to phylogenies for the major lepidopteran clades. We found that closely related herbivores fed on Inga with similar defenses rather than on closely related plants. Together, these results suggest that plant defenses might be more evolutionarily labile than the herbivore traits related to host association. Hence, there is an apparent asymmetry in the evolutionary interactions between Inga and its herbivores. Although plants may evolve under selection by herbivores, we hypothesize that herbivores may not show coevolutionary adaptations, but instead "chase" hosts based on the herbivore's own traits at the time that they encounter a new host, a pattern more consistent with resource tracking than with the arms race model of coevolution.},
author = {Endara, Mar{\'{i}}a-Jos{\'{e}} and Coley, Phyllis D. and Ghabash, Gabrielle and Nicholls, James A. and Dexter, Kyle G. and Donoso, David A. and Stone, Graham N. and Pennington, R. Toby and Kursar, Thomas A.},
doi = {10.1073/pnas.1707727114},
file = {:Users/alyssacirtwill/Documents/Papers/Endara et al.{\_}2017{\_}Proceedings of the National Academy of Sciences.pdf:pdf},
issn = {0027-8424},
journal = {Proceedings of the National Academy of Sciences},
number = {36},
pages = {E7499--E7505},
title = {{Coevolutionary arms race versus host defense chase in a tropical herbivore–plant system}},
volume = {114},
year = {2017}
}
@article{Karinho-Betancourt2015,
abstract = {Theory predicts patterns of defense across taxa based on notions of tradeoffs and synergism among defensive traits when plants and herbivores coevolve. Because the expression of char- acters changes ontogenetically, the evolution of plant strategies may be best understood by considering multiple traits along a trajectory of plant development. Here we addressed the ontogenetic expression of chemical and physical defenses in 12 Datura species, and tested for macroevolutionary correlations between defensive traits using phylogenetic analyses. We used liquid chromatography coupled to mass spectrometry to identify the toxic tropane alkaloids of Datura, and also estimated leaf trichome density. We report three major patterns. First, we found different ontogenetic trajectories of alkaloids and leaf trichomes, with alkaloids increasing in concentration at the reproductive stage, whereas trichomes were much more variable across species. Second, the dominant alkaloids and leaf trichomes showed correlated evolution, with positive and negative associations. Third, the correlations between defensive traits changed across ontogeny, with significant relationships only occurring during the juvenile phase. The patterns in expression of defensive traits in the genus Datura are suggestive of adapta- tion to complex selective environments varying in space and time.},
author = {Kari{\~{n}}ho-Betancourt, Eunice and Agrawal, Anurag A. and Halitschke, Rayko and N{\'{u}}{\~{n}}ez-Farf{\'{a}}n, Juan},
doi = {10.1111/nph.13300},
issn = {14698137},
journal = {New Phytologist},
keywords = {Chemical ecology,Comparative method,Datura,Leaf trichome,Ontogeny,Plant defense,Tradeoffs,Tropane alkaloids},
number = {2},
pages = {796--806},
title = {{Phylogenetic correlations among chemical and physical plant defenses change with ontogeny}},
volume = {206},
year = {2015}
}
@article{Sydenham2018,
abstract = {Aim Because the ecological similarity between species is expected to increase with relatedness and that speciation is a local process, phylogeny may provide a common measure for the influence of ecological and biogeographic processes on community assembly. We tested if similarities in floral visitation patterns within communities and the phylogenetic beta-diversity among communities were related to the position of bees within the bee phylogeny. Location Global. Methods We combined a genus level phylogeny with within-genera phylogenies for the bee species occurring within 18 globally distributed bee-flower networks. Networks consisted of a matrix of bee and plant species and information on whether or not a bee species had been observed visiting flowers of a given plant species. For each network, we used Abouheif's Cmean to test if the similarity in floral associations (niche similarity) between bees and the number of plant species visited displayed a significant phylogenetic signal. To test if biogeography influenced the relatedness among species within networks we tested if the phylogenetic beta-diversity increased with geographical distance and dissimilarity in climatic conditions among networks. Results We found a phylogenetic signal for niche similarity in only 50{\%} of the bee-flower networks. However, network size influenced the likelihood of observing a phylogenetic signal and for seven of the eight bee-flower networks with {\textgreater}20 species it was statistically significant. On a global scale, the phylogenetic beta-diversity increased with geographical distances and with climatic dissimilarity between sites. Main conclusions Bee communities are structured by processes of speciation and migration so that regional species pools are dominated by a subset of the global phylogenetic clades, resulting in increasing phylogenetic beta-diversity with geographical distance. Moreover, ecological filtering processes operating at both local (floral resource use) and continental (climatic constraints) scale determine the distribution of species among resources and geographical regions. The assembly of bee communities should therefore be understood as a product of both biogeographic and community ecological processes.},
author = {Sydenham, Markus Arne Kj{\ae}r and Eldegard, Katrine and Hegland, Stein Joar and Nielsen, Anders and Totland, {\O}rjan and Fjellheim, Siri and Moe, Stein R.},
doi = {10.1111/jbi.13103},
issn = {13652699},
journal = {Journal of Biogeography},
keywords = {beta-diversity,biogeography,community ecology,phylogenetic signal,plant–pollinator networks,wild bees},
number = {2},
pages = {461--472},
title = {{Community level niche overlap and broad scale biogeographic patterns of bee communities are driven by phylogenetic history}},
volume = {45},
year = {2018}
}
@article{Ibanez2016,
abstract = {Phylogenetically related species share a common evolutionary history and may therefore have similar traits. In terms of interaction networks, where traits are a major determinant, related species should therefore interact with other species which are also related. However, this prediction is challenged by current evidence that there is a weak, albeit significant, phylogenetic signal in species' taxonomic niche, i.e., the identity of interacting species. We studied mutualistic and antagonistic plant--insect interaction networks in species-rich alpine meadows and show that there is instead a very strong phylogenetic signal in species' functional niches---i.e., the mean functional traits of their interactors. This pattern emerges because related species tend to interact with species bearing certain traits that allow biotic interactions (pollination, herbivory) but not necessarily with species from all the same evolutionary lineages. Those traits define a set of potential interactors and show clear patterns of phylogenetic clustering on several portions of plants and insect phylogenies. Thus, this emerging pattern of low phylogenetic signal in taxonomic niches but high phylogenetic signal in functional niches may be driven by the interplay between functional trait convergence across plants' and insects' phylogenies and random sampling of the potential interactors.},
author = {Ibanez, S{\'{e}}bastien and Ar{\`{e}}ne, Fabien and Lavergne, S{\'{e}}bastien},
doi = {10.1007/s00442-016-3552-2},
file = {:Users/alyssacirtwill/Documents/Papers/Ibanez, Ar{\`{e}}ne, Lavergne{\_}2016{\_}Oecologia.pdf:pdf},
issn = {00298549},
journal = {Oecologia},
keywords = {Functional niche,Phylogenetic signal,Plant–herbivore,Plant–pollinator,Taxonomic niche},
number = {4},
pages = {989--1000},
publisher = {Springer Berlin Heidelberg},
title = {{How phylogeny shapes the taxonomic and functional structure of plant–insect networks}},
volume = {180},
year = {2016}
}
@misc{Levine2017,
abstract = {T he goal of ecological research on species coexistence is to explain how the tremendous diversity of species that we see in nature persists despite differences between species in competitive ability 1,2. However, empirically evaluating the interactions between a large set of competitors is logistically challenging, and many of the mathematical tools for analysing the interaction between a pair of competitors do not translate readily to large networks of competing species 3. As a consequence, coexistence research has focused overwhelmingly on mechanisms that operate between pairs of competitors. Although the emphasis on pairwise coexistence may prove valid, and great progress in understanding the maintenance of species diversity has been achieved through a pairwise approach 2 (Box 1), ecologists have had difficulty showing that the coexistence of many species in diverse ecosystems results from pairwise mechanisms. How probable is it that the more than 1,000 tropical tree species found in a 25-hectare plot in the Amazon rainforest coexist because of countless pairwise niche differences between the competitors 4,5 ? A tantalizing explanation for coexistence in species-rich communities involves mechanisms that emerge only in diverse systems of competitors. Indeed, systems of more than two competitors form a network of competitive relationships, the structure of which should influence the dynamics of the system as a whole 3,6. The degree to which studying coexistence between pairs of species in isolation can help us to understand the dynamics of complex competitive networks is simply not known 7,8. However, it is known that network structure can strongly determine the robust-ness of mutualistic and multitrophic networks to perturbations, as well as radically change the outcome of the pairwise interactions 9-11. We might therefore expect similarly powerful consequences of network structure for the dynamics of diverse competitive systems. In this Review, we propose that to understand the maintenance of species diversity, ecologists must better explore the coexistence mechanisms that result from the structure of diverse competitive networks. We further suggest that this understanding can be accelerated by applying lessons from the study of mutualistic and multitrophic networks to competitive systems. Importantly, a better understanding of coexistence mechanisms that emerge only in diverse systems would shed light on the stability of biodiversity. By definition, these coexistence mechanisms erode as species are lost. As a consequence, the loss of one competitor may lead to the subsequent loss of others, an extinction cascade well known from the theoretical study of trophic 12 and mutualistic 13,14 networks, but studied rarely in competitive systems (see ref. 15 for an example). Here, we discuss the theoretical and empirical literature on mechanisms of coexistence that emerge only in networks of more than two competitors. Despite these mechanisms being demonstrated with mathematical models almost fifty years ago, convincing empirical tests of their operation remain rare, which leaves the implications of these interactions for coexistence in nature unknown. Although we focus exclusively on the interactions between competitors, throughout the Review we highlight findings from the study of trophic and mutual-istic networks that help to show how diverse competitive networks operate and can be analysed. We then lay out a roadmap for advancing the understanding of coexistence mechanisms that emerge only in systems of more than two species. This involves developing a pre-dictive understanding of when such mechanisms are likely to operate, empirically evaluating their prevalence and importance in nature and demonstrating theoretically how they influence coexistence in truly diverse systems. Coexistence between more than two competitors Theory shows that two kinds of competitive dynamics-interaction chains and higher-order interactions-emerge only in networks of three or more species (Fig. 1). Such interactions do not necessarily stabilize , and can in fact destabilize, coexistence. Therefore, first we define these interactions and then we explain the conditions under which they promote species richness-our measure of diversity in this Review. Interaction chains emerge when pairwise competitive interactions are embedded in a network of other (still pairwise) interactions. As in a trophic cascade, the indirect effects that result arise from changes in the density of a third (or further) species that interacts with both species of the focal pair (Fig. 1b). Even when all direct pairwise interactions are negative, these indirect effects are often positive 16. The best-studied stabilizing competitive network involves intransitive competition among three species, as underpins a game of rock-paper-scissors 6,17. Although the interactions between the species remain fundamentally pairwise, the stabilized dynamics emerge from stringing these pairwise interactions together, so that changes in density propagate through the network to form a negative feedback loop that counteracts the initial perturbation. Higher-order interactions emerge when the interactions between species are no longer fundamentally pairwise. Instead, the per capita effect The tremendous diversity of species in ecological communities has motivated a century of research into the mechanisms that maintain biodiversity. However, much of this work examines the coexistence of just pairs of competitors. This approach ignores those mechanisms of coexistence that emerge only in diverse competitive networks. Despite the potential for these mechanisms to create conditions under which the loss of one competitor triggers the loss of others, we lack the knowledge needed to judge their importance for coexistence in nature. Progress requires borrowing insight from the study of multitrophic interaction networks, and coupling empirical data to models of competition. {\textcopyright} 2 0 1 7 M a c m i l l a n P u b l i s h e r s L i m i t e d , p a r t o f S p r i n g e r N a t u r e. A l l r i g h t s r e s e r v e d .},
author = {Levine, Jonathan M. and Bascompte, Jordi and Adler, Peter B. and Allesina, Stefano},
booktitle = {Nature},
doi = {10.1038/nature22898},
issn = {14764687},
number = {7656},
title = {{Beyond pairwise mechanisms of species coexistence in complex communities}},
volume = {546},
year = {2017}
}
@article{Ponisio2019,
author = {Ponisio, Lauren C. and de Valpine, Perry and M'Gonigle, Leithen K. and Kremen, Claire},
doi = {10.1111/ele.13257},
issn = {1461-023X},
journal = {Ecology Letters},
keywords = {2019,agriculture,ecology letters,graph,hedgerow,metapopulation,network,restoration,wild bee},
pages = {ele.13257},
title = {{Proximity of restored hedgerows interacts with local floral diversity and species' traits to shape long‐term pollinator metacommunity dynamics}},
url = {https://onlinelibrary.wiley.com/doi/abs/10.1111/ele.13257},
year = {2019}
}
@article{Deagle2019,
abstract = {Advances in DNA sequencing technology have revolutionised the field of molecular analysis of trophic interactions and it is now possible to recover counts of food DNA barcode sequences from a wide range of dietary samples. But what do these counts mean? To obtain an accurate estimate of the overall diet of a consumer should we work strictly with datasets summarising the frequency of occurrence of different food taxa, or is it possible to use the relative number of sequences? Both approaches are applied in the dietary metabarcoding literature, but occurrence data is often promoted as a more conservative and reliable option due to taxa-specific biases in recovery of sequences. Here, we point out that diet summaries based on occurrence data overestimate the importance of food consumed in small quantities (potentially including low-level contaminants) and are sensitive to the count threshold used to define an occurrence. Our simulations indicate that even with recovery biases incorporated, using relative read abundance (RRA) information can provide a more accurate view of population-level diet in many scenarios. The ideas presented here highlight the need to consider all sources of bias and to justify the methods used to interpret count data in dietary metabarcoding studies. We encourage researchers to continue to addressing methodological challenges, and acknowledge unanswered questions to help spur future investigations in this rapidly developing area of research.},
author = {Deagle, Bruce E. and Thomas, Austen C. and McInnes, Julie C. and Clarke, Laurence J. and Vesterinen, Eero J. and Clare, Elizabeth L. and Kartzinel, Tyler R. and Eveson, J. Paige},
doi = {10.1111/mec.14734},
file = {:Users/alyssacirtwill/Documents/Papers/Deagle et al.{\_}2019{\_}Molecular Ecology.pdf:pdf},
issn = {1365294X},
journal = {Molecular Ecology},
number = {2},
pages = {391--406},
title = {{Counting with DNA in metabarcoding studies: how should we convert sequence reads to dietary data?}},
volume = {28},
year = {2019}
}
@article{Mata2019,
abstract = {DNA metabarcoding is increasingly used in dietary studies to estimate diversity, composition and frequency of occurrence of prey items. However, few studies have assessed how technical and biological replication affect the accuracy of diet estimates. This study addresses these issues using the European free-tailed bat Tadarida teniotis, involving high-throughput sequencing of a small fragment of the COI gene in 15 separate faecal pellets and a 15-pellet pool per each of 20 bats. We investigated how diet descriptors were affected by variability among (a) individuals, (b) pellets of each individual and (c) PCRs of each pellet. In addition, we investigated the impact of (d) analysing separate pellets vs. pellet pools. We found that diet diversity estimates increased steadily with the number of pellets analysed per individual, with seven pellets required to detect {\~{}}80{\%} of prey species. Most variation in diet composition was associated with differences among individual bats, followed by pellets per individual and PCRs per pellet. The accuracy of frequency of occurrence estimates increased with the number of pellets analysed per bat, with the highest error rates recorded for prey consumed infrequently by many individuals. Pools provided poor estimates of diet diversity and frequency of occurrence, which were comparable to analysing a single pellet per individual, and consistently missed the less common prey items. Overall, our results stress that maximizing biological replication is critical in dietary metabarcoding studies and emphasize that analysing several samples per individual rather than pooled samples produce more accurate results.},
author = {Mata, Vanessa A. and Rebelo, Hugo and Amorim, Francisco and McCracken, Gary F. and Jarman, Simon and Beja, Pedro},
doi = {10.1111/mec.14779},
file = {:Users/alyssacirtwill/Documents/Papers/Archive/Mata et al.{\_}2018{\_}Molecular Ecology.pdf:pdf},
issn = {1365294X},
journal = {Molecular Ecology},
keywords = {bat ecology,metabarcoding,molecular diet analyses,replication,sampling design,trophic ecology},
number = {2},
pages = {165--175},
title = {{How much is enough? Effects of technical and biological replication on metabarcoding dietary analysis}},
volume = {28},
year = {2019}
}
@article{Brose2006,
abstract = {It has been suggested that differences in body size between consumer and resource species may have important implications for interaction strengths, population dynamics, and eventually food web structure, function, and evolution. Still, the general distribution of consumer-'resource body-size ratios in real ecosystems, and whether they vary systematically among habitats or broad taxonomic groups, is poorly understood. Using a unique global database on consumer and resource body sizes, we show that the mean body-size ratios of aquatic herbivorous and detritivorous consumers are several orders of magnitude larger than those of carnivorous predators. Carnivorous predator-prey body-size ratios vary across different habitats and predator and prey types (invertebrates, ectotherm, and endotherm vertebrates). Predator-prey body-size ratios are on average significantly higher (1) in freshwater habitats than in marine or terrestrial habitats, (2) for vertebrate than for invertebrate predators, and (3) for invertebrate than for ectotherm vertebrate prey. If recent studies that relate body-size ratios to interaction strengths are general, our results suggest that mean consumer-resource interaction strengths may vary systematically across different habitat categories and consumer types.},
archivePrefix = {arXiv},
arxivId = {10.1890/0012-9658(2006)87[2411:CBRINF]2.0.CO;2},
author = {Brose, Ulrich and Jonsson, Tomas and Berlow, Eric L. and Warren, Philip and Banasek-Richter, Carolin and Bersier, Louis F{\'{e}}lix and Blanchard, Julia L. and Brey, Thomas and Carpenter, Stephen R. and Blandenier, Marie France Cattin and Cushing, Lara and Dawah, Hassan A. and Dell, Tony and Edwards, Francois and Harper-Smith, Sarah and Jacob, Ute and Ledger, Mark E. and Martinez, Neo D. and Memmott, Jane and Mintenbeck, Katja and Pinnegar, John K. and Rall, Bj{\"{o}}rn C. and Rayner, Thomas S. and Reuman, Daniel C. and Ruess, Liliane and Ulrich, Werner and Williams, Richard J. and Woodward, Guy and Cohen, Joel E.},
doi = {10.1890/0012-9658(2006)87[2411:CBRINF]2.0.CO;2},
eprint = {0012-9658(2006)87[2411:CBRINF]2.0.CO;2},
isbn = {0012-9658},
issn = {00129658},
journal = {Ecology},
keywords = {Allometry,Body length,Body mass,Body-size ratio,Food webs,Parasitoid-host,Predation,Predator-prey},
number = {10},
pages = {2411--2417},
pmid = {17089649},
primaryClass = {10.1890},
title = {{Consumer-resource body-size relationships in natural food webs}},
url = {http://www.esajournals.org/doi/abs/10.1890/0012-9658(2006)87{\%}5B2411:CBRINF{\%}5D2.0.CO;2},
volume = {87},
year = {2006}
}
@article{Mora2018bioRxiv,
author = {Mora, Bernat Bramon and Cirtwill, Alyssa R. and Stouffer, Daniel B},
doi = {doi: https://doi.org/10.1101/364703},
journal = {bioRxiv},
title = {pymfinder: a tool for the motif analysis of binary and quantitative complex networks},
year = {2018}
}
@article{Kortsch2019,
abstract = {Large‐scale patterns in species diversity and community composition are associated with environmental gradients, but the implications of these patterns for food‐web structure are still unclear. Here, we investigated how spatial patterns in food‐web structure are associated with environmental gradients in the Barents Sea, a highly productive shelf sea of the Arctic Ocean. We compared food webs from 25 subregions in the Barents Sea and examined spatial correlations among food‐web metrics, and between metrics and spatial variability in seawater temperature, bottom depth and number of days with ice cover. Several food‐web metrics were positively associated with seawater temperature: connectance, level of omnivory, clustering, cannibalism, and high variability in generalism, while other food‐web metrics such as modularity and vulnerability were positively associated with sea ice and negatively with temperature. Food‐web metrics positively associated with habitat heterogeneity were: number of species, link density, omnivory, path length, and trophic level. This finding suggests that habitat heterogeneity promotes food‐web complexity in terms of number of species and link density. Our analyses reveal that spatial variation in food‐web structure along the environmental gradients is partly related to species turnover. However the higher interaction turnover compared to species turnover along these gradients indicates a consistent modification of food‐web structure, implying that interacting species may co‐vary in space. In conclusion, our study shows how environmental heterogeneity, via environmental filtering, influences not only turnover in species composition, but also the structure of food webs over large spatial scales.},
author = {Kortsch, Susanne and Primicerio, Raul and Aschan, Michaela and Lind, Sigrid and Dolgov, Andrey V. and Planque, Benjamin},
doi = {10.1111/ecog.03443},
file = {:Users/alyssacirtwill/Documents/Papers/Kortsch et al.{\_}2019{\_}Ecography.pdf:pdf},
isbn = {0000000264},
issn = {16000587},
journal = {Ecography},
keywords = {Arctic,Barents Sea,biogeography},
number = {2},
pages = {295--308},
title = {{Food-web structure varies along environmental gradients in a high-latitude marine ecosystem}},
volume = {42},
year = {2019}
}
@article{Valverde2016,
abstract = {The pollination success of animal-pollinated plants depends on the temporal coupling between flowering schedules and pollinator availability. Within a population, individual plants exhibiting disparate flowering schedules will be exposed to different pollinators when the latter exhibit temporal turnover. The temporal overlap between individual plants and pollinators will result in a turnover of interactions, which can be analyzed through a network approach. We have explored the temporal dynamics of individual-based plant networks resulting from pairwise similarities in pollinator composition. During two flowering seasons, we surveyed the phenology and pollinator fauna of the individual plants from a population of Erysimum mediohispanicum (Brassicaceae). We analyzed the topology of these networks by means of their modularity, clustering, and core-periphery structure. These metrics are related to network functional properties such as cohesion, transitivity and centralization respectively. Afterwards, we analyzed the influence of each pollinator functional group on network topology. We found that network topology varied widely over time as a consequence of the differences in plant phenology and the idiosyncratic and contextual effect of pollinators. When integrating all temporary networks, the network became cohesive (non modular), transitive (locally clusterized), and centralized (core-periphery topology). These topologies could entail important consequences for plant reproduction. Our results highlight the importance of considering the entire flowering season and the necessity of making comprehensive temporal sampling when trying to build reliable interaction networks.},
author = {Valverde, Javier and G{\'{o}}mez, Jos{\'{e}} Maria and Perfectti, Francisco},
doi = {10.1111/oik.02661},
file = {:Users/alyssacirtwill/Documents/Papers/Valverde, G{\'{o}}mez, Perfectti{\_}2016{\_}Oikos.pdf:pdf},
isbn = {2200822340},
issn = {16000706},
journal = {Oikos},
number = {4},
pages = {468--479},
pmid = {24151254},
title = {{The temporal dimension in individual-based plant pollination networks}},
volume = {125},
year = {2016}
}
@article{Tur2014,
abstract = {Most plant-pollinator network studies are conducted at species level, whereas little is known about network patterns at the individual level. In fact, nodes in traditional species-based interaction networks are aggregates of individuals establishing the actual links observed in nature. Thus, emergent properties of interaction networks might be the result of mechanisms acting at the individual level. Pollen loads carried by insect flower visitors from two mountain communities were studied to construct pollen-transport networks. For the first time, these community-wide pollen-transport networks were downscaled from species-species (sp-sp) to individuals-species (i-sp) in order to explore specialization, network patterns and niche variation at both interacting levels. We used a null model approach to account for network size differences inherent to the downscaling process. Specifically, our objectives were (i) to investigate whether network structure changes with downscaling, (ii) to evaluate the incidence and magnitude of individual specialization in pollen use and (iii) to identify potential ecological factors influencing the observed degree of individual specialization. Network downscaling revealed a high specialization of pollinator individuals, which was masked and unexplored in sp-sp networks. The average number of interactions per node, connectance, interaction diversity and degree of nestedness decreased in i-sp networks, because generalized pollinator species were composed of specialized and idiosyncratic conspecific individuals. An analysis with 21 pollinator species representative of two communities showed that mean individual pollen resource niche was only c. 46{\%} of the total species niche. The degree of individual specialization was associated with inter- and intraspecific overlap in pollen use, and it was higher for abundant than for rare species. Such niche heterogeneity depends on individual differences in foraging behaviour and likely has implications for community dynamics and species stability. Our findings highlight the importance of taking interindividual variation into account when studying higher-order structures such as interaction networks. We argue that exploring individual-based networks will improve our understanding of species-based networks and will enhance the link between network analysis, foraging theory and evolutionary biology.},
author = {Tur, Cristina and Vigalondo, Beatriz and Tr{\o}jelsgaard, Kristian and Olesen, Jens M. and Traveset, Anna},
doi = {10.1111/1365-2656.12130},
file = {:Users/alyssacirtwill/Documents/Papers/Tur et al.{\_}2014{\_}Journal of Animal Ecology.pdf:pdf},
isbn = {1365-2656 (Electronic)$\backslash$r0021-8790 (Linking)},
issn = {00218790},
journal = {Journal of Animal Ecology},
keywords = {Ecology of individuals,Foraging behaviour,Generalization,Individual specialization,Individual-based networks,Linkage level,Niche overlap,Pollen load analysis,Resource partition,Species-based networks},
number = {1},
pages = {306--317},
pmid = {24107193},
title = {{Downscaling pollen-transport networks to the level of individuals}},
volume = {83},
year = {2014}
}
@article{Gomez2012,
abstract = {The relationships among the members of a population can be visualized using individual networks, where each individual is a node connected to each other by means of links describing the interactions. The centrality of a given node captures its importance within the network. We hypothesize that in mutualistic networks, the centrality of a node should benefit its fitness. We test this idea studying eight individual-based networks originated from the interaction between Erysimum mediohispanicum and its flower visitors. In these networks, each plant was considered a node and was connected to conspecifics sharing flower visitors. Centrality indicates how well connected is a given E. mediohispanicum individual with the rest of the co-occurring conspecifics because of sharing flower visitors. The centrality was estimated by three network metrics: betweenness, closeness and degree. The complex relationship between centrality, phenotype and fitness was explored by structural equation modelling. We found that the centrality of a plant was related to its fitness, with plants occupying central positions having higher fitness than those occupying peripheral positions. The structural equation models (SEMs) indicated that the centrality effect on fitness was not merely an effect of the abundance of visits and the species richness of visitors. Centrality has an effect even when simultaneously accounting for these predictors. The SEMs also indicated that the centrality effect on fitness was because of the specific phenotype of each plant, with attractive plants occupying central positions in networks, in relation to the distribution of conspecific phenotypes. This finding suggests that centrality, owing to its dependence on social interactions, may be an appropriate surrogate for the interacting phenotype of individuals.},
author = {G{\'{o}}mez, Jos{\'{e}} M. and Perfectti, Francisco},
doi = {10.1098/rspb.2011.2244},
file = {:Users/alyssacirtwill/Documents/Papers/G{\'{o}}mez, Perfectti{\_}2012{\_}Proceedings of the Royal Society B Biological Sciences.pdf:pdf},
isbn = {1471-2954 (Electronic)$\backslash$r0962-8452 (Linking)},
issn = {14712970},
journal = {Proceedings of the Royal Society B: Biological Sciences},
keywords = {Centrality,Individual generalization,Individual-based networks,Interacting phenotype,Plant fitness,Pollination},
number = {1734},
pages = {1754--1760},
pmid = {22158957},
title = {{Fitness consequences of centrality in mutualistic individual-based networks}},
volume = {279},
year = {2012}
}
@article{Zeman2018,
author = {Zeman, Michal and Molcan, Lubos and Sutovska, Hana and Krajcirovicova, Valentina and Stebelova, Katarina and Okuliarova, Monika},
doi = {10.1016/j.pathophys.2018.07.015},
issn = {09284680},
journal = {Pathophysiology},
month = {aug},
number = {3},
pages = {164},
publisher = {Macmillan Publishers Limited, part of Springer Nature. All rights reserved.},
title = {{Artificial Light At Night As a Risk Factor for Health}},
url = {https://doi.org/10.1038/nature23288 http://10.0.4.14/nature23288 https://www.nature.com/articles/nature23288{\#}supplementary-information},
volume = {25},
year = {2018}
}
@article{Giling2019,
abstract = {Changes in the diversity of plant communities may undermine the economically and environmentally important consumer species they support. The structure of trophic interactions determines the sensitivity of food webs to perturbations, but rigorous assessments of plant diversity effects on network topology are lacking. Here, we use highly resolved networks from a grassland biodiversity experiment to test how plant diversity affects the prevalence of different food web motifs, the smaller recurrent sub-networks that form the building blocks of complex networks. We find that the representation of tri-trophic chain, apparent competition and exploitative competition motifs increases with plant species richness, while the representation of omnivory motifs decreases. Moreover, plant species richness is associated with altered patterns of local interactions among arthropod consumers in which plants are not directly involved. These findings reveal novel structuring forces that plant diversity exerts on food webs with potential implications for the persistence and functioning of multitrophic communities.},
author = {Giling, Darren P. and Ebeling, Anne and Eisenhauer, Nico and Meyer, Sebastian T. and Roscher, Christiane and Rzanny, Michael and Voigt, Winfried and Weisser, Wolfgang W. and Hines, Jes},
doi = {10.1038/s41467-019-08856-0},
issn = {2041-1723},
journal = {Nature Communications},
number = {1},
pages = {1226},
publisher = {Springer US},
title = {{Plant diversity alters the representation of motifs in food webs}},
url = {http://www.nature.com/articles/s41467-019-08856-0},
volume = {10},
year = {2019}
}
@article{Bartley2019,
abstract = {Climate change is asymmetrically altering environmental conditions in space, from local to global scales, creating novel heterogeneity. Here, we argue that this novel heterogeneity will drive mobile generalist consumer species to rapidly respond through their behaviour in ways that broadly and predictably reorganize — or rewire — food webs. We use existing theory and data from diverse ecosystems to show that the rapid behavioural responses of generalists to climate change rewire food webs in two distinct and critical ways. First, mobile generalist species are redistributing into systems where they were previously absent and foraging on new prey, resulting in topological rewiring — a change in the patterning of food webs due to the addition or loss of connections. Second, mobile generalist species, which navigate between habitats and ecosystems to forage, will shift their relative use of differentially altered habitats and ecosystems, causing interaction strength rewiring — changes that reroute energy and carbon flows through existing food web connections and alter the food web's interaction strengths. We then show that many species with shared traits can exhibit unified aggregate behavioural responses to climate change, which may allow us to understand the rewiring of whole food webs. We end by arguing that generalists' responses present a powerful and underutilized approach to understanding and predicting the consequences of climate change and may serve as much-needed early warning signals for monitoring the looming impacts of global climate change on entire ecosystems.},
author = {Bartley, Timothy J. and McCann, Kevin S. and Bieg, Carling and Cazelles, Kevin and Granados, Monica and Guzzo, Matthew M. and MacDougall, Andrew S. and Tunney, Tyler D. and McMeans, Bailey C.},
doi = {10.1038/s41559-018-0772-3},
file = {:Users/alyssacirtwill/Documents/Papers/Bartley et al.{\_}2019{\_}Nature Ecology and Evolution.pdf:pdf},
issn = {2397334X},
journal = {Nature Ecology and Evolution},
keywords = {Behavioural ecology,Climate,Ecological networks,Ecology,change ecology},
number = {3},
pages = {345--354},
publisher = {Springer US},
title = {{Food web rewiring in a changing world}},
url = {http://www.nature.com/articles/s41559-018-0772-3},
volume = {3},
year = {2019}
}
@article{Henriksen2019,
abstract = {Abstract 1.The accurate estimation of interaction network structure is essential for understanding network stability and function. A growing number of studies evaluate under-sampling as the degree of sampling completeness (proportional richness observed). How the relationship between network structural metrics and sampling completeness varies across networks of different sizes remains unclear, but this relationship has implications for the within- and between-system comparability of network structure. 2.Here we test the combined effects of network size and sampling completeness on the structure of spatially distinct networks (i.e. subwebs) in a host-parasitoid model system to better understand the within-system variability in metric bias. 3.Richness estimates were used to quantify a gradient of sampling completeness of species and interactions across randomly subsampled subwebs. The combined impacts of network size and sampling completeness on the estimated values of twelve unweighted and weighted network metrics were tested. 4.The robustness of network metrics to under-sampling was strongly related to network size and sampling completeness of interactions were generally a better predictor of metric bias than sampling completeness of species. Weighted metrics often performed better than unweighted metrics at low sampling completeness, however, this was mainly evident at large rather than small subweb size. 5.These outcomes highlight the significance of under-sampling for the comparability of both unweighted and weighted network metrics when networks are small and vary in size. This has implications for within-system comparability of species-poor networks and, more generally, reveals problems with under-sampling ecological networks that may otherwise be difficult to detect in species-rich networks. To mitigate the impacts of under-sampling, more careful considerations of system-specific variation in metric bias is needed. This article is protected by copyright. All rights reserved.},
author = {Henriksen, Marie V. and Chapple, David G. and Chown, Steven L. and McGeoch, Melodie A.},
doi = {10.1111/1365-2656.12912},
file = {:Users/alyssacirtwill/Documents/Papers/Henriksen et al.{\_}2019{\_}Journal of Animal Ecology.pdf:pdf},
issn = {13652656},
journal = {Journal of Animal Ecology},
keywords = {consumer–resource interactions,metric bias,network structure,subweb size,under-sampling},
number = {2},
pages = {211--222},
title = {{The effect of network size and sampling completeness in depauperate networks}},
volume = {88},
year = {2019}
}
@article{Planque2014,
abstract = {A novel approach to model food-web dynamics, based on a combination of chance (randomness) and necessity (system constraints), was presented by Mullon et al. in 2009. Based on simulations for the Benguela ecosystem, they concluded that observed patterns of ecosystem variability may simply result from basic structural constraints within which the ecosystem functions. To date, and despite the importance of these conclusions, this work has received little attention. The objective of the present paper is to replicate this original model and evaluate the conclusions that were derived from its simulations. For this purpose, we revisit the equations and input parameters that form the structure of the original model and implement a comparable simulation model. We restate the model principles and provide a detailed account of the model structure, equations, and parameters. Our model can reproduce several ecosystem dynamic patterns: pseudo-cycles, variation and volatility, diet, stock-recruitment relationships, and correlations between species biomass series. The original conclusions are supported to a large extent by the current replication of the model. Model parameterisation and computational aspects remain difficult and these need to be investigated further. Hopefully, the present contribution will make this approach available to a larger research community and will promote the use of non-deterministic-network-dynamics models as 'null models of food-webs' as originally advocated.},
author = {Planque, Benjamin and Lindstr{\o}m, Ulf and Subbey, Sam},
doi = {10.1371/journal.pone.0108243},
file = {:Users/alyssacirtwill/Documents/Papers/Planque, Lindstr{\o}m, Subbey{\_}2014{\_}PLoS ONE.PDF:PDF},
issn = {19326203},
journal = {PLoS ONE},
number = {10},
title = {{Non-deterministic modelling of food-web dynamics}},
volume = {9},
year = {2014}
}
@article{Cirtwill2018EcolLett,
abstract = {Food webs and meso-scale motifs allow us to understand the structure of ecological communities and define species' roles within them. This species-level perspective on networks permits tests for relationships between species' traits and their patterns of direct and indirect interactions. Such relationships could allow us to predict food-web structure based on more easily obtained trait information. Here, we calculated the roles of species (as vectors of motif position frequencies) in six well-resolved marine food webs and identified the motif positions associated with the greatest variation in species' roles. We then tested whether the frequencies of these positions varied with species' traits. Despite the coarse-grained traits we used, our approach identified several strong associations between traits and motifs. Feeding environment was a key trait in our models and may shape species' roles by affecting encounter probabilities. Incorporating environment into future food-web models may improve predictions of an unknown network structure.},
author = {Cirtwill, Alyssa R. and Ekl{\"{o}}f, Anna},
doi = {10.1111/ele.12955},
file = {:Users/alyssacirtwill/Documents/Papers/Cirtwill, Ekl{\"{o}}f{\_}2018{\_}Ecology Letters.pdf:pdf},
issn = {14610248},
journal = {Ecology Letters},
keywords = {Apparent competition,body mass,direct competition,feeding environment,food chain,indirect interactions,trophic level},
number = {6},
pages = {875--884},
pmid = {29611282},
title = {{Feeding environment and other traits shape species' roles in marine food webs}},
url = {http://doi.wiley.com/10.1111/ele.12955},
volume = {21},
year = {2018}
}
@article{Engelkes2008,
abstract = {Many species are currently moving to higher latitudes and altitudes. However, little is known about the factors that influence the future performance of range-expanding species in their new habitats. Here we show that range-expanding plant species from a riverine area were better defended against shoot and root enemies than were related native plant species growing in the same area. We grew fifteen plant species with and without non-coevolved polyphagous locusts and cosmopolitan, polyphagous aphids. Contrary to our expectations, the locusts performed more poorly on the range-expanding plant species than on the congeneric native plant species, whereas the aphids showed no difference. The shoot herbivores reduced the biomass of the native plants more than they did that of the congeneric range expanders. Also, the range-expanding plants developed fewer pathogenic effects in their root-zone soil than did the related native species. Current predictions forecast biodiversity loss due to limitations in the ability of species to adjust to climate warming conditions in their range. Our results strongly suggest that the plants that shift ranges towards higher latitudes and altitudes may include potential invaders, as the successful range expanders may experience less control by above-ground or below-ground enemies than the natives.},
author = {Engelkes, Tim and Morri{\"{e}}n, Elly and Verhoeven, Koen J F and Bezemer, T. Martijn and Biere, Arjen and Harvey, Jeffrey A. and McIntyre, Lauren M. and Tamis, Wil L M and {Van Der Putten}, Wim H.},
doi = {10.1038/nature07474},
isbn = {0028-0836},
issn = {00280836},
journal = {Nature},
number = {7224},
pages = {946--948},
title = {{Successful range-expanding plants experience less above-ground and below-ground enemy impact}},
volume = {456},
year = {2008}
}
@article{Wootton1994a,
abstract = {KEY WORDS species interactions. competition. predation, mutualism. interaction. modifications Abstract Indirect effects occur when the impact of one species on another requires the presence of a third species. They can arise in two general ways: through linked chains of direct interactions, and when a species changes the interactions among species. Indirect effects have been uncovered largely by experimental studies that have monitored the response of many species and discovered "unexpected results," although some studies have looked for specific indirect effects predicted from simple models. The characteristics of such approaches make it likely that the many indirect effects remain uncovered, but the appli­ cation of techniques such as path analysis may reduce this problem. Determin­ istic theory indicates that indirect effects should often be important, although stochastic models need exploration. Simulation models indicate that some indirect effects may stabilize multi-species assemblages. Five simple types of indirect effects have been regularly demonstrated in nature: exploitative com­ petition, trophic cascades, apparent competition, indirect mutualism, and in­ teraction modifications. Detailed experimental investigations of natural com­ munities have yielded complicated effects. Indirect effects have the potential to affect evolutionary patterns, but empirical examples are limited. Future directions in the study of indirect effects include developing techniques to estimate interaction strength in dynamic models, deriving more efficient ap­ proaches to detecting indirect effects, evaluating the effectiveness of ap-443 0066-4162/9411120-0443{\$}05.00},
archivePrefix = {arXiv},
arxivId = {0066-4162/9411120-0443},
author = {Wootton, J T},
doi = {10.1146/annurev.es.25.110194.002303},
eprint = {9411120-0443},
isbn = {0066-4162},
issn = {0066-4162},
journal = {Annual Review of Ecology and Systematics},
number = {1},
pages = {443--466},
pmid = {525},
primaryClass = {0066-4162},
title = {{The Nature and Consequences of Indirect Effects in Ecological Communities}},
volume = {25},
year = {2003}
}
@article{Brose2010,
abstract = {1. In community and population ecology, there is a chronic gap between the classic Eltonian ecology describing patterns in abundance and body mass across species and ecosystems and the more process oriented foraging ecology addressing interactions and quantitative population dynamics. However, this dichotomy is arbitrary, because body mass also determines most species traits affecting foraging interactions and population dynamics. 2. In this review, allometric (body-mass dependent) scaling of handling times and attack rates are documented, whereas body-mass effects on Hill exponents (varying between hyperbolic type II and sigmoid type III functional responses) and predator interference coefficients are lacking. This review describes how these allometric relationships define a biological plausible parameter space for population dynamic models. 3. Consistent with the classic Eltonian description, species co-existence in consumer-resource models and tri-trophic food chains is restricted to intermediate consumer-resource body-mass ratios. Allometric population dynamic models allow understanding the processes of energy limitation and unstable dynamics leading to this restriction. Complex food webs are stabilized by high predator-prey body-mass ratios, which are consistent with those found in natural ecosystems. These high body-mass ratios yield positive diversity-stability and complexity-stability relationships thus supporting the classic picture of ecosystem stability. 4. Allometric-trophic network models, based on body mass and trophic information from ecosystems, bridge the gap between Eltonian community patterns and process-oriented foraging ecology and provide a new means to describe the dynamics and functioning of natural ecosystems.},
author = {Brose, Ulrich},
doi = {10.1111/j.1365-2435.2009.01618.x},
file = {:Users/alyssacirtwill/Documents/Papers/Brose{\_}2010{\_}Functional Ecology.pdf:pdf},
isbn = {0269-8463},
issn = {02698463},
journal = {Functional Ecology},
keywords = {Allometric scaling,Attack rate,Food chain,Functional response,Handling time,Metabolic theory,Stability},
month = {feb},
number = {1},
pages = {28--34},
title = {{Body-mass constraints on foraging behaviour determine population and food-web dynamics}},
url = {http://doi.wiley.com/10.1111/j.1365-2435.2009.01618.x},
volume = {24},
year = {2010}
}
@article{Poisot2016,
abstract = {* There is a growing realization among community ecologists that interactions between species vary across space and time and that this variation needs to be quantified. Our current numerical framework to analyse the structure of species interactions, based on graph-theoretical approaches, usually do not consider the variability of interactions. As this variability has been show to hold valuable ecological information, there is a need to adapt the current measures of network structure so that they can exploit it. * We present analytical expressions of key measures of network structured, adapted so that they account for the variability of ecological interactions. We do so by modelling each interaction as a Bernoulli event; using basic calculus allows expressing the expected value, and when mathematically tractable, its variance. When applied to non-probabilistic data, the measures we present give the same results as their non-probabilistic formulations, meaning that they can be generally applied. * We present three case studies that highlight how these measures can be used, in re-analysing data that experimentally measured the variability of interactions, to alleviate the computational demands of permutation-based approaches, and to use the frequency at which interactions are observed over several locations to infer the structure of local networks. We provide a free and open-source implementation of these measures. * We discuss how both sampling and data representation of ecological networks can be adapted to allow the application of a fully probabilistic numerical network approach.},
author = {Poisot, Timoth{\'{e}}e and Cirtwill, Alyssa R. and Cazelles, K{\'{e}}vin and Gravel, Dominique and Fortin, Marie Jos{\'{e}}e and Stouffer, Daniel B.},
doi = {10.1111/2041-210X.12468},
file = {:Users/alyssacirtwill/Documents/Papers/Poisot et al.{\_}2016{\_}Methods in Ecology and Evolution.pdf:pdf},
isbn = {2041-210X},
issn = {2041210X},
journal = {Methods in Ecology and Evolution},
keywords = {Connectance,Degree distribution,Ecological networks,Modularity,Nestedness,Species interactions},
number = {3},
pages = {303--312},
title = {{The structure of probabilistic networks}},
volume = {7},
year = {2016}
}
@misc{Gounand2018,
abstract = {The meta-ecosystem framework demonstrates the significance of among-ecosystem spatial flows for ecosystem dynamics and has fostered a rich body of theory. The high level of abstraction of the models, however, impedes applications to empirical systems. We argue that further understanding of spatial dynamics in natural systems strongly depends on dense exchanges between field and theory. From empiricists, more and specific quantifications of spatial flows are needed, defined by the major categories of organismal movement (dispersal, foraging, life-cycle, and migration). In parallel, the theoretical framework must account for the distinct spatial scales at which these naturally common spatial flows occur. Integrating all levels of spatial connections among landscape elements will upgrade and unify landscape and meta-ecosystem ecology into a single framework for spatial ecology. Cross-ecosystem movements drive landscape dynamics. Among-ecosystem couplings are either dispersal- or resource-dominated. Dispersal-based couplings occur at the regional scale between similar habitat types. Resource-based couplings occur at the local scale between different habitat types.},
author = {Gounand, Isabelle and Harvey, Eric and Little, Chelsea J. and Altermatt, Florian},
booktitle = {Trends in Ecology and Evolution},
doi = {10.1016/j.tree.2017.10.006},
file = {:Users/alyssacirtwill/Documents/Papers/Gounand et al.{\_}2018{\_}Trends in Ecology and Evolution.pdf:pdf},
isbn = {1872-8383 (Electronic) 0169-5347 (Linking)},
issn = {01695347},
keywords = {animal movement,landscape,meta-ecosystem,metacommunity,resource flow,subsidy},
number = {1},
pages = {36--46},
pmid = {29102408},
title = {{Meta-Ecosystems 2.0: Rooting the Theory into the Field}},
volume = {33},
year = {2018}
}
@techreport{Spiller2010,
abstract = {The effect of resource subsidies on recipient food webs has received much recent attention. The purpose of this study was to measure the effects of significant seasonal seaweed deposition events, caused by hurricanes and other storms, on species inhabiting subtropical islands. The seaweed represents a pulsed resource subsidy that is consumed by amphipods and flies, which are eaten by lizards and predatory arthropods, which in turn consume terrestrial herbivores. Additionally, seaweed decomposes directly into the soil under plants. We added seaweed to six shoreline plots and removed seaweed from six other plots for three months; all plots were repeatedly monitored for 12 months after the initial manipulation. Lizard density (Anolis sagrei) responded rapidly, and the overall average was 63{\%} higher in subsidized than in removal plots. Stable-isotope analysis revealed a shift in lizard diet composition toward more marine-based prey in subsidized plots. Leaf damage was 70{\%} higher in subsidized than in removal plots after eight months, but subsequent damage was about the same in the two treatments. Foliage growth rate was 70{\%} higher in subsidized plots after 12 months. Results of a complementary study on the relationship between natural variation in marine subsidies and island food web components were consistent with the experimental results. We suggest two causal pathways for the effects of marine subsidies on terrestrial plants: (1) the "fertilization effect" in which seaweed adds nutrients to plants, increasing their growth rate, and (2) the "predator diet shift effect" in which lizards shift from eating local prey (including terrestrial herbivores) to eating mostly marine detritivores.},
author = {Avenue, One Shields and Avenue, One Shields and Spiller, David a. and Piovia-scott, Jonah and Wright, Amber N. and Yang, Louie H. and Schoener, Thomas W. and Iwata, Tomoya and Wright, N and Takimoto, Gaku and Schoener, Thomas W. and Iwata, Tomoya},
booktitle = {Ecology},
doi = {10.1890/09-0715.1},
file = {::},
isbn = {0012-9658},
issn = {00129658},
keywords = {Allochthonous inputs,Anolis sagrei,Apparent competition,Apparent mutualism,Central bahamas,Conocarpus erectus,Exumas,Food webs,Herbivory,Lizards,Marine subsidies,Resource pulse,Seaweed,allochthonous inputs,anolis sagrei,apparent competition,apparent mutualism,central bahamas,conocarpus erectus,exumas,food webs,herbivory,lizards,marine subsidies,resource},
number = {5},
pages = {1424--1434},
pmid = {20503874},
title = {{Marine subsidies have multiple effects on coastal food webs}},
volume = {91},
year = {2010}
}
@misc{Gravel2018,
abstract = {Biogeography has traditionally focused on the spatial distribution and abundance of species. Both are driven by the way species interact with one another, but also by the way these interactions vary across time and space. Here, we call for an integrated approach, adopting the view that community structure is best represented as a network of ecological interactions, and show how it translates to biogeography questions. We propose that the ecological niche should encompass the effect of the environment on species distribution (the Grinnellian dimension of the niche) and on the ecological interactions among them (the Eltonian dimension). Starting from this concept, we develop a quantitative theory to explain turnover of interactions in space and time i.e. a novel approach to interaction distribution modelling. We apply this framework to host parasite interactions across Europe and find that two aspects of the environment (temperature and precipitation) exert a strong imprint on species co-occurrence, but not on species interactions. Even where species co-occur, interaction proves to be stochastic rather than deterministic, adding to variation in realized network structure. We also find that a large majority of host-parasite pairs are never found together, thus precluding any inferences regarding their probability to interact. This first attempt to explain variation of network structure at large spatial scales opens new perspectives at the interface of species distribution modelling and community ecology.},
archivePrefix = {arXiv},
arxivId = {arXiv:1011.1669v3},
author = {Gravel, Dominique and Baiser, Benjamin and Dunne, Jennifer A. and Kopelke, Jens Peter and Martinez, Neo D. and Nyman, Tommi and Poisot, Timoth{\'{e}}e and Stouffer, Daniel B. and Tylianakis, Jason M. and Wood, Spencer A. and Roslin, Tomas},
booktitle = {Ecography},
doi = {10.1111/ecog.04006},
eprint = {arXiv:1011.1669v3},
file = {:Users/alyssacirtwill/Documents/Papers/Gravel et al.{\_}2019{\_}Ecography.pdf:pdf},
isbn = {9780874216561},
issn = {16000587},
keywords = {co-occurrence,networks,spatial ecology},
number = {3},
pages = {401--415},
pmid = {15991970},
title = {{Bringing Elton and Grinnell together: a quantitative framework to represent the biogeography of ecological interaction networks}},
volume = {42},
year = {2019}
}
@article{Desjardins-Proulx2017,
abstract = {Species interactions are a key component of ecosystems but we generally have an incomplete picture of who-eats-who in a given community. Different techniques have been devised to predict species interactions using theoretical models or abundances. Here, we explore the K nearest neighbour approach, with a special emphasis on recommendation, along with a supervised machine learning technique. Recommenders are algorithms developed for companies like Netflix to predict whether a customer will like a product given the preferences of similar customers. These machine learning techniques are well-suited to study binary ecological interactions since they focus on positive-only data. By removing a prey from a predator, we find that recommenders can guess the missing prey around 50{\%} of the times on the first try, with up to 881 possibilities. Traits do not improve significantly the results for the K nearest neighbour, although a simple test with a supervised learning approach (random forests) show we can predict interactions with high accuracy using only three traits per species. This result shows that binary interactions can be predicted without regard to the ecological community given only three variables: body mass and two variables for the species' phylogeny. These techniques are complementary, as recommenders can predict interactions in the absence of traits, using only information about other species' interactions, while supervised learning algorithms such as random forests base their predictions on traits only but do not exploit other species' interactions. Further work should focus on developing custom similarity measures specialized for ecology to improve the KNN algorithms and using richer data to capture indirect relationships between species.},
author = {Desjardins-Proulx, Philippe and Laigle, Idaline and Poisot, Timoth{\'{e}}e and Gravel, Dominique},
doi = {10.7717/peerj.3644},
file = {:Users/alyssacirtwill/Documents/Papers/Desjardins-Proulx et al.{\_}2017{\_}PeerJ.pdf:pdf},
issn = {2167-8359},
journal = {PeerJ},
pages = {e3644},
pmid = {28828250},
title = {{Ecological interactions and the Netflix problem}},
volume = {5},
year = {2017}
}
@inproceedings{Hamback2003,
abstract = {This review discusses the prevalence and potential for interactive effects between herbivory and competition on plant growth and biomass. and it is apparent that such effects typically arise when there is a mismatch between the spatial scale of herbivore behaviour (food or patch choice) and the spatial heterogeneity of the plant community. Historically, such interactive effects have been examined using two approaches. Studies using the first approach have excluded plant neighbors and herbivores in a factorial experiment, and scored effects on plant biomass. Studies using the second approach have observed herbivore abundance or herbivory on plants with or without plant neighbors, and have identified a large number of mechanisms underlying such interactive effects. The two types of studies have produced somewhat conflicting results, where interactive effects have been commonly observed in studies using the second approach and only rarely in studies using the first approach. This is most likely a consequence of a biased choice of study systems. where studies using the first approach have primarily Studied mammalian herbivory while studies using the second approach have been more focussed on insect herbivory. Moreover, studies using the first approach have typically been very small-scale manipulations and this probably precludes most possible interactive effects in systems with mammalian herbivory. This points to the fact that studies examining interactive effects of herbivory and plant competition should more carefully consider the behaviour and life history of herbivores included in the study prior to the design of removal experiments.},
author = {Hamb{\"{a}}ck, Peter A. and Beckerman, Andrew P.},
booktitle = {Oikos},
doi = {10.1034/j.1600-0706.2003.12568.x},
file = {:Users/alyssacirtwill/Documents/Papers/Hamb{\"{a}}ck, Beckerman{\_}2003{\_}Oikos.pdf:pdf},
isbn = {0030-1299},
issn = {00301299},
number = {1},
pages = {26--37},
pmid = {182106600004},
title = {{Herbivory and plant resource competition: A review of two interacting interactions}},
volume = {101},
year = {2003}
}
@article{Menge2018,
abstract = {Scientific communication relies on clear presentation of data. Logarithmic scales are used frequently for data presentation in many scientific disciplines, including ecology, but the degree to which they are correctly interpreted by readers is unclear. Analysing the extent of log scales in the literature, we show that 22{\%} of papers published in the journal Ecology in 2015 included at least one log-scaled axis, of which 21{\%} were log–log displays. We conducted a survey that asked members of the Ecological Society of America (988 responses, and 623 completed surveys) to interpret graphs that were randomly displayed with linear–linear or log–log axes. Many more respondents interpreted graphs correctly when the graphs had linear–linear axes than when they had log–log axes: 93{\%} versus 56{\%} for our all-around metric, although some of the individual item comparisons were even more skewed (for example, 86{\%} versus 9{\%} and 88{\%} versus 12{\%}). These results suggest that misconceptions about log-scaled data are rampant. We recommend that ecology curricula include explicit instruction on how to interpret log-scaled axes and equations, and we also recommend that authors take the potential for misconceptions into account when deciding how to visualize data.},
author = {Menge, Duncan N.L. and MacPherson, Anna C. and Bytnerowicz, Thomas A. and Quebbeman, Andrew W. and Schwartz, Naomi B. and Taylor, Benton N. and Wolf, Amelia A.},
doi = {10.1038/s41559-018-0610-7},
file = {:Users/alyssacirtwill/Documents/Papers/Menge et al.{\_}2018{\_}Nature Ecology and Evolution.pdf:pdf},
issn = {2397334X},
journal = {Nature Ecology and Evolution},
number = {9},
pages = {1393--1402},
title = {{Logarithmic scales in ecological data presentation may cause misinterpretation}},
volume = {2},
year = {2018}
}
@article{Clare2018,
author = {Clare, Elizabeth L. and Fazekas, Aron J. and Ivanova, Natalia V. and Floyd, Robin M. and Hebert, Paul D.N. and Adams, Amanda M. and Nagel, Juliet and Girton, Rebecca and Newmaster, Steven G. and Fenton, M. Brock},
doi = {10.1111/mec.14941},
file = {:Users/alyssacirtwill/Documents/Papers/Clare et al.{\_}2018{\_}Molecular Ecology.pdf:pdf},
issn = {1365294X},
journal = {Molecular Ecology},
keywords = {DNA barcoding,bats,food webs,high-throughput sequencing,interaction networks,metabarcoding},
number = {2},
pages = {503--519},
title = {{Approaches to integrating genetic data into ecological networks}},
volume = {28},
year = {2019}
}
@article{Deagle2018,
abstract = {Advances in DNA sequencing technology have revolutionised the field of molecular analysis of trophic interactions and it is now possible to recover counts of food DNA barcode sequences from a wide range of dietary samples. But what do these counts mean? To obtain an accurate estimate of the overall diet of a consumer should we work strictly with datasets summarising the frequency of occurrence of different food taxa, or is it possible to use the relative number of sequences? Both approaches are applied in the dietary metabarcoding literature, but occurrence data is often promoted as a more conservative and reliable option due to taxa-specific biases in recovery of sequences. Here, we point out that diet summaries based on occurrence data overestimate the importance of food consumed in small quantities (potentially including low-level contaminants) and are sensitive to the count threshold used to define an occurrence. Our simulations indicate that even with recovery biases incorporated, using relative read abundance (RRA) information can provide a more accurate view of population-level diet in many scenarios. The ideas presented here highlight the need to consider all sources of bias and to justify the methods used to interpret count data in dietary metabarcoding studies. We encourage researchers to continue to addressing methodological challenges, and acknowledge unanswered questions to help spur future investigations in this rapidly developing area of research.},
author = {Deagle, Bruce E. and Thomas, Austen C. and McInnes, Julie C. and Clarke, Laurence J. and Vesterinen, Eero J. and Clare, Elizabeth L. and Kartzinel, Tyler R. and Eveson, J. Paige},
doi = {10.1111/mec.14734},
file = {::},
isbn = {9780080969862},
issn = {1365294X},
journal = {Molecular Ecology},
number = {2},
pages = {391--406},
pmid = {29858539},
title = {{Counting with DNA in metabarcoding studies: How should we convert sequence reads to dietary data?}},
volume = {28},
year = {2019}
}
@article{Roslin2018,
abstract = {In the last paragraph of the Origin of Species, Darwin (1859) marvels at the diversity of life forms, the complexity of links between them and the forces creating this “tangled bank.” In this text, we may see the origins of community ecology—today defined as “the study of the interactions that determine the distribution and abundance of organisms” (Krebs, 2009). To capture and quantify the key elements of this concept of community structure, we may conveniently describe communities as ecological networks (Hagen et al., 2012). In such networks, the nodes are formed by species (or other taxonomic units) and the links by their interactions (Gravel et al., 2018). Modern molecular methods offer unique opportunities for describing both elements of community structure (Roslin {\&} Majaneva, 2016) and how they change in time and space (“community dynamics”).},
author = {Roslin, Tomas and Traugott, Michael and Jonsson, Mattias and Stone, Graham N. and Creer, Simon and Symondson, William O.C.},
doi = {10.1111/mec.14974},
file = {:Users/alyssacirtwill/Documents/Papers/Roslin et al.{\_}2018{\_}Molecular Ecology.pdf:pdf},
issn = {1365294X},
journal = {Molecular Ecology},
keywords = {antagonistic interactions,assembly processes,community ecology,ecological interaction networks,food webs,mutualistic interactions,species interactions},
month = {dec},
number = {2},
pages = {157--164},
title = {{Introduction: Special issue on species interactions, ecological networks and community dynamics – Untangling the entangled bank using molecular techniques}},
url = {https://onlinelibrary.wiley.com/doi/abs/10.1111/mec.14974},
volume = {28},
year = {2019}
}
@misc{Mata2018,
abstract = {DNA metabarcoding is increasingly used in dietary studies to estimate diversity, composition and frequency of occurrence of prey items. However, few studies have assessed how technical and biological replication affect the accuracy of diet estimates. This study addresses these issues using the European free-tailed bat Tadarida teniotis, involving high-throughput sequencing of a small fragment of the COI gene in 15 separate faecal pellets and a 15-pellet pool per each of 20 bats. We investigated how diet descriptors were affected by variability among (a) individuals, (b) pellets of each individual and (c) PCRs of each pellet. In addition, we investigated the impact of (d) analysing separate pellets vs. pellet pools. We found that diet diversity estimates increased steadily with the number of pellets analysed per individual, with seven pellets required to detect {\~{}}80{\%} of prey species. Most variation in diet composition was associated with differences among individual bats, followed by pellets per individual and PCRs per pellet. The accuracy of frequency of occurrence estimates increased with the number of pellets analysed per bat, with the highest error rates recorded for prey consumed infrequently by many individuals. Pools provided poor estimates of diet diversity and frequency of occurrence, which were comparable to analysing a single pellet per individual, and consistently missed the less common prey items. Overall, our results stress that maximizing biological replication is critical in dietary metabarcoding studies and emphasize that analysing several samples per individual rather than pooled samples produce more accurate results.},
author = {Mata, Vanessa A. and Rebelo, Hugo and Amorim, Francisco and Mccracken, Gary F. and Jarman, Simon and Beja, Pedro},
booktitle = {Molecular Ecology},
doi = {10.1111/mec.14779},
file = {:Users/alyssacirtwill/Documents/Papers/Archive/Mata et al.{\_}2018{\_}Molecular Ecology.pdf:pdf},
issn = {1365294X},
keywords = {Bat ecology,Metabarcoding,Molecular diet analyses,Replication,Sampling design,Trophic ecology},
pmid = {29940083},
title = {{How much is enough? Effects of technical and biological replication on metabarcoding dietary analysis}},
year = {2018}
}
@misc{Leyland2005a,
abstract = {This paper reviews empirical Bayes methods for disease mapping. A distinction is made between spatial models (which take into account the geographical distribution of disease) and nonspatial models. Several estimators are presented, and methods of estimation are described. Empirical Bayes methods are compared with full Bayes methods, and we argue that both have their place.},
author = {Leyland, Alastair H. and Davies, Carolyn A.},
booktitle = {Statistical Methods in Medical Research},
doi = {10.1191/0962280205sm387oa},
file = {:Users/alyssacirtwill/Documents/Papers/Leyland, Davies{\_}2005{\_}Statistical Methods in Medical Research.pdf:pdf},
isbn = {0962-2802 (Print)},
issn = {09622802},
pmid = {15690998},
title = {{Empirical Bayes methods for disease mapping}},
year = {2005}
}
@article{Carstensen2014,
abstract = {Interactions between species form complex networks that vary across space and time. Even without spatial or temporal constraints mutualistic pairwise interactions may vary, or rewire, across space but this variability is not well understood. Here, we quantify the beta diversity of species and interactions and test factors influencing the probability of turnover of pairwise interactions across space. We ask: 1) whether beta diversity of plants, pollinators, and interactions follow a similar trend across space, and 2) which interaction properties and site characteristics are related to the probability of turnover of pairwise interactions. Geographical distance was positively correlated with plant and interaction beta diversity. We find that locally frequent interactions are more consistent across space and that local flower abundance is important for the realization of pairwise interactions. While the identity of pairwise interactions is highly variable across space, some species-pairs form interactions that are locally frequent and spatially consistent. Such interactions represent cornerstones of interacting communities and deserve special attention from ecologists and conservation planners alike.},
author = {Carstensen, Daniel W. and Sabatino, Malena and Tr{\o}jelsgaard, Kristian and Morellato, Leonor Patricia C.},
doi = {10.1371/journal.pone.0112903},
file = {:Users/alyssacirtwill/Documents/Papers/Carstensen et al.{\_}2014{\_}PLoS ONE.pdf:pdf},
isbn = {1932-6203},
issn = {19326203},
journal = {PLoS ONE},
pmid = {25384058},
title = {{Beta diversity of plant-pollinator networks and the spatial turnover of pairwise interactions}},
year = {2014}
}
@article{Bartomeus2013a,
abstract = {The analysis of mutualistic networks has become a central tool in answering theoretical and applied questions regarding our understanding of ecological processes. Significant gaps in knowledge do however need to be bridged in order to effectively and accurately be able to describe networks. Main concern are the incorporation of species level information, accounting for sampling limitations and understanding linkage rules. Here I propose a simple method to combine plant pollinator effort-limited sampling with information about plant community to gain understanding of what drives linkage rules, while accounting for possible undetected linkages. I use hierarchical models to estimate the probability of detection of each plant-pollinator interaction in 12 Mediterranean plant-pollinator networks. As it is possible to incorporate plant traits as co-variables in the models, this method has the potential to be used for predictive purposes, such as identifying undetected links among existing species, as well as potential interactions with new plant species. Results show that pollinator detectability is very skewed and usually low. Nevertheless, 84{\%} of the models are enhanced by the inclusion of co-variables, with flower abundance and inflorescence type being the most commonly retained co-variables. The predicted networks increase network Connectance by 13{\%}, but not Nestedness, which is known to be robust to sampling effects. However, 46{\%} of the pollinator interactions in the studied networks comprised a single observation and hence could not be modeled. The hierarchical modeling approach suggested here is highly flexible and can be used on binary or frequency networks, accommodate different observers or include collection day weather variables as confounding factors. An R script is provided for a rapid adoption of this method.},
author = {Bartomeus, Ignasi},
doi = {10.1371/journal.pone.0069200},
file = {:Users/alyssacirtwill/Documents/Papers/Bartomeus{\_}2013{\_}PLoS ONE.pdf:pdf},
isbn = {1932-6203},
issn = {19326203},
journal = {PLoS ONE},
keywords = {Connectance,Linkage density,Nestedness,hierarchical models,occupancy models,pollination web,specialization},
pmid = {23840909},
title = {{Understanding Linkage Rules in Plant-Pollinator Networks by Using Hierarchical Models That Incorporate Pollinator Detectability and Plant Traits}},
year = {2013}
}
@article{Copas1972a,
author = {Methods, Empirical Bayes},
file = {::},
journal = {Biometrika},
number = {2},
pages = {349--360},
title = {{and the repeated use of a standard Empirical Bayes methods}},
volume = {59},
year = {2013}
}
@article{Preston2014,
abstract = {Most food webs use taxonomic or trophic species as building blocks, thereby collapsing variability in feeding linkages that occurs during the growth and development of individuals. This issue is particularly relevant to integrating parasites into food webs because parasites often undergo extreme ontogenetic niche shifts. Here, we used three versions of a freshwater pond food web with varying levels of node resolution (from taxonomic species to life stages) to examine how complex life cycles and parasites alter web properties, the perceived trophic position of organisms, and the fit of a probabilistic niche model. Consistent with prior studies, parasites increased most measures of web complexity in the taxonomic species web; however, when nodes were disaggregated into life stages, the effects of parasites on several network properties (e.g., connectance and nestedness) were reversed, due in part to the lower trophic generality of parasite life stages relative to free-living life stages. Disaggregation also reduced the trophic level of organisms with either complex or direct life cycles and was particularly useful when including predation on parasites, which can inflate trophic positions when life stages are collapsed. Contrary to predictions, disaggregation decreased network intervality and did not enhance the fit of a probabilistic niche model to the food webs with parasites. Although the most useful level of biological organization in food webs will vary with the questions of interest, our results suggest that disaggregating species-level nodes may refine our perception of how parasites and other complex life cycle organisms influence ecological networks.},
author = {Preston, Daniel L. and Jacobs, Abigail Z. and Orlofske, Sarah A. and Johnson, Pieter T J},
doi = {10.1007/s00442-013-2806-5},
file = {:Users/alyssacirtwill/Documents/Papers/Preston et al.{\_}2014{\_}Oecologia.pdf:pdf},
isbn = {1432-1939 (Electronic)$\backslash$r0029-8549 (Linking)},
issn = {00298549},
journal = {Oecologia},
keywords = {Community ecology,Food web model,Host-parasite interaction,Topology,Wetland},
pmid = {24258100},
title = {{Complex life cycles in a pond food web: Effects of life stage structure and parasites on network properties, trophic positions and the fit of a probabilistic niche model}},
year = {2014}
}
@article{Verschut2018,
abstract = {{\textless}section class="article-section article-section{\_}{\_}abstract" lang="en" data-lang="en" id="section-1-en"{\textgreater} {\textless}h3 class="article-section{\_}{\_}header main main"{\textgreater}Abstract{\textless}/h3{\textgreater} {\textless}div class="article-section{\_}{\_}content en main"{\textgreater} {\textless}p{\textgreater}Terrestrial predators on marine shores benefit from the inflow of organisms and matter from the marine ecosystem, often causing very high predator densities and indirectly affecting the abundance of other prey species on shores. This indirect effect may be particularly strong if predators shift diets between seasons. We therefore quantified the seasonal variation in diet of two wolf spider species that dominate the shoreline predator community, using molecular gut content analyses with general primers to detect the full prey range. Across the season, spider diets changed, with predominantly terrestrial prey from May until July and predominantly marine prey (mainly chironomids) from August until October. This pattern coincided with a change in the spider age and size structure, and prey abundance data and resource selection analyses suggest that the higher consumption of chironomids during autumn is due to an ontogenetic diet shift rather than to variation in prey abundance. The analyses suggested that small dipterans with a weak flight capacity, such as Chironomidae, Sphaeroceridae, Scatopsidae and Ephydridae, were overrepresented in the gut of small juvenile spiders during autumn, whereas larger, more robust prey, such as Lepidoptera, Anthomyidae and Dolichopodidae, were overrepresented in the diet of adult spiders during spring. The effect of this inflow may be that the survival and growth of juvenile spiders is higher in areas with high chironomid abundances, leading to higher densities of adult spiders and higher predation rates on the terrestrial prey next spring. {\textless}/p{\textgreater} {\textless}p{\textgreater}This article is protected by copyright. All rights reserved.{\textless}/p{\textgreater}},
author = {Verschut, Vasiliki and Strandmark, Alma and Esparza-Salas, Rodrigo and Hamb{\"{a}}ck, Peter A.},
doi = {10.1111/mec.14830},
file = {:Users/alyssacirtwill/Documents/Papers/Verschut et al.{\_}2018{\_}Molecular Ecology.pdf:pdf},
issn = {1365294X},
journal = {Molecular Ecology},
keywords = {DNA metabarcoding,Pardosa,marine inflow,molecular gut content analysis},
number = {2},
pages = {307--317},
title = {{Seasonally varying marine influences on the coastal ecosystem detected through molecular gut analysis}},
volume = {28},
year = {2019}
}
@article{Stouffer2005,
abstract = {We analyze the properties of model food webs and of fifteen community food webs from a variety of environments. We first perform a theoretical analysis of the niche model of Williams and Martinez. We derive analytical expressions for the distributions of species' number of prey, number of predators, and total number of trophic links and find that they follow universal functional forms. We also derive expressions for a number of other biologically relevant parameters, including the fraction of top, intermediate, basal, and cannibal species, the standard deviations of generality and vulnerability, the correlation coefficient between species' number of prey and number of predators, and assortativity. We show that our findings are robust under rather general conditions. We then use our analytical predictions as a guide to the analysis of fifteen of the most complete empirical food webs available. We uncover quantitative unifying patterns that describe the properties of the model food webs and most of the trophic webs considered. Our results support a strong new hypothesis that the empirical distributions of number of prey and number of predators follow universal functional forms that, without free parameters, match our analytical predictions. Further, we find that the empirically observed correlation coefficient, assortativity, and fraction of cannibal species are consistent with our analytical expressions and simulations of the niche model. Finally, we show that the average distance between nodes and the average clustering coefficient show a high degree of regularity for both the empirical data and simulations of the niche model. Our findings suggest that statistical physics concepts such as scaling and universality may be useful in the description of natural ecosystems.},
archivePrefix = {arXiv},
arxivId = {q-bio/0401023},
author = {Stouffer, D. B. and Camacho, J. and Guimer{\`{a}}, R. and Ng, C. A. and {Nunes Amaral}, L. A.},
doi = {10.1890/04-0957},
eprint = {0401023},
isbn = {0012-9658},
issn = {00129658},
journal = {Ecology},
keywords = {Complex networks,Food webs, model vs. empirical,Food-web patterns,Food-web structure,Network structure,Scaling,Universality},
month = {may},
number = {5},
pages = {1301--1311},
pmid = {228960000025},
primaryClass = {q-bio},
title = {{Quantitative patterns in the structure of model and empirical food webs}},
url = {http://www.esajournals.org/doi/pdf/10.1890/04-0957 http://www.esajournals.org/doi/abs/10.1890/04-0957},
volume = {86},
year = {2005}
}
@article{Markager1992,
abstract = {Light compensation points (I(c)) for growth were low (0.3 to 2.5 mumol m-2 s-1) for the temperate marine macroalgae Chondrus crispus, Fucus serratus, Petalonia fascia, Porphyra purpurea and Ulva lactuca measured at 7-degrees-C. These I(c)-values corresponded to those estimated by a physiological model including light absorption and quantum yield for growth to describe carbon gain, and weight specific dark respiration, dark loss rate and thallus specific carbon (mol C m-2 thallus) to describe carbon loss. Absorption and quantum yield were close to the theoretical maximum for all species and could not explain differences in I(c). Respiration and thallus specific carbon varied more than 15-fold and were the main factors responsible for variations in I(c). Experimental I(c)-values correspond to 0.12 to 0.61 {\%} of the yearly surface light dose in Denmark (56-degrees-N). These values agree with the {\%} of surface light ({\%}SI) available at the depth limits of leathery and foliose macroalgae at different latitudes. Hence, there is no surplus of energy to balance grazing and mechanical losses, and these factors must be of minor importance for macroalgae growing at great depths. A literature review of depth limits for marine macroalgae reveals an upper zone of mainly leathery algae with depth limits of about 0.5 {\%} SI, an intermediate zone of foliose and delicate algae with depth limits at about 0.10 {\%} SI, and a lower zone of encrusted algae extending down to about 0.01 {\%} SI. This zonation pattern is accompanied by a decrease in thallus specific carbon (i.e. thinner thalli) with increasing depth. The inverse relationship between growth rate at low light and thallus specific carbon suggests that a thin thallus is essential for growth and survival of marine macroalgae at great depths.},
author = {Markager, S. and Sand-Jensen, K.},
doi = {10.3354/meps088083},
isbn = {0171-8630},
issn = {01718630},
journal = {Marine Ecology Progress Series},
number = {1},
pages = {83--92},
pmid = {1081},
title = {{Light requirements and depth zonation of marine macroalgae}},
volume = {88},
year = {1992}
}
@article{Howarth1988,
abstract = {The question of nutrient limitation of primary production in estuaries and other marine ecosystems has engendered agreat deal of debate. Although other marine ecosystems has engendered a great deal of debate. Although nitrogen is often named as the primary limiting nutrient in seawater, this is by no means universally accepted. Many workers have argued that phosphorus is limiting, that both nitrogen and phosphorus can simultaneously be limiting, or that primary production can switch seasonally from being nitrogen-limited to phosphorus-limited. Others argue that nutrients are not limiting at all nmany marine ecosystems, including highly oligotrophic waters. To some extent these disagreements result from poor communication due to different definitions of nutrient limitation. Considerable argument also occurs over the various methods and approaches used to estimate nutrient limitation.},
author = {Howarth, R W},
doi = {10.1146/annurev.es.19.110188.000513},
issn = {0066-4162},
journal = {Annual Review of Ecology and Systematics},
number = {1},
pages = {89--110},
title = {{Nutrient Limitation of Net Primary Production in Marine Ecosystems}},
url = {http://www.princetonbrainandspine.com/brain/brain-anatomy},
volume = {19},
year = {2003}
}
@misc{Gelmanblog,
author = {Gelman, Andrew},
booktitle = {Statistical Modelling, Causal Inference, and Social Science},
title = {{Data-dependent prior as an approximation to hierarchical model}},
url = {https://andrewgelman.com/2016/03/25/28321/},
urldate = {2018-11-29},
year = {2016}
}
@article{Berger2006,
abstract = {Bayesian statistical practice makes extensive use of versions of ob- jective Bayesian analysis. We discuss why this is so, and address some of the criticisms that have been raised concerning objective Bayesian analysis. The dan- gers of treating the issue too casually are also considered. In particular, we suggest that the statistical community should accept formal objective Bayesian techniques with confidence, but should be more cautious about casual objective Bayesian techniques.},
author = {Berger, James},
doi = {10.1214/06-BA115},
file = {:Users/alyssacirtwill/Documents/Papers/Berger{\_}2006{\_}Bayesian Analysis.pdf:pdf},
isbn = {1931-6690},
issn = {19360975},
journal = {Bayesian Analysis},
keywords = {Coherency,Data dependent priors,Elicitation,Frequentist validation,History of objective Bayes,Information,Invariance,Jeffreys priors,Marginalization paradox,Matching priors,Reference priors,Subjective Bayes,Unification of statistics,Vague proper priors},
number = {3},
pages = {385--402},
title = {{The case for objective Bayesian analysis}},
volume = {1},
year = {2006}
}
@article{Brooks2015,
abstract = {Seasonal influenza epidemics cause consistent, considerable, widespread loss annually in terms of economic burden, morbidity, and mortality. With access to accurate and reliable forecasts of a current or upcoming influenza epidemic's behavior, policy makers can design and implement more effective countermeasures. We developed a framework for in-season forecasts of epidemics using a semiparametric Empirical Bayes framework, and applied it to predict the weekly percentage of outpatient doctors visits for influenza-like illness, as well as the season onset, duration, peak time, and peak height, with and without additional data from Google Flu Trends, as part of the CDC's 2013--2014 "Predict the Influenza Season Challenge". Previous work on epidemic modeling has focused on developing mechanistic models of disease behavior and applying time series tools to explain historical data. However, these models may not accurately capture the range of possible behaviors that we may see in the future. Our approach instead produces possibilities for the epidemic curve of the season of interest using modified versions of data from previous seasons, allowing for reasonable variations in the timing, pace, and intensity of the seasonal epidemics, as well as noise in observations. Since the framework does not make strict domain-specific assumptions, it can easily be applied to other diseases as well. Another important advantage of this method is that it produces a complete posterior distribution for any desired forecasting target, rather than mere point predictions. We report prospective influenza-like-illness forecasts that were made for the 2013--2014 U.S. influenza season, and compare the framework's cross-validated prediction error on historical data to that of a variety of simpler baseline predictors.},
archivePrefix = {arXiv},
arxivId = {1410.7350},
author = {Brooks, Logan C. and Farrow, David C. and Hyun, Sangwon and Tibshirani, Ryan J. and Rosenfeld, Roni},
doi = {10.1371/journal.pcbi.1004382},
eprint = {1410.7350},
file = {:Users/alyssacirtwill/Documents/Papers/Brooks et al.{\_}2015{\_}PLoS Computational Biology.pdf:pdf},
issn = {15537358},
journal = {PLoS Computational Biology},
number = {8},
pages = {1--18},
pmid = {26317693},
title = {{Flexible Modeling of Epidemics with an Empirical Bayes Framework}},
volume = {11},
year = {2015}
}
@article{Spiegelhalter2000,
abstract = {BACKGROUND:Bayesian methods may be defined as the explicit quantitative use of external evidence in the design, monitoring, analysis, interpretation and reporting of a health technology assessment. In outline, the methods involve formal combination through the use of Bayes's theorem of: 1. a prior distribution or belief about the value of a quantity of interest (for example, a treatment effect) based on evidence not derived from the study under analysis, with 2. a summary of the information concerning the same quantity available from the data collected in the study (known as the likelihood), to yield 3. an updated or posterior distribution of the quantity of interest. These methods thus directly address the question of how new evidence should change what we currently believe. They extend naturally into making predictions, synthesising evidence from multiple sources, and designing studies: in addition, if we are willing to quantify the value of different consequences as a 'loss function', Bayesian methods extend into a full decision-theoretic approach to study design, monitoring and eventual policy decision-making. Nonetheless, Bayesian methods are a controversial topic in that they may involve the explicit use of subjective judgements in what is conventionally supposed to be a rigorous scientific exercise. OBJECTIVES:This report is intended to provide: 1. a brief review of the essential ideas of Bayesian analysis 2. a full structured review of applications of Bayesian methods to randomised controlled trials, observational studies, and the synthesis of evidence, in a form which should be reasonably straightforward to update 3. a critical commentary on similarities and differences between Bayesian and conventional approaches 4. criteria for assessing the reporting of a Bayesian analysis 5. a comprehensive list of published 'three-star' examples, in which a proper prior distribution has been used for the quantity of primary interest 6. tutorial case studies of a variety of types 7. recommendations on how Bayesian methods and approaches may be assimilated into health technology assessments in a variety of contexts and by a variety of participants in the research process. METHODS:The BIDS ISI database was searched using the terms 'Bayes' or 'Bayesian'. This yielded almost 4000 papers published in the period 1990-98. All resultant abstracts were reviewed for relevance to health technology assessment; about 250 were so identified, and used as the basis for forward and backward searches. In addition EMBASE and MEDLINE databases were searched, along with websites of prominent authors, and available personal collections of references, finally yielding nearly 500 relevant references. A comprehensive review of all references describing use of 'proper' Bayesian methods in health technology assessment (those which update an informative prior distribution through the use of Bayes's theorem) has been attempted, and around 30 such papers are reported in structured form. There has been very limited use of proper Bayesian methods in practice, and relevant studies appear to be relatively easily identified. RESULTS:Bayesian methods in the health technology assessment context 1. Different contexts may demand different statistical approaches. Prior opinions are most valuable when the assessment forms part of a series of similar studies. A decision-theoretic approach may be appropriate where the consequences of a study are reasonably predictable. 2. The prior distribution is important and not unique, and so a range of options should be examined in a sensitivity analysis. Bayesian methods are best seen as a transformation from initial to final opinion, rather than providing a single 'correct' inference. 3. The use of a prior is based on judgement, and hence a degree of subjectivity cannot be avoided. However, subjective priors tend to show predictable biases, and archetypal priors may be useful for identifying a reasonable range of prior opinion.},
archivePrefix = {arXiv},
arxivId = {arXiv:1011.1669v3},
author = {Spiegelhalter, D. J. and Myles, J. P. and Jones, D. R. and Abrams, K. R.},
doi = {10.3310/hta4380},
eprint = {arXiv:1011.1669v3},
file = {:Users/alyssacirtwill/Library/Application Support/Mendeley Desktop/Downloaded/Spiegelhalter et al. - 2000 - Bayesian methods in health technology assessment a review.pdf:pdf},
isbn = {1366-5278 (Print)$\backslash$n1366-5278 (Linking)},
issn = {13665278},
journal = {Health Technology Assessment},
number = {38},
pages = {1--130},
pmid = {11134920},
title = {{Bayesian methods in health technology assessment: a review}},
volume = {4},
year = {2000}
}
@incollection{Ghosh1992,
address = {Hayward, CA},
author = {Ghosh, Malay},
booktitle = {Lecture Notes-Monograph Series},
doi = {10.1214/lnms/1215458844},
editor = {Ghosh, M and Pathak, P. K.},
pages = {151--177},
publisher = {IMS},
title = {{Hierarchical and empirical Bayes multivariate estimation}},
year = {2008}
}
@article{Casella1985,
abstract = {Empirical Bayes methods have been shown to be powerful data-analysis tools in recent years. The empirical Bayes model is much richer than either the classical or the oridnary Bayes model and often provides superior estimates of parameters. An introduction to some empirical Bayes methods is given, and these methods are illustrated with two examples.},
author = {Casella, George},
doi = {10.1080/00031305.1985.10479400},
file = {:Users/alyssacirtwill/Documents/Papers/Casella{\_}1985{\_}American Statistician.pdf:pdf},
isbn = {00031305},
issn = {15372731},
journal = {American Statistician},
keywords = {Binomial distribution,Normal distribution,Stein estimation},
number = {2},
pages = {83--87},
title = {{An introduction to empirical bayes data analysis}},
volume = {39},
year = {1985}
}
@article{Copas1972,
abstract = {The fact that more is generally known about a standard treatment in a clinical trial than about the test treatment is exploited in empirical Bayes estimates based on the results of using the same standard in other trials. Such estimates are proposed for the case of dichotomous response, and discussed in terms of an example in cancer research.},
author = {Copas, J. B.},
doi = {10.1093/biomet/59.2.349},
issn = {00063444},
journal = {Biometrika},
keywords = {Clinical trials,Concurrent and historical controls,Empirical Bayes methods,Partial prior information,Repeated use of a standard treatment,Two-by-two contingency table},
number = {2},
pages = {349--360},
title = {{Empirical Bayes methods and the repeated use of a standard}},
volume = {59},
year = {1972}
}
@incollection{Allen2007,
abstract = {Understanding the causes and consequences of variation in
biodiversity has long been a central focus of research in ecology
and biogeography (von Humboldt, 1808; Hutchinson, 1959;
MacArthur, 1969; Brown, 1981; Tilman, 1999; Hubbell, 2001;
Clarke, 2006). ...},
address = {Cambridge},
author = {Allen, Andrew P. and Gillooly, James F. and Brown, James H.},
booktitle = {Scaling Biodiversity},
doi = {10.1017/cbo9780511814938.016},
editor = {Storch, D. and Marquet, P. A. and Brown, J. H.},
isbn = {9780511814938},
issn = {0309-1740},
keywords = {bridge books online,cambridge,ebooks,edited by david storch,fromGyuri,http,james brown,org,pablo marquet,scaling biodiversity},
mendeley-tags = {fromGyuri},
pages = {283--299},
pmid = {21663804},
publisher = {Cambridge University Press},
title = {{Recasting the species–energy hypothesis: the different roles of kinetic and potential energy in regulating biodiversity}},
url = {http://www.nceas.ucsb.edu/{~}drewa/pubs/allen{\_}ap{\_}2006.pdf},
year = {2012}
}
@book{May1978,
abstract = {"May's "Stability and Complexity in Model Ecosystems" was undoubtedly the most influential treatise in theoretical ecology since the pioneering efforts of Volterra and Lotka. It transformed the subject by brokering a marriage between theory and fact that had been-and is still too often-missing in theoretical ecology. It is no coincidence that the full integration of theory into ecology has occurred since the original appearance of this landmark book. May's new introduction wonderfully places events in perspective."-Simon Levin, Princeton University},
address = {Princeton},
author = {Hassell, M. P. and May, Robert M.},
booktitle = {The Journal of Animal Ecology},
doi = {10.2307/3743},
isbn = {0691081255},
issn = {00218790},
number = {3},
pages = {931},
pmid = {4723571},
publisher = {Princeton University Press},
title = {{Stability and Complexity in Model Ecosystems}},
url = {http://ieeexplore.ieee.org/document/4309856/},
volume = {44},
year = {2006}
}
@article{Ley1999,
abstract = {Seasonal changes in freshwater inflow and other environmental conditions may induce changes in density and species composition of mangrove fishes along estuarine gradients. Fishes within mangrove habitats in a subtropical estuary were sampled monthly from May 1989 to May 1990, using block nets with rotenone and visual censuses. At 18 stations, temperature ranged from 22 to 34 °C, depth from 10 to 104 cm and underwater visibility from 1 to 13 m. Salinity ranged from 0 to 60 upstream, and 35 to 54 mid- and downstream. A total of 573 191 individuals (76 species) was observed or collected, with an average density of 6.5 fish m-2. Engraulidae, Atherinidae, Poeciliidae and Cyprinodontidae numerically dominated the assemblage. Distinct assemblages occurred up-, mid- and downstream and maintained coherent groups in these gradient positions over the seasons. Residents totalled 94.5{\%} of the individuals, estuarine transients comprised 5.1{\%} and occasional marine visitors were less than 0.4{\%}. Densities of resident fishes peaked in winter as temperatures and water levels fell, uncorrelated with changes in salinity. These observations suggest that mangrove habitats may sustain diverse and abundant fish communities dominated by euryhaline residents. Although estuarine transients were consistently rare in upstream sub-basins, downstream were found numerous sub-adults of species occurring as adults on nearby reefs (Lutjanidae, Haemulidae). Thus, reef-associated estuarine transients may be abundant in mangrove habitats having near-marine salinities. Contrary to expectations, mangrove habitats in northeastern Florida bay did not function as a nursery as defined under the nursery-ground paradigm: young-of-the-year juveniles of estuarine transient species did not seek low salinity sub-basins. However, northeastern Florida Bay may not be representative of most mangrove estuaries as the area: (1) is without lunar tides and related circulation; (2) has low and variable amounts of submersed vegetation; and (3) experiences severe hypersaline conditions.},
author = {Ley, J. A. and McIvor, C. C. and Montague, C. L.},
doi = {10.1006/ecss.1998.0459},
isbn = {0272-7714},
issn = {02727714},
journal = {Estuarine, Coastal and Shelf Science},
keywords = {Estuarine gradient,Fishes,Florida Bay,Mangrove habitats,Reefs},
number = {6},
pages = {701--723},
pmid = {1745},
title = {{Fishes in mangrove prop-root habitats of northeastern Florida Bay: Distinct assemblages across an estuarine gradient}},
volume = {48},
year = {1999}
}
@book{Vellend2016,
abstract = {A model is presented of sagittal plane jaw and hyoid motion based on the lambda model of motor control. The model, which is implemented as a computer simulation, includes central neural control signals, position- and velocity-dependent reflexes, reflex delays, and muscle properties such as the dependence of force on muscle length and velocity. The model has seven muscles (or muscle groups) attached to the jaw and hyoid as well as separate jaw and hyoid bone dynamics. According to the model, movements result from changes in neurophysiological control variables which shift the equilibrium state of the motor system. One such control variable is an independent change in the membrane potential of alpha-motoneurons (MNs); this variable establishes a threshold muscle length (lambda) at which MN recruitment begins. Motor functions may be specified by various combinations of lambda s. One combination of lambda s is associated with the level of coactivation of muscles. Others are associated with motions in specific degrees of freedom. Using the model, we study the mapping between control variables specified at the level of degrees of freedom and control variables corresponding to individual muscles. We demonstrate that commands can be defined involving linear combinations of lambda change which produce essentially independent movements in each of the four kinematic degrees of freedom represented in the model (jaw orientation, jaw position, vertical and horizontal hyoid position). These linear combinations are represented by vectors in lambda space which may be scaled in magnitude. The vector directions are constant over the jaw/hyoid workspace and result in essentially the same motion from any workspace position. The demonstration that it is not necessary to adjust control signals to produce the same movements in different parts of the workspace supports the idea that the nervous system need not take explicit account of musculo-skeletal geometry in planning movements. MESH Subject(s) below:},
address = {Princeton},
archivePrefix = {arXiv},
arxivId = {arXiv:1011.1669v3},
author = {Vellend, Mark},
booktitle = {The Theory of Ecological Communities (MPB-57)},
doi = {10.1515/9781400883790},
eprint = {arXiv:1011.1669v3},
isbn = {9781400883790},
issn = {00264423},
pages = {248},
pmid = {12301371},
publisher = {Princeton University Press},
title = {{The Theory of Ecological Communities (MPB-57)}},
url = {http://www.degruyter.com/view/books/9781400883790/9781400883790/9781400883790.xml},
year = {2016}
}
@article{Wainright1998,
abstract = {Dense, conspicuous colonies of seabirds and pinnipeds breed on ocean islands throughout the world. Such colonies have been shown to have local impacts on prey populations, but whether or not they a{\"{A}}ect nutrient cycling has been debated. We determined the natural abundance levels of the stable isotopes (C and N) of primary producers, seabirds and other consumers at and near St. Paul Island, Pribilof Islands, Bering Sea, in summer 1993. Marine primary producers (phytoplank- ton, as particulate organic matter, and kelp) collected near seabird colonies were ca. 6.5{\&}enriched in both 15N and 13C relative to those collected further from shore. Terrestrial plants collected near the seabird colonies were enriched in 15N(d15N ca. 22{\&}) compared with conspeci{\AE}cs collected away from the colonies (d15N ca. 11{\&}). The trend towards higher d15N values in marine and terrestrial plants near bird colonies is consistent with their uptake of ornithogenic N. This 15N-enrich- ment of plants using ornithogenic N can be attributed to a combination of two processes: trophic enrichment, and volatilization of ammonia produced during degradation of terrestrially deposited guano. Seabird breeding colo- nies at St. Paul Island appear to be signi{\AE}cant sources of recycled nitrogen for terrestrial plants in the vicinity of colonies and for phytoplankton in the nearshore zone.},
author = {Wainright, S. C. and Haney, J. C. and Kerr, C. and Golovkin, A. N. and Flint, M. V.},
doi = {10.1007/s002270050297},
isbn = {0025-3162},
issn = {00253162},
journal = {Marine Biology},
number = {1},
pages = {63--71},
title = {{Utilization of nitrogen derived from seabird guano by terrestrial and marine plants at St. Paul, Pribilof Islands, Bering Sea, Alaska}},
volume = {131},
year = {1998}
}
@incollection{Schiel1986,
address = {Aberdeen},
author = {{SCHIEL, DR (SCHIEL, DR); FOSTER, MS (FOSTER}, MS)},
booktitle = {Oceanography and Marine Biology},
chapter = {4},
edition = {Volume 42},
editor = {Barnes, Harold and Barnes, Margaret},
pages = {265--307},
publisher = {Aberdeen University Press},
title = {{Web THE STRUCTURE OF SUBTIDAL ALGAL STANDS IN TEMPERATE WATERSof Science [v.5.23.2] - Web of Science Core Collection Full Record}},
url = {http://apps.webofknowledge.com.proxy.library.uaf.edu/full{\_}record.do?product=WOS{\&}search{\_}mode=GeneralSearch{\&}qid=4{\&}SID=2CV6TSqpOP6NtxYtg7y{\&}page=1{\&}doc=1},
volume = {24},
year = {1986}
}
@article{VegaCendejas2004,
abstract = {Rio Lagartos, a tropical coastal lagoon in northern Yucatan Peninsula of Mexico, is characterized by high salinity during most of the year (55 psu annual average). Even though the area has been designated as a wetland of international importance because of its great biodiversity, fish species composition and distribution are unknown. To determine whether the salinity gradient was influencing fish assemblages or not, fish populations were sampled seasonally by seine and trawl from 1992 to 1993 and bimonthly during 1997. We identified 81 fish species, eight of which accounted for 53.1{\%} considering the Importance Value Index (Floridichthys polyommus, Sphoeroides testudineus, Eucinostomus argenteus, Eucinostomus gula, Fundulus majalis, Strongylura notata, Cyprinodon artifrons and Elops saurus). Species richness and density declined from the mouth to the inner zone where extreme salinity conditions are prominent ({\textgreater}80) and competitive interactions decreased. However, in Coloradas basin (53 average sanity) and in the inlet of the lagoon, the highest fish density and number of species were observed. Greater habitat heterogeneity and fish immigration were considered as the best explanation. Multivariate analysis found three zones distinguished by fish occurrence, abundance and distribution. Ichthyofaunal spatial differences were attributed to selective recruitment from the Gulf of Mexico due to salinity gradient and to changing climatic periods. Estuarine and euryhaline marine species are abundant, with estuarine dependent ones entering the system according to environmental preferences. This knowledge will contribute to the management of the Special Biosphere Reserve through baseline data to evaluate environmental and anthropogenic changes. {\textcopyright} 2004 Elsevier Ltd. All rights reserved.},
author = {Vega-Cendejas, Ma Eugenia and {Hern{\'{a}}ndez De Santillana}, Mireya},
doi = {10.1016/j.ecss.2004.01.005},
isbn = {0272-7714},
issn = {02727714},
journal = {Estuarine, Coastal and Shelf Science},
keywords = {Assemblages,Community structure,Fish,Hypersaline system,Rio Lagartos Reserve,Seasonal variations,Spatial distribution,Yucatan Peninsula},
number = {2},
pages = {285--299},
title = {{Fish community structure and dynamics in a coastal hypersaline lagoon: Rio Lagartos, Yucatan, Mexico}},
volume = {60},
year = {2004}
}
@article{Kolb2010,
abstract = {Seabirds concentrate nutrients from large marine areas on their nesting islands. The high nutrient load may cause runoff into surrounding waters and affect marine communities in similar ways to those reported from marine fertilization experiments. In order to test if cormorant colonies affect algae and invertebrates in surrounding coastal waters, we collected Fucus vesiculosus fronds, its epiphytic algae, and associated invertebrate fauna near abandoned and active cormorant nesting islands as well as reference islands without nesting cormorants in the Stockholm archipelago in the northern Baltic Sea, Sweden. First, we showed, with $\delta$15N analyses, that ornithogenic nitrogen provided a significant nitrogen source for algae and invertebrate consumers near islands with high nestdensity. Second, the nitrogen and phosphorus content of algae near active cormorant islands withhigh nest density was elevated, and epiphytic algae increased relative to F. vesiculosus. Third, 3 of5 invertebrate taxa (Jaera albifrons, Gammarus spp., and Chironomidae) showed increased biomasses near islands with high nest density; but, contrary to former fertilization studies, only J. albifrons increased in abundance compared to reference islands. We conclude that runoff from seabirdcolonies has a profound effect on primary producers and some consumers in the surrounding water,but only if the colonies exceed a certain nest density. Thus, seabirds not only affect marine communities via top-town forces as commonly assumed, but also via bottom-up forces by concentratingnutrients around their nesting islands. Consequently, seabird islands can be seen as natural fertilization experiments and give important insights to the effects of eutrophication of marine systems.},
author = {Davydov, S. Ya and Kosarev, N. P. and Valiev, N. G. and Boyarskikh, G. A. and Filatov, M. S.},
doi = {10.1007/s11148-017-0004-4},
isbn = {0171-8630},
issn = {10834877},
journal = {Refractories and Industrial Ceramics},
keywords = {Cylindrical form,Energy costs,Spherical rollers,Tribotechnical elements,Tubular belt conveyors},
number = {5},
pages = {462--466},
title = {{Prerequisites for the creation of energy-conserving constructions of tubular belt conveyors}},
url = {http://link.springer.com/10.1007/s11148-017-0004-4},
volume = {57},
year = {2017}
}
@article{Mendonca2018,
abstract = {Understanding the fundamental laws that govern complex food web networks over large ecosystems presents high costs and oftentimes unsurmountable logistical challenges. This way, it is crucial to find smaller systems that can be used as proxy food webs. Intertidal rock pool environments harbour particularly high biodiversity over small areas. This study aimed to analyse their food web networks to investigate their potential as proxies of larger ecosystems for food web networks research. Highly resolved food webs were compiled for 116 intertidal rock pools from cold, temperate, subtropical and tropical regions, to ensure a wide representation of environmental variability. The network properties of these food webs were compared to that of estuaries, lakes and rivers, as well as marine and terrestrial ecosystems (46 previously published complex food webs). The intertidal rock pool food webs analysed presented properties that were in the same range as the previously published food webs. The niche model predictive success was remarkably high (73–88{\%}) and similar to that previously found for much larger marine and terrestrial food webs. By using a large-scale sampling effort covering 116 intertidal rock pools in several biogeographic regions, this study showed, for the first time, that intertidal rock pools encompass food webs that share fundamental organizational characteristics with food webs from markedly different, larger, open and abiotically stable ecosystems. As small, self-contained habitats, intertidal rock pools are particularly tractable systems and therefore a large number of food webs can be examined with relatively low sampling effort. This study shows, for the first time that they can be useful models for the understanding of universal processes that regulate the complex network organization of food webs, which are harder or impossible to investigate in larger, open ecosystems, due to high costs and logistical difficulties.},
author = {Mendon{\c{c}}a, Vanessa and Madeira, Carolina and Dias, Marta and Vermandele, Fanny and Archambault, Philippe and Dissanayake, Awantha and Canning-Clode, Jo{\~{a}}o and Flores, Augusto A.V. and Silva, Ana and Vinagre, Catarina},
doi = {10.1371/journal.pone.0200066},
isbn = {1111111111},
issn = {19326203},
journal = {PLoS ONE},
number = {7},
pages = {e0200066},
title = {{What's in a tide pool? Just as much food web network complexity as in large open ecosystems}},
volume = {13},
year = {2018}
}
@article{Baxter2005,
abstract = {Streams and their adjacent riparian zones are closely linked by reciprocal flows of invertebrate prey. We review characteristics of these prey subsidies and their strong direct and indirect effects on consumers and recipient food webs. 2. Fluxes of terrestrial invertebrates to streams can provide up to half the annual energy budget for drift-feeding fishes such as salmonids, despite the fact that input occurs principally in summer. Inputs appear highest from closed-canopy riparian zones with deciduous vegetation and vary markedly with invertebrate phenology and weather. Two field experiments that manipulated this prey subsidy showed that it affected both foraging and local abundance of stream fishes. 3. Emergence of adult insects from streams can constitute a substantial export of benthic production to riparian consumers such as birds, bats, lizards, and spiders, and contributes 25–100{\%} of the energy or carbon to such species. Emergence typically peaks in early summer in the temperate zone, but also provides a low-level flux from autumn to spring in ice-free streams. This flux varies with in-stream productivity, and declines exponentially with distance from the stream edge. Some predators aggregate near streams and forage on these prey during periods of peak emergence, whereas others rely on the lower subsidy from autumn through spring when terrestrial prey are scarce. Several field experiments that manipulated this subsidy showed that it affected the short-term behaviour, growth, and abundance of terrestrial consumers. 4. Reciprocal prey subsidies also have important indirect effects on both stream and riparian food webs. Theory predicts that allochthonous prey should increase density of subsidised predators, thereby increasing predation on in situ prey and causing a negative indirect effect via apparent competition. However, short-term experiments have produced either positive or negative indirect effects. These contrasting results may be due to characteristics of the subsidies and individual consumers, but could also result from differences in experimental designs. 5. New study approaches are needed to better determine the direct and indirect effects of reciprocal prey subsidies. Experiments coupled with comparative research will be required to measure their effects on individual consumer fitness and population demographics. Future work should investigate whether reciprocal prey fluxes stabilise linked stream–riparian ecosystems, explore how landscape context affects the magnitude and importance of subsidies, and determine how impacts of human disturbance can propagate between streams and riparian zones via these trophic linkages. Study of these reciprocal connections is helping to define a more holistic perspective of catchments, and has the potential to shape new directions for ecology in general.},
author = {Baxter, Colden V. and Fausch, Kurt D. and Saunders, W. Carl},
doi = {10.1111/j.1365-2427.2004.01328.x},
isbn = {0046-5070},
issn = {00465070},
journal = {Freshwater Biology},
keywords = {Allochthonous inputs,Aquatic insects,Emergence,Food webs,Resource subsidies,Riparian ecology,Stream ecology,Terrestrial insects},
number = {2},
pages = {201--220},
pmid = {2964},
title = {{Tangled webs: Reciprocal flows of invertebrate prey link streams and riparian zones}},
volume = {50},
year = {2005}
}
@article{Eilola1999,
abstract = {During the last decade it has become increasingly obvious that the turnover of dissolved organic nitrogen DON in marine environments is quite vigorous. This paper quantifies the turnover of DON in the Baltic proper regarded as a biogeochemical reactor. In a nitrogen model for the reactor, dissolved inorganic nitrogen DIN, DON and molecular N, fixed by cyanobacteria, can be used for plant production. The decomposition of particulate organic matter is assumed to produce DON and DIN as end products in the proportions (1 - eta) to eta (0 less than or equal to eta less than or equal to 1). The model includes two internal sink processes, denitrification and sequestering in the bottom sediments and accounts for external sources and sinks by import and export of DIN and DON. The annual net production in the Baltic proper is about 12.8 10(6) ton C (50 gC m(-2)) requiring about 2.3 10(6) ton N. However only about 1.0 10(6) ton N are available as DW and the deficit has to be covered by an uptake of N from DON and/or fixed molecular nitrogen. The results of the model depend on the value of eta. With eta = 1 the use of DON for primary production is at a minimum (0.19 10(6) ton N) while there are maxima for nitrogen fixation (1.0 10(6) ton N) and denitrification (1.5 10(6) ton N). However, both these values are considered unrealistically large. A more likely value of eta is determined from the model in such a way that the annual rate of nitrogen fixation in the Baltic proper is in accordance with a recent estimate from the literature (0.11 10(6) ton N). This gives eta = 0.55 implying that about 0.67 10(6) ton N is denitrified, and 1.10 10(6) ton DON is used for net production, and 0.91 10(6) ton DON is produced by decomposition of particulate organic matter and the turnover time for DON is about 4 years. The finding that there is a vigorous turnover of DON on the reactor level has important consequences. Firstly, earlier estimates of denitrification rates were based on budgets for oxygen and DIN and overlooked the DON decomposition pathway, why denitrification rates are severely overestimated, often by a factor of 2 or greater. Secondly, the extensive use of DON for primary production in the Baltic proper in combination with abundance of DON, challenge the widely accepted opinion that nitrogen is the production-limiting nutrient on the systems (reactor) level in the Baltic proper.},
author = {Eilola, Kari and Stigebrandt, Anders},
doi = {10.1357/002224099321549648},
isbn = {0022-2402},
issn = {00222402},
journal = {Journal of Marine Research},
number = {4},
pages = {693--713},
title = {{On the seasonal nitrogen dynamics of the Baltic proper biogeochemical reactor}},
url = {http://www.ingentaselect.com/rpsv/cgi-bin/cgi?ini=xref{\&}body=linker{\&}reqdoi=10.1357/002224099321549648},
volume = {57},
year = {2003}
}
@article{Tittensor2010,
abstract = {Global patterns of species richness and their structuring forces have fascinated biologists since Darwin(1,2) and provide critical context for contemporary studies in ecology, evolution and conservation. Anthropogenic impacts and the need for systematic conservation planning have further motivated the analysis of diversity patterns and processes at regional to global scales(3). Whereas land diversity patterns and their predictors are known for numerous taxa(4,5), our understanding of global marine diversity has been more limited, with recent findings revealing some striking contrasts to widely held terrestrial paradigms(6-8). Here we examine global patterns and predictors of species richness across 13 major species groups ranging from zooplankton to marine mammals. Two major patterns emerged: coastal species showed maximum diversity in the Western Pacific, whereas oceanic groups consistently peaked across broad mid-latitudinal bands in all oceans. Spatial regression analyses revealed sea surface temperature as the only environmental predictor highly related to diversity across all 13 taxa. Habitat availability and historical factors were also important for coastal species, whereas other predictors had less significance. Areas of high species richness were disproportionately concentrated in regions with medium or higher human impacts. Our findings indicate a fundamental role of temperature or kinetic energy in structuring cross-taxon marine biodiversity, and indicate that changes in ocean temperature, in conjunction with other human impacts, may ultimately rearrange the global distribution of life in the ocean.},
archivePrefix = {arXiv},
arxivId = {arXiv:1011.1669v3},
author = {Tittensor, Derek P. and Mora, Camilo and Jetz, Walter and Lotze, Heike K. and Ricard, Daniel and Berghe, Edward Vanden and Worm, Boris},
doi = {10.1038/nature09329},
eprint = {arXiv:1011.1669v3},
isbn = {0028-0836},
issn = {00280836},
journal = {Nature},
number = {7310},
pages = {1098--1101},
pmid = {20668450},
publisher = {Nature Publishing Group},
title = {{Global patterns and predictors of marine biodiversity across taxa}},
url = {http://dx.doi.org/10.1038/nature09329},
volume = {466},
year = {2010}
}
@article{Cavanaugh1997,
abstract = {The Akaike (1973, 1974) information criterion, AIC, and the corrected Akaike information criterion (Hurvich and Tsai, 1989), AICc, were both designed as estimators of the expected Kullback-Leibler discrepancy between the model generating the data and a fitted candidate model. AIC is justified in a very general framework, and as a result, offers a crude estimator of the expected discrepancy: one which exhibits a potentially high degree of negative bias in small-sample applications (Hurvich and Tsai, 1989). AICc corrects for this bias, but is less broadly applicable than AIC since its justification depends upon the form of the candidate model (Hurvich and Tsai, 1989, 1993; Hurvich et al., 1990; Bedrick and Tsai, 1994). Although AIC and AICc share the same objective, the derivations of the criteria proceed along very different lines, making it difficult to reconcile how AICc improves upon the approximations leading to AIC. To address this issue, we present a derivation which unifies the justifications of AIC and AICc in the linear regression framework.},
author = {Cavanaugh, Joseph E.},
doi = {10.1016/s0167-7152(96)00128-9},
isbn = {0167-7152},
issn = {01677152},
journal = {Statistics {\&} Probability Letters},
keywords = {aic,aicc,information theory,kullback-leibler information,model selection},
number = {2},
pages = {201--208},
title = {{Unifying the derivations for the Akaike and corrected Akaike information criteria}},
url = {http://linkinghub.elsevier.com/retrieve/pii/S0167715296001289},
volume = {33},
year = {2003}
}
@article{Pausas2001,
abstract = {We review patterns of plant species richness with respect to variables related to resource availability and vari- ables that have direct physiological impact on plant growth or resource availability. This review suggests that there are a variety of patterns of species richness along environmental gradients reported in the literature. However, part of this diversity may be explained by the different types and lengths of gradients studied, and by the limited analysis applied to the data. To advance in understanding species richness pat- terns along environmental gradients, we emphasise the im- portance of: (1) using variables that are related to the growth of plants (latitudinal and altitudinal gradients have no direct process impact on plant growth); (2) using multivariate gra- dients, not single variables; (3) comparing patterns for dif- ferent life forms; and (4) testing for different shapes in the species richness response (not only linear) and for interaction between variables. Keywords:},
author = {Pausas, Juli G. and Austin, Mike P.},
doi = {10.2307/3236601},
isbn = {1100-9233},
issn = {11009233},
journal = {Journal of Vegetation Science},
keywords = {abbreviations,actual evapotranspiration,aet,diversity,environmental gradient,functional,life form,nutrient gradient,pet,potential evapotranspiration,temperature,type},
number = {2},
pages = {153--166},
pmid = {1853},
title = {{Patterns of plant species richness in relation to different environments: An appraisal}},
url = {http://doi.wiley.com/10.2307/3236601},
volume = {12},
year = {2007}
}
@article{Mccain2007,
abstract = {Aim  A global meta-analysis was used to elucidate a mechanistic understanding of elevational species richness patterns of bats by examining both regional and local climatic factors, spatial constraints, sampling and interpolation. Based on these results, I propose the first climatic model for elevational gradients in species richness, and test it using preliminary bat data for two previously unexamined mountains. Location  Global data set of bat species richness along elevational gradients from Old and New World mountains spanning 12.5° S to 38° N latitude. Methods  Bat elevational studies were found through an extensive literature search. Use was made only of studies sampling  70{\%} of the elevational gradient without significant sampling biases or strong anthropogenic disturbance. Undersampling and interpolation were explicitly examined with three levels of error analyses. The influence of spatial constraints was tested with a Monte Carlo simulation program, Mid-Domain Null. Preliminary bat species richness data sets for two test mountains were compiled from specimen records from 12 US museum collections. Results  Equal support was found for decreasing species richness with elevation and mid-elevation peaks. Patterns were robust to substantial amounts of error, and did not appear to be a consequence of spatial constraints. Bat elevational richness patterns were related to local climatic gradients. Species richness was highest where both temperature and water availability were high, and declined as temperature and water availability decreased. Mid-elevational peaks occurred on mountains with dry, arid bases, and decreasing species richness occurred on mountains with wet, warm bases. A preliminary analysis of bat richness patterns on elevational gradients in western Peru (dry base) and the Olympic Mountains, WA (wet base), supported the predictions of the climate model. Main conclusions  The relationship between species richness and combined temperature and water availability may be due to both direct (thermoregulatory constraints) and indirect (food resources) factors. Abundance was positively correlated with species richness, suggesting that bat species richness may also be related to productivity. The climatic model may be applicable to other taxonomic groups with similar ecological constraints, for instance certain bird, insect and amphibian clades.},
author = {McCain, Christy M.},
doi = {10.1111/j.1466-8238.2006.00263.x},
isbn = {1466-8238},
issn = {1466822X},
journal = {Global Ecology and Biogeography},
keywords = {Bats,Climate,Diversity,Elevational gradient,Mammals,Mid-domain effect,Species richness,Temperature,Water availability},
number = {1},
pages = {1--13},
pmid = {2102},
title = {{Could temperature and water availability drive elevational species richness patterns? A global case study for bats}},
volume = {16},
year = {2007}
}
@article{Shannon1948,
abstract = {The recent development of various methods of modulation such as PCM and PPM which exchange bandwidth for signal-to-noise ratio has intensified the interest in a general theory of communication. A basis for such a theory is contained in the important papers of Nyquist and Hartley on this subject. In the present paper we will extend the theory to include a number of new factors, in particular the effect of noise in the channel, and the savings possible due to the statistical structure of the original message and due to the nature of the final destination of the information. The fundamental problem of communication is that of reproducing at one point either exactly or approximately a message selected at another point. Frequently the messages have meaning; that is they refer to or are correlated according to some system with certain physical or conceptual entities. These semantic aspects of communication are irrelevant to the engineering problem. The significant aspect is that the actual message is one selected from a set of possible messages. The system must be designed to operate for each possible selection, not just the one which will actually be chosen since this is unknown at the time of design. If the number of messages in the set is finite then this number or any monotonic function of this number can be regarded as a measure of the information produced when one message is chosen from the set, all choices being equally likely. As was pointed out by Hartley the most natural choice is the logarithmic function. Although this definition must be generalized considerably when we consider the influence of the statistics of the message and when we have a continuous range of messages, we will in all cases use an essentially logarithmic measure.},
archivePrefix = {arXiv},
arxivId = {chao-dyn/9411012},
author = {Shannon, C E},
doi = {10.1002/j.1538-7305.1948.tb01338.x},
eprint = {9411012},
isbn = {0252725484},
issn = {0724-6811},
journal = {M.D. computing : computers in medical practice},
number = {4},
pages = {306--17},
pmid = {9230594},
primaryClass = {chao-dyn},
title = {{The mathematical theory of communication. 1963.}},
url = {http://www.ncbi.nlm.nih.gov/pubmed/9230594},
volume = {14},
year = {1948}
}
@article{Clegg2018,
author = {Clegg, Tom and Ali, Mohammad and Beckerman, Andrew P.},
doi = {10.1002/ecy.2523},
file = {:Users/alyssacirtwill/Documents/Papers/Clegg, Ali, Beckerman{\_}2018{\_}Ecology.pdf:pdf},
journal = {Ecology},
keywords = {ecological network,empirical food web,ontogenetic niche shift,ontogeny,predator–prey,stage structure},
number = {0},
pages = {2712--2720},
title = {{The impact of intraspecific variation on food web structure}},
url = {http://doi.wiley.com/10.1002/ecy.2523},
volume = {99},
year = {2018}
}
@article{Giron2018,
abstract = {{\textcopyright} 2018 The Netherlands Entomological Society. There is tremendous diversity of interactions between plants and other species. These relationships range from antagonism to mutualism. Interactions of plants with members of their ecological community can lead to a profound metabolic reconfiguration of the plants' physiology. This reconfiguration can favour beneficial organisms and deter antagonists like pathogens or herbivores. Determining the cellular and molecular dialogue between plants, microbes, and insects, and its ecological and evolutionary implications is important for understanding the options for each partner to adopt an adaptive response to its biotic environment. Moving forward, understanding how such ecological interactions are shaped by environmental change and how we potentially mitigate deleterious effects will be increasingly important. The development of integrative multidisciplinary approaches may provide new solutions to the major ecological and societal issues ahead of us. The rapid evolution of technology provides valuable tools and opens up novel ways to test hypotheses that were previously unanswerable, but requires that scientists master these tools, understand potential ethical problems flowing from their implementation, and train new generations of biologists with diverse technical skills. Here, we provide brief perspectives and discuss future promise and challenges for research on insect-plant interactions building on the 16th International Symposium on Insect-Plant interactions (SIP) meeting that was held in Tours, France (2-6 July 2017). Talks, posters, and discussions are distilled into key research areas in insect-plant interactions, highlighting the current state of the field and major challenges, and future directions for both applied and basic research.},
author = {Giron, David and Dubreuil, G{\'{e}}raldine and Bennett, Alison and Dedeine, Franck and Dicke, Marcel and Dyer, Lee A. and Erb, Matthias and Harris, Marion O. and Huguet, Elisabeth and Kaloshian, Isgouhi and Kawakita, Atsushi and Lopez-Vaamonde, Carlos and Palmer, Todd M. and Petanidou, Theodora and Poulsen, Michael and Sall{\'{e}}, Aur{\'{e}}lien and Simon, Jean Christophe and Terblanche, John S. and Thi{\'{e}}ry, Denis and Whiteman, Noah K. and Woods, H. Arthur and Pincebourde, Sylvain},
doi = {10.1111/eea.12679},
file = {:Users/alyssacirtwill/Documents/Papers/Giron et al.{\_}2018{\_}Entomologia Experimentalis et Applicata.pdf:pdf},
issn = {15707458},
journal = {Entomologia Experimentalis et Applicata},
keywords = {community ecology,ecological networks,evolutionary genomics,forests and agroecosystems,global change,insect effectors,multitrophic interactions,phylogenetics,plant response,symbionts,thermal ecology},
number = {5},
pages = {319--343},
title = {{Promises and challenges in insect–plant interactions}},
volume = {166},
year = {2018}
}
@article{Lopez2017,
abstract = {The irrigated area of Jaguaribe, Ceara, Brazil is considered important region of agribusiness of the country due to the installation of various fruit exporting companies. The present work has as main objective investigate twelve types of pesticides (molinate, atrazine, methyl parathion, malathion, chlorpyrifos, fenitrothion, pendimenthalin, triazophos, bentazone, azoxystrobin, propiconazole, difenoconazole) used in the region to assess the level of contamination of waters used for potable and irrigation. Analysis of pesticides were performed using chromatographic techniques (SPME-GC/MS and SPE-HPLC/DAD) through methodologies validated according to parameters recommended by ABNT. Among the 60 water samples, 48 were positive for at least one of the twelve active ingredients studied. Fungicides propiconazole and difenoconazole were detected more frequently. The total pesticide levels ranging from 0.11-17.30 µg/L were detected in the samples. The levels detected in surface and groundwater were lower than the limits established in Brazil, but 80{\%} of the samples analyzed were above total pesticide levels established by the European Community ({\textgreater}0.5 µg/L). Prolonged exposure to pesticides can cause adverse effects to human health and the aquatic ecosystem.},
author = {Lopez, Daniela N. and Camus, Patricio A. and Valdivia, Nelson and Estay, Sergio A.},
doi = {10.1111/oik.04285},
file = {:Users/alyssacirtwill/Documents/Papers/Lopez et al.{\_}2017{\_}Oikos.pdf:pdf},
isbn = {2012203566},
issn = {16000706},
journal = {Oikos},
number = {12},
pages = {1699--1707},
pmid = {28746801},
title = {{High temporal variability in the occurrence of consumer–resource interactions in ecological networks}},
volume = {126},
year = {2017}
}
@article{Currie1991,
abstract = {Many hypotheses have been proposed to explain the great variation among regions in species richness. These were tested by first examining patterns of species richness of birds, mammals, amphibians, and reptiles in 336 quadrats covering North America. These patterns were then compared with the regional variation of 21 descriptors of the environment suggested by the hypotheses. I found that, in the four vertebrate classes studied, 80{\%}-93{\%} of the variability in species richness could be statistically explained by a monotonically increasing function of a single variable: annual potential evapotranspiration (PET). In contrast, tree richness is more closely related to actual evapotranspiration (AET). Both AET and PET appear to be measures of available environmental energy. The relationships between tree and vertebrate richness are strikingly poor. Species richness in particular orders and families of the Vertebrata is also closely related to PET, but not always monotonically, often resembling a replacement series along an environmental gradient. The present results are consistent with the hypothesis that environmentally available energy limits regional species richness. However, my observations are not completely consistent with earlier species-energy theory. The energy-richness relationship appears to depend on scale, and it is affected differently by variations in area and in areal energy flux.},
author = {Currie, David J.},
doi = {10.1086/285144},
issn = {0003-0147},
journal = {The American Naturalist},
number = {1},
pages = {27--49},
title = {{Energy and Large-Scale Patterns of Animal- and Plant-Species Richness}},
volume = {137},
year = {2002}
}
@article{HELCOM2018,
author = {HELCOM},
journal = {Baltic Sea Environment Proceedings},
title = {{State of the Baltic Sea – Second HELCOM holistic assessment 2011-2016. Baltic Sea Environment Proceedings 155.}},
url = {www.helcom.fi/baltic-sea-trends/holistic-assessments/state-of-the-baltic-sea-2018/reports-and-materials/},
volume = {155},
year = {2018}
}
@article{Turner1987,
abstract = {JSTOR is a not-for-profit service that helps scholars, researchers, and students discover, use, and build upon a wide range of content in a trusted digital archive. We use information technology and tools to increase productivity and facilitate new forms of scholarship. For more information about JSTOR, please contact support@jstor.org. Wiley, Nordic Society Oikos are collaborating with JSTOR to digitize, preserve and extend access to Oikos This content downloaded from 35.8.11.3 on Mon, 25 Jul 2016 19:40:15 UTC All use subject to http://about.jstor.org/terms A. 1986. Does solar energy con-trol organic diversity? Butterflies, moths and the British climate. -Oikos 48: 195-205. In Britain, the diversity of butterflies, measured as the number of species present in an area of 900 km2, is highly correlated with sunshine and temperature during the summer months May to September. These two variables explain nearly eighty per-cent of the variance in diversity, and also explain it to some extent independently of latitude. Diversity shows a negative correlation with the average temperature for the three winter months December to February. Diversity (species number) therefore appears to be strongly influenced by the amount of energy available during the fa-vourable season. The simplest explanation for this relationship is the extreme ecto-thermic behaviour of adult butterflies, which depend on both warm air and direct sunshine to maintain normal activity. Heavy bodied, endothermic, nocturnal moths however show a very similar correlation with the climatic variables, suggesting that larvae as well as adults benefit from sunshine and warmth. Butterfly and moth diver-sity is shown also to be influenced by habitat diversity. It is suggested that these find-ings support the "species-energy hypothesis" of D. H. Wright, which explains global latidudinal gradients in diversity by the gradient of direct and atmospherically trans-ported solar energy. They appear also to explain some temporal fluctuations in but-terfly diversity.},
author = {Turner, John R. G. and Gatehouse, Catherine M. and Corey, Charlotte A.},
doi = {10.2307/3565855},
isbn = {00301299},
issn = {00301299},
journal = {Oikos},
number = {2},
pages = {195},
pmid = {716},
title = {{Does Solar Energy Control Organic Diversity? Butterflies, Moths and the British Climate}},
url = {http://www.jstor.org/stable/3565855?origin=crossref},
volume = {48},
year = {2006}
}
@article{Wells2013a,
abstract = {* Ecological network models based on aggregated data from species interactions are widely used to make inferences about species specialization, functionality and extinction risk. While increasing number of network data are available and are used in comparative studies, data quality and uncertainty have received little attention. Moreover, key individual-level information such as the proportion of individuals not involved in interactions and underlying processes driving interactions are ignored by aggregated data analysis. * We suggest an individual-level hierarchical interaction model as a more flexible approach to considering uncertainty, sampling effort and conditions under which interactions take place and from which network attributes can be derived. We performed a simulation exercise to compare inference under different sample sizes and from aggregated data matrices to those from our individual-level model. * Formalizing the process of network formation in an individual-level model made clear that per-species interaction frequencies are not independent of sample size and population pools and also ignore important information given by the proportion of non-interacting individuals. Hierarchical linear models are a possible solution to infer community-level attributes of network formation and allow various kinds of comprehensive model extensions to capture variation of per-individual interactions in space and time that shape upper level organization. * Individual-level hierarchical models provide the link between individual behaviour and interactions under variable environmental conditions and can be summarized into networks in a conceptually neat way. Such models may not only help to account for various sources of variation but also conceptualize aspects overlooked in aggregated data. In particular, the quantification of per-individual interactions under different sampling scenarios emphasizes that per-species interaction frequencies at the species level are not necessarily a surrogate of species abundance in natural systems under investigation},
author = {Wells, Konstans and O'Hara, Robert B.},
doi = {10.1111/j.2041-210x.2012.00249.x},
file = {:Users/alyssacirtwill/Documents/Papers/Wells et al.{\_}2013{\_}Oecologia.pdf:pdf},
isbn = {2041-210X},
issn = {2041210X},
journal = {Methods in Ecology and Evolution},
keywords = {Biotic interactions,Ecological fallacy,Ecological networks,Hierarchical models,Random graphs,Species specialization},
number = {1},
pages = {1--8},
title = {{Species interactions: Estimating per-individual interaction strength and covariates before simplifying data into per-species ecological networks}},
volume = {4},
year = {2013}
}
@article{Graham2018,
abstract = {Abstract Species interactions are fundamental to community dynamics and ecosystem processes. Despite significant progress in describing species interactions, we lack the ability to predict changes in interactions across space and time. We outline a Bayesian approach to separate the probability of species co-occurrence, interaction and detectability in influencing interaction betadiversity. We use a multi-year hummingbird?plant time series, divided into training and testing data, to show that including models of detectability and occurrence improves forecasts of mutualistic interactions. We then extend our model to explore interaction betadiversity across two distinct seasons. Despite differences in the observed interactions among seasons, there was no significant change in hummingbird occurrence or interaction frequency between hummingbirds and plants. These results highlight the challenge of inferring the causes of interaction betadiversity when interaction detectability is low. Finally, we highlight potential applications of our model for integrating observations of local interactions with biogeographic and evolutionary histories of co-occurring species. These advances will provide new insight into the mechanisms that drive variation in patterns of biodiversity.},
author = {Graham, Catherine H. and Weinstein, Ben G.},
doi = {10.1111/ele.13084},
file = {::},
isbn = {978-1-55563-732-3},
issn = {14610248},
journal = {Ecology Letters},
keywords = {Bayesian,dissimilarity,evolutionary history,interaction beta-diversity,interaction turnover,network ecology,prediction,rewiring,scale,species turnover},
number = {9},
pages = {1299--1310},
pmid = {21653401},
title = {{Towards a predictive model of species interaction beta diversity}},
volume = {21},
year = {2018}
}
@article{Pianka1966,
abstract = { The six major hypotheses of the control of species diversity are restated, examined, and some possible tests suggested. Although several of these mechanisms could be operating simultaneously, it is instructive to consider them separately, as this can serve to clarify our thinking, as well as assist in the choice of the best test situations for future examination.},
author = {Pianka, Eric R.},
doi = {10.1086/282398},
isbn = {00030147},
issn = {0003-0147},
journal = {The American Naturalist},
number = {910},
pages = {33--46},
pmid = {1034},
title = {{Latitudinal Gradients in Species Diversity: A Review of Concepts}},
url = {http://www.journals.uchicago.edu/doi/10.1086/282398},
volume = {100},
year = {2002}
}
@article{Bartomeus2013,
abstract = {The analysis of mutualistic networks has become a central tool in answering theoretical and applied questions regarding our understanding of ecological processes. Significant gaps in knowledge do however need to be bridged in order to effectively and accurately be able to describe networks. Main concern are the incorporation of species level information, accounting for sampling limitations and understanding linkage rules. Here I propose a simple method to combine plant pollinator effort-limited sampling with information about plant community to gain understanding of what drives linkage rules, while accounting for possible undetected linkages. I use hierarchical models to estimate the probability of detection of each plant-pollinator interaction in 12 Mediterranean plant-pollinator networks. As it is possible to incorporate plant traits as co-variables in the models, this method has the potential to be used for predictive purposes, such as identifying undetected links among existing species, as well as potential interactions with new plant species. Results show that pollinator detectability is very skewed and usually low. Nevertheless, 84{\%} of the models are enhanced by the inclusion of co-variables, with flower abundance and inflorescence type being the most commonly retained co-variables. The predicted networks increase network Connectance by 13{\%}, but not Nestedness, which is known to be robust to sampling effects. However, 46{\%} of the pollinator interactions in the studied networks comprised a single observation and hence could not be modeled. The hierarchical modeling approach suggested here is highly flexible and can be used on binary or frequency networks, accommodate different observers or include collection day weather variables as confounding factors. An R script is provided for a rapid adoption of this method.},
author = {Bartomeus, Ignasi},
doi = {10.1371/journal.pone.0069200},
file = {:Users/alyssacirtwill/Documents/Papers/Bartomeus{\_}2013{\_}PLoS ONE.pdf:pdf},
isbn = {1932-6203},
issn = {19326203},
journal = {PLoS ONE},
keywords = {Connectance,Linkage density,Nestedness,hierarchical models,occupancy models,pollination web,specialization},
number = {7},
pages = {1--8},
pmid = {23840909},
title = {{Understanding Linkage Rules in Plant-Pollinator Networks by Using Hierarchical Models That Incorporate Pollinator Detectability and Plant Traits}},
volume = {8},
year = {2013}
}
@article{Jordano2016,
abstract = {$\backslash$n$\backslash$n$\backslash$n$\backslash$n* Sampling ecological interactions presents similar challenges, problems, potential biases, and constraints as sampling individuals and species in biodiversity inventories. Robust estimates of the actual number of interactions (links) within diversified ecological networks require adequate sampling effort that needs to be explicitly gauged. Yet we still lack a sampling theory explicitly focusing on ecological interactions.$\backslash$n$\backslash$n$\backslash$n* While the complete inventory of interactions is likely impossible, a robust characterization of its main patterns and metrics is probably realistic. We must acknowledge that a sizeable fraction of the maximum number of interactions Imax among, say, A animal species and P plant species (i.e., Imax = AP) is impossible to record due to forbidden links, i.e., life-history restrictions. Thus, the number of observed interactions I in robustly sampled networks is typically I {\textless}{\textless} Imax, resulting in sparse interaction matrices with low connectance.$\backslash$n$\backslash$n$\backslash$n* Here I provide a review and outline a conceptual framework for interaction sampling by building an explicit analogue to individuals and species sampling, thus extending diversity-monitoring approaches to the characterization of complex networks of ecological interactions. Contrary to species inventories, a sizable fraction of non-observed pairwise interactions cannot be sampled, due to biological constraints that forbid their occurrence.$\backslash$n$\backslash$n$\backslash$n* Reasons for forbidden links are multiple but mainly stem from spatial and temporal uncoupling, size mismatches, and intrinsically low probabilities of interspecific encounter for most potential interactions of partner species. Adequately assessing the completeness of a network of ecological interactions thus needs knowledge of the natural history details embedded, so that forbidden links can be accounted for as a portion of the unobserved links when addressing sampling effort.$\backslash$n$\backslash$n$\backslash$n* Recent implementations of inference methods for unobserved species or for individual-based data can be combined with the assessment of forbidden links. This can help in estimating their relative importance, simply by the difference between the asymptotic estimate of interaction richness in a robustly-sampled assemblage and the maximum richness Imax of interactions. This is crucial to assess the rapid and devastating effects of defaunationdriven loss of key ecological interactions and the services they provide and the analogous losses related to interaction gains due to invasive species and biotic homogenization.$\backslash$n$\backslash$nThis article is protected by copyright. All rights reserved.},
author = {Jordano, Pedro},
doi = {10.1111/1365-2435.12763},
file = {:Users/alyssacirtwill/Library/Application Support/Mendeley Desktop/Downloaded/Jordano - 2016 - Sampling networks of ecological interactions.pdf:pdf},
isbn = {1365-2435},
issn = {13652435},
journal = {Functional Ecology},
keywords = {complex networks,food webs,frugivory,mutualism,plant–animal interactions,pollination,seed dispersal},
number = {12},
pages = {1883--1893},
title = {{Sampling networks of ecological interactions}},
volume = {30},
year = {2016}
}
@incollection{Kindlmann2007,
abstract = {A covalently branched nucleic acid can be synthesized by joining the 2'-hydroxyl of the branch-site ribonucleotide of a DNA or RNA strand to the activated 5'-phosphorus of a separate DNA or RNA strand. We have previously used deoxyribozymes to synthesize several types of branched nucleic acids for experiments in biotechnology and biochemistry. Here, we report in vitro selection experiments to identify improved deoxyribozymes for synthesis of branched DNA and RNA. Each of the new deoxyribozymes requires Mn(2+) as a cofactor, rather than Mg(2+) as used by our previous branch-forming deoxyribozymes, and each has an initially random region of 40 rather than 22 or fewer combined nucleotides. The deoxyribozymes all function by forming a three-helix-junction (3HJ) complex with their two oligonucleotide substrates. For synthesis of branched DNA, the best new deoxyribozyme, 8LV13, has k(obs) on the order of 0.1 min(-1), which is about two orders of magnitude faster than our previously identified 15HA9 deoxyribozyme. 8LV13 also functions at closer-to-neutral pH than does 15HA9 (pH 7.5 versus 9.0) and has useful tolerance for many DNA substrate sequences. For synthesis of branched RNA, two new deoxyribozymes, 8LX1 and 8LX6, were identified with broad sequence tolerances and substantial activity at pH 7.5, versus pH 9.0 for many of our previous deoxyribozymes that form branched RNA. These experiments provide new, and in key aspects improved, practical catalysts for preparation of synthetic branched DNA and RNA.},
address = {Cambridge},
author = {Kindlmann, Pavel and Sch{\"{o}}delbauerov{\'{a}}, Iva and Dixon, Anthony F. G.},
booktitle = {Scaling Biodiversity},
chapter = {12},
doi = {10.1017/cbo9780511814938.014},
isbn = {0521699371},
issn = {01695347},
number = {1969},
pages = {246--257},
pmid = {20739352},
publisher = {Cambridge University Press},
title = {{Inverse latitudinal gradients in species diversity}},
url = {http://linkinghub.elsevier.com/retrieve/pii/0169534789901638},
volume = {4},
year = {2012}
}
@article{Roy1994,
abstract = {This essay evaluates women's responses to classical myth against the background of third wave feminism, using Jo Shapcott's Of Mutability as a case study. Studies of female-authored responses to classical literature have overwhelmingly focussed on an impulse to give the female characters of antiquity a presence that is enriched through being penned by authors willing to acknowledge the importance of their experiences as women. The rhetoric is of 'rescue', 1 of 're-vision', 2 of reworking a male-dominated Western tradition so that it becomes a tradition that speaks of and for women also. Lively's chapter on third-wave feminism in Classics and the Uses of Reception usefully and lucidly charts the relationship between the classical world and three waves of feminism, the first of which began in 1792 and ended with a significant increase in women's right to the vote, 3 followed by the 'second wave' feminists of the 1960s and 1970s whose agenda were fuelled by a sense of victimhood and a wish to establish sisterly solidarity in order to counter their repression, and finally the third wave feminists who emerged in the mid-1980s and whose ethos was no longer to redress previous wrongs but to offer a more celebratory vision of feminism which welcomed the individual and no longer wanted to see her subsumed into a group identity. 4 On this reading feminist studies of the reception of the classical world to},
author = {Cox, Fiona M.},
doi = {10.1093/crj/cls017},
issn = {17595134},
journal = {Classical Receptions Journal},
number = {2},
pages = {163--175},
title = {{Metamorphosis, mutability and the third wave}},
volume = {4},
year = {2012}
}
@article{Rohde1992,
abstract = {Hypotheses that attempt to explain latitudinal gradients in species diversity are reviewed. Some hypotheses are circular, i.e. they are based on the assumption that some taxa have greater diversity in the tropics. These include explanations assuming different degrees of competition, mutualism, predation, epiphyte load, epidemics, biotic spatial heterogeneity, host diversity, population size, niche width, population growth rate, environmental harshness, and patchiness at different latitudes. Other explanations are not supported by sufficient evidence, i.e. there is no consistent correlation between species diversity and environmental stability, environmental predictability, productivity, abiotic rarefaction, physical heterogeneity, latitudinal decrease in the angle of the sun above the horizon, area, aridity, seasonality, number of habitats, and latitudinal ranges. The ecological and evolutionary time hypotheses, as usually understood, also cannot explain the gradients, nor does the temperature dependence of chemical reactions permit predictions on species richness. Only differences in solar energy are consistently correlated with diversity gradients along latitude, altitude and perhaps depth. It is concluded that greater species diversity is due to greater "effective" evolutionary time (evolutionary speed) in the tropics, probably as the result of shorter generation times, faster mutation rates, and faster selection at greater temperatures. There is an urgent need for experimental studies of temperature effects on speed of selection.},
author = {Rohde, Klaus},
doi = {10.2307/3545569},
isbn = {0030-1299},
issn = {00301299},
journal = {Oikos},
number = {3},
pages = {514},
pmid = {182},
title = {{Latitudinal Gradients in Species Diversity: The Search for the Primary Cause}},
url = {https://www.jstor.org/stable/3545569?origin=crossref},
volume = {65},
year = {1992}
}
@article{SosaLopez2007,
abstract = {Abstract Aim To analyse the relationship between fish species richness and salinity, and to provide a simple linear model for fish diversity trends across salinity gradients in a tropical coastal lagoon that can be compared with other similar ecosystems and other communities. To reinforce our conclusions, the salinity-fish richness relationship was investigated at different spatial scales (sampling station, set of stations and whole lagoon) and for two different periods, separated by 18 years. Location The Terminos coastal lagoon, a shallow tropical lagoon (mean maximum depths ranging between 3.5 and 4.5 m), is located in the southern Gulf of Mexico (18.5-18.8o N, 91.3-91.9o W). The lagoon is 70 km long and 30 km wide, with a surface area of 1700 km2. Methods Fish sampling, individual identification to the species level, and environmental variable measurements were carried out monthly at 17 sampling points. Multiple regression analysis with a backward selection procedure was used to relate fish species richness to environmental variables. Other statistical techniques, including cluster analysis and ancova, were applied to experimental data surveys. Results Among the different environmental variables, salinity was significantly and consistently related to fish species richness, whatever the period and the scale of observation. We found mainly significant negative correlations (P {\textless} 0.05) between fish species richness and salinity when sampling stations were analysed individually, and particularly for the river runoff zones with high variation in salinity throughout the year. For the entire lagoon, robust negative linear models were observed when fish species richness was organized into salinity ranges, with salinity explaining c. 8{\%} of the variation in mean fish species richness (in a multiple regression analysis; 63-93{\%} when considered in isolation). Main conclusions In the Terminos lagoon the relationship between fish species richness and salinity is mainly negative on any spatial scale. This result may be due partially to the penetration of freshwater fishes into estuarine areas following freshwater discharges, and partially to the dominance of estuarine taxa more able to tolerate low than high salinity values. Finally, we suggest that the 'realized' ecotone, where species from different origins really mix, is situated between 5 and 10{\%}, corresponding to the highest fish richness.},
author = {Sosa-L{\'{o}}pez, Atahualpa and Mouillot, David and Ramos-Miranda, Julia and Flores-Hernandez, Domingo and Chi, Thang Do},
doi = {10.1111/j.1365-2699.2006.01588.x},
isbn = {0305-0270},
issn = {03050270},
journal = {Journal of Biogeography},
keywords = {Ecotone,Experimental trawling,Fish species richness,Gulf of Mexico,Linear models,Regression analysis,Salinity gradient,Terminos lagoon},
number = {1},
pages = {52--61},
title = {{Fish species richness decreases with salinity in tropical coastal lagoons}},
volume = {34},
year = {2007}
}
@article{Eitzinger2018,
abstract = {In mammalian cells, mismatch recognition has been attributed to two partially redundant heterodimeric protein complexes of MutS homologues, MSH2-MSH3 and MSH2-MSH6. We have conducted a comparative analysis of Msh3 and Msh6 deficiency in mouse intestinal tumorigenesis by generating Apc1638N mice deficient in Msh3, Msh6 or both. We have found that Apc1638N mice defective in Msh6 show reduced survival and a 6-7- fold increase in intestinal tumor multiplicity. In contrast, Msh3- deficient Apc1638N mice showed no difference in survival and intestinal tumor multiplicity as compared with Apc1638N mice. However, when Msh3 deficiency is combined with Msh6 deficiency (Msh3(-/-)Msh6(-/- )Apc1638N), the survival rate of the mice was further reduced compared to Msh6(-/-)Apc(1638N) mice because of a high multiplicity of intestinal tumors at a younger age. Almost 90{\%} of the intestinal tumors from both Msh6(-/-)Apc1638N and Msh3(-/-)Msh6(-/-)Apc1638N mice contained truncation mutations in the wild-type Apc allele. Apc mutations in Msh6(-/-)Apc1638N mice consisted predominantly of base substitutions (93{\%}) creating stop codons, consistent with a major role for Msh6 in the repair of base-base mismatches. However, in Msh3(-/- )Msh6(-/-)Apc1638N tumors, we observed a mixture of base substitutions (46{\%}) and frameshifts (54{\%}), indicating that in Msh6(-/-)Apc1638N mice frameshift mutations in the Apc gene were suppressed by Msh3. Interestingly, all except one of the Apc mutations detected in mismatch repair-deficient intestinal tumors were located upstream of the third 20-amino acid beta-catenin binding repeat and before all of the Ser-Ala- Met-Pro repeats, suggesting that there is selection for loss of multiple domains involved in beta-catenin regulation. Our analysis therefore has revealed distinct mutational spectra and clarified the roles of Msh3 and Msh6 in DNA repair and intestinal tumorigenesis.},
author = {Eitzinger, Bernhard and Abrego, Nerea and Gravel, Dominique and Huotari, Tea and Vesterinen, Eero J. and Roslin, Tomas},
doi = {10.1111/mec.14872},
file = {:Users/alyssacirtwill/Documents/Papers/Eitzinger et al.{\_}2019{\_}Molecular Ecology(2).pdf:pdf},
isbn = {0008-5472 (Print)$\backslash$r0008-5472 (Linking)},
issn = {1365294X},
journal = {Molecular Ecology},
keywords = {altitudinal gradient, body mass,interaction probability,lycosidae,metabarcoding,predator–prey interaction},
number = {2},
pages = {266--280},
pmid = {11691815},
title = {{Assessing changes in arthropod predator–prey interactions through DNA-based gut content analysis—variable environment, stable diet}},
url = {http://doi.wiley.com/10.1111/mec.14872},
volume = {28},
year = {2019}
}
@article{Hiddink2008,
abstract = {http://dx.doi.org/10.1111/j.1365-2486.2007.01518.x},
author = {Hiddink, J. G. and ter Hofstede, R.},
doi = {10.1111/j.1365-2486.2007.01518.x},
isbn = {1365-2486},
issn = {13541013},
journal = {Global Change Biology},
keywords = {Biodiversity,Biogeography,Climate change,Extinction,Fisheries,Global warming,NorthSea,speciesrichness},
number = {3},
pages = {453--460},
pmid = {1039},
title = {{Climate induced increases in species richness of marine fishes}},
volume = {14},
year = {2008}
}
@article{Connolly1998,
abstract = {Intertidal systems have been models for the study of the roles of competition, predation, and disturbance in determining community structure. These systems exhibit considerable regional variability in percentage cover and in the strength of interspecific interactions, which may be due largely to effects of varying larval supply. In Oregon and Washington, experimental studies of space allocation among sessile invertebrates have emphasized the role of benthic processes such as competition and predation. In contrast, studies in central California have emphasized the importance of larval supply. In this article, we identify a gradient in percentage cover in the middle and upper intertidal zone that is consistent with an oceanographically based explanation for these differences: percentage cover of mussels and barnacles is much higher in Oregon, where nearshore circulation promotes high recruitment, than in California, where strong offshore currents inhibit recruitment. A mathematical model incorporating larval transport and interspecific competition for space offers an explanation for the one violation of the hypothesis-higher percentage cover of Chthamalus spp. in California. The findings illustrate that attempts to synthesize regional differences in community structure and dynamics can benefit from considering both the benthic adult and pelagic larval environments.},
author = {Connolly and Roughgarden},
doi = {10.2307/2463419},
isbn = {0003-0147},
issn = {00030147},
journal = {The American Naturalist},
keywords = {community struc-,intertidal community,recruitment},
number = {4},
pages = {311},
pmid = {18811323},
title = {{A Latitudinal Gradient in Northeast Pacific Intertidal Community Structure: Evidence for an Oceanographically Based Synthesis of Marine Community Theory}},
url = {http://www.jstor.org/stable/10.1086/286121{\%}5Cnhttp://www.ncbi.nlm.nih.gov/pubmed/18811323},
volume = {151},
year = {2017}
}
@article{Roy1998,
abstract = {Latitudinal diversity gradients are first-order expressions of diversity patterns both on land and in the oceans, although the current hypotheses that seek to explain them are based chiefly on terrestrial data. We have assembled a database of the geographic ranges of 3,916 species of marine prosobranch gastropods living on the shelves of the western Atlantic and eastern Pacific Oceans, from the tropics to the Arctic Ocean. Western Atlantic and eastern Pacific diversities are similar, and the diversity gradients are strikingly similar despite many important physical and historical differences between the oceans. This shared diversity pattern cannot be explained by: (i) latitudinal differences in species range-length (Rapoport's rule); (ii) species-area effects; or (iii) recent geologic histories. One parameter that does correlate significantly with diversity in both oceans is solar energy input, as represented by average sea surface temperature. If this correlation is causal, sea surface temperature is probably linked to diversity through some aspect of productivity. In this case, diversity is an evolutionary outcome of trophodynamic processes inherent in ecosystems, and not just a byproduct of physical geographies.},
author = {Roy, K. and Jablonski, D. and Valentine, J. W. and Rosenberg, G.},
doi = {10.1073/pnas.95.7.3699},
isbn = {00278424},
issn = {0027-8424},
journal = {Proceedings of the National Academy of Sciences},
number = {7},
pages = {3699--3702},
pmid = {9520429},
title = {{Marine latitudinal diversity gradients: Tests of causal hypotheses}},
url = {http://www.pnas.org/cgi/doi/10.1073/pnas.95.7.3699},
volume = {95},
year = {2002}
}
@article{Martino2003,
abstract = {This 3-year study provides a large-scale perspective of fish assemblage structure across an ocean-estuarine ecotone, given range of salinity encountered (0.1-32) based on sampling at 12 stations along 40 km from the Mullica River (river), Great Bay (bay), and the adjacent inner continental shelf (ocean) in southern New Jersey. Otter trawl (4.9 m, 6 mm mesh) collections were dominated by young-of-the-year of most of the 49 species encountered. Species richness and abundance appeared greatest in the ocean, decreased (with an increase in inter-station variability) in the bay, and appeared to increase again towards the uppermost river stations. The same areas contained three non-discrete, but identifiable, fish assemblages based on Detrended Correspondence Analysis. Members of the Triglidae and Stromateidae characterized the ocean and bay, whereas representatives of the Percichthyidae and Ictaluridae characterized the river. Several species, including Anchoa mitchilli and Cynoscion regalis, exhibited a ubiquitous distribution across the sampling area. Further analyses with Canonical Correspondence Analysis identified salinity and geographic distance, among the variables examined, as the most important determinants in shaping the assemblages. Other contributors included habitat heterogeneity and water depth. In summary, these observations indicate that large-scale patterns in the structure of this estuarine fish assemblage are primarily a result of individual species' responses to dominate environmental gradients, as well as ontogenetic migrations, whereas smaller-scale patterns appear to be the result of habitat associations that are most likely driven by foraging, competition, and/or predator avoidance. {\textcopyright} 2003 Elsevier Science B.V. All rights reserved.},
author = {Martino, Edward J. and Able, Kenneth W.},
doi = {10.1016/S0272-7714(02)00305-0},
isbn = {0272-7714},
issn = {02727714},
journal = {Estuarine, Coastal and Shelf Science},
keywords = {Estuarine,Fish assemblage structure,Inner continental shelf,Low-salinity,Species minimum},
number = {5-6},
pages = {969--987},
title = {{Fish assemblages across the marine to low salinity transition zone of a temperate estuary}},
volume = {56},
year = {2003}
}
@article{Edgar1999,
abstract = {The distributions of 390 taxa of benthic macroinvertebrates collected in forty-eight estuaries and 101 fish species collected in seventy-five Tasmanian estuaries were related to geographical and environmental variables. Distribution patterns for the two taxonomic groups were largely congruent at both between- and within-estuary scales. Faunal composition and the number of species collected at a site were primarily related to site salinity, the biomass of seagrass and tidal range. At the broader estuary scale, the distributions of macroinvertebrate and fish assemblages were primarily correlated with the presence of an entrance bar. Species richness varied with geographical location for both macrofauna and fishes, with highest numbers of species occurring in the Furneaux Group, north-eastern Tasmania and south-eastern Tasmania. These patterns primarily reflected differences in estuary type between regions rather than concentrations of locally endemic species. Although the majority of species collected during the study were marine vagrants, they constituted a very low proportion of total animal densities within estuaries. Only four species considered exotic to Tasmania were identifed. Nearly all species recorded from Tasmanian estuaries occurred widely within the state and have also been recorded in south-eastern Australia. Only 1{\%} of estuarine fish species and {\textless}5{\%} of invertebrate species were considered endemic to the state. The generally wide ranges of species around Tasmania were complicated by (i) the absence of most species from the west coast (ii) a small ({\textless}10{\%}) component of species that occurred only in the north-east and Furneaux Group (eastern Bass Strait), and (iii) a few species ({\textless} 5{\%}) restricted to other regions. The low number of species recorded from estuaries along the western Tasmanian coast reflected extremely low faunal biomass in that area. This depression in biomass on the west cease was attributed to unusually low concentrations of dissolved nutrients in rivers and dark tannin-stained waters which greatly restricted algal photosynthesis and primary production.},
author = {Edgar, G. J. and Barrett, N. S. and Last, P. R.},
doi = {10.1046/j.1365-2699.1999.00365.x},
isbn = {0305-0270},
issn = {03050270},
journal = {Journal of Biogeography},
keywords = {Biogeography,Estuaries,Fish,Macrofauna,Salinity,Seagrass,Tasmania},
number = {6},
pages = {1169--1189},
pmid = {5607774},
title = {{The distribution of macroinvertebrates and fishes in Tasmanian estuaries}},
volume = {26},
year = {1999}
}
@article{Lotze2002,
abstract = {The combined and interactive effects of climatic and ecological factors are rarely considered in marine communities. We designed a factorial field experiment to analyze (1) the interactive effects of ambient UV radiation and consumers; and (2) the effects of photosynthetically active radiation (PAR 400 to 700 nm), UVA (320 to 400 nm) and UVB (280 to 320 nm) radiation on a marine hard-bottom community in Nova Scotia, NW Atlantic. Species recruitment and succession on ceramic tiles were followed for 5 mo. We found strong negative UV effects on biomass and cover of the early colonizing macroalga Pilayella littoralis, whereas UVB was more harmful than UVA radiation. Consumers, mainly gammarid amphipods, increased P. littoralis biomass when UV was excluded, probably through fertilization. These initially strong and interacting UV and consumer effects on total biomass and cover diminished as species succession progressed. Species diversity was not affected by experimental treatments, but significant shifts in species composition occurred, especially at the recruitment stage. Red algae were most inhibited by UV, whereas sedentary invertebrates and some brown algae tended to increase under UV exposure. Consumers suppressed green and filamentous brown algae, but favored the other groups. Again, these effects diminished during the later stages of succession. We conclude that UV radiation can be a significant structuring force in early successional benthic communities, and that consumers can mediate its effects.},
author = {Lotze, Heike K. and Worm, Boris and Molis, Markus and Wahl, Martin},
doi = {10.3354/meps243057},
isbn = {0171-8630},
issn = {01718630},
journal = {Marine Ecology Progress Series},
keywords = {Community structure,Early life stages,Grazing,Productivity,Recruitment,Rocky shore,Species-specific sensitivity,UV stress},
pages = {57--66},
pmid = {5695717},
title = {{Effects of UV radiation and consumers on recruitment and succession of a marine macrobenthic community}},
volume = {243},
year = {2002}
}
@article{Pennekamp2018,
abstract = {Losses and gains in species diversity affect ecological stability1–7 and the sustainability of ecosystem functions and services8–13. Experiments and models have revealed positive, negative and no effects of diversity on individual components of stability, such as temporal variability, resistance and resilience2,3,6,11,12,14. How these stability components covary remains poorly understood15. Similarly, the effects of diversity on overall ecosystem stability16, which is conceptually akin to ecosystem multifunctionality17,18, remain unknown. Here we studied communities of aquatic ciliates to understand how temporal variability, resistance and overall ecosystem stability responded to diversity (that is, species richness) in a large experiment involving 690 micro-ecosystems sampled 19 times over 40 days, resulting in 12,939 samplings. Species richness increased temporal stability but decreased resistance to warming. Thus, two stability components covaried negatively along the diversity gradient. Previous biodiversity manipulation studies rarely reported such negative covariation despite general predictions of the negative effects of diversity on individual stability components3. Integrating our findings with the ecosystem multifunctionality concept revealed hump- and U-shaped effects of diversity on overall ecosystem stability. That is, biodiversity can increase overall ecosystem stability when biodiversity is low, and decrease it when biodiversity is high, or the opposite with a U-shaped relationship. The effects of diversity on ecosystem multifunctionality would also be hump- or U-shaped if diversity had positive effects on some functions and negative effects on others. Linking the ecosystem multifunctionality concept and ecosystem stability can transform the perceived effects of diversity on ecological stability and may help to translate this science into policy-relevant information.},
author = {Pennekamp, Frank and Pontarp, Mikael and Tabi, Andrea and Altermatt, Florian and Alther, Roman and Choffat, Yves and Fronhofer, Emanuel A. and Ganesanandamoorthy, Pravin and Garnier, Aur{\'{e}}lie and Griffiths, Jason I. and Greene, Suzanne and Horgan, Katherine and Massie, Thomas M. and M{\"{a}}chler, Elvira and Palamara, Gian Marco and Seymour, Mathew and Petchey, Owen L.},
doi = {10.1038/s41586-018-0627-8},
isbn = {1476-4687 (Electronic)
0028-0836 (Linking)},
issn = {14764687},
journal = {Nature},
number = {7729},
pages = {109--112},
pmid = {30333623},
title = {{Biodiversity increases and decreases ecosystem stability}},
url = {http://www.nature.com/articles/s41586-018-0627-8},
volume = {563},
year = {2018}
}
@article{Dormann2018,
abstract = {Ten international laboratories specializing in the determination of marine pigment concentrations using high performance liquid chromatography (HPLC) were intercompared using in situ samples and a mixed pigment sample. Although prior Sea-viewing Wide Field-of-view Sensor (SeaWiFS) High Performance Liquid Chromatography (HPLC) Round-Robin Experiment (SeaHARRE) activities conducted in open-ocean waters covered a wide dynamic range in productivity, and some of the samples were collected in the coastal zone, none of the activities involved exclusively coastal samples. Consequently, SeaHARRE-4 was organized and executed as a strictly coastal activity and the field samples were collected from primarily eutrophic waters within the coastal zone of Denmark. The more restrictive perspective limited the dynamic range in chlorophyll concentration to approximately one and a half orders of magnitude (previous activities covered more than two orders of magnitude). The method intercomparisons were used for the following objectives: a) estimate the uncertainties in quantitating individual pigments and higher-order variables formed from sums and ratios; b) confirm if the chlorophyll a accuracy requirements for ocean color validation activities (approximately 25{\%}, although 15{\%} would allow for algorithm refinement) can be met in coastal waters; c) establish the reduction in uncertainties as a result of applying QA procedures; d) show the importance of establishing a properly defined referencing system in the computation of uncertainties; e) quantify the analytical benefits of performance metrics, and f) demonstrate the utility of a laboratory mix in understanding method performance. In addition, the remote sensing requirements for the in situ determination of total chlorophyll a were investigated to determine whether or not the average uncertainty for this measurement is being satisfied.},
author = {Dormann, Carsten F. and Calabrese, Justin M. and Guillera-Arroita, Gurutzeta and Matechou, Eleni and Bahn, Volker and Barto{\'{n}}, Kamil and Beale, Colin M. and Ciuti, Simone and Elith, Jane and Gerstner, Katharina and Guelat, J{\'{e}}r{\^{o}}me and Keil, Petr and Lahoz-Monfort, Jos{\'{e}} J. and Pollock, Laura J. and Reineking, Bj{\"{o}}rn and Roberts, David R. and Schr{\"{o}}der, Boris and Thuiller, Wilfried and Warton, David I. and Wintle, Brendan A. and Wood, Simon N. and W{\"{u}}est, Rafael O. and Hartig, Florian},
doi = {10.1002/ecm.1309},
isbn = {Technical Memorandum NASA/TM-2005-212787},
issn = {15577015},
journal = {Ecological Monographs},
keywords = {AIC weights,ensemble,model averaging,model combination,nominal coverage,prediction averaging,uncertainty},
number = {4},
pages = {485--504},
pmid = {11317470},
title = {{Model averaging in ecology: a review of Bayesian, information-theoretic, and tactical approaches for predictive inference}},
url = {http://doi.wiley.com/10.1002/ecm.1309},
volume = {88},
year = {2018}
}
@article{Bieg2018,
abstract = {There is a widely acknowledged need to explicitly include humans in our conceptual and mathematical models of food webs. However, a simple and generalized method for incorporating humans into fisheries food webs has yet to be established. We developed a simple graphical framework for defining whole-system inland fishery food webs that includes a continuum of fishery behaviors. This range of behaviors mimics those of generalist to specialist predators, which differentially influence ecosystem diversity, sustainability, and functioning. Fishery behaviors in this food-web context are predicted to produce a range of “fishery types” – from targeted (ie specialist) to multispecies (ie generalist) inland fisheries – and relate to the socioeconomic status of fishery participants. Fishery participants in countries with low Human Development Index (HDI) values are highly connected through fisheries food webs relative to humans in more developed countries. Our framework shows that fisheries can occupy a variety of roles within a food-web model and may thereby affect food-web stability in different ways. This realization could help to improve sustainable fisheries management at a global scale.},
author = {Bieg, Carling and McCann, Kevin S. and McMeans, Bailey C. and Rooney, Neil and Holtgrieve, Gordon W. and Lek, Sovan and Bun, Ngor Peng and Krishna, B. K.C. and Fraser, Evan},
doi = {10.1002/fee.1933},
issn = {15409309},
journal = {Frontiers in Ecology and the Environment},
number = {7},
pages = {412--420},
title = {{Linking humans to food webs: a framework for the classification of global fisheries}},
url = {http://doi.wiley.com/10.1002/fee.1933},
volume = {16},
year = {2018}
}
@article{Grueber2011,
abstract = {Information theoretic approaches and model averaging are increasing in popularity, but this approach can be difficult to apply to the realistic, complex models that typify many ecological and evolutionary analyses. This is especially true for those researchers without a formal background in information theory. Here, we highlight a number of practical obstacles to model averaging complex models. Although not meant to be an exhaustive review, we identify several important issues with tentative solutions where they exist (e.g. dealing with collinearity amongst predictors; how to compute model-averaged parameters) and highlight areas for future research where solutions are not clear (e.g. when to use random intercepts or slopes; which information criteria to use when random factors are involved). We also provide a worked example of a mixed model analysis of inbreeding depression in a wild population. By providing an overview of these issues, we hope that this approach will become more accessible to those investigating any process where multiple variables impact an evolutionary or ecological response.},
author = {Grueber, C. E. and Nakagawa, S. and Laws, R. J. and Jamieson, I. G.},
doi = {10.1111/j.1420-9101.2010.02210.x},
isbn = {1420-9101},
issn = {1010061X},
journal = {Journal of Evolutionary Biology},
keywords = {Akaike Information Criterion,Generalized linear mixed models,Inbreeding,Information theory,Lethal equivalents,Model averaging,Random factors,Standardized predictors},
number = {4},
pages = {699--711},
pmid = {21272107},
title = {{Multimodel inference in ecology and evolution: Challenges and solutions}},
volume = {24},
year = {2011}
}
@article{Jansson1967,
author = {Jansson, Ann-mari},
journal = {Helgol{\"{a}}nder Wissenschaftliche Meeresuntersuchungen},
number = {1-4},
pages = {574--588},
title = {{The food-web of the Cladophora-belt fauna}},
volume = {15},
year = {1967}
}
@article{Patrick2018,
abstract = {Accurately characterizing spatial patterns on landscapes is necessary to understand the processes that generate biodiversity, a problem that has applications in ecological theory, conservation planning, ecosystem restoration, and ecosystem management. However, the measurement of biodiversity patterns and the ecological and evolutionary processes that underlie those patterns is highly dependent on the study unit size, boundary placement, and number of observations. These issues, together known as the modifiable areal unit problem, are well known in geography. These factors limit the degree to which results from different metacommunity and macro‐ecological studies can be compared to draw new inferences, and yet these types of comparisons are widespread in community ecology. Using aquatic community datasets, we demonstrate that spatial context drives analytical results when landscapes are sub‐divided. Next, we present a framework for using resampling and neighborhood smoothing to standardize datasets to allow for inferential comparisons. We then provide examples for how addressing these issues enhances our ability to understand the processes shaping ecological communities at landscape scales and allows for informative meta‐analytical synthesis. We conclude by calling for greater recognition of issues derived from the modifiable areal unit problem in community ecology, discuss implications of the problem for interpreting the existing literature, and identify tools and approaches for future research. This article is protected by copyright. All rights reserved.},
author = {Patrick, Christopher J. and Yuan, Lester L.},
doi = {10.1111/oik.05802},
issn = {16000706},
journal = {Oikos},
keywords = {biodiversity,metacommunity,spatial scale},
number = {3},
pages = {297--308},
title = {{The challenges that spatial context present for synthesizing community ecology across scales}},
url = {https://onlinelibrary.wiley.com/doi/abs/10.1111/oik.05802?af=R},
volume = {128},
year = {2019}
}
@misc{ICESweb,
author = {{International Council for the Exploration of the Sea (ICES)}},
title = {{Oceanography}},
year = {2018}
}
@article{Cirtwill2018FoodWebs,
abstract = {Many different concepts have been used to describe species' roles in food webs (i.e., the ways in which species participate in their communities as consumers and resources). As each concept focuses on a different aspect of food-web structure, it can be difficult to relate these concepts to each other and to other aspects of ecology. Here we use the Eltonian niche as an overarching framework, within which we summarize several commonly-used role concepts (degree, trophic level, motif roles, and centrality). We focus mainly on the topological versions of these concepts but, where dynamical versions of a role concept exist, we acknowledge these as well. Our aim is to highlight areas of overlap and ambiguity between different role concepts and to describe how these roles can be used to group species according to different strategies (i.e., equivalence and functional roles). The existence of “gray areas” between role concepts make it essential for authors to carefully consider both which role concept(s) are most appropriate for the analyses they wish to conduct and what aspect of species' niches (if any) they wish to address. The ecological meaning of differences between species' roles can change dramatically depending on which role concept(s) are used.},
author = {Cirtwill, Alyssa R. and {Dalla Riva}, Giulio Valentino and Gaiarsa, Marilia P. and Bimler, Malyon D. and Cagua, E. Fernando and Coux, Camille and Dehling, D. Matthias},
doi = {10.1016/j.fooweb.2018.e00093},
file = {:Users/alyssacirtwill/Documents/Papers/Cirtwill et al.{\_}2018{\_}Food Webs(2).pdf:pdf},
issn = {23522496},
journal = {Food Webs},
keywords = {Eltonian niche,Network structure},
pages = {e00093},
publisher = {Elsevier Inc.},
title = {{A review of species role concepts in food webs}},
url = {https://doi.org/10.1016/j.fooweb.2018.e00093},
volume = {16},
year = {2018}
}
@article{Gardner1970,
author = {Gardner, Mark R. and Ashby, W. Ross},
journal = {Nature},
pages = {1--2},
title = {{228 - 5282.Indd}},
volume = {228},
year = {2004}
}
@article{Wootton2016a,
abstract = {Given current levels of biodiversity loss and environmental change, studies of how food webs respond to disturbance should broaden their focus beyond short-term disturbances to explore the effects of long-term, “press” disturbances. Press disturbances often disproportionately impact one or a few species, but these impacts invariably propagate to the remaining species in the food web. Additionally, the way species interact with each other within the food web influences the impact they have on the rest of the food web if it is disturbed. Here, we investigate the effect of species-level press disturbances in a large set of model food webs. We simulated disturbances as a reduction in growth rate of a single species within the food web, which is analogous to a targeted disturbance such as selective fishing. In these simulations, we were particularly interested in the resistance of the food web—the magnitude of disturbance it could tolerate before any species went extinct. We found that more highly connected and biodiverse food webs had lower resistance and were more likely to lose species at a low level of disturbance than sparsely connected food webs with few species. Food-web complexity also influenced which species were likely to go extinct due to the disturbance. At low species richness and/or low connectance, food webs could tolerate a large disturbance, and it was usually the focal species which went extinct. In contrast, webs were less stable at higher levels of complexity and a small disturbance rapidly propagated and caused the extinction of a non-focal species. Lastly, the disturbed species' traits were also important: Disturbance of a species with few interactions usually resulted in its own extinction, while disturbance of a species with many interactions more often caused the extinction of the disturbed species' predator(s). Likewise, the trophic level of the disturbed species influenced which species went extinct, although this was modulated by the complexity of the food web. Overall, our study indicates that both the traits of disturbed species and the complexity of the food web need to be considered in attempts to predict or manage the ecological impact of press disturbances.},
author = {Wootton, K. L. and Stouffer, D. B.},
doi = {10.1002/ecs2.1518},
isbn = {9781943580019},
issn = {21508925},
journal = {Ecosphere},
keywords = {Degree,Extinction,Resistance,Stability,Trophic level},
number = {11},
pages = {1--13},
title = {{Species' traits and food-web complexity interactively affect a food web's response to press disturbance}},
volume = {7},
year = {2016}
}
@article{Woodland2016,
abstract = {Many consumers display flexible feeding strategies that vary among individuals or populations, through life-history, or spatiotemporally. Despite the recognized influence of flexible feeding on the structure and dynamics of food webs, the consequences of these feeding strategies on the actual shape and characteristics of trophic position distributions have received less attention. We proposed and tested several a priori hypotheses to predict the likely effect of niche-dependent (e.g. herbivore, secondary consumer) foraging on the shape and statistical properties of consumer trophic position distributions using natural abundance stable isotope data from a diverse dataset of consumers. We found evidence that the structural characteristics of consumer trophic position distributions varied as a function of trophic niche. Herbivores and tertiary consumers tended to be ‘packed' closely near their mean trophic position, with few individuals realizing trophic positions markedly higher or lower than the mean. Conversely, secondary consumers often displayed broad trophic position distributions with many individuals dispersed away from the center of the distribution. We examined the effect of applying constant versus dynamic isotope trophic fractionation models and found that both models yielded similar although not identical results. Our findings suggest that trophic level omnivory supports a larger fraction of consumer diet at intermediate trophic positions than at either the lowest or the highest positions in aquatic food webs. These results suggest that vertical trophic niche declines among higher order consumers despite general evidence that the range of potential foraging options (i.e. horizontal trophic niche) tends to increase at higher trophic positions. Although further work is needed to test the generality of these patterns in other ecosystems, proactively examining trophic position distributions and reporting appropriate measures of central tendency (e.g. arithmetic versus geometric means) will increase the accuracy of individual trophic studies as well as the applicability of results for meta-analytical food web models. This article is protected by copyright. All rights reserved.},
author = {Woodland, Ryan J. and Warry, Fiona Y. and Woodland, Ryan J. and Evrard, Victor and Clarke, Rohan H. and Reich, Paul and Cook, Perran L.M.},
doi = {10.1111/oik.02486},
isbn = {1600-0706},
issn = {16000706},
journal = {Oikos},
keywords = {Stockholm spiders},
mendeley-tags = {Stockholm spiders},
number = {4},
pages = {556--565},
title = {{Niche-dependent trophic position distributions among primary, secondary and tertiary consumers}},
volume = {125},
year = {2016}
}
@article{Baiser2016,
abstract = {The assembly of local communities from regional pools is a multifaceted process that involves the confluence of interactions and environmental conditions at the local scale and biogeographic and evolutionary history at the regional scale. Understanding the relative influence of these factors on community structure has remained a challenge and mechanisms driving community assembly are often inferred from patterns of taxonomic, functional, and phylogenetic diversity. Moreover, community assembly is often viewed through the lens of competition and rarely includes trophic interactions or entire food webs. Here, we use motifs – subgraphs of nodes (e.g. species) and links (e.g. predation) whose abundance within a network deviates significantly as compared to a random network topology – to explore the assembly of food web networks found in the leaves of the northern pitcher plant Sarracenia purpurea. We compared counts of three-node motifs across a hierarchy of scales to a suite of null models to determine if motifs are over-, under-, or randomly represented. We then assessed if the pattern of representation of a motif in a given network matched that of the network it was assembled from. We found that motif representation in over 70{\%} of site networks matched the continental network they were assembled from and over 75{\%} of local networks matched the site networks they were assembled from for the majority of null models. This suggests that the same processes are shaping networks across scales. To generalize our results and effectively use a motif perspective to study community assembly, a theoretical framework detailing potential mechanisms for all possible combinations of motif representation is necessary.},
author = {Baiser, Benjamin and Elhesha, Rasha and Kahveci, Tamer},
doi = {10.1111/oik.02532},
isbn = {1600-0706},
issn = {16000706},
journal = {Oikos},
number = {4},
pages = {480--491},
title = {{Motifs in the assembly of food web networks}},
volume = {125},
year = {2016}
}
@article{Schwarzmuller2015,
author = {Schwarzm{\"{u}}ller, Florian Andreas},
number = {October},
title = {{Global change effects on the stability of food-web motifs Dissertation zur Erlangung des Doktorgrades der}},
year = {2015}
}
@article{Delmas2017,
abstract = {1. Food webs are the backbone upon which biomass flows through ecosystems. Dynamical models of biomass can reveal how the structure of food webs is involved in many key ecosystem properties, such as persistence, stability , etc. 2. Inthiscontribution,wepresent BioEnergeticFoodWebs,animplementationofYodzis{\&}Innes(TheAmeri-canNaturalist139,11511175,1992)bio-energeticmodel,inthehigh-performancecomputinglanguageJulia. 3. We illustrate how this package can be used to conduct numerical experiments in a reproducible and standard way. 4. A reference implementation of this widely used model will ease reproducibility and comparison of results across studies.},
author = {Delmas, Eva and Brose, Ulrich and Gravel, Dominique and Stouffer, Daniel B. and Poisot, Timoth{\'{e}}e},
doi = {10.1111/2041-210X.12713},
isbn = {4955139574},
issn = {2041210X},
journal = {Methods in Ecology and Evolution},
keywords = {community ecology,food webs,modelling},
number = {7},
pages = {881--886},
pmid = {28199780},
title = {{Simulations of biomass dynamics in community food webs}},
volume = {8},
year = {2017}
}
@article{Ameztegui2017,
abstract = {1. Despite being instrumental in forest ecology, the definition and nature of shade tolerance are complex and not beyond controversy. Moreover, the role it plays in the trait–demography relationship remains unclear. 2. Here, we hypothesize that shade tolerance can be achieved by alternative combinations of traits depending on the species' functional group (evergreen vs. deciduous species) and that its ability to explain the array of traits involved in demography will also vary between these two groups. 3. We used dimension reduction to identify the main trait spectra for 48 tree species, including 23 evergreens and 25 deciduous – dispersed across 21 genera and 13 families. We assessed the relationship between functional traits, shade tolerance, and demographic performance at high and low light using structural equation modelling. 4. The dimensions found corresponded to the trait spectra previously observed in the literature and were significantly related to measures of demography. However, our results support the existence of a divergence between evergreen and deciduous species in the way shade tolerance relates to the demography of species along light gradients. 5. We show that shade tolerance can be attained through different combination of traits depending on the functional and geographical context, and thus, its utilization as a predictor of forest dynamics and species coexistence requires previous knowledge on the role it plays in the demographic performance of the species under study.},
author = {Ameztegui, Aitor and Paquette, Alain and Shipley, Bill and Heym, Michael and Messier, Christian and Gravel, Dominique},
doi = {10.1111/1365-2435.12804},
issn = {13652435},
journal = {Functional Ecology},
keywords = {SORTIE,boreal forests,demographic performance,functional ecology,structural equation modelling,temperate forests,trait spectra,tree life-histories},
number = {4},
pages = {821--830},
title = {{Shade tolerance and the functional trait: demography relationship in temperate and boreal forests}},
volume = {31},
year = {2017}
}
@article{Borrelli2015a,
abstract = {Food web structure can be characterized by the particular frequencies of subgraphs found within them. Although there are thirteen possible configurations of three species subgraphs, some are consistently over-represented in empirical food webs. This is a robust pattern that is found across marine, freshwater or terrestrial environments. The preferential elimination of unstable subgraphs during the assembly of the food web can explain the observed pattern. It follows from this hypothesis that there should be differences in the stability of different subgraphs, and that stability should be positively correlated to their frequency in food webs. Using 50 food webs collected from a variety of databases I determined the frequency of each of the thirteen possible subgraphs with respect to randomized webs. Then by numerical simulation I determined the quasi sign stability (QSS) of each subgraph. My results clearly show a positive correlation between QSS and over-representation of the different subgraphs in empirical food webs.},
author = {Borrelli, Jonathan J.},
doi = {10.1111/oik.02176},
file = {:Users/alyssacirtwill/Documents/Papers/Borrelli{\_}2015{\_}Oikos.pdf:pdf},
isbn = {1600-0706},
issn = {16000706},
journal = {Oikos},
number = {12},
pages = {1583--1588},
title = {{Selection against instability: Stable subgraphs are most frequent in empirical food webs}},
volume = {124},
year = {2015}
}
@article{Smith2015,
abstract = {Neutel {\&} Thorne (Ecology Letters, 17:651–661, June 2014) provide an approximation for the leading eigenvalue of a food web community matrix involving coefficients of its characteristic polynomial. Though valuably incorporating three-way species interactions, two critical problems emerge when one considers the dimensions of the system, calling the approach's accuracy and precision into question},
author = {Smith, Matthew J. and Sander, Elizabeth and Barab{\'{a}}s, Gy{\"{o}}rgy and Allesina, Stefano},
doi = {10.1111/ele.12416},
isbn = {1461-0248 (Electronic)$\backslash$r1461-023X (Linking)},
issn = {14610248},
journal = {Ecology Letters},
keywords = {Complexity,Food webs,Negative feedback loop,Population dynamics,Stability},
number = {6},
pages = {593--595},
pmid = {25847355},
title = {{Stability and feedback levels in food web models}},
volume = {18},
year = {2015}
}
@article{Dambrot2017,
author = {Dambrot, Mason},
number = {August},
pages = {1--6},
title = {{Trophic coherence explains why networks have few feedback loops and high stability}},
year = {2017}
}
@article{Rooney2012,
abstract = {Given the unprecedented rate of species extinctions facing the planet, understanding the causes and consequences of species diversity in ecosystems is of paramount importance. Ecologists have investigated both the influence of environmental variables on species diversity and the influence of species diversity on ecosystem function and stability. These investigations have largely been carried out without taking into account the overarching stabilizing structures of food webs that arise from evolutionary and successional processes and that are maintained through species interactions. Here, we argue that the same large-scale structures that have been purported to convey stability to food webs can also help to understand both the distribution of species diversity in nature and the relationship between species diversity and food web stability. Specifically, the allocation of species diversity to slow energy channels within food webs results in the skewed distribution of interactions strengths that has been shown to confer stability to complex food webs. We end by discussing the processes that might generate and maintain the structured, stable and diverse food webs observed in nature. {\textcopyright} 2011.},
archivePrefix = {arXiv},
arxivId = {arXiv:1408.1149},
author = {Rooney, Neil and McCann, Kevin S.},
doi = {10.1016/j.tree.2011.09.001},
eprint = {arXiv:1408.1149},
isbn = {0169-5347},
issn = {01695347},
journal = {Trends in Ecology and Evolution},
number = {1},
pages = {40--46},
pmid = {21944861},
title = {{Integrating food web diversity, structure and stability}},
volume = {27},
year = {2012}
}
@article{Rip2010,
abstract = {Large-scale changes to the world's ecosystem are resulting in the deterioration of biostructure-the complex web of species interactions that make up ecological communities. A difficult, yet crucial task is to identify food web structures, or food web motifs, that are the building blocks of this baroque network of interactions. Once identified, these food web motifs can then be examined through experiments and theory to provide mechanistic explanations for how structure governs ecosystem stability. Here, we synthesize recent ecological research to show that generalist consumers coupling resources with different interaction strengths, is one such motif. This motif amazingly occurs across an enormous range of spatial scales, and so acts to distribute coupled weak and strong interactions throughout food webs. We then perform an experiment that illustrates the importance of this motif to ecological stability. We find that weak interactions coupled to strong interactions by generalist consumers dampen strong interaction strengths and increase community stability. This study takes a critical step by isolating a common food web motif and through clear, experimental manipulation, identifies the fundamental stabilizing consequences of this structure for ecological communities.},
author = {Rip, Jason M.K. and McCann, Kevin S. and Lynn, Denis H. and Fawcett, Sonia},
doi = {10.1098/rspb.2009.2191},
isbn = {0962-8452},
issn = {14712970},
journal = {Proceedings of the Royal Society B: Biological Sciences},
keywords = {Interaction strength,Module,Motif,Stability,Weak interactions},
number = {1688},
pages = {1743--1749},
pmid = {20129988},
title = {{An experimental test of a fundamental food web motif}},
volume = {277},
year = {2010}
}
@article{Kuiper2015,
abstract = {A principal aim of ecologists is to identify critical levels of environmental change beyond which ecosystems undergo radical shifts in their functioning. Both food-web theory and alternative stable states theory provide fundamental clues to mechanisms conferring stability to natural systems. Yet, it is unclear how the concept of food-web stability is associated with the resilience of ecosystems susceptible to regime change. Here, we use a combination of food web and ecosystem modelling to show that impending catastrophic shifts in shallow lakes are preceded by a destabilizing reorganization of interaction strengths in the aquatic food web. Analysis of the intricate web of trophic interactions reveals that only few key interactions, involving zooplankton, diatoms and detritus, dictate the deterioration of food-web stability. Our study exposes a tight link between food-web dynamics and the dynamics of the whole ecosystem, implying that trophic organization may serve as an empirical indicator of ecosystem resilience.},
author = {Kuiper, Jan J. and {Van Altena}, Cassandra and {De Ruiter}, Peter C. and {Van Gerven}, Luuk P.A. and Janse, Jan H. and Mooij, Wolf M.},
doi = {10.1038/ncomms8727},
isbn = {2041-1723},
issn = {20411723},
journal = {Nature Communications},
pages = {1--7},
pmid = {26173798},
publisher = {Nature Publishing Group},
title = {{Food-web stability signals critical transitions in temperate shallow lakes}},
url = {http://dx.doi.org/10.1038/ncomms8727},
volume = {6},
year = {2015}
}
@article{Klaise2017,
abstract = {Food webs have been found to exhibit remarkable motif profiles, patterns in the relative prevalences of all possible three-species sub-graphs, and this has been related to ecosystem properties such as stability and robustness. Analysing 46 food webs of various kinds, we find that most food webs fall into one of two distinct motif families. The separation between the families is well predicted by a global measure of hierarchical order in directed networks - trophic coherence. We find that trophic coherence is also a good predictor for the extent of omnivory, defined as the tendency of species to feed on multiple trophic levels. We compare our results to a network assembly model that admits tunable trophic coherence via a single free parameter. The model is able to generate food webs in either of the two families by varying this parameter, and correctly classifies almost all the food webs in our database. This establishes a link between global order and local preying patterns in food webs.},
archivePrefix = {arXiv},
arxivId = {1609.04318},
author = {Klaise, Janis and Johnson, Samuel},
doi = {10.1038/s41598-017-15496-1},
eprint = {1609.04318},
isbn = {0217-2445},
issn = {20452322},
journal = {Scientific Reports},
number = {1},
pages = {1--11},
publisher = {Springer US},
title = {{The origin of motif families in food webs}},
url = {http://dx.doi.org/10.1038/s41598-017-15496-1},
volume = {7},
year = {2017}
}
@article{Gross2009,
abstract = {Insights into what stabilizes natural food webs have always been limited by a fundamental dilemma: Studies either need to make unwarranted simplifying assumptions, which undermines their relevance, or only examine few replicates of small food webs, which hampers the robustness of findings. We used generalized modeling to study several billion replicates of food webs with nonlinear interactions and up to 50 species. In this way, first we show that higher variability in link strengths stabilizes food webs only when webs are relatively small, whereas larger webs are instead destabilized. Second, we reveal a new power law describing how food-web stability scales with the number of species and their connectance. Third, we report two universal rules: Food-web stability is enhanced when (i) species at a high trophic level feed on multiple prey species and (ii) species at an intermediate trophic level are fed upon by multiple predator species.},
author = {Gross, Thilo and Rudolf, Lars and Levin, Simon A and Dieckmann, Ulf},
journal = {Science},
number = {5941},
pages = {747--750},
title = {{Generalized Models Reveal Stabilizing Factors in Food Webs$\backslash$r10.1126/science.1173536}},
url = {http://www.sciencemag.org/cgi/content/abstract/325/5941/747},
volume = {325},
year = {2009}
}
@article{Monteiro2016,
abstract = {In this paper, we analyzed the occurrence of motifs (modules) in empirical food-webs from different ecosystem types. Differently from previous studies, our analysis did not relied on randomized networks with specific a priori assumptions, which has been demonstrated to produce inconsistent patterns. We aimed to evaluate the interplay between population dynamics and food-web topology, and its consequences to module occurrences in complex food-webs. We evaluated 13 arrangements of three-species modules and 199 arrangements of four-species modules. For each module, we assembled, a corresponding Jacobian predation matrix, and evaluated the arrangements expected to persist after a disturbance in the equilibrium of the populations dynamics (local stability). Our general results were that (1) a limited set of stable arrangements occurs most frequently; (2) the omnivory module is the only three-species module expected to occur both in the stable and unstable region; (3) connectance and omnivory affects the proportion of stable modules; and (4) the type of ecosystem influence the proportion of stable modules. Further, we demonstrated that food-web topology and population dynamics influenced module occurrences in natural communities; presented a function for the ways that local stability increases the probability of module occurrence; and highlighted the use of omnivory degree to access the effect of feeding at more than one trophic level on food-web stability.},
author = {Monteiro, Angelo B. and Faria, Lucas Del Bianco},
doi = {10.1016/j.jtbi.2016.09.006},
issn = {10958541},
journal = {Journal of Theoretical Biology},
keywords = {Compartmentalization,Complexity/stability,Food-web structure,Local stability,Network,Omnivory},
pages = {165--171},
publisher = {Elsevier},
title = {{The interplay between population stability and food-web topology predicts the occurrence of motifs in complex food-webs}},
url = {http://dx.doi.org/10.1016/j.jtbi.2016.09.006},
volume = {409},
year = {2016}
}
@article{Borrelli2014,
abstract = {Food chains are short, rarely more than five trophic levels long. The cause of this pattern remains unresolved, and no current hypothesis fully explains this phenomenon. We offer an explanation based on the stability of food chains that have been shifted away from linearity to be more web-like. We start with a simple example of food webs of two to six species arranged so that species consume all those with a trophic level less than their own. The probability of stability, for such "universal omnivory" chains declined strongly with chain length, and was as low as 1{\%} with six level chains but highest for two and three level chains. We further explored the influence of chain length on food web stability by testing food webs with varying levels of connectance that were constructed either randomly or with the niche model. By additionally altering the relative impacts of predators on prey, and vice-versa, we test the role of our assumptions on the relationship between chain length and stability. Food webs characterized by low to moderate degrees of connectance, asymmetrical interactions, and relatively weak density dependence showed a pattern of reduced stability with longer trophic chains. The simple view that food webs characterized by long trophic chains are less stable seems to resolve the long-standing question of why there are so few trophic levels in nature.},
author = {Borrelli, Jonathan J. and Ginzburg, Lev R.},
doi = {10.1016/j.fooweb.2014.11.002},
isbn = {23522496},
issn = {23522496},
journal = {Food Webs},
keywords = {Food chain,Food web,Stability,Trophic level},
number = {1-4},
pages = {10--17},
publisher = {Elsevier Inc.},
title = {{Why there are so few trophic levels: Selection against instability explains the pattern}},
url = {http://dx.doi.org/10.1016/j.fooweb.2014.11.002},
volume = {1},
year = {2014}
}
@article{Kefi2015,
abstract = {How multiple types of non-trophic interactions map onto trophic networks in real communities remains largely unknown. We present the first effort, to our knowledge, describing a comprehensive ecological network that includes all known trophic and diverse non-trophic links among {\textgreater}100 coexisting species for the marine rocky intertidal community of the central Chilean coast. Our results suggest that non-trophic interactions exhibit highly nonrandom structures both alone and with respect to food web structure. The occurrence of different types of interactions, relative to all possible links, was well predicted by trophic structure and simple traits of the source and target species. In this community, competition for space and positive interactions related to habitat/refuge provisioning by sessile and/or basal species were by far the most abundant non-trophic interactions. If these patterns are corroborated in other ecosystems, they may suggest potentially important dynamic constraints on the combined architec...},
author = {K{\'{e}}fi, Sonia and Berlow, Eric L. and Wieters, Evie A. and Joppa, Lucas N. and Wood, Spencer A. and Brose, Ulrich and Navarrete, Sergio A.},
doi = {10.1890/13-1424.1},
isbn = {0012-9658},
issn = {00129658},
journal = {Ecology},
keywords = {Competition,Ecological communities,Ecological networks,Facilitation,Food webs,Multiplex networks,Mutualism,Negative interaction,Positive interaction,Rocky intertidal},
number = {1},
pages = {291--303},
pmid = {26236914},
title = {{Network structure beyond food webs: Mapping non-trophic and trophic interactions on Chilean rocky shores}},
volume = {96},
year = {2015}
}
@article{Foote1995,
abstract = {Information is given on the habitat distribution, larval feeding habits, and economic roles of the dipterous family Ephydridae. This family has broad ecological tolerances and is commonly encountered in such physiologically stressful habitats as oil pools, inland alkaline and saline marshes, hot springs, and cold thermal springs, and coastal salt marshes and mangrove swamps. Other ephydrid species occur in freshwater marshes, moist to xeric grasslands, muddy and sandy shores, and rocky substrates along streams and lakes. Larvae display adaptive radiation in their use of particulate food substrates, with different groups of species feeding on Cyanobacteria (blue-green algae), algae, detritus, and decomposing carcasses. A few species are predators, whereas others are leaf miners. Economically, certain species are pests in greenhouses and others damage crops such as rice and sugar beets. Other species are being investigated as agents of biological control of aquatic weeds.},
author = {Foote, B. A.},
doi = {10.1146/annurev.ento.40.1.417},
isbn = {0066-4170},
issn = {00664170},
journal = {Annual Review of Entomology},
keywords = {diptera,economic importance,ephydridae,feeding habits,habitats},
number = {1},
pages = {417--442},
title = {{Biology of Shore Flies}},
url = {http://ento.annualreviews.org/cgi/doi/10.1146/annurev.ento.40.1.417},
volume = {40},
year = {2002}
}
@incollection{DeMoor1988,
abstract = {TRICHOPTERA: CADDISFLIES Definition -Trichoptera means hairy winged ('trichia' = hairs, 'ptera' = wings). They look rather like those moths which hold their wings over the back like a sharply-angled roof, but those have scales on the wings. The antennae are long and thread-like. -Larvae are aquatic (with one exception) and they make fixed or portable homes to live in. -The larvae have jaws for biting and chewing. Adults rarely feed. -The life cycle is complete :-egg, larva, pupa and adult. What they do {\&} where they live -They breed in many types of pond, lake or water-filled ditch where there is aquatic or marginal vegetation suitable as larval food and for case-making. Many specialise in streams and rivers, such as those that make nets for catching down-drift food items, including any small animals. Adults are found on water-side vegetation and tree foliage. -Providing waters are not polluted and have a reasonably natural-look (even if artificial) there is likely to be a caddis fauna. Many of the species characterise particular types of water ecology. Number of species -In Britain 189 species. -The world fauna is 7,000 species.},
address = {Cape Town},
author = {Marzolf, G. R.},
booktitle = {Ecology},
doi = {10.2307/1936088},
editor = {Lubke, R. A. and de Moor, Irene},
isbn = {9780900881688},
issn = {0028-0836},
number = {3},
pages = {648--649},
pmid = {999999},
publisher = {University of Cape Town Press},
title = {{Fresh-Water Invertebrates}},
url = {http://www.nature.com/doifinder/10.1038/186345b0},
volume = {60},
year = {2006}
}
@article{Kajak1979,
author = {Kajak, Zdzislaw and Rybak, Jadwiga},
doi = {10.1002/iroh.19790640310},
issn = {15222632},
journal = {Internationale Revue der gesamten Hydrobiologie und Hydrographie},
number = {3},
pages = {361--378},
title = {{The Feeding of Chaoborus flavicans Meigen (Diptera, Chaoboridae). and its Predation on Lake Zooplankton}},
volume = {64},
year = {1979}
}
@article{Weerekoon1953,
author = {Weerekoon, A. C.J.},
doi = {10.1111/j.1365-3032.1953.tb00649.x},
issn = {13653032},
journal = {Proceedings of the Royal Entomological Society of London. Series A, General Entomology},
number = {7-9},
pages = {85--92},
title = {{on the Behaviour of Certain Ceratopogonidae (Diptera)}},
volume = {28},
year = {1953}
}
@article{Olivier1971,
abstract = {Information on the life history of the Chironomidae is scattered through a wide variety of papers in many languages. The present article is based mainly on papers that have been published since 1950 as the earlier papers were reviewed thoroughly in Chironomus by Thienemann (101). Additional references and information on most of the topics covered here may be found in this excellent book. Even with this restriction a high degree of selection is necessary and most of the papers on genetics, physiology, pro­ ductivity and biomass, and fish food are not considered. Many of the papers cited have been chosen because they are most likely to be useful for fur­ ther reference.},
author = {Oliver, D. R.},
doi = {10.1146/annurev.en.16.010171.001235},
issn = {0066-4170},
journal = {Annual Review of Entomology},
number = {1},
pages = {211--230},
title = {{Life History of the Chironomidae}},
url = {http://www.annualreviews.org/doi/10.1146/annurev.en.16.010171.001235},
volume = {16},
year = {2003}
}
@article{Mora2018,
abstract = {Although the structure of empirical food webs can differ between ecosystems, there is growing evidence of multiple ways in which they also exhibit common topological properties. To reconcile these contrasting observations, we postulate the existence of a backbone of interactions underlying all ecological networks—a common substructure within every network comprised of species playing similar ecological roles—and a periphery of species whose idiosyncrasies help explain the differences between networks. To test this conjecture, we introduce a new approach to investigate the structural similarity of 411 food webs from multiple environments and biomes. We first find significant differences in the way species in different ecosystems interact with each other. Despite these differences, we then show that there is compelling evidence of a common backbone of interactions underpinning all food webs. We expect that identifying a backbone of interactions will shed light on the rules driving assembly of different ecological communities.},
author = {Mora, Bernat Bramon and Gravel, Dominique and Gilarranz, Luis J. and Poisot, Timoth{\'{e}}e and Stouffer, Daniel B.},
doi = {10.1038/s41467-018-05056-0},
isbn = {1545-4428},
issn = {20411723},
journal = {Nature Communications},
number = {1},
pages = {2603},
publisher = {Springer US},
title = {{Identifying a common backbone of interactions underlying food webs from different ecosystems}},
url = {http://www.nature.com/articles/s41467-018-05056-0},
volume = {9},
year = {2018}
}
@book{Andersen1976,
abstract = {SCF-MO  calculations  were carried  out  on  aniline,  N-methylaniline, N,Y-dimethylaniline,  2-methyl-dY,N-dimethylaniline, 2,6-dimethyl-N,N-dimethylaniline1 and 2,6-dimethylaniline, using the CNDO/2 approximation. The results  on  triphenylmethylation have established  the  sequence  (CH8)LV {\textgreater} CH,NH  {\textgreater} NHg  in  activating power.  This result indicates a  considerable  amount  of  charge transfer in the transition state in agreement with  the idea of  a late transition state. The  results  show further  that  the  tritylation  does not  proceed via direct  attack  at the para position. Nitrogen  inversion  barriers  were  calculated,  but were found  to be  too  high  (aniline, 6.4 kcal, experimentally estimated, 2  kcal).  The frontier  electron  theory  correctly  predicts  the reactivity  of  ortho-substituted  aniline and N,Y-dimethylaniline in electrophilic aromatic substitution.  Experiments on  the  tritylation of  2,6-dimethyl-aniline  and 2,6-dimethyl-N,hr-dimethylaniline  are reported. The calculations  show a decrease  in  the HOMO  energy levels in  this sequence. },
address = {Amsterdam},
author = {Eberhardt, Manfred K. and Chuchani, Gabriel},
booktitle = {Journal of Organic Chemistry},
doi = {10.1021/jo00796a018},
editor = {Cheng, Lanna},
isbn = {0022-3263},
issn = {15206904},
number = {23},
pages = {3649--3653},
publisher = {North-Holland Publishing Company},
title = {{Electrophilic Aromatic Triphenylmethylation. Self-Consistent Field-Molecular Orbital Calculations on Aniline, N-Methylaniline, N,N-Dimethylaniline, and Ortho-Substituted Anilines}},
url = {http://www.vliz.be/imisdocs/publications/224066.pdf{\#}page=500},
volume = {37},
year = {1972}
}
@article{Petrozhitskaya2010,
abstract = {The distribution of preimaginal stages of amphibiotic insects (mayflies, stoneflies, caddisflies, and black flies) in the running waters of the mountain-steppe landscapes of western Tuva has been investigated. The basin of the Hemchik River is 500-2200 meters above sea level. The taxonomic composition and spatial distribution are determined; higlnnountain, middle- mountain and low-nountain plain types of communities are detailed. The trophic structure of amphibiotic communities is analyzed along the ecological profiles from the upper to the lower reaches of the rivers. {\textcopyright} 2010 Pleiades Publishing, Ltd.},
author = {Petrozhitskaya, L. V. and Rodkina, V. I. and Zaika, V. V.},
doi = {10.1134/s1995082910020045},
isbn = {1995-0829},
issn = {1995-0829},
journal = {Inland Water Biology},
keywords = {Amphibiotic insects,Mountains and steppe,Rivers of Western Tuva,Spatial distribution,Taxonomic and trophic structures},
number = {2},
pages = {126--134},
title = {{Distribution of amphibiotic insects of different trophic groups in mountainous and steppe rivers of Western Tuva}},
url = {http://www.scopus.com/inward/record.url?eid=2-s2.0-77954171355{\&}partnerID=tZOtx3y1},
volume = {3},
year = {2012}
}
@article{Helcom2017,
abstract = {The purpose of this study was to investigate the role of the tibialis anterior (TA) in the walk-to-run transition (WRT) by means of an experimental manipulation that allows increasing or decreasing muscular effort of the TA around heel contact. Eight subjects performed five WRTs on an accelerating treadmill wearing a powered ankle-foot exoskeleton. There was a trend towards a lower WRT-speed in the condition in which the TA was resisted (2.06+/-0.09 m s(-1)) than in the control condition (2.10+/-0.10 m s(-1)). This finding could not be extrapolated in the opposite direction, as there was no significant difference between the assist and control condition. The TA activation burst around heel contact showed a pattern that led to the hypothesis that the TA activation reaches a critical level at the fourth last heel contact before the WRT which triggers the WRT. The fact that the results comply with previous transition studies emphasises the role of the TA as a determinant of the WRT.},
author = {och Vattenmyndighetens, Havs-},
doi = {10.1016/j.gaitpost.2008.05.016},
isbn = {0966-6362},
issn = {0966-6362},
journal = {Available at: http://stateofthebalticsea.helcom.fi},
keywords = {assessment,baltic,ecosystem,management,status},
number = {June 2017},
pages = {1--155},
pmid = {18620862},
title = {{State of the Baltic Sea}},
year = {2013}
}
@article{Meier2012,
abstract = {The combined future impacts of climate change and industrial and agricultural practices in the Baltic Sea catchment on the Baltic Sea ecosystem were assessed. For this purpose 16 transient simulations for 1961–2099 using a coupled physical-biogeochemical model of the Baltic Sea were performed. Four climate scenarios were combined with four nutrient load scenarios ranging from a pessimistic business-as-usual to a more optimistic case following the Baltic Sea Action Plan (BSAP). Annual and seasonal mean changes of climate parameters and ecological quality indicators describing the environmental status of the Baltic Sea like bottom oxygen, nutrient and phytoplankton concentrations and Secchi depths were studied. Assuming present-day nutrient concentrations in the rivers, nutrient loads from land increase during the twenty first century in all investigated scenario simulations due to increased volume flows caused by increased net precipitation in the Baltic catchment area. In addition, remineralization rates increase due to increased water temperatures causing enhanced nutrient flows from the sediments. Cause-and-effect studies suggest that both processes may play an important role for the biogeochemistry of eutrophicated seas in future climate partly counteracting nutrient load reduction efforts like the BSAP.},
author = {Meier, H. E.M. and Hordoir, R. and Andersson, H. C. and Dieterich, C. and Eilola, K. and Gustafsson, B. G. and H{\"{o}}glund, A. and Schimanke, S.},
doi = {10.1007/s00382-012-1339-7},
isbn = {0930-7575},
issn = {09307575},
journal = {Climate Dynamics},
keywords = {Baltic Sea,Baltic Sea Action Plan,Climate change,Eutrophication,Marine ecosystems,Numerical modeling,Scenarios},
number = {9-10},
pages = {2421--2441},
title = {{Modeling the combined impact of changing climate and changing nutrient loads on the Baltic Sea environment in an ensemble of transient simulations for 1961-2099}},
volume = {39},
year = {2012}
}
@article{Harvey2003,
abstract = {During the past decade, regional changes in the dynamics of the Atlanto-Iberian stock of sardine, and its exploitation by Portuguese and Spanish purse-seine fisheries, have increased the uncertainties in estimated trends of spawning biomass, stock abundance, and fishing mortality. Together with recent evidence for lack of discontinuities in the distribution of sardine eggs at the edges of the stock area, this casts doubts on the hypothesis that the stock is a panmictic, closed population. Sardine morphometric data (truss variables and landmark data) from 14 samples spanning the northeastern Atlantic and the western Mediterranean were analysed by multivariate and geometric methods. The analyses explored the homogeneity of sardine shape within the area studied, as well as its relation to that of adjacent and distant populations (Azores and northwestern Mediterranean). Principal components analysis on size-corrected truss variables and cluster analysis of mean fish shape using landmark data indicate that the shape of sardine off southern Iberia and Morocco is distinct from the shape of sardine in the rest of the area. The two groups of sardine are significantly separated by discriminant analysis, and their validity was confirmed by large percentages of correct classifications of test fish (87 and 86{\%} of fish from the test sample were correctly classified into each group, respectively). There was also some evidence that fish from the western Mediterranean and the Azores form a separate morphometric group. These results question both the homogeneity within the Atlanto- Iberian sardine stock and the validity of its current boundaries},
archivePrefix = {arXiv},
arxivId = {https://doi.org/10.1016/S1054-3139(03)00117-6},
author = {Swain, Douglas P and Benoı, Hugues P},
doi = {10.1016/S1054},
eprint = {/doi.org/10.1016/S1054-3139(03)00117-6},
isbn = {4415025242},
issn = {00282448},
journal = {ICES Journal of Marine Science},
keywords = {depth-dependent,diel variation in catchability,length-dependent,trawl survey},
number = {03},
pages = {1298--1317},
pmid = {2200},
primaryClass = {https:},
title = {{Accounting for length- and depth-dependent diel variation in catchability of fish and invertebrates in an annual bottom-trawl survey}},
url = {http://icesjms.oxfordjournals.org/content/60/6/1352.short},
volume = {3139},
year = {2003}
}
@article{McDonaldMadden2016,
abstract = {Food-web theory can be a powerful guide to the management of complex ecosystems. However, we show that indices of species importance common in food-web and network theory can be a poor guide to ecosystem management, resulting in significantly more extinctions than necessary. We use Bayesian Networks and Constrained Combinatorial Optimization to find optimal management strategies for a wide range of real and hypothetical food webs. This Artificial Intelligence approach provides the ability to test the performance of any index for prioritizing species management in a network. While no single network theory index provides an appropriate guide to management for all food webs, a modified version of the Google PageRank algorithm reliably minimizes the chance and severity of negative outcomes. Our analysis shows that by prioritizing ecosystem management based on the network-wide impact of species protection rather than species loss, we can substantially improve conservation outcomes.},
author = {McDonald-Madden, E. and Sabbadin, R. and Game, E. T. and Baxter, P. W.J. and Chad{\`{e}}s, I. and Possingham, H. P.},
doi = {10.1038/ncomms10245},
isbn = {2041-1723 (Electronic) 2041-1723 (Linking)},
issn = {20411723},
journal = {Nature Communications},
number = {May 2015},
pages = {1--8},
pmid = {26776253},
title = {{Using food-web theory to conserve ecosystems}},
volume = {7},
year = {2016}
}
@article{Sandberg2000,
abstract = {The brackish Baltic Sea has been seen as particularly suitable for studies of food webs. Compared to fully marine ecosystems, it has low species diversity, which means fewer trophic linkages to analyse. The Baltic Sea is also one of the best-studied areas of the world, suggesting that most data requirements for food web models should be fulfilled. Nevertheless, the influence of physical and biological factors on trophic interactions and biogeochemical patterns varies spatially in the Baltic Sea, adding considerable complexity to food web studies. Food web structure and processes can be described and compared quantitatively between areas by estimating the flow of matter or energy through the organisms. Most such models have been based on carbon, though studies of complementary flows of other elements limiting production, such as nitrogen and phosphorus would be desirable. However, since ratios between carbon and other elements are used in calculating these flows, it is crucial, as a first step, to quantify the flows of carbon as accurately as possible. In this study, we used the EcopathII software (ver 3.1) to analyse models of carbon flow through the food webs in the three main areas of the Baltic Sea; the Baltic proper, Bothnian Sea and Bothnian Bay. A previously published study on carbon flow in the Baltic Sea [Elmgren, R. 1984. Trophic dynamics in the enclosed, brackish Baltic Sea. Rapp. P.-V. Reun. Cons. Int. Explor. Mer. (183) 152-169.] was complemented with the data on respiration and flow to detritus [Wulff, F., Ulanowicz, R. 1989. A comparative anatomy of the Baltic Sea and Chesapeeake Bay ecosystems. In: F. Wulff, J.G. Field, K.H. Mann (Eds.), Flow Analysis of Marine Ecosystems: Theory and Practice. New York: Springer-Verlag.] in order to present complete mass balance models of carbon. The purpose of re-evaluating previous models with new analytic tools was to check how well their carbon flows balance, and to provide a basis for improved mass balance models using more recent data, including nutrients other than carbon. The resulting mass balance networks for the Baltic proper, Bothnian Sea and the Bothnian Bay were shown to deviate from steady state. There was an organic carbon surplus of 45, 25 and 18 g C m-2 year-1 in the pelagic zones of the Baltic proper, Bothnian Sea and Bothnian Bay, respectively. The Ecopath network analysis confirmed that the overall carbon flow was highest in the Baltic proper, somewhat lower in the Bothnian Sea and much lower in the Bothnian Bay. The only clear differences in food web structure between the basins was that the average trophic level was lower for demersal fish in the Bothnian Sea and higher for macrofauna in the Bothnian Bay, compared to the other basins. The analysis showed weakness in our current understanding in Baltic Sea food webs and highlighted areas where improvements could be made with more recent data. (C) 2000 Elsevier Science B.V.},
author = {Sandberg, J. and Elmgren, R. and Wulff, F.},
doi = {10.1016/S0924-7963(00)00019-1},
file = {:Users/alyssacirtwill/Documents/Papers/Sandberg, Elmgren, Wulff{\_}2000{\_}Journal of Marine Systems.pdf:pdf},
isbn = {0014-3820},
issn = {09247963},
journal = {Journal of Marine Systems},
keywords = {Baltic Sea,Carbon,Ecopath,Food web,Model,Re-evaluation},
number = {3-4},
pages = {249--260},
pmid = {15058728},
title = {{Carbon flows in Baltic Sea food webs - A re-evaluation using a mass balance approach}},
volume = {25},
year = {2000}
}
@article{Allesina2009a,
abstract = {A major challenge in ecology is forecasting the effects of species' extinctions, a pressing problem given current human impacts on the planet. Consequences of species losses such as secondary extinctions are difficult to forecast because species are not isolated, but interact instead in a complex network of ecological relationships. Because of their mutual dependence, the loss of a single species can cascade in multiple coextinctions. Here we show that an algorithm adapted from the one Google uses to rank web-pages can order species according to their importance for coextinctions, providing the sequence of losses that results in the fastest collapse of the network. Moreover, we use the algorithm to bridge the gap between qualitative (who eats whom) and quantitative (at what rate) descriptions of food webs. We show that our simple algorithm finds the best possible solution for the problem of assigning importance from the perspective of secondary extinctions in all analyzed networks. Our approach relies on network structure, but applies regardless of the specific dynamical model of species' interactions, because it identifies the subset of coextinctions common to all possible models, those that will happen with certainty given the complete loss of prey of a given predator. Results show that previous measures of importance based on the concept of "hubs" or number of connections, as well as centrality measures, do not identify the most effective extinction sequence. The proposed algorithm provides a basis for further developments in the analysis of extinction risk in ecosystems.},
author = {Allesina, Stefano and Pascual, Mercedes},
doi = {10.1371/journal.pcbi.1000494},
isbn = {1553-734X},
issn = {1553734X},
journal = {PLoS Computational Biology},
number = {9},
pages = {e1000494},
pmid = {19730676},
title = {{Googling food webs: Can an eigenvector measure species' importance for coextinctions?}},
volume = {5},
year = {2009}
}
@article{Jensen1906,
abstract = {I. Des fonctions convexes et concaves. Ddfinition. Exemples. Dans sa e{\~{}}l{\~{}}bre Analyse alg{\~{}}brique (note IX, pp. 457--59) CAucHY dgmontre que ,Aa moyenne gdom6trique entre plusieurs nombres est toujours infdrieure h leur moyenne alg{\~{}}brique,. La mdfhode employge par CAVCHY est extrgmement dldgante, et elle h passd sans changement dans tous los traitds d'analyse algdbrique. Elle eonsiste, comme on sait, en ceci, que, de l'in6galit6 --'(a+b), {\~{}}/{\~{}}b {\textless} 2 oh a et b sont des nombres positifs, on est conduit {\~{}}t l'in6galitd analogue pour quatre hombres, savoir {\~{}}/.b{\~{}}a {\textless} -' (a + b+ c + d), 4 et aux suivantes, pour 8, I6,..., 2 m nombres, apr{\~{}}s quoi ee nombre, par un artifice, est r6duit {\~{}} un hombre arbitraire inf6rieur, n. Cette m6thode simple a gt6 mon point de d6part dans les recherches suivantes, qui con-duisent, par une voie en r6alit6 tr6s simple et 616mentaire, {\~{}} des r6sultats 9 " imp generaux et non sans ort{\~{}}nee. 1 Oonf6rence faite {\~{}} la Soci6t6 math6matique danoise le x 7 janvier I9O S.},
author = {Jensen, J. L. W. V.},
doi = {10.1007/bf02418571},
issn = {0001-5962},
journal = {Acta Mathematica},
number = {0},
pages = {175--193},
title = {{Sur les fonctions convexes et les in{\'{e}}galit{\'{e}}s entre les valeurs moyennes}},
volume = {30},
year = {2006}
}
@article{Li2018,
abstract = {Aim: Ecological communities are composed of both species and the biotic relationships (interactions or spatial associations) among them. Biotic homogenization in species composition (i.e., increased site-to-site similarity) is recognized as a common consequence of global change, but less is known about how the similarity of species relationships changes over space and time. Does homogenization of species composition lead to homogenization of species relationships or are the dynamics of species relationships decoupled from changes in species composition? Location: Wisconsin, USA. Time period: 1950-2012. Major taxa studied: Vascular plants. Methods: We used long-term resurvey data to analyse changes in plant species association patterns between the 1950s and 2000s at 266 sites distributed among three community types in Wisconsin, USA. We used species associations (quantified via local co-occurrence patterns) to represent one type of relationship among species. Species pairs that co-occur more or less than expected by chance have positive or negative associations, respectively. We then measured beta diversity in both species composition and species association networks over time and space. Results: Shifts in species associations consistently exceeded the shifts observed in species composition. Less disturbed forests of northern Wisconsin have converged somewhat in species composition but little in species associations. In contrast, forests in central Wisconsin succeeding from pine barrens to closed-canopy forests have strongly homogenized in both species composition and species associations. More fragmented forests in southern Wisconsin also tended to converge in species composition and in the species' negative associations, but their positive associations diverged over the last half century. Species composition and associations are generally affected by a similar set of environmental variables. Their relative importance, however, has changed over time. Main conclusions: Long-term shifts in species relationships appear to be decoupled from shifts in species composition despite being affected by similar environmental variables. K E Y W O R D S beta diversity, long-term changes, networks, species composition, species co-occurrence, species interactions},
author = {Li, Daijiang and Poisot, Timoth{\'{e}}e and Waller, Donald M. and Baiser, Benjamin},
doi = {10.1111/geb.12825},
issn = {14668238},
journal = {Global Ecology and Biogeography},
keywords = {beta diversity,long-term changes,networks,species co-occurrence,species composition,species interactions},
number = {12},
pages = {1481--1491},
title = {{Homogenization of species composition and species association networks are decoupled}},
volume = {27},
year = {2018}
}
@article{Jumars2015,
abstract = {Polychaetes are common in most marine habitats and dominate many infaunal communities. Functional guild classification based on taxonomic identity and morphology has linked community structure to ecological function. The functional guilds now include osmotrophic siboglinids as well as sipunculans, echiurans, and myzostomes, which molecular genetic analyses have placed within Annelida. Advances in understanding of encounter mechanisms explicitly relate motility to feeding mode. New analyses of burrowing mechanics explain the prevalence of bilateral symmetry and blur the boundary between surface and subsurface feeding. The dichotomy between microphagous deposit and suspension feeders and macrophagous carnivores, herbivores, and omnivores is further supported by divergent digestive strategies. Deposit feeding appears to be limited largely to worms longer than 1 cm, with juveniles and small worms in general restricted to ingesting highly digestible organic material and larger, rich food items, blurring the macrophage-microphage dichotomy that applies well to larger worms. Copyright {\textcopyright} 2015 by Annual Reviews. All rights reserved.},
archivePrefix = {arXiv},
arxivId = {```````````````````````````````````````````````````````````````````````````````````````````````````````n},
author = {Jumars, Peter A. and Dorgan, Kelly M. and Lindsay, Sara M.},
doi = {10.1146/annurev-marine-010814-020007},
eprint = {```````````````````````````````````````````````````````````````````````````````````````````````````````n},
isbn = {1941-1405},
issn = {1941-1405},
journal = {Annual Review of Marine Science},
number = {1},
pages = {497--520},
pmid = {25251269},
title = {{Diet of Worms Emended: An Update of Polychaete Feeding Guilds}},
url = {http://www.annualreviews.org/doi/10.1146/annurev-marine-010814-020007},
volume = {7},
year = {2015}
}
@article{Poisot2018,
abstract = {Both species and their interactions are affected by changes that occur at evolutionary time-scales, and shape both ecological communities and their phylogenetic structure. But because extent ecological community structure is contingent upon random chance, environmental filters, and local effects, it is unclear how much ecological signal local communities should retain. Here we show that, in a host–parasite system where species interactions vary substantially over a continental gradient, the ecological significance of individual interactions is maintained across different scales. Notably, this occurs despite the fact that observed community variation at the local scale frequently tends to weaken or remove community-wide phylogenetic signal. When considered in terms of the interplay between community ecology and coevolutionary theory, our results demonstrate that individual interactions are capable and indeed likely to show a consistent signature of past evolutionary history even when woven into communities that do not. This article is protected by copyright. All rights reserved.},
author = {Poisot, Timoth{\'{e}}e and Stouffer, Daniel B.},
doi = {10.1111/oik.03788},
isbn = {0000000207355},
issn = {16000706},
journal = {Oikos},
keywords = {community phylogenetics,cophylogeny -,phylogenetic congruence - network,species interactions - host-parasites},
number = {2},
pages = {230--238},
title = {{Interactions retain the co-phylogenetic matching that communities lost}},
volume = {127},
year = {2018}
}
@article{Leicht2008,
abstract = {We consider the problem of finding communities or modules in directed networks. The most common approach to this problem in the previous literature has been simply to ignore edge direction and apply methods developed for community discovery in undirected networks, but this approach discards potentially useful information contained in the edge directions. Here we show how the widely used benefit function known as modularity can be generalized in a principled fashion to incorporate the information contained in edge directions. This in turn allows us to find communities by maximizing the modularity over possible divisions of a network, which we do using an algorithm based on the eigenvectors of the corresponding modularity matrix. This method is shown to give demonstrably better results than previous methods on a variety of test networks, both real and computer-generated.},
archivePrefix = {arXiv},
arxivId = {0709.4500},
author = {Leicht, E. A. and Newman, M. E.J.},
doi = {10.1103/PhysRevLett.100.118703},
eprint = {0709.4500},
isbn = {0031-9007},
issn = {00319007},
journal = {Physical Review Letters},
number = {11},
pages = {1--4},
pmid = {18517839},
title = {{Community structure in directed networks}},
volume = {100},
year = {2008}
}
@article{Simmons2018,
abstract = {Bipartite networks are widely-used to represent a diverse range of species interactions, such as pollination, herbivory, parasitism and seed dispersal. The structure of these networks is usually characterised by calculating one or more metrics that capture different aspects of network architecture. While these metrics capture useful properties of networks, they only consider structure at the scale of the whole network (the macro-scale) or individual species (the micro-scale). {\{}$\backslash$textquoteright{\}}Meso-scale{\{}$\backslash$textquoteright{\}} structure between these scales is usually ignored, despite representing ecologically-important interactions. Network motifs are a framework for capturing this meso-scale structure and are gaining in popularity. However, there is no software available in R, the most popular programming language among ecologists, for conducting motif analyses in bipartite networks. Similarly, no mathematical formalisation of bipartite motifs has been developed. Here we introduce bmotif: a package for counting motifs, and species positions within motifs, in bipartite networks. Our code is primarily an R package, but we also provide MATLAB and Python code of the core functionality. The software is based on a mathematical framework where, for the first time, we derive formal expressions for motif frequencies and the frequencies with which species occur in different positions within motifs. This framework means that analyses with bmotif are fast, making motif methods compatible with the permutational approaches often used in network studies, such as null model analyses. We describe the package and demonstrate how it can be used to conduct ecological analyses, using two examples of plant-pollinator networks. We first use motifs to examine the assembly and disassembly of an Arctic plant-pollinator community, and then use them to compare the roles of native and introduced plant species in an unrestored site in Mauritius. bmotif will enable motif analyses of a wide range of bipartite ecological networks, allowing future research to characterise these complex networks without discarding important meso-scale structural detail.},
author = {Simmons, Benno I and Sweering, Michelle J M and Dicks, Lynn V and Sutherland, William J and {Di Clemente}, Riccardo},
doi = {10.1101/302356},
journal = {bioRxiv},
pages = {doi:10.1101/302356},
title = {{Bmotif: a Package for Counting Motifs in Bipartite Networks}},
url = {https://www.biorxiv.org/content/early/2018/04/17/302356},
year = {2018}
}
@incollection{Barnes2008,
address = {Plymouth},
author = {Barnes, M. K. S.},
booktitle = {Marine Life Information Network: Biology and Sensitivity Key Information Reviews, [on-line]},
editor = {Tyler-Walters, H and Hiscock, K},
pages = {Available from http://marlin.ac.uk/species/detail/},
publisher = {Marine Biological Association of the United Kingdom},
title = {{$\backslash$emph{\{}Beroe cucumis{\}} A comb jelly}},
year = {2008}
}
@article{Reed2016,
author = {Reed, Daniel C and Harrison, John A},
doi = {10.1002/2015GB005303.Received},
file = {::},
issn = {08866236},
journal = {Global Biogeochem. Cycles},
keywords = {10.1002/2015GB005303 and bottom water oxygen,coastal ocean,environmental change,eutrophication,hypoxia,nutrient loading},
pages = {447--459},
title = {{Global Biogeochemical Cycles ocean : A new global scale model}},
year = {2016}
}
@article{Hong2017,
abstract = {Photovoltaic conversion efficiency of a crystalline silicon cell is investigated as a function of its temperature and taking into account complete thermal and irradiation operating conditions. The spectral radiative transfer problem is solved through a gray per band approach and a separated treatment of the collimated and diffuse components of radiation fluxes. The heat transfer modeling includes local heat sources due to radiation absorption and thermal emission, non-radiative recombinations and excess power release of photo-generated carriers. Continuity equations for minority carriers are solved to provide the current-voltage characteristic. A detailed analysis of the electrical and thermal behaviors demonstrates that proper adjustment and control of both thermal and surroundings radiative operating conditions are likely to provide guidelines for the improvement of photovoltaic cell performances. (c) 2006 Elsevier Ltd. All rights reserved.},
author = {Hong, Bongghi and Swaney, Dennis P. and McCrackin, Michelle and Svanb{\"{a}}ck, Annika and Humborg, Christoph and Gustafsson, Bo and Yershova, Alexandra and Pakhomau, Aliaksandr},
doi = {10.1007/s10533-017-0330-0},
isbn = {0017-9310},
issn = {1573515X},
journal = {Biogeochemistry},
keywords = {Anthropogenic nutrient inputs,Baltic watershed,NANI,NAPI,Total nitrogen,Total phosphorus},
number = {3},
pages = {245--261},
title = {{Advances in NANI and NAPI accounting for the Baltic drainage basin: spatial and temporal trends and relationships to watershed TN and TP fluxes}},
volume = {133},
year = {2017}
}
@article{Ojaveer2010,
abstract = {The brackish Baltic Sea hosts species of various origins and environmental tolerances. These immigrated to the sea 10,000 to 15,000 years ago or have been introduced to the area over the relatively recent history of the system. The Baltic Sea has only one known endemic species. While information on some abiotic parameters extends back as long as five centuries and first quantitative snapshot data on biota (on exploited fish populations) originate generally from the same time, international coordination of research began in the early twentieth century. Continuous, annual Baltic Sea-wide long-term datasets on several organism groups (plankton, benthos, fish) are generally available since the mid-1950s. Based on a variety of available data sources (published papers, reports, grey literature, unpublished data), the Baltic Sea, incl. Kattegat, hosts altogether at least 6,065 species, including at least 1,700 phytoplankton, 442 phytobenthos, at least 1,199 zooplankton, at least 569 meiozoobenthos, 1,476 macrozoobenthos, at least 380 vertebrate parasites, about 200 fish, 3 seal, and 83 bird species. In general, but not in all organism groups, high sub-regional total species richness is associated with elevated salinity. Although in comparison with fully marine areas the Baltic Sea supports fewer species, several facets of the system's diversity remain underexplored to this day, such as micro-organisms, foraminiferans, meiobenthos and parasites. In the future, climate change and its interactions with multiple anthropogenic forcings are likely to have major impacts on the Baltic biodiversity.},
author = {Ojaveer, Henn and Jaanus, Andres and MacKenzie, Brian R. and Martin, Georg and Olenin, Sergej and Radziejewska, Teresa and Telesh, Irena and Zettler, Michael L. and Zaiko, Anastasija},
doi = {10.1371/journal.pone.0012467},
file = {::},
isbn = {1932-6203},
issn = {19326203},
journal = {PLoS ONE},
number = {9},
pages = {e12467},
pmid = {20824189},
title = {{Status of Biodiversity in the Baltic Sea}},
volume = {5},
year = {2010}
}
@article{Meier2012a,
abstract = {Multi-model ensemble simulations for the marine biogeochemistry and food web of the Baltic Sea were performed for the period 1850–2098, and projected changes in the future climate were compared with the past climate environment. For the past period 1850–2006, atmospheric, hydrological and nutrient forcings were reconstructed, based on historical measurements. For the future period 1961–2098, scenario simulations were driven by regionalized global general circulation model (GCM) data and forced by various future greenhouse gas emission and air- and riverborne nutrient load scenarios (ranging from a pessimistic ‘business-as-usual' to the most optimistic case). To estimate uncertainties, different models for the various parts of the Earth system were applied. Assuming the IPCC greenhouse gas emission scenarios A1B or A2, we found that water temperatures at the end of this century may be higher and salinities and oxygen concentrations may be lower than ever measured since 1850. There is also a tendency of increased eutrophication in the future, depending on the nutrient load scenario. Although cod biomass is mainly controlled by fishing mortality, climate change together with eutrophication may result in a biomass decline during the latter part of this century, even when combined with lower fishing pressure. Despite considerable shortcomings of state-of-the-art models, this study suggests that the future Baltic Sea ecosystem may unprecedentedly change compared to the past 150 yr. As stakeholders today pay only little attention to adaptation and mitigation strategies, more information is needed to raise public awareness of the possible impacts of climate change on marine ecosystems.},
author = {Meier, H. E.Markus and Andersson, Hel{\'{e}}n C. and Arheimer, Berit and Blenckner, Thorsten and Chubarenko, Boris and Donnelly, Chantal and Eilola, Kari and Gustafsson, Bo G. and Hansson, Anders and Havenhand, Jonathan and H{\"{o}}glund, Anders and Kuznetsov, Ivan and MacKenzie, Brian R. and M{\"{u}}ller-Karulis, B{\"{a}}rbel and Neumann, Thomas and Niiranen, Susa and Piwowarczyk, Joanna and Raudsepp, Urmas and Reckermann, Marcus and Ruoho-Airola, Tuija and Savchuk, Oleg P. and Schenk, Frederik and Schimanke, Semjon and V{\"{a}}li, Germo and Weslawski, Jan Marcin and Zorita, Eduardo},
doi = {10.1088/1748-9326/7/3/034005},
file = {:Users/alyssacirtwill/Documents/Papers/Meier et al.{\_}2012{\_}Environmental Research Letters.pdf:pdf},
isbn = {1748-9326},
issn = {17489326},
journal = {Environmental Research Letters},
keywords = {Baltic Sea,Baltic Sea Action Plan,climate change,decision support system,ensemble modeling,eutrophication,marine biogeochemical cycles,marine food web,numerical modeling,scenarios},
number = {3},
title = {{Comparing reconstructed past variations and future projections of the Baltic Sea ecosystem - First results from multi-model ensemble simulations}},
volume = {7},
year = {2012}
}
@article{Conley2002,
abstract = {Deep-water oxygen concentrations in the Baltic Sea are influenced by eutrophication, but also by saltwater inflows from the North Sea. In the last two decades, only two major inflows have been recorded and the lack of major inflows is believed to have resulted in a long-term stagnation of the deepest bottom water. Analyzing data from 1970 to 2000 at the basin scale, we show that the estimated volume of water with oxygen, {\textless}2 mL L(-1), was actually at a minimum at the end of the longest so-called stagnation period on record. We also show that annual changes in dissolved inorganic phosphate water pools were positively correlated to the area of bottom covered by hypoxic water, but not to changes in total phosphorus load, thus addressing the legacy of eutrophication on a basinwide scale. The variations in phosphorus pools that have occurred during the past decades do not reflect any human action to reduce inputs. The long residence time and internally controlled variation of the large P pool in the Baltic Sea has important implications for management of both N and P inputs into this eutrophicated enclosed basin.},
author = {Conley, Daniel J. and Humborg, Christoph and Rahm, Lars and Savchuk, Oleg P. and Wulff, Fredrik},
doi = {10.1021/es025763w},
file = {::},
isbn = {0013-936X},
issn = {0013936X},
journal = {Environmental Science and Technology},
number = {24},
pages = {5315--5320},
pmid = {12521155},
title = {{Hypoxia in the baltic sea and basin-scale changes in phosphorus biogeochemistry}},
volume = {36},
year = {2002}
}
@article{Paavola2005,
abstract = {European brackish water seas (Baltic Sea, Black Sea and Sea of Azov, Caspian Sea) are subject to intense invasion of non-indigenous species (NIS). In these seas, salinity is the most important range limiting factor and native species seem to reach a minimum species richness at intermediate salinities. This trend, revealed by Remane in 1934 and later on confirmed by many other scientists, was compared to the salinity range of already established NIS in the European brackish water seas. It turned out that most NIS are well adapted to the salinities holding lowest native species richness, already in their native area, and that NIS richness maximum in brackish water seas occurs in the salinity intervals of native species richness minimum. A predictable pattern in the salinity range of NIS can be used as a tool in initial risk assessment of future invasions in brackish water seas, especially when mapping highly potential donor and recipient areas. A product of empty niches, suitable environmental conditions, and availability of proper vectors might be the most effective predictor for the invasibility of brackish water areas. {\textcopyright} 2005 Elsevier Ltd. All rights reserved.},
author = {Paavola, Marjo and Olenin, Sergej and Lepp{\"{a}}koski, Erkki},
doi = {10.1016/j.ecss.2005.03.021},
isbn = {0272-7714},
issn = {02727714},
journal = {Estuarine, Coastal and Shelf Science},
keywords = {Baltic Sea,Black Sea,Caspian Sea,Non-indigenous species (NIS),Risk assessment,Salinity,Venice System},
number = {4},
pages = {738--750},
title = {{Are invasive species most successful in habitats of low native species richness across European brackish water seas?}},
volume = {64},
year = {2005}
}
@article{Zettler2007,
abstract = {The need to assess the environmental status of marine and coastal waters according to the EU Water Framework Directive (WFD) encouraged the design of specific biotic indices to evaluate the response of benthic communities to human-induced changes in water quality. In the present study three of these indices, the traditional Shannon Wiener Index (H′) and the more recently published AMBI (AZTI′ Marine Biotic Index) and BQI (Benthic Quality Index), were tested along a salinity gradient in the southern Baltic Sea. The comparison of the three indices demonstrates that in the southern Baltic Sea the ecological quality (EcoQ) classification based on macrozoobenthic communities as indicator greatly depends on the biotic index chosen. We found a significant positive relation between species number, H′, BQI and salinity resulting in EcoQ status of "Bad", "Poor" or "Moderate" in areas with a salinity value below 10 psu. The AMBI was less dependent on salinity but appear to partly overestimate the EcoQ status. Presently none of these biotic indices appear to be adjusted for application in a gradient system as given in the southern Baltic Sea. A potential approach describing how to overcome this limitation is discussed. {\textcopyright} 2006 Elsevier Ltd. All rights reserved.},
author = {Zettler, Michael L. and Schiedek, Doris and Bobertz, Bernd},
doi = {10.1016/j.marpolbul.2006.08.024},
isbn = {0025-326X},
issn = {0025326X},
journal = {Marine Pollution Bulletin},
keywords = {Biotic indices,Ecological quality,Macrozoobenthos,Oxygen depletion,Salinity gradient,Southern Baltic Sea,Water Framework Directive},
number = {1-6},
pages = {258--270},
pmid = {17010998},
title = {{Benthic biodiversity indices versus salinity gradient in the southern Baltic Sea}},
volume = {55},
year = {2007}
}
@article{Long2011,
abstract = {P{\textgreater}1. Using a subtidal marine food web as a model system, we examined how food chain length (predators present or absent) and the prevalence of omnivory influenced temporal stability (and its components) of herbivores and plants. We held the density of top predators constant but manipulated their identity to generate a gradient in omnivory prevalence. 2. We measured temporal stability as the inverse of the coefficient of variation of abundance over time. Predators and omnivory could influence temporal stability through effects on abundance (the 'abundance' effect), summed variance across taxa (the 'portfolio effect') or summed covariances among taxa (the 'covariance effect'). 3. We found that increasing food chain length by predator addition destabilized aggregate herbivore abundance through their cascading effects on abundances. Thus, predators destabilized herbivores through the overyielding effect. We also found that the stability of herbivore abundance and microalgae declined with increasing prevalence of omnivory among top predators. Aggregate macroalgae was not affected, but the stability of one algal taxon increased with the prevalence of omnivory. 4. Our results suggest that herbivores are more sensitive than plants to changes in food web structure because of predator additions by invasion or deletions such as might occur via harvesting and habitat loss.},
author = {Long, Zachary T. and Bruno, John F. and Duffy, J. Emmett},
doi = {10.1111/j.1365-2656.2010.01800.x},
file = {::},
isbn = {0021-8790},
issn = {00218790},
journal = {Journal of Animal Ecology},
keywords = {Covariance effect,Food chain length,Food web,Lagodon rhombiode,Omnivory,Portfolio effect,Stability,Subtidal,Trophic cascade},
number = {3},
pages = {586--594},
pmid = {21250990},
title = {{Food chain length and omnivory determine the stability of a marine subtidal food web}},
volume = {80},
year = {2011}
}
@article{Webster2006,
abstract = {1. The influence of salinity, temperature and physiological development on habitat choice by juvenile salmon is poorly understood. We measured differences in the net energetic cost of habitats that differed in salinity or temperature using behavioural titration of juvenile salmon and correlated these costs with body size and osmoregulatory enzyme activity to quantify how costs change with physiological development. 2. Juvenile salmon showed a strong preference for saline water (27 or 15 vs 0) and for cold water (9 degrees C vs 14 degrees C). It was estimated to be 0.15 W and 0.11 W more costly for salmon to forage in fresh water than in 15 and 27 salt water, respectively, and 0.09 W more costly to forage in warm than in cold water. 3. We expected fish to prefer 15 salt water to fresh water regardless of enzyme activity because they are iso-osmotic with this salt concentration. In contrast, preference for higher salt concentrations should increase with enzyme activity. Consistent with our expectations, enzyme activity was not correlated with preference for 15 salt water, but was positively correlated with preference for 27 salt water. 4. The cost of changing salinity vs changing temperature were very similar, indicating that knowledge of both of these costs are necessary to understand juvenile salmon habitat choice.},
author = {Webster, S. J. and Dill, L. M.},
doi = {10.1111/j.1365-2435.2006.01128.x},
file = {::},
isbn = {1365-2435},
issn = {02698463},
journal = {Functional Ecology},
keywords = {Foraging,Habitat choice,Ideal free distribution,K +-ATPase activity,Na+},
number = {4},
pages = {621--629},
title = {{The energetic equivalence of changing salinity and temperature to juvenile salmon}},
volume = {20},
year = {2006}
}
@article{Herlemann2011,
abstract = {Salinity is a major factor controlling the distribution of biota in aquatic systems, and most aquatic multicellular organisms are either adapted to life in saltwater or freshwater conditions. Consequently, the saltwater-freshwater mixing zones in coastal or estuarine areas are characterized by limited faunal and floral diversity. Although changes in diversity and decline in species richness in brackish waters is well documented in aquatic ecology, it is unknown to what extent this applies to bacterial communities. Here, we report a first detailed bacterial inventory from vertical profiles of 60 sampling stations distributed along the salinity gradient of the Baltic Sea, one of world's largest brackish water environments, generated using 454 pyrosequencing of partial (400 bp) 16S rRNA genes. Within the salinity gradient, bacterial community composition altered at broad and finer-scale phylogenetic levels. Analogous to faunal communities within brackish conditions, we identified a bacterial brackish water community comprising a diverse combination of freshwater and marine groups, along with populations unique to this environment. As water residence times in the Baltic Sea exceed 3 years, the observed bacterial community cannot be the result of mixing of fresh water and saltwater, but our study represents the first detailed description of an autochthonous brackish microbiome. In contrast to the decline in the diversity of multicellular organisms, reduced bacterial diversity at brackish conditions could not be established. It is possible that the rapid adaptation rate of bacteria has enabled a variety of lineages to fill what for higher organisms remains a challenging and relatively unoccupied ecological niche.},
archivePrefix = {arXiv},
arxivId = {arXiv:gr-qc/9809069v1},
author = {Herlemann, Daniel P.R. and Labrenz, Matthias and J{\"{u}}rgens, Klaus and Bertilsson, Stefan and Waniek, Joanna J. and Andersson, Anders F.},
doi = {10.1038/ismej.2011.41},
eprint = {9809069v1},
isbn = {1751-7362},
issn = {17517362},
journal = {ISME Journal},
keywords = {454 pyrosequencing,SAR11,Verrucomicrobia,bacterial diversity,bacterioplankton,brackish water},
number = {10},
pages = {1571--1579},
pmid = {21472016},
primaryClass = {arXiv:gr-qc},
title = {{Transitions in bacterial communities along the 2000 km salinity gradient of the Baltic Sea}},
volume = {5},
year = {2011}
}
@article{Telesh2010,
abstract = {The salinity gradient is one of the main features characteristic of any estuarine ecosystem. Within this gradient in a critical salinity range of 5-8 PSU the major biotic and abiotic processes demonstrate non-linear dynamics of change in rates and directions. In estuaries, this salinity range acts as both external ecological factor and physiological characteristics of internal environment of aquatic organisms; it divides living conditions appropriate for freshwater and marine faunas, separates invertebrate communities with different osmotic regulation types, and defines the distribution range of high taxa. In this paper, the non-linearity of biotic processes within the estuarine salinity gradient is illustrated by the data on zooplankton from the Baltic estuaries. The non-tidal Baltic Sea provides a good demonstration of the above phenomena due to gradual changes of environmental factors and relatively stable isohalines. The non-linearity concept coupled with the ecosystem approach served the basis for a new definition of an estuary proposed by the authors. {\textcopyright} 2010 Elsevier Ltd.},
author = {Telesh, Irena V. and Khlebovich, Vladislav V.},
doi = {10.1016/j.marpolbul.2010.02.008},
file = {::},
isbn = {0025-326X},
issn = {0025326X},
journal = {Marine Pollution Bulletin},
keywords = {Baltic Sea,Critical salinity,Definition of an estuary,Neva Estuary,Salinity gradient,Zooplankton},
number = {4-6},
pages = {149--155},
pmid = {20304437},
publisher = {Elsevier Ltd},
title = {{Principal processes within the estuarine salinity gradient: A review}},
url = {http://dx.doi.org/10.1016/j.marpolbul.2010.02.008},
volume = {61},
year = {2010}
}
@article{Cognetti2000,
abstract = {A comparative analysis of estuaries, lagoons and coastal ponds focusing on population differentiation, and community structure is necessary to correctly address the issue of brackish water biology. Although the different biotopes all present similar features of environmental unpredictability and the common presence of the hypohalobic contingent (artenminimum), they each have their own characteristics, due to the evolution of peculiar balances in their relation to the sea on the one hand and inland waters on the other. In addition to euryhaline species, locally adapted populations of stenohaline species typical of marine habitats, as well as some recently introduced species, are also found. These species have given rise to euryhaline populations, reaching their maximum development in an optimal site. This situation occurs between basins with essentially similar ecological features and probably depends both on the different degree of adaptability of many species to a specific environmental parameter and the type of biocoenoses adjacent to the brackish basin. These populations possess genotypes allowing adaptation to brackish waters, which have resulted in the differentiation, through selection, of individuals capable of fine-grained perception of environmental unpredictability. Experimental works demonstrated the existence of genetically differentiated populations, or, ultimately, sibling species complexes, in several brackish species with broad geographical distribution and belonging to a wide range of taxonomic groups. The conceptions regarding the uniformity of brackish elements and the doubts concerning the existence of a specific brackish fauna come from the fact that attention generally focuses on species in the traditional meaning of the term, that is to say at the macrosystematic level. Comparative analyses of very fine morpho-physiological changes and genetic analyses result in a rather different picture, leading to the conclusion that in brackish waters a given species of marine origin often consists of many different forms at various levels of differentiation. Copyright (C) 2000 Elsevier Science Ltd.},
archivePrefix = {arXiv},
arxivId = {arXiv:1011.1669v3},
author = {Cognetti, Giuseppe and Maltagliati, Ferruccio},
doi = {10.1016/S0025-326X(99)00173-3},
eprint = {arXiv:1011.1669v3},
file = {::},
isbn = {0025-326X},
issn = {0025326X},
journal = {Marine Pollution Bulletin},
keywords = {Adaptive strategies,Conservation,Environmental unpredictability,Exotic species,Genetic variation},
number = {1},
pages = {7--14},
pmid = {270},
title = {{Biodiversity and adaptive mechanisms in brackish water fauna}},
volume = {40},
year = {2000}
}
@article{Society2010,
author = {Society, British Ecological},
file = {::},
number = {2},
pages = {267--290},
title = {{A Darwinian Approach to Plant Ecology Author ( s ): J . L . Harper Source : Journal of Applied Ecology , Vol . 4 , No . 2 ( Nov ., 1967 ), pp . 267-290 Published by : British Ecological Society Stable URL : http://www.jstor.org/stable/2401336 Accessed : 1}},
volume = {4},
year = {2016}
}
@article{Attrill2002,
author = {Society, British Ecological and Ecology, Animal},
journal = {Society},
number = {2},
pages = {262--269},
title = {{A testable in estuaries linear model for diversity trends}},
volume = {71},
year = {2010}
}
@article{Zanne2014,
abstract = {Early flowering plants are thought to have been woody species restricted to warm habitats. This lineage has since radiated into almost every climate, with manifold growth forms. As angiosperms spread and climate changed, they evolved mechanisms to cope with episodic freezing. To explore the evolution of traits underpinning the ability to persist in freezing conditions, we assembled a large species-level database of growth habit (woody or herbaceous; 49,064 species), as well as leaf phenology (evergreen or deciduous), diameter of hydraulic conduits (that is, xylem vessels and tracheids) and climate occupancies (exposure to freezing). To model the evolution of species' traits and climate occupancies, we combined these data with an unparalleled dated molecular phylogeny (32,223 species) for land plants. Here we show that woody clades successfully moved into freezing-prone environments by either possessing transport networks of small safe conduits and/or shutting down hydraulic function by dropping leaves during freezing. Herbaceous species largely avoided freezing periods by senescing cheaply constructed aboveground tissue. Growth habit has long been considered labile, but we find that growth habit was less labile than climate occupancy. Additionally, freezing environments were largely filled by lineages that had already become herbs or, when remaining woody, already had small conduits (that is, the trait evolved before the climate occupancy). By contrast, most deciduous woody lineages had an evolutionary shift to seasonally shedding their leaves only after exposure to freezing (that is, the climate occupancy evolved before the trait). For angiosperms to inhabit novel cold environments they had to gain new structural and functional trait solutions; our results suggest that many of these solutions were probably acquired before their foray into the cold.},
archivePrefix = {arXiv},
arxivId = {NIHMS150003},
author = {Zanne, Amy E. and Tank, David C. and Cornwell, William K. and Eastman, Jonathan M. and Smith, Stephen A. and FitzJohn, Richard G. and McGlinn, Daniel J. and O'Meara, Brian C. and Moles, Angela T. and Reich, Peter B. and Royer, Dana L. and Soltis, Douglas E. and Stevens, Peter F. and Westoby, Mark and Wright, Ian J. and Aarssen, Lonnie and Bertin, Robert I. and Calaminus, Andre and Govaerts, Rafa{\"{e}}l and Hemmings, Frank and Leishman, Michelle R. and Oleksyn, Jacek and Soltis, Pamela S. and Swenson, Nathan G. and Warman, Laura and Beaulieu, Jeremy M.},
doi = {10.1038/nature12872},
eprint = {NIHMS150003},
isbn = {1476-4687 (Electronic)$\backslash$r0028-0836 (Linking)},
issn = {00280836},
journal = {Nature},
number = {7486},
pages = {89--92},
pmid = {24362564},
title = {{Three keys to the radiation of angiosperms into freezing environments}},
volume = {506},
year = {2014}
}
@article{Sun2001,
abstract = {Zoonotic viruses, many originating in wild mammals, pose a serious threat to global public health. Peter Daszak and colleagues create a comprehensive database of mammalian hostvirus relationships, which they analyse to determine patterns of virus and zoonotic virus distribution in mammals. They identify various factors that influence the number and diversity of viruses that infect a given species as well as factors that predict the proportion of zoonotic viruses per species. In doing so, they identify mammalian species and geographic locations where novel zoonoses are likely to be found.},
archivePrefix = {arXiv},
arxivId = {NIHMS150003},
author = {Singh, J. B. and Behari, Pradeep and Yadava, R. B.},
doi = {10.1038/nature},
eprint = {NIHMS150003},
file = {::},
isbn = {0000017892},
issn = {00113891},
journal = {Current Science},
keywords = {Osteoporosis,Postmenopausal,Sclerostin},
number = {1},
pages = {17--19},
pmid = {1000084880},
title = {{Single cell Hi-C method}},
volume = {93},
year = {2007}
}
@article{Ysebaert2003,
abstract = {Few macrobenthic studies have dealt simultaneously with the two major gradients in estuarine benthic habitats: the salinity gradient along the estuary (longitudinal) and the gradients from high intertidal to deep subtidal sites (vertical gradient). In this broad-scale study, a large data set (3112 samples) of the Schelde estuary allowed a thorough analysis of these gradients, and to relate macrobenthic species distributions and community structure to salinity, depth, current velocities and sediment characteristics. Univariate analyses clearly revealed distinct gradients in diversity, abundance, and biomass along the vertical and longitudinal gradients. In general, highest diversity and biomass were observed in the intertidal, polyhaline zone and decreased with decreasing salinity. Abundance did not show clear trends and varied between spring and autumn. In all regions, very low values for all measures were observed in the subtidal depth strata. Abundance in all regions was dominated by both surface deposit feeders and sub-surface deposit feeders. In contrast, the biomass of the different feeding guilds showed clear gradients in the intertidal zone. Suspension feeders dominated in the polyhaline zone and showed a significant decrease with decreasing salinity. Surface deposit feeders and sub-surface deposit feeders showed significantly higher biomass values in the polyhaline zone as compared with the mesohaline zone. Omnivores showed an opposite trend. Multivariate analyses showed a strong relationship between the macrobenthic assemblages and the predominant environmental gradients in the Schelde estuary. The most important environmental factor was depth, which reflected also the hydrodynamic conditions (current velocities). A second gradient was related to salinity and confirms the observations from the univariate analyses. Additionally, sediment characteristics (mud content) explained a significant part of the macrobenthic community structure not yet explained by the two other main gradients. The different assemblages are further described in terms of indicator species and abiotic characteristics. The results showed that at a large, estuarine scale a considerable fraction of the variation in abundance and biomass of the benthic macrofauna correlated very well with environmental factors (depth, salinity, tidal current velocity, sediment composition). {\textcopyright} 2003 Elsevier Science B.V. All rights reserved.},
author = {Ysebaert, T. and Herman, P. M.J. and Meire, P. and Craeymeersch, J. and Verbeek, H. and Heip, C. H.R.},
doi = {10.1016/S0272-7714(02)00359-1},
file = {::},
isbn = {0272-7714},
issn = {02727714},
journal = {Estuarine, Coastal and Shelf Science},
keywords = {Benthic macrofauna,Canonical correspondence analysis,Depth,Environmental gradients,Estuarine habitats,Salinity,Schelde estuary,Suspension and deposit feeders,Variation partitioning},
number = {1-2},
pages = {335--355},
title = {{Large-scale spatial patterns in estuaries: Estuarine macrobenthic communities in the Schelde estuary, NW Europe}},
volume = {57},
year = {2003}
}
@misc{Alon2018,
author = {Biology, Molecular Cell},
booktitle = {Weizmann Institute of Science},
pages = {11--12},
title = {{Uri Alon Lab}},
url = {http://www.weizmann.ac.il/mcb/UriAlon/},
urldate = {2018-05-01},
year = {2010}
}
@article{Godoy2018,
abstract = {The quest for understanding how species interactions modulate diversity has progressed by theoretical and empirical advances following niche and network theories. Yet, niche studies have been limited to describe coexistence within tropic levels despite incorporating information about multi-trophic interactions. Network approaches could address this limitation, but they have ignored the structure of species interactions within trophic levels. Here we call for the integration of niche and network theories to reach new frontiers of knowledge exploring how interactions within and across trophic levels promote species coexistence. This integration is possible due to the strong parallelisms in the historical development, ecological concepts, and associated mathematical tools of both theories. We provide a guideline to integrate this framework with observational and experimental studies.},
author = {Godoy, Oscar and Bartomeus, Ignasi and Rohr, Rudolf P. and Saavedra, Serguei},
doi = {10.1016/j.tree.2018.01.007},
file = {::},
isbn = {0169-5347},
issn = {01695347},
journal = {Trends in Ecology and Evolution},
keywords = {Coexistence,feasibility,multi-trophic networks,species interactions,stability},
number = {4},
pages = {287--300},
pmid = {29471971},
publisher = {Elsevier Ltd},
title = {{Towards the Integration of Niche and Network Theories}},
url = {http://dx.doi.org/10.1016/j.tree.2018.01.007},
volume = {33},
year = {2018}
}
@article{Johnson2004,
abstract = {Recently, researchers in several areas of ecology and evolution have begun to change the way in which they analyze data and make biological inferences. Rather than the traditional null hypothesis testing approach, they have adopted an approach called model selection, in which several competing hypotheses are simultaneously confronted with data. Model selection can be used to identify a single best model, thus lending support to one particular hypothesis, or it can be used to make inferences based on weighted support from a complete set of competing models. Model selection is widely accepted and well developed in certain fields, most notably in molecular systematics and mark-recapture analysis. However, it is now gaining support in several other areas, from molecular evolution to landscape ecology. Here, we outline the steps of model selection and highlight several ways that it is now being implemented. By adopting this approach, researchers in ecology and evolution will find a valuable alternative to traditional null hypothesis testing, especially when more than one hypothesis is plausible.},
archivePrefix = {arXiv},
arxivId = {10.1016/j.tree.2003.10.013},
author = {Johnson, Jerald B. and Omland, Kristian S.},
doi = {10.1016/j.tree.2003.10.013},
eprint = {j.tree.2003.10.013},
file = {::},
isbn = {0169-5347},
issn = {01695347},
journal = {Trends in Ecology and Evolution},
number = {2},
pages = {101--108},
pmid = {16701236},
primaryClass = {10.1016},
title = {{Model selection in ecology and evolution}},
volume = {19},
year = {2004}
}
@article{Donatti2011,
abstract = {Mutualistic interactions involving pollination and ant-plant mutualistic networks typically feature tightly linked species grouped in modules. However, such modularity is infrequent in seed dispersal networks, presumably because research on those networks predominantly includes a single taxonomic animal group (e.g. birds). Herein, for the first time, we examine the pattern of interaction in a network that includes multiple taxonomic groups of seed dispersers, and the mechanisms underlying modularity. We found that the network was nested and modular, with five distinguishable modules. Our examination of the mechanisms underlying such modularity showed that plant and animal trait values were associated with specific modules but phylogenetic effect was limited. Thus, the pattern of interaction in this network is only partially explained by shared evolutionary history. We conclude that the observed modularity emerged by a combination of phylogenetic history and trait convergence of phylogenetically unrelated species, shaped by interactions with particular types of dispersal agents.},
author = {Donatti, Camila I. and Guimar{\~{a}}es, Paulo R. and Galetti, Mauro and Pizo, Marco Aur{\'{e}}lio and Marquitti, Fl{\'{a}}via M.D. and Dirzo, Rodolfo},
doi = {10.1111/j.1461-0248.2011.01639.x},
isbn = {1461-023X},
issn = {1461023X},
journal = {Ecology Letters},
keywords = {Birds,Body mass,Complex networks,Fish,Fruit diameter,Mammals,Nestedness,Phylogenetic analyses,Reptiles},
month = {aug},
number = {8},
pages = {773--781},
pmid = {21699640},
title = {{Analysis of a hyper-diverse seed dispersal network: Modularity and underlying mechanisms}},
url = {http://www.ncbi.nlm.nih.gov/pubmed/21699640},
volume = {14},
year = {2011}
}
@article{Emmerson2004,
abstract = {Empirical studies have shown that, in real ecosystems, species-interaction strengths are generally skewed in their distribution towards weak interactions. Some theoretical work also suggests that weak interactions, especially in omnivorous links, are important for the local stability of a community at equilibrium. However, the majority of theoretical studies use uniform distributions of interaction strengths to generate artificial communities for study. We investigate the effects of the underlying interaction-strength distribution upon the return time, permanence and feasibility of simple Lotka-Volterra equilibrium communities. We show that a skew towards weak interactions promotes local and global stability only when omnivory is present. It is found that skewed interaction strengths are an emergent property of stable omnivorous communities, and that this skew towards weak interactions creates a dynamic constraint maintaining omnivory. Omnivory is more likely to occur when omnivorous interactions are skewed towards weak interactions. However, a skew towards weak interactions increases the return time to equilibrium, delays the recovery of ecosystems and hence decreases the stability of a community. When no skew is imposed, the set of stable omnivorous communities shows an emergent distribution of skewed interaction strengths. Our results apply to both local and global concepts of stability and are robust to the definition of a feasible community. These results are discussed in the light of empirical data and other theoretical studies, in conjunction with their broader implications for community assembly.},
author = {Emmerson, Mark and Yearsley, Jon M.},
doi = {10.1098/rspb.2003.2592},
isbn = {0962-8452},
issn = {14712970},
journal = {Proceedings of the Royal Society B: Biological Sciences},
keywords = {Extinction,Food webs,Interaction strength,Local stability, permanence,Omnivory},
month = {feb},
number = {1537},
pages = {397--405},
pmid = {15101699},
title = {{Weak interactions, omnivory and emergent food-web properties}},
volume = {271},
year = {2004}
}
@article{Stouffer2014,
abstract = {There is increasing world-wide concern about the impact of the introduction of exotic species on ecological communities. Since many exotic plants depend on native pollinators to successfully establish, it is of paramount importance that we understand precisely how exotic species integrate into existing plant-pollinator communities. In this manuscript, we have studied a global data base of empirical pollination networks to determine whether community, network, species or interaction characteristics can help identify invaded communities. We found that a limited number of community and network properties showed significant differences among the empirical data sets - namely networks with exotic plants present are characterized by greater total, plant and pollinator richness, as well as higher values of relative nestedness. We also observed significant differences in terms of the pollinators that interact with the exotic plants. In particular, we found that specialist pollinators that are also weak contributors to community nestedness are far more likely to interact with exotic plants than would be expected by chance alone. By virtue of their interactions, it appears that exotic plants may provide a key service to a community's specialist pollinators as well as fill otherwise vacant 'coevolutionary niches'.},
author = {Stouffer, Daniel B. and Cirtwill, Alyssa R. and Bascompte, Jordi},
doi = {10.1111/1365-2745.12310},
editor = {Bartomeus, Ignasi},
isbn = {1365-2745},
issn = {13652745},
journal = {Journal of Ecology},
keywords = {Coevolution,Competition,Extinction,Generalists,Indirect facilitation,Invasion ecology,Mutualistic networks,Nestedness,Plant-animal interactions,Specialization},
month = {sep},
number = {6},
pages = {1442--1450},
pmid = {25558089},
title = {{How exotic plants integrate into pollination networks}},
url = {http://doi.wiley.com/10.1111/1365-2745.12310},
volume = {102},
year = {2014}
}
@article{Stouffer2011,
abstract = {1. The idea that species occupy distinct niches is a fundamental concept in ecology. Classically, the niche was described as an n-dimensional hypervolume where each dimension represents a biotic or abiotic characteristic. More recently, it has been hypothesised that a single dimension may be sufficient to explain the system-level organization of trophic interactions observed between species in a community. 2. Here, we test the hypothesis that species body mass is that single dimension. Specifically, we determine how the intervality of food webs ordered by body size compares to that of randomly ordered food webs. We also extend this analysis beyond the community level to the effect of body mass in explaining the diets of individual species. 3. We conclude that body mass significantly explains the ordering of species and the contiguity of diets in empirical communities. 4. At the species-specific level, we find that the degree to which body mass is a significant explanatory variable depends strongly on the phylogenetic history, suggesting that other evolutionarily conserved traits partly account for species' roles in the food web. 5. Our investigation of the role of body mass in food webs thus helps us to better understand the important features of community food-web structure and the evolutionary forces that have led us to the communities we observe.},
author = {Stouffer, Daniel B. and Rezende, Enrico L. and Amaral, Lu{\'{i}}s A.Nunes},
doi = {10.1111/j.1365-2656.2011.01812.x},
isbn = {0021-8790},
issn = {00218790},
journal = {Journal of Animal Ecology},
keywords = {Complex networks,Food webs,Intervality,Niche dimension,Species phylogenetics},
month = {may},
number = {3},
pages = {632--639},
pmid = {21401590},
title = {{The role of body mass in diet contiguity and food-web structure}},
url = {http://www.ncbi.nlm.nih.gov/pubmed/21401590 http://doi.wiley.com/10.1111/j.1365-2656.2011.01812.x},
volume = {80},
year = {2011}
}
@article{Cirtwill2015,
abstract = {Previous analyses of empirical food webs (the networks of who eats whom in a community) have revealed that parasites exert a strong influence over observed food web structure and alter many network properties such as connectance and degree distributions. It remains unclear, however, whether these community-level effects are fully explained by differences in the ways that parasites and free-living species interact within a food web. To rigorously quantify the interrelationship between food web structure, the types of species in a web and the distinct types of feeding links between them, we introduce a shared methodology to quantify the structural roles of both species and feeding links. Roles are quantified based on the frequencies with which a species (or link) appears in different food web motifs - the building blocks of networks. We hypothesized that different types of species (e.g. top predators, basal resources, parasites) and different types of links between species (e.g. classic predation, parasitism, concomitant predation on parasites along with their hosts) will show characteristic differences in their food web roles. We found that parasites do indeed have unique structural roles in food webs. Moreover, we demonstrate that different types of feeding links (e.g. parasitism, predation or concomitant predation) are distributed differently in a food web context. More than any other interaction type, concomitant predation appears to constrain the roles of parasites. In contrast, concomitant predation links themselves have more variable roles than any other type of interaction. Together, our results provide a novel perspective on how both species and feeding link composition shape the structure of an ecological community and vice versa.},
author = {Cirtwill, Alyssa R. and Stouffer, Daniel B.},
doi = {10.1111/1365-2656.12323},
isbn = {1365-2656},
issn = {13652656},
journal = {Journal of Animal Ecology},
keywords = {Interaction roles,Network motifs,Role dispersion,Role diversity,Species roles},
number = {3},
pages = {734--744},
pmid = {25418425},
title = {{Concomitant predation on parasites is highly variable but constrains the ways in which parasites contribute to food web structure}},
url = {http://doi.wiley.com/10.1111/1365-2656.12323},
volume = {84},
year = {2015}
}
@article{Power1992,
abstract = {Predation by fish (roach, Hesperoleucas symmetricus, and steelhead Oncorhynchus mykiss) produced strong cascading effects on biota associated with boulder-bedrock substrates in pools of a northern California [USA] river, but not on gravel-dwelling biota. Enclosure-exclosure experiments in the South Fork Eel River of northern California (39.degree.44'N, 123.degree.39'W) showed that fish, by suppressing densities of damselfly nymphs and other small predators, released algivorous chironomids (Pseudochironomus richardsoni) from predation. Chironomids in turn dramatically reduced algal standing crops. In contrast, fish had little effect on algae or invertebrates associated with gravel. Gravel-dwelling heptageniid mayflies were behaviorally inhibited from using tops of stones in fish enclosures, and stone surfaces had more chironomid tubes in fish enclosures than in fish exclosures. However, no effects on epilithic algae or densities of invertebrates comparable to those of biota on boulder-bedrock substrates were detected. These spatially varying predator effects in a river parallel results from marine benthic systems, where strong effects of large predators documented for rocky intertidal habitats and unvegetated soft bottoms are not conspicuous in seagrass beds.},
author = {Power, Mary E.},
journal = {Ecology},
number = {3},
pages = {733--746},
title = {{Power 1992 top down bottom up}},
volume = {73},
year = {1992}
}
@article{Cirtwill2015PRSB,
abstract = {Several properties of food webs-the networks of feeding links between species-are known to vary systematically with the species richness of the underlying community. Under the 'latitude-niche breadth hypothesis', which predicts that species in the tropics will tend to evolve narrower niches, one might expect that these scaling relationships could also be affected by latitude. To test this hypothesis, we analysed the scaling relationships between species richness and average generality, vulnerability and links per species across a set of 196 empirical food webs. In estuarine, marine and terrestrial food webs there was no effect of latitude on any scaling relationship, suggesting constant niche breadth in these habitats. In freshwater communities, on the other hand, there were strong effects of latitude on scaling relationships, supporting the latitude-niche breadth hypothesis. These contrasting findings indicate that it may be more important to account for habitat than latitude when exploring gradients in food-web structure.},
author = {Cirtwill, Alyssa R. and Stouffer, Daniel B. and Romanuk, Tamara N.},
doi = {10.1098/rspb.2015.1589},
isbn = {0962-8452},
issn = {14712954},
journal = {Proceedings of the Royal Society B: Biological Sciences},
keywords = {Food webs,Generality,Link density,Scaling,Trophic level,Vulnerability},
number = {1819},
pages = {20151589},
pmid = {26559955},
title = {{Latitudinal gradients in biotic niche breadth vary across ecosystem types}},
url = {http://rspb.royalsocietypublishing.org/content/282/1819/20151589},
volume = {282},
year = {2015}
}
@article{Williams2000,
abstract = {Several of the most ambitious theories in ecology describe food webs that document the structure of strong and weak trophic links that is responsible for ecological dynamics among diverse assemblages of species. Early mechanism-based theory asserted that food webs have little omnivory and several properties that are independent of species richness. This theory was overturned by empirical studies that found food webs to be much more complex, but these studies did not provide mechanistic explanations for the complexity. Here we show that a remarkably simple model fills this scientific void by successfully predicting key structural properties of the most complex and comprehensive food webs in the primary literature. These properties include the fractions of species at top, intermediate and basal trophic levels, the means and variabilities of generality, vulnerability and food-chain length, and the degrees of cannibalism, omnivory, looping and trophic similarity. Using only two empirical parameters, species number and connectance, our 'niche model' extends the existing 'cascade model and improves its fit ten-fold by constraining species to consume a contiguous sequence of prey in a one-dimensional trophic niche.},
author = {Williams, Richard J. and Martinez, Neo D.},
doi = {10.1038/35004572},
isbn = {0028-0836},
issn = {00280836},
journal = {Nature},
keywords = {Biological,Ecology,Food Chain,Models,Monte Carlo Method},
month = {mar},
number = {6774},
pages = {180--183},
pmid = {10724169},
title = {{Simple rules yield complex food webs}},
url = {http://www.ncbi.nlm.nih.gov/pubmed/10724169},
volume = {404},
year = {2000}
}
@article{Cirtwill2018c,
abstract = {Food webs and meso-scale motifs allow us to understand the structure of ecological communities and define species' roles within them. This species-level perspective on networks permits tests for relationships between species' traits and their patterns of direct and indirect interactions. Such relationships could allow us to predict food-web structure based on more easily obtained trait information. Here, we calculated the roles of species (as vectors of motif position frequencies) in six well-resolved marine food webs and identified the motif positions associated with the greatest variation in species' roles. We then tested whether the frequencies of these positions varied with species' traits. Despite the coarse-grained traits we used, our approach identified several strong associations between traits and motifs. Feeding environment was a key trait in our models and may shape species' roles by affecting encounter probabilities. Incorporating environment into future food-web models may improve predictions of an unknown network structure.},
author = {Cirtwill, Alyssa R. and Ekl{\"{o}}f, Anna},
doi = {10.1111/ele.12955},
file = {:Users/alyssacirtwill/Documents/Papers/Cirtwill, Ekl{\"{o}}f{\_}2018{\_}Ecology Letters.pdf:pdf},
issn = {14610248},
journal = {Ecology Letters},
keywords = {Apparent competition,body mass,direct competition,feeding environment,food chain,indirect interactions,trophic level},
number = {6},
pages = {875--884},
pmid = {29611282},
title = {{Feeding environment and other traits shape species' roles in marine food webs}},
volume = {21},
year = {2018}
}
@article{Allesina2008b,
abstract = {A central problem in ecology is determining the processes that shape the complex networks known as food webs formed by species and their feeding relationships. The topology of these networks is a major determinant of ecosystems' dynamics and is ultimately responsible for their responses to human impacts. Several simple models have been proposed for the intricate food webs observed in nature. We show that the three main models proposed so far fail to fully replicate the empirical data, and we develop a likelihood-based approach for the direct comparison of alternative models based on the full structure of the network. Results drive a new model that is able to generate all the empirical data sets and to do so with the highest likelihood.},
author = {Allesina, Stefano and Alonso, David and Pascual, Mercedes},
doi = {10.1126/science.1156269},
isbn = {0036-8075},
issn = {00368075},
journal = {Science},
number = {5876},
pages = {658--661},
pmid = {18451301},
title = {{A general model for food web structure}},
url = {www.scopus.com},
volume = {320},
year = {2008}
}
@article{Bartomeus2016a,
abstract = {Species interactions, ranging from antagonisms to mutualisms, form the architecture of biodiversity and determine ecosystem functioning. Understanding the rules responsible for who interacts with whom, as well as the functional consequences of these interspecific interactions, is central to predict community dynamics and stability. Species traits sensu lato may affect different ecological processes by determining species interactions through a two-step process. First, ecological and life-history traits govern species distributions and abundance, and hence determine species co-occurrence and the potential for species to interact. Secondly, morphological or physiological traits between co-occurring potential interaction partners should match for the realization of an interaction. Here, we review recent advances on predicting interactions from species co-occurrence and develop a probabilistic model for inferring trait matching. The models proposed here integrate both neutral and trait-matching constraints, while using only information about known interactions, thereby overcoming problems originating from undersampling of rare interactions (i.e. missing links). They can easily accommodate qualitative or quantitative data and can incorporate trait variation within species, such as values that vary along developmental stages or environmental gradients. We use three case studies to show that the proposed models can detect strong trait matching (e.g. predator–prey system), relaxed trait matching (e.g. herbivore–plant system) and barrier trait matching (e.g. plant–pollinator systems). Only by elucidating which species traits are important in each process (i.e. in determining interaction establishment and frequency), we can advance in explaining how species interact and the consequences of these interactions for ecosystem functioning.},
archivePrefix = {arXiv},
arxivId = {1011.1669},
author = {Bartomeus, Ignasi and Gravel, Dominique and Tylianakis, Jason M. and Aizen, Marcelo A. and Dickie, Ian A. and Bernard-Verdier, Maud},
doi = {10.1111/1365-2435.12666},
eprint = {1011.1669},
file = {::},
isbn = {9788578110796},
issn = {13652435},
journal = {Functional Ecology},
keywords = {functional traits,herbivory,interaction networks,mutualisms,parasitism,pollination,predation,trait matching,trophic interactions},
number = {12},
pages = {1894--1903},
pmid = {25246403},
title = {{A common framework for identifying linkage rules across different types of interactions}},
volume = {30},
year = {2016}
}
@article{Cirtwill2018Oikos,
abstract = {BACKGROUND: Attention deficit hyperactivity disorder (ADHD) symptoms and autistic traits often occur together. The pattern and etiology of co-occurrence are largely unknown, particularly in adults. This study investigated the co-occurrence between both traits in detail, and subsequently examined the etiology of the co-occurrence, using two independent adult population samples. Method Data on ADHD traits (Inattention and Hyperactivity/Impulsivity) were collected in a population sample (S1, n = 559) of unrelated individuals. Data on Attention Problems (AP) were collected in a population-based family sample of twins and siblings (S2, n = 560). In both samples five dimensions of autistic traits were assessed (social skills, routine, attentional switching, imagination, patterns).$\backslash$n$\backslash$nRESULTS: Hyperactive traits (S1) did not correlate substantially with the autistic trait dimensions. For Inattention (S1) and AP (S2), the correlations with the autistic trait dimensions were low, apart from a prominent correlation with the attentional switching scale (0.47 and 0.32 respectively). Analyses in the genetically informative S2 revealed that this association could be explained by a shared genetic factor.$\backslash$n$\backslash$nCONCLUSIONS: Our findings suggest that the co-occurrence of ADHD traits and autistic traits in adults is not determined by problems with hyperactivity, social skills, imagination or routine preferences. Instead, the association between those traits is due primarily to shared attention-related problems (inattention and attentional switching capacity). As the etiology of this association is purely genetic, biological pathways involving attentional control could be a promising focus of future studies aimed at unraveling the genetic causes of these disorders.},
author = {Cirtwill, Alyssa R. and Roslin, Tomas and Rasmussen, Claus and Olesen, Jens M. and Stouffer, Daniel B.},
doi = {doi:10.1111/oik.05074},
journal = {Oikos},
number = {8},
pages = {1163----1176},
title = {{Between-year changes in community composition shape species' roles in an Arctic plant-polliator network}},
url = {http://doi.wiley.com/10.1111/oik.05074},
volume = {127},
year = {2018}
}
@article{Knop2017,
abstract = {Pollinator numbers are declining worldwide. Alongside factors such as land use change and agricultural intensification, artificial light at night has been proposed to contribute to this loss. Eva Knop and colleagues put this theory to the test in a field experiment in Switzerland. They exposed ruderal meadows to artificial light at night and monitored nocturnal pollinator behaviour. Pollinator visits to plants fell by 62{\%} in the illuminated plots, and fruit production by a focal plant fell by 13{\%}. The findings suggest that artificial light at night, which is spreading at an estimated rate of 6{\%} per year, poses yet another threat to global pollinator health.},
author = {Knop, Eva and Zoller, Leana and Ryser, Remo and Gerpe, Christopher and H{\"{o}}rler, Maurin and Fontaine, Colin},
doi = {10.1038/nature23288},
file = {:Users/alyssacirtwill/Documents/Papers/Knop et al.{\_}2017{\_}Nature.pdf:pdf},
isbn = {1476-46871476-4687},
issn = {14764687},
journal = {Nature},
number = {7666},
pages = {206--209},
publisher = {Nature Publishing Group},
title = {{Artificial light at night as a new threat to pollination}},
url = {http://doi.wiley.com/10.1111/ele.12955},
volume = {548},
year = {2017}
}
@article{Hutchinson2018,
abstract = {Despite the fact that natural selection underlies both traits and interactions, evolutionary models often neglect that ecological interactions may, and in many cases do, influence the evolution of traits. Here, we explore the interdependence of ecological interactions and functional traits in the pollination associations of hawkmoths and flowering plants. Specifically, we develop an adaptation of the Ornstein-Uhlenbeck model of trait evolution that allows us to study the influence of plant corolla depth and observed hawkmoth-plant interactions on the evolution of hawkmoth proboscis length. Across diverse modelling scenarios, we find that the inclusion of contemporary interactions can provide a better description of trait evolution than the null expectation. Moreover, we show that the pollination interactions provide more-likely models of hawkmoth trait evolution when interactions are considered at increasingly finescale groups of hawkmoths. Finally, we demonstrate how the results of best-fit modelling approaches can implicitly support the association between interactions and trait evolution that our method explicitly examines. In showing that contemporary interactions can provide insight into the historical evolution of hawkmoth proboscis length, we demonstrate the clear utility of incorporating additional ecological information to models designed to study past trait evolution.},
author = {Hutchinson, Matthew C. and Gaiarsa, Mar{\'{i}}lia P. and Stouffer, Daniel B.},
doi = {10.1093/sysbio/syy012},
file = {:Users/alyssacirtwill/Documents/Papers/Hutchinson, Gaiarsa, Stouffer{\_}2018{\_}Systematic Biology.pdf:pdf},
isbn = {0816646627},
issn = {1076836X},
journal = {Systematic Biology},
keywords = {Macroevolution,Mutualism,Ornstein-Uhlenbeck,Pollination,Sphingidae},
number = {5},
pages = {861--872},
title = {{Contemporary ecological interactions improve models of past trait evolution}},
url = {https://academic.oup.com/sysbio/advance-article/doi/10.1093/sysbio/syy012/4877123},
volume = {67},
year = {2018}
}
@article{Hutchinson2017,
abstract = {Introduction: D-dimer assay, generally evaluated according to cutoff points calibrated for VTE exclusion, is used to estimate the individual risk of recurrence after a first idiopathic event of venous thromboembolism (VTE). Methods: Commercial D-dimer assays, evaluated according to predetermined cutoff levels for each assay, specific for age (lower in subjects {\textless}70 years) and gender (lower in males), were used in the recent DULCIS study. The present analysis compared the results obtained in the DULCIS with those that might have been had using the following different cutoff criteria: traditional cutoff for VTE exclusion, higher levels in subjects aged ≥60 years, or age multiplied by 10. Results: In young subjects, the DULCIS low cutoff levels resulted in half the recurrent events that would have occurred using the other criteria. In elderly patients, the DULCIS results were similar to those calculated for the two age-adjusted criteria. The adoption of traditional VTE exclusion criteria would have led to positive results in the large majority of elderly subjects, without a significant reduction in the rate of recurrent event. Conclusion: The results confirm the usefulness of the cutoff levels used in DULCIS.},
archivePrefix = {arXiv},
arxivId = {arXiv:physics/0608246v3},
author = {Hutchinson, Matthew C. and Cagua, Edgar Fernando and Stouffer, Daniel B.},
doi = {10.1002/ecy.1955},
eprint = {0608246v3},
file = {:Users/alyssacirtwill/Documents/Papers/Hutchinson, Cagua, Stouffer{\_}2017{\_}Ecology.pdf:pdf},
isbn = {4955139574},
issn = {00129658},
journal = {Ecology},
keywords = {co-speciation,compartmentalization,modularity,mutualism,mutualistic networks,phylogenetic structure,pollination syndromes},
number = {10},
pages = {2640--2652},
pmid = {28199780},
primaryClass = {arXiv:physics},
title = {{Cophylogenetic signal is detectable in pollination interactions across ecological scales}},
url = {http://doi.wiley.com/10.1002/ecy.1955},
volume = {98},
year = {2017}
}
@article{Hicks2016,
abstract = {Planted meadows are increasingly used to improve the biodiversity and aesthetic amenity value of urban areas. Although many 'pollinator-friendly' seed mixes are available, the floral resources these provide to flower-visiting insects, and how these change through time, are largely unknown. Such data are necessary to compare the resources provided by alternative meadow seed mixes to each other and to other flowering habitats. We used quantitative surveys of over 2 million flowers to estimate the nectar and pollen resources offered by two exemplar commercial seed mixes (one annual, one perennial) and associated weeds grown as 300m2 meadows across four UK cities, sampled at six time points between May and September 2013. Nectar sugar and pollen rewards per flower varied widely across 65 species surveyed, with native British weed species (including dandelion, Taraxacum agg.) contributing the top five nectar producers and two of the top ten pollen producers. Seed mix species yielding the highest rewards per flower included Leontodon hispidus, Centaurea cyanus and C. nigra for nectar, and Papaver rhoeas, Eschscholzia californica and Malva moschata for pollen. Perennial meadows produced up to 20x more nectar and up to 6x more pollen than annual meadows, which in turn produced far more than amenity grassland controls. Perennial meadows produced resources earlier in the year than annual meadows, but both seed mixes delivered very low resource levels early in the year and these were provided almost entirely by native weeds. Pollen volume per flower is well predicted statistically by floral morphology, and nectar sugar mass and pollen volume per unit area are correlated with flower counts, raising the possibility that resource levels can be estimated for species or habitats where they cannot be measured directly. Our approach does not incorporate resource quality information (for example, pollen protein or essential amino acid content), but can easily do so when suitable data exist. Our approach should inform the design of new seed mixes to ensure continuity in floral resource availability throughout the year, and to identify suitable species to fill resource gaps in established mixes.},
archivePrefix = {arXiv},
arxivId = {NIHMS150003},
author = {Hicks, Damien M. and Ouvrard, Pierre and Baldock, Katherine C.R. and Baude, Mathilde and Goddard, Mark A. and Kunin, William E. and Mitschunas, Nadine and Memmott, Jane and Morse, Helen and Nikolitsi, Maria and Osgathorpe, Lynne M. and Potts, Simon G. and Robertson, Kirsty M. and Scott, Anna V. and Sinclair, Frazer and Westbury, Duncan B. and Stone, Graham N.},
doi = {10.1371/journal.pone.0158117},
eprint = {NIHMS150003},
file = {::},
isbn = {1932-6203 (Electronic)$\backslash$r1932-6203 (Linking)},
issn = {19326203},
journal = {PLoS ONE},
number = {6},
pages = {e0158117},
pmid = {27341588},
title = {{Food for pollinators: Quantifying the nectar and pollen resources of urban flower meadows}},
volume = {11},
year = {2016}
}
@article{Brakker2014,
abstract = {Some vertebrate species have evolved means of extending their visual sensitivity beyond the range of human vision. One mechanism of enhancing sensitivity to long-wavelength light is to replace the 11-cis retinal chromophore in photopigments with 11-cis 3,4-didehydroretinal. Despite over a century of research on this topic, the enzymatic basis of this perceptual switch remains unknown. Here, we show that a cytochrome P450 family member, Cyp27c1, mediates this switch by converting vitamin A1 (the precursor of 11-cis retinal) into vitamin A2 (the precursor of 11-cis 3,4-didehydroretinal). Knockout of cyp27c1 in zebrafish abrogates production of vitamin A2, eliminating the animal's ability to red-shift its photoreceptor spectral sensitivity and reducing its ability to see and respond to near-infrared light. Thus, the expression of a single enzyme mediates dynamic spectral tuning of the entire visual system by controlling the balance of vitamin A1 and A2 in the eye.},
author = {Enright, Jennifer M. and Toomey, Matthew B. and Sato, Shin Ya and Temple, Shelby E. and Allen, James R. and Fujiwara, Rina and Kramlinger, Valerie M. and Nagy, Leslie D. and Johnson, Kevin M. and Xiao, Yi and How, Martin J. and Johnson, Stephen L. and Roberts, Nicholas W. and Kefalov, Vladimir J. and {Peter Guengerich}, F. and Corbo, Joseph C.},
doi = {10.1016/j.cub.2015.10.018},
isbn = {0960-9822},
issn = {09609822},
journal = {Current Biology},
number = {23},
pages = {3048--3057},
pmid = {26549260},
title = {{Cyp27c1 red-shifts the spectral sensitivity of photoreceptors by converting Vitamin A1 into A2}},
volume = {25},
year = {2015}
}
@article{Dee2018,
abstract = {CONTEXT: Dyspnea is one of the most distressing symptoms in patients with cancer, and often worsens with breakthrough episodes on exertion. We hypothesized that fentanyl given prophylactically may alleviate breakthrough dyspnea. OBJECTIVES: To determine the feasibility of conducting a randomized trial of subcutaneous fentanyl in patients with cancer, and examine the effects of fentanyl on dyspnea, walk distance, vital signs, and adverse events. METHODS: In this double-blind, randomized, controlled trial, we asked ambulatory patients with breakthrough dyspnea to perform a baseline six minute walk test (6MWT), and then assigned them to either subcutaneous fentanyl or placebo 15 minutes before a second 6MWT. We documented the change in dyspnea Numeric Rating Scale (NRS) score, walk distance, vital signs, and adverse events between the first and second 6MWT. RESULTS: A total of 20 patients were enrolled (1:1 ratio) without attrition. Comparison between baseline and second walk showed that fentanyl was associated with significant improvements in dyspnea NRS score at the end of the 6MWT (mean [95{\%} CI] -1.8 [-3.2, -0.4]), dyspnea NRS score at rest of 15 minutes after drug administration (-0.9 [-1.8, -0.04]), Borg Scale fatigue score at the end of the 6MWT (-1.3 [-2.4, -0.2]), 6MWT distance (+37.2m [5.8, 68.6]), and respiratory rate (-2.4 [-4.5, -0.3]). Nonstatistically significant improvements also were observed in the placebo arm, with no difference between the two study arms. No significant adverse effects were observed. CONCLUSION: Prophylactic fentanyl was safe and improved dyspnea, fatigue, walk distance, and respiratory rate. We also observed a large placebo effect. Our results justify larger randomized controlled trials with higher fentanyl doses (clinicaltrials.gov registration: NCT01515566).},
author = {Dee, Laura E. and Thompson, Ross and Massol, Fran{\c{c}}ois and Guerrero, Angela and Bohan, David A.},
doi = {10.1016/j.tree.2017.06.001},
file = {::},
issn = {01695347},
journal = {Trends in Ecology and Evolution},
keywords = {ecological networks,ecosystem services,governance,motifs,networks,socioecological systems},
number = {8},
pages = {549--552},
pmid = {28651896},
publisher = {Elsevier Ltd},
title = {{Do Social–Ecological Syndromes Predict Outcomes for Ecosystem Services? – a Reply to Bodin et al.}},
url = {http://dx.doi.org/10.1016/j.tree.2017.06.001},
volume = {32},
year = {2017}
}
@article{Elmqvist2015,
abstract = {Cities are a key nexus of the relationship between people and nature and are huge centers of demand for ecosystem services and also generate extremely large environmental impacts. Current projections of rapid expansion of urban areas present fundamental challenges and also opportunities to design more livable, healthy and resilient cities (e.g. adaptation to climate change effects). We present the results of an analysis of benefits of ecosystem services in urban areas. Empirical analyses included estimates of monetary benefits from urban ecosystem services based on data from 25 urban areas in the USA, Canada, and China. Our results show that investing in ecological infrastructure in cities, and the ecological restoration and rehabilitation of ecosystems such as rivers, lakes, and woodlands occurring in urban areas, may not only be ecologically and socially desirable, but also quite often, economically advantageous, even based on the most traditional economic approaches.},
archivePrefix = {arXiv},
arxivId = {arXiv:1011.1669v3},
author = {Elmqvist, T. and Set{\"{a}}l{\"{a}}, H. and Handel, S. N. and van der Ploeg, S. and Aronson, J. and Blignaut, J. N. and G{\'{o}}mez-Baggethun, E. and Nowak, D. J. and Kronenberg, J. and de Groot, R.},
doi = {10.1016/j.cosust.2015.05.001},
eprint = {arXiv:1011.1669v3},
file = {::},
isbn = {1877-3435},
issn = {18773435},
journal = {Current Opinion in Environmental Sustainability},
pages = {101--108},
pmid = {22030279},
title = {{Benefits of restoring ecosystem services in urban areas}},
volume = {14},
year = {2015}
}
@article{Consortium2016,
abstract = {The ecosystem services (EcoS) concept is being used increasingly to attach values to natural systems and the multiple benefits they provide to human societies. Ecosystem processes or functions only become EcoS if they are shown to have social and/or economic value. This should assure an explicit connection between the natural and social sciences, but EcoS approaches have been criticized for retaining little natural science. Preserving the natural, ecological science context within EcoS research is challenging because the multiple disciplines involved have very different traditions and vocabularies (common-language challenge) and span many organizational levels and temporal and spatial scales (scale challenge) that define the relevant interacting entities (interaction challenge). We propose a network-based approach to transcend these discipline challenges and place the natural science context at the heart of EcoS research.},
author = {Bohan, D. A.},
doi = {10.1016/j.tree.2015.12.003},
file = {:Users/alyssacirtwill/Documents/Papers/Bohan{\_}2016{\_}Trends in Ecology and Evolution.pdf:pdf},
isbn = {0169-5347},
issn = {01695347},
journal = {Trends in Ecology and Evolution},
number = {2},
pages = {105--115},
pmid = {21227770},
publisher = {Elsevier Ltd},
title = {{Networking Our Way to Better Ecosystem Service Provision}},
url = {http://dx.doi.org/10.1016/j.tree.2015.12.003},
volume = {31},
year = {2016}
}
@article{bipartite,
abstract = {Infectious disease counts from surveillance systems are typically observed in several administrative geographical areas. In this paper, a non-linear model for the analysis of such multiple time series of counts is discussed. To account for heterogeneous incidence levels or varying transmission of a pathogen across regions, region-specific and possibly spatially correlated random effects are introduced. Inference is based on penalized likelihood methodology for mixed models. Since the use of classical model choice criteria such as AIC or BIC can be problematic in the presence of random effects, models are compared by means of one-step-ahead predictions and proper scoring rules. In a case study, the model is applied to monthly counts of meningococcal disease cases in 94 departments of France (excluding Corsica) and weekly counts of influenza cases in 140 administrative districts of Southern Germany. The predictive performance improves if existing heterogeneity is accounted for by random effects.},
author = {Paul, M. and Held, L.},
doi = {10.1002/sim.4177},
file = {::},
issn = {02776715},
journal = {Statistics in Medicine},
keywords = {Infectious diseases,Multivariate time series of counts,Proper scoring rules,Random effects},
number = {10},
pages = {1118--1136},
title = {{Predictive assessment of a non-linear random effects model for multivariate time series of infectious disease counts}},
url = {https://cran.r-project.org/web/packages/bipartite/bipartite.pdf},
volume = {30},
year = {2011}
}
@article{Dennis2006,
abstract = {The process of seed dispersal has a profound effect on vegetation structure and diversity in tropical forests. However, our understanding of the process and our ability to predict its outcomes at a community scale are limited by the frequently large number of interactions associated with it. Here, we outline an approach to dealing with this complexity that reduces the number of unique interactions considered by classifying the participants according to their functional similarity. We derived a classification of dispersers based on the nature of the dispersal service they provide to plants. We described the quantities of fruit handled, the quality of handling and the diversity of plants to which the service is provided. We used ten broad disperser traits to group 26 detailed measures for each disperser. We then applied this approach to vertebrate dispersers in Australia's tropical forests. Using this we also develop a classification that may be more generally applicable. For each disperser, data relating to each trait was obtained either from the field or published literature. First, we identified dispersers whose service outcomes were so distinct that statistical analysis was not required and assigned them to functional groups. The remaining dispersers were assigned to functional groups using cluster analysis. The combined processes created 15 functional groups from 65 vertebrate dispersers in Australian tropical forests. Our approach--grouping dispersers on the basis of the type of dispersal service provided and the fruit types it is provided to--represents a means of reducing the complexity encountered in tropical seed dispersal systems and could be effectively applied in community level studies. It also represents a useful tool for exploring changes in dispersal services when the distribution and abundance of animal populations change due to human impacts.},
author = {Dennis, Andrew J. and Westcott, David A.},
doi = {10.1007/s00442-006-0475-3},
file = {::},
isbn = {0442006047},
issn = {00298549},
journal = {Oecologia},
keywords = {Ecosystem,Frugivores,Granivores,Landscape,Rainforest},
number = {4},
pages = {620--634},
pmid = {16858588},
title = {{Reducing complexity when studying seed dispersal at community scales: A functional classification of vertebrate seed dispersers in tropical forests}},
volume = {149},
year = {2006}
}
@article{Luczkovich2003a,
abstract = {We present a graph theoretic model of analysing food web structure called regular equivalence. Regular equivalence is a method for partitioning the species in a food web into "isotrophic classes" that play the same structural roles, even if they are not directly consuming the same prey or if they do not share the same predators. We contrast regular equivalence models, in which two species are members of the same trophic group if they have trophic links to the same set of other trophic groups, with structural equivalence models, in which species are equivalent if they are connected to the exact same other species. Here, the regular equivalence approach is applied to two published food webs: (1) a topological web (Malaysian pitcher plant insect food web) and (2) a carbon-flow web (St. Marks, Florida seagrass ecosystem food web). Regular equivalence produced a more satisfactory set of classes than did the structural approach, grouping basal taxa with other basal taxa and not with top predators. Regular equivalence models provide a way to mathematically formalize trophic position, trophic group and trophic niche. These models are part of a family of models that includes structural models used extensively by ecologists now. Regular equivalence models uncover similarities in trophic roles at a higher level of organization than do the structural models. The approach outlined is useful for measuring the trophic roles of species in food web models, measuring similarity in trophic relations of two or more species, comparing food webs over time and across geographic regions, and aggregating taxa into trophic groups that reduce the complexity of ecosystem feeding relations without obscuring network relationships. In addition, we hope the approach will prove useful in predicting the outcome of predator-prey interactions in experimental studies. {\textcopyright} 2003 Elsevier Science Ltd. All rights reserved.},
author = {Luczkovich, Joseph J. and Borgatti, Stephen P. and Johnson, Jeffrey C. and Everett, Martin G.},
doi = {10.1006/jtbi.2003.3147},
file = {:Users/alyssacirtwill/Documents/Papers/Luczkovich et al.{\_}2003{\_}Journal of Theoretical Biology.pdf:pdf},
isbn = {1252328184},
issn = {00225193},
journal = {Journal of Theoretical Biology},
number = {3},
pages = {303--321},
pmid = {12468282},
title = {{Defining and measuring trophic role similarity in food webs using regular equivalence}},
volume = {220},
year = {2003}
}
@article{Estrada2007a,
abstract = {An important question in the network representation of ecological systems is to determine how direct and indirect interactions between species determine the positional importance of species in the ecosystem. Here we present a quantitative analysis of the similarities and differences of six different topological centrality measures as indicators of keystone species in 17 food webs. These indicators account for local, global and "meso-scale" - intermediate between local and global - topological information about species in the food webs. Using factor analysis we shown that most of these centrality indices share a great deal of topological information, which range from 75{\%} to 96{\%}. A generalized keystone indicator is then proposed by considering the factor loadings of the six-centrality measures, which contains most of the information encoded by these indices. However, the individual ordering of species according to these criteria display significant differences in most food webs. We simulate the effects of species extinction by removing species ranked according to a local and a "meso-scale" centrality indicator. The differences observed on three network characteristics - size, average distance and clustering coefficient of the largest component - after the removal of the most central nodes indicate that the consideration of these indices have different impacts for the ranking of species with conservational biology purposes. The "meso-scale" indicator appears to play an important role in determining the relative importance of species in epidemic spread and parasitism rates. {\textcopyright} 2007 Elsevier B.V. All rights reserved.},
archivePrefix = {arXiv},
arxivId = {arXiv:1311.2433v1},
author = {Estrada, Ernesto},
doi = {10.1016/j.ecocom.2007.02.018},
eprint = {arXiv:1311.2433v1},
file = {:Users/alyssacirtwill/Documents/Papers/Estrada{\_}2007{\_}Ecological Complexity.pdf:pdf},
isbn = {1476-945X},
issn = {1476945X},
journal = {Ecological Complexity},
keywords = {Betweenness,Centrality,Closeness,Ecoinformatics,Eigenvector,Network structure,Robustness,Subgraph centrality},
number = {1-2},
pages = {48--57},
title = {{Characterization of topological keystone species. Local, global and "meso-scale" centralities in food webs}},
volume = {4},
year = {2007}
}
@article{Vasas2006,
abstract = {There is increasing evidence that non-trophic interspecific interactions play an at least as important role in community dynamics as trophic relationships. More and more studies on pollination, mutualism and facilitation are published but these effects are interpreted more like alternative explanations than being synthesized with results of trophic analyses. Here, we construct and analyze the interaction web of the well-studied Chesapeake Bay mesohaline ecosystem. By interaction web we mean a food web completed by a carefully selected set of non-trophic links. We quantify the interaction structure of the web and the positional importance of nodes by different network indices. We perform the suitable analyses for different variants of the network: combinations of direction, sign and weights, as well as considering also non-trophic links result in a set of webs of different information content. We also create a semi-quantitative variant of the web, in which only the order of magnitude of the mass flows are considered. The appropriate network indices for each web variant are calculated and compared. Finally, however our paper is primarily of methodological nature, we present some findings about the fish community of the Bay. We suggest that the multiple techniques presented here, adapted even from social network analysis, can help field conservation efforts by suggesting optimal preferences for data collection. {\textcopyright} 2006 Elsevier B.V. All rights reserved.},
author = {Vasas, Vera and Jord{\'{a}}n, Ferenc},
doi = {10.1016/j.ecolmodel.2006.02.024},
file = {::},
isbn = {0304-3800},
issn = {03043800},
journal = {Ecological Modelling},
keywords = {Food web,Interaction web,Keystone species,Network theory,Positive interactions},
number = {3-4},
pages = {365--378},
pmid = {75},
title = {{Topological keystone species in ecological interaction networks: Considering link quality and non-trophic effects}},
volume = {196},
year = {2006}
}
@article{Funk2008,
abstract = {One of the greatest challenges for ecological restoration is to create or reassemble plant communities that are resistant to invasion by exotic species. We examine how concepts pertaining to the assembly of plant communities can be used to strengthen resistance to invasion in restored communities. Community ecology theory predicts that an invasive species will be unlikely to establish if there is a species with similar traits present in the resident community or if available niches are filled. Therefore, successful restoration efforts should select native species with traits similar to likely invaders and include a diversity of functional traits. The success of trait-based approaches to restoration will depend largely on the diversity of invaders, on the strength of environmental factors and on dispersal dynamics of invasive and native species. {\textcopyright} 2008 Elsevier Ltd. All rights reserved.},
archivePrefix = {arXiv},
arxivId = {arXiv:1704.01415v1},
author = {Funk, Jennifer L. and Cleland, Elsa E. and Suding, Katherine N. and Zavaleta, Erika S.},
doi = {10.1016/j.tree.2008.07.013},
eprint = {arXiv:1704.01415v1},
file = {::},
isbn = {0169-5347},
issn = {01695347},
journal = {Trends in Ecology and Evolution},
number = {12},
pages = {695--703},
pmid = {18951652},
title = {{Restoration through reassembly: plant traits and invasion resistance}},
volume = {23},
year = {2008}
}
@article{Lai2015,
abstract = {Species importance is often defined by how central the role a species plays in a food web. Here we argue that species importance should also incorporate the concept of uniqueness of its network position. By developing a previous methodology, we propose a new approach for species uniqueness. Our methodology quantifies the interaction structure of species, separates strong and weak interactors in a threshold-dependent way and measures the similarity of strong interactors' identity for pairs of species. By exploring various threshold values systematically, the profile of a species' trophic overlap can be constructed. Summing up the extent of trophic overlap across the whole profile for a species, we then obtain a new measure for species uniqueness. A unique species should overlap less with other species and therefore has a low value for this new measure. We demonstrate our methodology by using the food web representation of Prince Williams Sound ecosystem. Drastic differences between our new result and that derived from the previous approach are found. We further compare the result with other network indices and found that the information generated from our new approach differs from the majority others. This in turn implies our approach offer an alternative view on species importance that may complement the existing methodologies in the literature.},
author = {Kudeken, N. and Kawakami, K. and Kakazu, T. and Takushi, Y. and Kakazu, T. and Fukuhara, H. and Nakamura, H. and Kaneshima, H. and Saito, A. and Toda, T.},
doi = {10.1016/j.ecolmodel.2014.12.014},
file = {:Users/alyssacirtwill/Documents/Papers/Kudeken et al.{\_}1993{\_}Japanese Journal of Thoracic Diseases.pdf:pdf},
issn = {03043800},
journal = {Japanese Journal of Thoracic Diseases},
number = {12},
pages = {1585--1590},
publisher = {Elsevier B.V.},
title = {{A case of acetaminophen-induced pneumonitis}},
url = {http://linkinghub.elsevier.com/retrieve/pii/S0304380014006206},
volume = {31},
year = {1993}
}
@article{Kolar2001,
abstract = {Predicting which species are probable invaders has been a long-standing goal of ecologists, but only recently have quantitative methods been used to achieve such a goal. Although restricted to few taxa, these studies reveal clear relationships between the characteristics of releases and the species involved, and the successful establishment and spread of invaders. For example, the probability of bird establishment increases with the number of individuals released and the number of release events. Also, the probability of plant invasiveness increases if the species has a history of invasion and reproduces vegetatively. These promising quantitative approaches should be more widely applied to allow us to predict patterns of invading species more successfully.},
author = {Kolar, Cynthia S. and Lodge, David M.},
doi = {10.1016/S0169-5347(01)02101-2},
file = {::},
isbn = {0169-5347},
issn = {01695347},
journal = {Trends in Ecology and Evolution},
number = {4},
pages = {199--204},
pmid = {11245943},
title = {{Progress in invasion biology: Predicting invaders}},
volume = {16},
year = {2001}
}
@article{Pocock2012,
abstract = {Understanding species' interactions and the robustness of interaction networks to species loss is essential to understand the effects of species' declines and extinctions. In most studies, different types of networks (such as food webs, parasitoid webs, seed dispersal networks, and pollination networks) have been studied separately. We sampled such multiple networks simultaneously in an agroecosystem. We show that the networks varied in their robustness; networks including pollinators appeared to be particularly fragile. We show that, overall, networks did not strongly covary in their robustness, which suggests that ecological restoration (for example, through agri-environment schemes) benefitting one functional group will not inevitably benefit others. Some individual plant species were disproportionately well linked to many other species. This type of information can be used in restoration management, because it identifies the plant taxa that can potentially lead to disproportionate gains in biodiversity.},
author = {Pocock, Michael J O and Evans, Darren M. and Memmott, Jane},
doi = {10.1126/science.1214915},
file = {::},
isbn = {0036-8075},
issn = {10959203},
journal = {Science},
number = {6071},
pages = {973--977},
pmid = {22363009},
title = {{The robustness and restoration of a network of ecological networks}},
volume = {335},
year = {2012}
}
@article{CotteeJones2012,
abstract = {Nowhere are tensions between motorists, bicyclists and buses higher than in San Francisco, the birthplace of the freeway revolts, the Transit First ordinance, and Critical Mass. In Street Fight, geographer Jason Henderson offers a fresh perspective into the battle for limited urban road space, delving into the ideologies underlying the politics of mobility. Released this spring, his first book proves a provocative read for those engaged in sustainability and urban livability debates},
archivePrefix = {arXiv},
arxivId = {arXiv:physics/0608246v3},
author = {Pines, Jesse and Isserman, Joshua and Kelly, John},
doi = {10.5811/westjem.2011.5.6700},
eprint = {0608246v3},
file = {::},
isbn = {9780980200454},
issn = {1936900X},
journal = {Western Journal of Emergency Medicine},
keywords = {Biogeography,Ecology,community composition,definitions,dominant species,ecosystem engineer,keystone concept,keystone species},
number = {1},
pages = {1--10},
pmid = {24696751},
primaryClass = {arXiv:physics},
title = {{Perceptions of Emergency Department Crowding in the Commonwealth of Pennsylvania}},
url = {http://www.escholarship.org/uc/item/5d91b6r1},
volume = {14},
year = {2013}
}
@article{Newman2000,
abstract = {Using computer databases of scientific papers in physics, biomedical research, and computer science, we have constructed networks of collaboration between scientists in each of these disciplines. In these networks two scientists are considered connected if they have coauthored one or more papers together. Here we study a variety of nonlocal statistics for these networks, such as typical distances between scientists through the network, and measures of centrality such as closeness and betweenness. We further argue that simple networks such as these cannot capture variation in the strength of collaborative ties and propose a measure of collaboration strength based on the number of papers coauthored by pairs of scientists, and the number of other scientists with whom they coauthored those papers.},
archivePrefix = {arXiv},
arxivId = {cond-mat/0011144},
author = {Newman, M. E.J.},
doi = {10.1103/PhysRevE.64.016132},
eprint = {0011144},
file = {::},
isbn = {978-3-540-44485-5},
issn = {1063651X},
journal = {Physical Review E - Statistical Physics, Plasmas, Fluids, and Related Interdisciplinary Topics},
number = {1},
pages = {7},
pmid = {1348},
primaryClass = {cond-mat},
title = {{Scientific collaboration networks. II. Shortest paths, weighted networks, and centrality}},
url = {http://arxiv.org/abs/cond-mat/0011144{\%}0Ahttp://dx.doi.org/10.1103/PhysRevE.64.016132},
volume = {64},
year = {2001}
}
@book{Jolliffe2002,
abstract = {Principal component analysis is one of the most important and powerful methods in chemometrics as well as in a wealth of other areas.Principal component analysis is one of the most important and powerful methods in chemometrics as well as in a wealth of other areas. This paper provides a description of how to understand, use, and interpret principal component analysis. The paper focuses on the use of principal component analysis in typical chemometric areas but the results are generally applicable.},
address = {New York},
archivePrefix = {arXiv},
arxivId = {arXiv:1011.1669v3},
author = {Wang, Qianqian and Gao, Quanxue and Gao, Xinbo and Nie, Feiping},
booktitle = {IJCAI International Joint Conference on Artificial Intelligence},
doi = {10.1039/c3ay41907j},
edition = {Second Edi},
eprint = {arXiv:1011.1669v3},
isbn = {9780999241103},
issn = {10450823},
number = {3},
pages = {2936--2942},
pmid = {21811560},
publisher = {Springer-Verlag},
title = {{Angle principal component analysis}},
url = {http://onlinelibrary.wiley.com/doi/10.1002/0470013192.bsa501/full},
volume = {30},
year = {2017}
}
@article{fitdistrplus,
abstract = {The package fitdistrplus provides functions for fitting univariate distributions to different types of data (continuous censored or non-censored data and discrete data) and allowing different estimation methods (maximum likelihood, moment matching, quantile matching and maximum goodness-of-fit estimation). Outputs of fitdist and fitdistcens functions are S3 objects, for which kind generic methods are provided, including summary, plot and quantile. This package also provides various functions to compare the fit of several distributions to a same data set and can handle bootstrap of parameter estimates. Detailed examples are given in food risk assessment, ecotoxicology and insurance contexts.},
author = {Delignette-Muller, Marie Laure and Dutang, Christophe},
doi = {10.18637/jss.v064.i04},
isbn = {9781420065213},
issn = {1548-7660},
journal = {Journal of Statistical Software},
number = {4},
pages = {1 -- 34},
title = {{fitdistrplus : An R Package for Fitting Distributions }},
volume = {64},
year = {2015}
}
@incollection{Jacob2011,
abstract = {Human-induced habitat destruction, overexploitation, introduction of alien species and climate change are causing species to go extinct at unprecedented rates, from local to global scales. There are growing concerns that these kinds of disturbances alter important functions of ecosystems. Our current understanding is that key parameters of a community (e.g. its functional diversity, species composition, and presence/absence of vulnerable species) reflect an ecological network's ability to resist or rebound from change in response to pressures and disturbances, such as species loss. If the food web structure is relatively simple, we can analyse the roles of different species interactions in determining how environmental impacts translate into species loss. However, when ecosystems harbour species-rich communities, as is the case in most natural systems, then the complex network of ecological interactions makes it a far more challenging task to perceive how species' functional roles influence the consequences of species loss. One approach to deal with such complexity is to focus on the functional traits of species in order to identify their respective roles: for instance, large species seem to be more susceptible to extinction than smaller species. Here, we introduce and analyse the marine food web from the high Antarctic Weddell Sea Shelf to illustrate the role of species traits in relation to network robustness of this complex food web. Our approach was threefold: firstly, we applied a new classification system to all species, grouping them by traits other than body size; secondly, we tested the relationship between body size and food web parameters within and across these groups and finally, we calculated food web robustness. We addressed questions regarding (i) patterns of species functional/trophic roles, (ii) relationships between species functional roles and body size and (iii) the role of species body size in terms of network robustness. Our results show that when analyzing relationships between trophic structure, body size and network structure, the diversity of predatory species types needs to be considered in future studies. {\textcopyright} 2011 Elsevier Ltd.},
address = {Amsterdam},
author = {Jacob, Ute and Thierry, Aaron and Brose, Ulrich and Arntz, Wolf E. and Berg, Sofia and Brey, Thomas and Fetzer, Ingo and Jonsson, Tomas and Mintenbeck, Katja and M{\"{o}}llmann, Christian and Petchey, Owen L. and Riede, Jens O. and Dunne, Jennifer A.},
booktitle = {Advances in Ecological Research},
doi = {10.1016/B978-0-12-386475-8.00005-8},
editor = {Belgrano, Andrea and Reiss, Julia},
file = {::},
isbn = {9780123864758},
issn = {00652504},
keywords = {Ecosystem,Food web,Network structure,Weddell Sea},
pages = {181--223},
publisher = {Academic Press},
title = {{The Role of Body Size in Complex Food Webs. A Cold Case}},
volume = {45},
year = {2011}
}
@article{Cohen2009,
abstract = {Many studies have aimed to understand food webs by investigating components such as trophic links (one consumer taxon eats one resource taxon), tritrophic interactions (one consumer eats an intermediate taxon, which eats a resource), or longer chains of links. We show here that none of these components (links, tritrophic interactions, and longer chains), individually or as an ensemble, accounts fully for the properties of the next higher level of organization. As a cell is more than its molecules, as an organ is more than its cells, and as an organism is more than its organs, in a food web, new structure emerges at every organizational level up to and including the whole web. We demonstrate the emergence of properties at progressively higher levels of structure by using all of the directly observed, appropriately organized, publicly available food web datasets with relatively complete trophic link data and with average body mass and population density data for each taxon. There are only three such webs, those of Tuesday Lake, Michigan, in 1984 and 1986, and Ythan Estuary, Scotland. We make the data freely available online with this report. Differences in web patterns between Tuesday Lake and Ythan Estuary, and similarities of Tuesday Lake in 1984 and 1986 despite 50{\%} turnover of species, suggest that the patterns we describe respond to major differences between ecosystem types.},
archivePrefix = {arXiv},
arxivId = {arXiv:1408.1149},
author = {Cohen, Joel E. and Schittler, Daniella N. and Raffaelli, David G. and Reuman, Daniel C.},
doi = {10.1073/pnas.0910582106},
eprint = {arXiv:1408.1149},
file = {::},
isbn = {0910582106},
issn = {0027-8424},
journal = {Proceedings of the National Academy of Sciences},
number = {52},
pages = {22335--22340},
pmid = {20018774},
title = {{Food webs are more than the sum of their tritrophic parts}},
url = {http://www.pnas.org/cgi/doi/10.1073/pnas.0910582106},
volume = {106},
year = {2009}
}
@article{LimaMendez2015,
abstract = {Species interaction networks are shaped by abiotic and biotic factors. Here, as part of the Tara Oceans project, we studied the photic zone interactome using environmental factors and organismal abundance profiles and found that environmental factors are incomplete predictors of community structure. We found associations across plankton functional types and phylogenetic groups to be nonrandomly distributed on the network and driven by both local and global patterns. We identified interactions among grazers, primary producers, viruses, and (mainly parasitic) symbionts and validated network-generated hypotheses using microscopy to confirm symbiotic relationships. We have thus provided a resource to support further research on ocean food webs and integrating biological components into ocean models.},
archivePrefix = {arXiv},
arxivId = {arXiv:1011.1669v3},
author = {Lima-Mendez, G. and Faust, K. and Henry, N. and Decelle, J. and Colin, S. and Carcillo, F. and Chaffron, S. and Ignacio-Espinosa, J. C. and Roux, S. and Vincent, F. and Bittner, L. and Darzi, Y. and Wang, J. and Audic, S. and Berline, L. and Bontempi, G. and Cabello, A. M. and Coppola, L. and Cornejo-Castillo, F. M. and D'Ovidio, F. and {De Meester}, L. and Ferrera, I. and Garet-Delmas, M.-J. and Guidi, L. and Lara, E. and Pesant, S. and Royo-Llonch, M. and Salazar, G. and Sanchez, P. and Sebastian, M. and Souffreau, C. and Dimier, C. and Picheral, M. and Searson, S. and Kandels-Lewis, S. and Gorsky, G. and Not, F. and Ogata, H. and Speich, S. and Stemmann, L. and Weissenbach, J. and Wincker, P. and Acinas, S. G. and Sunagawa, S. and Bork, P. and Sullivan, M. B. and Karsenti, E. and Bowler, C. and de Vargas, C. and Raes, J.},
doi = {10.1126/science.1262073},
eprint = {arXiv:1011.1669v3},
file = {::},
isbn = {1095-9203 (Electronic) 0036-8075 (Linking)},
issn = {0036-8075},
journal = {Science},
number = {6237},
pages = {1262073--1262073},
pmid = {25999517},
title = {{Determinants of community structure in the global plankton interactome}},
url = {http://www.sciencemag.org/content/348/6237/1262073.full.pdf},
volume = {348},
year = {2015}
}
@article{Eklof2011,
abstract = {Explaining the structure of ecosystems is one of the great challenges of ecology. Simple models for food web structure aim at disentangling the complexity of ecological interaction networks and detect the main forces that are responsible for their shape. Trophic interactions are influenced by species traits, which in turn are largely determined by evolutionary history. Closely related species are more likely to share similar traits, such as body size, feeding mode and habitat preference than distant ones. Here, we present a theoretical framework for analysing whether evolutionary history—represented by taxonomic classification—provides valuable information on food web structure. In doing so, we measure which taxonomic ranks better explain species interactions. Our analysis is based on partitioning of the species into taxonomic units. For each partition, we compute the likelihood that a probabilistic model for food web structure reproduces the data using this information. We find that taxonomic partitions produce significantly higher likelihoods than expected at random. Marginal likelihoods (Bayes factors) are used to perform model selection among taxonomic ranks. We show that food webs are best explained by the coarser taxonomic ranks (kingdom to class). Our methods provide a way to explicitly include evolutionary history in models for food web structure.},
author = {Ekl{\"{o}}f, Anna and Helmus, Matthew R. and Moore, M. and Allesina, Stefano},
doi = {10.1098/rspb.2011.2149},
file = {::},
isbn = {1471-2954 (Electronic)$\backslash$n0962-8452 (Linking)},
issn = {14712970},
journal = {Proceedings of the Royal Society B: Biological Sciences},
keywords = {Complex networks,Dimension,Food webs,Species traits,Taxonomy},
number = {1733},
pages = {1588--1596},
pmid = {22090387},
title = {{Relevance of evolutionary history for food web structure}},
url = {http://rspb.royalsocietypublishing.org/cgi/doi/10.1098/rspb.2011.2149},
volume = {279},
year = {2012}
}
@article{Brown2016,
author = {Vickery, Peter D and Hunter, Malcolm L and Wells, Jeffrey V and Vickery, Peter D and Hunter, Malcolm L},
file = {::},
number = {4},
pages = {706--710},
title = {{Linked references are available on JSTOR for this article : IS DENSITY AN INDICATOR OF BREEDING SUCCESS ?}},
volume = {109},
year = {2016}
}
@misc{Gerritsen1977,
abstract = {Predator–prey interactions between swimming animals of the zooplankton are studied in a mathematical model. The assumptions are: 1) the animals are points in a 1-m3 homogeneous space, 2) the animals move at random and are randomly distributed, and 3) the predator animal has an encounter radius given by its sensory system. The mathematics of encounter probabilities are developed for a 3-dimensional space. The results show two optimal strategies: 1) cruising predators which prey upon slow moving animals (herbivores), and 2) ambush (nonmoving) predators which prey upon fast cruising prey. Of the variables used (population densities, speeds of the two animal species, and encounter radius) the encounter radius has the greatest influence on the encounter probabilities. The results suggest a simple community structure and point to the importance of studies on live zooplankton.},
archivePrefix = {arXiv},
arxivId = {https://doi.org/10.2307/1943058},
author = {Gerritsen, Jeroen and Strickler, J. Rudi},
booktitle = {Journal of the Fisheries Research Board of Canada},
doi = {10.1139/f77-008},
eprint = {/doi.org/10.2307/1943058},
file = {::},
isbn = {0706-652X},
issn = {0015-296X},
number = {1},
pages = {73--82},
pmid = {148},
primaryClass = {https:},
title = {{Encounter Probabilities and Community Structure in Zooplankton: a Mathematical Model}},
url = {http://www.nrcresearchpress.com/doi/10.1139/f77-008},
volume = {34},
year = {2011}
}
@article{Servedio2011,
abstract = {Speciation with gene flow is greatly facilitated when traits subject to divergent selection also contribute to non-random mating. Such traits have been called 'magic traits', which could be interpreted to imply that they are rare, special, or unrealistic. Here, we question this assumption by illustrating that magic traits can be produced by a variety of mechanisms, including ones in which reproductive isolation arises as an automatic by-product of adaptive divergence. We also draw upon the theoretical literature to explore whether magic traits have a unique role in speciation or can be mimicked in their effects by physically linked trait-complexes. We conclude that magic traits are more frequent than previously perceived, but further work is needed to clarify their importance. {\textcopyright} 2011 Elsevier Ltd.},
author = {Servedio, Maria R. and Doorn, G. Sander Van and Kopp, Michael and Frame, Alicia M. and Nosil, Patrik},
doi = {10.1016/j.tree.2011.04.005},
file = {::},
isbn = {0169534711001},
issn = {01695347},
journal = {Trends in Ecology and Evolution},
number = {8},
pages = {389--397},
pmid = {21592615},
title = {{Magic traits in speciation: 'magic' but not rare?}},
volume = {26},
year = {2011}
}
@article{Beauchamp1999,
abstract = {We developed a visual foraging model for piscivores that predicts search volume as a function of light and turbidity. We combined this model with diel hydroacoustic measurements of depth-specific prey fish densities during summer stratification in Lake Tahoe, Lake Washington, and Strawberry Reservoir to examine differences in diel, depth-specific visual encounter rates of prey. These study sites were selected to represent gradients of increasing limnetic prey fish density and declining transparency. The model predicted over a 30-fold difference in maximum depth-specific diel encounter rates among the three lakes. Lake Washington, which was characterized by intermediate transparency and moderate limnetic prey density, had the highest predicted prey encounter rates. The pattern of prey encounter rates among the three lakes was similar to the proportional contribution of limnetic prey fishes observed in the diet of piscivores from these waters. This approach may be used to construct temporally and spatially explicit trophic interaction models for examining mechanisms underlying predator and prey distributions or to predict the response of existing or introduced predators to changing environmental conditions, prey abundance, or distribution.},
author = {Beauchamp, David A and Baldwin, Casey M and Vogel, Jason L and Gubala, Chad P},
doi = {10.1139/f99-217},
file = {::},
isbn = {0706-652X},
issn = {0706-652X},
journal = {Canadian Journal of Fisheries and Aquatic Sciences},
number = {S1},
pages = {128--139},
pmid = {2206013154731626671},
title = {{Estimating diel, depth-specific foraging opportunities with a visual encounter rate model for pelagic piscivores}},
url = {http://www.nrc.ca/cgi-bin/cisti/journals/rp/rp2{\_}abst{\_}e?cjfas{\_}f99-217{\_}56{\_}ns{\_}nf{\_}cjfas56-99},
volume = {56},
year = {2011}
}
@article{Gremillet1999,
abstract = {カワウの採餌行動の個体差と潜水行動の研究。 Diving seabirds should evolve a variety of foraging characteristics which enable them to minimize energy expenditure and to maximize net energy gain while searching for prey underwater. In order to assess the related ecological adaptations in a marine predator, we studied the at-sea distribution and the diving behaviour of 23 cormorants Phalacrocorax carbo (Linnaeus) breeding at the Chausey Islands (France) using VHE-telemetry and data loggers recording hydrostatic pressure. Birds foraged within an area of approximately 1131 km(2) situated north-east of the breeding colony. This zone represents only 25 {\%} of the maximal potentially available area that the birds may utilize considering their maximum foraging range of 35 km. Individual birds remained within restricted individual foraging areas (on average 18 and 10{\%} of the total utilized area in 1994 and 1995, respectively) throughout the study period. Moreover, the cormorants studied conducted an average of 42 dives per foraging trip, lasting for an average of 40 s (maximum 152 s), and reached an average maximum dive depth of 6.1 m (maximum 32 m) with median descent and ascent angles calculated to be 18.7 degrees and 20.3 degrees, respectively. Overall, 64{\%} of all dives were U-shaped dives and 36{\%} V-shaped dives. We use these results to demonstrate how both specialization and opportunism may support the remarkably high foraging efficiency of this marine predator.},
author = {Gr{\'{e}}millet, David and Wilson, Rory Paul and Storch, Sandra and Gary, Yann},
doi = {10.3354/meps183263},
file = {::},
isbn = {0171-8630},
issn = {01718630},
journal = {Marine Ecology Progress Series},
keywords = {Diving behaviour,Foraging strategy,Phalacrocorax carbo,Specialization vs opportunism,Wildlife telemetry},
number = {1},
pages = {263--273},
pmid = {249},
title = {{Three-dimensional space utilization by a marine predator}},
volume = {183},
year = {1999}
}
@article{Roth2007a,
abstract = {Trees that fall into lakes from riparian forest become habitat for aquatic organisms, and are potentially an important link between terrestrial and aquatic ecosystems. Coarse woody habitat (CWH) promotes the production of benthic invertebrate prey and offers refuge for prey fishes, which are in turn consumed by piscivorous fishes. We used a simulation model to explore responses of an aquatic food web to changes in CWH caused by lakeshore residential development and windstorm, as well as to harvest of adult piscivores. Residential development had a negative effect on fishes, and could lead to extirpation of benthivorous prey fish species. In contrast, pulsed addition of CWH following a windstorm had little effect on the aquatic food web. Our results suggest that CWH is more important as shelter for prey fishes than as a substrate for benthic invertebrate production. However, piscivore harvest can supersede the role of CWH as prey shelter, leading to piscivore collapse and prey persistence even when CWH levels are low enough to promote piscivore dominance. Thus, the effects of lakeshore residential development on fishes can be masked by angler harvest of top piscivores. {\textcopyright} 2007 Elsevier B.V. All rights reserved.},
author = {Roth, Brian M. and Kaplan, Isaac C. and Sass, Greg G. and Johnson, Pieter T. and Marburg, Anna E. and Yannarell, Anthony C. and Havlicek, Tanya D. and Willis, Theodore V. and Turner, Monica G. and Carpenter, Stephen R.},
doi = {10.1016/j.ecolmodel.2006.12.005},
file = {::},
isbn = {0304-3800},
issn = {03043800},
journal = {Ecological Modelling},
keywords = {Coarse woody habitat,Food web,Residential development},
number = {3-4},
pages = {439--452},
pmid = {732275},
title = {{Linking terrestrial and aquatic ecosystems: The role of woody habitat in lake food webs}},
volume = {203},
year = {2007}
}
@article{Brown1999,
abstract = {Larger areas support more species. To test the application of this biogeographic principle to ponds, we consider the relationship between size and diversity for 80 ponds in Switzerland, using richness (number of species) and conservation value (score for all species present, according to their degree of rarity) of aquatic plants, molluscs (Gastropoda, Sphaeriidae), Coleoptera, Odonata (adults) and Amphibia. Pond size was found to be important only for Odonata and explained 31{\%} of the variability of their species richness. Pond size showed only a feeble relationship with the species richness of all other groups, particularly the Coleoptera and Amphibia. The weakness of this relationship was also indicated by the low z-values obtained ({\textless} 0.13). The SLOSS analyses showed that a set of ponds of small size has more species and has a higher conservation value than a single large pond of the same total area. But we also show that large ponds harbour species missing in the smaller ponds. Finally, we conclude that in a global conservation policy (protection, restoration, management), all size ranges of ponds should be promoted. {\textcopyright} 2002 Elsevier Science Ltd. All rights reserved.},
author = {Oertli, Beat and Joye, Dominique Auderset and Castella, Emmanuel and Juge, Rapha{\"{e}}lle and Cambin, Diana and Lachavanne, Jean Bernard},
doi = {10.1016/S0006-3207(01)00154-9},
file = {::},
isbn = {00063207 (ISSN)},
issn = {00063207},
journal = {Biological Conservation},
keywords = {Amphibia,Aquatic plants,Biodiversity conservation,Coleoptera,Gastropoda,Odonata,Species richness},
number = {1},
pages = {59--70},
pmid = {5351},
title = {{Does size matter? The relationship between pond area and biodiversity}},
volume = {104},
year = {2002}
}
@article{Brown2001,
abstract = {We revisit the problem of interval estimation of a binomial proportion. The erratic behavior of the coverage probability of the standard Wald confidence interval has previously been remarked on in the literature (Blyth and Still, Agresti and Coull, Santner and others). We begin by showing that the chaotic coverage properties of the Wald interval are far more persistent than is appreciated. Furthermore, common textbook prescriptions regarding its safety are misleading and defective in several respects and cannot be trusted. This leads us to consideration of alternative intervals. A number of natural alternatives are presented, each with its motivation and context. Each interval is examined for its coverage probability and its length. Based on this analysis, we recommend the Wilson interval or the equal-tailed Jeffreys prior interval for small n and the interval suggested in Agresti and Coull for larger n. We also provide an additional frequentist justification for use of the Jeffreys interval.},
archivePrefix = {arXiv},
arxivId = {arXiv:1303.1288v1},
author = {DasGupta, Anirban and Cai, T. Tony and Brown, Lawrence D.},
doi = {10.1214/ss/1009213286},
eprint = {arXiv:1303.1288v1},
isbn = {0883-4237},
issn = {0883-4237},
journal = {Statistical Science},
keywords = {and phrases,bayes,binomial distribution,confidence,coverage probability,edgeworth expansion,expected length,intervals,jeffreys prior,normal approximation,posterior},
number = {2},
pages = {101--133},
pmid = {17566141},
title = {{Interval Estimation for a Binomial Proportion}},
url = {http://projecteuclid.org/euclid.ss/1009213286},
volume = {16},
year = {2001}
}
@article{Olden2004,
abstract = {Biotic homogenization, the gradual replacement of native biotas by locally expanding non-natives, is a global process that diminishes floral and faunal distinctions among regions. Although patterns of homogenization have been well studied, their specific ecological and evolutionary consequences remain unexplored. We argue that our current perspective on biotic homogenization should be expanded beyond a simple recognition of species diversity loss, towards a synthesis of higher order effects. Here, we explore three distinct forms of homogenization (genetic, taxonomic and functional), and discuss their immediate and future impacts on ecological and evolutionary processes. Our goal is to initiate future research that investigates the broader conservation implications of homogenization and to promote a proactive style of adaptive management that engages the human component of the anthropogenic blender that is currently mixing the biota on Earth.},
author = {Olden, Julian D. and Poff, N. Le Roy and Douglas, Marlis R. and Douglas, Michael E. and Fausch, Kurt D.},
doi = {10.1016/j.tree.2003.09.010},
isbn = {0169-5347},
issn = {01695347},
journal = {Trends in Ecology and Evolution},
number = {1},
pages = {18--24},
pmid = {16701221},
title = {{Ecological and evolutionary consequences of biotic homogenization}},
volume = {19},
year = {2004}
}
@article{Layman2012,
abstract = {Population sizes of generalist consumers are increasing in many ecosystems because of various human activities, and it is critical to understand the trophic role of these generalist species if we are to predict how they may affect food web structure and ecosystem function. Lionfish Pterois volitans/miles have spread throughout the Western Atlantic and Gulf of Mexico and they may have significant effects on native faunal communities. We characterized the trophic ecology of lionfish in back reef habitats on Abaco Island, Bahamas, drawing on recently developed analytical tools that employ both direct diet information and stable isotope data. Although d 15N and d 13C bi-plot data appeared to suggest substantial niche overlap with native gray snapper and schoolmaster snapper, Bayesian analytical tools suggested differences in core isotopic niches among the species. This was consistent with direct diet information, as lionfish fed almost exclusively on small prey fishes and snapper fed more commonly on crustaceans. When combining empirical isotope and diet data in a simulation model, individual lionfish appear to be more specialized in their diets than schoolmaster snapper. We suggest that this pattern may be driven by high site-fidelity of lionfish, in conjunction with distinct prey assemblages at the patch scale. Lionfish are widely considered to be generalist predators, and our data reveal aspects of this trophic generality that must be considered as the role of lionfish in their invaded habitats continues to be examined. {\^{A}}{\textcopyright} Inter-Research 2012.},
author = {Layman, Craig A. and Allgeier, Jacob E.},
doi = {10.3354/meps09511},
isbn = {0171-8630$\backslash$n1616-1599},
issn = {01718630},
journal = {Marine Ecology Progress Series},
keywords = {Food web,Individual specialization,Invasive species,Optimal foraging,Predator-prey interaction,Pterois miles,Pterois volitans},
pages = {131--141},
pmid = {12650459},
title = {{Characterizing trophic ecology of generalist consumers: A case study of the invasive lionfish in the Bahamas}},
volume = {448},
year = {2012}
}
@article{Wirsing2008,
abstract = {Objective Neurometabolic disorders are an important group of diseases that mostly are presented in newborns and infants$\backslash$nNeurological manifestations are the prominent signs and symptoms in this group of diseases. Seizures are a common sign and are often refractory to antiepileptic drugs in untreated neurometabolic patients.$\backslash$nThe onset of symptoms for neurometabolic disorders appears after an interval of normal or near normal growth and development.Additionally, affected children may fare well until a catabolic crisis occurs.$\backslash$nPatients with neurometabolic disorders during metabolic decompensation have severe clinical presentation, which include poor feeding, vomiting, lethargy, seizures, and loss of consciousness.$\backslash$nThis symptom is often fatal but severe neurological insult and regression in neurodevelopmental milestones can result as a prominent sign in patients who survived.$\backslash$nAcute symptoms should be immediately treated regardless of the cause.$\backslash$nA number of patients with neurometabolic disorders respond favorably and, in some instances, dramatically respond to treatment.$\backslash$nEarly detection and early intervention is invaluable in some patients to prevent catabolism and normal or near normal neurodevelopmental milestones.$\backslash$nThis paper discusses neurometabolic disorders, approaches to this group of diseases (from the view of a pediatric neurologist), clinical and neurological manifestations, neuroimaging and electroencephalography findings, early detection, and early treatment.},
archivePrefix = {arXiv},
arxivId = {arXiv:1011.1669v3},
author = {Wirsing, Aaron J. and Heithaus, Michael R. and Frid, Alejandro and Dill, Lawrence M.},
doi = {10.1111/j.1748-7692.2007.00167.x},
eprint = {arXiv:1011.1669v3},
isbn = {9781908477316},
issn = {08240469},
journal = {Marine Mammal Science},
keywords = {Bottlenose dolphin,Dugong,Dugong dugon,Harbor seals,Intimidation,Phoca vitulina,Predation risk,Time allocation,Trait-mediated indirect interactions,Tursiops},
number = {1},
pages = {1--15},
pmid = {25767534},
title = {{Seascapes of fear: Evaluating sublethal predator effects experienced and generated by marine mammals}},
volume = {24},
year = {2008}
}
@article{Luttbeg2005,
abstract = {Although the principles of disruptive colouration are widely believed to explain a variety of animal colour patterns, there has been no field evidence that it works to reduce the detection rates of natural prey. In a recent paper, Cuthill et al. successfully address this shortfall, separating the benefits of background matching from those of disruptive colouration. Their results provide the first definitive field support for this long-recognized phenomenon and suggest several new avenues of research. {\textcopyright} 2005 Elsevier Ltd. All rights reserved.},
author = {Rosenberg, Miriam I. and Desplan, Claude},
doi = {10.1126/science.1192769},
isbn = {9781315681375},
issn = {00368075},
journal = {Science},
number = {5989},
pages = {284--285},
pmid = {16701405},
title = {{Hiding in plain sight}},
volume = {329},
year = {2010}
}
@article{Laundre2010,
abstract = {“Predation risk” and “fear” are concepts well established in animal behavior literature. We expand these concepts to develop the model of the “landscape of fear”. The landscape of fear represents relative levels of predation risk as peaks and valleys that reflect the level of fear of predation a prey experiences in different parts of its area of use. We provide observations in support of this model regarding changes in predation risk with respect to habitat types, and terrain characteristics. We postulate that animals have the ability to learn and can respond to differing levels of predation risk. We propose that the landscape of fear can be quantified with the use of well documented existing methods such as giving- up densities, vigilance observations, and foraging surveys of plants. We conclude that the landscape of fear is a useful visual model and has the potential to become a unifying ecological concept.},
archivePrefix = {arXiv},
arxivId = {arXiv:1011.1669v3},
author = {Laundre, John W. and Hernandez, Lucina and Ripple, William J.},
doi = {10.2174/1874213001003030001},
eprint = {arXiv:1011.1669v3},
isbn = {1874-2130},
issn = {18742130},
journal = {The Open Ecology Journal},
keywords = {fear,fear a,introduction to this special,issue,killed by its predator,landscape of fear,may affect the basic,predation risk,predators,prey,prey has of being,this special issue attempts,to investigate how the},
number = {3},
pages = {1--7},
pmid = {25246403},
title = {{The Landscape of Fear: Ecological Implications of Being Afraid{\~{}}!2009-09-09{\~{}}!2009-11-16{\~{}}!2010-02-02{\~{}}!}},
url = {http://benthamopen.com/ABSTRACT/TOECOLJ-3-3-1},
volume = {3},
year = {2010}
}
@article{Gounand2017,
abstract = {The meta-ecosystem framework demonstrates the significance of among-ecosystem spatial flows for ecosystem dynamics and has fostered a rich body of theory. The high level of abstraction of the models, however, impedes applications to empirical systems. We argue that further understanding of spatial dynamics in natural systems strongly depends on dense exchanges between field and theory. From empiricists, more and specific quantifications of spatial flows are needed, defined by the major categories of organismal movement (dispersal, foraging, life-cycle, and migration). In parallel, the theoretical framework must account for the distinct spatial scales at which these naturally common spatial flows occur. Integrating all levels of spatial connections among landscape elements will upgrade and unify landscape and meta-ecosystem ecology into a single framework for spatial ecology. Cross-ecosystem movements drive landscape dynamics. Among-ecosystem couplings are either dispersal- or resource-dominated. Dispersal-based couplings occur at the regional scale between similar habitat types. Resource-based couplings occur at the local scale between different habitat types.},
author = {Gounand, Isabelle and Harvey, Eric and Little, Chelsea J. and Altermatt, Florian},
doi = {10.1016/j.tree.2017.10.006},
file = {:Users/alyssacirtwill/Documents/Papers/Gounand et al.{\_}2018{\_}Trends in Ecology and Evolution.pdf:pdf},
isbn = {1872-8383 (Electronic) 0169-5347 (Linking)},
issn = {01695347},
journal = {Trends in Ecology and Evolution},
keywords = {animal movement,landscape,meta-ecosystem,metacommunity,resource flow,subsidy},
number = {1},
pages = {36--46},
pmid = {29102408},
publisher = {Elsevier Ltd},
title = {{Meta-Ecosystems 2.0: Rooting the Theory into the Field}},
url = {http://dx.doi.org/10.1016/j.tree.2017.10.006},
volume = {33},
year = {2018}
}
@article{Kankaanpaa2018,
abstract = {Snow conditions are important drivers of the distribution and phenology of Arctic flora and fauna, but the extent and effects of local variation in snowmelt are still inadequately studied. We analyze snowmelt patterns within the Zackenberg valley in northeast Greenland. Drawing on landscape- level snowmelt dates and meteorological data from a central climate station, we model snowmelt trends during 1998–2014. We then use time-lapse photographs to examine consistency in spatiotemporal snowmelt patterns during 2006–2014. Finally, we use monitoring data on arthro- pods and plants for 1998–2014 to investigate how snowmelt date affects the phenology of Arctic organisms. Despite large interannual variation in snowmelt timing, we find consistency in the relative order of snowmelt among sites within the landscape. With a slight overall advancement in snowmelt during the study period, early melting locations have advanced more than late-melting ones. Individual organism groups differ greatly in how their phenology shifts with snowmelt, with much variance attributable to variation in life history and diet. Overall, we note that local variation in snowmelt patterns may drive important ecological processes, and that more attention should be paid to variability within landscapes. Areas optimal for a given taxon vary between years, thereby creating spatial structure in a seemingly uniform landscape.},
author = {Kankaanp{\"{a}}{\"{a}}, Tuomas and Skov, Kirstine and Abrego, Nerea and Lund, Magnus and Schmidt, Niels Martin and Roslin, Tomas},
doi = {10.1080/15230430.2017.1415624},
issn = {19384246},
journal = {Arctic, Antarctic, and Alpine Research},
keywords = {Snowmelt,climate change,high Arctic,phenological mismatch,spatiotemporal variability},
number = {1},
title = {{Spatiotemporal snowmelt patterns within a high Arctic landscape, with implications for flora and fauna}},
volume = {50},
year = {2018}
}
@phdthesis{Bakerthesis,
author = {Baker, Nick J.},
pages = {99},
school = {University of Canterbury},
title = {{A quantitative exploration of the meso-scale structure of ecological networks}},
year = {2015}
}
@article{Dobson2014,
abstract = {The reintroduction of wolves to Yellowstone has provided fascinating insights into the ways species interactions within food webs structure ecosystems. Recent controversies about whether wolves are responsible for all observed changes in prey and plant abundance suggest that we need many more such studies, as they throw considerable light on the forces that structure the parts of the universe that are of vital importance to humans.},
author = {Dobson, Andy P.},
doi = {10.1371/journal.pbio.1002025},
isbn = {1545-7885},
issn = {15457885},
journal = {PLoS Biology},
number = {12},
pages = {e1002025},
pmid = {25535737},
title = {{Yellowstone Wolves and the Forces That Structure Natural Systems}},
volume = {12},
year = {2014}
}
@article{Rodriguez-Rodriguez2017,
abstract = {Introduction: D-dimer assay, generally evaluated according to cutoff points calibrated for VTE exclusion, is used to estimate the individual risk of recurrence after a first idiopathic event of venous thromboembolism (VTE). Methods: Commercial D-dimer assays, evaluated according to predetermined cutoff levels for each assay, specific for age (lower in subjects {\textless}70 years) and gender (lower in males), were used in the recent DULCIS study. The present analysis compared the results obtained in the DULCIS with those that might have been had using the following different cutoff criteria: traditional cutoff for VTE exclusion, higher levels in subjects aged ≥60 years, or age multiplied by 10. Results: In young subjects, the DULCIS low cutoff levels resulted in half the recurrent events that would have occurred using the other criteria. In elderly patients, the DULCIS results were similar to those calculated for the two age-adjusted criteria. The adoption of traditional VTE exclusion criteria would have led to positive results in the large majority of elderly subjects, without a significant reduction in the rate of recurrent event. Conclusion: The results confirm the usefulness of the cutoff levels used in DULCIS.},
author = {Rodr{\'{i}}guez-Rodr{\'{i}}guez, Mar{\'{i}}a C. and Jordano, Pedro and Valido, Alfredo},
doi = {10.1002/ecy.1756},
isbn = {4955139574},
issn = {00129658},
journal = {Ecology},
keywords = {Canary Islands,Isoplexis canariensis,antagonist,bird pollination,female reproductive success,individual-based pollination networks,interaction strength,mating network,mutualist},
number = {5},
pages = {1266--1276},
pmid = {16040241},
title = {{Functional consequences of plant-animal interactions along the mutualism-antagonism gradient}},
volume = {98},
year = {2017}
}
@article{Arim2004,
abstract = {Intraguild predation (IGP), defined as killing and eating among potential competitors, seems to be a ubiquitous interaction, differing from competition or predation. In the present study we assess the frequency of IGP among 763 potential intraguild prey and 599 potential intraguild predators. Our results indicate that IGP is common in nature, reaching frequencies between 58.4 and 86.7{\%}. A null model suggests that IGP in different groups of predators and prey (i.e. carnivores, omnivores, herbivores, detritivores, or top and intermediate species) have different deviations from a chance expectation, indicating these attributes of species biology as main determinants of IGP persistence. We suggest that IGP satisfies two basic requirements to be considered as important to the trophic structuring of communities. First, its occurrence is not random, rather it is associated with well-defined attributes of species biology, and secondly, it is a widespread interaction.},
author = {Arim, Mat{\'{i}}as and Marquet, Pablo A.},
doi = {10.1111/j.1461-0248.2004.00613.x},
isbn = {1461-023X},
issn = {1461023X},
journal = {Ecology Letters},
keywords = {Community modules,Food webs,Intraguild predation,Null model,Omnivory,Species interaction},
number = {7},
pages = {557--564},
pmid = {221853400006},
title = {{Intraguild predation: A widespread interaction related to species biology}},
volume = {7},
year = {2004}
}
@article{Gravel2013,
abstract = {* Current global changes make it important to be able to predict which interactions will occur in the emerging ecosystems. Most of the current methods to infer the existence of interactions between two species require a good knowledge of their behaviour or a direct observation of interactions. In this paper, we overcome these limitations by developing a method, inspired from the niche model of food web structure, using the statistical relationship between predator and prey body size to infer the matrix of potential interactions among a pool of species. * The novelty of our approach is to infer, for any species of a given species pool, the three species-specific parameters of the niche model. The method applies to both local and metaweb scales. It allows one to evaluate the feeding interactions of a new species entering the community. * We find that this method gives robust predictions of the structure of food webs and that its efficiency is increased when the strength of the body–size relationship between predators and preys increases. * We finally illustrate the potential of the method to infer the metaweb structure of pelagic fishes of the Mediterranean sea under different global change scenarios.},
author = {Gravel, Dominique and Poisot, Timoth{\'{e}}e and Albouy, Camille and Velez, Laure and Mouillot, David},
doi = {10.1111/2041-210X.12103},
isbn = {2041-210X},
issn = {2041210X},
journal = {Methods in Ecology and Evolution},
keywords = {Body size,Food web,Metaweb,Niche model},
number = {11},
pages = {1083--1090},
title = {{Inferring food web structure from predator-prey body size relationships}},
volume = {4},
year = {2013}
}
@article{Gravel2016,
abstract = {There is a growing interest in using trait-based approaches to characterize the functional structure of animal communities. Quantitative methods have been derived mostly for plant ecology, but it is now common to characterize the functional composition of various systems such as soils, coral reefs, pelagic food webs or terrestrial vertebrate communities. With the ever-increasing availability of distribution and trait data, a quantitative method to represent the different roles of animals in a community promise to find generalities that will facilitate cross-system comparisons. There is, however, currently no theory relating the functional composition of food webs to their dynamics and properties. The intuitive interpretation that more functional diversity leads to higher resource exploitation and better ecosystem functioning was brought from plant ecology and does not apply readily to food webs. Here we appraise whether there are interpretable metrics to describe the functional composition of food webs that could foster a better understanding of their structure and functioning. We first distinguish the various roles that traits have on food web topology, resource extraction (bottom-up effects), trophic regulation (top-down effects), and the ability to keep energy and materials within the community. We then discuss positive effects of functional trait diversity on food webs, such as niche construction and bottom-up effects. We follow with a discussion on the negative effects of functional diversity, such as enhanced competition (both exploitation and apparent) and top-down control. Our review reveals that most of our current understanding of the impact of functional trait diversity on food web properties and functioning comes from an over-simplistic representation of network structure with well-defined levels. We, therefore, conclude with propositions for new research avenues for both theoreticians and empiricists.},
archivePrefix = {arXiv},
arxivId = {http://dx.doi.org/10.1098/rstb.2015.0268},
author = {Gravel, Dominique and Albouy, Camille and Thuiller, Wilfried},
doi = {10.1098/rstb.2015.0268},
eprint = {/dx.doi.org/10.1098/rstb.2015.0268},
file = {:Users/alyssacirtwill/Documents/Papers/Gravel, Albouy, Thuiller{\_}2016{\_}Philosophical Transactions of the Royal Society B Biological Sciences.pdf:pdf},
isbn = {0962-8436},
issn = {14712970},
journal = {Philosophical Transactions of the Royal Society B: Biological Sciences},
keywords = {Biodiversity,Ecological networks,Ecosystem functioning,Trait matching},
number = {1694},
pages = {20150268},
pmid = {27114571},
primaryClass = {http:},
title = {{The meaning of functional trait composition of food webs for ecosystem functioning}},
volume = {371},
year = {2016}
}
@article{Albouy2014,
abstract = {Climate change is inducing deep modifications in species geographic ranges worldwide. However, the consequences of such changes on community structure are still poorly understood, particularly the impacts on food-web properties. Here, we propose a new framework, coupling species distribution and trophic models, to predict climate change impacts on food-web structure across the Mediterranean Sea. Sea surface temperature was used to determine the fish climate niches and their future distributions. Body size was used to infer trophic interactions between fish species. Our projections reveal that 54 fish species of 256 endemic and native species included in our analysis would disappear by 2080-2099 from the Mediterranean continental shelf. The number of feeding links between fish species would decrease on 73.4{\%} of the continental shelf. However, the connectance of the overall fish web would increase on average, from 0.26 to 0.29, mainly due to a differential loss rate of feeding links and species richness. This result masks a systematic decrease in predator generality, estimated here as the number of prey species, from 30.0 to 25.4. Therefore, our study highlights large-scale impacts of climate change on marine food-web structure with potential deep consequences on ecosystem functioning. However, these impacts will likely be highly heterogeneous in space, challenging our current understanding of climate change impact on local marine ecosystems.},
author = {Albouy, Camille and Velez, Laure and Coll, Marta and Colloca, Francesco and {Le Loc'h}, Fran{\c{c}}ois and Mouillot, David and Gravel, Dominique},
doi = {10.1111/gcb.12467},
isbn = {1365-2486},
issn = {13541013},
journal = {Global Change Biology},
keywords = {Climate change,Connectance,Fish body size,Food-webs,Generality,Mediterranean Sea,Metaweb,Niche model,Vulnerability},
number = {3},
pages = {730--741},
pmid = {24214576},
title = {{From projected species distribution to food-web structure under climate change}},
volume = {20},
year = {2014}
}
@article{Yodzis1992,
author = {Weiss, Bernard and Environmental, Source and Perspectives, Health and Jun, No and Park, College and Williams, Gary M},
doi = {10.1289/ehp.lll82},
journal = {The American Naturalist},
number = {6},
pages = {27--29},
title = {{Brogan {\&} Partners Food Additives and Hyperactivity Linked references are available on JSTOR for this article : All use subject to JSTOR Terms and Conditions Correspondence Carcinogenicity of Aspartame : Soffritti Responds}},
volume = {116},
year = {2016}
}
@article{Poisot2016a,
abstract = {The increased availability of both open ecological data, and software to interact with it, allows the fast collection and integration of information at all spatial and taxonomic scales. This offers the opportunity to address macroecological questions in a cost-effective way. In this contribution, we illustrate this approach by forecasting the structure of a stream food web at the global scale. In so doing, we highlight the most salient issues needing to be addressed before this approach can be used with a high degree of confidence.},
author = {Poisot, Timoth{\'{e}}e and Gravel, Dominique and Leroux, Shawn and Wood, Spencer A. and Fortin, Marie Jos{\'{e}}e and Baiser, Benjamin and Cirtwill, Alyssa R. and Ara{\'{u}}jo, Miguel B. and Stouffer, Daniel B.},
doi = {10.1111/ecog.01941},
file = {:Users/alyssacirtwill/Documents/Papers/Poisot et al.{\_}2016{\_}Ecography.pdf:pdf},
isbn = {1600-0587},
issn = {16000587},
journal = {Ecography},
number = {4},
pages = {402--408},
pmid = {20348093},
title = {{Synthetic datasets and community tools for the rapid testing of ecological hypotheses}},
volume = {39},
year = {2016}
}
@article{Cirtwill2015b,
abstract = {Variations in levels of parasitism among individuals in a population of hosts underpin the importance of parasites as an evolutionary or ecological force. Factors influencing parasite richness (number of parasite species) and load (abundance and biomass) at the individual host level ultimately form the basis of parasite infection patterns. In fish, diet range (number of prey taxa consumed) and prey selectivity (proportion of a particular prey taxon in the diet) have been shown to influence parasite infection levels. However, fish diet is most often characterized at the species or fish population level, thus ignoring variation among conspecific individuals and its potential effects on infection patterns among individuals. Here, we examined parasite infections and stomach contents of New Zealand freshwater fish at the individual level. We tested for potential links between the richness, abundance and biomass of helminth parasites and the diet range and prey selectivity of individual fish hosts. There was no obvious link between individual fish host diet and helminth infection levels. Our results were consistent across multiple fish host and parasite species and contrast with those of earlier studies in which fish diet and parasite infection were linked, hinting at a true disconnect between host diet and measures of parasite infections in our study systems. This absence of relationship between host diet and infection levels may be due to the relatively low richness of freshwater helminth parasites in New Zealand and high host–parasite specificity.},
author = {CIRTWILL, ALYSSA R. and STOUFFER, DANIEL B. and POULIN, ROBERT and LAGRUE, CL{\'{E}}MENT},
doi = {10.1017/s003118201500150x},
isbn = {0031182015001},
issn = {0031-1820},
journal = {Parasitology},
keywords = {fish diet,helminth parasites,individual host,infection levels,transmission mode},
number = {01},
pages = {75--86},
pmid = {16185695},
title = {{Are parasite richness and abundance linked to prey species richness and individual feeding preferences in fish hosts?}},
volume = {143},
year = {2015}
}
@article{Cirtwill2016a,
abstract = {Aim MacArthur and Wilson's original formulation of the theory of island biogeography (TIB) included the corollary hypothesis that species richness might affect immigration and extinction rates. Building on this, other researchers have suggested additional top-down and bottom-up effects. We compare these hypotheses to identify the strongest candidates for inclusion in a ‘trophic TIB'. Location Six mangrove islands in the Florida Keys, USA Methods We studied a classic island biogeography time series featuring lists of species observed on six mangrove islands during roughly 16 censuses each across 700 days. We first used this time series to determine the number of opportunities for species to immigrate to an island for the first time (n = 18,420), to go locally extinct (n = 1943) or to re-immigrate to an island after having previously gone extinct (n = 1813). We then leveraged information on the predators and prey of those species to estimate the potential for top-down and bottom-up interactions during each census period. Finally, we constructed statistical models to test for species richness, top-down, and bottom-up effects on per-species immigration and extinction probabilities and validated them by comparing each model with a similar model based on the classic TIB. Results We found that models including bottom-up effects gave the greatest improvement over the classic TIB models. Extinction probability in particular decreased sharply for species with both basal resources and animal prey available. Species richness and top-down effects had far weaker impacts on per-species probabilities of immigration and extinction. Main conclusions Our findings suggest that incorporating information on the trophic structure of island communities – particularly the species-specific availability of resources – can substantially alter predictions of extinction probabilities. Immigration probability, on the contrary, appeared largely stochastic. Incorporating trophic information into predictions of extinction rates therefore represents the most promising and best-supported way to extend the TIB.},
author = {Cirtwill, Alyssa R. and Stouffer, Daniel B.},
doi = {10.1111/geb.12332},
file = {:Users/alyssacirtwill/Documents/Papers/Cirtwill, Stouffer{\_}2016{\_}Global Ecology and Biogeography(2).pdf:pdf},
isbn = {1466-8238},
issn = {14668238},
journal = {Global Ecology and Biogeography},
keywords = {Bottom-up effects,community assembly,food web,predator–prey interactions,species richness,theory of island biogeography,top-down effects},
number = {7},
pages = {900--911},
title = {{Knowledge of predator–prey interactions improves predictions of immigration and extinction in island biogeography}},
volume = {25},
year = {2016}
}
@article{VizentinBugoni2014,
abstract = {Understanding the relative importance of multiple processes on structuring species interactions within communities is one of the major challenges in ecology. Here, we evaluated the relative importance of species abundance and forbidden links in structuring a hummingbird-plant interaction network from the Atlantic rainforest in Brazil. Our results show that models incorporating phenological overlapping and morphological matches were more accurate in predicting the observed interactions than models considering species abundance. This means that forbidden links, by imposing constraints on species interactions, play a greater role than species abundance in structuring the ecological network. We also show that using the frequency of interaction as a proxy for species abundance and network metrics to describe the detailed network structure might lead to biased conclusions regarding mechanisms generating network structure. Together, our findings suggest that species abundance can be a less important driver of species interactions in communities than previously thought.},
archivePrefix = {arXiv},
arxivId = {arXiv:gr-qc/9809069v1},
author = {Vizentin-Bugoni, Jeferson and Maruyama, Pietro Kiyoshi and Sazima, Marlies},
doi = {10.1098/rspb.2013.2397},
eprint = {9809069v1},
isbn = {9780874216561},
issn = {14712954},
journal = {Proceedings of the Royal Society B: Biological Sciences},
keywords = {Atlantic rainforest,Morphological match,Neutral-based processes,Phenological overlap,Plant-pollinator networks},
number = {1780},
pages = {20132397--20132397},
pmid = {24552835},
primaryClass = {arXiv:gr-qc},
title = {{Processes entangling interactions in communities: Forbidden links are more important than abundance in a hummingbird-plant network}},
url = {http://rspb.royalsocietypublishing.org/cgi/doi/10.1098/rspb.2013.2397},
volume = {281},
year = {2014}
}
@article{Simanonok2014a,
abstract = {Ecologists have taken two distinct approaches in studying the distribution and diversity of communities: a species-centric focus and an interaction-network based approach. A current frontier in community-level studies is the integration of these perspectives by investigating both simultaneously; one method for achieving this is evaluating the relative contributions of species turnover and host switching towards interaction turnover (i.e., the dissimilarity in interactions between two networks). We performed observations of plant-pollinator interactions to investigate (1) patterns in interaction turnover across spatial, temporal, and environmental gradients and (2) the relative contribution of pollinator species turnover, floral turnover, simultaneous pollinator {\&} floral turnover, and host switching towards interaction turnover. Field work was conducted on the Beartooth Plateau, an alpine ecosystem in Montana and Wyoming, with weekly observations of plant-pollinator interactions across one growing season. Interaction turnover increased through time, with magnitudes consistently greater than 80{\%}, even at time intervals as short as one week. Floral species turnover (41{\%}) and simultaneous floral and pollinator species turnover (36{\%}) accounted for almost all interaction turnover while host switching accounted for only 5{\%}. Interaction turnover also significantly increased with spatial and elevational distance, albeit with lesser magnitudes than with temporal distance. The marginal spatial pattern was present for only some taxa (Bombus spp. and solitary bee species), potentially indicating variable habitat use by pollinators across the landscape. Weak environmental trends may be a consequence of unmeasured environmental variables, yet our finding that environmental gradients structure plant-pollinator partitions had not previously been tested with empirical data. Our observations suggest that host switching does not readily occur at the scales of alpine flowering phenology (i.e. {\~{}}1 week); however, whether lack of host switching is indicative of inflexible pollinator foraging, or, more likely, a lack of opportunity or necessity to switch hosts, requires further investigation.},
author = {Simanonok, Michael P. and Burkle, Laura A.},
doi = {10.1890/ES14-00323.1},
isbn = {2150-8925},
issn = {21508925},
journal = {Ecosphere},
keywords = {Beta diversity,Elevation,Spatiotemporal},
number = {11},
pages = {1--17},
title = {{Partitioning interaction turnover among alpine pollination networks: Spatial, temporal, and environmental patterns}},
url = {http://doi.wiley.com/10.1890/ES14-00323.1},
volume = {5},
year = {2014}
}
@article{Carnicer2009a,
abstract = {Ecological network patterns are influenced by diverse processes that operate at different temporal rates. Here we analyzed whether the coupled effect of local abundance variation, seasonally phenotypic plastic responses, and species evolutionary adaptations might act in concert to shape network patterns. We studied the temporal variation in three interaction properties of bird species (number of interactions per species, interaction strength, and interaction asymmetry) in a temporal sequence of 28 plant-frugivore interaction networks spanning two years in a Mediterranean shrubland community. Three main hypotheses dealing with the temporal variation of network properties were tested, examining the effects of abundance, switching behavior between alternative food resources, and morphological traits in determining consumer interaction patterns. Our results demonstrate that temporal variation in consumer interaction patterns is explained by short-term variation in resource and bird abundances and seasonal dietary switches between alternative resources (fleshy fruits and insects). Moreover, differences in beak morphology are associated with differences in switching behavior between resources, suggesting an important role of foraging adaptations in determining network patterns. We argue that beak shape adaptations might determine generalist and specialist feeding behaviors and thus the positions of consumer species within the network. Finally, we provide a preliminary framework to interpret phylogenetic signal in plant-animal networks. Indeed, we show that the strength of the phylogenetic signal in networks depends on the relative importance of abundance, behavioral, and morphological variables. We show that these variables strongly differ in their phylogenetic signal. Consequently, we suggest that moderate and significant phylogenetic effects should be commonly observed in networks of species interactions.},
author = {Carnicer, Jofre and Jordano, Pedro and Melian, Carlos J.},
doi = {10.1890/07-1939.1},
issn = {00129658},
journal = {Ecology},
keywords = {Abundance,Asymmetry,Frugivorous birds,Generalist vs. specialist,Interaction network,Mediterranean shrubland,Morphological traits,Phytogeny,Resource pulse,Switching behavior},
number = {7},
pages = {1958--1970},
title = {{The temporal dynamics of resource use by frugivorous birds: a network approach}},
url = {https://doi.org/10.1890/07-1939.1},
volume = {90},
year = {2009}
}
@article{Burkle2013a,
abstract = {Using historic data sets, we quantified the degree to which global change over 120 years disrupted plant-pollinator interactions in a temperate forest understory community in Illinois, USA. We found degradation of interaction network structure and function and extirpation of 50{\%} of bee species. Network changes can be attributed to shifts in forb and bee phenologies resulting in temporal mismatches, nonrandom species extinctions, and loss of spatial co-occurrences between extant species in modified landscapes. Quantity and quality of pollination services have declined through time. The historic network showed flexibility in response to disturbance; however, our data suggest that networks will be less resilient to future changes.},
archivePrefix = {arXiv},
arxivId = {http://links.jstor.org/sici?sici=0036-8075{\%}2819780324{\%}293{\%}3A199{\%}3A4335{\%}3C1302{\%}3ADITRFA{\%}3E2.0.CO{\%}3B2-2},
author = {Burkle, Laura A. and Marlin, John C. and Knight, Tiffany M.},
doi = {10.1126/science.1232728},
eprint = {/links.jstor.org/sici?sici=0036-8075{\%}2819780324{\%}293{\%}3A199{\%}3A4335{\%}3C1302{\%}3ADITRFA{\%}3E2.0.CO{\%}3B2-2},
file = {:Users/alyssacirtwill/Documents/Papers/Burkle, Marlin, Knight{\_}2013{\_}Science.pdf:pdf},
isbn = {0036-8075},
issn = {10959203},
journal = {Science},
number = {6127},
pages = {1611--1615},
pmid = {23449999},
primaryClass = {http:},
title = {{Plant-pollinator interactions over 120 years: Loss of species, co-occurrence, and function}},
url = {http://www.sciencemag.org/cgi/doi/10.1126/science.1232728},
volume = {340},
year = {2013}
}
@article{Fort2016,
abstract = {A frequent observation in plant–animal mutualistic networks is that abundant species tend to be more generalised, interacting with a broader range of interaction partners than rare species. Uncovering the causal relationship between abundance and generalisation has been hindered by a chicken-and-egg dilemma: is generalisation a by-product of being abundant, or does high abun- dance result from generalisation? Here, we analyse a database of plant–pollinator and plant–seed disperser networks, and provide strong evidence that the causal link between abundance and generalisation is uni-directional. Specifically, species appear to be generalists because they are more abundant, but the converse, that is that species become more abundant because they are generalists, is not supported by our analysis. Furthermore, null model analyses suggest that abun- dant species interact with many other species simply because they are more likely to encounter potential interaction partners.},
author = {Fort, Hugo and V{\'{a}}zquez, Diego P. and Lan, Boon Leong},
doi = {10.1111/ele.12535},
isbn = {1461-0248},
issn = {14610248},
journal = {Ecology Letters},
keywords = {Causality,Generalisation,Mutualistic networks,Plant-animal interactions,Pollination,Seed dispersal,Specialisation},
number = {1},
pages = {4--11},
pmid = {26498731},
title = {{Abundance and generalisation in mutualistic networks: Solving the chicken-and-egg dilemma}},
volume = {19},
year = {2016}
}
@article{VizentinBugoni2014,
abstract = {Understanding the relative importance of multiple processes on structuring species interactions within communities is one of the major challenges in ecology. Here, we evaluated the relative importance of species abundance and forbidden links in structuring a hummingbird-plant interaction network from the Atlantic rainforest in Brazil. Our results show that models incorporating phenological overlapping and morphological matches were more accurate in predicting the observed interactions than models considering species abundance. This means that forbidden links, by imposing constraints on species interactions, play a greater role than species abundance in structuring the ecological network. We also show that using the frequency of interaction as a proxy for species abundance and network metrics to describe the detailed network structure might lead to biased conclusions regarding mechanisms generating network structure. Together, our findings suggest that species abundance can be a less important driver of species interactions in communities than previously thought.},
archivePrefix = {arXiv},
arxivId = {arXiv:gr-qc/9809069v1},
author = {Vizentin-Bugoni, Jeferson and Maruyama, Pietro Kiyoshi and Sazima, Marlies},
doi = {10.1098/rspb.2013.2397},
eprint = {9809069v1},
isbn = {9780874216561},
issn = {14712954},
journal = {Proceedings of the Royal Society B: Biological Sciences},
keywords = {Atlantic rainforest,Morphological match,Neutral-based processes,Phenological overlap,Plant-pollinator networks},
number = {1780},
pages = {20132397--20132397},
pmid = {24552835},
primaryClass = {arXiv:gr-qc},
title = {{Processes entangling interactions in communities: Forbidden links are more important than abundance in a hummingbird-plant network}},
url = {http://rspb.royalsocietypublishing.org/cgi/doi/10.1098/rspb.2013.2397},
volume = {281},
year = {2014}
}
@article{CaraDonna2017,
abstract = {The 6M BaIrO(3) with the distorted hexagonal BaTiO(3) structure was synthesized by high-pressure sintering. Through Rietveld refinement of the powder X-ray diffraction data, the lattice parameters of a = 5.7459(1) A, b = 9.9289(2) A, c = 14.3433(2) A, and beta = 91.340(1) degrees were obtained. In the Ir(2)O(9) dioctahedron, the average Ir-O distance and direct Ir-Ir distance were equal to 2.067(19) and 2.719(1) A, respectively. The temperature dependence of electrical resistivity shows that the 6M BaIrO(3) is a new metallic iridate. It is an abnormal metal, being deviated from the Fermi liquid behavior, following a linear relationship of rho versus T below 20 K. Both magnetic susceptibility and specific heat data indicate that it is an exchange-enhanced Pauli paramagnet, because of the electron-electron correlation effect.},
author = {CaraDonna, Paul J. and Petry, William K. and Brennan, Ross M. and Cunningham, James L. and Bronstein, Judith L. and Waser, Nickolas M. and Sanders, Nathan J.},
doi = {10.1111/ele.12740},
isbn = {0020-1669},
issn = {14610248},
journal = {Ecology Letters},
keywords = {Adaptive foraging,beta-diversity,community composition,food webs,interaction turnover,mutualism,networks,null models,optimal foraging theory,phenology},
number = {3},
pages = {385--394},
pmid = {19366187},
title = {{Interaction rewiring and the rapid turnover of plant–pollinator networks}},
volume = {20},
year = {2017}
}
@article{Carnicer2008,
abstract = {Theory shows that the presence of behavioural switching between alternative resources can contribute to coexistence when competitors differ in trophic-related traits. In addition, switching can generate disruptive selection on such traits in a low-diversity community, increasing the number of species. Both of these processes should produce communities in which species differ in their values of the trophic trait, and display corresponding differences in the time-course of their switching from one resource to another. Here we present evidence for widespread switching behaviour for a diverse Mediterranean scrubland bird community. We show that species differ in a beak character related to their relative use of insect and fruit resource channels, and that the timing of switching is correlated with the relative use of resources. These patterns are consistent with theoretical predictions, suggesting a possible role of switching behaviour in promoting avian coexistence and diversification.},
author = {Carnicer, Jofre and Abrams, Peter A. and Jordano, Pedro},
doi = {10.1111/j.1461-0248.2008.01195.x},
isbn = {1461-0248},
issn = {1461023X},
journal = {Ecology Letters},
keywords = {Adaptation,Birds,Coexistence,Diversification,Foraging theory,Fruits,Insects,Resource seasonality,Species richness,Switching behaviour},
number = {8},
pages = {802--808},
pmid = {18445033},
title = {{Switching behavior, coexistence and diversification: Comparing empirical community-wide evidence with theoretical predictions}},
volume = {11},
year = {2008}
}
@article{Carnicer2009,
abstract = {Ecological network patterns are influenced by diverse processes that operate at different temporal rates. Here we analyzed whether the coupled effect of local abundance variation, seasonally phenotypic plastic responses, and species evolutionary adaptations might act in concert to shape network patterns. We studied the temporal variation in three interaction properties of bird species (number of interactions per species, interaction strength, and interaction asymmetry) in a temporal sequence of 28 plant-frugivore interaction networks spanning two years in a Mediterranean shrubland community. Three main hypotheses dealing with the temporal variation of network properties were tested, examining the effects of abundance, switching behavior between alternative food resources, and morphological traits in determining consumer interaction patterns. Our results demonstrate that temporal variation in consumer interaction patterns is explained by short-term variation in resource and bird abundances and seasonal dietary switches between alternative resources (fleshy fruits and insects). Moreover, differences in beak morphology are associated with differences in switching behavior between resources, suggesting an important role of foraging adaptations in determining network patterns. We argue that beak shape adaptations might determine generalist and specialist feeding behaviors and thus the positions of consumer species within the network. Finally, we provide a preliminary framework to interpret phylogenetic signal in plant-animal networks. Indeed, we show that the strength of the phylogenetic signal in networks depends on the relative importance of abundance, behavioral, and morphological variables. We show that these variables strongly differ in their phylogenetic signal. Consequently, we suggest that moderate and significant phylogenetic effects should be commonly observed in networks of species interactions.},
author = {Carnicer, Jofre and Jordano, Pedro and Melian, Carlos J.},
doi = {10.1890/07-1939.1},
isbn = {0012-9658},
issn = {00129658},
journal = {Ecology},
keywords = {Abundance,Asymmetry,Frugivorous birds,Generalist vs. specialist,Interaction network,Mediterranean shrubland,Morphological traits,Phytogeny,Resource pulse,Switching behavior},
number = {7},
pages = {1958--1970},
pmid = {19694143},
title = {{The temporal dynamics of resource use by frugivorous birds: a network approach}},
volume = {90},
year = {2009}
}
@article{Burkle2016,
abstract = {The effects of climate change on species interactions are poorly understood. Investigating the mechanisms by which species interactions may shift under altered environmental conditions will help form a more predictive understanding of such shifts. In particular, components of climate change have the potential to strongly influence floral volatile organic compounds (VOCs) and, in turn, plant-pollinator interactions. In this study, we experimentally manipulated drought and herbivory for four forb species to determine effects of these treatments and their interactions on (1) visual plant traits traditionally associated with pollinator attraction, (2) floral VOCs, and (3) the visitation rates and community composition of pollinators. For all forbs tested, experimental drought universally reduced flower size and floral display, but there were species-specific effects of drought on volatile emissions per flower, the composition of compounds produced, and subsequent pollinator visitation rates. Moreover, the community of pollinating visitors was influenced by drought across forb species (i.e., some pollinator species were deterred by drought while others were attracted). Together, these results indicate that VOCs may provide more nuanced information to potential floral visitors and may be relatively more important than visual traits for pollinator attraction, particularly under shifting environmental conditions. This article is protected by copyright. All rights reserved.},
author = {Burkle, Laura A. and Runyon, Justin B.},
doi = {10.1111/gcb.13149},
isbn = {0098-0331},
issn = {13652486},
journal = {Global Change Biology},
keywords = {Campanula rotundifolia,Climate change,Floral display,Floral scent,Heterotheca villosa,Phacelia hastata,Plant-pollinator interactions,Pollinator community,Potentilla recta,Volatile organic compounds},
number = {4},
pages = {1644--1654},
pmid = {26546275},
title = {{Drought and leaf herbivory influence floral volatiles and pollinator attraction}},
volume = {22},
year = {2016}
}
@article{Poisot2012a,
abstract = {1. Ecological specialization is a unifying concept in the biological sciences. While there are reliable ways to characterize specificity at individual and community levels, the evaluation of population and species-level measures is lacking. There is a need for such assessments given that populations and species are the relevant scales for most ecological and evolutionary processes. 2. Using examples of simulated and empirical data sets of bipartite networks representing a continuum of biological interactions, we evaluate six indices of specificity in terms of their robustness to incomplete sampling and information they extract from data. 3. Robustness differed between the measures and in their ability to differentiate specialists and generalists along a full continuum. On the empirical data sets, indices were less separated by their informativity than on the simulated data sets, which may be due to the heterogeneity of the former. 4. Based on these different evaluations for species-level (or population-level) specificity, we recommend the use of Resource range when no quantitative data are available and Paired Difference Index otherwise. These results will assist both applied and fundamental researchers in the characterization and interpretation of species specificity.},
author = {Poisot, Timoth{\'{e}}e and Canard, Elsa and Mouquet, Nicolas and Hochberg, Michael E.},
doi = {10.1111/j.2041-210X.2011.00174.x},
file = {:Users/alyssacirtwill/Documents/Papers/Poisot et al.{\_}2012{\_}Methods in Ecology and Evolution.pdf:pdf},
isbn = {2041-210X},
issn = {2041210X},
journal = {Methods in Ecology and Evolution},
keywords = {Bipartite networks,Methodology,Sampling,Specialization,Specificity},
number = {3},
pages = {537--544},
title = {{A comparative study of ecological specialization estimators}},
volume = {3},
year = {2012}
}
@article{Petanidou2008,
abstract = {We analysed the dynamics of a plant-pollinator interaction network of a scrub community surveyed over four consecutive years. Species composition within the annual networks showed high temporal variation. Temporal dynamics were also evident in the topology of the network, as interactions among plants and pollinators did not remain constant through time. This change involved both the number and the identity of interacting partners. Strikingly, few species and interactions were consistently present in all four annual plant-pollinator networks (53{\%} of the plant species, 21{\%} of the pollinator species and 4.9{\%} of the interactions). The high turnover in species-to-species interactions was mainly the effect of species turnover (c. 70{\%} in pairwise comparisons among years), and less the effect of species flexibility to interact with new partners (c. 30{\%}). We conclude that specialization in plant-pollinator interactions might be highly overestimated when measured over short periods of time. This is because many plant or pollinator species appear as specialists in 1 year, but tend to be generalists or to interact with different partner species when observed in other years. The high temporal plasticity in species composition and interaction identity coupled with the low variation in network structure properties (e.g. degree centralization, connectance, nestedness, average distance and network diameter) imply (i) that tight and specialized coevolution might not be as important as previously suggested and (ii) that plant-pollinator interaction networks might be less prone to detrimental effects of disturbance than previously thought. We suggest that this may be due to the opportunistic nature of plant and animal species regarding the available partner resources they depend upon at any particular time.},
author = {Petanidou, Theodora and Kallimanis, Athanasios S. and Tzanopoulos, Joseph and Sgardelis, Stefanos P. and Pantis, John D.},
doi = {10.1111/j.1461-0248.2008.01170.x},
isbn = {1461-023X},
issn = {1461023X},
journal = {Ecology Letters},
keywords = {Apparent vs. real specialization,Coevolution,Ecological networks,Food web structure,Mediterranean scrub,Nestedness analysis,Network analysis,Sampling effort},
number = {6},
pages = {564--575},
pmid = {18363716},
title = {{Long-term observation of a pollination network: Fluctuation in species and interactions, relative invariance of network structure and implications for estimates of specialization}},
volume = {11},
year = {2008}
}
@article{CuartasHernandez2015,
abstract = {{\textless}p{\textgreater}Understanding the factors determining the spatial and temporal variation of ecological networks is fundamental to the knowledge of their dynamics and functioning. In this study, we evaluate the effect of elevation and time on the structure of plant-flower-visitor networks in a Colombian mountain forest. We examine the level of generalization of plant and animal species and the identity of interactions in 44 bipartite matrices obtained from eight altitudinal levels, from 2200 to 2900 m during eight consecutive months. The contribution of altitude and time to the overall variation in the number of plant ({\textless}italic{\textgreater}P{\textless}/italic{\textgreater}) and pollinator ({\textless}italic{\textgreater}A{\textless}/italic{\textgreater}) species, network size ({\textless}italic{\textgreater}M{\textless}/italic{\textgreater}), number of interactions ({\textless}italic{\textgreater}I{\textless}/italic{\textgreater}), connectance ({\textless}italic{\textgreater}C{\textless}/italic{\textgreater}), and nestedness was evaluated. In general, networks were small, showed high connectance values and non-nested patterns of organization. Variation in {\textless}italic{\textgreater}P{\textless}/italic{\textgreater}, {\textless}italic{\textgreater}M{\textless}/italic{\textgreater}, {\textless}italic{\textgreater}I{\textless}/italic{\textgreater} and {\textless}italic{\textgreater}C{\textless}/italic{\textgreater} was better accounted by time than elevation, seemingly related to temporal variation in precipitation. Most plant and insect species were specialists and the identity of links showed a high turnover over months and at every 100 m elevation. The partition of the whole system into smaller network units allowed us to detect small-scale patterns of interaction that contrasted with patterns commonly described in cumulative networks. The specialized but erratic pattern of network organization observed in this tropical mountain suggests that high connectance coupled with opportunistic attachment may confer robustness to plant-flower-visitor networks occurring at small spatial and temporal units.{\textless}/p{\textgreater}},
author = {Cuartas-Hern{\'{a}}ndez, Sandra and Medel, Rodrigo},
doi = {10.1371/journal.pone.0141804},
isbn = {978-131-001-4},
issn = {19326203},
journal = {PLoS ONE},
number = {10},
pages = {e0141804},
title = {{Topology of plant - Flower-visitor networks in a tropical mountain forest: Insights on the role of altitudinal and temporal variation}},
volume = {10},
year = {2015}
}
@article{Burkle2015,
abstract = {Wildfires influence many temperate terrestrial ecosystems worldwide.$\backslash$nHistorical environmental heterogeneity created by wildfires has been$\backslash$naltered by human activities and will be impacted by future climate$\backslash$nchange. Our ability to predict the impact of wildfire-created$\backslash$nheterogeneity on biodiversity is limited because few studies have$\backslash$ninvestigated variation in community composition (beta-diversity) in$\backslash$nresponse to fire. Wildfires may influence beta-diversity through several$\backslash$necological mechanisms. First, highseverity fires may decrease$\backslash$nbeta-diversity by homogenizing species composition when they create$\backslash$nlandscapes dominated by disturbance-tolerant or rapidly colonizing$\backslash$nspecies. In contrast, mixed-severity fires may increase beta-diversity$\backslash$nby creating mosaic landscapes containing habitats that support species$\backslash$nwith differing environmental tolerances and dispersal traits. Moreover,$\backslash$nthe effects of fire severity on betadiversity may change depending on$\backslash$nsite conditions. Disturbance is hypothesized to increase local species$\backslash$nrichness at higher productivity and decrease local species richness at$\backslash$nlower productivity, a process that can have important, but largely$\backslash$nunexamined, consequences on beta-diversity in fire-prone ecosystems. We$\backslash$ntested these hypotheses by comparing patterns of beta-diversity and$\backslash$nspecies richness across 162 plant communities in three sites that span a$\backslash$nlarge-scale gradient in climate and productivity in the Northern Rockies$\backslash$nof Montana. Within each site, we used spatially explicit fire-severity$\backslash$ndata to stratify sampling across unburned forests and forests burned$\backslash$nwith mixed-and high-severity wildfires. We found that betadiversity$\backslash$n(community dispersion) of forbs was higher in mixed-severity compared to$\backslash$nhigh-severity fire, regardless of productivity. Counter to our$\backslash$npredictions, local species richness of forbs was higher in burned$\backslash$nlandscapes compared to unburned landscapes at the low-productivity site,$\backslash$nbut lower in burned landscapes at the high-productivity site. This$\backslash$npattern may be explained by rapid regeneration of woody plants after$\backslash$nfire in high-productivity forests. Moreover, forbs and woody plants had$\backslash$ndisproportionately higher overall species richness in mixed-severity$\backslash$nfire compared to high-severity fire, but only at the low-productivity$\backslash$nsite. These patterns suggest that mixed-severity fires promote higher$\backslash$nlandscape-level biodiversity in lowproductivity sites by increasing$\backslash$nspecies turnover across landscapes with a diverse mosaic of habitats.$\backslash$nOur study illustrates the importance of understanding the mechanisms by$\backslash$nwhich patterns of wildfire severity interact with environmental$\backslash$ngradients to influence patterns of biodiversity across spatial scales.},
author = {Burkle, Laura A. and Myers, Jonathan A. and Belote, R. Travis and Peters, D. P.C.},
doi = {10.1890/ES15-00438.1},
isbn = {2150-8925},
issn = {21508925},
journal = {Ecosphere},
keywords = {Beta-diversity,Community assembly,Disturbance severity,Fire management,Homogenization,Landscape ecology,Mixed-severity wildfire,Net primary productivity,Northern rockies ecoregion,Plant community composition,Restoration ecology,Spatial scale},
number = {10},
pages = {535},
title = {{Wildfire disturbance and productivity as drivers of plant species diversity across spatial scales}},
url = {http://doi.wiley.com/10.1890/ES15-00438.1},
volume = {6},
year = {2015}
}
@article{Simanonok2014,
abstract = {Ecologists have taken two distinct approaches in studying the distribution and diversity of communities: a species-centric focus and an interaction-network based approach. A current frontier in community-level studies is the integration of these perspectives by investigating both simultaneously; one method for achieving this is evaluating the relative contributions of species turnover and host switching towards interaction turnover (i.e., the dissimilarity in interactions between two networks). We performed observations of plant-pollinator interactions to investigate (1) patterns in interaction turnover across spatial, temporal, and environmental gradients and (2) the relative contribution of pollinator species turnover, floral turnover, simultaneous pollinator {\&} floral turnover, and host switching towards interaction turnover. Field work was conducted on the Beartooth Plateau, an alpine ecosystem in Montana and Wyoming, with weekly observations of plant-pollinator interactions across one growing season. Interaction turnover increased through time, with magnitudes consistently greater than 80{\%}, even at time intervals as short as one week. Floral species turnover (41{\%}) and simultaneous floral and pollinator species turnover (36{\%}) accounted for almost all interaction turnover while host switching accounted for only 5{\%}. Interaction turnover also significantly increased with spatial and elevational distance, albeit with lesser magnitudes than with temporal distance. The marginal spatial pattern was present for only some taxa (Bombus spp. and solitary bee species), potentially indicating variable habitat use by pollinators across the landscape. Weak environmental trends may be a consequence of unmeasured environmental variables, yet our finding that environmental gradients structure plant-pollinator partitions had not previously been tested with empirical data. Our observations suggest that host switching does not readily occur at the scales of alpine flowering phenology (i.e. {\~{}}1 week); however, whether lack of host switching is indicative of inflexible pollinator foraging, or, more likely, a lack of opportunity or necessity to switch hosts, requires further investigation.},
author = {Simanonok, Michael P. and Burkle, Laura A.},
doi = {10.1890/ES14-00323.1},
isbn = {2150-8925},
issn = {21508925},
journal = {Ecosphere},
keywords = {Beta diversity,Elevation,Spatiotemporal},
number = {11},
pages = {art149--art149},
title = {{Partitioning interaction turnover among alpine pollination networks: Spatial, temporal, and environmental patterns}},
url = {http://doi.wiley.com/10.1890/ES14-00323.1},
volume = {5},
year = {2014}
}
@article{Ponisio2017,
abstract = {Species and interactions are being lost at alarming rates and it is imperative to understand how communities assemble if we have to prevent their collapse and restore lost interactions. Using an 8-year dataset comprising nearly 20 000 pollinator visitation records, we explore the assembly of plant-pollinator communities at native plant restoration sites in an agricultural landscape. We find that species occupy highly dynamic network positions through time, causing the assembly process to be punctuated by major network reorganisations. The most persistent pollinator species are also the most variable in their network positions, contrary to what preferential attachment the most widely studied theory of ecological network assembly - predicts. Instead, we suggest assembly occurs via an opportunistic attachment process. Our results contribute to our understanding of how communities assembly and how species interactions change through time while helping to inform efforts to reassemble robust communities.},
author = {Ponisio, Lauren C. and Gaiarsa, Marilia P. and Kremen, Claire},
doi = {10.1111/ele.12821},
isbn = {1461-0248},
issn = {14610248},
journal = {Ecology Letters},
keywords = {Change points,community assembly,modularity,mutualism,nestedness,preferential attachment,restoration,robustness},
number = {10},
pages = {1261--1272},
title = {{Opportunistic attachment assembles plant–pollinator networks}},
volume = {20},
year = {2017}
}
@article{Olesen2008a,
abstract = {Despite a strong current interest in ecological networks, the bulk of studies are static descriptions of the structure of networks, and very few analyze their temporal dynamics. Yet, understanding network dynamics is important in order to relate network patterns to ecological processes. We studied the day-to-day dynamics of an arctic pollination interaction network over two consecutive seasons. First, we found that new species entering the network tend to interact with already well-connected species, although there are deviations from this trend due, for example, to morphological mismatching between plant and pollinator traits and nonoverlapping phenophases of plant and pollinator species. Thus, temporal dynamics provides a mechanistic explanation for previously reported network patterns such as the heterogeneous distribution of number of interactions across species. Second, we looked for the ecological properties most likely to be mediating this dynamical process and found that both abundance and phenoph...},
author = {Olesen, Jens M. and Bascompte, Jordi and Elberling, Heidi and Jordano, Pedro},
doi = {10.1890/07-0451.1},
isbn = {0012-9658},
issn = {00129658},
journal = {Ecology},
keywords = {Abundance,Arctic,Constraint,Ecological network,Linkages between species,Mutualistic network,Phenology,Pollination,Preferential attachment},
number = {6},
pages = {1573--1582},
pmid = {18589522},
title = {{Temporal dynamics in a pollination network}},
volume = {89},
year = {2008}
}
@article{Nunez2010,
abstract = {Exotic herbivores represent a serious threat to native biodiversity, producing large scale changes in native communities and altering ecosystem processes. In this special issue, we present a series of case studies and reviews from different areas of the world that highlight (1) the consequences of herbivore introductions are a global problem; (2) they can result in wholesale shifts in the distribution of dominant plants on the landscape and; (3) the effects of herbivore introductions extend from the population to the community and ecosystem level. These studies suggest that introduced herbivores often retard ecosystem recovery after disturbance, facilitate invasion of plant species and can act as selective agents on native plant communities. These studies also suggest that several topics, including facilitation between exotic herbivores and exotic plants and animals (i.e., invasional meltdown) and the effect of exotic herbivores on ecosystem processes, require more research attention. Overall the papers in this special feature suggest that introduced herbivores are a global problem with wide-ranging ecological and evolutionary effects.},
author = {Nu{\~{n}}ez, Martin A. and Bailey, Joseph K. and Schweitzer, Jennifer A.},
doi = {10.1007/s10530-009-9626-x},
isbn = {1387-3547},
issn = {13873547},
journal = {Biological Invasions},
keywords = {Ecosystem ecology,Exotic species,Introduced herbivores,Invasional meltdown},
number = {2},
pages = {297--301},
pmid = {273},
title = {{Population, community and ecosystem effects of exotic herbivores: A growing global concern}},
volume = {12},
year = {2010}
}
@article{Gurevitch2000,
abstract = {Ecologists working with a range of organisms and environments have carried out manipulative field experiments that enable us to ask questions about the interaction between competition and predation (including herbivory) and about the relative strength of competition and predation in the field. Evaluated together, such a collection of studies can offer insight into the importance and function of these factors in nature. Using a new factorial meta-analysis technique, we combined the results of 20 articles reporting on 39 published field experiments to ask whether the presence of predators affects the intensity of competitive effects and to compare the average effects of competition and predation. Across all studies, the effects of competition in the presence of predators were less than in the absence of predators, and the interaction between competition and predation for most response variables was statistically significant. Removal of competitors had much more positive effects on organisms' growth and mass than did exclusion of predators. Predator exclusion had much more beneficial effects on organisms' survival than did competition. The mean effects of competition and predation on density did not differ from one another. The results differed among trophic levels. Further understanding would benefit greatly from more field experiments that manipulate both competition and predation, that focus on a wider range of organisms and environments, that focus on population-level parameters such as density, and that report results more completely, including data such as sample sizes and variances.},
author = {Gurevitch and Morrison and Hedges},
doi = {10.2307/3078927},
isbn = {0003-0147},
issn = {00030147},
journal = {The American Naturalist},
keywords = {be addressed,bio,competition,e-mail,ecological experiments,edu,herbivory,jgurvtch,life,meta-analysis,predation,statistical interac-,sunysb,tion,to whom correspondence should},
number = {4},
pages = {435},
pmid = {10753073},
title = {{The Interaction between Competition and Predation: A Meta-Analysis of Field Experiments}},
url = {http://www.journals.uchicago.edu/doi/10.1086/303337},
volume = {155},
year = {2017}
}
@article{Kopelke2017,
abstract = {Communities consist of species and their interactions. They can thus be described as networks, with species as nodes and interactions as links. Within such networks, the diversity of nodes and the distribution of links may affect patterns of energy transfer between trophic levels, the dynamics of the system, and the outcome in terms of ecosystem functioning. To date, most descriptions of networks have focused on single or relatively few sites, and have oftentimes been built on poorly resolved nodes and links. Yet, comparisons of local interaction networks reveal variation in space and in time, thus spurring interest in methods and theory for understanding patterns, drivers, and consequences of this variation. Progress in this field relies on access to replicate samples of comparable food webs across large spatiotemporal scales, resolved to species rather than to compound nodes. Due to the massive efforts required, high-quality data sets are still scarce. We created a data set on a single community type sampled across Europe: willow species (Salix), willow-galling sawflies (Hymenoptera: Tenthredinidae: Nematinae: Euurina), and their natural enemies (hymenopteran parasitoids and coleopteran, lepidopteran, dipteran, and hymenopteran inquilines). Each sample was referenced in space and time, and each node resolved with the highest possible resolution, including taxonomic affinity, gall type (for herbivores), and mode of parasitism (for natural enemies). Galler survival and link structure were resolved by dissection and rearing of gall inhabitants. In total, the data set is based on 641 site visits over 29 years, and on 165,424 galls representing 96 herbivore nodes and 52 plant nodes. The dissections and rearings yielded 42,129 natural enemies belonging to 126 species, and revealed 1,173 different links. The spatiotemporal and taxonomic resolution of these data make them amenable to analyses of both ecological and evolutionary processes of network assembly. Thus, this data set will facilitate testing of important hypotheses in recent community theory, concerning, e.g., the sampling effort needed to adequately describe interaction structure within ecological communities, the impact of environmental conditions and biotic filters on the distribution of species and their interactions, and the relationship between the global “metaweb” and its local realizations.},
author = {Kopelke, Jens Peter and Nyman, Tommi and Cazelles, K{\'{e}}vin and Gravel, Dominique and Vissault, Steve and Roslin, Tomas},
doi = {10.1002/ecy.1832},
isbn = {4955139574},
issn = {00129658},
journal = {Ecology},
keywords = {Europe,Salix,decadal census,ecological interaction networks,food webs,galler,inquiline,parasitoid,spatiotemporal community structure,trans-continental survey,trophic interactions},
number = {6},
pages = {1730},
pmid = {28199780},
title = {{Food-web structure of willow-galling sawflies and their natural enemies across Europe}},
volume = {98},
year = {2017}
}
@article{Poisot2012,
abstract = {In a context of global changes, and amidst the perpetual modification of community structure undergone by most natural ecosystems, it is more important than ever to understand how species interactions vary through space and time. The integration of biogeography and network theory will yield important results and further our understanding of species interactions. It has, however, been hampered so far by the difficulty to quantify variation among interaction networks. Here, we propose a general framework to study the dissimilarity of species interaction networks over time, space or environments, allowing both the use of quantitative and qualitative data. We decompose network dissimilarity into interactions and species turnover components, so that it is immediately comparable to common measures of $\beta$-diversity. We emphasise that scaling up $\beta$-diversity of community composition to the $\beta$-diversity of interactions requires only a small methodological step, which we foresee will help empiricists adopt this method. We illustrate the framework with a large dataset of hosts and parasites interactions and highlight other possible usages. We discuss a research agenda towards a biogeographical theory of species interactions.},
archivePrefix = {arXiv},
arxivId = {arXiv:0707.1616v3},
author = {Poisot, Timoth{\'{e}}e and Canard, Elsa and Mouillot, David and Mouquet, Nicolas and Gravel, Dominique},
doi = {10.1111/ele.12002},
eprint = {arXiv:0707.1616v3},
file = {:Users/alyssacirtwill/Documents/Papers/Poisot et al.{\_}2012{\_}Ecology Letters.pdf:pdf},
isbn = {1461-0248},
issn = {1461023X},
journal = {Ecology Letters},
keywords = {Food web,Metaweb,Species interaction networks,$\beta$-diversity},
number = {12},
pages = {1353--1361},
pmid = {22994257},
title = {{The dissimilarity of species interaction networks}},
volume = {15},
year = {2012}
}
@article{Frost2016,
abstract = {Species have strong indirect effects on others, and predicting these effects is a central challenge in ecology. Prey species sharing an enemy (predator or parasitoid) can be linked by apparent competition, but it is unknown whether this process is strong enough to be a community-wide structuring mechanism that could be used to predict future states of diverse food webs. Whether species abundances are spatially coupled by enemy movement across different habitats is also untested. Here, using a field experiment, we show that predicted apparent competitive effects between species, mediated via shared parasitoids, can significantly explain future parasitism rates and herbivore abundances. These predictions are successful even across edges between natural and managed forests, following experimental reduction of herbivore densities by aerial spraying of insecticide over 20[thinsp]hectares. This result shows that trophic indirect effects propagate across networks and habitats in important, predictable ways, with implications for landscape planning, invasion biology and biological control.},
author = {Frost, Carol M. and Peralta, Guadalupe and Rand, Tatyana A. and Didham, Raphael K. and Varsani, Arvind and Tylianakis, Jason M.},
doi = {10.1038/ncomms12644},
isbn = {2041-1723},
issn = {20411723},
journal = {Nature Communications},
number = {August},
pages = {1--12},
pmid = {27577948},
publisher = {Nature Publishing Group},
title = {{Apparent competition drives community-wide parasitism rates and changes in host abundance across ecosystem boundaries}},
url = {http://dx.doi.org/10.1038/ncomms12644},
volume = {7},
year = {2016}
}
@article{Gotelli2000,
author = {Nicholas, J. Gotelli},
doi = {10.1890/0012-9658(2000)081[2606:NMAOSC]2.0.CO;2},
isbn = {1939-9170},
issn = {0012-9658},
journal = {Ecology},
keywords = {absence matrix,assembly rules,checkerboard distribution,co-occurrence,coexistence,com-,community structure,monte carlo simulation,null model,petition,presence,randomization,species combinations,test},
number = {9},
pages = {2606--2621},
title = {{Null Model Analysis of Species Co-Occurrence Patterns}},
url = {http://www.esajournals.org/doi/abs/10.1890/0012-9658(2000)081[2606:NMAOSC]2.0.CO;2},
volume = {81},
year = {2000}
}
@article{Haugaasen2010,
abstract = { We know surprisingly little about the fate of seeds of the Brazil nut tree ( Bertholletia excelsa ) under natural conditions. Here we investigate seed removal, predation and caching of Brazil nuts by scatter-hoarding rodents in the wet and dry seasons, based on an experimental approach using 900 thread-marked seeds. We tracked the fate of seeds handled by these animals to examine how seasonal food availability may influence caching rates, dispersal distances and cache longevity. Most seeds exposed to dispersal trials were removed by scatter-hoarders during the first week in both seasons and seeds were generally buried intact in single-seeded caches within 10 m of seed stations. Seeds were removed significantly faster and buried at greater distances during the dry season. The proportion of seeds buried intact was considerably higher in the wet season (74.4{\%}) than in the dry season (38.2{\%}). Most (99.4{\%}) of the 881 primary caches monitored were recovered, but these had a significantly shorter lifetime in the dry season. Our results show that rodents are highly skilled at retrieving buried Brazil nuts and that caching behaviour appears to be affected by seasonal resource abundance. Reduced seed availability due to intensive harvest could potentially create a dry-season scenario where most seeds succumb to pre-dispersal predation, thereby adversely affecting the natural regeneration of Brazil nut trees. },
author = {Haugaasen, Joanne M. Tuck and Haugaasen, Torbj{\o}rn and Peres, Carlos A. and Gribel, Rogerio and Wegge, Per},
doi = {10.1017/s0266467410000027},
isbn = {0266-4674},
issn = {0266-4674},
journal = {Journal of Tropical Ecology},
keywords = {Agouti,Bertholletia excelsa,Dasyprocta fuliginosa,Fruit abundance,Myoprocta pratti,Neotropical forest,Resource seasonality,Secondary caches,Seed germination,Seed predation},
number = {03},
pages = {251--262},
title = {{Seed dispersal of the Brazil nut tree ( Bertholletia excelsa) by scatter-hoarding rodents in a central Amazonian forest}},
url = {http://www.journals.cambridge.org/abstract{\_}S0266467410000027},
volume = {26},
year = {2010}
}
@article{McCann1997,
abstract = {Under equilibrium conditions, previous theory has shown that the presence of omnivory destabilizes food webs. Correspondingly, omnivory ought to be rare in real food webs. Although, early food web data appeared to verify this, recently many ecologists have found omnivory to be ubiquitous in food web data gathered at a high taxonomic resolution. In this paper, we re-investigate the role of omnivory in food webs using a non-equilibrium perspective. We find that the addition of omnivory to a simple food chain model (thus a simple food web) locally stabilizes the food web in a very complete way. First, non-equilibrium dynamics (e.g. chaos) tend to be eliminated or bounded further away from zero via period-doubling reversals invoked by the omnivorous trophic link. Second, food chains without interior attractors tend to gain a stable interior attractor with moderate amounts of omnivory.},
author = {McCann, K. and Hastings, A.},
doi = {10.1098/rspb.1997.0172},
isbn = {09628452},
issn = {14712970},
journal = {Proceedings of the Royal Society B: Biological Sciences},
number = {1385},
pages = {1249--1254},
title = {{Re-evaluating the omnivory-stability relationship in food webs}},
url = {http://rspb.royalsocietypublishing.org/cgi/doi/10.1098/rspb.1997.0172},
volume = {264},
year = {1997}
}
@article{Morris2014,
abstract = {An increase in species richness with decreasing latitude is a prominent pattern in nature. However, it remains unclear whether there are corresponding latitudinal gradients in the properties of ecological interaction networks. We investigated the structure of 216 quantitative antagonistic networks comprising insect hosts and their parasitoids, drawn from 28 studies from the High Arctic to the tropics. Key metrics of network structure were strongly affected by the size of the interaction matrix (i.e. the total number of interactions documented between individuals) and by the taxonomic diversity of the host taxa involved. After controlling for these sampling effects, quantitative networks showed no consistent structural patterns across latitude and host guilds, suggesting that there may be basic rules for how sets of antagonists interact with resource species. Furthermore, the strong association between network size and structure implies that many apparent spatial and temporal variations in network structure may prove to be artefacts.},
author = {Morris, Rebecca J. and Gripenberg, Sofia and Lewis, Owen T. and Roslin, Tomas},
doi = {10.1111/ele.12235},
file = {:Users/alyssacirtwill/Documents/Papers/Morris et al.{\_}2014{\_}Ecology Letters.pdf:pdf},
isbn = {1461-0248 (Electronic)$\backslash$r1461-023X (Linking)},
issn = {14610248},
journal = {Ecology Letters},
keywords = {Antagonistic network,Guild,Host-parasitoid,Latitude,Matrix size,Network metrics,Network structure,Quantitative food web,Specialisation,Taxonomic diversity},
number = {3},
pages = {340--349},
pmid = {24354432},
title = {{Antagonistic interaction networks are structured independently of latitude and host guild}},
volume = {17},
year = {2014}
}
@article{Macgregor2017,
abstract = {Background: The analysis of ecological networks can be affected by sampling effort, potentially leading to bias. Ecological network structure is often summarised by descriptive metrics but these metrics can vary according to the proportion of the total interactions that have been observed. Therefore, to know the likely degree of bias, it is valuable to estimate the total number of interactions in a network, and so calculate the proportion of interactions that have been observed (sampling completeness of interactions). Existing approaches to estimate sampling completeness of interactions use the Chao family of asymptotic species richness estimators to predict the total number of interactions, but do not fully utilise information about the relative specialisation of species within the network. Results: Here, we propose a modification of previously-used methods, that places equal weight on each interaction (whether or not it has been observed), rather than on each species. Our approach is therefore equivalent to weighting the interaction sampling completeness of each species in the network according to its relative specialisation. We demonstrate that, for the subset of species that are observed and when assuming that species richness estimators accurately project the number of unobserved interactions per observed species, our approach is mathematically more accurate. Our approach can be universally applied to any quantitative, bipartite network. We propose two methods to estimation using our approach, using abundance-based and incidence-based species richness estimators respectively, and give recommendations when each should be applied. We discuss the effect of unobserved species and the potential use of a threshold of minimum abundance for species inclusion. Finally, we consider these advances in the context of some of the main issues surrounding estimation of interaction sampling completeness in network ecology. Conclusions: We recommend that future studies of bipartite networks utilise our approach and methods to estimate the sampling completeness of interactions, to assist with the quantitative and comparative analysis and interpretation of network properties.},
author = {Macgregor, Callum J. and Evans, Darren M. and Pocock, Michael J. O.},
doi = {10.1101/195917},
journal = {bioRxiv},
pages = {195917},
title = {{Estimating sampling completeness of interactions in quantitative bipartite ecological networks: incorporating variation in species specialisation}},
url = {https://www.biorxiv.org/content/early/2017/09/29/195917},
year = {2017}
}
@article{Pellissier2017,
abstract = {Knowledge of species composition and their interactions, in the form of interaction networks, is required to understand processes shaping their distribution over time and space. As such, comparing ecological networks along environmental gradients represents a promising new research avenue to understand the organization of life. Variation in the position and intensity of links within networks along environmental gradients may be driven by turnover in species composition, by variation in species abundances and by abiotic influences on species interactions. While investigating changes in species composition has a long tradition, so far only a limited number of studies have examined changes in species interactions between networks, often with differing approaches. Here, we review studies investigating variation in network structures along environmental gradients, highlighting how methodological decisions about standardization can influence their conclusions. Due to their complexity, variation among ecological networks is frequently studied using properties that summarize the distribution or topology of interactions such as number of links, connectance, or modularity. These properties can either be compared directly or using a procedure of standardization. While measures of network structure can be directly related to changes along environmental gradients, standardization is frequently used to facilitate interpretation of variation in network properties by controlling for some co-variables, or via null models. Null models allow comparing the deviation of empirical networks from random expectations and are expected to provide a more mechanistic understanding of the factors shaping ecological networks when they are coupled with functional traits. As an illustration, we compare approaches to quantify the role of trait matching in driving the structure of plant–hummingbird mutualistic networks, i.e. a direct comparison, standardized by null models and hypothesis-based metaweb. Overall, our analysis warns against a comparison of studies that rely on distinct forms of standardization, as they are likely to highlight different signals. Fostering a better understanding of the analytical tools available and the signal they detect will help produce deeper insights into how and why ecological networks vary along environmental gradients.},
author = {Pellissier, Lo{\"{i}}c and Albouy, Camille and Bascompte, Jordi and Farwig, Nina and Graham, Catherine and Loreau, Michel and Maglianesi, Maria Alejandra and Meli{\'{a}}n, Carlos J. and Pitteloud, Camille and Roslin, Tomas and Rohr, Rudolf and Saavedra, Serguei and Thuiller, Wilfried and Woodward, Guy and Zimmermann, Niklaus E. and Gravel, Dominique},
doi = {10.1111/brv.12366},
isbn = {1469-185X},
issn = {1469185X},
journal = {Biological Reviews},
keywords = {environmental gradient,metaweb,motif,network,network comparison,network properties,null model,rarefaction analysis},
number = {2},
pages = {785--800},
pmid = {28941124},
title = {{Comparing species interaction networks along environmental gradients}},
url = {http://doi.wiley.com/10.1111/brv.12366},
volume = {93},
year = {2018}
}
@article{Dormann2017,
abstract = {How ecological niche breadth evolves is central to adaptation and speciation and has been a topic of perennial interest. Niche breadth evolution research has occurred within environmental, ecological, evolutionary, and biogeo-graphical contexts, and although some generalities have emerged, critical knowledge gaps exist. Performance breadth trade-offs, although long in-voked, may not be common determinants of niche breadth evolution or limits. Niche breadth can expand or contract from specialist or generalist lin-eages, and so specialization need not be an evolutionary dead end. Whether niche breadth determines diversification and distribution breadth and how niche breadth is partitioned among individuals and populations within a species are important but particularly understudied topics. Molecular ge-netic and phylogenetic techniques have greatly expanded understanding of niche breadth evolution, but field studies of how niche breadth evolves are essential for providing mechanistic details and allowing the development of comprehensive theory and improved prediction of biological responses under global change.},
author = {Sexton, Jason P and Montiel, Jorge and Shay, Jackie E and Stephens, Molly R and Slatyer, Rachel A},
doi = {10.1146/annurev-ecolsys-110316},
issn = {07306784},
journal = {Annu. Rev. Ecol. Evol. Syst},
keywords = {adaptation,fundamental niche,niche evolution,performance breadth trade-offs,realized niche,specialization,speciation},
number = {4},
pages = {183--206},
title = {{Evolution of Ecological Niche BreadthSexton, J. P., Montiel, J., Shay, J. E., Stephens, M. R., {\&}Slatyer, R. A. (2017). Evolution of Ecological Niche Breadth. Annu. Rev. Ecol. Evol. Syst, 48, 183–206. https://doi.org/10.1146/annurev-ecolsys-110316}},
url = {https://doi.org/10.1146/annurev-ecolsys-110316-023003},
volume = {48},
year = {2017}
}
@article{Morales-Castilla2015,
abstract = {Inferring biotic interactions from functional, phylogenetic and geographical proxies remains one great challenge in ecology. We propose a conceptual framework to infer the backbone of biotic interaction networks within regional species pools. First, interacting groups are identified to order links and remove forbidden interactions between species. Second, additional links are removed by examination of the geographical context in which species co-occur. Third, hypotheses are proposed to establish interaction probabilities between species. We illustrate the framework using published food-webs in terrestrial and marine systems. We conclude that preliminary descriptions of the web of life can be made by careful integration of data with theory.},
author = {Morales-Castilla, Ignacio and Matias, Miguel G. and Gravel, Dominique and Ara{\'{u}}jo, Miguel B.},
doi = {10.1016/j.tree.2015.03.014},
isbn = {1872-8383 (Electronic)$\backslash$r0169-5347 (Linking)},
issn = {01695347},
journal = {Trends in Ecology and Evolution},
number = {6},
pages = {347--356},
pmid = {25922148},
title = {{Inferring biotic interactions from proxies}},
volume = {30},
year = {2015}
}
@article{Roslin2016,
abstract = {By depicting who eats whom, food webs offer descriptions of how groupings in nature (typically species or populations) are linked to each other. For asking questions on how food webs are built and work, we need descriptions of food webs at different levels of resolution. DNA techniques provide opportunities for highly-resolved webs. In this paper, we offer an expos{\'{e}} of how DNA-based techniques, and DNA barcodes in particular, have recently been used to construct food web structure in both terrestrial and aquatic systems. We highlight how such techniques can be applied to simultaneously improve the taxonomic resolution of the nodes of the web (i.e. the species), and the links between them (i.e. who eats whom). We end by proposing how DNA barcodes and DNA information may allow new approaches to the construction of larger interaction webs, and overcome some hurdles to achieving adequate sample size. Most importantly, we propose that the joint adoption and development of these techniques may serve to unite ap...},
archivePrefix = {arXiv},
arxivId = {arXiv:1408.1149},
author = {Roslin, Tomas and Majaneva, Sanna},
doi = {10.1139/gen-2015-0229},
eprint = {arXiv:1408.1149},
file = {:Users/alyssacirtwill/Documents/Papers/Roslin, Majaneva{\_}2016{\_}Genome.pdf:pdf},
isbn = {0831-2796},
issn = {0831-2796},
journal = {Genome},
keywords = {afin de d{\'{e}}terminer comment,ces r{\'{e}}seaux sont constitu{\'{e}}s,de qui mange qui,diff{\'{e}}rents niveaux de r{\'{e}}solution,dna barcodes,ecological networks,en brossant le tableau,esp{\`{e}}ces ou de populations,et,food webs,les,les r{\'{e}}seaux trophiques livrent,qui,r{\'{e}}sum{\'{e}},species delimitation,species identification,trophic links,une description des liens,unissent des groupes d},
number = {9},
pages = {603--628},
pmid = {27484156},
title = {{The use of DNA barcodes in food web construction—terrestrial and aquatic ecologists unite!}},
url = {http://www.nrcresearchpress.com/doi/10.1139/gen-2015-0229},
volume = {59},
year = {2016}
}
@article{Weinstein2017a,
abstract = {By specialising on specific resources, species evolve advantageous morphologies to increase the efficiency of nutrient acquisition. However, many specialists face variation in resource availability and composition. Whether specialists respond to these changes depends on the composition of the resource pulses, the cost of foraging on poorly matched resources, and the strength of interspecific competition. We studied hummingbird bill and plant corolla matching during seasonal variation in flower availability and morphology. Using a hierarchical Bayesian model, we accounted for the detectability and spatial overlap of hummingbird-plant interactions. We found that despite sea- sonal pulses of flowers with short-corollas, hummingbirds consistently foraged on well-matched flowers, leading to low niche overlap. This behaviour suggests that the costs of searching for rare and more specialised resources are lower than the benefit of switching to super-abundant resources. Our results highlight the trade-off between foraging efficiency and interspecific competi- tion, and underline niche partitioning in maintaining tropical diversity.},
author = {Weinstein, Ben G. and Graham, Catherine H.},
doi = {10.1111/ele.12730},
file = {:Users/alyssacirtwill/Documents/Papers/Weinstein, Graham{\_}2017{\_}Ecology Letters.pdf:pdf},
isbn = {9780511542268},
issn = {14610248},
journal = {Ecology Letters},
keywords = {Co-evolution,Ecuador,hummingbirds,networks,species interactions},
number = {3},
pages = {326--335},
title = {{Persistent bill and corolla matching despite shifting temporal resources in tropical hummingbird-plant interactions}},
volume = {20},
year = {2017}
}
@article{Kaartinen2011,
abstract = {With habitat fragmentation spreading around the world, there is a pressing need to understand its impacts on local food webs. To date, few studies have examined the effects of landscape context on multiple local communities in a quantitative, spatially realistic setting. To examine how the isolation of a food web affects its structure, we construct local food webs of specialist herbivores and their natural enemies on 82 individual oaks (Quercus robur) growing in different landscape contexts. Across this set of webs, we find that communities in isolated habitat patches not only contained fewer species than did well-connected ones, but also differed in species composition. Surprisingly, the effects observed in terms of species composition were not reflected in the quantitative interaction structure of local food webs: landscape context had no detectable effect on either the interaction evenness, linkage density, connectance, generality or vulnerability of local webs. We conclude that the quantitative structure of food webs may be stable in the face of habitat fragmentation, despite clear-cut impacts on individual species. This finding offers hope-inspiring news for conservation, but should clearly be verified by empirical studies across both naturally and more recently fragmented systems},
author = {Kaartinen, Riikka and Roslin, Tomas},
doi = {10.1111/j.1365-2656.2011.01811.x},
isbn = {1365-2656},
issn = {00218790},
journal = {Journal of Animal Ecology},
keywords = {Community structure,Connectivity,Gall wasp,Leaf-miner,Metacommunity,Oak,Parasitoid,Quantitative food web,Quercus robur,Species richness},
number = {3},
pages = {622--631},
pmid = {21323923},
title = {{Shrinking by numbers: Landscape context affects the species composition but not the quantitative structure of local food webs}},
volume = {80},
year = {2011}
}
@article{Weinstein2017,
abstract = {To determine the causes, frequency and stability of biotic interactions, ecologists often use field observations to generate networks of interacting species. The challenge is in determining whether unobserved interactions were undetected due to sampling, or whether they truly do not occur. This uncertainty makes it difficult to predict interactions based on ecological mechanisms. By using hierarchical Bayesian N-mixture models adapted from wildlife ecology, we differentiated the probability of detecting a species from the intensity of species interactions. We began with a brief simulation to illustrate the consequences of ignoring detectability when building species networks. We then applied our model to an empirical dataset on Andean hummingbirds to compare trait-matching and plant abundance in explaining flower visitation patterns. The daily probability of detecting a hummingbird interaction ranged from 14{\%} to 70{\%} among species, underlining the importance of multiple days of sampling when estimating species interactions. The trait-matching model best fit hummingbird visitation rates among plants and pollinators. The plant abundance model was inconsistent across hummingbird species and did not support neutral interactions based on increasing plant abundance. We recommend that network studies account for interaction detectability by separately modeling observation and process mechanisms, focusing on species-level discrepancy in assessing model adequacy, and estimating uncertainty when computing network statistics.},
author = {Weinstein, Ben G. and Graham, Catherine H.},
doi = {10.1016/j.fooweb.2017.05.002},
file = {:Users/alyssacirtwill/Documents/Papers/Weinstein, Graham{\_}2017{\_}Food Webs.pdf:pdf},
issn = {23522496},
journal = {Food Webs},
keywords = {Bayesian,Ecuador,Hummingbirds,N-mixture,Trait-matching},
number = {January},
pages = {17--25},
publisher = {Elsevier},
title = {{On comparing traits and abundance for predicting species interactions with imperfect detection}},
url = {http://dx.doi.org/10.1016/j.fooweb.2017.05.002},
volume = {11},
year = {2017}
}
@article{Emmerson2004b,
abstract = {1. We examined the empirical relationship between predator-prey body size ratio and interaction strength in the Ythan Estuary food web. 2. We have refined a previously published version of the food web and explored how size-based predatory effects might affect food web dynamics. To do so, we used four predatory species Crangon crangon (Linnaeus), Carcinus maenas (Linnaeus), Pomatoschistus microps (Kroyer) and Platichthys flesus (Linnaeus) and one common prey species Corophium volutator (Pallas) from the food web. 3. All predators and prey were sorted into small, medium and large size classes and placed into mesocosms in all possible pairwise combinations of size and species identity to determine per capita effects of predators on prey (a(ij)). 4. Using Lotka-Volterra dynamics the empirical body size relationships obtained from these experiments and other relationships already available for the Ythan Estuary, we parameterized a food web model for this system. The local stability properties of the resulting food web models were then determined. 5. We found that by choosing interaction strengths using an empirically defined scaling law, the resulting food web models are always dynamically stable, despite the residual uncertainties in the modelling approach. This contrasts with the statistical expectation that random webs with random parameters have a vanishingly improbable chance of stability. 6. The patterning of predator and prey body sizes in real ecosystems affects the arrangement of interaction strengths, which in turn determines food web stability},
author = {Newton, A.M.J. and Lakshmanan, Prabakaran},
doi = {10.1111/j.0021-8790.2004.00818.x},
isbn = {0021-8790},
issn = {00218790},
journal = {Recent Patents on Drug Delivery {\&} Formulation},
keywords = {Allometric,Community,Ecosystem,Functioning,Power law},
number = {1},
pages = {46--62},
pmid = {5957},
title = {{Effect of HPMC - E15 LV premium Polymer on Release Profile and Compression Characteristics of Chitosan/ Pectin Colon Targeted Mesalamine Matrix Tablets and in vitro Study on Effect of pH Impact on the Drug Release Profile}},
url = {http://www.eurekaselect.com/openurl/content.php?genre=article{\&}issn=1872-2113{\&}volume=8{\&}issue=1{\&}spage=46},
volume = {8},
year = {2014}
}
@article{Jonsson1998,
abstract = {The effects of predator-prey body size ratios on the resilience and probability of stability in linear Lotka-Volterra food chains have been analysed. The prey per capita interaction strengths of the model is assumed to be negatively correlated to the relative size difference between a predator and its prey. The relationship between prey interaction strength and predator-prey body size ratios is motivated by energetical arguments. Analytical results show that, given this assumption (on prey interaction strengths) and if average (relative) size differences between predators and their prey decrease with the trophic position of the consumer (as found in a large number of 'real food webs') the probability of local stability in model food chains is increased (when compared to model chains with a constant predator-prey body size ratio). Numerical simulations show that in most cases, the effect on the probability of stability is accompanied by an increase in resilience. For example, as model food chain length is increased from two to three trophic levels in one simulation, the return time increases by more than two orders of magnitude with a constant predator-prey body mass ratio while chains longer than four are not feasible. With a decreasing predator-prey body mass ratio on the other hand, the return time does not increase as rapidly and feasible equilibria exist for longer chains. The relationship between resilience and food chain length is, in this model, affected by the relationship between the predator-prey body mass ratio and the trophic position of the predator, that is, how fast this ratio decreases with increasing trophic height. The effect of body mass on consumer mortality rates, and subsequently on the probability of stability and resilience is also analysed. Decreasing mortality rates with increasing body size do not change the results qualitatively, it only increases the probability that an equilibrium is feasible.},
author = {Jonsson, Tomas and Ebenman, Bo},
doi = {10.1006/jtbi.1998.0708},
isbn = {0022-5193},
issn = {00225193},
journal = {Journal of Theoretical Biology},
keywords = {article,body size,ecosystem,food chain,model,mortality,predator,predator-prey interaction,priority journal,resilience,stability analysis},
number = {3},
pages = {407--417},
pmid = {9735269},
title = {{Effects of predator-prey body size ratios on the stability of food chains}},
url = {http://www.scopus.com/inward/record.url?eid=2-s2.0-0032493594{\&}partnerID=40{\&}md5=16ac360663c3ac5a7826d0618d32dfd0},
volume = {193},
year = {1998}
}
@article{Basset2007,
abstract = {Body size is a major phenotypic trait of individuals that commonly differentiates co-occurring species. We analyzed inter-specific competitive interactions between a large consumer and smaller competitors, whose energetics, selection and giving-up behaviour on identical resource patches scaled with individual body size. The aim was to investigate whether pure metabolic constraints on patch behaviour of vagile species can determine coexistence conditions consistent with existing theoretical and experimental evidence. We used an individual- based spatially explicit simulation model at a spatial scale defined by the home range of the large consumer, which was assumed to be parthenogenic and semelparous. Under exploitative conditions, competitive coexistence occurred in a range of body size ratios between 2 and 10. Asymmetrical competition and the mechanism underlying asymmetry, determined by the scaling of energetics and patch behaviour with consumer body size, were the proximate determinant of inter-specific coexistence. The small consumer exploited patches more efficiently, but searched for profitable patches less effectively than the larger competitor. Therefore, body- size related constraints induced niche partitioning, allowing competitive coexistence within a set of conditions where the large consumer maintained control over the small consumer and resource dynamics. The model summarises and extends the existing evidence of species coexistence on a limiting resource, and provides a mechanistic explanation for decoding the size-abundance distribution patterns commonly observed at guild and community levels.},
author = {Basset, Alberto and {L. Angelis}, Donald},
doi = {10.1111/j.0030-1299.2007.15702.x},
file = {:Users/alyssacirtwill/Documents/Papers/Basset, L. Angelis{\_}2007{\_}Oikos.pdf:pdf},
isbn = {0030-1299},
issn = {16000706},
journal = {Oikos},
number = {8},
pages = {1363--1377},
title = {{Body size mediated coexistence of consumers competing for resources in space}},
volume = {116},
year = {2007}
}
@article{Barbour2016,
abstract = {Theory predicts that intraspecific genetic variation can increase the complexity of an ecological network. To date, however, we are lacking empirical knowledge of the extent to which genetic variation determines the assembly of ecological networks, as well as how the gain or loss of genetic variation will affect network structure. To address this knowledge gap, we used a common garden experiment to quantify the extent to which heritable trait variation in a host plant determines the assembly of its associated insect food web (network of trophic interactions). We then used a resampling procedure to simulate the additive effects of genetic variation on overall food-web complexity. We found that trait variation among host-plant genotypes was associated with resistance to insect herbivores, which indirectly affected interactions between herbivores and their insect parasitoids. Direct and indirect genetic effects resulted in distinct compositions of trophic interactions associated with each host-plant genotype. Moreover, our simulations suggest that food-web complexity would increase by 20{\%} over the range of genetic variation in the experimental population of host plants. Taken together, our results indicate that intraspecific genetic variation can play a key role in structuring ecological networks, which may in turn affect network persistence.},
author = {Barbour, Matthew A. and Fortuna, Miguel A. and Bascompte, Jordi and Nicholson, Joshua R. and Julkunen-Tiitto, Riitta and Jules, Erik S. and Crutsinger, Gregory M.},
doi = {10.1073/pnas.1513633113},
isbn = {0027-8424},
issn = {0027-8424},
journal = {Proceedings of the National Academy of Sciences},
number = {8},
pages = {2128--2133},
pmid = {26858398},
title = {{Genetic specificity of a plant–insect food web: Implications for linking genetic variation to network complexity}},
url = {http://www.pnas.org/lookup/doi/10.1073/pnas.1513633113},
volume = {113},
year = {2016}
}
@article{Martinez1999,
author = {Martinez, Neo D and Hawkins, Bradford A and Dawah, Hassan Ali and Feifarek, Brian P},
file = {:Users/alyssacirtwill/Documents/Papers/Martinez et al.{\_}1999{\_}Ecology.pdf:pdf},
journal = {Ecology},
keywords = {food webs,observation effort,quantitative food-web patterns,sampling effort,scale,scale dependence,scale invariance,trophic species},
number = {3},
pages = {1044--1055},
title = {{Effects of Sampling Effort on Characterization of Food-Web Structure}},
volume = {80},
year = {1999}
}
@article{Dickman1988,
author = {Dickman, C R and Jun, No},
journal = {Ecology},
keywords = {antechinus,body size,community structure,foraging microhabitat,in-,insectivore,parantechinus,prey size,sminthopsis,sorex,terspecific competition},
number = {3},
pages = {569--580},
title = {{Body Size , Prey Size , and Community Structure in Insectivorous Mammals}},
volume = {69},
year = {2007}
}
@article{Barbour2016Dryad,
abstract = {Theory predicts that intraspecific genetic variation can increase the complexity of an ecological network. To date, however, we are lacking empirical knowledge of the extent to which genetic variation determines the assembly of ecological networks, as well as how the gain or loss of genetic variation will affect network structure. To address this knowledge gap, we used a common garden experiment to quantify the extent to which heritable trait variation in a host plant determines the assembly of its associated insect food web (network of trophic interactions). We then used a resampling procedure to simulate the additive effects of genetic variation on overall food-web complexity. We found that trait variation among host-plant genotypes was associated with resistance to insect herbivores, which indirectly affected interactions between herbivores and their insect parasitoids. Direct and indirect genetic effects resulted in distinct compositions of trophic interactions associated with each host-plant genotype. Moreover, our simulations suggest that food-web complexity would increase by 20{\%} over the range of genetic variation in the experimental population of host plants. Taken together, our results indicate that intraspecific genetic variation can play a key role in structuring ecological networks, which may in turn affect network persistence.},
author = {Barbour, Matthew A. and Fortuna, Miguel A. and Bascompte, Jordi and Nicholson, Joshua R. and Julkunen-Tiitto, Riitta and Jules, Erik S. and Crutsinger, Gregory M.},
doi = {10.1073/pnas.1513633113},
isbn = {0027-8424},
issn = {0027-8424},
journal = {Proceedings of the National Academy of Sciences},
number = {8},
pages = {2128--2133},
pmid = {26858398},
title = {{Genetic specificity of a plant–insect food web: Implications for linking genetic variation to network complexity}},
url = {http://www.pnas.org/lookup/doi/10.1073/pnas.1513633113},
volume = {113},
year = {2016}
}
@article{Bowers1982,
author = {Bowers, Michael A and Brown, James H and Apr, No},
journal = {New York},
keywords = {and community,body size,coexistence,community,competition,desert rodent,empirical foundations,geographic range,guild,hutchinson,recently the theoretical and,s ratio,standing of interspecific interactions},
number = {2},
pages = {391--400},
title = {{Body Size and Coexistnce in Desert Rodents : Chance or Community Structure ? BODY SIZE AND COEXISTENCE IN DESERT RODENTS : CHANCE OR COMMUNITY STRUCTURE ?'}},
volume = {63},
year = {2007}
}
@article{Basset1995,
abstract = {This paper investigates whether body size-related constraints on home-range resource harvesting could lead to coexistence between interspecific competitors of different body size under conditions of complete niche overlap. With this objective, I analyzed the influence of body size on induced and sustainable resource limitation and its implications on the interaction between an individual and a fixed biomass of larger competitors in a four-dimensional space, consisting of two spatial dimensions describing competitor home range, resource availability, and time. It was shown that body size-related spatiotemporal constraints on home-range resource harvesting, trophic optimization, and relativity of resource availability determine absolute and relative amounts of unused home-range resources, restricting the influence of resource limitation induced by an individual to a definite size distance around its size. It was therefore concluded that: (1) when competition occurs asymmetrically with a superiority of large animals, size differences alone could allow coexistence, independently of any kind of resource partitioning; and (2) superiority of large animals should result from the resource density control that a larger competitor imposes on the smaller one whenever size differences for stable coexistence occur. Existing evidence of an inefficiency in home-range resource exploitation seems large enough to suggest a view of guilds based on a hierarchy of inclusive home ranges.},
author = {Basset, A.},
doi = {10.2307/1940913},
isbn = {0012-9658},
issn = {00129658},
journal = {Ecology},
keywords = {BEHAVIOUR-FEEDING-FORAGING,POPULATION-DISTRIBUTION},
number = {4},
pages = {1027--1035},
pmid = {23014334},
title = {{Body size-related coexistence: An approach through allometric constraints on home-range use}},
volume = {76},
year = {1995}
}
@article{Cirtwill2017,
abstract = {Introduction: D-dimer assay, generally evaluated according to cutoff points calibrated for VTE exclusion, is used to estimate the individual risk of recurrence after a first idiopathic event of venous thromboembolism (VTE). Methods: Commercial D-dimer assays, evaluated according to predetermined cutoff levels for each assay, specific for age (lower in subjects {\textless}70 years) and gender (lower in males), were used in the recent DULCIS study. The present analysis compared the results obtained in the DULCIS with those that might have been had using the following different cutoff criteria: traditional cutoff for VTE exclusion, higher levels in subjects aged ≥60 years, or age multiplied by 10. Results: In young subjects, the DULCIS low cutoff levels resulted in half the recurrent events that would have occurred using the other criteria. In elderly patients, the DULCIS results were similar to those calculated for the two age-adjusted criteria. The adoption of traditional VTE exclusion criteria would have led to positive results in the large majority of elderly subjects, without a significant reduction in the rate of recurrent event. Conclusion: The results confirm the usefulness of the cutoff levels used in DULCIS.},
author = {Cirtwill, Alyssa R. and Lagrue, Clement and Poulin, Robert and Stouffer, Daniel B.},
doi = {10.1002/ecy.1927},
file = {:Users/alyssacirtwill/Documents/Papers/Archive/Cirtwill, Stouffer{\_}2015{\_}Journal of Animal Ecology.pdf:pdf},
isbn = {4955139574},
issn = {00129658},
journal = {Ecology},
keywords = {concomitant predation,food-web dynamics,food-web structure,host specificity,link properties,trematodes},
number = {9},
pages = {2401--2412},
pmid = {27935037},
title = {{Host taxonomy constrains the properties of trophic transmission routes for parasites in lake food webs}},
volume = {98},
year = {2017}
}
@article{Bartomeus2017,
abstract = {The response and effect trait framework, if supported empirically, would provide for powerful and general predictions about how biodiversity loss leads to loss in ecosystem function. This framework proposes that species traits will explain how different species respond to disturbance (i.e. response traits) as well as their contribution to ecosystem function (i.e. effect traits). However, predictive response and effect traits remain elusive for most systems. Here, we use data on crop pollination services provided by native, wild bees to explore the role of six commonly used species traits in determining both species' response to land-use change and the subsequent effect on crop pollination. Analyses were conducted in parallel for three crop systems (watermelon, cranberry, and blueberry) located within the same geographical region (mid-Atlantic USA). Bee species traits did not strongly predict species' response to land-use change, and the few traits that were weakly predictive were not consistent across crops. Similarly, no trait predicted species' overall functional contribution in any of the three crop systems, although body size was a good predictor of per capita efficiency in two systems. Overall we were unable to make generalizable predictions regarding species responses to land-use change and its effect on the delivery of crop pollination services. Pollinator traits may be useful for understanding ecological processes in some systems, but thus far the promise of traits-based ecology has yet to be fulfilled for pollination ecology. This article is protected by copyright. All rights reserved. },
author = {Bartomeus, Ignasi and Cariveau, Daniel P. and Harrison, Tina and Winfree, Rachael},
doi = {10.1111/oik.04507},
isbn = {1600-0706},
issn = {16000706},
journal = {Oikos},
number = {2},
pages = {306--315},
pmid = {17922114},
title = {{On the inconsistency of pollinator species traits for predicting either response to land-use change or functional contribution}},
url = {http://doi.wiley.com/10.1111/oik.04507},
volume = {127},
year = {2018}
}
@article{Hanssen-Bauer1998,
abstract = {Observations from the Norwegian Arctic show positive trends in annual$\backslash$nmean temperatures from 1912 to the 1930s and from the 1960s to 1996.$\backslash$nBetween these periods there was a negative trend, and there is no$\backslash$nstatistically significant trend in the record as a whole. The present$\backslash$ntemperature is approximately the same as in the 1920s, and lower$\backslash$nthan during the 1930s and 1950s. Spring is the only season which$\backslash$nshows a statistically significant warming from 1912 to 1996. Annual$\backslash$nprecipitation, on the other hand, has increased in the Norwegian$\backslash$nArctic. At Spitsbergen the measurements show a statistically significant$\backslash$nincrease in annual and in spring, summer and autumn precipitation.$\backslash$nMonthly values of mean sea level pressure of 4 grid points were used$\backslash$nto develop models for monthly mean temperature and monthly precipitation$\backslash$nat Spitsbergen. During the period 1912 to 1993 the temperature model$\backslash$naccounts for 30 to 45{\%} of the variance in the seasonal mean temperatures.$\backslash$nThe correlation between observed and modelled values is at a minimum$\backslash$nin the summer and at a maximum in the autumn. The precipitation model$\backslash$naccounts for 15 to 35{\%} of the variance in seasonal precipitation$\backslash$nsums. The correlation between observed and modelled values is lowest$\backslash$nin winter, when the problems with drifting and blowing snow are greatest.$\backslash$nEven though the observed and modelled seasonal values in most cases$\backslash$nare better correlated for temperature than for precipitation, the$\backslash$nprecipitation model accounts for more of the decadal scale variability$\backslash$nand long-term trends. The precipitation model reproduces the observed$\backslash$npositive precipitation trends on both a seasonal and annual basis.$\backslash$nConcerning decadal scale variability, most of the main observed features$\backslash$nare also modelled satisfactorily. It is concluded that the major$\backslash$nobserved features concerning decadal scale variability and trends$\backslash$nin precipitation at Spitsbergen are connected to variability in the$\backslash$natmospheric circulation pattern. The temperature model reproduces$\backslash$nreasonably well the observed positive trend during the last 3 decades$\backslash$nof the series. The very low temperature before 1920 and the high$\backslash$nvalues in the 1930s and the 1950s, on the other hand, are not modelled$\backslash$nsatisfactorily. Thus, while the temperature increase of the later$\backslash$ndecades may mainly be explained as a result of changes in advection,$\backslash$nthe temperature increases in the Norwegian Arctic from the beginning$\backslash$nof the measurements to the 1930s cannot be explained in this way.},
author = {F{\o}rland, EJ and Hanssen-Bauer, I},
doi = {10.3354/cr010143},
isbn = {0936-577X},
issn = {0936-577X},
journal = {Climate Research},
keywords = {arctic,atmospheric circulation,climate variation,precipitation,temperature},
number = {1998},
pages = {143--153},
title = {{Long-term trends in precipitation and temperature in the Norwegian Arctic: can they be explained by changes in atmospheric circulation patterns?}},
volume = {10},
year = {1998}
}
@article{Kearns1998,
abstract = {The pollination of flowering plants by animals represents a critical ecosystem service of great value to humanity, both monetary and otherwise. However, the need for active conservation of pollination interactions is only now being appreciated. Pollination systems are under increasing threat from anthropogenic sources, including fragmentation of habitat, changes in land use, modern agricultural practices, use of chemicals such as pesticides and herbicides, and invasions of non-native plants and animals. Honeybees, which themselves are non-native pollinators on most continents, and which may harm native bees and other polli-nators, are nonetheless critically important for crop pollination. Recent declines in honeybee numbers in the United States and Europe bring home the importance of healthy pollination systems, and the need to further develop native bees and other animals as crop pollinators. The "pollination crisis" that is evident in declines of honeybees and native bees, and in damage to webs of plant-pollinator interaction, may be ameliorated not only by cultivation of a diversity of crop pollinators, but also by changes in habitat use and agricultural practices, species reintroductions and removals, and other means. In addition, ecologists must redouble efforts to study basic aspects of plant-pollinator interactions if optimal 83},
author = {Kearns, Carol A and Inouye, David W and Waser, Nickolas M},
doi = {10.1146/annurev.ecolsys.29.1.83},
isbn = {0066-4162},
issn = {0066-4162},
journal = {Annual Review of Ecology and Systematics},
keywords = {agriculture,ecosystem services,fragmentation,habitat alteration,species invasions},
number = {1998},
pages = {83--112},
pmid = {10196},
title = {{ENDANGERED MUTUALISMS: The Conservation of Plant-Pollinator Interactions}},
url = {www.annualreviews.org},
volume = {29},
year = {1998}
}
@article{Wirta2015a,
abstract = {How food webs are structured has major implications for their stability and dynamics. While poorly studied to date, arctic food webs are commonly assumed to be simple in structure, with few links per species. If this is the case, then different parts of the web may be weakly connected to each other, with populations and species united by only a low number of links. We provide the first highly resolved description of trophic link structure for a large part of a high-arctic food web. For this purpose, we apply a combination of recent techniques to describing the links between three predator guilds (insectivorous birds, spiders, and lepidopteran parasitoids) and their two dominant prey orders (Diptera and Lepidoptera). The resultant web shows a dense link structure and no compartmentalization or modularity across the three predator guilds. Thus, both individual predators and predator guilds tap heavily into the prey community of each other, offering versatile scope for indirect interactions across different parts of the web. The current description of a first but single arctic web may serve as a benchmark toward which to gauge future webs resolved by similar techniques. Targeting an unusual breadth of predator guilds, and relying on techniques with a high resolution, it suggests that species in this web are closely connected. Thus, our findings call for similar explorations of link structure across multiple guilds in both arctic and other webs. From an applied perspective, our description of an arctic web suggests new avenues for understanding how arctic food webs are built and function and of how they respond to current climate change. It suggests that to comprehend the community-level consequences of rapid arctic warming, we should turn from analyses of populations, population pairs, and isolated predator-prey interactions to considering the full set of interacting species.},
author = {Wirta, Helena K. and Vesterinen, Eero J. and Hamb{\"{a}}ck, Peter A. and Weingartner, Elisabeth and Rasmussen, Claus and Reneerkens, Jeroen and Schmidt, Niels M. and Gilg, Olivier and Roslin, Tomas},
doi = {10.1002/ece3.1647},
isbn = {2045-7758},
issn = {20457758},
journal = {Ecology and Evolution},
keywords = {Calidris,DNA barcoding,Generalism,Greenland,Hymenoptera,Molecular diet analysis,Pardosa,Plectrophenax,Specialism,Xysticus},
number = {17},
pages = {3842--3856},
pmid = {26380710},
title = {{Exposing the structure of an Arctic food web}},
volume = {5},
year = {2015}
}
@article{Cazelles2016,
abstract = {The study of species co-occurrences has been central in community ecology since the foundation of the discipline. Co-occurrence data are, nevertheless, a neglected source of information to model species distributions and biogeographers are still debating about the impact of biotic interactions on species distributions across geographical scales. We argue that a theory of species co-occurrence in ecological networks is needed to better inform interpretation of co-occurrence data, to formulate hypotheses for different community assembly mechanisms, and to extend the analysis of species distributions currently focused on the relationship between occurrences and abiotic factors. The main objective of this paper is to provide the first building blocks of a general theory for species co-occurrences. We formalize the problem with definitions of the different probabilities that are studied in the context of co-occurrence analyses. We analyze three species interactions modules and conduct multi-species simulations in order to document five principles influencing the associations between species within an ecological network: (i) direct interactions impact pairwise co-occurrence, (ii) indirect interactions impact pairwise co-occurrence, (iii) pairwise co-occurrence rarely are symmetric, (iv) the strength of an association decreases with the length of the shortest path between two species, and (v) the strength of an association decreases with the number of interactions a species is experiencing. Our analyses reveal the difficulty of the interpretation of species interactions from co-occurrence data. We discuss whether the inference of the structure of interaction networks is feasible from co-occurrence data. We also argue that species distributions models could benefit from incorporating conditional probabilities of interactions within the models as an attempt to take into account the contribution of biotic interactions to shaping individual distributions of species.},
archivePrefix = {arXiv},
arxivId = {arXiv:1011.1669v3},
author = {Cazelles, K{\'{e}}vin and Ara{\'{u}}jo, Miguel B. and Mouquet, Nicolas and Gravel, Dominique},
doi = {10.1007/s12080-015-0281-9},
eprint = {arXiv:1011.1669v3},
isbn = {1874-1738$\backslash$r1874-1746},
issn = {18741746},
journal = {Theoretical Ecology},
keywords = {Biogeography,Co-occurrence,Ecological networks,Indirect interactions,Null models},
number = {1},
pages = {39--48},
pmid = {11507039},
title = {{A theory for species co-occurrence in interaction networks}},
url = {https://link.springer.com/article/10.1007/s12080-015-0281-9},
volume = {9},
year = {2016}
}
@article{Garlachelli2003,
abstract = {Regeneration of the magnetic field by convection in the core places demands on heat flow into the base of the mantle. If the heat flow is too low, thermal convection is shut off and the rate of generation of compositional buoyancy by the solidification of the core becomes too low to sustain the geodynamo. Conversely, a large heat flow causes rapid growth of the inner core, so that convection prior to the appearance of the inner core must be sustained by thermal buoyancy alone. The attendant requirements on primordial heat become more severe as the age of the inner core decreases. We show that estimates of the present-day heat flow satisfy the power requirements for the geodynamo. However, the resulting thermal history is incompatible with estimates of mantle temperatures prior to 3 Ga. This discrepancy can be resolved by accumulating radioactive isotopes in D or adding heat sources to the core.},
author = {Buffett, Bruce A.},
doi = {10.1029/2001GL014649},
isbn = {0028-0836},
issn = {1022386X},
journal = {Geophysical Research Letters},
number = {12},
pages = {165--168},
pmid = {12736684},
title = {{Estimates of heat flow in the deep mantle based on the power requirements for the geodynamo}},
url = {http://doi.wiley.com/10.1029/2001GL014649},
volume = {29},
year = {2002}
}
@article{Dominguez-Garcia2016,
abstract = {Food webs -- networks of predators and prey -- have long been known to exhibit "intervality": species can generally be ordered along a single axis in such a way that the prey of any given predator tend to lie on unbroken compact intervals. Although the meaning of this axis -- identified with a "niche" dimension -- has remained a mystery, it is assumed to lie at the basis of the highly non-trivial structure of food webs. With this in mind, most trophic network modelling has for decades been based on assigning species a niche value by hand. However, we argue here that intervality should not be considered the cause but rather a consequence of food-web structure. First, analysing a set of {\$}46{\$} empirical food webs, we find that they also exhibit {\{}$\backslash$it predator{\}} intervality: the predators of any given species are as likely to be contiguous as the prey are, but in a different ordering. Furthermore, this property is not exclusive of trophic networks: several networks of genes, neurons, metabolites, cellular machines, airports, and words are found to be approximately as interval as food webs. We go on to show that a simple model of food-web assembly which does not make use of a niche axis can nevertheless generate significant intervality. Therefore, the niche dimension (in the sense used for food-web modelling) could in fact be the consequence of other, more fundamental structural traits, such as trophic coherence. We conclude that a new approach to food-web modelling is required for a deeper understanding of ecosystem assembly, structure and function, and propose that certain topological features thought to be specific of food webs are in fact common to many complex networks.},
archivePrefix = {arXiv},
arxivId = {1603.03767},
author = {Dom{\'{i}}nguez-Garc{\'{i}}a, Virginia and Johnson, Samuel and Mu{\~{n}}oz, Miguel A.},
doi = {10.1063/1.4953163},
eprint = {1603.03767},
isbn = {9781424483433},
issn = {10541500},
journal = {Chaos},
number = {6},
title = {{Intervality and coherence in complex networks}},
volume = {26},
year = {2016}
}
@article{Hambright1991,
abstract = {—Piscivorous fish are size-selective predators. Although sizes of prey selectively in-gested by piscivores traditionally have been measured in terms of prey length relative to predator length, the relationship between prey body depth (measured dorsoventrally) and piscivore mouth gape may be a more appropriate measure of prey size selection. In 2-d feeding trials with three sizes of largemouth bass Micropterus salmoides, I offered various sizes of shallow-bodied fathead minnows Pimephales promelas and deep-bodied pumpkinseeds Lepomis gibbosus in assemblages of one or both species. All sizes of predators preferred pumpkinseeds with body depths well below the maximum size ingestible. Small predators also preferred fathead minnows with body depths below the maximum size ingestible, whereas intermediate and large predators selectively ingested the largest fathead minnows offered. Largemouth bass never ingested prey of body depth greater than their own external mouth width. Although lengths of selectively ingested fathead minnows and pumpkinseeds differed, largemouth bass showed highest preferences for prey of similar body depths regardless of taxonomic identity. These results suggest that, in addition to setting constraints on maximum sizes of prey that can be ingested by piscivores, the relationship between prey body depth and piscivore mouth gape may also be important in selection of prey within the range of ingestible sizes. Therefore, body depth may be more useful than the traditional measure of prey length as a common measure for examining prey selection by gape-limited piscivores over a wide array of prey species.},
author = {Hambright, K. David},
doi = {10.1577/1548-8659(1991)120<0500:EAOPSB>2.3.CO;2},
isbn = {0002-8487},
issn = {0002-8487},
journal = {Transactions of the American Fisheries Society},
number = {4},
pages = {500--508},
pmid = {2319},
title = {{Experimental Analysis of Prey Selection by Largemouth Bass: Role of Predator Mouth Width and Prey Body Depth}},
url = {http://dx.doi.org/10.1577/1548-8659(1991)120{\%}3C0500:EAOPSB{\%}3E2.3.CO;2},
volume = {120},
year = {1991}
}
@article{Nowlin2006a,
abstract = {Top–down control of phytoplankton biomass through piscivorous fish manipulation has been explored in numerous ecological and biomanipulation experiments. Piscivores are gape-limited predators and it is hypothesized that the distribution of gape sizes relative to distribution of body depths of prey fish may restrict piscivore effects cascading to plankton. We examined the top–down effects of piscivorous largemouth bass on nutrients, turbidity, phytoplankton, zooplankton and fish in ponds containing fish assemblages with species representing a range of body sizes and feeding habits (western mosquitofish, bluegill, channel catfish, gizzard shad and common carp). The experimental design consisted of three replicated treatments: fishless ponds (NF), fish community without largemouth bass (FC), and fish community with largemouth bass (FCB). Turbidity, chlorophyll a, cyclopoid copepodid and copepod nauplii densities were significantly greater in FC and FCB ponds than in NF ponds. However, these response variables were not significantly different in FC and FCB ponds. The biomass and density of shallow-bodied western mosquitofish were reduced and bluegill body depths shifted toward larger size classes in the presence of largemouth bass, but the biomass and density of all other fish species and of the total fish community were unaffected by the presence of largemouth bass. Our results show that top–down impacts of largemouth bass in ecosystems containing small- and deep-bodied fish species may be most intense at the top of the food web and alter the size distribution and species composition of the fish community. However, these top–down effects may not cascade to the level of the plankton when large-bodied benthivorous fish species are abundant.},
author = {Nowlin, Weston H. and Drenner, Ray W. and Guckenberger, Kirk R. and Lauden, Mark A. and Alonso, G. Todd and Fennell, Joseph E. and Smith, Judson L.},
doi = {10.1007/s10750-006-0024-4},
isbn = {0018-8158},
issn = {00188158},
journal = {Hydrobiologia},
keywords = {Benthivorous fish,Largemouth bass,Piscivore,Prey refuge,Trophic cascade},
number = {1},
pages = {357--369},
title = {{Gape limitation, prey size refuges and the top-down impacts of piscivorous largemouth bass in shallow pond ecosystems}},
volume = {563},
year = {2006}
}
@article{DeMott1982,
abstract = {The feeding selectivities and feeding rates of Daphnia rosea and Bosmina longirostris were measured in mixtures of ' " C-labeled algae (Chlamydomonas reinhardi) and 311-labelcd bacteria (Aerohacter aerogenes). Daphnia showed no preference for one over the other. Se-lectivity coefficients (algal clearance rate : bacterial clearance rate) for Bosmina ranged from 2.8 to 13.7 depending on both previous feeding history and th{\&} relative abundance of the two foods. At high concentrations of Chlamydomonas alone the ingestion rates of the two cladoc-erans per unit body weight were not significantly different, while at low food concentrations (500-10,000 cells. ml--') th c relative ingestion rate of Bosmina was 4.8 to 1.6{\~{}} higher than that of Daphnia. Differences in feeding behavior arc' attributed to interspecific differences in morphology and feeding mode. The ability of Bosmina to feed efficiently at low food concentrations and to feed selectively may invalidate inferences about zooplankton compc-tition from theoretical models of optimal body size alone.},
author = {DeMott, William R.},
doi = {10.4319/lo.1982.27.3.0518},
isbn = {0024-3590},
issn = {19395590},
journal = {Limnology and Oceanography},
number = {3},
pages = {518--527},
pmid = {736},
title = {{Feeding selectivities and relative ingestion rates of Daphnia and Bosmina}},
volume = {27},
year = {1982}
}
@article{Fauchald1979,
abstract = {A sequential injection system which consists of a syringe pump, a selector valve, a multi-port valve, a gas-liquid separator and a solenoid valve for the determination of arsenic by hydride generation atomic absorption spectrometry using tetrahydroborate as reductant was developed. The reduction time of sample with tetrahydoborate has increased by keeping the reactant in gas-liquid separator by using the solenoid valve. Various parameters affecting the performance of the sequential injection system were optimized, including reaction-time, carrier gas flow, sample volume, tetrahydroborate volume and concentration. Established sequential injection hydride generation technique was simple and automated operation. A sample throughput of 112/h was achieved with 400 $\mu$L samples with a precision of 2.0{\%} RSD at 4 $\mu$g/L As (n = 10) and a detection limit of 0.09 $\mu$g/L. Good agreement with the certified values was obtained for the determination of arsenic in standard reference materials.},
archivePrefix = {arXiv},
arxivId = {arXiv:1011.1669v3},
author = {Yin, Xuefeng and Zhang, Jianjun and Wang, Xiaofang},
doi = {10.1017/CBO9781107415324.004},
eprint = {arXiv:1011.1669v3},
isbn = {9788578110796},
issn = {02533820},
journal = {Fenxi Huaxue},
keywords = {Arsenic,Hydride generation atomic absorption spectrometry,Sequential injection},
number = {10},
pages = {1365--1367},
pmid = {25246403},
title = {{Sequential injection analysis system for the determination of arsenic by hydride generation atomic absorption spectrometry}},
url = {https://www.researchgate.net/profile/Peter{\_}Jumars/publication/255608624{\_}The{\_}Diet{\_}of{\_}Worms{\_}A{\_}Study{\_}of{\_}Polychaete{\_}Feeding{\_}Guilds/links/02e7e5371f8f32da46000000.pdf},
volume = {32},
year = {2004}
}
@article{Kaspari1990,
abstract = {The Erasmus Experiment Archive is an electronic database, accessible via the internet, which collects in single reference repository information regarding all experiments performed in the facilities of the ESA Directorate of Human Spaceflight, Microgravity and Exploration. The archive was developed and is maintained and kept up-to-date by the Erasmus User Centre, which forms part of the Directorate and is located at ESTEC in Noordwijk (NL). The archive addresses the scientists interested in utilizing experiment facilities for their own research activities. It also addresses project engineers and managers involved in developing, building or operating similar facilities.},
author = {Isakeit, Dieter and Sabbatini, Massimo and Carey, William},
issn = {03764265},
journal = {European Space Agency Bulletin},
number = {120},
pages = {34--39},
title = {{Sharing ESA's knowledge and experience - The erasmus experiment archive}},
volume = {40},
year = {2004}
}
@article{Costa2014,
abstract = {Based on geographical and home range sizes, physiology, and gape limitation, a positive relationship between predator size and diet breadth is expected. Alternatively, larger predators might avoid smaller prey; in this case no relationship would be found. Here, I used a large data set on the diets of marine predators to describe and identify mechanisms responsible for the relationships among predator body size, diet breadth, and the mean, minimum, maximum, and variance of prey size. I found no relationship between predator size and diet breadth. Mean, minimum, maximum, and variance of prey size were all positively associated with predator size. I found that larger predators increase their minimum and maximum prey size with similar slopes, which explains the lack of relationship between predator size and diet breadth. The results support predictions of the hypothesis that optimal foraging is the main factor constraining the shape of the relationships among predator size, prey size, and diet breadth. Future research should focus on examining the relationship between body size and the breadth of different niche axis across different groups of organisms to assess whether a positive relationship between body size and niche breadth is a general rule in macroecology.},
author = {Costa, Gabriel C and Costa, Gabriel C},
doi = {doi:10.1890/08-1150.1},
journal = {Ecology},
keywords = {also,and consume both small,and large,body size,brown and maurer 1989,can detect,capture,diet,gain,large predators,macroecology,marine predators,niche breadth,optimal foraging,phylogenetic contrast,predator,prey size},
number = {7},
pages = {2014--2019},
title = {{Predator Size , Prey Size , and Dietary Niche Breadth Relationships in Marine Predators}},
volume = {90},
year = {2016}
}
@article{Wirta2014,
abstract = {How networks of ecological interactions are structured has a major impact on their functioning. However, accurately resolving both the nodes of the webs and the links between them is fraught with difficulties. We ask whether the new resolution conferred by molecular information changes perceptions of network structure. To probe a network of antagonistic interactions in the High Arctic, we use two complementary sources of molecular data: parasitoid DNA sequenced from the tissues of their hosts and host DNA sequenced from the gut of adult parasitoids. The information added by molecular analysis radically changes the properties of interaction structure. Overall, three times as many interaction types were revealed by combining molecular information from parasitoids and hosts with rearing data, versus rearing data alone. At the species level, our results alter the perceived host specificity of parasitoids, the parasitoid load of host species, and the web-wide role of predators with a cryptic lifestyle. As the northernmost network of host-parasitoid interactions quantified, our data point exerts high leverage on global comparisons of food web structure. However, how we view its structure will depend on what information we use: compared with variation among networks quantified at other sites, the properties of our web vary as much or much more depending on the techniques used to reconstruct it. We thus urge ecologists to combine multiple pieces of evidence in assessing the structure of interaction webs, and suggest that current perceptions of interaction structure may be strongly affected by the methods used to construct them.},
archivePrefix = {arXiv},
arxivId = {arXiv:1408.1149},
author = {Wirta, Helena K. and Hebert, Paul D. N. and Kaartinen, Riikka and Prosser, Sean W. and V{\'{a}}rkonyi, Gergely and Roslin, Tomas},
doi = {10.1073/pnas.1316990111},
eprint = {arXiv:1408.1149},
file = {:Users/alyssacirtwill/Documents/Papers/Wirta et al.{\_}2014{\_}Proceedings of the National Academy of Sciences.pdf:pdf},
isbn = {0027-8424},
issn = {0027-8424},
journal = {Proceedings of the National Academy of Sciences},
number = {5},
pages = {1885--1890},
pmid = {24449902},
title = {{Complementary molecular information changes our perception of food web structure}},
url = {http://www.pnas.org/lookup/doi/10.1073/pnas.1316990111},
volume = {111},
year = {2014}
}
@article{Huxel1998,
abstract = {ABSTRACT In nature, fluxes across habitats often bring both nutrient and energetic resources into areas of low productivity from areas of higher productivity. These inputs can alter consumption rates of consumer and predator species in the recipient food webs, thereby influencing food web stability. Starting from a well‐studied tritrophic food chain model, we investigated the impact of allochthonous inputs on the stability of a simple food web model. We considered the effects of allochthonous inputs on stability of the model using four sets of biologically plausible parameters that represent different dynamical outcomes. We found that low levels of allochthonous inputs stabilize food web dynamics when species preferentially feed on the autochthonous sources, while either increasing the input level or changing the feeding preference to favor allochthonous inputs, or both, led to a decoupling of the food chain that could result in the loss of one or all species. We argue that allochthonous inputs are important sources of productivity in many food webs and their influence needs to be studied further. This is especially important in the various systems, such as caves, headwater streams, and some small marine islands, in which more energy enters the food web from allochthonous inputs than from autochthonous inputs.},
author = {Huxel, Gary R. and McCann, Kevin},
doi = {10.1086/286182},
isbn = {00030147},
issn = {0003-0147},
journal = {The American Naturalist},
keywords = {allochthonous inputs,energetic tri-,factors that impinge on,food web structure is,food webs,greatly influenced by a,number of,polis and,population densities,stability,trophic cascades,trophic model},
number = {3},
pages = {460--469},
pmid = {18811452},
title = {{Food Web Stability: The Influence of Trophic Flows across Habitats}},
url = {http://www.journals.uchicago.edu/doi/10.1086/286182},
volume = {152},
year = {1998}
}
@article{Bascompte2005a,
abstract = {The stability of ecological communities largely depends on the strength of interactions between predators and their prey. Here we show that these interaction strengths are structured nonran- domly in a large Caribbean marine food web. Specifically, the cooccurrence of strong interactions on two consecutive levels of food chains occurs less frequently than expected by chance. Even when they occur, these strongly interacting chains are accompa- nied by strong omnivory more often than expected by chance. By using a food web model, we show that these interaction strength combinations reduce the likelihood of trophic cascades after the overfishing of top predators. However, fishing selectively removes predators that are overrepresented in strongly interacting chains. Hence, the potential for strong community-wide effects remains a threat},
archivePrefix = {arXiv},
arxivId = {arXiv:1011.1669v3},
author = {Bascompte, J. and Melian, C. J. and Sala, E.},
doi = {10.1073/pnas.0501562102},
eprint = {arXiv:1011.1669v3},
isbn = {0027-8424},
issn = {0027-8424},
journal = {Proceedings of the National Academy of Sciences},
number = {15},
pages = {5443--5447},
pmid = {15802468},
title = {{Interaction strength combinations and the overfishing of a marine food web}},
url = {http://www.pnas.org/cgi/doi/10.1073/pnas.0501562102},
volume = {102},
year = {2005}
}
@article{Brown1989,
author = {Brown, James H and Maurer, Brian A},
journal = {Science},
number = {4895},
pages = {1145--1150},
title = {{Macroecology: The division of food and space amor species on continents}},
volume = {243},
year = {2007}
}
@article{Nowlin2006,
abstract = {Top–down control of phytoplankton biomass through piscivorous fish manipulation has been explored in numerous ecological and biomanipulation experiments. Piscivores are gape-limited predators and it is hypothesized that the distribution of gape sizes relative to distribution of body depths of prey fish may restrict piscivore effects cascading to plankton. We examined the top–down effects of piscivorous largemouth bass on nutrients, turbidity, phytoplankton, zooplankton and fish in ponds containing fish assemblages with species representing a range of body sizes and feeding habits (western mosquitofish, bluegill, channel catfish, gizzard shad and common carp). The experimental design consisted of three replicated treatments: fishless ponds (NF), fish community without largemouth bass (FC), and fish community with largemouth bass (FCB). Turbidity, chlorophyll a, cyclopoid copepodid and copepod nauplii densities were significantly greater in FC and FCB ponds than in NF ponds. However, these response variables were not significantly different in FC and FCB ponds. The biomass and density of shallow-bodied western mosquitofish were reduced and bluegill body depths shifted toward larger size classes in the presence of largemouth bass, but the biomass and density of all other fish species and of the total fish community were unaffected by the presence of largemouth bass. Our results show that top–down impacts of largemouth bass in ecosystems containing small- and deep-bodied fish species may be most intense at the top of the food web and alter the size distribution and species composition of the fish community. However, these top–down effects may not cascade to the level of the plankton when large-bodied benthivorous fish species are abundant.},
author = {Nowlin, Weston H. and Drenner, Ray W. and Guckenberger, Kirk R. and Lauden, Mark A. and Alonso, G. Todd and Fennell, Joseph E. and Smith, Judson L.},
doi = {10.1007/s10750-006-0024-4},
isbn = {0018-8158},
issn = {00188158},
journal = {Hydrobiologia},
keywords = {Benthivorous fish,Largemouth bass,Piscivore,Prey refuge,Trophic cascade},
number = {1},
pages = {357--369},
title = {{Gape limitation, prey size refuges and the top-down impacts of piscivorous largemouth bass in shallow pond ecosystems}},
volume = {563},
year = {2006}
}
@article{Thompson2000,
abstract = {From a behavioral as well as neurobiological point of view, sleep and consciousness are intimately connected. A better understanding of sleep cycles and sleep architecture of patients suffering from disorders of consciousness (DOC) might therefore improve the clinical care for these patients as well as our understanding of the neural correlations of consciousness. Defining sleep in severely brain-injured patients is however problematic as both their electrophysiological and sleep patterns differ in many ways from healthy individuals. This paper discusses the concepts involved in the study of sleep of patients suffering from DOC and critically assesses the applicability of standard sleep criteria in these patients. The available literature on comatose and vegetative states as well as that on locked-in and related states following traumatic or non-traumatic severe brain injury will be reviewed. A wide spectrum of sleep disturbances ranging from almost normal patterns to severe loss and architecture disorganization are reported in cases of DOC and some patterns correlate with diagnosis and prognosis. At the present time the interactions of sleep and consciousness in brain-injured patients are a little studied subject but, the authors suggest, a potentially very interesting field of research. ?? 2009 Elsevier Ltd. All rights reserved.},
author = {Thompson, Ross M. and Townsend, Colin R.},
doi = {10.1046/j.1365-2427.2000.00579.x},
isbn = {1532-2955 (Electronic)$\backslash$n1087-0792 (Linking)},
issn = {00465070},
journal = {Freshwater Biology},
keywords = {Connectance,Food chain length,Linkage complexity,Macroinvertebrates,Stream algae},
number = {3},
pages = {413--422},
pmid = {19524464},
title = {{Is resolution the solution?: The effect of taxonomic resolution on the calculated properties of three stream food webs}},
volume = {44},
year = {2000}
}
@article{Newsome2012,
abstract = {JSTOR is a not-for-profit service that helps scholars, researchers, and students discover, use, and build upon a wide range of content in a trusted digital archive. We use information technology and tools to increase productivity and facilitate new forms of scholarship. For more information about JSTOR, please contact support@jstor.org. Stable isotope analysis of fossil materials has become an increasingly important method for gathering dietary and environmental information from extinct species in terrestrial and aquatic ecosystems. The benefits of these analyses stem from the geochemical fingerprint that an animal's environment leaves in its bones, teeth, and tissues. Ongoing study of living mammals has found the stable isotopie composition of several light (hydrogen, carbon, nitrogen, oxygen, and sulfur) and even a few heavy (calcium and strontium) elements to be useful tracers of ecological and physiological information; many of these can be similarly applied to the study of fossil mammals. For instance, the carbon isotopie composition of an animal's tissues tracks that of the food it eats, whereas the oxygen isotopie compositions of the carbonate and phosphate in an animal's bones and teeth are primarily controlled by that of the surface water it drinks or the water in the food it ingests. These stable isotope proxies for diet and habitat information are independent of inferences based on morphological characters and thus provide a means of testing ecological interpretations drawn from the fossil record. As such, when well-preserved specimens are available, any dietary study of fossil species should seriously consider including this approach. To illustrate the potential benefits associated with applying these methods to paleontological research, a review of current work on the ecological and evolutionary history of fossil mammals through geochemical analysis is presented. After a brief introduction to issues associated with the preservation of stable isotopie information in soft and mineralized tissues, a series of case studies involving the application of stable isotope analysis to fossil mammal research is discussed. These studies were selected to highlight the versatility of this analytical method to paleontological research and are complemented by a discussion of new techniques and instrumentation in stable isotope analysis (e.g., laser ablation and compound-specific isotope ratio mass spectrometry, and calcium and clumped isotopes), which represent the latest advances in the extension of these geochemical tools to the paleontology of fossil mammals. stable isotopie composition of vertebrate fossil remains, paleon-tologists gained a valuable tool for studying fossil mammals from ancient marine and terrestrial communities. Because di-rect observation of extinct species within a community is not possible, stable isotope analysis has become an increasingly important tool for paleontologists interested in the paleoecology of ancient mammals (Cerling et al. 1997; Clementz et al. 2003b; Hoppe et al. 1999; MacFadden et al. 2004). Prior to the initial application of this analytical tool to archaeological (Van der Merwe and Vogel 1978; Vogel and Van der Merwe 1977) and subsequently paleontological (DeNiro and Epstein 1978; Ericson et al. 1981; Schoeninger and DeNiro 1982a, 1982b) research in the late 1970s and early 1980s, ecological inter-pretations of fossil mammals were primarily restricted to interpretations based on either examination of the morphology of the specimens or careful study of sedimentary environ-ments in which fossils were deposited. Because morphological},
author = {Clementz, Mark T},
doi = {10.1644/1},
isbn = {1545-1542},
issn = {0022-2372},
journal = {Source: Journal of Mammalogy Journal of Mammalogy This},
keywords = {bioapatite,calcium isotopes,collagen,migration,paleodietary reconstruction,strontium isotopes},
number = {12},
pages = {368--38039},
title = {{American Society of Mammalogists New insight from old bones: stable isotope analysis of fossil mammals New insight from old bones: stable isotope analysis of fossil mammals With the discovery of measurable natural variation in the}},
url = {http://www.jstor.org/stable/41480347{\%}0Ahttp://www.jstor.org/stable/41480347?seq=1{\&}cid=pdf-reference{\#}references{\_}tab{\_}contents{\%}0Ahttp://about.jstor.org/terms},
volume = {9317},
year = {2012}
}
@article{Phillips2003,
abstract = {{\textless}sec{\textgreater} {\textless}title{\textgreater}Background{\textless}/title{\textgreater} {\textless}p{\textgreater}Stable isotope analysis is increasingly being utilised across broad areas of ecology and biology. Key to much of this work is the use of mixing models to estimate the proportion of sources contributing to a mixture such as in diet estimation.{\textless}/p{\textgreater} {\textless}/sec{\textgreater}{\textless}sec{\textgreater} {\textless}title{\textgreater}Methodology{\textless}/title{\textgreater} {\textless}p{\textgreater}By accurately reflecting natural variation and uncertainty to generate robust probability estimates of source proportions, the application of Bayesian methods to stable isotope mixing models promises to enable researchers to address an array of new questions, and approach current questions with greater insight and honesty.{\textless}/p{\textgreater} {\textless}/sec{\textgreater}{\textless}sec{\textgreater} {\textless}title{\textgreater}Conclusions{\textless}/title{\textgreater} {\textless}p{\textgreater}We outline a framework that builds on recently published Bayesian isotopic mixing models and present a new open source R package, SIAR. The formulation in R will allow for continued and rapid development of this core model into an all-encompassing single analysis suite for stable isotope research.{\textless}/p{\textgreater} {\textless}/sec{\textgreater}},
archivePrefix = {arXiv},
arxivId = {962},
author = {Phillips, Donald L. and Gregg, Jillian W.},
doi = {10.1007/s00442-003-1218-3},
eprint = {962},
isbn = {1932-6203},
issn = {00298549},
journal = {Oecologia},
keywords = {Mixing model,Source partitioning,Stable isotope},
number = {2},
pages = {261--269},
pmid = {20300637},
title = {{Source partitioning using stable isotopes: Coping with too many sources}},
volume = {136},
year = {2003}
}
@article{Halley2016,
abstract = {Species extinction following habitat loss is well documented. However, these extinctions do not happen immediately. The biodiversity surplus (extinction debt) declines with some delay through the process of relaxation. Estimating the time constants of relaxation, mainly the expected time to first extinction and the commonly used time for half the extinction debt to be paid off (half-life), is crucial for conservation purposes. Currently, there is no agreement on the rate of relaxation and the factors that it depends on. Here we find that half-life increases with area for all groups examined in a large meta-analysis of extinction data. A common pattern emerges if we use average number of individuals per species before habitat loss as an area index: for mammals, birds, reptiles and plants, the relationship has an exponent close to a half. We also find that the time to first determined extinction is short and increases slowly with area.},
author = {Halley, John M. and Monokrousos, Nikolaos and Mazaris, Antonios D. and Newmark, William D. and Vokou, Despoina},
doi = {10.1038/ncomms12283},
issn = {20411723},
journal = {Nature Communications},
pages = {12283},
publisher = {Nature Publishing Group},
title = {{Dynamics of extinction debt across five taxonomic groups}},
url = {http://www.nature.com/doifinder/10.1038/ncomms12283},
volume = {7},
year = {2016}
}
@article{Klaise2016,
abstract = {Food webs have been found to exhibit remarkable motif profiles, patterns in the relative prevalences of all possible three-species sub-graphs, and this has been related to ecosystem properties such as stability and robustness. Analysing 46 food webs of various kinds, we find that most food webs fall into one of two distinct motif families. The separation between the families is well predicted by a global measure of hierarchical order in directed networks - trophic coherence. We find that trophic coherence is also a good predictor for the extent of omnivory, defined as the tendency of species to feed on multiple trophic levels. We compare our results to a network assembly model that admits tunable trophic coherence via a single free parameter. The model is able to generate food webs in either of the two families by varying this parameter, and correctly classifies almost all the food webs in our database. This establishes a link between global order and local preying patterns in food webs.},
archivePrefix = {arXiv},
arxivId = {1609.04318},
author = {Klaise, Janis and Johnson, Samuel},
doi = {10.1038/s41598-017-15496-1},
eprint = {1609.04318},
isbn = {0217-2445},
issn = {20452322},
journal = {Scientific Reports},
number = {1},
pages = {24--26},
title = {{The origin of motif families in food webs}},
url = {http://arxiv.org/abs/1609.04318},
volume = {7},
year = {2017}
}
@article{Leibold1998,
abstract = {The notion of 'community-wide character displacement' hypothesizes that locally co-existing sets of competing species should be less similar than expected when compared to random expectations from a broader regional species pool. Here I use a mechanistic approach to the niche concept to show how this expectation is dependent on the types of traits involved. I investigate how two different niche components, those that relate to species' requirements (or responses to environmental factors) versus those that relate to species' impacts (or effects on environmental factors), affect predictions about the similarity of locally co-existing species. In contrast with more conventional approaches that focus on species impacts, I focus on species responses to conclude that locally co-existing species should be more similar in such traits than expected on the basis of random asortment from a larger equilibrium regional biota. In addition, I explore the evolutionary implications of exceptions that might favour the co-existence of species with dissimilar traits (especially those that determine species' impacts on the environment) and conclude that these implications differ when species compete for shared resources, interact via shared predators, or interact via both mechanisms. The analysis developed in this paper shows that the co-existence of species that are more similar than expected by chance is not incompatible with the notion of strongly interacting species in saturated local communities near equilibrium.},
author = {Leibold, Mathew A.},
doi = {10.1023/A:1006511124428},
isbn = {0269-7653},
issn = {02697653},
journal = {Evolutionary Ecology},
keywords = {Apparent competition,Character displacement,Co-existence,Keystone predator,Limiting similarity,Regional biotas,Resource competition},
number = {1},
pages = {95--110},
pmid = {3902237},
title = {{Similarity and local co-existence of species in regional biotas}},
volume = {12},
year = {1998}
}
@article{Pimm1991,
abstract = {A food web is a map that describes which kinds of organisms in a community eat which other kinds. A web helps picture how a community is put together and how it works. Although webs were often initially reported in despair at ever understanding ecological complexity, recently discovered widespread patterns in the shapes of webs, and theoretical explanations for these patterns, indicate that webs are orderly and intelligible, and have some foreseeable consequences for the dynamics of communities.},
author = {Pimm, Stuart L. and Lawton, John H. and Cohen, Joel E.},
doi = {10.1038/350669a0},
isbn = {0028-0836},
issn = {00280836},
journal = {Nature},
number = {6320},
pages = {669--674},
pmid = {1834949},
title = {{Food web patterns and their consequences}},
url = {http://www.nature.com/doifinder/10.1038/350669a0},
volume = {350},
year = {1991}
}
@article{Morris2004,
abstract = {The herbivorous insects of tropical forests constitute some of the most diverse communities of living organisms. For this reason it has been difficult to discover the degree to which these communities are structured, and by what processes. Interspecific competition for resources does occur, but its contemporary importance is limited because most pairs of potentially competing insects feed on different host plants. An alternative way in which species can interact is through shared natural enemies, a process called apparent competition. Despite extensive theoretical discussion there are few field demonstrations of apparent competition, and none in hyper-diverse tropical communities. Here, we experimentally removed two species of herbivore from a community of leaf-mining insects in a tropical forest. We predicted that other species that share natural enemies with the two removed species would experience lower parasitism and have higher population densities in treatment compared with control sites. In both cases (on removal of a dipteran and a coleopteran leaf-miner species) we found significantly lower parasitism, and in one case (removal of the dipteran) we found significantly higher abundance a year after the manipulation. Our results suggest that apparent competition may be important in structuring tropical insect communities.},
author = {Morris, Rebecca J. and Lewis, Owen T. and Godfray, H. Charles J.},
doi = {10.1038/nature02394},
isbn = {0028-0836},
issn = {00280836},
journal = {Nature},
number = {6980},
pages = {310--313},
pmid = {15029194},
title = {{Experimental evidence for apparent competition in a tropical forest food web}},
url = {http://www.nature.com/doifinder/10.1038/nature02394},
volume = {428},
year = {2004}
}
@article{Hay1988,
abstract = {Herbivory has a profound effect on seaweeds in both temperate and tropical$\backslash$r$\backslash$ncommunities (II, 17,21,33,43,47,80, 124). This is especially true on coral$\backslash$r$\backslash$nreefs where 60-97{\%} (II, 42) of the total seaweed production may be removed$\backslash$r$\backslash$nby herbivores. To persist in marine communities, seaweeds must escape,$\backslash$r$\backslash$ndeter, or tolerate herbivory. The ecological and evolutionary importance of$\backslash$r$\backslash$nspatial and temporal escapes has been extensively studied for seaweeds and$\backslash$r$\backslash$nadequately reviewed in the recent literature.$\backslash$r$\backslash$nApproximately 500--600 secondary compounds have been isolated from$\backslash$r$\backslash$nmarine algae (23, 24, 123). Many of these compounds possess strong biological$\backslash$r$\backslash$nactivities in laboratory assays (3, 23, 24, 27, 45, 52, 93, 104); however,$\backslash$r$\backslash$ntheir ecological functions under natural conditions have been addressed only$\backslash$r$\backslash$nrecently. Seaweed secondary metabolites may function as herbivore deterrents$\backslash$r$\backslash$n(see below) or as allelopathic and antifouling agents (57, 85, 1 28).$\backslash$r$\backslash$nRecent investigations (52, 1 16) suggest that the primary function of many$\backslash$r$\backslash$nof these compounds is to deter herbivory. However, secondary metabolites$\backslash$r$\backslash$nalso may have multiple or alternate functions. Below we review the available$\backslash$r$\backslash$ninformation on chemically mediated seaweed-herbivore interactions ,$\backslash$r$\backslash$ncompare their effects in marine vs terrestrial communities, and interpret the$\backslash$r$\backslash$n.robustness of present plant-herbivore theory in light of the emerging marine$\backslash$r$\backslash$npatterns. $\backslash$r$\backslash$n},
archivePrefix = {arXiv},
arxivId = {arXiv:1512.03704v2},
author = {Hay, M E and Fenical, W},
doi = {10.1146/annurev.es.19.110188.000551},
eprint = {arXiv:1512.03704v2},
isbn = {0066-4162},
issn = {0066-4162},
journal = {Annual Review of Ecology and Systematics},
number = {1},
pages = {111--145},
pmid = {745},
title = {{Marine Plant-Herbivore Interactions: The Ecology of Chemical Defense}},
url = {http://www.annualreviews.org/doi/10.1146/annurev.es.19.110188.000551},
volume = {19},
year = {1988}
}
@article{Prill2005,
abstract = {Biological networks, such as those describing gene regulation, signal transduction, and neural synapses, are representations of large-scale dynamic systems. Discovery of organizing principles of biological networks can be enhanced by embracing the notion that there is a deep interplay between network structure and system dynamics. Recently, many structural characteristics of these non-random networks have been identified, but dynamical implications of the features have not been explored comprehensively. We demonstrate by exhaustive computational analysis that a dynamical property--stability or robustness to small perturbations--is highly correlated with the relative abundance of small subnetworks (network motifs) in several previously determined biological networks. We propose that robust dynamical stability is an influential property that can determine the non-random structure of biological networks.},
author = {Prill, Robert J. and Iglesias, Pablo A. and Levchenko, Andre},
doi = {10.1371/journal.pbio.0030343},
isbn = {1545-7885 (Linking)},
issn = {15449173},
journal = {PLoS Biology},
number = {11},
pages = {1881--1892},
pmid = {16187794},
title = {{Dynamic properties of network motifs contribute to biological network organization}},
volume = {3},
year = {2005}
}
@article{Allesina2008a,
abstract = {Large, complex networks of ecological interactions with random structure tend invariably to instability. This mathematical relationship between complexity and local stability ignited a debate that has populated ecological literature for more than three decades. Here we show that, when species interact as predators and prey, systems as complex as the ones observed in nature can still be stable. Moreover, stability is highly robust to perturbations of interaction strength, and is largely a property of structure driven by predator–prey loops with the stability of these small modules cascading into that of the whole network. These results apply to empirical food webs and models that mimic the structure of natural systems as well. These findings are also robust to the inclusion of other types of ecological links, such as mutualism and interference competition, as long as consumer–resource interactions predominate. These considerations underscore the influence of food web structure on ecological dynamics and challenge the current view of interaction strength and long cycles as main drivers of stability in natural communities.$\backslash$r$\backslash$n$\backslash$r$\backslash$nKeywo},
author = {Allesina, Stefano and Pascual, Mercedes},
doi = {10.1007/s12080-007-0007-8},
isbn = {1874-1738},
issn = {18741738},
journal = {Theoretical Ecology},
keywords = {Complexity/stability,Food webs,Predator - prey,Sign-stability,Weak interactions},
number = {1},
pages = {55--64},
pmid = {20003464},
title = {{Network structure, predator - Prey modules, and stability in large food webs}},
volume = {1},
year = {2008}
}
@article{Schmitz1997,
abstract = {Press perturbations, in which one or more species densities are experimentally altered and held at higher or lower levels, are common field approaches used to understand community dynamics. The outcomes of such experiments are often difficult to anticipate solely on the basis of intuition. This is because the effects of a perturbation may pass through a complex network of direct and indirect pathways in a food web, and the outcome may be highly sensitive to the strength of interactions among species. One solution to understanding outcomes of press experiments is to quantify first the community matrix, the matrix of measured direct interactions between all species in a food web, and then obtain the inverse of this matrix. The inverse of the community matrix predicts the effect of all species presses on all other species. I evaluated the utility of the inverse community matrix in predicting the outcomes of press experiments in an old-field food web. I used data from field and laboratory experiments to quantify the interaction strengths between grasshoppers, four old-field plants, and nitrogen supply. These values were used to parameterize the community matrix and obtain its inverse in a Monte Carlo simulation. The simulation was used to predict the mean and standard error in the outcome of a simultaneous nitrogen and herbivore press on food-web structure and dynamics. The predictions were compared with data from an enclosure experiment in the field in which I manipulated nitrogen supply and herbivore abundance. There was a high degree of uncertainty predicted and observed in the study system. Despite this, I show that the degree and the sources of uncertainty were predictable for each species. This suggests that the inverse community matrix offers a useful theoretical benchmark for understanding the outcome of field press experiments.},
author = {Schmitz, Oswald J.},
doi = {10.1890/0012-9658(1997)078{[}0055:PPATPO]2.0.CO;2},
isbn = {0012-9658},
issn = {00129658},
journal = {Ecology},
keywords = {Community matrix and inverse,Ecological interactions, predictability of,Field experiment,Food-web dynamics,Grasses, grasshoppers, and nitrogen,Manipulative experiments,Old-field food web,Parry Sound, Ontario, Canada,Press perturbation experiment},
number = {1},
pages = {55--69},
pmid = {1013},
title = {{Press perturbations and the predictability of ecological interactions in a food web}},
volume = {78},
year = {1997}
}
@article{DunneDryad,
abstract = {Comparative research on food web structure has revealed generalities in trophic organization, produced simple models, and allowed assessment of robustness to species loss. These studies have mostly focused on free-living species. Recent research has suggested that inclusion of parasites alters structure. We assess whether such changes in network structure result from unique roles and traits of parasites or from changes to diversity and complexity. We analyzed seven highly resolved food webs that include metazoan parasite data. Our analyses show that adding parasites usually increases link density and connectance (simple measures of complexity), particularly when including concomitant links (links from predators to parasites of their prey). However, we clarify prior claims that parasites "dominate" food web links. Although parasites can be involved in a majority of links, in most cases classic predation links outnumber classic parasitism links. Regarding network structure, observed changes in degree distributions, 14 commonly studied metrics, and link probabilities are consistent with scale-dependent changes in structure associated with changes in diversity and complexity. Parasite and free-living species thus have similar effects on these aspects of structure. However, two changes point to unique roles of parasites. First, adding parasites and concomitant links strongly alters the frequency of most motifs of interactions among three taxa, reflecting parasites' roles as resources for predators of their hosts, driven by trophic intimacy with their hosts. Second, compared to free-living consumers, many parasites' feeding niches appear broader and less contiguous, which may reflect complex life cycles and small body sizes. This study provides new insights about generic versus unique impacts of parasites on food web structure, extends the generality of food web theory, gives a more rigorous framework for assessing the impact of any species on trophic organization, identifies limitations of current food web models, and provides direction for future structural and dynamical models.},
author = {Dunne, Jennifer A. and Lafferty, Kevin D. and Dobson, Andrew P. and Hechinger, Ryan F. and Kuris, Armand M. and Martinez, Neo D. and McLaughlin, John P. and Mouritsen, Kim N. and Poulin, Robert and Reise, Karsten and Stouffer, Daniel B. and Thieltges, David W. and Williams, Richard J. and Zander, Claus Dieter},
doi = {10.1371/journal.pbio.1001579},
isbn = {1545-7885},
issn = {15449173},
journal = {PLoS Biology},
number = {6},
pmid = {23776404},
title = {{Parasites Affect Food Web Structure Primarily through Increased Diversity and Complexity}},
url = {http://dx.doi.org/10.5061/dryad.b8r5c},
volume = {11},
year = {2013}
}
@article{Ramirez1992,
author = {Ramirez, N B Y},
journal = {Botanical Journal of the Linnean Society},
keywords = {-},
pages = {----},
title = {{Pollination Biology in a Palm Swamp Community in the Venezuelan Central Plains. Botanical Journal of the Linnean Society 110: 277-302p0493}},
volume = {110},
year = {1992}
}
@article{Bluthgen2006a,
abstract = {Stick insects (Phasmida) are important herbivores in tropical ecosystems, but have been poorly investigated in their natural environment. We studied phasmids and their food plants in a tropical lowland rain forest in Borneo (Danum Valley, Sabah, Malaysia). Thirty species of phasmid were collected from 49 plant species during nocturnal surveys in the forest understorey. In most cases (35 plant species), experiments confirmed that these phasmids fed on those plant species from which they were collected. Partitioning of phasmid species among food plant species was highly significant. Two common species had a largely restricted diet: Asceles margaritatus occurred mainly on Mallotus spp. (Euphorbiaceae) and Dinophasma ruficornis on Leea indica (Leeaceae). Other phasmids fed on a broad spectrum of plant families and can be considered polyphagous (e.g. Haaniella echinata, Lonchodes hosei herberti). Feeding experiments were performed on captive phasmids using leaves from eight plant species. Asceles margaritatus showed a significantly higher consumption rate for Mallotus miquelianus leaves than for other plants, while H. echinata showed the opposite trend and the lowest consumption for M. miquelianus. However, A. margaritatus readily accepted foliage from several plant families, particularly when Mallotus was not offered at the same time. Therefore, studies on host specialisation by herbivores need to include their distribution in the natural vegetation.},
author = {Bl{\"{u}}thgen, Nico and Metzner, Anika and Ruf, Daniel},
doi = {10.1017/S0266467405002749},
issn = {14697831},
journal = {Journal of Tropical Ecology},
keywords = {Asceles margaritatus,Haaniella echinata,Herbivory,Mallotus,Phasmatodea,Phasmida,Specialization,Tropical rain forest},
number = {1},
pages = {35--40},
pmid = {234740700004},
title = {{Food plant selection by stick insects (Phasmida) in a Bornean rain forest}},
volume = {22},
year = {2006}
}
@incollection{Basset1996,
address = {Amsterdam},
author = {Basset, Yves and Samuelson, G. Allan},
booktitle = {SPB Academic Publishing},
editor = {Jolivet, P. H. A. and Cox, M. L.},
pages = {243--262},
publisher = {SPB Academic Publishing},
title = {{Ecological characteristics of an arboreal community of Chrysomelidae in Papua New Guinea}},
year = {1996}
}
@article{Ueckert1971,
abstract = {Overlap in the diets of 14 species of grasshoppers on mixed-grass prairie in northeastern Colorado was estimated by microscopic examination of crop contents. Food availability estimates, taken by a weight-estimate method, facilitated the determination of food preferences. Two cases of high overlap in diets and period of habitat occupancy occurred between species which consumed equal or almost equal numbers of different foods (foodniche breadth) and had almost identical foraging behavior. The first case occurred early in the season between Xanthippus corallipes and Arphia conspersa with foodniche breadths of 14 and 11, respectively. The second case occurred later in the season between Amphitornus coloradus and Trachyrhachys kiowa both of which have foodniche breadths of 6. The amount of movement between plants while feeding was equal or almost equal in both cases, thus we observed high dietary overlap between species with equal ecological efficiencies. Dietary overlap between the 14 grasshopper species decreased as foodniche dimensions increased and the grasshopper population as a whole utilized each food resource approximately proportional to its relative abundance. Factors influencing the interpretation and importance of dietary overlap in this study included 1. non-synchronous life cycles, 2. seasonal changes in diets and food preferences, 3. complexity of the food resources, 4. variations in foodniche dimensions among grasshopper species, and 5. variations in diets between males and females within a species. Our findings agree with MacArthur and Levins theory that the number of competing species which can coexist is proportional to the total range of the environment divided by the niche breadth of the species. {\textcopyright} 1971 Springer-Verlag.},
author = {Ueckert, D. N. and Hansen, R. M.},
doi = {10.1007/BF00346475},
issn = {00298549},
journal = {Oecologia},
number = {3},
pages = {276--295},
title = {{Dietary overlap of grasshoppers on sandhill rangeland in northeastern Colorado}},
volume = {8},
year = {1971}
}
@article{Otte1976,
author = {Otte, Daniel},
journal = {Proceedings of the Academy of Natural Sciences of Philadelphia},
pages = {89--126},
title = {{On feeding patterns in desert grasshoppers and the evolution of specialized diets.}},
volume = {128},
year = {1976}
}
@article{Sheldon1978,
author = {Sheldon, J. K. and Rogers, L. E.},
doi = {10.1007/BF00344692},
issn = {00298549},
journal = {Oecologia},
number = {1},
pages = {85--92},
title = {{Grasshopper food habits within a shrub-steppe community}},
volume = {32},
year = {1978}
}
@article{Novotny2012,
abstract = {Classical niche theory explains the coexistence of species through their exploitation of different resources. Assemblages of herbivores coexisting on a particular plant species are thus expected to be dominated by species from host-specific guilds with narrow, coexistence-facilitating niches rather than by species from generalist guilds. Exactly the opposite pattern is observed for folivores feeding on trees in New Guinea. The least specialized mobile chewers were the most species rich, followed by the moderately specialized semiconcealed and exposed chewers. The highly specialized miners and mesophyll suckers were the least species-rich guilds. The Poisson distribution of herbivore species richness among plant species in specialized guilds and the absence of a negative correlation between species richness in different guilds on the same plant species suggest that these guilds are not saturated with species. We show that herbivore assemblages are enriched with generalists because these are more completely sampled from regional species pools. Herbivore diversity increases as a power function of plant diversity, and the rate of increase is inversely related to host specificity. The relative species diversity among guilds is thus scale dependent, as the importance of specialized guilds increases with plant diversity. Specialized insect guilds may therefore comprise a larger component of overall diversity in the tropics (where they are also poorly known taxonomically) than in the temperate zone, which has lower plant diversity.},
author = {Novotny, Vojtech and Miller, Scott E. and Hrcek, Jan and Baje, Leontine and Basset, Yves and Lewis, Owen T. and Stewart, Alan J. A. and Weiblen, George D.},
doi = {10.1086/664082},
isbn = {0003-0147},
issn = {0003-0147},
journal = {The American Naturalist},
keywords = {Adaptation,Animals,Biodiversity,Biological,Biological: physiology,Food Chain,Host-Parasite Interactions,Insects,Insects: physiology,Models,New Guinea,Species Specificity,Trees,Trees: parasitology},
number = {3},
pages = {351--362},
pmid = {22322223},
title = {{Insects on Plants: Explaining the Paradox of Low Diversity within Specialist Herbivore Guilds}},
url = {http://www.journals.uchicago.edu/doi/10.1086/664082},
volume = {179},
year = {2012}
}
@article{Ibanez2013,
abstract = {Despite their potential to provide a mechanistic understanding of ecosystem processes, the functional traits that govern interaction networks remain poorly understood. We investigated the extent to which biomechanical traits are related to consumption in a plantgrasshopper herbivory network. Using a choice experiment, we assessed the feeding patterns of 26 grasshopper species for 24 common plant species from subalpine grasslands. We quantified shear and punch toughness for each plant species, while grasshopper incisive and molar strengths were estimated by a lever mechanics model, following the measurement of mandibular traits. Models incorporating co-phylogenetic effects showed that the ratio between the grasshopper incisive strength and plant toughness, that reflects the cutting effort, is correlated with the mass of plant eaten. Moreover, a strong relationship between the incisive strength of the grasshoppers and the weighed mean toughness of the plants they eat was found. Our results suggest that biomechanical constraints imposed by plants influence the evolution of grasshoppers' mandibular traits. Such scaling relationships offer promising avenues towards the understanding of trait function links in interaction networks.},
author = {Ibanez, S{\'{e}}bastien and Lavorel, Sandra and Puijalon, Sara and Moretti, Marco},
doi = {10.1111/1365-2435.12058},
isbn = {0269-8463},
issn = {02698463},
journal = {Functional Ecology},
keywords = {Cafeteria experiment,Functional trait,Incisive and molar strength,Interaction network,Phylogenetic model,Shear and punch toughness},
number = {2},
pages = {479--489},
pmid = {12455678},
title = {{Herbivory mediated by coupling between biomechanical traits of plants and grasshoppers}},
volume = {27},
year = {2013}
}
@article{Coley2006,
abstract = {A survey of 85 species of Lepidoptera feeding on 40 hosts on Barro Colorado Island, Panama showed that growth and defensive traits of caterpillars were correlated with the nutritional and defensive traits of their hosts. Growth rates were faster on young than mature leaves, reflecting the higher nitrogen and water content of the former. Growth was also positively correlated with leaf expansion rate, partially because of higher nitrogen and water contents of fast-expanding young leaves. Specialists grew faster than generalists, but both responded positively to nutritional quality. There was no effect of lepidopteran family on growth. In analyses where the effects of nitrogen and water were removed, the residuals for growth rate were greater for young than for mature leaves and were positively correlated with expansion rates of young leaves. This suggests that traits other than nutrition were also important. As young, expanding leaves cannot use toughness as a defense, one possible explanation for the differences in growth is differences in chemical defenses. Growth rate residuals for both specialists and generalists were higher for the more poorly defended fast-expanders, but the effect was greatest for generalists, perhaps because generalists were more sensitive to secondary metabolites. We predicted that slow growth for caterpillars would increase their risk to natural enemies and would select for higher defenses. Generalists had more defensive traits than specialists and were less preferred in feeding trials with ants. Similarly, species feeding on mature leaves were the most defended and those feeding on fast-expanding young leaves were the least defended and most preferred by ants. Thus the effects of plant secondary metabolites and nutrients dictate herbivore growth rates, which in turn influence their susceptibility to the third trophic level and the importance of defenses},
author = {Coley, P. D. and Bateman, M. L. and Kusar, T. A.},
doi = {10.1111/j.2006.0030-1299.14928.x},
isbn = {0030-1299},
issn = {00301299},
journal = {Oikos},
number = {2},
pages = {219--228},
pmid = {241198800003},
title = {{The effects of plant quality on caterpillar growth and defense against natural enemies}},
volume = {115},
year = {2006}
}
@article{Bodner2010,
abstract = {During four months of field surveys at the Reserva Biol{\'{o}}gica San Francisco in the south Ecuadorian Andes, caterpillars of 59 Geometridae species were collected in a montane rainforest between 1800 and 2800m altitude and reared to adults. The resulting data on host plant affiliations of these species was collated. The preimaginal stages of 58 and adult stages of all 59 species are depicted in colour plates. Observations on morphology and behaviour are briefly described. Five species, documented for the first time in the study area by means of larval collections, had not been previously collected by intensive light-trap surveys. Together with published literature records, life-history data covers 8.6{\%} of the 1271 geometrid species observed so far in the study area. For 50 species these are the first records of their early stages, and for another 7 the data significantly extend known host plant ranges. Most larvae were collected on shrubs or trees, but more unusual host plant affiliations, such as ferns (6 geometrid species) and lichens (3 geometrid species), were also recorded. Thirty-four percent of the caterpillars were infested by wasp or tachinid parasitoids.},
author = {Bodner, Florian and Brehm, Gunnar and Homeier, J{\"{u}}rgen and Strutzenberger, Patrick and Fiedler, Konrad},
doi = {10.1673/031.010.6701},
isbn = {1536-2442},
issn = {1536-2442},
journal = {Journal of Insect Science},
keywords = {andes,host plant affiliations,insect herbivores,larval morphology,neotropical,pupal morphology},
number = {67},
pages = {1--22},
pmid = {20672985},
title = {{Caterpillars and Host Plant Records for 59 Species of Geometridae (Lepidoptera) from a Montane Rainforest in Southern Ecuador}},
url = {https://academic.oup.com/jinsectscience/article-lookup/doi/10.1673/031.010.6701},
volume = {10},
year = {2010}
}
@article{SmithRamirez2005,
author = {Dmitrieva, L. A. and Romanov, A. E. and Tsarev, V. N. and Ushakov, R. V. and Karnaukhov, A. T.},
issn = {00391735},
journal = {Stomatologiia},
number = {2},
pages = {26--27},
title = {{Sravnitel'naia kharakteristika antibakterial'noi aktivnosti novykh antiseptikov i perspektivy ikh primeneniia v stomatologicheskoi praktike.}},
volume = {76},
year = {1997}
}
@article{Inoue1990,
author = {Forest, Temperate Deciduous and Overview, An and Phenology, Flowering and Pattern, Seasonal and Visits, Insect},
journal = {Contributions from the Biological Laboratory, Kyoto University},
number = {August},
pages = {377--463},
title = {{Relationship in the Temperate Deciduous Forest of Kibune , Kyoto : An Overview of the Flowering Phenology}},
volume = {27},
year = {1990}
}
@article{Elberling1999,
author = {Elberling, H. and Olesen, J. M.},
journal = {Ecography},
pages = {314--323},
title = {{The structure of a high latitude plant-pollinator system: The dominance of flies}},
volume = {22},
year = {1999}
}
@article{Memmott1999,
author = {Letters, Ecology},
journal = {Ecology Letters},
keywords = {food web,generalization,pollination,pollinators,specialization},
pages = {276--280},
title = {{The structure of a plant-pollinator food web}},
volume = {2},
year = {1999}
}
@article{Peralta2014,
abstract = {Complementary resource use and redundancy of species that fulfill the same ecological role are two mechanisms that can respectively increase and stabilize process rates in ecosystems. For example, predator complementarity and redundancy can determine prey consumption rates and their stability, yet few studies take into account the multiple predator species attacking multiple prey at different rates in natural communities. Thus, it remains unclear whether these biodiversity mechanisms are important determinants of consumption in entire predator–prey assemblages, such that food-web interaction structure determines community-wide consumption and stability. Here, we use empirical quantitative food webs to study the community-wide effects of functional complementarity and redundancy of consumers (parasitoids) on herbivore control in temperate forests. We find that complementarity in host resource use by parasitoids was a strong predictor of absolute parasitism rates at the community level and that redundancy i...},
archivePrefix = {arXiv},
arxivId = {arXiv:1011.1669v3},
author = {Peralta, Guadalupe and Frost, Carol M. and Rand, Tatyana A. and Didham, Raphael K. and Tylianakis, Jason M.},
doi = {10.1890/13-1569.1},
eprint = {arXiv:1011.1669v3},
isbn = {0012-9658},
issn = {00129658},
journal = {Ecology},
keywords = {Biodiversity,Ecosystem Functioning,Food-Web Structure,Insurance Hypothesis,Niche Partitioning,Parasitism,Phylogenetic Species Variability,Spatial Variability,Temporal Variability},
number = {7},
pages = {1888--1896},
pmid = {25163121},
title = {{Complementarity and redundancy of interactions enhance attack rates and spatial stability in host-parasitoid food webs}},
volume = {95},
year = {2014}
}
@article{Small1976,
author = {Small, Ernest},
doi = {citeulike-article-id:563757},
journal = {Canadian Field Naturalist},
number = {1},
pages = {22--28},
title = {{Insect pollinators of the Mer Bleue peat bog of Ottawa}},
volume = {90},
year = {1982}
}
@phdthesis{Petanidou1991,
author = {Petanidou, Theodora},
pages = {Ph.D. dissertation},
school = {Aristotelian University, Thessaloniki},
title = {{Pollination ecology in a phryganic ecosystem}},
type = {Ph.D. thesis},
year = {1991}
}
@article{Philipp2006,
author = {Philipp, Marianne and B{\"{o}}cher, Jens and Siegismund, Hans R and Nielsen, Lene R and Philipp, Marianne and Bcher, Jens and Siegismund, Hans R},
journal = {Ecography},
number = {4},
pages = {531--540},
title = {{Structure of a Plant-Pollinator Network on a Pahoehoe Lava Desert of the Gal{\'{a}}pagos Islands network on a pahoehoe lava desert of a plant-pollinator of Structure the Galapagos Islands}},
volume = {29},
year = {2016}
}
@article{White1994,
abstract = {This paper generalizes Freeman's geodesic centrality measures for betweenness on undirected graphs to the more general directed case. Four steps are taken. The point centrality measure is first generalized for directed graphs. Second, a unique maximally centralized graph is defined for directed graphs, holding constant the numbers of points with reciprocatable (incoming and outgoing) versus only unreciprocatable (outgoing only or incoming only) arcs, and focusing the measure on the maximally central arrangement of arcs within these constraints. Alternatively, one may simply normalize on the number of arcs. This enables the third step of defining the relative betweenness centralities of a point, independent of the number of points. This normalization step for directed centrality measures removes Gould's objection that centrality measures for directed graphs are not interpretable because they lack a standard for maximality. The relative directed centrality converges with Freeman's betweenness measure in the case of undirected graphs with no isolates. The fourth step is to define the measures of this concept of graph centralization in terms of the dominance of the most central point. {\textcopyright} 1994.},
archivePrefix = {arXiv},
arxivId = {arXiv:1011.1669v3},
author = {White, Douglas R. and Borgatti, Stephen P.},
doi = {10.1016/0378-8733(94)90015-9},
eprint = {arXiv:1011.1669v3},
isbn = {0378-8733},
issn = {03788733},
journal = {Social Networks},
number = {4},
pages = {335--346},
pmid = {19756229},
title = {{Betweenness centrality measures for directed graphs}},
volume = {16},
year = {1994}
}
@article{Primack1983,
abstract = {A wide variety of lepidopterans, bees, flies, and beetles visit the flowers of most species of New Zealand montane plants. Of the 82 plant species which were well-collected, 4 species (Pterostylis sp., Pratia macrodon, Mazus radicans, and Dracophyllum acerosum) exhibited specialised pollination relationships with an insect order; 3 species (Cyathodes fraserii, Microtis unifolia, and Thelymitra venosa) are “not apparently visited by insects; and the remaining 75 species are visited by a variety of insects in 2 or more orders. Introduced plant species in the Composite family are visited predominantly by introduced bumblebees. Bees in the genera Lasioglossum and Leioproctus are abundant flower visitors. The most common lepidopteran flower visitors are in the families Noctuidae, Geometridae, and Pyralidae, and the genus Lycaena. Dipterans, in particular tachinids and syrphids, are numerically the most abundant flower visitors, and visit a wide range of species. The syrphid Melangyna novaezelandiae visits the flowers of more plant species than any other flower visitor. Beetles are typically not abundant, and do not move often between flowers. Species in the genus Hebe might be expected to have different insect pollinators from species in the Composite family, because of the quite different floral characteristics in these groups. However, there is no general difference in the insects visiting the flowers in the genus Hebe, the family Compositae, and the remaining species, indicating a lack of specialisation for particular pollinators. Individual flowers of most species last more than 4 days, so that several consecutive days of bad weather need not prevent a flower from being pollinated. There is no obvious relationship between the biogeographical origin of plant species and the types of insects visiting its flowers.},
author = {Primack, Richard B.},
doi = {10.1080/0028825X.1983.10428561},
isbn = {0028-825X},
issn = {11758643},
journal = {New Zealand Journal of Botany},
keywords = {Insect pollination,Insects,Mountain flora,New zealand flora,Pollination,Reproductive behaviour},
number = {3},
pages = {317--333},
title = {{Insect pollination in the New Zealand mountain flora}},
url = {http://www.tandfonline.com/doi/abs/10.1080/0028825X.1983.10428561},
volume = {21},
year = {1983}
}
@article{Ramirez1989,
abstract = {The pollination ecology of tropical shrubland was studied in the Venezuelan Guayana Highlands. Most flowers are white or pink. Flower length is less than 0.5 cm for 45.4 percent of plant species (N = 55). The diameter has a similar distribution. Of the 62 visiting agents recorded on 55 plant species, 80.6 percent are pollinators and 19.4 percent are visitors only. Four families of bees are represented. Pollen is normally carried on only one location, frequently ventral, although it can also be carried on the head or legs. Pollen transportation by wasps (four families) is frequently ventral on the body and legs. Lepidopterans (three families) carry pollen grains on their proboscis. Coleopterans are represented by two species of Scarabaeidae which transport pollen ventrally. They as well as the dipterans visit only one plant species. Three bird species visit plant species in the shrubland. Politmus milleri (Trochilidae) pollinates seven plant species and the pollen is transported on the beak and head feathers. At the community level, 56.2 percent of plant species are pollinated by hymenopterans, 10.9 percent by lepidopterans and 9.6 percent by dipterans, and one plant species (2.7{\%}) by beetles. Bird-pollinated plants represent an important part of plants in the shrubland (12.3{\%}). The anemophilous plants are 8.2 percent of total. From 148 species of visitors recorded on 49 zoophilous plant species, 37.6 percent of plant species are pollinated by medium size bees, 16.9 percent by large bees and 10.1 percent by small bees. The small wasps pollinate 8.1 percent and large wasps pollinate 6.1 percent of plant species. Large flies visit 1.4 percent and small flies 2.7 percent of plant species.},
author = {Ramirez, Nelson},
doi = {10.2307/2388282},
isbn = {00063606},
issn = {00063606},
journal = {Biotropica},
number = {4},
pages = {319},
title = {{Biologia de Polinizacion en una Comunidad Arbustiva Tropical de la Alta Guayana Venezolana}},
url = {https://www.jstor.org/stable/2388282?origin=crossref},
volume = {21},
year = {1989}
}
@article{Yamazaki2003,
abstract = {The hillsides of Mt, Yufu, located in Kyusyu, Japan, is a dormant yolcano, are coyered with natural and semi-natural grasslands; the latter of which are maintained by traditional mowing and burning. Both the natura1 and semi-natura1 grasslands are inhabited by many grassland-specific plant species, some of which are now endangered in Japan. To understand pollination mutualisms in the grassland ecosystem, we investigated the flowering phenology and anthophilous insect communities on 149 plant species from 49 different plant families, from April to October 2001. In tota1, 1192 individuals from 308 species, 83 families and 10 orders of Insecta were observed on flowers of 101 plant species. The most abundant insect order was Hymenoptera (37.89o of individuals), followed by Diptera (32.59o), Coleoptera (22.79o) and Lepidoptera (6.29o), The proportions of Coleoptera and Lepidoptera were respectively smaller and greater than in forested habitats, suggesting that many anthophilous beetles depend on woody plants during their larval stages and that anthophilous butterflies (especially Nymphalidae) are associated with grassland-specific perennials (especially Viola spp.) in their larval stages. The bee fauna consisted of 54 species, from 1O genera and 6 families; the bee community was characterized by an absence of cavity-nesting Hylaeus and Xylocopa and by the predominance of long-tongued Tetratonia in the early spring. The bumblebee community was characterized by the predominance of a short-haired Bombus ignitus, uncommon in forested habhats. The dominant pollination syndrome, among 70 plant species for which pollinators svere inferred, was melittophily (829o), followed by myophily (149o), psychophily (1.49o), phalaenophily (I.49o) and anemophily (1.49o). Among the melittophilous species, small-bee-pol]inated species (459o) dominated, followed by Bombus-(369o), Apis-(8.69o), Tetratonia-(6.99o), megachilid-(1.79o) and wasp-(1.79o) pollinated species. These data on community-level plant-pollinator interactions at Mt. Yufu will contribute to the conservation of endangered grassland ecosystems.},
author = {Yamazaki, Kyoko and Kato, Makoto},
issn = {0452-9987},
journal = {Contributions from the Biological Laboratory, Kyoto University},
number = {3},
pages = {255--318},
title = {{Flowering phenology and anthophilous insect community in a grassland ecosystem at Mt. Yufu, western Japan}},
volume = {29},
year = {2003}
}
@article{Schemske1978,
abstract = {JSTOR is a not-for-profit service that helps scholars, researchers, and students discover, use, and build upon a wide range of content in a trusted digital archive. We use information technology and tools to increase productivity and facilitate new forms of scholarship. For more information about JSTOR, please contact support@jstor.org. This content downloaded from 159.189.96.89 on Mon, 11 Jan 2016 15:13:29 UTC All use subject to JSTOR Terms and Conditions Ecology, 59(2), 1978, pp. 351-366 Abstract. Fecundity characteristics, phenology, and behavior of insect flower-visitors were stud-ied for 7 early flowering woodland herbs: Claytonia virginica, Dentaria laciniata, Dicentra canaden-sis, Dicentra cucullaria, Erythronium albidum, Isopyrum biternatum, and Sanguinaria canadensis. Sanguinaria canadensis is facultatively autogamous, the Dicentras are obligate outcrossers, and the remainder are self-compatible, at least within a stem. All are insect pollinated except sometimes S. canadensis. The numbers of ovules per flower and flowers per stem tended to be inversely correlated, and large-seeded species (S. canadensis, E. albidum, I. biternatum) had lower numbers of potential seeds per stem than did small-seeded species. Flowering of all species typically occurred during the first prolonged period of weather suitable for pollinator activity and ceased by the time the canopy closed. Annual differences in flowering times were associated with differences in average temperatures (i.e., early blooming in a warm, early spring), but cumulative degree-hours or degree-days of air or soil temperatures were not well corre-lated with flowering times. Other constraints on flowering phenology are discussed, including the predictability of suitable conditions, a proposed "fail-safe" mechanism that may assure flowering before canopy closure even if temperatures are abnormally low, and the importance of nontemperature factors in defining suitable conditions. Flowering time was not very finely tuned to the temperature regime and pollinator activity; flowers blooming during the flowering peak often had low seed pro-duction and the fertilization rate of most species was low. Evidence that seed production may have been pollinator limited for several species was obtained by comparing the success of hand pollination and of natural pollination, rarity of certain specialized pollinators, and estimates of the abortion rates of fertilized ovules. We suggest that flowering in early spring is a high-risk option in terms of insect-mediated sexual reproduction. Certain flower-visiting insects favored D. laciniata out of proportion to its abundance, but no effect on seed set of other species was detectable. Honeybees were abundant and active flower visitors with the potential for disrupting ecological/evolutionary relationshps between native insects and flowers.},
author = {Schemske, Douglas W. and Willson, Mary F. and Melampy, Michael N. and Miller, Linda J. and Verner, Louis and Schemske, Kathleen M. and Best, Louis B.},
doi = {10.2307/1936379},
isbn = {0012-9658},
issn = {00129658},
journal = {Ecology},
keywords = {adensis,breeding systems,claytonia virginica,competition,dentaria laciniata,dicentra can-,dicentra cucullaria,erythronium albidum,illinois,isopyrum biternatum,nation,phenology,polli-,seed production,temperature,woodland herbs},
number = {2},
pages = {351--366},
title = {{Flowering Ecology of Some Spring Woodland Herbs}},
url = {http://doi.wiley.com/10.2307/1936379},
volume = {59},
year = {1978}
}
@phdthesis{Witt1998,
author = {Witt, P.},
school = {University of Aarhus, Denmark},
title = {{Witt, P}},
type = {BSc thesis},
year = {1998}
}
@phdthesis{Vazquez2002a,
abstract = {*[Ants recorded visiting flowers om milkweed.]},
author = {Vazquez, Diego},
booktitle = {Doctoral Dissertations},
school = {University of Tennessee, Knoxville},
title = {{Interactions among Introduced Ungulates, Plants, and Pollinators: A Field Study in the Temperate Forest of the Southern Andes}},
type = {Ph.D. thesis},
url = {http://trace.tennessee.edu/utk{\_}graddiss/2169{\%}5Cnhttp://trace.tennessee.edu/utk{\_}graddiss/2169/},
year = {2002}
}
@phdthesis{Stald2003,
author = {Stald, Line},
pages = {1--75},
school = {University of Aarhus, Denmark},
title = {{Struktur og dynamik i rum og tid af et best{\o}vningsnetv{\ae}rk p{\aa} Tenerife, De Kanariske {\O}er}},
type = {Master's thesis},
year = {2003}
}
@article{Yletyinen2016,
abstract = {Species composition and habitats are changing at unprecedented rates in the world's oceans, potentially causing entire food webs to shift to structurally and functionally different regimes. Despite the severity of these regime shifts, elucidating the precise nature of their underlying processes has remained difficult. We address this challenge with a new analytic approach to detect and assess the relative strength of different driving processes in food webs. Our study draws on complexity theory, and integrates the network-centric exponential random graph modelling (ERGM) framework developed within the social sciences with community ecology. In contrast to previous research, this approach makes clear assumptions of direction of causality and accommodates a dynamic perspective on the emergence of food webs. We apply our approach to analysing food webs of the Baltic Sea before and after a previously reported regime shift. Our results show that the dominant food web processes have remained largely the same, although we detect changes in their magnitudes. The results indicate that the reported regime shift may not be a system-wide shift, but instead involve a limited number of species. Our study emphasizes the importance of community-wide analysis on marine regime shifts and introduces a novel approach to examine food webs.},
author = {Yletyinen, Johanna and Bodin, {\"{O}}rjan and Weigel, Benjamin and Nordstr{\"{o}}m, Marie C. and Bonsdorff, Erik and Blenckner, Thorsten},
doi = {10.1098/rspb.2015.2569},
isbn = {0962-8452},
issn = {14712954},
journal = {Proceedings of the Royal Society B: Biological Sciences},
keywords = {Baltic Sea,Complex adaptive systems,Exponential random graph model,Food web,Motifs,Regime shift},
number = {1825},
pages = {20152569},
pmid = {26888032},
title = {{Regime shifts in marine communities: A complex systems perspective on food web dynamics}},
url = {http://www.ncbi.nlm.nih.gov/pubmed/26888032{\%}5Cnhttp://www.pubmedcentral.nih.gov/articlerender.fcgi?artid=PMC4810827},
volume = {283},
year = {2016}
}
@article{Inouye1988,
author = {Inouye, David W. and Pyke, Graham H.},
journal = {Australian Journal of Ecology},
pages = {191--210},
title = {{Pollination biology in the Snowy Mountains of Australia comparisons with montane Colorado, USA. Australian Journal of Ecology.pdf}},
volume = {13},
year = {1988}
}
@article{Herrera1988,
abstract = {(1) Pollination relationships were investigated for fourteen months in a southern Spanish Mediterranean coastal scrub community, composed of thirty plant species, at Reserva Biol6gica de Dofiana, Dofiana National Park. (2) Flowering encompassed the whole year, as did insect visits to flowers. Distinct seasonal changes, however, in both the number and identity of insect taxa, and in the number of plant species in bloom were apparent: maximum plant and insect richness occurred in spring. (3) Insect visitors mainly included small beetles, honeybees, small halictid bees, syrphids and bombylids. The overall species richness of the pollinator array was very high (187 taxa). (4) Plant species with specialized pollination mechanisms were relatively infrequent. Most plants had non-restrictive or small flowers, or both. Species relying on pollen to attract pollinators outweighed those relying on nectar as the main reward. (5) Joint analysis of flower attributes, blooming phenology and pollination vectors demonstrated that species flowering at about the same time of year tend to have their flowers visited by the same insects, irrespective of floral features. (6) It is hypothesized that fruit set is more resource- than pollen-limited and that to achieve maximum fruit set most plants have unspecialized pollination relationships. The generalized nature of pollination systems may have been a major factor contributing to the survival and weedy behaviour of many Mediterranean scrub species.},
author = {Herrera, Javier},
doi = {10.2307/2260469},
isbn = {0022-0477},
issn = {00220477},
journal = {The Journal of Ecology},
number = {1},
pages = {274},
title = {{Pollination Relationships in Southern Spanish Mediterranean Shrublands}},
url = {https://www.jstor.org/stable/2260469?origin=crossref},
volume = {76},
year = {1988}
}
@article{Kato2000,
author = {KAto, M},
journal = {Euphorbiaceae, Glochidium, Glyphipterygidae, active pollination},
number = {2},
pages = {157--254},
title = {{Anthophilous insect community and plant-pollinator interactions on Amami Islands in Ryukyu..}},
volume = {29},
year = {2000}
}
@article{Dupont2003,
abstract = {Confined within a volcanic caldera at 2000 m a.s.l., the sub-alpine desert of Tenerife, Canary Islands, harbors a distinct biota. At this altitude the climate is harsh and the growing season short. Hence, plant and animal communities, constituting the sub- alpine plant–flower-visitor network, are clearly delimited, both spatially and tempo- rally. We investigated species composition and interaction structure of this system. A total of 11 plant species (91{\%} endemics) and 37 flower-visiting animal species (62{\%} endemics) formed 108 interactions. Numbers of interactions among species varied ten-fold within both plant and animal communities. Generalization level of a species was positively correlated with its local abundance. Two separate network analyses revealed a significantly nested structure. In relation to a plant–flower-visitor system, nestedness implies that specialized species (animals or plants) interact with a subset of the species pool visiting (animals) or being visited (plants) by more generalized species. Therefore, specialized, locally rare plants tend to be visited by generalized, locally abundant animals, and specialized, locally rare animals tend to utilize generalized, locally abundant food plants. Such patterns could have implications for conservation of the sub-alpine network, and stress the importance of preserving not only rare species, but also the more abundant ones, which may be key food resources or pollinators in the plant–flower-visitor network.},
author = {{Yoko L. Dupont}, Dennis M. Hansen and Jens M. Olesen},
journal = {Ecography},
number = {3},
pages = {301--310},
title = {{Structure of a plant–flower-visitor network in the high-altitude}},
volume = {26},
year = {2003}
}
@article{Medan2002,
abstract = {The assemblages of visitors to angiosperms flowering at a montane and at a high alpine site in the Andes of Mendoza, Argentina (33-34degreesS) were described and the plant-flower visitor matrices were analyzed and compared to other systems, in particular those located at a similar latitude on the western slope of the Andes. In the low-attitude (montane) habitat, 23 plant species had a total of 126 interactions with 71 taxa of insects and one hummingbird, and at the higher site 21 plants and 45 insect species had 83 interactions. Connectances of the visitor matrices were 7.6 and 8.7, respectively. Diptera and Hymenoptera dominated the visitor assemblages at both sites without change of proportions with attitude, while Lepidoptera significantly increased at the higher site. Flies were more species-rich than expected at the sites' latitude and more constant across altitudes than is usually observed. Lack of a significant decrease with altitude of (1) the frequency of Hymenoptera and (2) the number of interactions per plant were the main differences with a comparable gradient in the Chilean Andes. The proportion of self-compatible species increased with altitude, however. use of phylogenetically-independent contrasts showed that the prevalence of selfers at higher altitudes does not reflect a generalized reaction pattern but results from two speciose families (Asteraceae and Fabaceae) showing more self-compatibility at high altitudes.},
author = {Medan, Diego and Montaldo, Norberto H. and Devoto, Mariano and Mantese, Anita and Vasellati, Viviana and Roitman, German G. and Bartoloni, Norberto H.},
doi = {10.2307/1552480},
isbn = {1523-0430},
issn = {15230430},
journal = {Arctic, Antarctic, and Alpine Research},
number = {3},
pages = {233},
title = {{Plant-Pollinator Relationships at Two Altitudes in the Andes of Mendoza, Argentina}},
url = {http://www.jstor.org/stable/1552480?origin=crossref},
volume = {34},
year = {2002}
}
@article{Lundgren2005,
author = {Lundgren, R and Olesen, Jens M.},
journal = {Arctic, Antarctic, and Alpine Research},
pages = {51},
title = {{The dense and highly connected world of Greenland's plants and their pollinators}},
volume = {37},
year = {2005}
}
@article{Bartomeus2008,
abstract = {JSTOR is a not-for-profit service that helps scholars, researchers, and students discover, use, and build upon a wide range of content in a trusted digital archive. We use information technology and tools to increase productivity and facilitate new forms of scholarship. For more information about JSTOR, please contact support@jstor.org. This content downloaded from 216.125.80.183 on Mon, 08 Jun 2015 16:07:26 UTC All use subject to JSTOR Terms and Conditions Oecologia (2008) 155:761-770 Abstract The structural organization of mutualism net-works, typified by interspecific positive interactions, is important to maintain community diversity. However, there is little information available about the effect of introduced species on the structure of such networks. We compared uninvaded and invaded ecological communities, to examine how two species of invasive plants with large and showy flowers {\{}Carpobrotus affine acinaciformis and Opuntia stricta) affect the structure of Mediterranean plant-pollina-tor networks. To attribute differences in pollination to the direct presence of the invasive species, areas were surveyed that contained similar native plant species cover, diversity and floral composition, with or without the invaders. Both invasive plant species received significantly more pollina-tor visits than any native species and invaders interacted strongly with pollinators. Overall, the pollinator commu-nity richness was similar in invaded and uninvaded plots, and only a few generalist pollinators visited invasive spe-cies exclusively. Invasive plants acted as pollination super generalists. The two species studied were visited by 43{\%} and 31{\%} of the total insect taxa in the community, respec-tively, suggesting they play a central role in the plant-polli-nator networks. Carpobrotus and Opuntia had contrasting effects on pollinator visitation rates to native plants: Carpo-brotus facilitated the visit of pollinators to native species, whereas Opuntia competed for pollinators with native spe-cies, increasing the nestedness of the plant-pollinator net-work. These results indicate that the introduction of a new species to a community can have important consequences for the structure of the plant-pollinator network.}},
author = {Bartomeus, Ignasi and Vil{\`{a}}, Montserrat and Santamar{\'{i}}a, Lu{\'{i}}s},
doi = {10.1007/s00442-007-0946-l},
journal = {Source: Oecologia},
keywords = {Carpobrotus affine acinaciformis @BULLET,Ecological networks @BULLET,Generalization @BULLET,Interaction strength @BULLET,Invasive species},
number = {4},
pages = {761--770},
title = {{International Association for Ecology Contrasting Effects of Invasive Plants in Plant-Pollinator Networks Contrasting effects of invasive plants in plant-pollinator networks}},
volume = {155},
year = {2008}
}
@article{Kato1990,
abstract = {In 1985-1987 insect visitors to flowers were weekly or biweekly surveyed on a total of 113 plant species or 48 families in the campus of Kyoto University in Kyo-to city, Japan. Although the total number of plant species was nearly equal to those in Ashu and Kibune, native species were only 25, due to urbanization and disturbance. Flowering started from cultivated plants, e.g. Prunus spachina, in early April and ended also in cultivated plants, e.g. Camellia sazanqua in late November. The total number of plant species at flowering peaked in May. The flowering period of a single species was 17 days on average. A total of 2109 individuals of 320 species in nine orders of Insecta and two orders in Arachnoidea were collected in our samples. The total number of arthropod species was estimated to be 790 by the Preston's octave method and thus 40.5{\%} were in our samples. The most abundant order was Hymenoptera (50{\%} of individuals), followed by Coleoptera (26{\%}) and Diptera (16{\%}). The number of species was highest in Diptera (34{\%}), followed by Hymenoptera (33{\%}) and Coleoptera (14{\%}). Compared with the undisturbed areas, Ashu and Kibune, two dominant Coleopteran families, Cerambycidae and Nitidulidae were quite rare here. In Hymenoptera, Megachilidae were quite abundant on exotic cultivated plants. The estimated total number of bee species (170 sp.) was more than those in the undisturbed areas. The number of insect species peaked twice in June and September, while the total number of individuals peaked in May and September. Coleoptera peaked in May and June, Diptera peaked in June and October, while Hymenoptera appeared rather constantly throughout the flowering season. Cluster analysis separated 48 plant families into four groups: 30 families mainly visited by Hymenoptera, 6 families by Diptera, 9 families by Coleoptera and the others (3 families) by Lepidoptera.},
author = {Kakutani, T and Inoue, T and Kato, M and Ichihashi, H},
doi = {citeulike-article-id:548578},
isbn = {0452-9987},
issn = {0452-9987},
journal = {Contributions from the Biological Laboratory Kyoto University},
keywords = {Activity patterns,Araneae,Araneae (Arachnida).,Asia,Associations,Behaviour,Diet,Distribution within habitat,Ecology,Eurasia,Habitat,Insecta [Food plants / / Seasonal flower visiting,Insecta.,Land zones,Man made habitat,Mutualism,Nutrition,Opiliones (Arachnida).,Opiliones [Pollination / / Seasonal occurrence in,Palaearctic region,Terrestrial,[Plant and vegetation habitats / / Seasonal occurr,[Seasonal activity / / Flower visiting] [Pollinati,habitat,habitat / / ] [Japan / / Kyoto].,university campus],university campus] [Seasonal distribution within h},
number = {4},
pages = {465--521},
title = {{Insect-flower relationship in the campus of Kyoto University, Kyoto: an overview of the flowering phenology and the seasonal pattern of insect visits}},
volume = {27},
year = {1990}
}
@article{Kato1996,
abstract = {Nakaikemi marsh, located in Fukui Prefecture, is one of only a few natural lowland marshlands left in western Japan, and harbors many endangered marsh plants and animals. Flowering phenology and anthophilous insect communities on 64 plant species of 35 families were studied in the marsh in 1994-95. A total of 936 individuals of 215 species in eight orders of Insecta were collected on flowers from mid April to mid October. The anthophilous insect community was characterized by dominance of Diptera (58{\%} of individuals) and relative paucity of Hymenoptera (26{\%}), Hemiptera (6{\%}), Lepidoptera (5{\%}), and Coleoptera (5{\%}). Syrphidae was the most abundant family and probably the most important pollination agents. Bee community was characterized by dominance of an aboveground nesting bee genus, Hylaeus (Colletidae), the most abundant species of which was a minute, rare little-recorded species. Cluster analysis on flower-visiting insect spectra grouped 64 plant species into seven clusters, which were respectively characterized by dominance of small or large bees (18 spp.), syrphid flies (13 spp.), Calyptrate and other flies (11 spp.), wasps and middle-sized bees (8 spp.), Lepidoptera (2 spp.), Coleoptera (1 sp.) and a mixture of these various insects (11 spp.). These flower guilds largely coincided with pollination guilds with some exceptions such as anemophilous grasses visited by specific syrphid flies. The flower-insect relationship in the marsh was discriminated from that in woodlands by rarity of specialized relationships and by prevalence of relationships between flowers and flies, most larvae of which grow in waterlogged habitats. Nakaikemi marsh is regarded as a rare, important wetland habitat not only harboring many endangered plant and anthophilous insect species but also fostering unique insect-flower relationships. The presence of some plant species originally pollinated by bumblebees nesting at forest floor suggests that the marshland should be conserved as a whole ecosystem uniting the marshland and the neighboring woodlands.},
author = {Kato, M},
issn = {0452-9987},
journal = {Contribution from Biological Laboratory Kyoto University},
keywords = {1 bees,1 wetland 1 marsh,flowering phenology 1 pollination,syrphidae},
number = {March},
pages = {1--48},
title = {{Flowering phenology and anthophilous insect community at a threatened natural lowland marsh at Nakaikemi in Tsuruga, Japan}},
url = {http://www.mendeley.com/research/flowering-phenology-anthophilous-insect-community-threatened-natural-lowland-marsh-nakaikemi-tsuruga-japan/},
volume = {29},
year = {1996}
}
@article{Kakutani1990,
abstract = {In 1985-1987 insect visitors to flowers were weekly or biweekly surveyed on a total of 113 plant species or 48 families in the campus of Kyoto University in Kyo-to city, Japan. Although the total number of plant species was nearly equal to those in Ashu and Kibune, native species were only 25, due to urbanization and disturbance. Flowering started from cultivated plants, e.g. Prunus spachina, in early April and ended also in cultivated plants, e.g. Camellia sazanqua in late November. The total number of plant species at flowering peaked in May. The flowering period of a single species was 17 days on average. A total of 2109 individuals of 320 species in nine orders of Insecta and two orders in Arachnoidea were collected in our samples. The total number of arthropod species was estimated to be 790 by the Preston's octave method and thus 40.5{\%} were in our samples. The most abundant order was Hymenoptera (50{\%} of individuals), followed by Coleoptera (26{\%}) and Diptera (16{\%}). The number of species was highest in Diptera (34{\%}), followed by Hymenoptera (33{\%}) and Coleoptera (14{\%}). Compared with the undisturbed areas, Ashu and Kibune, two dominant Coleopteran families, Cerambycidae and Nitidulidae were quite rare here. In Hymenoptera, Megachilidae were quite abundant on exotic cultivated plants. The estimated total number of bee species (170 sp.) was more than those in the undisturbed areas. The number of insect species peaked twice in June and September, while the total number of individuals peaked in May and September. Coleoptera peaked in May and June, Diptera peaked in June and October, while Hymenoptera appeared rather constantly throughout the flowering season. Cluster analysis separated 48 plant families into four groups: 30 families mainly visited by Hymenoptera, 6 families by Diptera, 9 families by Coleoptera and the others (3 families) by Lepidoptera.},
author = {Kakutani, T and Inoue, T and Kato, M and Ichihashi, H},
doi = {citeulike-article-id:548578},
isbn = {0452-9987},
issn = {0452-9987},
journal = {Contributions from the Biological Laboratory Kyoto University},
keywords = {Activity patterns,Araneae,Araneae (Arachnida).,Asia,Associations,Behaviour,Diet,Distribution within habitat,Ecology,Eurasia,Habitat,Insecta [Food plants / / Seasonal flower visiting,Insecta.,Land zones,Man made habitat,Mutualism,Nutrition,Opiliones (Arachnida).,Opiliones [Pollination / / Seasonal occurrence in,Palaearctic region,Terrestrial,[Plant and vegetation habitats / / Seasonal occurr,[Seasonal activity / / Flower visiting] [Pollinati,habitat,habitat / / ] [Japan / / Kyoto].,university campus],university campus] [Seasonal distribution within h},
number = {4},
pages = {465--521},
title = {{Insect-flower relationship in the campus of Kyoto University, Kyoto: an overview of the flowering phenology and the seasonal pattern of insect visits}},
volume = {27},
year = {1990}
}
@article{Bezerra2009,
abstract = {1. In the Neotropics, most plants depend on animals for pollination. Solitary bees are the most important vectors, and among them members of the tribe Centridini depend on oil from flowers (mainly Malpighiaceae) to feed their larvae. This specialized relationship within 'the smallest of all worlds' (a whole pollination network) could result in a 'tiny world' different from the whole system. This 'tiny world' would have higher nestedness, shorter path lengths, lower modularity and higher resilience if compared with the whole pollination network. 2. In the present study, we contrasted a network of oil-flowers and their visitors from a Brazilian steppe ('caatinga') to whole pollination networks from all over the world. 3. A network approach was used to measure network structure and, finally, to test fragility. The oil-flower network studied was more nested (NODF = 0.84, N = 0.96) than all of the whole pollination networks studied. Average path lengths in the two-mode network were shorter (one node, both for bee and plant one-mode network projections) and modularity was lower (M = 0.22 and four modules) than in all of the whole pollination networks. Extinctions had no or small effects on the network structure, with an average change in nestedness smaller than 2{\%} in most of the cases studied; and only two species caused coextinctions. The higher the degree of the removed species, the stronger the effect and the higher the probability of a decrease in nestedness. 4. We conclude that the oil-flower subweb is more cohesive and resilient than whole pollination networks. Therefore, the Malpighiaceae have a robust pollination service in the Neotropics. Our findings reinforce the hypothesis that each ecological service is in fact a mosaic of different subservices with a hierarchical structure ('webs within webs').},
author = {Bezerra, Elis{\^{a}}ngela L S and MacHado, Isabel C. and Mello, Marco A R},
doi = {10.1111/j.1365-2656.2009.01567.x},
isbn = {0021-8790},
issn = {00218790},
journal = {Journal of Animal Ecology},
keywords = {Community structure,Melittophyly,Modules,Mutualism,Subwebs},
number = {5},
pages = {1096--1101},
pmid = {19515098},
title = {{Pollination networks of oil-flowers: A tiny world within the smallest of all worlds}},
volume = {78},
year = {2009}
}
@article{Kato1993,
abstract = {We studied flowering phenology and anthophilous insect commumttes bimonthly in 1990-1991 in the primary cool-temperate subalpine forests and meadows at Mt. Kushigata, Yamanashi Prefecture, Japan. One hundred and fifty-one plant species of 41 families flowered sequentially from late May to mid September. A total of 2127 individuals of 370 species in eight orders of Insecta were collected. The most abundant order was Hymenoptera (35{\%} of individuals) and followed by Diptera (33{\%}), Coleoptera (28{\%}) and Lepidoptera (4{\%}). The number of species was highest in Diptera (47{\%}) and followed by Hymenoptera (24{\%}), Coleoptera (18{\%}) and Lepidoptera (9{\%}). The numbers of both spe-cies and individuals peaked in late July and early August. Bee fauna was composed of six families, nine genera and 34 species, lacking Xylocopinae and wild Apinae. The most abundant genus in bees was Bombus (76.7{\%} of individuals) and followed by Lasioglossum (20.2{\%}). Cluster analysis on flower-visiting insect order spectra separated 30 plant families into four groups: nine families (Geraniaceae, Elaeagnaceae, Onagraceae, Ericaceae, Labiatae, Scrophulariaceae, Campanulaceae, Liliaceae and Iridaceae) were visited mainly by Hyme-noptera, one (Violaceae) by Lepidoptera, five (Celastraceae, Umbelliferae, Polemoniaceae, Dipsacaceae and Gramineae) by Coleoptera and 15 by Diptera and/or various orders. Cluster analysis on flower-visiting insect order spectra of 91 plant species separated them into five flower guilds: hymenopterous (36 plant species), dipterous (30 spp.), coleopterous (14 spp.), lepidopterous (two spp.) and general flowers (nine spp.). Significant correlations were detected between violet flower color and hymenopterous flowers and between tubular corolla and hymenopterous flowers. Seventy-three {\%} of hymenopterous flowers and 93{\%} of dipterous flowers were visited by bumblebees and hoverflies, respectively. Cluster analysis on flower-visiting bumblebee species spectra separated 42 plant species into five flower guilds: longest-tongued bumblebee flowers (eight spp.), B. honshuensis flowers (eight spp.), B. ardens flowers (three spp.), B. beaticola flowers (15 spp.), B. hypocrita flowers (eight spp.). The number of coflowering plant species within each flower guild was usually kept less than five and, at most, eight in B. beaticola flowers which sometimes shared a few bumblebee species. Flower-visiting patterns of anthophilous insects were compared among insect orders, families and bumblebee species. The most preferred plant family was Compositae in Hymenoptera, Diptera and Lepidoptera, and Saxifragaceae in Coleoptera. Niche segrega-tion as to floral host utilization was detected among six bumblebee species, although there were overlaps. The two longest-tongued bumblebee species visited similar plant species, but the second longest-tongued B. diversus, was largely expelled from the flowers of the same guild by the longest-tongued B. consobrinus, and the flower-visiting pattern of B. di-versus was rather similar to the third longest-tongued B. honshuensis. The high bumblebee species diversity and niche segregation among them are thought to be a reason of high spe-cies diversity of herbaceous plants at cool-temperate subalpine forests and meadows. KEY WORDS flowering phenology/ anthophilous insect community/ bumblebee/ flower guild/ subalpine meadow 120 M. KATO et al.},
author = {Kato, Makoto and Matsumoto, Masamichi and Kato, Toru},
issn = {0452-9987},
journal = {Contributions from the Biological Laboratory, Kyoto University},
number = {2},
pages = {119--172},
title = {{Flowering phenology and anthophilous insect community in the cool-temperate subalpine forests and meadows at Mt. Kushigata in the central part of Japan}},
url = {http://repository.kulib.kyoto-u.ac.jp/dspace/bitstream/2433/156107/1/cbl02802{\_}119.pdf},
volume = {28},
year = {1993}
}
@article{Arroyo1982,
abstract = {JSTOR is a not-for-profit service that helps scholars, researchers, and students discover, use, and build upon a wide range of content in a trusted digital archive. We use information technology and tools to increase productivity and facilitate new forms of scholarship. For more information about JSTOR, please contact support@jstor.org. This content downloaded from 129.81.226.78 on Mon, 11 Jan 2016 03:51:18 UTC All use subject to JSTOR Terms and Conditions},
author = {Arroyo, Mary T. Kalin and Primack, Richard and Armesto, Juan},
doi = {10.2307/2442833},
isbn = {0002-9122},
issn = {00029122},
journal = {American Journal of Botany},
number = {1},
pages = {82--97},
pmid = {21269816},
title = {{Community Studies in Pollination Ecology in the High Temperate Andes of Central Chile. I. Pollination Mechanisms and Altitudinal Variation}},
url = {http://doi.wiley.com/10.1002/j.1537-2197.1982.tb13237.x},
volume = {69},
year = {1982}
}
@article{Dupont2009a,
abstract = {1. Co-existing plants and flower-visiting animals often form complex interaction networks. A long-standing question in ecology and evolutionary biology is how to detect nonrandom subsets (compartments, blocks, modules) of strongly interacting species within such networks. Here we use a network analytical approach to (i) detect modularity in pollination networks, (ii) investigate species composition of modules, and (iii) assess the stability of modules across sites. 2. Interactions between entomophilous plants and their flower-visitors were recorded throughout the flowering season at three heathland sites in Denmark, separated by {\textgreater}or= 10 km. Among sites, plant communities were similar, but composition of flower-visiting insect faunas differed. Visitation frequencies of visitor species were recorded as a measure of insect abundance. 3. Qualitative (presence-absence) interaction networks were tested for modularity. Modules were identified, and species classified into topological roles (peripherals, connectors, or hubs) using 'functional cartography by simulated annealing', a method recently developed by Guimer{\`{a}} {\&} Amaral (2005a). 4. All networks were significantly modular. Each module consisted of 1-6 plant species and 18-54 insect species. Interactions aggregated around one or two hub plant species, which were largely identical at the three study sites. 5. Insect species were categorized in taxonomic groups, mostly at the level of orders. When weighted by visitation frequency, each module was dominated by one or few insect groups. This pattern was consistent across sites. 6. Our study adds support to the conclusion that certain plant species and flower-visitor groups are nonrandomly and repeatedly associated. Within a network, these strongly interacting subgroups of species may exert reciprocal selection pressures on each other. Thus, modules may be candidates for the long-sought key units of co-evolution.},
author = {Dupont, Yoko L. and Olesen, Jens M.},
doi = {10.1111/j.1365-2656.2008.01501.x},
isbn = {1365-2656},
issn = {00218790},
journal = {Journal of Animal Ecology},
keywords = {Coevolutionary units,Community structure,Compartmentalization,Mutualistic webs,Network topology},
number = {2},
pages = {346--353},
pmid = {19021779},
title = {{Ecological modules and roles of species in heathland plant-insect flower visitor networks}},
volume = {78},
year = {2009}
}
@article{Hocking1968,
abstract = {665 measurements of nectar concentration and 983 of nectar volume per day, distributed among 37 out of 43 species of flowering plants examined, are recorded and analysed. Nectar production per unit area per season was substantially less at Lake Hazen, 82° N, than at Churchill, 58°N. Nectar yield in mg sugar/flower/day was higher at Lake Hazen than at Churchill in eight of the ten species for which data were obtained at both localities. There is competition between flowers for pollinators rather than among pollinators for nectar. Heliotropic flowers, notably Dryas and Papaver, focus sunlight falling on them in the region of the germ cells; it is shown that the thermal increments obtainable by black insects resting in these flowers can be important. 184 different plant species - insect species associations are reported, based on about 350 observations and 760 insect specimens; these associations fall into 9 activity categories (some into more than one), as follows: ambush (6), basking (4), flying over (20), hidden in (20), courtship behaviour (1), nectar feeding (23), ovipositing (2), pollen feeding or collecting (12), resting on or uncertain (96). It is concluded that flowers and floral groups are important as aggregation centres for insect populations in this environment. /// На 37 из 43 видов цветущих растений проводили определения концентрации и об"ема нектара в цветках (665 определений концентрации нектара и 983 определения об"ема нектара). Установлено, что продукция нектара на единицу площади в течение сезона в Лейк-Хазен (82° с.ш.) значительно ниже, чем в Черчхилле (58° с.ш.). Количество нектара, выраженное в мг сахар/цветок/день в Лейк-Хазон выше, чем в Черчхилле у 8 из 10 видов растений, для которых получены данные в обоих пунктах. Предполагается, что конкуренция между цветами из-за опылителей более вероятна, чем между опылителями изза нектара. Гелиотропные цветы, например. Dryas и Papaver, фокусируют падающий на них солнечный свет на зародышевых клетках. Показано, что повышение температуры, которое способны вызывать черные насекомые, отдыхающие на цветах, может иметь значение для жизнедеятельности цветов. Установлено 184 вида связей "вид растения" - "вид насекомого", которые основаны на 350 наблюдениях за 760 видами насекомых. Выделено 9 категорий связей насекомых с цветами: засада в цветке (6), обогревание на солнце (4), полет над цветком (20), укрывание в цветке (20), брачные игры (1), питание нектаром (23), откладка яиц (2), питание пыльцой или ее сбор (12), отдых или неопределенное поведение (96). Установлено, что цветы и цветущие группы растений имеют значение как центры скопления популяций насекомых.},
author = {Hocking, Brian},
doi = {10.2307/3565022},
isbn = {00301299},
issn = {00301299},
journal = {Oikos},
number = {2},
pages = {359},
title = {{Insect-Flower Associations in the High Arctic with Special Reference to Nectar}},
url = {http://www.jstor.org/stable/3565022?origin=crossref},
volume = {19},
year = {1968}
}
@article{Dicks2002,
author = {Dicks, L V and Corbet, S A},
journal = {Journal of Animal Ecology},
number = {1},
pages = {32--43},
title = {{in plant-insect Compartmentalization flower visitor webs}},
volume = {71},
year = {2012}
}
@article{Motten1986,
abstract = {This content downloaded from 136.177.28.15 on Tue, 23 Feb 2016 16:28:04 UTC All use subject to JSTOR Terms and Conditions Abstract. I studied the spring wildflower community of mesic deciduous forests in piedmont North Carolina to determine (a) the extent to which fecundity is pollination-limited in the community, (b) the importance of competition for pollination in affecting seed-set, and (c) the characteristics of plants and their floral visitors that most contribute to full pollination. Although inadequate pollination seems likely in the community, supplemental hand-pollination significantly improved fecundity in just 3 of the 12 species I examined. Pollination-limited reproductive success was evident only in a distinctive subset of the community, species pollinated primarily by queen bumble bees. The majority of wildflower species are pollinated by flies and solitary bees. Measurements of visitation rates and pollinator effectiveness on these plants confirmed that they are usually adequately pollinated in spite of a short blooming season, considerable overlap in flowering times, extensive pollinator sharing by concurrently blooming species, and inclement weather that frequently interrupts insect activity. Many of the flies and solitary bees are inconstant foragers, yet competition for pollination among wildflower species through differential pollinator attraction or interspecific pollinator movements usually does not significantly decrease the seed-set of plants with shared visitors. Competition may act with other causes of insufficient pollination, however, as a selective force to maintain a characteristic set of floral biology traits within the community, including autogamy and self-compatibility, extended receptivity, and pollination by a variety of visitor types. That these floral traits contribute significantly to the successful pollination of vernal herbs was demonstrated by observations of visitor behavior, plant caging experiments that excluded visitors or restricted their access to selected flowers, and measure-ments of floral lifetimes and seed-set for individual plants. These traits are effective regardless of the source of pollination-limited fecundity, and it is the prevalence of such traits, rather than floral specialization or character displacement, that distinguishes the forest spring wildflower community from other communities with potentially inadequate pollinator service.},
author = {Motten, Alexander F.},
doi = {10.2307/2937269},
isbn = {00129615},
issn = {00129615},
journal = {Ecological Monographs},
keywords = {bees,bombylius major,competition,deciduous forest,floral biology,forest herbs,north carolina,plant community,pollination,seed-set,spring wildflowers},
number = {1},
pages = {21--42},
pmid = {14839495},
title = {{Pollination Ecology of the Spring Wildflower Community of a Temperate Deciduous Forest}},
url = {http://doi.wiley.com/10.2307/2937269},
volume = {56},
year = {1986}
}
@article{Percival1974,
abstract = {Species flowering during the dry season (late December to early April) form three main pollinator groups, namely, butterfly, solitary bee, and hummingbird flowers. The animal vectors present match these three flower syndromes. The daily and seasonal rhythms of flower opening and pollination activity are synchronized. Butterflies and solitary bees, active chiefly in the morning, visit flowers blooming only half a day. The number of seeds per fruit and the pollen-carrying capacity of the pollen vector are correlated. Butterflies, carrying few pollen grains, chiefly pollinate flowers with one to four seeds. Solitary bees, carrying many grains, chiefly pollinate flowers with many seeds per flower. These close links between the flowers and their visitors result in a high degree of successful pollination and a high percentage fruit set. A group of species, attracting no visitors, was characterized by a very narrow 'pollination gap,' making self-pollination almost inevitable. Heavy fruiting indicated self-fertility. Heterocorollary in Cordia sebestena, a double keel in Stylosanthes hamata, and several examples of andromonoecy are recorded for the first time. The presence of the adventive honeybee, Apis mellifera, in the association appeared detrimental to the native pollinators when forage was scarce. The whole scrub association appears balanced with a distinctive pattern of floral ecology. This circumstance, in terms of pollinators, is designated a Butterfly-Solitary bee-Hummingbird association.},
archivePrefix = {arXiv},
arxivId = {arXiv:1011.1669v3},
author = {Percival, Mary},
doi = {10.2307/2989824},
eprint = {arXiv:1011.1669v3},
isbn = {00063606},
issn = {00063606},
journal = {Biotropica},
number = {2},
pages = {104},
pmid = {21675331},
title = {{Floral Ecology of Coastal Scrub in Southeast Jamaica}},
url = {https://www.jstor.org/stable/2989824?origin=crossref},
volume = {6},
year = {1974}
}
@article{Mosquin1967,
author = {Martin, H},
journal = {Canadian Field Naturalist},
pages = {201--205},
title = {{Observations on the Pollination Biology of Plants on Melville Island , N . W . T ., Canada}},
volume = {81},
year = {1965}
}
@article{Olesen2002a,
abstract = {The structure of pollination networks is described for two oceanic islands, the Azorean Flores and the Mauritian Ile aux Aigrettes. At each island site, all interactions between endemic, non-endemic native and introduced plants and pollinators were mapped. Linkage level, i.e. number of species interactions per species, was significantly higher for endemic species than for non-endemic native and introduced species. Linkage levels of the two latter categories were similar. Nine types of interaction may be recognized among endemic, non-endemic native and introduced plants and pollinators. Similar types had similar frequencies in the two networks. Specifically, we looked for the presence of 'invader complexes' of mutualists, defined as groups of introduced species interacting more with each other than expected by chance and thus facilitating each other's establishment. On both islands, observed frequencies of interactions between native (endemic and non-endemic) and introduced pollinators and plants differed from random. Introduced pollinators and plants interacted less than expected by chance. Thus, the data did not support the existence of invader complexes. Instead, our study suggested that endemic super-generalist species, i.e. pollinators or plant species with a very wide pollination niche, include new invaders in their set of food plants or pollinators and thereby improve establishment success of the invaders. Reviewing other studies, super generalists seem to be a widespread island phenomenon, i.e. island pollination networks include one or a few species with a very high generalization level compared to co-occurring species. Low density of island species may lead to low interspecific competition, high abundance and ultimately wide niches and super generalization.},
author = {Olesen, Jens M. and Eskildsen, Louise I. and Venkatasamy, Shadila},
doi = {10.1046/j.1472-4642.2002.00148.x},
isbn = {1366-9516},
issn = {13669516},
journal = {Diversity and Distributions},
keywords = {Azores,Biological invasions,Endemism,Mauritius,Mutualisms,Pollination,Super generalist},
number = {3},
pages = {181--192},
pmid = {502},
title = {{Invasion of pollination networks on oceanic islands: Importance of invader complexes and endemic super generalists}},
volume = {8},
year = {2002}
}
@phdthesis{Motten1982,
abstract = {This content downloaded from 136.177.28.15 on Tue, 23 Feb 2016 16:28:04 UTC All use subject to JSTOR Terms and Conditions Abstract. I studied the spring wildflower community of mesic deciduous forests in piedmont North Carolina to determine (a) the extent to which fecundity is pollination-limited in the community, (b) the importance of competition for pollination in affecting seed-set, and (c) the characteristics of plants and their floral visitors that most contribute to full pollination. Although inadequate pollination seems likely in the community, supplemental hand-pollination significantly improved fecundity in just 3 of the 12 species I examined. Pollination-limited reproductive success was evident only in a distinctive subset of the community, species pollinated primarily by queen bumble bees. The majority of wildflower species are pollinated by flies and solitary bees. Measurements of visitation rates and pollinator effectiveness on these plants confirmed that they are usually adequately pollinated in spite of a short blooming season, considerable overlap in flowering times, extensive pollinator sharing by concurrently blooming species, and inclement weather that frequently interrupts insect activity. Many of the flies and solitary bees are inconstant foragers, yet competition for pollination among wildflower species through differential pollinator attraction or interspecific pollinator movements usually does not significantly decrease the seed-set of plants with shared visitors. Competition may act with other causes of insufficient pollination, however, as a selective force to maintain a characteristic set of floral biology traits within the community, including autogamy and self-compatibility, extended receptivity, and pollination by a variety of visitor types. That these floral traits contribute significantly to the successful pollination of vernal herbs was demonstrated by observations of visitor behavior, plant caging experiments that excluded visitors or restricted their access to selected flowers, and measure-ments of floral lifetimes and seed-set for individual plants. These traits are effective regardless of the source of pollination-limited fecundity, and it is the prevalence of such traits, rather than floral specialization or character displacement, that distinguishes the forest spring wildflower community from other communities with potentially inadequate pollinator service.},
author = {Motten, Alexander F.},
booktitle = {Ecological Monographs},
doi = {10.2307/2937269},
isbn = {00129615},
issn = {00129615},
number = {1},
pages = {21--42},
pmid = {14839495},
school = {Duke University, USA},
title = {{Pollination Ecology of the Spring Wildflower Community of a Temperate Deciduous Forest}},
type = {Ph.D. thesis},
url = {http://doi.wiley.com/10.2307/2937269},
volume = {56},
year = {1986}
}
@article{Ollerton2003,
abstract = {The KwaZulu-Natal region of South Africa hosts a large diversity of asclepiads (Apocynaceae: Asclepiadoideae), many of which are endemic to the area. The asclepiads are of particular interest because of their characteristically highly evolved floral morphology. During 3 months of fieldwork (November 2000 to January 2001) the flower visitors and pollinators to an assemblage of nine asclepiads at an upland grassland site were studied. These observations were augmented by laboratory studies of flower morphology (including scanning electron microscopy) and flower colour (using a spectrometer). Two of the specialized pollination systems that were documented are new to the asclepiads: fruit chafer pollination and pompilid wasp pollination. The latter is almost unique in the angiosperms. Taxa possessing these specific pollination systems cluster together in multidimensional phenotype space, suggesting that there has been convergent evolution in response to similar selection to attract identical pollinators. Pollination niche breadth varied from the very specialized species, with only one pollinator, to the more generalized, with up to ten pollinators. Pollinator sharing by the specialized taxa does not appear to have resulted in niche differentiation in terms of the temporal or spatial dimensions, or with regards to placement of pollinaria. Nestedness analysis of the data set showed that there was predictability and structure to the pattern of plant-pollinator interactions, with generalist insects visiting specialized plants and vice versa. The research has shown that there is still much to be learned about plant-pollinator interactions in areas of high plant diversity such as South Africa.},
author = {Ollerton, Jeff and Johnson, Steven D. and Cranmer, Louise and Kellie, Sam},
doi = {10.1093/aob/mcg206},
isbn = {0305-7364 (Print)$\backslash$r0305-7364 (Linking)},
issn = {03057364},
journal = {Annals of Botany},
keywords = {Apocynaceae,Asclepiadaceae,Community structure,Floral morphology,Grassland,Mutualism,Nestedness,Niche,Plant assemblage,Pollination,SEM,South Africa,Species interactions},
number = {6},
pages = {807--834},
pmid = {14612378},
title = {{The pollination ecology of an assemblage of grassland asclepiads in South Africa}},
volume = {92},
year = {2003}
}
@article{Bouillon1989,
abstract = {Eighteen genera of metazoan parasites (Monogenea, 1; Digenea, 7; Cestoda, 4; Nematoda, 4; Acanthocephala, 1; Copepoda, 1) were collected from 172 landlocked and anadromous Arctic charr (Salvelinus alpinus Linnaeus) in northern Labrador. Four species (Lecithaster gibbosus, Diphyllobothrium dendriticum, Diphyllobothrium ditremum, and Salmincola carpionis) and one genus (Tetraonchus) have not previously been reported from Arctic charr in Labrador. The dominant parasites in the landlocked charr were Diplostomum sp., Crepidostomum farionis, and Diphyllobothrium ditremum. In the sea-run charr, Bothrimonus sturionis and Brachyphallus crenatus were dominant. Regression analyses indicated that the numbers of parasites were significantly correlated with host age (P {\textless} 0.001) for these species. All landlocked charr sampled were infected with parasites by age 1+ years whereas all sea-run charr were infected by age 3+. At 1+ years, 43{\%} of the Arctic charr collected in the Ikarut River were infected with marine or brack...},
author = {Bouillon, Daniel R. and Dempson, J. Brian},
doi = {10.1139/z89-350},
issn = {0008-4301},
journal = {Canadian Journal of Zoology},
number = {10},
pages = {2478--2485},
pmid = {900500225},
title = {{Metazoan parasite infections in landlocked and anadromous Arctic charr ( Salvelinus alpinus Linnaeus ), and their use as indicators of movement to sea in young anadromous charr}},
url = {http://www.nrcresearchpress.com/doi/abs/10.1139/z89-350},
volume = {67},
year = {1989}
}
@book{McDowall1990,
abstract = {The non-diadromous Galaxias vulgaris species complex, comprising about 10 genetic lineages. is confined to South and Stewart Islands, primarily to the east of the Southern Alps, though there are a few populations to the northwest of the northern Southern Alps. They fall into two morphotypes, informally called 'flatheads' (6 lineages) and 'roundheads' (4 lineages). Species richness is low in the north and greatest in the southern sector of the South Island Greatest diversity in the south, is on an area that is regarded by some as a residual emergent island during the Oligocene marine submergence of much of New Zealand, where 8 of the lineages are found. Distributions of lineages overlap broadly in the south, though sympatry of lineages is only occasional, and where there is sympatry there is minimal evidence for hybridisation Patterns of distribution tend to connect strongly to existing river catchments, though there are interesting instances where apparently anomalous occurrences relate to know changes in riverine connections associated with changes in earth history and topography},
address = {Auckland, New Zealand},
author = {McDowall, R.M.},
doi = {10.1007/978-90-481-9271-7},
isbn = {978-90-481-9270-0},
issn = {1367-8396},
pages = {553},
publisher = {Heinemann Reed/MAF Publishing Group},
title = {{New Zealand Freshwater Fishes}},
url = {http://link.springer.com/10.1007/978-90-481-9271-7},
year = {2010}
}
@article{Svanback2015,
abstract = {Among-individual diet variation is common in natural populations and may occur at any trophic level within a food web. Yet, little is known about its variation among trophic levels and how such variation could affect phenotypic divergence within populations. In this study we investigate the relationships between trophic position (the population's range and average) and among-individual diet variation. We test for diet variation among individuals and across size classes of Eurasian perch (Perca fluviatilis), a widespread predatory freshwater fish that undergoes ontogenetic niche shifts. Second, we investigate among-individual diet variation within fish and invertebrate populations in two different lake communities using stable isotopes. Third, we test potential evolutionary implications of population trophic position by assessing the relationship between the proportion of piscivorous perch (populations of higher trophic position) and the degree of phenotypic divergence between littoral and pelagic perch sub-populations. We show that among-individual diet variation is highest at intermediate trophic positions, and that this high degree of among-individual variation likely causes an increase in the range of trophic positions among individuals. We also found that phenotypic divergence was negatively related to trophic position in a population. This study thus shows that trophic position is related to and may be important for among-individual diet variation as well as to phenotypic divergence within populations.},
author = {Svanb{\"{a}}ck, Richard and Quevedo, Mario and Olsson, Jens and Ekl{\"{o}}v, Peter},
doi = {10.1007/s00442-014-3203-4},
isbn = {0029-8549},
issn = {00298549},
journal = {Oecologia},
keywords = {Communities,Eco-evolutionary feedback,Evolution,Populations,Trophic position},
number = {1},
pages = {103--114},
pmid = {25651804},
publisher = {Springer Berlin Heidelberg},
title = {{Individuals in food webs: the relationships between trophic position, omnivory and among-individual diet variation}},
url = {http://dx.doi.org/10.1007/s00442-014-3203-4},
volume = {178},
year = {2015}
}
@article{Rosenblatt2015,
abstract = {Individual niche specialization (INS) is increas- ingly recognized as an important component of ecological and evolutionary dynamics. However, most studies that have investigated INS have focused on the effects of niche width and inter- and intraspecific competition on INS in small-bodied species for short time periods, with less atten- tion paid to INS in large-bodied reptilian predators and the effects of available prey types on INS. We investigated the prevalence, causes, and consequences of INS in foraging behaviors across different populations of American alliga- tors (Alligator mississippiensis), the dominant aquatic apex predator across the southeast US, using stomach contents and stable isotopes. Gut contents revealed that, over the short term, although alligator populations occupied wide ranges of the INS spectrum, general patterns were apparent. Alligator populations inhabiting lakes exhibited lower INS than coastal populations, likely driven by variation in habi- tat type and available prey types. Stable isotopes revealed that over longer time spans alligators exhibited remark- ably consistent use of variable mixtures of carbon pools (e.g., marine and freshwater food webs). We conclude that INS in large-bodied reptilian predator populations is likely affected by variation in available prey types and habitat het- erogeneity, and that INS should be incorporated into man- agement strategies to efficiently meet intended goals. Also, ecological models, which typically do not consider behav- ioral variability, should include INS to increase model real- ism and applicability.},
author = {Rosenblatt, Adam E. and Nifong, James C. and Heithaus, Michael R. and Mazzotti, Frank J. and Cherkiss, Michael S. and Jeffery, Brian M. and Elsey, Ruth M. and Decker, Rachel A. and Silliman, Brian R. and Guillette, Louis J. and Lowers, Russell H. and Larson, Justin C.},
doi = {10.1007/s00442-014-3201-6},
isbn = {0029-8549},
issn = {00298549},
journal = {Oecologia},
keywords = {Alligator mississippiensis,American alligator,Food web,Stable isotope analysis,Stomach content analysis},
number = {1},
pages = {5--16},
pmid = {25645268},
publisher = {Springer Berlin Heidelberg},
title = {{Factors affecting individual foraging specialization and temporal diet stability across the range of a large “generalist” apex predator}},
url = {http://dx.doi.org/10.1007/s00442-014-3201-6},
volume = {178},
year = {2015}
}
@article{Lile1998,
abstract = {Eighteen species of helminths were analysed in relation to host-parasite specificity and the effect of host ecological preferences on the establishment of the parasite fauna in the alimentary tract of four pleuronectid flatfish, flounder Pleuronectes flesus, witch flounder Glyptocephalus cynoglossus, American plaice Hippoglossoides platessoides, and Atlantic halibut Hippoglossus hippoglossus in northern Norway. Thirteen species were generalists and the diversity of the parasite faunas decreased with increasing depths inhabited by the host. Similarities were greatest between the parasite faunas of flounder and American plaice and least between flounder and witch flounder. Host ecology, rather than phylogenetic relationships of these hosts, mainly influences the composition and diversity of the parasite communities of flatfishes in northern Norway.},
author = {Lile, N. K.},
doi = {10.1006/jfbi.1998.0769},
issn = {00221112},
journal = {Journal of Fish Biology},
keywords = {Depth distribution,Diet,Habitat selection,Macroparasites},
number = {5},
pages = {945--953},
title = {{Alimentary tract helminths of four pleuronectid flatfish in relation to host phylogeny and ecology}},
volume = {53},
year = {1998}
}
@article{Johnson2004b,
abstract = {We used data from parasites and stable isotopes of yellow perch, Perca flavescens, to determine trophic status in four small Canadian Shield lakes as parasites allow the identification of both prey and non-prey dietary components in the host''s community. Stable C isotope ratios for all perch ranged from sim –34 to –19permil while stable N isotope ratios ranged from sim4.5 to 12.5permil. These ranges are larger than those observed in many other fish species. Perch age was the most significant predictor of stable C isotope ratio and perch parasite fauna was the most significant predictor of stable N ratios. In particular, parasite fauna indicative of zooplanktivorous or piscivorous perch were most accurate for predicting fish trophic position and thus stable isotope ratio. Fish length and age showed fewer significant relationships with isotope ratios than parasite infracommunity or diet and suggests that trophic category for perch cannot always be predicted based on size.},
author = {Johnson, Michael W. and Hesslein, Ray H. and Dick, Terry A.},
doi = {10.1007/s10641-004-4189-2},
issn = {03781909},
journal = {Environmental Biology of Fishes},
keywords = {Experimental Lakes Area,carbon isotopes,freshwater fish,nitrogen isotopes,parasite communities,trophic feeding},
number = {4},
pages = {379--388},
title = {{Host length, age, diet, parasites and stable isotopes as predictors of yellow perch (Perca flavescens Mitchill), trophic status in nutrient poor Canadian Shield lakes}},
volume = {71},
year = {2004}
}
@article{Kennedy1986,
abstract = {Recently, some authors (Kennedy, 1981; Price {\&} Clancy, 1983) have$\backslash$nargued that there are fundamental differences between the communities$\backslash$nof helminths in fish and bird hosts. Such differences are foreshadowed$\backslash$nby the work of Dogiel (1964) and are apparent from survey data (e.g.$\backslash$nThrelfall, 1967; Bakke, 1972; Hair {\&} Holmes, 1975 on birds, and compare$\backslash$nChubb, 1963; Mishra {\&} Chubb, 1969; Wootten, 1973; Ingham {\&} Dronen,$\backslash$n1980 on fish). Questions still remain, however, as to whether the$\backslash$ndistinctions are truly justified and whether the differences are$\backslash$nreally fundamental. In this paper, we address these questions by$\backslash$nexamining helminth diversity in a series of hosts. We then discuss$\backslash$nand provide explanations for the observed differences.},
author = {Kennedy, C. R. and Bush, A. O. and Aho, J. M.},
doi = {10.1017/S0031182000049945},
isbn = {1469-8161},
issn = {14698161},
journal = {Parasitology},
number = {1},
pages = {205--215},
pmid = {3748613},
title = {{Patterns in helminth communities: Why are birds and fish different?}},
volume = {93},
year = {1986}
}
@article{Parker2003,
abstract = {The fundamental question of how complex life cycles--where there is typically more than one host-evolve in host--parasite systems remains largely unexplored. We suggest that complex cycles in helminths without penetrative infective stages evolve by two essentially different processes, depending on where in the cycle a new host is inserted. In 'upward incorporation', a new definitive host, typically higher up a food web and which preys on the original definitive host, is added. Advantages to the parasite are avoidance of mortality due to the predator, greater body size at maturity and higher fecundity. The original host typically becomes an intermediate host, in which reproduction is suppressed. In 'downward incorporation', a new intermediate host is added at a lower trophic level; this reduces mortality and facilitates transmission to the original definitive host. These two processes should also apply in helminths with penetrative infective stages, although the mathematical conditions differ.},
author = {Parker, Geoff A. and Chubb, Jimmy C. and Ball, Michael A. and Roberts, Guy N.},
doi = {10.1038/nature02012},
isbn = {0028-0836},
issn = {00280836},
journal = {Nature},
number = {6957},
pages = {480--484},
pmid = {14523438},
title = {{Evolution of complex life cycles in helminth parasites}},
volume = {425},
year = {2003}
}
@article{Fodrie2015,
abstract = {Many mobile marine species are presumed to utilize a broad spectrum of habitats, but this seemingly generalist life history may arise from conspecifics specializing on distinct habitat alternatives to exploit foraging, resting/refuge, or reproductive opportunities. We acoustically tagged 34 red drum, and mapped sand, seagrass, marsh, or oyster (across discrete landscape contexts) use by each uniquely coded individual. Using 144,000 acoustic detections, we recorded differences in habitat use among red drum: proportional use of seagrass habitat ranged from 0 to 100 {\%}, and use of oyster-bottom types also varied among fish. WIC/TNW and IS metrics (previously applied vis-{\`{a}}-vis diet specialization) consistently indicated that a typical red drum overlapped {\textgreater}70 {\%} with population-level niche exploitation. Monte Carlo permutations showed these values were lower than expected had fish drawn from a common habitat-use distribution, but longitudinal comparisons did not provide evidence of temporally consistent individuality, suggesting that differences among individuals were plastic and not reflective of true specialization. Given the range of acoustic detections we captured (from tens to 1,000s per individual), which are substantially larger sample sizes than in many diet studies, we extended our findings by serially reducing or expanding our data in simulations to evaluate sample-size effects. We found that the results of null hypothesis testing for specialization were highly dependent on sample size, with thresholds in the relationship between sample size and associated P-values. These results highlight opportunities and potential caveats in exploring individuality in habitat use. More broadly, exploring individual specialization in fine-scale habitat use suggests that, for mobile marine species, movement behaviors over shorter (≤weeks), but not longer (≥months), timescales may serve as an underlying mechanism for other forms of resource specialization.},
author = {Fodrie, F. Joel and Yeager, Lauren A. and Grabowski, Jonathan H. and Layman, Craig A. and Sherwood, Graham D. and Kenworthy, Matthew D.},
doi = {10.1007/s00442-014-3212-3},
isbn = {0044201432123},
issn = {00298549},
journal = {Oecologia},
keywords = {Foraging theory,Individual specialization,Landscape context,Niche variation,Scieanops ocellatus},
number = {1},
pages = {75--87},
pmid = {25669451},
publisher = {Springer Berlin Heidelberg},
title = {{Measuring individuality in habitat use across complex landscapes: approaches, constraints, and implications for assessing resource specialization}},
url = {http://dx.doi.org/10.1007/s00442-014-3212-3},
volume = {178},
year = {2015}
}
@article{Layman2015,
abstract = {See, stats, and : https : / / www . researchgate . net / publication / 272513462 Individual - level populations : emerging Article DOI : 10 . 1007 / s00442 - 014 - 3209 - y : PubMed CITATIONS 12 READS 185 3 : Some : After : exploring extinction . View Effects Craig Florida 109 , 436 SEE Seth . Newsome University 115 , 173 SEE Tara University 6 SEE All . The .},
author = {Layman, Craig A. and Newsome, Seth D. and {Gancos Crawford}, Tara},
doi = {10.1007/s00442-014-3209-y},
isbn = {0029-8549},
issn = {00298549},
journal = {Oecologia},
number = {1},
pages = {1--4},
pmid = {25690712},
publisher = {Springer Berlin Heidelberg},
title = {{Individual-level niche specialization within populations: emerging areas of study}},
url = {http://dx.doi.org/10.1007/s00442-014-3209-y},
volume = {178},
year = {2015}
}
@article{Knudsen1997,
abstract = {In this study from Fjellfrosvatn, an oligotrophic lake in northern Norway, the parasite communities in two sympatric Arctic charr populations were compared. The dwarf morph, which inhabits the profundal zone, exhibited the lowest parasite diversity, seven species, and 72{\%} of these charr harboured only one or two parasite species. In contrast, 10 parasite species were encountered in the larger normal charr, and between 5 and 8 species were present in 73{\%} of these fish, which also utilised a broader food and habitat niche. Proteocephalus sp. was by far the most abundant species in the dwarf charr, probably because this morph fed intensively upon the benthic copepod Acanthocyclops gigas. On the other hand, parasites that are transmitted with littoral benthic prey (i.e., Phyllodistomum umblae, Cyathocephalus truncatus, Cystidicola farionis, and Crepidostomum spp.) were almost absent in the dwarf charr, though they were common in the normal morph. Also, Diphyllobothrium spp. were more prevalent in the normal charr, and this was attributed to their feeding upon limnetic copepods in the pelagic zone. The only recorded parasite with a direct life cycle, the copepod Salmincola edwardsii, had relatively similar abundances in the two morphs. The considerable differences in parasite community structure and abundance between the two charr populations were closely related to differences in the width and composition of the habitat and food niches between the morphs.},
author = {Knudsen, R. and Kristoffersen, R. and Amundsen, P.-A.},
doi = {10.1139/z97-833},
isbn = {0008-4301},
issn = {0008-4301},
journal = {Canadian Journal of Zoology},
number = {12},
pages = {2003--2009},
title = {{Parasite communities in two sympatric morphs of Arctic charr, {\textless}i{\textgreater}Salvelinus alpinus{\textless}/i{\textgreater} (L.), in northern Norway}},
url = {http://www.nrcresearchpress.com/doi/abs/10.1139/z97-833},
volume = {75},
year = {1997}
}
@article{Morand2000,
abstract = {Ecological factors may influence the number of parasites encountered and, thus, parasite species richness. These factors include diet, gregarity, conspecific and total host density, habitat, body size, vagility, and migration. One means of examining the influence of these factors on parasite species richness is through a comparative analysis of the parasites of different, but related, host species. In contrast to most comparative studies of parasite species richness of fish, which have been conducted by using data from the literature, the present study uses data obtained by the investigators. Coral reef fishes vary widely in the above ecological factors and are frequently parasitized by a diverse array of parasites. We, therefore, chose to investigate how the above ecological factors influence parasite species richness in coral reef fishes. We investigated the endoparasite species richness of 21 species of butterfly fishes (Chaetodontidae) of New Caledonia. We mapped the diet characters on the existing butterfly fish phylogeny and found that omnivory appears to be ancestral. We also mapped the estimated endoparasite species richness, coded from low to high parasite species richness, on the existing butterfly fish phylogeny and found that low parasite species richness appears to be associated with the ancestral state of omnivory. Different dietary and social strategies appear to have evolved more than once, with the exception of obligate coralivory, which appears to have evolved only once. Finally, after controlling for phylogenetic relationships, we found that only the percentage of plankton in the diet and conspecific host density were positively correlated with endoparasite species richness.},
author = {Perpignan, Universitet De and Pacifique, Universitet and Pratique, Ecole and Cedex, Perpignan and Morand, S and Cribb, T H and Kulbicki, M and Rigby, M C and Chauvet, C and Dufour, V and Faliex, E and Galzin, R and Lo, C M and Lo-Yat, a and Pichelin, S and Sasal, P},
issn = {0031-1820},
journal = {Parasitology},
keywords = {Animal,Animal Nutritional Physiological Phenomena,Animals,Diet,Ecosystem,Fish Diseases,Fish Diseases: parasitology,Fishes,Fishes: classification,Fishes: genetics,Fishes: parasitology,Host-Parasite Interactions,Host-Parasite Interactions: physiology,New Caledonia,Parasites,Parasites: classification,Parasites: physiology,Parasitic Diseases,Phylogeny,chaetodontidae,coral reef,host density,host diet,parasite species richness},
pages = {65--73},
pmid = {11085226},
title = {{Endoparasite species richness of new caledonian butterfly fishes: host density and diet matter.}},
url = {http://www.ncbi.nlm.nih.gov/pubmed/11085226},
volume = {121 ( Pt 1},
year = {2000}
}
@article{Marcogliese2002,
abstract = {Helminth parasites of fish in marine systems are often considered to be generalists, lacking host specificity for both intermediate and definitive hosts. In addition, many parasites in marine waters possess life cycles consisting of long-lived larval stages residing in intermediate and paratenic hosts. These properties are believed to be adaptations to the long food chains and the low densities of organisms distributed over broad spatial scales that are characteristic of open marine systems. Moreover, such properties are predicted to lead to the homogenization of parasite communities among fish species. Yet, these communities can be relatively distinct among marine fishes. For benthos, the heterogeneous horizontal distribution of invertebrates and fish with respect to sediment quality and water depth contributes to the formation of distinct parasite communities. Similarly, for the pelagic realm, vertical partitioning of animals with depth will lead to the segregation of parasites among fish hosts. Within each habitat, resource partitioning in terms of dietary preferences of fish further contributes to the establishment of distinct parasite assemblages. Parasite distributions are predicted to be superimposed on distributional patterns of free-living animals that participate as hosts in parasite life cycles. The purpose of this review is first, to summarize distribution patterns of invertebrates and fish in the marine environment and relate these patterns to helminth transmission. Second, patterns of transmission in marine systems are interpreted in the context of food web structure. Consideration of the structure and dynamics of food webs permits predictions about the distribution and abundance of parasites. Lastly, parasites that influence food web structure by regulating the abundance of dominant host species are briefly considered in addition to the effects of pollution and exploitation on food webs and parasite transmission.},
author = {MARCOGLIESE, D. J.},
doi = {10.1017/S003118200200149X},
isbn = {0031-1820},
issn = {0031-1820},
journal = {Parasitology},
keywords = {benthos,fish,food webs,marine,parasites,transmission,zooplankton},
number = {07},
pages = {S83--S99},
pmid = {12396218},
title = {{Food webs and the transmission of parasites to marine fish}},
url = {http://www.journals.cambridge.org/abstract{\_}S003118200200149X},
volume = {124},
year = {2002}
}
@article{Zelmer2014,
abstract = {The tendency to attribute species-area relationships to "island biogeography" effectively bypasses the examination of specific mechanisms that act to structure parasite communities. Positive covariation between fish size and infrapopulation richness should not be examined within the typical extinction-based paradigm, but rather should be addressed from the standpoint of differences in colonization potential among individual hosts. Although most mechanisms producing the aforementioned pattern constitute some variation of passive sampling, the deterministic aspects of the accumulation of parasite individuals by fish hosts makes untenable the suggestion that infracommunities of freshwater fishes are stochastic assemblages. At the component community level, application of extinction-dependent mechanisms might be appropriate, given sufficient time for colonization, but these structuring forces likely act indirectly through their effects on the host community to increase the probability of parasite persistence. At all levels, the passive sampling hypothesis is a relevant null model. The tendency for mechanisms that produce species-area relationships to produce nested subset patterns means that for most systems, the passive sampling hypothesis can be addressed through the application of appropriate null models of nested subset structure.},
author = {Zelmer, Derek A.},
doi = {10.1645/14-534.1},
isbn = {0022-3395},
issn = {0022-3395},
journal = {Journal of Parasitology},
number = {5},
pages = {561--568},
pmid = {24820194},
title = {{Size, Time, and Asynchrony Matter: The Species–Area Relationship for Parasites of Freshwater Fishes}},
url = {http://www.bioone.org/doi/abs/10.1645/14-534.1},
volume = {100},
year = {2014}
}
@article{Locke2013,
abstract = {Disease-mediated threats posed by exotic species to native counterparts are not limited to introduced parasites alone, since exotic hosts frequently acquire native parasites with possible consequences for infection patterns in native hosts. Several biological and geographical factors are thought to explain both the richness of parasites in native hosts, and the invasion success of free-living exotic species. However, the determinants of native parasite acquisition by exotic hosts remain unknown. Here, we investigated native parasite communities of exotic freshwater fish to determine which traits influence acquisition of native parasites by exotic hosts. Model selection suggested that five factors (total body length, time since introduction, phylogenetic relatedness to the native fish fauna, trophic level and native fish species richness) may be linked to native parasite acquisition by exotic fish, but 95{\%} confidence intervals of coefficient estimates indicated these explained little of the variance in parasite richness. Based on R2-values, weak positive relationships may exist only between the number of parasites acquired and either host size or time since introduction. Whilst our results suggest that factors influencing parasite richness in native host communities may be less important for exotic species, it seems that analyses of general ecological factors currently fail to adequately incorporate the physiological and immunological complexity of whether a given animal species will become a host for a new parasite.},
author = {Locke, Sean A. and Mclaughlin, J. Daniel and Marcogliese, David J.},
doi = {10.1111/j.1600-0706.2012.20211.x},
isbn = {0030-1299},
issn = {00301299},
journal = {Oikos},
number = {1},
pages = {73--83},
title = {{Predicting the similarity of parasite communities in freshwater fishes using the phylogeny, Ecology and proximity of hosts}},
volume = {122},
year = {2013}
}
@article{Wilson1996,
abstract = {In some lakes, bluegill sunfish (Lepomis macrochirus, family Centrarchidae) captured from the open water and littoral zone are morphologically and behaviorally specialized for foraging in their respective habitats. Here, we examine the degree to which individuals move between habitats of a single lake by using parasites as an indicator of long-term habitat use. Fish captured from the open water and littoral zone of Holcomb Lake, Michigan, were morphologically different, confirming the results of an earlier study. Twelve parasite species infected bluegill from the littoral zone whereas seven species infected fish from the open water. Two digenetic trematode species (Neascus sp. and Posthodiplostomum minimum) that utilize snails as intermediate hosts were over five times more abundant in littoral-zone fish than in open-water fish, whereas one cestode species (Proteocephalus ambloplitis) that utilizes copepods as intermediate hosts was almost twice as abundant in open-water fish than in littoral-zone fish. Within habitats, there was no relationship between morphology and parasite abundance. Our results suggest that most fish (including those with intermediate morphology) commit themselves to one habitat or the other and that open-water specialists seldom visit the littoral zone.},
author = {Sloan, David and Muzzall, Patrick M and Ehlinger, Timothy J and Wilson, David Sloan},
doi = {10.2307/1446850},
isbn = {00458511},
issn = {00458511},
journal = {Copeia},
number = {2},
pages = {348--354},
title = {{Parasites , Morphology , and Habitat Use in a Bluegill Sunfish ( Lepomis macrochirus ) Population}},
volume = {1996},
year = {1996}
}
@article{Hadfield2010,
abstract = {This was the day we visited two CBOs in Changsha. A couple of things: the first visit was mtg with MSM and straight PLHA, and the issue of marriage; the use of the term "A" to describe AIDS (instead of gan ran zhe); this is the one with the young guy who was HIV positive from his BF, and stopped dating people altogether, and had convo with H about it; issues of being public with your status and/or infection; the issue of getting married even if you are HIV positive; the story of the guy who got divorced and now has a BF; the CBO guy that is in competition with the other Changsha CBO and did workshops for the CDC under GF. The second meeting was with XL, and we talked about the gou mai fu wu that they are doing with the CDC. Some of the main points: fear of ppl finding out in their home towns, would rather forego ART; 30{\%} first time testers; them not accepting fees for the testing, and us convincing them that they should; the details of their gou mai fu wu from the CDC; the pressures of getting married still even if MSM and come out to parents (XL said he will probably marry a lala).},
archivePrefix = {arXiv},
arxivId = {arXiv:1501.0228},
author = {Hadfield, J D},
doi = {10.1016/S1474-4422(11)70296-4},
eprint = {arXiv:1501.0228},
isbn = {6000000200},
issn = {14744422},
journal = {The Lancet Neurology},
keywords = {animal model,linear mixed model,mcmc,multivariate,pedigree,phylogeny},
number = {1},
pages = {31},
pmid = {22028239},
title = {{October, 2011}},
volume = {11},
year = {2012}
}
@article{Novak2015,
abstract = {Many populations consist of individuals that differ substantially in their diets. Quantification of the magnitude and temporal consistency of such intraspecific diet variation is needed to understand its importance, but the extent to which different approaches for doing so reflect instantaneous vs. time-aggregated measures of individual diets may bias inferences. We used direct observations of sea otter individuals (Enhydra lutris nereis) to assess how: (1) the timescale of sampling, (2) under-sampling, and (3) the incidence- vs. frequency-based consideration of prey species affect the inferred strength and consistency of intraspecific diet variation. Analyses of feeding observations aggregated over hourly to annual intervals revealed a substantial bias associated with time aggregation that decreases the inferred magnitude of specialization and increases the inferred consistency of individuals' diets. Time aggregation also made estimates of specialization more sensitive to the consideration of prey frequency, which decreased estimates relative to the use of prey incidence; time aggregation did not affect the extent to which under-sampling contributed to its overestimation. Our analyses demonstrate the importance of studying intraspecific diet variation with an explicit consideration of time and thereby suggest guidelines for future empirical efforts. Failure to consider time will likely produce inconsistent predictions regarding the effects of intraspecific variation on predator-prey interactions.},
author = {Novak, Mark and Tinker, M. Tim},
doi = {10.1007/s00442-014-3213-2},
isbn = {1432-1939
0029-8549},
issn = {00298549},
journal = {Oecologia},
keywords = {Individual variation,Predation,Prey switching,Seasonal prey selection,Time-averaging},
number = {1},
pages = {61--74},
pmid = {25656583},
publisher = {Springer Berlin Heidelberg},
title = {{Timescales alter the inferred strength and temporal consistency of intraspecific diet specialization}},
url = {http://dx.doi.org/10.1007/s00442-014-3213-2},
volume = {178},
year = {2015}
}
@article{Crofton1971,
author = {Ng, Vicky and Sargeant, Jan},
journal = {Parasitology},
number = {2},
pages = {179--193},
title = {{a Quantitative Approach To Prioritizing}},
volume = {62},
year = {2012}
}
@article{Zelmer1998,
author = {Lake, Garner and Zelmer, Author D A and Arai, H P},
journal = {Journal of Parasitology},
number = {1},
pages = {24--28},
title = {{the Contributions of Host Age and Size To the Aggregated Distribution of Parasites in Yellow Perch , Perca Flavescens , From}},
volume = {84},
year = {2015}
}
@article{Curtis1995,
abstract = {Analyses of statistical associations between the stomach contents and endoparasites of Arctic char, Salvelinus alpinus, from a small (9 ha) lake in northern Quebec revealed that food items found in fish stomachs at the time of capture frequently consisted of intermediate hosts for the parasites infecting the fish. Thus the stomach contents of Arctic char infected by Diphyllobothrium ditremum, D. dendriticum, and Eubothrium salvelini tended to include copepods, while fish infected by the digenean Crepidostomum farionis more frequently contained insect larvae (ephemeropterans) and fish infected by the acanthocephalan Echinorhynchus lateralis most often had amphipods in their stomachs. Moreover, strong quantitative associations were evident between parasites utilizing intermediate hosts from either the benthic or the limnetic zone of the lake. This suggests that some degree of persistent feeding specialization was present among members of the Arctic char population over an extended period of time, with individual fish predominantly feeding upon prey organisms in either limnetic or benthic habitats. In this manner an allopatric Arctic char population may function analogously to more diverse fish communities, where specialist foraging behavior is developed to most efficiently exploit the food web.},
author = {Curtis, Mark A. and B{\'{e}}rub{\'{e}}, Michel and Stenzel, Andreas},
doi = {10.1139/f95-526},
isbn = {0706-652X},
issn = {0706-652X},
journal = {Canadian Journal of Fisheries and Aquatic Sciences},
number = {S1},
pages = {186--194},
pmid = {950809977},
title = {{Parasitological evidence for specialized foraging behavior in lake-resident Arctic char ( {\textless}i{\textgreater}Salvelinus alpinus{\textless}/i{\textgreater} )}},
url = {http://www.nrcresearchpress.com/doi/10.1139/f95-526},
volume = {52},
year = {1995}
}
@article{Bell1991,
abstract = {(1) This paper describes and analyses the mean number of parasitic helminth species per host individual among Canadian freshwater fish. (2) Helminth diversity varies among host species. (3) The diversities of different taxa of helminths are correlated both within and among host species. (4) Helminth diversity is positively correlated with host size, longevity, diet and geographical range. Size (or longevity) produces the strongest correlations; when the effect of size is removed, only diet remains correlated with helminth diversity. Size and diet together explain about 40{\%} of variance in helminth diversity among host species; the model is most successful for cestodes, where about 60{\%} of the variance is explained. (5) Most of the ecological covariance of helminth diversity is displayed at rather high taxonomic levels, perhaps because of lag between host evolution and parasite adaptation.},
author = {Bell, Graham and Burt, Austin},
doi = {10.2307/5430},
isbn = {00218790},
issn = {00218790},
journal = {The Journal of Animal Ecology},
number = {3},
pages = {1047},
title = {{The Comparative Biology of Parasite Species Diversity: Internal Helminths of Freshwater Fish}},
url = {https://www.jstor.org/stable/5430?origin=crossref},
volume = {60},
year = {1991}
}
@article{Dick2009,
abstract = {Shorthorn sculpin (Myoxocephalus scorpius) from Frobisher Bay, Baffin Island, is a slow growing long-lived species. A wide range of diet items were present in the stomachs of the shorthorn sculpins sampled but 2-3 diet items (amphipod species) comprised 99.5{\%} of total food consumed, These amphipods were present in the stomachs in similar proportions among all age classes of shorthorn sculpin. Several new host records for parasites were reported and mean numbers of parasite species increased with shorthorn sculpin age. The increased diversity of parasite species and higher delta N-15 values in older/larger individuals suggest that their diets were more diverse and the prey items consumed had higher delta N-15 values. By contrast, the value of delta C-13 in dominant diet items masked the delta C-13 values of minor diet items. We conclude that parasites and stable isotope values provide complementary data on feeding patterns of the shorthorn sculpin. The ubiquitous marine acanthocephalan, Echinorhynchus gadi, was found at high prevalences (87-100{\%}) and mean intensities (28-35), and were localized in the midgut. In contrast to other studies on acanthocephalans, E. gadi did not influence fish condition as measured by condition factor, liver somatic and gonado-somatic indices.},
author = {Dick, R. and Chambers, C. and Gallagher, C.P.},
doi = {10.1051/parasite/2009164297},
isbn = {1252-607X},
issn = {1252607X},
journal = {Parasite},
keywords = {diet,myoxocephalus scorpius,parasites,shorthorn sculpin,stable},
pages = {297--304},
pmid = {20092061},
title = {{Parasites, diet and stable isotopes of shorthorn sculpin (Myoxocephalus scorpius) from Frobisher Bay, Canada}},
url = {https://www.parasite-journal.org/articles/parasite/pdf/2009/04/parasite2009164p297.pdf},
volume = {16},
year = {2009}
}
@article{Lo1998,
abstract = {The parasite communities of three coral-reef fish species (Stegastes nigricans, Dascyllus aruanus and Cephalopholis argus) were on Tiahura reef, French Polynesia. The age and growth of each fish was analysed by otolith increment counts and a significant correlation between these variables was found. Stegastes nigricans was parasitised by six adult parasite species, D. aruanus by two adult parasite species and C. argus by five adult parasite species. The most common parasite species were found in all fish size classes. Ectoparasites showed a positive relationship between their abundance and host body length for all three reef fish species. A positive relationship was found only between host size and parasite abundance for common endoparasite species. Parasite species richness, Brillouins diversity index, and host size and age were positively related. Finally, we discuss the influence of different biological (host diet, host immune response, parasite life-cycle) and ecological factors on parasite community structure in these three reef fishes. Host diet quality seems to be one of the major factors affecting the endoparasite community structure in these reef fishes. Ectoparasite communities seem to be influenced more by biological factors such as, for example, host immunity for the caligid larvae or parasite life-cycle for the gnathiid praniza larvae. In addition, the effect of ecological factors such as cleaning symbiosis on these ectoparasites cannot be dismissed. Copyright (C) 1998 Australian Society for Parasitology.},
author = {Lo, C{\'{e}}drik M. and Morand, Serge and Galzin, Ren{\'{e}}},
doi = {10.1016/S0020-7519(98)00140-4},
isbn = {0020-7519},
issn = {00207519},
journal = {International Journal for Parasitology},
keywords = {Cephalopholis argus,Dascyllus aruanus,Diet,Diversity,Ectoparasites,Endoparasites,Host age,Host size,Species richness,Stegastes nigricans},
number = {11},
pages = {1695--1708},
pmid = {9846606},
title = {{Parasite diversity$\backslash$host age and size relationship in three coral-reef fishes from French Polynesia}},
volume = {28},
year = {1998}
}
@article{Araujo2011,
abstract = {Stone marten (Martes foina Erxleben) seems to be one of the most successful, adaptable mammal. All the studies, dealing with the feeding ecology of this marten, emphasize and verify the fact that it is an omnivorous, euriphagous, opportunistic predator species of the family Mustelidae. Within its huge biogeographical area it found its rule in the food chain of urban environment, too. The frequency and biomass of different items are miscellaneous, according to sources of environment, but the quantity of the urban garbage is insignificant, against its high abundance and easy availability.},
author = {Schlegl, Thomas and Bretterklieber, Thomas and M{\"{u}}hlbacher-Karrer, Stephan and Zangl, Hubert},
doi = {10.1111/j.1461-0248.2011.01662.x},
isbn = {0164-0291},
issn = {11785608},
journal = {International Journal on Smart Sensing and Intelligent Systems},
keywords = {Capacitive sensing,ECT,Leakage effect,Open environment,Reconstruction},
number = {4},
pages = {1579--1594},
pmid = {21790933},
title = {{Simulation of the leakage effect in capacitive sensing}},
volume = {7},
year = {2014}
}
@article{Bolnick2002,
abstract = {Intermolecular effects in the infrared spectra of crystalline n-paraffins, n-C20H42 through n-C30H62 are reported. Absorption bands in the spectra of the triclinic form are singlets. Extensive doubling of bands occurs for the mono clinic and orthorhombic structures in accordance with selection rules based on the factor group symmetry. In particular, the methylene rocking mode bands of the orthorhombic structure display a striking pattern of splitting. Frequency separations of the components can be expressed by a simple function of a single parameter, k$\pi$ (m + 1), where m is the number of methylene groups of the n-paraffin and k is an integer which characterizes a particular intramolecular mode. From these data several intermolecular force constants have been evaluated. In relating these force constants to specific methylene interactions, methylenes in adjacent planes were found to interact more strongly than those in the same plane. By extending Stein's theory that the splitting arises from short range repulsive forces, to include additional interactions, the force constants are explained satisfactorily. However, interactions which involve only dipole-dipole forces are found to be inadequate. {\textcopyright} 1961.},
archivePrefix = {arXiv},
arxivId = {1011.1669v3},
author = {Snyder, Robert G.},
doi = {10.1016/0022-2852(61)90347-2},
eprint = {1011.1669v3},
isbn = {9788578110796},
issn = {1096083X},
journal = {Journal of Molecular Spectroscopy},
keywords = {Adaptive variation,Diet analysis,Gut contents,Individual specialization,Niche variation,Proportional similarity,Resource partitioning},
number = {1-6},
pages = {116--144},
pmid = {25246403},
title = {{Vibrational spectra of crystalline n-paraffins. II. Intermolecular effects}},
volume = {7},
year = {1961}
}
@article{Knudsen2008,
abstract = {The habitat and diet choice and the infection (prevalence and abundance) of trophically transmitted parasites were compared in Arctic charr and brown trout living sympatrically in two lakes in northern Norway. Arctic charr were found in all main lake habitats, whereas the brown trout were almost exclusively found in the littoral zone. In both lakes the parasite fauna reflected the niche segregation between trout and charr. Surface insects were most common in the diet of trout, but transmit few parasites, and accordingly the brown trout had a relatively low diversity and abundance of parasites. Parasites transmitted by benthic prey such as Gammarus and insect larva, were common in both salmonid host species. Copepod transmitted parasites were much more common in Arctic charr, as brown trout did not include zooplankton in their diets. Parasite species that may use small fish as transport hosts, were far more abundant in piscivorous fish, especially brown trout. The seasonal dynamics in parasite infection were also consistent with the developments in the diet throughout the year. The study demonstrates that the structure of parasite communities of charr and the trout is highly dependent on shifts in habitat and diet of their hosts both on an annual base and through the ontogeny, in addition to the observed niche segregation between the two salmonid species.},
author = {Knudsen, Rune and Amundsen, Per Arne and Nilsen, Rune and Kristoffersen, Roar and Klemetsen, Anders},
doi = {10.1007/s10641-007-9216-7},
isbn = {0378-1909},
issn = {03781909},
journal = {Environmental Biology of Fishes},
keywords = {Feeding,Parasite community,Salmo trutta,Salvelinus alpinus},
number = {1},
pages = {107--116},
title = {{Food borne parasites as indicators of trophic segregation between Arctic charr and brown trout}},
volume = {83},
year = {2008}
}
@article{Wainwright1995,
author = {Wainwright, Peter C and Wainwright, Peter C and Richard, Barton A},
doi = {10.1007/BF00005909},
journal = {Environmental Biology of Fishes},
keywords = {allometry,centrarchidae,ecomorphology,gape limited feeding,jaw mechanics,serranidae},
number = {March},
pages = {97--113},
title = {{Predicting patterns of use from morphology of fishes Predicting patterns of prey use from morphology of fishes}},
volume = {44},
year = {2017}
}
@article{Bertrand2008,
abstract = {Stomach contents, parasite assemblages and morphometrics were compared in brook charr Salvelinus fontinalis from the littoral and pelagic zone of two adjacent lakes on the Canadian Shield. In lac Baie des Onze {\^{}} Iles, fish from the littoral zone had greater abundance of benthic prey in their stomach and were more heavily infected by parasites that use intermediate hosts associated with the littoral zone than fish captured in the pelagic zone. Littoral and pelagic brook charr from this lake also differed in regard to body shape and fin length, with each group being anatomically adapted to exploit their respective habitats. The highly significant correlation between morphometric and parasite canonical scores supports the hypothesis of functional diversification of individuals within lac Baie des Onze {\^{}} Iles. While fish from littoral and pelagic zones of lac Caribou did not differ in terms of diet, parasite assemblages or morphometrics, they were different to fish from lac Baie des Onze {\^{}} Iles in that they were less frequently infected with parasites that use gastropods as intermediate hosts, and had shorter pectoral fins. The inter-lake comparisons suggested that parasite assemblages and morpho- metrics of brook charr reflected the dominance of the limnetic and littoral habitats in lacs Caribou and lac Baie des Onze {\^{}} Iles, respectively.},
author = {Bertrand, M. and Marcogliese, D. J. and Magnan, P.},
doi = {10.1111/j.1095-8649.2007.01720.x},
isbn = {0022-1112},
issn = {00221112},
journal = {Journal of Fish Biology},
keywords = {Biological indicators,Competition,Habitat,Intermediate host,Lake morphometry,Population discrimination},
number = {3},
pages = {555--572},
title = {{Trophic polymorphism in brook charr revealed by diet, parasites and morphometrics}},
volume = {72},
year = {2008}
}
@article{Knudsen2003,
abstract = {Infection patterns of trophically transmitted helminth parasites were compared with feeding ecology in two sympatric whitefish Coregonus lavaretus morphs from two lake systems in northern Norway. In both lakes, the pelagic morph was an obligate zooplanktivore, while the benthic morph utilized both the benthivore and zooplanktivore trophic niches. The differences in niche utilization between the two morphs were associated with differences in trophic morphology (gill raker numbers), suggesting that they were genetically dissimilar and reproductively isolated. The benthic morph had the highest number of helminth species, probably because they exhibited a broader niche width compared to the pelagic morph. In both lakes, the species composition and intensities of helminths reflected the trophic diversification of the whitefish ecotypes with respect to different habitat choice (benthic v. pelagic) and dietary specialization (benthivore v. zooplanktivore feeding strategies within the benthic whitefish morph). Zooplanktivorous fish from both morphs acquired parasites mainly from pelagic copepods and in almost equal quantities. The benthivore feeders within the benthic morph had the highest proportion of parasites with transmission stages from benthic organisms. Host feeding behaviour seemed to be a major determinant of the helminth community structure, and helminths appeared to be useful indicators of long-term trophic specialization of whitefish ecotypes},
author = {Knudsen, R. and Amundsen, P. A. and Klemetsen, A.},
doi = {10.1046/j.1095-8649.2003.00069.x},
isbn = {0022-1112},
issn = {00221112},
journal = {Journal of Fish Biology},
keywords = {Coregonus lavaretus,Helminths,Sympatric morphs,Trophic niches},
number = {4},
pages = {847--859},
title = {{Inter- and intra-morph patterns in helminth communities of sympatric whitefish morphs}},
volume = {62},
year = {2003}
}
@article{Marques2011,
abstract = {The extent to which host biology, ecology and phylogeny determine the diversity of macroparasite assemblages has been investigated in recent years in several taxa, including fish. However, consensus has not been reached probably as a result of data being collected from different sources, different temporal scales or host and parasite biogeography and phylogeny having greater influence than expected. The present study evaluates the relative importance of 27 biological, ecological and phylogenetic characteristics of 14 flatfish species on the diversity of their ecto- and endoparasite assemblages, comprising a total of 53 taxa. Redundancy analyses were applied to the mean abundance of each parasite taxa infecting each host and to the richness, taxonomic distinctness and variance in taxonomic distinctness calculated for each assemblage within each host. Only a few host characteristics contributed significantly to the observed patterns: host distribution was more important in determining the type and mean abundance of ectoparasites present in an assemblage, whereas diversity of these assemblages were mainly related to the host's maximum size. Endoparasite mean abundance and diversity were mostly influenced by the number of food items ingested and by the presence of Crustacea and Polychaeta in the diet. However, the sympatric occurrence of related hosts also played an important role in the diversity values found in macroparasite assemblages. Results showed that a host characteristic has different importance according to the host-parasite relationship being examined, suggesting an important role for host-parasite co-evolution on the diversity of extant macroparasite assemblages.},
author = {Marques, J. F. and Santos, M. J. and Teixeira, C. M. and Batista, M. I. and Cabral, H. N.},
doi = {10.1017/S0031182010001009},
isbn = {0031-1820},
issn = {00311820},
journal = {Parasitology},
keywords = {diversity,flatfish,host-characteristics,macroparasites,redundancy analyses,richness},
number = {1},
pages = {107--121},
pmid = {20819241},
title = {{Host-parasite relationships in flatfish (Pleuronectiformes) - The relative importance of host biology, ecology and phylogeny}},
volume = {138},
year = {2011}
}
@article{Hyndes1997,
abstract = {The dietary compositions and breadths of sequential 50 mm size classes of the six whiting species found in nearshore ({\textless}1.5 m), shallow inner-shelf (5 to 15 m) and/or deep inner-shelf (20 to 35 m) waters of the lower west coast of Australia were determined. Comparisons between the results of principal components analysis of head and mouth dimensions and the dietary compositions of Sillago bassensis , S. vittata , S. burrus , S. schomburgkii , S. robusta and Sillaginodespunctata suggests that any differences in the dietary composition of similar-sized representatives of different species, when they occur in the same habitat, are more likely to be due to differences in foraging behaviour than mouth morphology. Classification, ordination and Schoener's overlap indices showed that, in nearshore waters, the juveniles of Sillago bassensis , which colonise relatively exposed areas, have a different diet to those of the smallest representatives of the other whiting species that occupy more sheltered habitats. S. bassensis consumes mainly amphipods, whereas the smaller representatives of S. vittata , S. burrus , S. schomburgkii and Sillaginodes punctata ingest large volumes of copepods, which are typically abundant in protected nearshore waters. Although the mouth dimensions of S. punctata tend to be smaller than those of Sillago schomburgkii , the larger individuals of the former species ingest greater quantities of larger prey, such as crabs and carid shrimps. As S. bassensis , S. vittata and S. burrus increase in size and migrate out into shallow inner-shelf waters, the latter two species tend to concentrate more on benthic prey, while the former species ingests fauna that is more epibenthic. The largest S. bassensis subsequently migrate out into deep inner-shelf waters, where they co-occur with S. robusta , which is restricted to those waters. In these waters, S. bassensis feeds to a far greater extent on large benthic prey, whereas S. robusta consumes a greater quantity of small epibenthic crustaceans, differences that reflect the far larger lengths of the former species in that region. The above data emphasise that the distribution and ontogenetic movements of the six abundant species of whiting play a major role in facilitating a partitioning of food resources amongst these species found in coastal waters of the lower west coast of Australia.},
author = {Hyndes, G. A. and Platell, M. E. and Potter, I. C.},
doi = {10.1007/s002270050125},
isbn = {0025-3162},
issn = {00253162},
journal = {Marine Biology},
number = {4},
pages = {585--598},
pmid = {1640},
title = {{Relationships between diet and body size, mouth morphology, habitat and movements of six sillaginid species in coastal waters: Implications for resource partitioning}},
volume = {128},
year = {1997}
}
@book{mass,
abstract = {... Long 1997; Venables and Ripley 1997), and negative binomial (Long 1997; Venables and Ripley 1997). All three of these regression models are parametric ...},
address = {New York},
archivePrefix = {arXiv},
arxivId = {arXiv:1011.1669v3},
author = {Venables, W. N. and Ripley, B. D.},
doi = {10.1007/978-0-387-21706-2},
edition = {Fourth},
eprint = {arXiv:1011.1669v3},
isbn = {978-1-4419-3008-8},
issn = {0040-1706},
pmid = {2002022925},
publisher = {Springer},
title = {{Modern Applied Statistics with S}},
url = {http://link.springer.com/10.1007/978-0-387-21706-2},
year = {2002}
}
@book{lmerTest,
author = {Kuznetsova, Alexandra and {Bruun Brockoff}, Per and Christensen, Rune H B},
booktitle = { http://CRAN.R-project.org/package=lmerTest},
publisher = {R package version 2.0-11},
title = {{lmerTest: Tests for random and fixed effects for linear mixed effect$\backslash$r  models (lmer objects of lme4 package). R package version 2.0-3.}},
year = {2013}
}
@article{Simkova2001,
abstract = {Local, regional and global influences on the patterns of parasite species richness of 39 freshwater fish species from Central Europe were investigated. Host local abundance and host occurrence were considered respectively as local and regional factors, while host geographical range in longitude and latitude was considered as a global factor. Influences of size, ecology and behavior of hosts were also included in a comparative analysis using the independent contrasts method. We considered host habitat, host diet, host shoaling behavior and mobility. We found a positive relationship between local occurrence of fish and global range of their distribution. We confirmed previous findings showing the importance of host behavior and ecology on the variability of parasite species richness. Second, we showed how a global pattern, such as host geographical range, may affect the variability in parasite species richness through its effects on local abundance and distribution of hosts. A negative relationship between endoparasite species richness and host longitudinal range was found. This suggests that fish with eastern distribution live in the boundary of their distribution in Central Europe far from their center of distribution, which should also be characterized by a higher diversity of parasites.},
author = {Simkov{\'{a}}, Andrea and Morand, Serge and Matejusov{\'{a}}, Iveta and Jurajda, Pavel and Gelnar, Milan},
doi = {10.1023/A:1016658427730},
isbn = {0960-3115},
issn = {09603115},
journal = {Biodiversity and Conservation},
keywords = {Comparative analysis,Fish behavior,Fish distribution,Freshwater fish,Independent contrasts,Local and regional influences,Parasites},
number = {4},
pages = {511--525},
title = {{Local and regional influences on patterns of parasite species richness of central European fishes}},
volume = {10},
year = {2001}
}
@article{Carney2000,
abstract = {Twenty-eight parasite species were recorded from 504 yellow perch (Perca flavescens) collected from Dauphin Lake and Beaufort Lake, Manitoba, and Lake Winnebago, Green Bay, and Lake Michigan, Wisconsin. Four parasite species, Diplostomum spp., Urocleidus adspectus, Proteocephalus pearsei, and Raphidascaris acus, occurred in perch from all localities. Infracommunities and component communities were low in richness. The Dauphin Lake and Beaufort Lake samples had the richest parasite communities, while those in the Green Bay and Lake Michigan samples were the least rich. The effect of host size and age on parasite community structure was equivocal. A positive association between P. pearsei and Bothriocephalus cuspidatus and more multispecies infracommunities than expected provide evidence of nonrandom associations in the Manitoba samples, while the Wisconsin infracommunities were random associations. Significant infracommunity nestedness in all samples indicated nonrandom community organization and structure. Parasite faunas were richer in samples with complex invertebrate communities than in samples with complex fish communities. The trophic status of the aquatic system indirectly affected the parasite communities by limiting the variety of potential intermediate hosts. Predictions regarding relationships between parasite community structure and lake trophic status were not supported. We show that predictable patterns at the fine-scale local level of the parasite infracommunity and component communities of perch are best explained by a rich invertebrate community upon which the host feeds.},
author = {Carney, J.P. and Dick, T.A.},
doi = {10.1139/cjz-78-4-538},
issn = {1480-3283},
journal = {Canadian Journal of Zoology},
number = {4},
pages = {538--555},
pmid = {20000808889},
title = {{Helminth communities of yellow perch ({\textless}i{\textgreater}Perca flavescens{\textless}/i{\textgreater} (Mitchill)): determinants of pattern}},
url = {http://www.nrc.ca/cgi-bin/cisti/journals/rp/rp2{\_}abst{\_}e?cjz{\_}z99-222{\_}78{\_}ns{\_}nf{\_}cjz78-00},
volume = {78},
year = {2000}
}
@article{Klimpel2006,
abstract = {Silver scabbard fish Lepidopus caudatus (Euphrasen, 1788) (Trichiuridae) from the Great Meteor Seamount (GMS) in the central eastern Atlantic were studied for diet composition and metazoan parasites. A total of 36 specimens with lengths between 39.1 and 52.2 cm were sampled, which had taken 14 different prey items belonging to 4 major taxonomic groups (Chaetognatha, Crustacea, Mollusca and Teleostei). The most abundant prey organisms were Myctophidae and Euphausiacea, followed by Copepoda (Calanoida), Decapoda, Chaetognatha and Cephalopoda. Fishes were also the dominant prey in terms of biomass. Cannibalism was observed in 7 specimens of subadult L. caudatus. A total of 11 parasite species were identified in/on L. caudatus. We established 9 new host and 8 new locality records. Infestation rates were congruent with diet composition, indicating that parasites were ingested via mesopelagic prey organisms serving as intermediate hosts. The rich parasite fauna in L. caudatus reflects a high diversity of mesopelagic species at the GMS, providing niches for parasites and their intermediate hosts. While several species such as Paradiplectanotrema lepidopi (Monogenea) and Nybelinia lingualis (Cestoda) are typical parasites of L. caudatus, other species such as Sphyriocephalus tergestinus (Cestoda), Anisakis simplex (Nematoda) and Bolbosoma vasculosum (Acanthocephala) seem to be transferred by hosts migrating into the area, indicating an important role of the GMS in the transoceanic distribution patterns of such parasites},
author = {Klimpel, Sven and R{\"{u}}ckert, Sonja and Piatkowski, Uwe and Palm, Harry W. and Hanel, Reinhold},
doi = {10.3354/meps315249},
issn = {01718630},
journal = {Marine Ecology Progress Series},
keywords = {Biotic relationships,Diet composition,Diversity,Lepidopus caudatus,Metazoan parasites,Seamount,Zoogeography},
pages = {249--257},
title = {{Diet and metazoan parasites of silver scabbard fish Lepidopus caudatus from the Great Meteor Seamount (North Atlantic)}},
volume = {315},
year = {2006}
}
@article{Bolnick2003,
abstract = {Most empirical and theoretical studies of resource use and population dynamics treat conspecific individuals as ecologically equivalent. This simplification is only justified if interindividual niche variation is rare, weak, or has a trivial effect on ecological processes. This article reviews the incidence, degree, causes, and implications of individual-level niche variation to challenge these simplifications. Evidence for individual specialization is available for 93 species distributed across a broad range of taxonomic groups. Although few studies have quantified the degree to which individuals are specialized relative to their population, between-individual variation can sometimes comprise the majority of the population's niche width. The degree of individual specialization varies widely among species and among populations, reflecting a diverse array of physiological, behavioral, and ecological mechanisms that can generate intrapopulation variation. Finally, individual specialization has potentially important ecological, evolutionary, and conservation implications. Theory suggests that niche variation facilitates frequency-dependent interactions that can profoundly affect the population's stability, the amount of intraspecific competition, fitness-function shapes, and the population's capacity to diversify and speciate rapidly. Our collection of case studies suggests that individual specialization is a widespread but underappreciated phenomenon that poses many important but unanswered questions.},
archivePrefix = {arXiv},
arxivId = {arXiv:1011.1669v3},
author = {Bolnick, Daniel I. and Svanb{\"{a}}ck, Richard and Fordyce, James A. and Yang, Louie H. and Davis, Jeremy M. and Hulsey, C. Darrin and Forister, Matthew L.},
doi = {10.1086/343878},
eprint = {arXiv:1011.1669v3},
isbn = {00030147},
issn = {0003-0147},
journal = {The American Naturalist},
keywords = {Animals,Biological,Conservation of Natural Resources,Ecology,Evolution,Genetic Variation,Models,Sample Size,Species Specificity},
number = {1},
pages = {1--28},
pmid = {12650459},
title = {{The Ecology of Individuals: Incidence and Implications of Individual Specialization}},
url = {http://www.journals.uchicago.edu/doi/10.1086/343878},
volume = {161},
year = {2003}
}
@article{Locke2014,
abstract = {Recent studies of aquatic food webs show that parasite diversity is concentrated in nodes that likely favour transmission. Various aspects of parasite diversity have been observed to be correlated with the trophic level, size, diet breadth, and vulnerability to predation of hosts. However, no study has attempted to distinguish among all four correlates, which may have differential importance for trophically transmitted parasites occurring as larvae or adults. We searched for factors that best predict the diversity of larval and adult endoparasites in 4105 fish in 25 species studied over a three-year period in the Bothnian Bay, Finland. Local predator-prey relationships were determined from stomach contents, parasites, and published data in 8,229 fish in 31 species and in seals and piscivorous birds. Fish that consumed more species of prey had more diverse trophically transmitted adult parasites. Larval parasite diversity increased with the diversity of both prey and predators, but increases in predator diversity had a greater effect. Prey diversity was more strongly associated with the diversity of adult parasites than with that of larvae. The proportion of parasite species present as larvae in a host species was correlated with the diversity of its predators. There was a notable lack of association with the diversity of any parasite guild and fish length, trophic level, or trophic category. Thus, diversity is associated with different nodal properties in larval and adult parasites, and association strengths also differ, strongly reflecting the life cycles of parasites and the food chains they follow to complete transmission.},
author = {Locke, Sean A. and Marcogliese, David J. and {Tellervo Valtonen}, E.},
doi = {10.1007/s00442-013-2757-x},
issn = {00298549},
journal = {Oecologia},
keywords = {Food webs,Helminths,Predator-prey,Species richness,Trophic transmission},
number = {1},
pages = {253--262},
pmid = {24026499},
title = {{Vulnerability and diet breadth predict larval and adult parasite diversity in fish of the Bothnian Bay}},
volume = {174},
year = {2014}
}
@article{Knudsen2004,
abstract = {Individual Arctic charr (Salvelinus alpinus) from Fjellfr{\o}svatn, northern Norway, could be categorized by their stomach contents as zooplanktivores or benthivores. Feeding specialization among these fish was evident from negative correlations between helminths transmitted by pelagic copepods (Diphyllobothrium dendriticum and D. ditremum) and those transmitted by the benthic amphipod Gammarus lacustris (Cystidicola farionis and Cyathocephalus truncatus). Occurrences of parasite species acquired from the same types of invertebrate were positively correlated in the fish. Strong relationships among habitat use, diet, and helminth infections among the Arctic charr indicated persistent foraging patterns involving long-term habitat use and feeding specialization. The distribution of all parasite species was highly aggregated in the fish samples, measured by the exponent k of the fitted negative binomial distributions (range: 0.5-7.5) and the variance-to-mean ratios (s2/mean, range: 5-85). Charr specializing on either copepods or Gammarus predominantly contributed to high-intensity class intervals within the overall frequency distributions of the corresponding parasite species. Such fish had low infection intensities of helminths transmitted by other prey organisms. The detailed analyses of the parasite frequency distributions for fish with different habitat or feeding preferences evidently show how heterogeneity in trophic behavior contributes strongly to the commonly observed aggregation of helminths among hosts under natural conditions.},
author = {Knudsen, Rune and Curtis, Mark A. and Kristoffersen, Roar},
doi = {10.1645/GE-3184},
isbn = {0022-3395},
issn = {0022-3395},
journal = {Journal of Parasitology},
number = {1},
pages = {1--7},
pmid = {15040660},
title = {{Aggregation of Helminths: the Role of Feeding Behavior of Fish Hosts}},
url = {http://www.bioone.org/doi/abs/10.1645/GE-3184},
volume = {90},
year = {2004}
}
@article{Kristoffersen1994,
abstract = {This article deals with mathematical tools used for solving equations of the improved mathematical model on the micro-scale for the process of supercritical fluid extraction of essential oils from glandular trichomes. Glandular trichomes are secretory structures of Lamiaceae plant family and as such represent the sites of essential oil synthesis and storage. It was previously noticed that during the extraction with carbon dioxide these secretory structures undergo cracking due to the solvent dissolving into the essential oil phase. In this study, the process of extraction is thoroughly analysed and mathematically presented on the fixed bed scale as well as on the single trichome scale. The finite differences method was applied for solving differential equations of the model. This included dividing the extractor vessel into twenty spatial, and extraction time into ten thousand time increments. Cracking time distribution of glandular trichomes in the form of Gamma distribution was incorporated in each of the twenty spatial increments. The model was applied to simulate experimental results of supercritical extraction from several species of the Lamiaceae family. The deviation of the model results from the experimental data was 9.6-35.7{\%} lower for the improved model than for the model without the cracking time distribution function. {\textcopyright} 2012 Elsevier Ltd.},
author = {Kristoffersen, Knut and Halvorsen, Morten and J{\o}rgensen, Lisbeth},
doi = {10.1139/f94-123},
isbn = {0706-652X},
issn = {0706-652X},
journal = {Canadian Journal of Fisheries and Aquatic Sciences},
number = {6},
pages = {1229--1246},
pmid = {19492483},
title = {{Influence of Parr Growth, Lake Morphology, and Freshwater Parasites on the Degree of Anadromy in Different Populations of Arctic Char ( {\textless}i{\textgreater}Salvelinus alpinus{\textless}/i{\textgreater} ) in Northern Norway}},
url = {http://www.nrcresearchpress.com/doi/abs/10.1139/f94-123},
volume = {51},
year = {1994}
}
@article{Johnson2004a,
abstract = {Seventeen parasite species were recovered from perch in four small Canadian Shield lakes with different fish species compositions. Parasite species such as Urocleidus adspectus Mullier, 1936, Bunodera sacculata Van Cleave and Muller, 1932, and Proteocephalus pearsei La Rue, 1919 are phylogenetically associated with perch and (or) percids; others, such as Crepidostomum cooperi Hopkins, 1931, Spinitectus gracilis Ward and Magath, 1917, and Echinorhynchus salmonis Muller, 1784, are related to dietary sharing; and larval species such as Apophallus brevis Ransom, 1920 and Raphidascaris acus (Bloch, 1779) are phylogenetically tied to perch but also to bird and fish definitive hosts. Variable patterns of dominance were dependent on trophic level and other host species in the system. As matrix fill increased with host age, the dependence of the parasite component communities on the given infracommunities decreased, confirming that predictable component communities depend on repetitive infracommunities. Shifts in dietary preference with age and (or) size and dietary sharing among host species were important in producing repetitive infracommunities. Host diet and age contributed significantly to the rate of parasite species accumulation. Parasite composition at the infracommunity scale changes with local community structure across the watershed and regardless of lake connectivity. The presence of ecologically derived parasite species is strongly influenced by local factors.},
author = {Johnson, M W and Nelson, P A and Dick, T A},
doi = {10.1139/Z04-092},
isbn = {0008-4301},
issn = {0008-4301},
journal = {Canadian Journal of Zoology-Revue Canadienne De Zoologie},
keywords = {DIVERSITY,FAUNA,FLUVIATILIS,FRESH-WATER FISHES,GASTRIC EVACUATION,HELMINTH COMMUNITIES,LAKE,MITCHILL,PATTERNS,RICHNESS},
number = {8},
pages = {1291--1301},
title = {{Structuring mechanisms of yellow perch (Perca flavescens) parasite communities: host age, diet, and local factors}},
volume = {82},
year = {2004}
}
@article{Carney1999,
abstract = {Component communities of perch (Perca fluviatilis L) in Eurasia and the North American yellow perch (Perca flavescens Mitchill) were examined to determine the nature of their parasite communities. The scale of this investigation is continental and includes data collected across the distribution of each host species. Data were compiled from the literature and from 5 sample sites in North America. Four parasite species were found to occur frequently in the helminth component communities of P. flavescens. The cestodes Bothriocephalus cuspidatus and Proteocephalus pearsei, the digenean Crepidostomum cooperi, and the nematode Dichelyne cotylophora comprised a suite of species of which some or all occurred in most samples. Similarly, a group of 4 predictable parasite species was identified for P. fluviatilis in Eurasia, the digenean Bunodera luciopercae, the nematode Camallanus lacustris, the cestode Proteocephalus percae, and the acanthocephalan Acanthocephalus lucii. Specificity was not a requirement for predictability. Despite geographical isolation for millions of years, and different fish species interactions within and between continents, the predictability of these parasite assemblages indicates they are shaped by a biology, especially feeding patterns, common to both perch species. This is evidence that parasite assemblages comprised of nonhost-specific parasites in freshwater fishes are not merely stochastic assemblages but have key components that are predictable at this broad continental scale.},
author = {Carney, J P and Dick, T A},
doi = {10.2307/3285812},
isbn = {0022-3395 (Print)$\backslash$r0022-3395 (Linking)},
issn = {00223395},
journal = {J Parasitol},
keywords = {Animals,Asia,Europe,Fish Diseases/*parasitology,Fresh Water,Helminthiasis, Animal/*parasitology,Helminths/classification/*physiology,Host-Parasite Interactions,Intestinal Diseases, Parasitic/parasitology/*veter,North America,Perches/*parasitology,Software,Species Specificity,Stochastic Processes},
number = {5},
pages = {785--795},
pmid = {10577711},
title = {{Enteric helminths of perch (Perca fluviatilis L.) and yellow perch (Perca flavescens Mitchill): stochastic or predictable assemblages?}},
url = {http://www.ncbi.nlm.nih.gov/pubmed/10577711},
volume = {85},
year = {1999}
}
@article{Albon2002,
abstract = {Even though theoretical models show that parasites may regulate host population densities, few empirical studies have given support to this hypothesis. We present experimental and observational evidence for a host-parasite interaction where the parasite has sufficient impact on host population dynamics for regulation to occur. During a six year study of the Svalbard reindeer and its parasitic gastrointestinal nematode Ostertagia gruehneri we found that anthelminthic treatment in April-May increased the probability of a reindeer having a calf in the next year, compared with untreated controls. However, treatment did not influence the over-winter survival of the reindeer. The annual variation in the degree to which parasites depressed fecundity was positively related to the abundance of O. gruehneri infection the previous October, which in turn was related to host density two years earlier. In addition to the treatment effect, there was a strong negative effect of winter precipitation on the probability of female reindeer having a calf. A simple matrix model was parameterized using estimates from our experimental and observational data. This model shows that the parasite-mediated effect on fecundity was sufficient to regulate reindeer densities around observed host densities.},
author = {Albon, S. D. and Stien, A. and Irvine, R. J. and Langvatn, R. and Ropstad, E. and Halvorsen, O.},
doi = {10.1098/rspb.2002.2064},
isbn = {0962-8452 (Print)$\backslash$n0962-8452 (Linking)},
issn = {14712970},
journal = {Proceedings of the Royal Society B: Biological Sciences},
keywords = {Calving rates,Nematodes,Ostertagia,Rangifer tarandus},
number = {1500},
pages = {1625--1632},
pmid = {12184833},
title = {{The role of parasites in the dynamics of a reindeer population}},
url = {http://www.pubmedcentral.nih.gov/articlerender.fcgi?artid=1691070{\&}tool=pmcentrez{\&}rendertype=abstract},
volume = {269},
year = {2002}
}
@article{Gonzalez2005,
abstract = {The search for consistent patterns of organisation in parasite communities remains a central theme in parasite community ecology. However, to date, much evidence comes from studies without replication in both space and time; when replicate communities are examined, repeatable patterns are rarely observed. Here we determine, using nested subset analyses, whether the infracommunities of ectoparasites and endoparasites of a benthic marine fish (Sebastes capensis) show non-random structure. Then we examine the spatial repeatability of parasite community structure across the host's distribution in the southern Pacific, and the temporal repeatability of ectoparasite community structure from one locality. In total, 537 fish were captured from different latitudes (between 11°S and 52°S) along the Pacific coast of South America; a further 122 specimens were captured in two other years from one of the sampling localities, Valdivia (40°S). In spite of variation in fish sizes among samples, fish size generally did not correlate with either ecto- or endoparasite species richness. The ecto- and endoparasite species richness of the component communities were also not correlated with fish sample size across the nine localities. Significant nested patterns were found in the ectoparasite communities of S. capensis at all eight localities, except at latitude 52°S. Significant nested patterns were also found in the endoparasite infracommunities of S. capensis at seven of the nine localities, the exceptions being those from latitudes 11°S and 20°S. On a temporal scale, significant nestedness was observed in the ectoparasite infracommunities of S. capensis during each of the 3 years of sampling at Valdivia. In general, the same parasite species are responsible for the repeatability of nested patterns, though their importance varies among localities. The spatial and temporal predictability of the parasite community structure in S. capensis may be associated with the fish's benthic habitat and territorial behavior, suggesting that host biology may be a key determinant of the structure of parasite communities. {\textcopyright} 2005 Australian Society for Parasitology Inc. Published by Elsevier Ltd. All rights reserved.},
author = {Gonz{\'{a}}lez, M. T. and Poulin, R.},
doi = {10.1016/j.ijpara.2005.07.016},
isbn = {0020-7519 (Print)$\backslash$r0020-7519 (Linking)},
issn = {00207519},
journal = {International Journal for Parasitology},
keywords = {Infracommunity structure,Nestedness subset analyses,Sebastes capensis,Southeastern Pacific},
number = {13},
pages = {1369--1377},
pmid = {16185695},
title = {{Spatial and temporal predictability of the parasite community structure of a benthic marine fish along its distributional range}},
volume = {35},
year = {2005}
}
@article{Ebert2000,
abstract = {Parasites have been shown to reduce host density and to induce host population extinction in some cases but not in others. Epidemiological models suggest that variable effects of parasites on individual hosts can explain this variability on the population level. Here, we aim to support this hypothesis with a specific epidemiological model using a cross-parasite species approach. We compared the effect of six parasites on host fecundity and survival to their effects on density and risk of extinction of clonal host populations. We contrast our empirical results of population density with predictions from a deterministic model and contrast our empirical results of host and parasite extinction rates with those predicted by a stochastic model. Five horizontally transmitted microparasites (two bacteria: white bacterial disease, Pasteuria ramosa; two microsporidia: Glugoides intestinalis, Ordospora colligata; one fungus: Metschnikowiella biscuspidata); and six strains of a vertically transmitted microsporidium (Flabelliforma magnivora) of the planktonic crustacean Daphnia magna were used. In life table experiments, we quantified fecundity and survival in individual parasitized and healthy hosts and compared these with the effect of the parasites on host population density and on the likelihood of host population extinction in microcosm populations. Parasite species varied strongly in their effects on host fecundity, host survival, host density reduction, and the frequency with which they drove host populations to extinction. The fewer offspring an infected host produced, the lower the density of an infected host population. This effect on host density was relatively stronger for the vertically transmitted parasite strains than for the horizontally transmitted parasites. As predicted by the stochastic simulations, strong effects of a parasite on individual host survival and fecundity increased the risk of host population extinction. The same was true for parasite extinctions. Our results have implications for the use of microparasites in biological control programs and for the role parasites play in driving small populations to extinction.},
author = {Ebert, Dieter and Lipsitch, Marc and Mangin, Katrina L.},
doi = {10.1086/303404},
isbn = {0003-0147},
issn = {0003-0147},
journal = {The American Naturalist},
keywords = {daphnia,experimental epidemiology,host extinction,host regulation,individual population,magna,microparasites},
number = {5},
pages = {459--477},
pmid = {17411497},
title = {{The Effect of Parasites on Host Population Density and Extinction: Experimental Epidemiology with {\textless}i{\textgreater}Daphnia{\textless}/i{\textgreater} and Six Microparasites}},
url = {http://www.journals.uchicago.edu/doi/10.1086/303404},
volume = {156},
year = {2000}
}
@article{Poulin2000,
abstract = {In a meta-analysis, the overall mean correlation between fish length and the intensity of parasitic infections derived from 76 different host-parasite species was positive but weak and non-significant, following corrections for sample size. Whether the parasites were acquired by ingestion or by skin contact had no influence on the strength of the relationship. For cestodes, larval digeneans, and gnathiid isopods, however, the mean correlation between fish length and intensity of infection was significant. Some statistical parameters influenced the strength of the raw correlations computed within samples and thus led to over- or under-estimation of the true relationship. Sample size correlated negatively with the value of the correlation coefficients, whereas range in both fish lengths and intensities of infection correlated positively with the value of the correlation coefficients. Distinguishing between statistical noise and the biological processes shaping the size v. intensity relationship will be important if this relationship is to be incorporated into fish population models. (C) 2000 The Fisheries Society of the British Isles.},
author = {Poulin, R.},
doi = {10.1006/jfbi.1999.1146},
isbn = {0022-1112},
issn = {00221112},
journal = {Journal of Fish Biology},
keywords = {Helminths,Meta-analysis,Parasitism,Sample size,Size-dependent infection},
number = {1},
pages = {123--137},
title = {{Variation in the intraspecific relationship between fish length and intensity of parasitic infection: Biological and statistical causes}},
volume = {56},
year = {2000}
}
@article{Marcogliese2004,
abstract = {Effective management of our natural resources requires an understanding of ecosystem structure and function; effectively, an ecosystem-based approach to management. Parasites occur, albeit cryptically, in almost all ecosystems, yet they are usually neglected in studies on populations and communties of organisms. Parasites can have pronounced or subtle effects on hosts affecting host behavior, growth, fecundity, and mortality. Furthermore, parasites may regulate host population dynamics and influence community structure. Many parasites have complex life cycles and depend for transmission on the presence of a variety of invertebrate and vertebrate intermediate hosts. Often transmission involves predator–prey interactions. Thus, parasites reflect the host's position in the food web and are indicative of changes in ecosystem structure and function. Parasites can provide information on population structure, evolutionary hypotheses, environmental stressors, trophic interactions, biodiversity, and climatic conditions. I use examples from diverse freshwater and marine systems to demonstrate that parasites should be incorporated into research and monitoring programs to maximize information gathered in ecosystem-based studies and resource management.},
author = {Marcogliese, D.J.},
doi = {10.1007/s10393-004-0028-3},
isbn = {1612-9202},
issn = {1612-9202},
journal = {EcoHealth},
keywords = {ecosystems,food webs,fresh water,marine,parasite-induced host mortalit,parasite-induced host mortality,parasitism,stress},
number = {2},
pages = {151--164},
pmid = {24272363},
title = {{Parasites: Small Players with Crucial Roles in the Ecological Theater}},
url = {http://link.springer.com/10.1007/s10393-004-0028-3},
volume = {1},
year = {2004}
}
@article{McMullen1993,
author = {McMullen, C K},
journal = {Pan-Pacific Entomology},
keywords = {Community Galapagos},
pages = {----},
title = {{Flower-visiting insects of the Galapagos Islands. Pan-Pacific Entomologist 69: 95-106F2392}},
volume = {69},
year = {1993}
}
@phdthesis{Montero2005,
author = {Montero, Ana Cast{\~{a}}no},
school = {University of Aarhus, Denmark},
title = {{The Ecology of pollination networks, MSc.-thesis}},
type = {Master's thesis},
year = {2005}
}
@phdthesis{Kevan1970,
author = {Kevan, P G},
doi = {10.1109/ICC.2013.6655289},
isbn = {9781467331227},
issn = {15503607},
pages = {191--205},
pmid = {1202},
school = {University of Alberta, Edmonton, Canada},
title = {{High arctic insect-flower relations: the interrelationships of arthropods and flowers at Lake Hazen, Ellesmere island, NWT, canada.}},
type = {Ph.D. thesis},
year = {1970}
}
@article{Gauzens2015,
abstract = {Within food webs, species can be partitioned into groups according to various criteria. Two notions have received particular attention: trophic groups, which have been used for decades in the ecological literature, and more recently, modules. The relationship between these two group definitions remains unknown in empirical food webs because they have so far been studied separately. While recent developments in network theory have led to efficient methods for detecting modules in food webs, the determination of trophic groups (sets of species that are functionally similar) is based on subjective expert knowledge. Here, we develop a novel algorithm for trophic group detection. We apply this method to several well-resolved empirical food webs, and show that aggregation into trophic groups allows the simplification of food webs while preserving their information content. Furthermore, we reveal a 2-level hierarchical structure where modules partition food webs into large bottom-top trophic pathways whereas trophic groups further partition these pathways into sets of species with similar trophic connections. Bringing together trophic groups and modules provides new perspectives to the study of dynamical and functional consequences of food-web structure, bridging topological analysis and dynamical systems. Trophic groups have a clear ecological meaning in terms of trophic similarity, and are found to provide a trade-off between network complexity and information loss.},
archivePrefix = {arXiv},
arxivId = {1403.2352},
author = {Gauzens, Benoit and Th{\'{e}}bault, Elisa and Lacroix, G{\'{e}}rard and Legendre, St{\'{e}}phane},
doi = {10.1098/rsif.2014.1176},
eprint = {1403.2352},
isbn = {1742-5689},
issn = {17425662},
journal = {Journal of the Royal Society Interface},
keywords = {Clustering method,Community detection,Food webs,Key species,Trophic groups},
number = {106},
pages = {20141176},
pmid = {25878127},
title = {{Trophic groups and modules: Two levels of group detection in food webs}},
volume = {12},
year = {2015}
}
@article{Schleuning2014,
abstract = {Modularity is a recurrent and important property of bipartite ecological networks. Although well-resolved ecological networks describe interaction frequencies between species pairs, modularity of bipartite networks has been analysed only on the basis of binary presence-absence data. We employ a new algorithm to detect modularity in weighted bipartite networks in a global analysis of avian seed-dispersal networks. We define roles of species, such as connector values, for weighted and binary networks and associate them with avian species traits and phylogeny. The weighted, but not binary, analysis identified a positive relationship between climatic seasonality and modularity, whereas past climate stability and phylogenetic signal were only weakly related to modularity. Connector values were associated with foraging behaviour and were phylogenetically conserved. The weighted modularity analysis demonstrates the dominating impact of ecological factors on the structure of seed-dispersal networks, but also underscores the relevance of evolutionary history in shaping species roles in ecological communities.},
author = {Schleuning, Matthias and Ingmann, Lili and Strau{\ss}, Rouven and Fritz, Susanne A. and Dalsgaard, Bo and {Matthias Dehling}, D. and Plein, Michaela and Saavedra, Francisco and Sandel, Brody and Svenning, Jens Christian and B{\"{o}}hning-Gaese, Katrin and Dormann, Carsten F.},
doi = {10.1111/ele.12245},
isbn = {1461-0248 (Electronic)$\backslash$n1461-023X (Linking)},
issn = {14610248},
journal = {Ecology Letters},
keywords = {Avian seed dispersal,Current and past climate,Ecological networks,Evolutionary history,Macroecology,Modularity,Phylogeny,Seasonality,Traits,Weighted bipartite networks},
number = {4},
pages = {454--463},
pmid = {24467289},
title = {{Ecological, historical and evolutionary determinants of modularity in weighted seed-dispersal networks}},
volume = {17},
year = {2014}
}
@article{Peralta2016,
abstract = {The occurrence of complex networks of interactions among species not only relies on species co-occurrence, but also on inherited traits and evolutionary events imprinted in species phylogenies. The phylogenetic signal found in ecological networks suggests that evolution plays an important role in determining community assembly and hence could inform about the underpinning mechanisms. The aim of this study was to review the main findings and methodological approaches used for detecting phylogenetic signal in species interaction networks, particularly in different aspects of network structure: conservatism of interactions, modularity, connectivity and nestedness. In general, studies show that species phylogenies determine interacting partners, module composition, species roles and nested patterns, although these influences are not always consistent across different interaction types. The relative importance of phylogeny to network structure, as well as the scale-dependence of phylogenetic signal, denote key areas for future research. Phylogenetically-informed network ecology represents a promising field for understanding species interaction patterns, community assembly processes and dynamics. It can also provide important information for predicting community changes and improving management practices. This article is protected by copyright. All rights reserved.},
author = {Peralta, Guadalupe},
doi = {10.1111/1365-2435.12669},
isbn = {1365-2435},
issn = {13652435},
journal = {Functional Ecology},
keywords = {antagonistic network,conservatism of interactions,module,mutualistic network,nestedness,network cohesion,phylogenetic signal,phylogeny},
number = {12},
pages = {1917--1925},
title = {{Merging evolutionary history into species interaction networks}},
volume = {30},
year = {2016}
}
@phdthesis{Ingversen2006,
author = {Ingversen, Tanja Toftemark},
school = {University of Aarhus, Denmark},
title = {{Plant-Pollinator Interactions on Jamaica and Dominica}},
type = {Master's thesis},
year = {2006}
}
@article{Barrett1987,
abstract = {ABSTRACT: BACKGROUND: Exosomes consist of membrane vesicles that are secreted by several cell types, including tumors and have been found in biological fluids. Exosomes interact with other cells and may serve as vehicles for the transfer of protein and RNA among cells. METHODS: SKOV3 exosomes were labelled with carboxyfluoresceine diacetate succinimidyl-ester and collected by ultracentrifugation. Uptake of these vesicles, under different conditions, by the same cells from where they originated was monitored by immunofluorescence microscopy and flow cytometry analysis. Lectin analysis was performed to investigate the glycosylation properties of proteins from exosomes and cellular extracts. RESULTS: In this work, the ovarian carcinoma SKOV3 cell line has been shown to internalize exosomes from the same cells via several endocytic pathways that were strongly inhibited at 4 degrees C, indicating their energy dependence. Partial colocalization with the endosome marker EEA1 and inhibition by chlorpromazine suggested the involvement of clathrin-dependent endocytosis. Furthermore, uptake inhibition in the presence of 5-ethyl-N-isopropyl amiloride, cytochalasin D and methyl-beta-cyclodextrin suggested the involvement of additional endocytic pathways. The uptake required proteins from the exosomes and from the cells since it was inhibited after proteinase K treatments. The exosomes were found to be enriched in specific mannose- and sialic acid-containing glycoproteins. Sialic acid removal caused a small but non-significant increase in uptake. Furthermore, the monosaccharides D-galactose, alpha-L-fucose, alpha-D-mannose, D-N-acetylglucosamine and the disaccharide beta-lactose reduced exosomes uptake to a comparable extent as the control D-glucose. CONCLUSIONS: In conclusion, exosomes are internalized by ovarian tumor cells via various endocytic pathways and proteins from exosomes and cells are required for uptake. On the other hand, exosomes are enriched in specific glycoproteins that may constitute exosome markers. This work contributes to the knowledge about the properties and dynamics of exosomes in cancer.},
archivePrefix = {arXiv},
arxivId = {arXiv:nucl-ex/0406003v2},
author = {Barrett, Spencer C. H. and Helenurm, Kaius},
doi = {10.1139/b87-278},
eprint = {0406003v2},
isbn = {1471-2407 (Electronic)$\backslash$r1471-2407 (Linking)},
issn = {0008-4026},
journal = {Canadian Journal of Botany},
number = {10},
pages = {2036--2046},
pmid = {21439085},
primaryClass = {arXiv:nucl-ex},
title = {{The reproductive biology of boreal forest herbs. I. Breeding systems and pollination}},
url = {http://www.nrcresearchpress.com/doi/10.1139/b87-278},
volume = {65},
year = {1987}
}
@book{Clements1923,
address = {Washington, D.C.},
author = {{Clements Frederic E. (Frederic Edward)}, 1874 -1945},
pages = {Page 148},
publisher = {Carnegie Institution of Washington},
title = {{Experimental pollination; an outline of the ecology of flowers and insects, by Frederic E. Clements and Frances L. Long.}},
url = {http://www.biodiversitylibrary.org/page/18070794},
year = {1923}
}
@phdthesis{Bek2006,
author = {Bek, S{\o}ren},
booktitle = {Darwin},
keywords = {Network a,Pollination ecology,Population ecology},
school = {University of Aarhus, Denmark},
title = {{A pollination network from a Danish forest meadow Tabel of content}},
type = {Master's thesis},
year = {2006}
}
@phdthesis{Bundgaard2003,
author = {Bundgaard, M},
school = {University of Aarhus, Denmark},
title = {{Tidslig og rummelig variation i et plante-best{\o}vernetv{\ae}rk. MSc.-thesis.}},
type = {Master's thesis},
year = {2003}
}
@article{Wootton1997,
abstract = {Predicting the dynamics of natural food webs requires estimates of the strength of interactions among species. The ability to estimate per capita interaction strength from observational data is desirable because of the logistical difficulty of using experimental manipulations to obtain such measures for all species within complex natural communities. In this paper, I derive observational measures of per capita interaction strength having units matching those of dynamic food web models (per capita consumption and assimilation rates). I also highlight the difference between per capita interaction strength (a parameter used in theoretical models) and species impact (empirical measures of total species effect). I then use behavioral observations and population censuses in a rocky intertidal community to estimate both per capita interaction strengths and species impacts on invertebrate prey of Glaucous-winged Gulls (Larus glaucescens), American Black Oystercatchers (Haema-topus bachmani), and Northwestern Crows (Corvus caurinus). Estimated per capita inter-action strengths exhibited a skewed distribution with many weak interactions and few strong interactions: mean Ϯ 1 SD of log 10 (interaction strength) ϭ Ϫ1.95 Ϯ 1.40 (bird-day/m of shore) Ϫ1 . Per capita interaction strength correlated poorly (r 2 ϭ 0.152–0.157) and nonlinearly with both consumption rates and percentage contribution of a prey species to the diet. Using my observational estimates of per capita interaction strengths, I predicted the species impact of bird predation on different prey taxa. Predictions included strong effects of birds on goose barnacles (Pollicipes polymerus), limpets (Lottia and Tectura spp.), sea urchins (Strongylocentrotus spp.), and large starfish (Pycnopodia helianthoides and Solaster stimp-soni), but little effect on mussels (Mytilus californianus and M. trossulus), dogwhelk snails (Nucella spp.), and acorn barnacles (Semibalanus cariosus). I compared nine of the pre-dictions with 126 results of experimental manipulations of birds. The predictions agreed both qualitatively and quantitatively with the experimental results. These findings suggest that observational measures of interaction strength that have units matching those of dy-namical food web models may be reasonable to use in estimating those found in natural communities.},
author = {Wotton, J},
doi = {10.1890/0012-9615(1997)067[0045:EATOPC]2.0.CO;2},
issn = {0012-9615},
journal = {Ecological Monographs},
keywords = {communities,corvus caurinus,food webs,foraging behavior,haematopus bachmani,larus glaucescens,population dynamics,predation,rocky intertidal,species interactions},
number = {1},
pages = {45--64},
title = {{Estimates and Tests of Per Capita Interaction Strength :}},
volume = {67},
year = {1997}
}
@article{Reiss2009,
abstract = {Two decades of intensive research have provided compelling evidence for a link between biodiversity and ecosystem functioning (B-EF). Whereas early B-EF research concentrated on species richness and single processes, recent studies have investigated different measures of both biodiversity and ecosystem functioning, such as functional diversity and joint metrics of multiple processes. There is also a shift from viewing assemblages in terms of their contribution to particular processes toward placing them within a wider food web context. We review how the responses and predictors in B-EF experiments are quantified and how biodiversity effects are shaped by multitrophic interactions. Further, we discuss how B-EF metrics and food web relations could be addressed simultaneously. We conclude that addressing traits, multiple processes and food web interactions is needed to capture the mechanisms that underlie B-EF relations in natural assemblages. {\textcopyright} 2009 Elsevier Ltd. All rights reserved.},
author = {Reiss, Julia and Bridle, Jon R. and Montoya, Jos{\'{e}} M. and Woodward, Guy},
doi = {10.1016/j.tree.2009.03.018},
isbn = {0169-5347},
issn = {01695347},
journal = {Trends in Ecology and Evolution},
number = {9},
pages = {505--514},
pmid = {19595476},
title = {{Emerging horizons in biodiversity and ecosystem functioning research}},
volume = {24},
year = {2009}
}
@article{Neutel2002,
abstract = {Increasing evidence that the strengths of interactions among populations in biological communities form patterns that are crucial for system stability requires clarification of the precise form of these patterns, how they come about, and why they influence stability. We show that in real food webs, interaction strengths are organized in trophic loops in such a way that long loops contain relatively many weak links. We show and explain mathematically that this patterning enhances stability, because it reduces maximum "loop weight" and thus reduces the amount of intraspecific interaction needed for matrix stability. The patterns are brought about by biomass pyramids, a feature common to most ecosystems. Incorporation of biomass pyramids in 104 food-web descriptions reveals that the tow weight of the tong loops stabilizes complex food webs. Loop-weight analysis could be a useful toot for exploring the structure and organization of complex communities.},
author = {Neutel, Anje Margriet and Heesterbeek, Johan A.P. and {De Ruiter}, Peter C.},
doi = {10.1074/jbc.M113.501056},
isbn = {0036-8075},
issn = {00219258},
journal = {Science},
number = {5570},
pages = {1120--1123},
pmid = {12004131},
title = {{Stability in real food webs: Weak links in long loops}},
volume = {296},
year = {2002}
}
@article{Levine1980,
abstract = {A continuous measure of "trophic position" is introduced, based on average function of an ecosystem component. Two measures of trophic position variance are defined, and the ideas of trophic specialists and generalists are introduced. The analysis is based on a Markov chain model of energy flows. Along with a number of simple ecosystem structures, the model is also applied to data on the North Sea ecosystem. The model developed in this paper allows for describing in trophic level terms ecosystems which differ substantially from food chains. It is envisioned that such a description would play a useful role in the comparative analysis of ecosystems. {\textcopyright} 1980.},
author = {Levine, Stephen},
doi = {10.1016/0022-5193(80)90288-X},
issn = {10958541},
journal = {Journal of Theoretical Biology},
number = {2},
pages = {195--207},
title = {{Several measures of trophic structure applicable to complex food webs}},
volume = {83},
year = {1980}
}
@article{Dormann2011,
abstract = {The analysis of ecological networks has gained a very prominent foothold in ecology over the last years. While many publications try to elucidate patterns about the networks, others are primarily concerned with the role of specific species in the network. The core challenge here is to tell specialists from generalists. While field data and observations can be used to directly assess specialisation levels, the indirect way through networks is burdened with problems. Here, I review eight measures to quantify specialisation in pollination networks (degree, node specialisation, betweenness, closeness, strength, pollination support, Shannons H and discrimination d), the first four being based on binary, the others on weighted network data. All data and R-code are available as supplement and can be applied beyond pollination networks. The indices convey different concepts of specialisation and hence quantify different aspects. Still, there is some redundancy, with node specialisation and closeness quantifying the same properties, as do degree, betweenness and Shannons H. Using artificial and real network data, I illustrate the interpretation of the different indices and the importance of using a null model to correct for expectations given the different observed frequencies of interactions. For a well-described network the distributions of specialisation values do not differ from null model expectations for most indices. Finally, I investigate the effect of cattle grazing on the specialisation of an important pollinator in eight replicated pollination networks as an illustration of how to employ the specialisation indices, null models and permutation-based statistics in the analysis of specialisation in pollination networks.},
author = {Dormann, Carsten F F CF},
doi = {10.0000/issn-2220-8879-networkbiology-2011-v1-0001},
isbn = {0123456789},
issn = {2220-8879},
journal = {Network Biology},
keywords = {bipartite network,degree,discrimination,index,node specialisation index,pollination service,pollinator,strength,two-mode network},
number = {1},
pages = {1--20},
title = {{How to be a specialist? Quantifying specialisation in pollination networks}},
url = {http://portal.uni-freiburg.de/biometrie/mitarbeiter/dormann/publications-dormann/Dormann2011NetworkBiology.pdf},
volume = {1},
year = {2011}
}
@article{Cousins1987,
abstract = {The development of the trophic level concept1is described and the causes of its failure as a predictive model in ecology are examined. A defence2of and modifications to the trophic level concept are reviewed. A trend towards taxonomic food web analysis is identified leading to models that are independent of the trophic level approach. {\textcopyright} 1987.},
author = {Cousins, Steve},
doi = {10.1016/0169-5347(87)90086-3},
isbn = {L},
issn = {01695347},
journal = {Trends in Ecology and Evolution},
number = {10},
pages = {312--316},
pmid = {21227874},
title = {{The decline of the trophic level concept}},
volume = {2},
year = {1987}
}
@article{Darnell1961,
author = {Darnell, Rezneat M.},
doi = {10.2307/1932242},
issn = {00129658},
journal = {Ecology},
number = {3},
pages = {553--568},
title = {{Trophic Spectrum of an Estuarine Community, Based on Studies of Lake Pontchartrain, Louisiana}},
url = {http://doi.wiley.com/10.2307/1932242},
volume = {42},
year = {1961}
}
@article{Lindeman1942,
abstract = {A sequential injection system which consists of a syringe pump, a selector valve, a multi-port valve, a gas-liquid separator and a solenoid valve for the determination of arsenic by hydride generation atomic absorption spectrometry using tetrahydroborate as reductant was developed. The reduction time of sample with tetrahydoborate has increased by keeping the reactant in gas-liquid separator by using the solenoid valve. Various parameters affecting the performance of the sequential injection system were optimized, including reaction-time, carrier gas flow, sample volume, tetrahydroborate volume and concentration. Established sequential injection hydride generation technique was simple and automated operation. A sample throughput of 112/h was achieved with 400 $\mu$L samples with a precision of 2.0{\%} RSD at 4 $\mu$g/L As (n = 10) and a detection limit of 0.09 $\mu$g/L. Good agreement with the certified values was obtained for the determination of arsenic in standard reference materials.},
archivePrefix = {arXiv},
arxivId = {arXiv:1011.1669v3},
author = {Yin, Xuefeng and Zhang, Jianjun and Wang, Xiaofang},
doi = {10.1017/CBO9781107415324.004},
eprint = {arXiv:1011.1669v3},
isbn = {9788578110796},
issn = {02533820},
journal = {Fenxi Huaxue},
keywords = {Arsenic,Hydride generation atomic absorption spectrometry,Sequential injection},
number = {10},
pages = {1365--1367},
pmid = {25246403},
title = {{Sequential injection analysis system for the determination of arsenic by hydride generation atomic absorption spectrometry}},
volume = {32},
year = {2004}
}
@article{Peterson2011,
abstract = {Aim  To evaluate the evolutionary conservatism of coarse-resolution Grinnellian (or scenopoetic) ecological niches. Location  Global. Methods  I review a broad swathe of literature relevant to the topic of niche conservatism or differentiation, and illustrate some of the resulting insights with examplar analyses. Results  Ecological niche characteristics are highly conserved over short-to-moderate time spans (i.e. from individual life spans up to tens or hundreds of thousands of years); little or no ecological niche differentiation is discernible as part of the processes of invasion or speciation. Main conclusions  Although niche conservatism is widespread, many methodological complications obscure this point. In particular, niche models are frequently over-interpreted: too often, they are based on limited occurrence data in high-dimensional environmental spaces, and cannot be interpreted robustly to indicate niche differentiation.},
archivePrefix = {arXiv},
arxivId = {arXiv:1011.1669v3},
author = {Peterson, A. Townsend},
doi = {10.1111/j.1365-2699.2010.02456.x},
eprint = {arXiv:1011.1669v3},
isbn = {1365-2699},
issn = {03050270},
journal = {Journal of Biogeography},
keywords = {Ecological niche,Evolutionary change,Grinnell,Niche conservatism,Niche identity,Niche similarity,Overfitting,Speciation,Temporal scale},
number = {5},
pages = {817--827},
pmid = {20739916},
title = {{Ecological niche conservatism: A time-structured review of evidence}},
volume = {38},
year = {2011}
}
@book{Chase2003,
abstract = {{\textless}div{\textgreater}Why do species live where they live? What determines the abundance and diversity of species in a given area? What role do species play in the functioning of entire ecosystems? All of these questions share a single core concept—the ecological niche. Although the niche concept has fallen into disfavor among ecologists in recent years, Jonathan M. Chase and Mathew A. Leibold argue that the niche is an ideal tool with which to unify disparate research and theoretical approaches in contemporary ecology.{\textless}br{\textgreater}{\textless}br{\textgreater}Chase and Leibold define the niche as including both what an organism needs from its environment and how that organism's activities shape its environment. Drawing on the theory of consumer-resource interactions, as well as its graphical analysis, they develop a framework for understanding niches that is flexible enough to include a variety of small- and large-scale processes, from resource competition, predation, and stress to community structure, biodiversity, and ecosystem function. Chase and Leibold's synthetic approach will interest ecologists from a wide range of subdisciplines.{\textless}/div{\textgreater}},
address = {Chicago},
author = {Chase, Jonathan M. and Leibold, Mathew A.},
doi = {10.7208/chicago/9780226101811.001.0001},
isbn = {9780226101804},
issn = {6596},
pages = {212},
publisher = {University of Chicago Press},
title = {{Ecological Niches}},
url = {http://www.bibliovault.org/BV.landing.epl?ISBN=9780226101804},
year = {2003}
}
@article{Vazquez2003,
abstract = {The linear preferential attachment hypothesis has been shown to be quite successful to explain the existence of networks with power-law degree distributions. It is then quite important to determine if this mechanism is the consequence of a general principle based on local rules. In this work it is claimed that an effective linear preferential attachment is the natural outcome of growing network models based on local rules. It is also shown that the local models offer an explanation to other properties like the clustering hierarchy and degree correlations recently observed in complex networks. These conclusions are based on both analytical and numerical results of different local rules, including some models already proposed in the literature.},
archivePrefix = {arXiv},
arxivId = {cond-mat/0211528},
author = {V{\'{a}}zquez, Alexei},
doi = {10.1103/PhysRevE.67.056104},
eprint = {0211528},
isbn = {1539-3755 (Print)$\backslash$r1539-3755 (Linking)},
issn = {1063651X},
journal = {Physical Review E - Statistical Physics, Plasmas, Fluids, and Related Interdisciplinary Topics},
number = {5},
pages = {15},
pmid = {12786217},
primaryClass = {cond-mat},
title = {{Growing network with local rules: Preferential attachment, clustering hierarchy, and degree correlations}},
volume = {67},
year = {2003}
}
@article{VanderZanden1996,
author = {Monographs, Source Ecological and Nov, No},
journal = {America},
keywords = {biomagnification,food chain,food web,lake trout,mercury,omnivory,pcbs,smelt,trophic level,trophic position,trophic structure},
number = {4},
pages = {451--477},
title = {{A Trophic Position Model of Pelagic Food Webs : Impact on Contaminant Bioaccumulation in Lake Trout A TROPHIC POSITION MODEL OF PELAGIC FOOD WEBS : IMPACT ON CONTAMINANT BIOACCUMULATION IN LAKE TROUT '}},
volume = {66},
year = {2012}
}
@article{Boecklen2011,
abstract = {Stable isotope analysis (SIA) has proven to be a useful tool in reconstructing diets, characterizing trophic relationships, elucidating patterns of resource allocation, and constructing food webs. Consequently, the number of studies using SIA in trophic ecology has increased exponentially over the past decade. Several subdisciplines have developed, including isotope mixing models, incorporation dynamics models, lipid-extraction and correction methods, isotopic routing models, and compound-specific isotopic analysis. As with all tools, there are limitations to SIA. Chief among these are multiple sources of variation in isotopic signatures, unequal taxonomic and ecosystem coverage, over-reliance on literature values for key parameters, lack of canonical models, untested or unrealistic assumptions, low predictive power, and a paucity of experimental studies. We anticipate progress in SIA resulting from standardization of methods and models, calibration of model parameters through experimentation, and continued development of several recent approaches such as isotopic routing models and compound-specific isotopic analysis.},
author = {Boecklen, William J. and Yarnes, Christopher T. and Cook, Bethany A. and James, Avis C.},
doi = {10.1146/annurev-ecolsys-102209-144726},
isbn = {1543-592X 978-0-8243-1442-2},
issn = {1543-592X},
journal = {Annual Review of Ecology, Evolution, and Systematics},
keywords = {compound-specific isotope analysis,diet reconstru},
number = {1},
pages = {411--440},
pmid = {17136720},
title = {{On the Use of Stable Isotopes in Trophic Ecology}},
url = {http://www.annualreviews.org/doi/10.1146/annurev-ecolsys-102209-144726},
volume = {42},
year = {2011}
}
@article{Petchey2002,
abstract = {Functional diversity is an important component of biodiversity, yet in comparison to taxonomic diversity, methods of quantifying functional diversity are less well developed. Here, we propose a means for quantifying functional diversity that may be particularly useful for determining how functional diversity is related to ecosystem functioning. This measure of functional diversity ''FD'' is defined as the total branch length of a functional dendrogram. Various characteristics of FD make it preferable to other measures of functional diversity, such as the number of functional groups in a community. Simulating species' trait values illustrates how the relative importance of richness and composition for FD depends on the effective dimensionality of the trait space in which species separate. Fewer dimensions increase the importance of community composition and functional redundancy. More dimensions increase the importance of species richness and decreases functional redundancy. Clumping of species in trait space increases the relative importance of community composition. Five natural communities show remarkably similar relationships between FD and species richness.},
author = {Petchey, Owen L. and Gaston, Kevin J.},
doi = {10.1046/j.1461-0248.2002.00339.x},
isbn = {1461-0248},
issn = {1461023X},
journal = {Ecology Letters},
keywords = {Clustering,Community composition,FD,Functional dendrogram,Functional diversity,Functional groups,Functional redundancy,Species identity,Species richness,Traits},
number = {3},
pages = {402--411},
pmid = {3114},
title = {{Functional diversity (FD), species richness and community composition}},
volume = {5},
year = {2002}
}
@article{Bossart2011a,
abstract = {The long-term consequences of climate change and potential environmental degradation are likely to include aspects of disease emergence in marine plants and animals. In turn, these emerging diseases may have epizootic potential, zoonotic implications, and a complex pathogenesis involving other cofactors such as anthropogenic contaminant burden, genetics, and immunologic dysfunction. The concept of marine sentinel organisms provides one approach to evaluating aquatic ecosystem health. Such sentinels are barometers for current or potential negative impacts on individual- and population-level animal health. In turn, using marine sentinels permits better characterization and management of impacts that ultimately affect animal and human health associated with the oceans. Marine mammals are prime sentinel species because many species have long life spans, are long-term coastal residents, feed at a high trophic level, and have unique fat stores that can serve as depots for anthropogenic toxins. Marine mammals may be exposed to environmental stressors such as chemical pollutants, harmful algal biotoxins, and emerging or resurging pathogens. Since many marine mammal species share the coastal environment with humans and consume the same food, they also may serve as effective sentinels for public health problems. Finally, marine mammals are charismatic megafauna that typically stimulate an exaggerated human behavioral response and are thus more likely to be observed.},
author = {Bossart, G},
doi = {10.5670/oceanog.2006.77},
isbn = {1544-2217 (Electronic)$\backslash$r0300-9858 (Linking)},
issn = {10428275},
journal = {Oceanography},
keywords = {as the effects of,better characterized,climate change and potential,degradation are debated and,ecosystem health,environmental,human health,marine mammals,sentinel species,worldwide},
number = {2},
pages = {134--137},
pmid = {21160025},
title = {{Marine Mammals as Sentinal Species for Oceans and Human Health}},
url = {papers2://publication/uuid/D14C1B19-6350-4893-9980-3B726A4F85CA},
volume = {19},
year = {2006}
}
@article{Cabana1994,
abstract = {THE nitrogen pools of animals are enriched in N-15 relative to their food(1), with the top predators having the highest concentrations of this stable isotope(2). The use of delta(15) N to indicate trophic position depends on the degree to which it reflects variation in the underlying food-web structure, rather than variable fractionation along the food chain. Here we compare adult lake trout, a top pelagic predator, from a series of lakes, and find that delta(15)N values vary from 7.5 to 17.5 parts per thousand, a surprisingly wide range for one species. The length of the food chain can explain this variation, supporting the idea that delta(15)N is a food-web descriptor. Food-chain length was measured by the presence or absence of two intermediate trophic levels, pelagic forage fish and the macrozooplankter, Mysis relicta, each of which when present contributes about three delta(15)N units to the trout signature. We find that delta(15)N can be used as a continuous, integrative measure of trophic position, which is supported by its correlation to mercury levels in lake trout.},
archivePrefix = {arXiv},
arxivId = {nature.vol.342.30nov1989},
author = {Cabana, Gilbert and Rasmussen, Joseph B.},
doi = {10.1038/372255a0},
eprint = {nature.vol.342.30nov1989},
isbn = {0028-0836},
issn = {00280836},
journal = {Nature},
number = {6503},
pages = {255--257},
pmid = {22904156},
title = {{Modelling food chain structure and contaminant bioaccumulation using stable nitrogen isotopes}},
volume = {372},
year = {1994}
}
@incollection{Tilman2001,
abstract = {Functional diversity refers to those components of biodiversity that influence how an ecosystem operates or functions. The biological diversity, or biodiversity, of a habitat is much broader and includes all the species living in a site, all of the genotypic and phenotypic variation within each species, and all the spatial and temporal variability in the communities and ecosystems that these species form. Functional diversity, which is a subset of this, is measured by the values and range in the values, for the species present in an ecosystem, of those organismal traits that influence one or more aspects of the functioning of an ecosystem. Functional diversity is of ecological importance because it, by definition, is the component of diversity that influences ecosystem dynamics, stability, productivity, nutrient balance, and other aspects of ecosystem functioning.},
address = {San Diego, CA},
archivePrefix = {arXiv},
arxivId = {arXiv:1011.1669v3},
author = {Tilman, David},
booktitle = {Encyclopedia of Biodiversity: Second Edition},
doi = {10.1016/B978-0-12-384719-5.00061-7},
editor = {Levin, Simon Asher and Colwell, Robert and Daily, Gretchen and Lubchenco, Jane and Mooney, Harold A. and Schulze, Ernst-Detlef and Tilman, G. David},
eprint = {arXiv:1011.1669v3},
isbn = {9780123847195},
issn = {00652598},
keywords = {Ecosystem processes,Functional diversity,Functional group,Sampling effect},
number = {3},
pages = {587--596},
pmid = {20410934},
publisher = {Academic Press},
title = {{Functional Diversity}},
url = {http://www.scielo.br/scielo.php?script=sci{\_}arttext{\&}pid=S0104-59702009000300009{\&}lng=pt{\&}nrm=iso{\&}tlng=pt},
volume = {3},
year = {2013}
}
@article{OReilly2002,
abstract = {Lake Tanganyika, East Africa, has a simple pelagic food chain, and trophic relationships have been established previously from gut-content analysis. Instead of expected isotopic enrichment from phytoplankton to upper level consumers, there was a depletion of N-15 in August 1999. The isotope signatures of the lower trophic levels were an indicator of a recent upwelling event, identified by wind speed and nitrate concentration data, that occurred over a 4-d period several days prior to sampling. The isotope structure of the food web suggests that upwelled nitrate is a nutrient source rapidly consumed by phytoplankton, but the distinctive signature of this nitrate is diluted by time averaging in the upper trophic levels. This time averaging is a consequence of the fact that the isotopic signature of an organism is related to variable nitrogen sources used throughout the life of the organism. This study illustrates the importance of recognizing differences in time averaging among trophic levels.},
author = {O'Reilly, C. M. and Hecky, R. E. and Cohen, A. S. and Plisnier, P. D.},
doi = {10.4319/lo.2002.47.1.0306},
isbn = {0024-3590},
issn = {00243590},
journal = {Limnology and Oceanography},
number = {1},
pages = {306--309},
pmid = {5334650},
title = {{Interpreting stable isotopes in food webs: Recognizing the role of time averaging at different trophic levels}},
volume = {47},
year = {2002}
}
@article{Peterson1987,
abstract = {JSTOR is a not-for-profit service that helps scholars, researchers, and students discover, use, and build upon a wide range of content in a trusted digital archive. We use information technology and tools to increase productivity and facilitate new forms of scholarship. For more information about JSTOR, please contact support@jstor.org.},
author = {Peterson, B J and Fry, B},
doi = {10.1146/annurev.es.18.110187.001453},
isbn = {8711120029},
issn = {0066-4162},
journal = {Annual Review of Ecology and Systematics},
number = {1},
pages = {293--320},
pmid = {17468571},
title = {{Stable Isotopes in Ecosystem Studies}},
url = {http://www.annualreviews.org/doi/10.1146/annurev.es.18.110187.001453},
volume = {18},
year = {1987}
}
@article{VanderZanden2015,
abstract = {Stable isotope analysis is a useful tool to track animal movements in both terrestrial and marine environments. These intrinsic markers are assimilated through the diet and may exhibit spatial gradients as a result of biogeochemical processes at the base of the food web. In the marine environment, maps to predict the spatial distribution of stable isotopes are limited, and thus determining geographic origin has been reliant upon integrating satellite telemetry and stable isotope data. Migratory sea turtles regularly move between foraging and reproductive areas. Whereas most nesting populations can be easily accessed and regularly monitored, little is known about the demographic trends in foraging populations. The purpose of the present study was to examine migration patterns of loggerhead nesting aggregations in the Gulf of Mexico (GoM), where sea turtles have been historically understudied. Two methods of geographic assignment using stable isotope values in known-origin samples from satellite telemetry were compared: (1) a nominal approach through discriminant analysis and (2) a novel continuous-surface approach using bivariate carbon and nitrogen isoscapes (isotopic landscapes) developed for this study. Tissue samples for stable isotope analysis were obtained from 60 satellite-tracked individuals at five nesting beaches within the GoM. Both methodological approaches for assignment resulted in high accuracy of foraging area determination, though each has advantages and disadvantages. The nominal approach is more appropriate when defined boundaries are necessary, but up to 42{\%} of the individuals could not be considered in this approach. All individuals can be included in the continuous-surface approach, and individual results can be aggregated to identify geographic hotspots of foraging area use, though the accuracy rate was lower than nominal assignment. The methodological validation provides a foundation for future sea turtle studies in the region to inexpensively determine geographic origin for large numbers of untracked individuals. Regular monitoring of sea turtle nesting aggregations with stable isotope sampling can be used to fill critical data gaps regarding habitat use and migration patterns. Probabilistic assignment to origin with isoscapes has not been previously used in the marine environment, but the methods presented here could also be applied to other migratory marine species.},
author = {Zanden, Hannah B.Vander and Tucker, Anton D. and Hart, Kristen M. and Lamont, Margaret M. and Fujisaki, Ikuko and Addison, David S. and Mansfield, Katherine L. and Phillips, Katrina F. and Wunder, Michael B. and Bowen, Gabriel J. and Pajuelo, Mariela and Bolten, Alan B. and Bjorndal, Karen A.},
doi = {10.1890/14-0581.1.sm},
isbn = {1051-0761},
issn = {19395582},
journal = {Ecological Applications},
keywords = {Carbon,Caretta caretta,Gulf of Mexico,Isoscapes,Loggerhead,Migration,Nitrogen,Satellite telemetry,Scute,Sea turtle},
number = {2},
pages = {320--335},
pmid = {26263657},
title = {{Determining origin in a migratory marine vertebrate: A novel method to integrate stable isotopes and satellite tracking}},
volume = {25},
year = {2015}
}
@article{Kling1992,
abstract = {Actual food-web structure or function is difficult to determine based on visual observation, gut analyses, or the feeding interactions expected from a given list of species. We used C and N stable-isotope distributions to define food-web structure in arctic lakes, and we compared that structure with results based on more traditional analyses. Although zooplankton species composition was similar across the eight lakes studied, the food-web structure varied greatly. In some lakes the copepod predator Heterocope fed on the herbivorous copepod Diaptomus as expected in a conventional food web. In most lakes, however, @?{\^{}}1{\^{}}5N data were consistent with Heterocope functioning as an herbivore rather than a predator. These inferences were supported by evidence from carbon isotopes and energy-flow data. Our study indicates that only two or three trophic levels exist in the macrozoopolankton of these lakes, in comparison to five or six trophic levels reported in temperate lakes. Isotope analyses showed that actual food-web structure is poorly predicted from simple consideration of species lists and potential trophic interactions.},
author = {O'Brien, W. John and Kling, George W. and Fry, Brian},
doi = {10.2307/1940762},
issn = {00129658},
journal = {Ecology},
keywords = {arctic lakes,copepod predation,energyflow,food web,food webs,omnivory,potential,realized vs,stable isotopes,trophic structure,zooplankton},
number = {2},
pages = {561--566},
title = {{Stable Isotopes and Planktonic Trophic Structure in Arctic Lakes}},
url = {http://www.jstor.org.ezproxy.library.uvic.ca/stable/1940762{\%}5Cnhttp://www.jstor.org.ezproxy.library.uvic.ca/stable/pdfplus/1940762.pdf},
volume = {73},
year = {1992}
}
@article{White1983,
abstract = {The algebraic definitions presented here are motivated by our search for an adequate formalization of the concepts of social roles as regularities in social network patterns. The theorems represent significant homomorphic reductions of social networks which are possible using these definitions to capture the role structure of a network. The concepts build directly on the pioneering work of S.F. Nadel (1957) and the pathbreaking approach to blockmodeling introduced by Lorrain and White (1971) and refined in subsequent years (White, Boorman and Breiger 1976;Boorman and White 1976; Arabie, Boorman and Levitt, 1978; Sailer, 1978). Blockmodeling is one of the predominant techniques for deriving structural models of social networks. When a network is represented by a directed multigraph, a blockmodel of the multigraph can be characterized as mapping points and edges onto their images in a reduced multigraph. The relations in a network or multigraph can also be composed to form a semigroup. In the first part of the paper we examine "graph" homomorphisms, or homomorphic mappings of the points or actors in a network. A family of basic concepts of role equivalence are introduced, and theorems presented to show the structure preserving properties of their various induced homomorphisms. This extends the "classic" approach to blockmodeling via the equivalence of positions. Lorrain and White (1971), Pattison (1980), Boyd (1980, 1982), and most recently Bonacich (1982) have explored the topic taken up in the second part of this paper, namely the homomorphic reduction of the semigroup of relations on a network, and the relation between semigroup and graph homomorphisms. Our approach allows us a significant beginning in reducing the complexity of a multigraph by collapsing relations which play a similar "role" in the network. {\textcopyright} 1983.},
author = {White, Douglas R. and Reitz, Karl P.},
doi = {10.1016/0378-8733(83)90025-4},
isbn = {0378-8733},
issn = {03788733},
journal = {Social Networks},
number = {2},
pages = {193--234},
title = {{Graph and semigroup homomorphisms on networks of relations}},
volume = {5},
year = {1983}
}
@book{Borgatti2013,
abstract = {Borgatti, S.P., Everett, M.G. and Freeman, L.C. 2002. Ucinet for Windows: Software for Social Network Analysis. Harvard, MA: Analytic Technologies.},
address = {London},
author = {Leguina, Adrian},
booktitle = {International Journal of Research {\&} Method in Education},
doi = {10.1080/1743727X.2016.1205836},
isbn = {978-1446247419},
issn = {1743-727X},
number = {4},
pages = {446--447},
publisher = {SAGE Publications Limited},
title = {{Analysing social networks}},
url = {http://www.tandfonline.com/doi/full/10.1080/1743727X.2016.1205836},
volume = {39},
year = {2016}
}
@article{Borgatti2002,
abstract = {This article introduces a new method for statistically comparing pairs of aggregate data series. Aggregate data series refers to a set of values, each of which is averaged or otherwise aggregated across respondents. The motivating problem is the comparison of aggregate proximity matrices, such as those obtained from pile sort exercises. The standard approach to this problem uses the nonparametric, permutation- based quadratic assignment program (QAP) technique. However, the null distribution that QAP is based on is inappropriate for comparing subsamples of a data set and may lead to misleading conclusions. The new method can yield different results than QAP, results more in line with researchers' intuition. Furthermore, the method can be applied to a variety of data types beyond those appropriate for QAP.},
author = {Borgatti, Stephen P.},
doi = {10.1177/1525822X02014001006},
issn = {1525822X},
journal = {Field Methods},
number = {1},
pages = {88--107},
title = {{A Statistical Method for Comparing Aggregate Data Across A Priori Groups}},
volume = {14},
year = {2002}
}
@article{Dehling2016,
abstract = {Species' functional roles in key ecosystem processes such as predation, pollination or seed dispersal are determined by the resource use of consumer species. An interaction between resource and consumer species usually requires trait matching (e.g. a congruence in the morphologies of interaction partners). Species' morphology should therefore determine species' functional roles in ecological processes mediated by mutualistic or antagonistic inter- actions. We tested this assumption for Neotropical plant–bird mutualisms. We used a new analytical framework that assesses a species's functional role based on the analysis of the traits of its interaction partners in a multidimen- sional trait space. We employed this framework to test (i) whether there is correspondence between the morphology of bird species and their functional roles and (ii) whether morphologically specialized birds fulfil specialized functional roles. We found that morphological differences between bird species reflected their functional differences: (i) bird species with different morphologies foraged on distinct sets of plant species and (ii) morphologically distinct bird species fulfilled specialized functional roles. These findings encourage further assessments of species' functional roles through the analysis of their interaction partners, and the proposed analytical framework facilitates a wide range of novel analyses for network and community ecology.},
author = {Dehling, D. Matthias and Jordano, Pedro and Schaefer, H. Martin and B{\"{o}}hning-Gaese, Katrin and Schleuning, Matthias},
doi = {10.1098/rspb.2015.2444},
isbn = {0962-8452},
issn = {14712954},
journal = {Proceedings of the Royal Society B: Biological Sciences},
keywords = {Ecological networks,Fleshy-fruited plants,Frugivorous birds,Functional diversity,Mutualistic interactions,Traits},
number = {1823},
pages = {20152444},
pmid = {26817779},
title = {{Morphology predicts species' functional roles and their degree of specialization in plant–Frugivore interactions}},
volume = {283},
year = {2016}
}
@article{Carscallen2012,
abstract = {Structural or binary approaches, based on presence-absence of feeding links, are the most common method of assembling food webs and form the basis of the most well explored food web models. Binary approaches to assembling feeding links are often criticized as being less powerful and accurate than flow-based methods. To test this assumption we compared binary estimates of trophic position with estimates based on stable isotope values of nitrogen (d 15 N). For 366 species from eight marine and estuarine food webs we compared trophic position estimates based on binary ( presence-absence) feeding links with estimates based on the stable isotope of nitrogen (d 15 N). For a subset of 127 fish species in four of the webs we further compared trophic position estimates based on gut content analysis using a flow-based algorithm using data from FishBase.org with binary andd 15 N estimates. Across all species and webs binary estimates of trophic position were strongly correlated (R¼0.644) withd 15 N estimates. On average binary estimates differed from baseline correctedd 15 N estimates by 2.33{\%}for mean trophic position and 6.57{\%}for maximum trophic position. On average the difference between binaryd 15 N estimates was 0.14 of a trophic level. For the subset of 127 fish species binary estimates performed similarly or more accurately in predictingd 15 N values than the flow-based estimates. Binary approaches to assembling feeding links are often criticized as being less powerful and accurate than flow-based methods. Our results show a high concordance between binary andd 15 N estimates of trophic position as well as showing that in some cases binary estimates are better predictors ofd 15 N than flow-based estimates, reaffirming the robustness of the structural approach to assembling food webs. Additional cross-validation studies in other ecosystems are necessary to determine whether our results can be generalized to terrestrial and freshwater ecosystems.},
author = {Carscallen, W. Mather A. and Vandenberg, Kristen and Lawson, Julia M. and Martinez, Neo D. and Romanuk, Tamara N.},
doi = {10.1890/ES11-00224.1},
isbn = {2150-8925},
issn = {2150-8925},
journal = {Ecosphere},
keywords = {binary,cross-validation,feeding interactions,fish,food webs,gut content analysis,nitrogen,stable isotope,structural,trophic position},
number = {3},
pages = {art25},
title = {{Estimating trophic position in marine and estuarine food webs}},
url = {http://doi.wiley.com/10.1890/ES11-00224.1},
volume = {3},
year = {2012}
}
@article{Allesina2011,
abstract = {Few food web theory hypotheses/predictions can be readily tested using likelihoods of reproducing the data. Simple probabilistic models for food web structure, however, are an exception as their likelihoods were recently derived. Here I test the performance of a more complex model for food web structure that is grounded in the allometric scaling of interactions with body size and the theory of optimal foraging (Allometric Diet Breadth Model-ADBM). This deterministic model has been evaluated by measuring the fraction of trophic relations it correctly predicts. I contrasted this value with that produced by simpler models based on body sizes and found that the quantitative information on allometric scaling and optimal foraging does not significantly increase model fit. Also, I present a method to compute the p-value for the fraction of trophic interactions correctly predicted by the ADBM, or any other model, with respect to three probabilistic models. I find that the ADBM predicts significantly more links than random graphs, but other models can outperform it. Although optimal foraging and allometric scaling may improve our understanding of food webs, the ADBM needs to be modified or replaced to find support in the data. {\textcopyright} 2010 Elsevier Ltd.},
archivePrefix = {arXiv},
arxivId = {0911.2021},
author = {Allesina, Stefano},
doi = {10.1016/j.jtbi.2010.06.040},
eprint = {0911.2021},
isbn = {0022-5193},
issn = {00225193},
journal = {Journal of Theoretical Biology},
keywords = {Allometric relation,Food webs,Likelihoods,Model selection,Optimal foraging},
number = {1},
pages = {161--168},
pmid = {20633565},
publisher = {Elsevier},
title = {{Predicting trophic relations in ecological networks: A test of the Allometric Diet Breadth Model}},
url = {http://dx.doi.org/10.1016/j.jtbi.2010.06.040},
volume = {279},
year = {2011}
}
@article{Wootton1994,
abstract = {KEY WORDS species interactions. competition. predation, mutualism. interaction. modifications Abstract Indirect effects occur when the impact of one species on another requires the presence of a third species. They can arise in two general ways: through linked chains of direct interactions, and when a species changes the interactions among species. Indirect effects have been uncovered largely by experimental studies that have monitored the response of many species and discovered "unexpected results," although some studies have looked for specific indirect effects predicted from simple models. The characteristics of such approaches make it likely that the many indirect effects remain uncovered, but the appli­ cation of techniques such as path analysis may reduce this problem. Determin­ istic theory indicates that indirect effects should often be important, although stochastic models need exploration. Simulation models indicate that some indirect effects may stabilize multi-species assemblages. Five simple types of indirect effects have been regularly demonstrated in nature: exploitative com­ petition, trophic cascades, apparent competition, indirect mutualism, and in­ teraction modifications. Detailed experimental investigations of natural com­ munities have yielded complicated effects. Indirect effects have the potential to affect evolutionary patterns, but empirical examples are limited. Future directions in the study of indirect effects include developing techniques to estimate interaction strength in dynamic models, deriving more efficient ap­ proaches to detecting indirect effects, evaluating the effectiveness of ap-443 0066-4162/9411120-0443{\$}05.00},
archivePrefix = {arXiv},
arxivId = {0066-4162/9411120-0443},
author = {Wootton, J. T.},
doi = {10.1146/annurev.es.25.110194.002303},
eprint = {9411120-0443},
file = {:Users/alyssacirtwill/Documents/Papers/Wootton{\_}1994{\_}Annual Review of Ecology and Systematics.pdf:pdf},
isbn = {0066-4162},
issn = {0066-4162},
journal = {Annual Review of Ecology and Systematics},
keywords = {competition,interaction,mutualism,predation,species interactions},
number = {1},
pages = {443--466},
pmid = {525},
primaryClass = {0066-4162},
title = {{The Nature and Consequences of Indirect Effects in Ecological Communities}},
url = {http://ecolsys.annualreviews.org/cgi/doi/10.1146/annurev.ecolsys.25.1.443},
volume = {25},
year = {1994}
}
@article{Zook2011,
abstract = {Food webs, the networks describing "who eats whom" in an ecosystem, are nearly interval, i.e. there is a way to order the species so that almost all the resources of each consumer are adjacent in the ordering. This feature has important consequences, as it means that the structure of food webs can be described using a single (or few) species' traits. Moreover, exploiting the quasi-intervality found in empirical webs can help build better models for food web structure. Here we investigate which species trait is a good proxy for ordering the species to produce quasi-interval orderings. We find that body size produces a significant degree of intervality in almost all food webs analyzed, although it does not match the maximum intervality for the networks. There is also a great variability between webs. Other orderings based on trophic levels produce a lower level of intervality. Finally, we extend the concept of intervality from predator-centered (in which resources are in intervals) to prey-centered (in which consumers are in intervals). In this case as well we find that body size yields a significant, but not maximal, level of intervality. These results show that body size is an important, although not perfect, trait that shapes species interactions in food webs. This has important implications for the formulation of simple models used to construct realistic representations of food webs. {\textcopyright} 2010 Elsevier Ltd.},
author = {Zook, Alexander E. and Eklof, Anna and Jacob, Ute and Allesina, Stefano},
doi = {10.1016/j.jtbi.2010.11.045},
isbn = {0022-5193},
issn = {00225193},
journal = {Journal of Theoretical Biology},
keywords = {Body size,Ecological networks,Food web structure,Intervality,Niche dimension},
number = {1},
pages = {106--113},
pmid = {21144853},
publisher = {Elsevier},
title = {{Food webs: Ordering species according to body size yields high degree of intervality}},
url = {http://dx.doi.org/10.1016/j.jtbi.2010.11.045},
volume = {271},
year = {2011}
}
@article{Rohr2010,
abstract = {A stochastic epidemic model featuring fixed-length latent periods, gamma-distributed infectious periods and randomly varying heterogeneity among susceptibles is considered. A Markov chain Monte Carlo algorithm is developed for performing Bayesian inference for the parameters governing the infectious-period length and the hyper-parameters governing the heterogeneity of susceptibility. This method of analysis applies to a wider class of diseases than methods proposed previously. An application to smallpox data confirms results about heterogeneity suggested by an earlier analysis that relied on less realistic assumptions.},
author = {Rohr, Rudolf Philippe and Scherer, Heike and Kehrli, Patrik and Mazza, Christian and Bersier, Louis‐F{\'{e}}lix},
doi = {10.1086/653667},
isbn = {0003-0147},
issn = {0003-0147},
journal = {The American Naturalist},
keywords = {biological network,body,community structure,latent variable,phylogeny,size,statistical model},
number = {2},
pages = {170--177},
pmid = {12933559},
title = {{Modeling Food Webs: Exploring Unexplained Structure Using Latent Traits}},
url = {http://www.journals.uchicago.edu/doi/10.1086/653667},
volume = {176},
year = {2010}
}
@article{Williams2008,
abstract = {Four models of network structure are combined with models of bioenergetic dynamics to study the role of food web topology and nonlinear dynamics on species coexistence in complex ecological networks. Network models range from the highly structured niche model to loosely constrained energetically feasible random networks. Bioenergetic models differ in how they represent primary production, functional responses, and consumption by generalists. Network structure weakly influenced the ability of species to coexist. Species persistence is strongly affected by functional responses and generalists' consumption rates but weakly affected by models and amounts of primary production. Despite these generalities, specific mechanisms that determine persistence under one dynamical regime, such as top-down control by consumers, may play an insignificant role under different dynamical conditions. Future research is needed to strengthen the weak empirical basis for various functional forms and parameter values that strongly influence whether species can coexist in complex food webs.},
author = {Williams, Richard J.},
doi = {10.1007/s12080-008-0013-5},
isbn = {1208000800135},
issn = {18741738},
journal = {Theoretical Ecology},
keywords = {Bioenergetic dynamics,Food web,Persistence,Stability},
number = {3},
pages = {141--151},
title = {{Effects of network and dynamical model structure on species persistence in large model food webs}},
volume = {1},
year = {2008}
}
@article{Williams2010,
abstract = {The niche model has been widely used to model the structure of complex food webs, and yet the ecological meaning of the single niche dimension has not been explored. In the niche model, each species has three traits, niche position, diet position and feeding range. Here, a new probabilistic niche model, which allows the maximum likelihood set of trait values to be estimated for each species, is applied to the food web of the Benguela fishery. We also developed the allometric niche model, in which body size is used as the niche dimension. About 80{\%} of the links in the empirical data are predicted by the probabilistic niche model, a significant improvement over recent models. As in the niche model, species are uniformly distributed on the niche axis. Feeding ranges are exponentially distributed, but diet positions are not uniformly distributed below the predator. Species traits are strongly correlated with body size, but the allometric niche model performs significantly worse than the probabilistic niche model. The best-fit parameter set provides a significantly better model of the structure of the Benguela food web than was previously available. The methodology allows the identification of a number of taxa that stand out as outliers either in the model's poor performance at predicting their predators or prey or in their parameter values. While important, body size alone does not explain the structure of the one-dimensional niche.},
author = {Williams, Richard J. and Anandanadesan, Ananthi and Purves, Drew},
doi = {10.1371/journal.pone.0012092},
file = {:Users/alyssacirtwill/Documents/Papers/Williams, Anandanadesan, Purves{\_}2010{\_}PLoS ONE.pdf:pdf},
isbn = {1932-6203},
issn = {19326203},
journal = {PLoS ONE},
number = {8},
pages = {e12092},
pmid = {20711506},
title = {{The probabilistic niche model reveals the niche structure and role of body size in a complex food web}},
volume = {5},
year = {2010}
}
@article{Rohr2016,
abstract = {Networks play a prominent role in the study of complex systems of interacting entities in biology, sociology, and economics. Despite this diversity, we demonstrate here that a statistical model decomposing networks into matching and centrality components provides a comprehensive and unifying quantification of their architecture. First we show, for a diverse set of networks, that this decomposition provides an extremely tight fit to observed networks. Consequently, the model allows very accurate prediction of missing links in partially known networks. Second, when node characteristics are known, we show how the matching-centrality decomposition can be related to this external information. Consequently, it offers a simple and versatile tool to explore how node characteristics explain network architecture. Finally, we demonstrate the efficiency and flexibility of the model to forecast the links that a novel node would create if it were to join an existing network.},
archivePrefix = {arXiv},
arxivId = {1310.4633},
author = {Rohr, Rudolf P. and Naisbit, Russell E. and Mazza, Christian and Bersier, Louis F{\'{E}}lix},
doi = {10.1098/rspb.2015.2702},
eprint = {1310.4633},
file = {:Users/alyssacirtwill/Documents/Papers/Rohr et al.{\_}2016{\_}Proceedings of the Royal Society B Biological Sciences.pdf:pdf},
issn = {14712954},
journal = {Proceedings of the Royal Society B: Biological Sciences},
keywords = {Complex networks,Ecological networks,Metabolic networks,Missing links,Predicting networks,Social networks},
number = {1824},
pages = {20152702},
pmid = {26842568},
title = {{Matching-centrality decomposition and the forecasting of new links in networks}},
url = {http://arxiv.org/abs/1310.4633{\%}5Cnhttp://rspb.royalsocietypublishing.org/lookup/doi/10.1098/rspb.2015.2702},
volume = {283},
year = {2016}
}
@article{Dalsgaard2009,
abstract = {Floral phenotype and pollination system of a plant may be influenced by the abiotic environment and the local pollinator assemblage. This was investigated in seven plant-hummingbird assemblages on the West Indian islands of Grenada, Dominica and Puerto Rico. We report all hummingbird and insect pollinators of 49 hummingbird-pollinated plant species, as well as six quantitative and semi-quantitative floral characters that determine visitor restriction, attraction and reward. Using nonmetric multidimensional scaling analysis, we show that hummingbird-pollinated plants in the West Indies separate in floral phenotypic space into two gradients—one associated with the abiotic environment and another with hummingbird size. Plants pollinated by large, long-billed hummingbirds had flowers with long corolla tube, large amounts of nectar and showy orange-red colouration. These attracted few or no insect species, whereas plants pollinated by small, short-billed hummingbirds were frequently pollinated by insects, particularly lepidopterans. The separation of plants related to environmental factors showed that species in the wet and cold highlands produced large amounts of dilute nectar, possessed no or a weak odour, and were associated with few insects, particularly few hymenopterans, compared to plants in the dry and warm lowlands. The most specialised hummingbird-pollinated plants are found in the West Indian highlands where they are pollinated by mainly large, long-billed hummingbirds. At the other extreme, highly generalised plants growing in the dry and warm lowlands are pollinated by small, short-billed hummingbirds and numerous insect species. This illustrates that, even within the hummingbird-pollinated flora, pollination syndrome and the degree of specialisation may vary tremendously depending on pollinator morphology and environment.},
author = {Dalsgaard, Bo and {Mart{\'{i}}n Gonz{\'{a}}lez}, Ana M. and Olesen, Jens M. and Ollerton, Jeff and Timmermann, Allan and Andersen, Laila H. and Tossas, Adrianne G.},
doi = {10.1007/s00442-008-1255-z},
isbn = {00298549},
issn = {00298549},
journal = {Oecologia},
keywords = {Floral phenotype,Insect pollinator,Ornithophilous syndrome,Plant-pollinator interaction,Specialisation gradient},
number = {4},
pages = {757--766},
pmid = {19132403},
title = {{Plant-hummingbird interactions in the West Indies: Floral specialisation gradients associated with environment and hummingbird size}},
volume = {159},
year = {2009}
}
@article{Stiles1975,
abstract = {Nine hummingbird-pollinated species of Heliconia occur together at Finca La Selva, in the wet Caribbean lowlands of Costa Rica. In forest habitats, Heliconia clumps (clones) are typically small; in more open areas, many clumps attain large size. This probably reflects differences in light intensity and degree of vegetative competition in these habitats. Nine species of hummingbirds regularly visit Heliconia flowers at La Selva. The four hermits are nonterritorial, traplining foragers with long, curved bills. Non-hermits frequently hold territories at Heliconia clumps, and have short, straight bills. Pollination by hermits tends to produce more cross-pollination; territorial hummingbirds increase self-pollination. Different Heliconia species appear to be specialized for pollination by either hermits on non-hermits, largely through components of the caloric phenotype: amount and timing of nectar production, rate of inflorescene and flower production, and morphological paramerters that affect the energetic efficiency of nectar-harvesting hummingbirds. Habitat may influence pollination systems through its effects on clump size and thus on the number of flowers a clump can have at any one time. Ultimately, specialization for hermits or non-hermits may depend on the degree of self-compatibility of the different Heliconia species. Hermit-pollinated Heliconia mostly show sequential and nonoverlapping flowering peaks, probably resulting from competition for pollinators and/or selection against hybridization. Two hermit-pollinated species bloom simultanesoulsy, thereby inducing the birds to utilize an other-wise little-used microhabitat. Heliconia species pollinated by non-hermits bloom in the early to middle rainy season, and are mostly separated by habitat. Isolating mechanisms among sympatric Heliconia species involve both spatial and temporal patterns of partioning available pollinators. Floral parameters include mechanical (different site of pollen deposition on the bird) and ethological (caloric and visual factors affecting flower choice) mechanisms. Selection for pollinator specificity may result in convergence of blooming peaks, provided that other isolating mechanisms are present. Human activity has broken down some habitat barriers by producing large areas of second growth.},
author = {Stiles, F. Gary},
doi = {10.2307/1934961},
isbn = {0012-9658},
issn = {00129658},
journal = {Ecology},
number = {2},
pages = {285--301},
title = {{Ecology, Flowering Phenology, and Hummingbird Pollination of Some Costa Rican Heliconia Species}},
url = {http://doi.wiley.com/10.2307/1934961},
volume = {56},
year = {1975}
}
@article{Wolf1976,
abstract = {(1) We studied the ecological organization of a four-species hummingbird community in the highlands of Costa Rica, Central America. (2) Three of the four bird species (Colibri, Eugenes, Selasphorus) have marked seasonal cycles of abundance while one species (Panterpe) is a year-long resident. Panterpe breeds during the rainy season when an important food plant blooms before the arrival of most of the migratory bird populations. These arrive and breed in relation to blooming of other plant species. (3) The greatest diversity and density of hummingbirds occurs during the dry season which is also the time of peak bloom of the flowers visited by the hummingbirds. (4) Two of the four species (Colibri, Eugenes) tend to visit few plant species while two (Panterpe, Selasphorus) visit more plant species during a year. (5) When the bird species co-occur the principal division of the nectar resource involves: (a) a bill-corolla size interaction; (b) efficiency differences in exploiting the nectar; (c) dominance interactions among the bird species; and (d) the availability of alternative resources. (6) The efficiency of foraging is influenced by co-evolutionary selective forces. (7) The availability of several pollinator species has apparently led to co-evolutionary relationships that further tend to restrict the resource range used by a bird species. (8) Panterpe, the dominant species, is viewed as the organizer member of the guild. Its population size in relation to nectar availability effectively determines the ability of individuals of other species to maintain a position in the guild.},
author = {Wolf, Larry L and Stiles, F Gary and Hainsworth, F Reed},
doi = {10.2307/3879},
isbn = {00218790},
issn = {00218790},
journal = {Journal of Animal Ecology},
number = {2},
pages = {349--379},
pmid = {17182748},
title = {{Ecological organization of a tropical, highland hummingbird community}},
url = {http://www.jstor.org/stable/3879},
volume = {45},
year = {1976}
}
@article{Furness1988,
abstract = {On Foula, Shetland, sheep were observed biting off the legs, wings or head of unfledged young Arctic terns. Large numbers of tern chicks and a few Arctic skua chicks were found with amputations characteristic of these attacks by sheep. On Rhum, Inner Hebrides, red deer were watched biting the heads off manx shearwater chicks and occasionally also chewing the shearwaters' legs and wings to excise bone. Killing of birds and the selective ingestion of bone-rich parts by ruminants has not previously been widely documented. It is presumably a response to mineral deficiencies in the vegetation, and it may only occur in rare situations where ruminants feed on mineral-deficient vegetation on which there are dense colonies of ground-nesting birds.},
author = {FURNESS, R. W.},
doi = {10.1111/j.1469-7998.1988.tb02451.x},
isbn = {1469-7998},
issn = {14697998},
journal = {Journal of Zoology},
number = {3},
pages = {565--573},
title = {{Predation on ground‐nesting seabirds by island populations of red deer Cervus elaphus and sheep Ovis}},
volume = {216},
year = {1988}
}
@article{Stang2009,
abstract = {Background and Aims$\backslash$r$\backslash$nMany recent studies show that plant–pollinator interaction webs exhibit consistent structural features such as long-tailed distributions of the degree of generalization, nestedness of interactions and asymmetric interaction dependencies. Recognition of these shared features has led to a variety of mechanistic attempts at explanation. Here it is hypothesized that beside size thresholds and species abundances, the frequency distribution of sizes (nectar depths and proboscis lengths) will play a key role in determining observed interaction patterns.$\backslash$r$\backslash$nMethods$\backslash$r$\backslash$nTo test the influence of size distributions, a new network parameter is introduced: the degree of size matching between nectar depth and proboscis length. The observed degree of size matching in a Spanish plant–pollinator web was compared with the expected degree based on joint probability distributions, integrating size thresholds and abundance, and taking the sampling method into account.$\backslash$r$\backslash$nKey Results$\backslash$r$\backslash$nNectar depths and proboscis lengths both exhibited right-skewed frequency distributions across species and individuals. Species-based size matching was equally close for plants, independent of nectar depth, but differed significantly for pollinators of dissimilar proboscis length. The observed patterns were predicted well by a model considering size distributions across species. Observed size matching was closer when relative abundances of species were included, especially for flowers with openly accessible nectar and pollinators with long proboscises, but was predicted somewhat less successfully by the model that included abundances.$\backslash$r$\backslash$nConclusions$\backslash$r$\backslash$nThe results suggest that in addition to size thresholds and species abundances, size distributions are important for understanding interaction patterns in plant–pollinator webs. It is likely that the understanding will be improved further by characterizing for entire communities how nectar production of flowers and energetic requirements of pollinators covary with size, and how sampling methods influence the observed interaction patterns.},
author = {Stang, Martina and Klinkhamer, Peter G.L. and Waser, Nickolas M. and Stang, Ingo and {Van Der Meijden}, Eddy},
doi = {10.1093/aob/mcp027},
isbn = {0305-7364},
issn = {10958290},
journal = {Annals of Botany},
keywords = {Plantpollinator community,body size,flower morphology,generalization,nectar,pollination network,size matching,specialization},
number = {9},
pages = {1459--1469},
pmid = {19228701},
title = {{Size-specific interaction patterns and size matching in a plant-pollinator interaction web}},
volume = {103},
year = {2009}
}
@article{Junker2013,
abstract = {* Biotic interactions do not occur in isolation but are imbedded in a network of species interactions. Network analysis facilitates the compilation and understanding of the complexity found in natural ecosystems and is a powerful tool to reveal information on the degree of specialization of the interacting partners and their niches. The indices measuring these properties are based on qualitative or quantitative observations of interactions between partners from different trophic levels, which informs about the structure of network patterns, but not about the underlying mechanisms. Functional traits may control the interaction strength between partners and also the (micro-) structure of networks. Here, we ask whether flower visitors specialize on certain plant traits and how this trait specialization contributes to niche partitioning and interaction partner diversity. * We introduce two novel statistical approaches suited to evaluate the dimension of the realized niche and to analyse which traits determine niches. As basis for our analysis, we measured 10 quantitative flower traits and evaluated whether 31 arthropod taxa i visited flowers displaying only subsets of the available trait characteristics, indicating a specialization on these traits by narrow trait-widths 〈Si〉. The product of 10 trait- and species-specific trait-widths 〈Si〉 was defined as trait-volume Vi (expansion of a n-dimensional hypervolume) occupied by each taxon i. These indices are applicable beyond flower–visitor interactions to quantify realized niches based on various biotic and abiotic factors. * Each flower visitor species showed some degree of specialization to a unique set of flower traits (realized niche). Overall, our data suggested a hierarchical sequence of flower traits influencing the flower visitors' behaviour and thus network structure: flowering phenology was found to have the strongest effect, followed by flower height, nectar-tube depth and floral reflectance. Less important were pollen-mass/flower, sugar/flower, anther position, phylogeny, display size and abundance. * The species-specific specialization on traits suggests that plant communities with more diverse floral niches may sustain a larger number of flower visitors with non-redundant fundamental niches. Our study and statistical approach provide a basis for a better understanding of how plant traits shape interactions between flowers and their visitors and thus network structure.},
author = {Junker, Robert R. and Bl??thgen, Nico and Brehm, Tanja and Binkenstein, Julia and Paulus, Justina and {Martin Schaefer}, H. and Stang, Martina},
doi = {10.1111/1365-2435.12005},
isbn = {1365-2435},
issn = {02698463},
journal = {Functional Ecology},
keywords = {Floral filters,Floral resources,Flower colour,Hierarchy of traits,Morphology,Niche,Pollination,n-dimensional hypervolume},
number = {2},
pages = {329--341},
title = {{Specialization on traits as basis for the niche-breadth of flower visitors and as structuring mechanism of ecological networks}},
volume = {27},
year = {2013}
}
@article{Coux2016,
abstract = {Species roles in ecological networks combine to generate their architecture, which contributes to their stability. Species trait diversity also affects ecosystem functioning and resilience, yet it remains unknown whether species' contributions to functional diversity relate to their network roles. Here, we use 21 empirical pollen transport networks to characterise this relationship. We found that, apart from a few abundant species, pollinators with original traits either had few interaction partners or interacted most frequently with a subset of these partners. This suggests that narrowing of interactions to a subset of the plant community accompanies pollinator niche specialisation, congruent with our hypothesised trade-off between having unique traits vs. being able to interact with many mutualist partners. Conversely, these effects were not detected in plants, potentially because key aspects of their flowering traits are conserved at a family level. Relating functional and network roles can provide further insight into mechanisms underlying ecosystem functioning.},
author = {Coux, Camille and Rader, Romina and Bartomeus, Ignasi and Tylianakis, Jason M.},
doi = {10.1111/ele.12612},
isbn = {1461-0248},
issn = {14610248},
journal = {Ecology Letters},
keywords = {Biodiversity,Ecosystem functioning,Interaction,Mutualistic network,Resilience,Stability,Web},
pages = {762--770},
pmid = {27169359},
title = {{Linking species functional roles to their network roles}},
url = {http://doi.wiley.com/10.1111/ele.12612},
volume = {19},
year = {2016}
}
@article{Maglianesi2014,
abstract = {Ecological communities are organized in complex ecological networks. Trait-based analyses of the structure of these networks in highly diversified species assemblages are crucial for improving our understanding of the ecological and evolutionary processes causing specialization in mutualistic networks. In this study, we assessed the importance of morphological traits for structuring plant–hummingbird networks in Neotropical forests by using a novel combination of quantitative analytical approaches. We recorded the visitation of hummingbirds to plant species over an entire year at three different elevations in Costa Rica and constructed quantitative networks based on interaction frequencies. Three morphological traits were measured in hummingbirds (bill length, bill curvature, and body mass) and plants (corolla length, curvature, and volume). We tested the effects of avian morphological traits and abundance on ecological specialization of hummingbird species. All three morphological traits of hummingbirds were positively associated with ecological specialization, especially bill curvature. We tested whether interaction strength in the networks was associated with the degree of trait matching between corresponding pairs of morphological traits in plant and hummingbird species and explore whether this was related to resource handling times by hummingbird species. We found strong and significant associations between interaction strength and the degree of trait matching. Moreover, the degree of trait matching, particularly between bill and corolla length, was associated with the handling time of nectar resources by hummingbirds. Our findings show that bill morphology structures tropical plant–hummingbird networks and patterns of interactions are closely associated with morphological matches between plant and bird species and the efficiency of hummingbirds' resource use. These results are consistent with the findings of seminal studies in plant–hummingbird systems from the neotropics. We conclude that trait-based analyses of quantitative networks contribute to a better mechanistic understanding of the causes of specialization in ecological networks and could be valuable for studying processes of complementary trait evolution in highly diversified species assemblages. Read More: http://www.esajournals.org/doi/abs/10.1890/13-2261.1},
author = {Maglianesi, Mar{\'{i}}a Alejandra and Bl{\"{u}}Thgen, Nico and B{\"{o}}Hning-Gaese, Katrin and Schleuning, Matthias},
doi = {10.1890/13-2261.1},
isbn = {0012-9658},
issn = {00129658},
journal = {Ecology},
keywords = {Biotic interactions,Costa Rica,Fourth-corner analysis,Hummingbirds,Mutualistic networks,Neotropical forest,Optimal foraging,Pollination,Specialization,Trait complementarity,Trochilidae},
number = {12},
pages = {3325--3334},
title = {{Morphological traits determine specialization and resource use in plant-hummingbird networks in the neotropics}},
volume = {95},
year = {2014}
}
@article{Kefi2016,
abstract = {Species are linked to each other by a myriad of positive and negative interactions. This com- plex spectrum of interactions constitutes a network of links that mediates ecological com- munities' response to perturbations, such as exploitation and climate change. In the last decades, there have been great advances in the study of intricate ecological networks. We have, nonetheless, lacked both the data and the tools to more rigorously understand the patterning of multiple interaction types between species (i.e., “multiplex networks”), as well as their consequences for community dynamics. Using network statisticalmodeling applied to a comprehensive ecological network, which includes trophic and diverse non-trophic links, we provide a first glimpse at what the full “entangled bank” of species looks like. The community exhibits clear multidimensional structure, which is taxonomically coherent and broadly predictable from species traits. Moreover, dynamic simulations suggest that this non-randompatterning of how diverse non-trophic interactionsmap onto the food web could allow for higher species persistence and higher total biomass than expected by chance and tends to promote a higher robustness to extinctions. Author},
author = {K{\'{e}}fi, Sonia and Miele, Vincent and Wieters, Evie A. and Navarrete, Sergio A. and Berlow, Eric L.},
doi = {10.1371/journal.pbio.1002527},
isbn = {1545-7885},
issn = {15457885},
journal = {PLoS Biology},
number = {8},
pages = {e1002527},
pmid = {27487303},
title = {{How Structured Is the Entangled Bank? The Surprisingly Simple Organization of Multiplex Ecological Networks Leads to Increased Persistence and Resilience}},
url = {http://dx.plos.org/10.1371/journal.pbio.1002527},
volume = {14},
year = {2016}
}
@book{Barlow2000,
abstract = {Ecological science is changing because of a recent discovery: Every field,$\backslash$n forest, and park is full of living organisms adapted for relationships$\backslash$n with creatures that have long been extinct. In this book, the author shows$\backslash$n how this idea of" missing partners" in nature evolved from ...},
address = {New York},
author = {Barlow, C},
booktitle = {Basic Books},
doi = {10.1007/BF02344736},
isbn = {0-465-00552-7},
issn = {01400118},
pages = {304},
publisher = {Basic Books},
title = {{The Ghosts of Evolution}},
year = {2000}
}
@article{Janzen1982,
author = {Janzen-martin},
journal = {Science},
number = {4528},
pages = {19--27},
title = {{Neotropical Anachronisms : The Fruits the Gomphotheres Ate Neotropical Anachronisms : The Fruits the Gomphotheres Ate}},
volume = {215},
year = {1982}
}
@article{Aizen2012a,
abstract = {Functional elucidation of causal genetic variants and elements requires precise genome editing technologies. The type II prokaryotic CRISPR (clustered regularly interspaced short palindromic repeats)/Cas adaptive immune system has been shown to facilitate RNA-guided site-specific DNA cleavage. We engineered two different type II CRISPR/Cas systems and demonstrate that Cas9 nucleases can be directed by short RNAs to induce precise cleavage at endogenous genomic loci in human and mouse cells. Cas9 can also be converted into a nicking enzyme to facilitate homology-directed repair with minimal mutagenic activity. Lastly, multiple guide sequences can be encoded into a single CRISPR array to enable simultaneous editing of several sites within the mammalian genome, demonstrating easy programmability and wide applicability of the RNA-guided nuclease technology.},
archivePrefix = {arXiv},
arxivId = {20},
author = {Kim, Sangjin and Brostr{\"{o}}mer, Erik and Xing, Dong and Jin, Jianshi and Chong, Shasha and Ge, Hao and Wang, Siyuan and Gu, Chan and Yang, Lijiang and Gao, Yi Qin and Su, Xiao Dong and Sun, Yujie and Xie, X. Sunney},
doi = {10.1126/science.1229223},
eprint = {20},
isbn = {0036-8075},
issn = {10959203},
journal = {Science},
number = {6121},
pages = {816--819},
pmid = {23287718},
title = {{Probing allostery through DNA}},
volume = {339},
year = {2013}
}
@article{Dehling2014,
abstract = {Aim$\backslash$n$\backslash$nFunctional relationships between species groups on macroecological scales have often been inferred from comparisons of species numbers across space. On large spatial scales, however, it is difficult to assess whether correlations of species numbers represent actual functional relationships. Here, we investigated the functional relationship between a feeding guild (fruit-eating birds) and its resource (fleshy-fruited plants) by studying the matching of their functional traits across spatial scales, from individual interactions to regional patterns.$\backslash$n$\backslash$n$\backslash$nLocation$\backslash$n$\backslash$nA 3000-m elevational gradient in the tropical Andes.$\backslash$n$\backslash$n$\backslash$nMethods$\backslash$n$\backslash$nWe sampled plant–bird interactions at two sites along the elevational gradient, and using multivariate statistics (fourth-corner analysis) we identified corresponding morphological traits of birds and plants that influenced which bird species fed from which plant species. We then tested whether the functional trait diversities of the bird species assemblages matched those of the plant species assemblages along the elevational gradient.$\backslash$n$\backslash$n$\backslash$nResults$\backslash$n$\backslash$nCorresponding functional traits of birds and plants were closely and significantly correlated on the scale of individual plant–bird interactions. On the regional scale, the functional diversities, but not species numbers, of bird and plant assemblages correlated significantly along the elevational gradient.$\backslash$n$\backslash$n$\backslash$nMain conclusions$\backslash$n$\backslash$nThe analysis of species interaction networks with multivariate statistics was a powerful tool for identifying relationships between functional traits of interacting species. The close functional relationships between birds and plants on the scale of individual interactions and on the regional scale show that comparisons of functional trait diversities, based on matching traits of interacting species, are better suited than correlations of species numbers to reveal the mechanisms behind large-scale diversity patterns of interacting species. The identification of functional interdependences between interacting species on large spatial scales will be important for improving predictive models of species distributions in space and time.},
author = {Dehling, D. Matthias and T{\"{o}}pfer, Till and Schaefer, H. Martin and Jordano, Pedro and B{\"{o}}hning-Gaese, Katrin and Schleuning, Matthias},
doi = {10.1111/geb.12193},
isbn = {1466-8238},
issn = {14668238},
journal = {Global Ecology and Biogeography},
keywords = {Andes,Elevational gradient,Fleshy-fruited plants,Fourth-corner analysis,Frugivorous birds,Functional diversity,Interaction networks,Man{\'{u}} National Park,Mutualism,Seed dispersal},
number = {10},
pages = {1085--1093},
title = {{Functional relationships beyond species richness patterns: Trait matching in plant-bird mutualisms across scales}},
volume = {23},
year = {2014}
}
@article{Grier1982,
abstract = {Reproduction of bald eagles in northwestern Ontario declined from 1.26 young per breeding area in 1966 to a low of 0.46 in 1974 and then increased to 1.12 in 1981. Residues of DDE in addled eggs showed a significant inverse relation, confirming the effects of this toxicant on bald eagle reproduction at the population level and the effectiveness of the ban on DDT. The recovery from DDE contamination in bald eagles appears to be occurring much ore rapidly than predicted.},
author = {Grier, J.},
doi = {10.1126/science.7146905},
isbn = {0036-8075},
issn = {0036-8075},
journal = {Science},
number = {4578},
pages = {1232--1235},
pmid = {7146905},
title = {{Ban of DDT and subsequent recovery of Reproduction in bald eagles}},
url = {http://www.sciencemag.org/cgi/doi/10.1126/science.7146905},
volume = {218},
year = {1982}
}
@article{BeltranPedreros2011,
abstract = {Thirty-two species of commercially important fish from three trophic levels and nine trophic categories were sampled at a floodplain lake of the Solim{\~{o}}es River (Lago Grande de Manacapuru). The fish were analyzed to determine their Hg level and the bioaccumulation, bioconcentration, and biomagnification of this element. The observed increase in mean concentration of mercury (49.6 ng.g-1 for omnivores, 418.3 ng.g-1 for piscivores, and 527.8 ng.g-1 for carnivores/necrophages) furnished evidence of biomagnification. Primary, secondary, and tertiary consumers presented biomagnification factors of 0.27, 0.33, and 0.47, respectively. Significant differences in the bioconcentration and concentration of total Hg occurred between the categories of the third trophic level and the other categories. Plagioscion squamosissimus (carnivorous/piscivorous) and Calophysus macropterus (carnivorous/ necrophagous) showed levels of total Hg above those permitted by Brazilian law (500 ng.g-1 ). Six other species also posed risks to human health because their Hg levels exceeded 300 ng.g-1 . Fifteen species showed bioaccumulation, but only eight presented significant correlations between the concentration of Hg and the length and/or the weight of the fish.},
author = {Beltran-Pedreros, Sandra and Zuanon, Jansen and Leite, Rosseval Galdino and Peleja, Jose Reinaldo Pacheco and Mendon{\c{c}}a, Alessandra Barros and Forsberg, Bruce Rider},
doi = {10.1590/S1679-62252011000400022},
isbn = {1679-6225},
issn = {16796225},
journal = {Neotropical Ichthyology},
keywords = {Biomagnification,Ecotoxicology,Fish quality,Floodplain,Mercury contamination},
number = {4},
pages = {901--908},
title = {{Mercury bioaccumulation in fish of commercial importance from different trophic categories in an Amazon floodplain lake}},
volume = {9},
year = {2011}
}
@article{Bossart2011,
abstract = {The long-term consequences of climate change and potential environmental degradation are likely to include aspects of disease emergence in marine plants and animals. In turn, these emerging diseases may have epizootic potential, zoonotic implications, and a complex pathogenesis involving other cofactors such as anthropogenic contaminant burden, genetics, and immunologic dysfunction. The concept of marine sentinel organisms provides one approach to evaluating aquatic ecosystem health. Such sentinels are barometers for current or potential negative impacts on individual- and population-level animal health. In turn, using marine sentinels permits better characterization and management of impacts that ultimately affect animal and human health associated with the oceans. Marine mammals are prime sentinel species because many species have long life spans, are long-term coastal residents, feed at a high trophic level, and have unique fat stores that can serve as depots for anthropogenic toxins. Marine mammals may be exposed to environmental stressors such as chemical pollutants, harmful algal biotoxins, and emerging or resurging pathogens. Since many marine mammal species share the coastal environment with humans and consume the same food, they also may serve as effective sentinels for public health problems. Finally, marine mammals are charismatic megafauna that typically stimulate an exaggerated human behavioral response and are thus more likely to be observed.},
author = {Bossart, G},
doi = {10.5670/oceanog.2006.77},
isbn = {1544-2217 (Electronic)$\backslash$r0300-9858 (Linking)},
issn = {10428275},
journal = {Oceanography},
keywords = {as the effects of,better characterized,climate change and potential,degradation are debated and,ecosystem health,environmental,human health,marine mammals,sentinel species,worldwide},
number = {2},
pages = {134--137},
pmid = {21160025},
title = {{Marine Mammals as Sentinal Species for Oceans and Human Health}},
url = {papers2://publication/uuid/D14C1B19-6350-4893-9980-3B726A4F85CA},
volume = {19},
year = {2006}
}
@article{Gezon2016,
abstract = {Climate change has had numerous ecological effects, including species range shifts and altered phenology. Altering flowering phenology often affects plant reproduction, but the mechanisms behind these changes are not well-understood. To investigate why altering flowering phenology affects plant reproduction, we manipulated flowering phenology of the spring herb Claytonia lanceolata (Portulacaceae) using two methods: in 2011-2013 by altering snow pack (snow-removal vs. control treatments), and in 2013 by inducing flowering in a greenhouse before placing plants in experimental outdoor arrays (early, control, and late treatments). We measured flowering phenology, pollinator visitation, plant reproduction (fruit and seed set), and pollen limitation. Flowering occurred approx. 10 days earlier in snow-removal than control plots during all years of snow manipulation. Pollinator visitation patterns and strength of pollen limitation varied with snow treatments, and among years. Plants in the snow removal treatment were more likely to experience frost damage, and frost-damaged plants suffered low reproduction despite lack of pollen limitation. Plants in the snow removal treatment that escaped frost damage had higher pollinator visitation rates and reproduction than controls. The results of the array experiment supported the results of the snow manipulations. Plants in the early and late treatments suffered very low reproduction due either to severe frost damage (early treatment) or low pollinator visitation (late treatment) relative to control plants. Thus, plants face tradeoffs with advanced flowering time. While early-flowering plants can reap the benefits of enhanced pollination services, they do so at the cost of increased susceptibility to frost damage that can overwhelm any benefit of flowering early. In contrast, delayed flowering results in dramatic reductions in plant reproduction through reduced pollination. Our results suggest that climate change may constrain the success of early-flowering plants not through plant-pollinator mismatch but through the direct impacts of extreme environmental conditions.},
author = {Gezon, Zachariah J. and Inouye, David W. and Irwin, Rebecca E.},
doi = {10.1111/gcb.13209},
isbn = {1365-2486},
issn = {13652486},
journal = {Global Change Biology},
keywords = {Claytonia lanceolata,Climate change,Phenological mismatch,Phenology,Plant reproduction,Pollen limitation,Pollination},
number = {5},
pages = {1779--1793},
pmid = {26833694},
title = {{Phenological change in a spring ephemeral: Implications for pollination and plant reproduction}},
volume = {22},
year = {2016}
}
@article{Hua2016,
abstract = {Global climate change is known to affect the assembly of ecological communities by altering species' spatial distribution patterns, but little is known about how climate change may affect community assembly by changing species' temporal co-occurrence patterns, which is highly likely given the widely observed phenological shifts associated with climate change. Here, we analyzed a 29-year phenological data set comprising community-level information on the timing and span of temporal occurrence in 11 seasonally occurring animal taxon groups from 329 local meteorological observatories across China. We show that widespread shifts in phenology have resulted in community-wide changes in the temporal overlap between taxa that are dominated by extensions, and that these changes are largely due to taxa's altered span of temporal occurrence rather than the degree of synchrony in phenological shifts. Importantly, our findings also suggest that climate change may have led to less phenological mismatch than generally presumed, and that the context under which to discuss the ecological consequences of phenological shifts should be expanded beyond asynchronous shifts.},
author = {Hua, Fangyuan and Hu, Junhua and Liu, Yang and Giam, Xingli and Lee, Tien Ming and Luo, Hao and Wu, Jia and Liang, Qiaoyi and Zhao, Jian and Long, Xiaoyan and Pang, Hong and Wang, Biao and Liang, Wei and Zhang, Zhengwang and Gao, Xuejie and Zhu, Jiang},
doi = {10.1111/gcb.13199},
isbn = {1609258029},
issn = {13652486},
journal = {Global Change Biology},
keywords = {China,Climate change,Community assembly,Interspecific temporal overlap,Phenological shift,Temporal occurrence window},
number = {5},
pages = {1746--1754},
pmid = {26680152},
title = {{Community-wide changes in intertaxonomic temporal co-occurrence resulting from phenological shifts}},
url = {https://onlinelibrary.wiley.com/doi/abs/10.1111/gcb.13199},
volume = {22},
year = {2016}
}
@article{Menzel2006,
author = {Alm-kubler, K and Bissollik, P and Braslavska, O and Briede, A and Chmielewski, f m and Crepinsek, Z and Curnel, Y and Dahl, A and Defila, C and Donnelly, A and Filella, Y and Jatczak, K and Mage, F and Mestre, A and Nordli, {\O} and Penuelas, J and Pirinen, P and Remisova, V and Scheifinger, H and Striz, M and Susnik, A and Van vliet, a j h and Wielgolaski, F-e and Zach, S and Zust, A},
doi = {10.1111/j.1365-2486.2006.01193.x},
issn = {1354-1013},
journal = {Global Change Biology},
keywords = {climate change,europe,growing season,meta analysis,phenology,season,temperature},
number = {10},
pages = {1969--1976},
title = {{European phenological response to climate change matches the warming pattern}},
url = {http://doi.wiley.com/10.1111/j.1365-2486.2006.01193.x},
volume = {12},
year = {2006}
}
@incollection{MeyerOrtmanns2015,
abstract = {This paper addresses a set of issues involved in modeling systems across many orders of magnitude in spatial and temporal scales. In particular, it focuses on the question of how one can explain and understand the relative autonomy and safety of models at contin- uum scales. The typical battle line between reductive “bottom-up” modeling and “top-down” modeling from phenomenological theories is shown to be overly simplistic. Multi-scale models are beginning to succeed in showing how to upscale from statistical atomistic/molecular models to continuum/hydrodynamics models. The consequences for our understanding of the debate between reductionism and emergence will be examined.},
address = {Heidelberg},
author = {Meyer-Ortmanns, Hildegard},
booktitle = {Why More Is Different},
chapter = {2},
doi = {10.1007/978-3-662-43911-1},
editor = {Falkenburg, Brigitte and Morrison, Margaret},
isbn = {978-3-662-43910-4},
issn = {1612-3018},
pages = {201--226},
publisher = {Springer-Verlag},
title = {{Why More Is Different}},
url = {http://link.springer.com/10.1007/978-3-662-43911-1},
year = {2015}
}
@article{Burkle2013,
abstract = {Using historic data sets, we quantified the degree to which global change over 120 years disrupted plant-pollinator interactions in a temperate forest understory community in Illinois, USA. We found degradation of interaction network structure and function and extirpation of 50{\%} of bee species. Network changes can be attributed to shifts in forb and bee phenologies resulting in temporal mismatches, nonrandom species extinctions, and loss of spatial co-occurrences between extant species in modified landscapes. Quantity and quality of pollination services have declined through time. The historic network showed flexibility in response to disturbance; however, our data suggest that networks will be less resilient to future changes.},
archivePrefix = {arXiv},
arxivId = {http://links.jstor.org/sici?sici=0036-8075{\%}2819780324{\%}293{\%}3A199{\%}3A4335{\%}3C1302{\%}3ADITRFA{\%}3E2.0.CO{\%}3B2-2},
author = {Burkle, Laura A. and Marlin, John C. and Knight, Tiffany M.},
doi = {10.1126/science.1232728},
eprint = {/links.jstor.org/sici?sici=0036-8075{\%}2819780324{\%}293{\%}3A199{\%}3A4335{\%}3C1302{\%}3ADITRFA{\%}3E2.0.CO{\%}3B2-2},
file = {:Users/alyssacirtwill/Documents/Papers/Burkle, Marlin, Knight{\_}2013{\_}Science.pdf:pdf},
isbn = {0036-8075},
issn = {10959203},
journal = {Science},
keywords = {Animals,Bees,Bees: physiology,Biological,Bombus occidentalis,Climate,Climate change,Colorado,Community,Community assembly,Community composition,Community dynamics,Connectance,Delphinium nuttallianum,Demography,Ecosystem service,Ecosystem services,Experiment,Extinction,Flowers,Flowers: growth {\&} development,Generalization,Global change,Habitat fragmentation,Illinois,Indirect effects,Insect,Ipomopsis aggregata,Life history,Linaria vulgaris,Long-term data,Mesocosm experiments,Meta-analysis,Missouri,Models,Mutualism,Mycorrhizal fungi,Nectar robbing,Network,Nutrient addition,Nutrient limitation,Observation,Ozark glades,Phenology,Pioneer species,Plant,Poaceae,Poaceae: growth {\&} development,Pollen limitation,Pollination,Portulacaceae,Portulacaceae: growth {\&} development,Potentilla pulcherrima,Productivity-diversity relationships,QK900 Plant ecology,QK926 Pollination,Restoration,Self-compatibility,Simulation,Specialization,Specialization-generalization continuum,Species interactions,Trees,Trees: growth {\&} development,USA,Vulnerability,accepted 15 july 2015,accepted 17 september 2014,and evolutionary biologists for,arrhenius,behavioral plasticity,beta diversity,beta-diversity,biodiversity,biogeographic gradient,c,community assembly,conservation,corresponding editor,d,disturbance severity,ecologists,ecology,ecosystem function,elevation,environmental gradient,fire management,have inspired natural historians,homogenization,interaction turnover,landscape,mixed-severity wildfire,net primary productivity,northern rockies ecoregion,over a century,p,peters,plant,plant community composition,pollination services,pollinator network,published 26 november 2014,published 28 october 2015,received 12 july 2015,received 15 september 2014,restoration ecology,spatial patterns of biodiversity,spatial scale,spatiotemporal,species pool},
month = {feb},
number = {6127},
pages = {1611--1615},
pmid = {23449999},
primaryClass = {http:},
title = {{Plant-pollinator interactions over 120 years: Loss of species, co-occurrence, and function}},
url = {http://www.sciencemag.org/cgi/doi/10.1126/science.1232728 http://www.amjbot.org/content/early/2015/11/20/ajb.1500079.abstract http://www.pollinationecology.org/index.php?journal=jpe http://dx.doi.org/10.1016/B978-0-12-384703-4.00416-0},
volume = {340},
year = {2013}
}
@incollection{Poulin2010,
abstract = {Many taxa of parasites modify the behavior of their hosts in ways that improve their probability of transmission. Regardless of its evolutionary origins or underlying mechanisms, host manipulation is a widespread adaptive strategy yielding fitness benefits for parasites with various life cycles and transmission modes. This chapter focuses on recent developments that are expanding our understanding of this phenomenon. Currently, growing attention is being paid to the effect of parasites on whole suites of host behavioral traits as opposed to single traits, and to correlations among behaviors, which may be the target of manipulation instead of the traits themselves. At the same time, variation in the use of manipulation is being explored both among and within parasite species. On the one hand, models that take into account the potential costs of manipulation predict under what circumstances manipulation is likely to evolve as a transmission strategy. On the other hand, within manipulative species, manipulation may be a flexible strategy only adopted by individual parasites in certain conditions dictated by other parasites and by the host itself. This inter- and intraspecific variation in the use of host manipulation for transmission is due in large part to its unreliable effectiveness within complex natural systems where dead-ends await many manipulative parasites. Finally, recent neurological and proteomic studies of the pathways used by parasites to alter host behavior offer new insights into the evolution of manipulation. The chapter ends with a list of promising directions that provide an agenda for future research. {\textcopyright} 2010 Elsevier Inc.},
address = {Burlington},
author = {Poulin, Robert},
booktitle = {Advances in the Study of Behavior},
doi = {10.1016/S0065-3454(10)41005-0},
edition = {1},
editor = {Brockmann, H. Jane},
isbn = {9780123808929},
issn = {00653454},
keywords = {Adaptiveness,Behavioral syndrome,Costs,Life cycles,Neurobiology,Phenotypic alteration,Proteomics,Transmission},
number = {C},
pages = {151--186},
pmid = {11863559},
publisher = {Elsevier Inc.},
title = {{Parasite Manipulation of Host Behavior: An Update and Frequently Asked Questions}},
url = {http://www.sciencedirect.com/science/article/pii/S0065345410410050},
volume = {41},
year = {2010}
}
@article{Whittaker1972,
abstract = {As little is known of association(s) between gut microbiota profiles and host immunological markers, we explored these in children with and without multiple sclerosis (MS). Children ≤18 years provided stool and blood. MS cases were within 2-years of onset. Fecal 16S rRNA gene profiles were generated on an Illumina Miseq platform. Peripheral blood mononuclear cells were isolated, and Treg (CD4+CD25hiCD127lowFoxP3+) frequency and CD4+ T-cell intracellular cytokine production evaluated by flow cytometry. Associations between microbiota diversity, phylum-level abundances and immune markers were explored using Pearson's correlation and adjusted linear regression. Twenty-four children (15 relapsing-remitting, nine controls), averaging 12.6 years were included. Seven were on a disease-modifying drug (DMD) at sample collection. Although immune markers (e.g. Th2, Th17, Tregs) did not differ between cases and controls (p {\textgreater} 0.05), divergent gut microbiota associations occurred; richness correlated positively with Th17 for cases (r = +0.665, p = 0.018), not controls (r = −0.644, p = 0.061). Bacteroidetes inversely associated with Th17 for cases (r = −0.719, p = 0.008), not controls (r = +0.320, p = 0.401). Fusobacteria correlated with Tregs for controls (r = +0.829, p = 0.006), not cases (r = −0.069, p = 0.808). Our observations motivate further exploration to understand disruption of the microbiota-immune balance so early in the MS course.},
author = {Obiako, M. N. and Ebigbo, P. O.},
doi = {10.2307/1218190},
isbn = {00400262},
issn = {01455613},
journal = {Ear, Nose and Throat Journal},
number = {7},
pages = {379--381},
pmid = {73},
title = {{Psychodynamic observations on Globus hystericus among Nigerians}},
url = {http://www.jstor.org/stable/1218190},
volume = {61},
year = {1982}
}
@article{Platt2013,
abstract = {Abstract Saurochory (seed dispersal by reptiles) among crocodilians has largely been ignored, probably because these reptiles are generally assumed to be obligate carnivores incapable of digesting vegetable proteins and polysaccharides. Herein we review the literature on crocodilian diet, foraging ecology, digestive physiology and movement patterns, and provide additional empirical data from recent dietary studies of Alligator mississippiensis. We found evidence of frugivory in 13 of 18 (72.2{\%}) species for which dietary information was available, indicating this behav- ior is widespread among the Crocodylia. Thirty-four families and 46 genera of plants were consumed by crocodilians. Fruit types consumed by crocodilians varied widely; over half (52.1{\%}) were fleshy fruits. Some fruits are consumed as gastroliths or ingested incidental to prey capture; however, there is little doubt that on occasion, fruit is deliberately consumed, often in large quantities. Sensory cues involved in crocodilian frugivory are poorly understood, although airborne and waterborne cues as well as surface disturbances seem important. Crocodilians likely accrue nutritional benefits from frugivory and there are no a priori reasons to assume otherwise. Ingested seeds are regurgitated, retained in the stomach for indefinite and often lengthy periods, or passed through the digestive tract and excreted in feces. Chemical and mechanical scarification of seeds probably occurs in the stomach, but what effects these processes have on seed viability remain unknown. Because crocodilians have large territories and undertake lengthy movements, seeds are likely transported well beyond the parent plant before being voided. Little is known about the ultimate fate of seeds ingested by crocodilians; however, however, deposition sites could prove suitable for seed germination. Although there is no evidence for a crocodilian-specific dispersal syndrome similar to that described for other reptiles, our review strongly suggests that crocodilians func- tion as effective agents of seed dispersal. Crocodilian saurochory offers a fertile ground for future research.},
author = {Platt, S. G. and Elsey, R. M. and Liu, H. and Rainwater, T. R. and Nifong, J. C. and Rosenblatt, A. E. and Heithaus, M. R. and Mazzotti, F. J.},
doi = {10.1111/jzo.12052},
isbn = {1469-7998},
issn = {09528369},
journal = {Journal of Zoology},
keywords = {Alligator mississippiensis,Crocodylia,Diet,Foraging ecology,Frugivory,Saurochory,Seed dispersal},
number = {2},
pages = {87--99},
title = {{Frugivory and seed dispersal by crocodilians: An overlooked form of saurochory?}},
volume = {291},
year = {2013}
}
@article{Pietz2000,
abstract = {The seaweed Sargassum polyceratium Montagne inhabits a broad spectrum of subtidal and intertidal habitats. Genetic diversity and spatial genetic structure were examined within and among 12 stands using random amplified polymorphic DNA (RAPD) phenotypes. Data were analyzed using analysis of molecular variance (AMOVA) and Shannon's information measure. In both analyses, 60-75{\%} of the variation occurred within stands and 25-40{\%} between stands. These values are consistent with out-crossing, high-dispersal species. Significant differentiation was found among bays ca. 25 kin apart (Shannon's G'(st) averaged 0.37 and pairwise AMOVA Phi (st) values averaged 0.272) and among stands 150-200 m apart within bays (AMOVA Phi (st) values averaged 0.149). Effects of shore (windward vs. leeward), depth, and bay on population structure were tested. These analyses revealed that the factor depth is confounded with shore, and that bays show significant differentiation from each other but are not completely isolated from one another. Mantel tests for differentiation-by-distance were significant along both sides of the island but stronger along the windward side. A neighbor-joining analysis of genetic distances among stands showed that the effects of currents around both tips of the island were especially important for shallow populations. For S. polyceratium, depth and bay promote population differentiation along shores, yet dispersal around the tips of the island simultaneously connects these populations to varying degrees. This study highlights the importance of investigating the relative contribution of habitat factorsin relation to island-scale population structure.},
archivePrefix = {arXiv},
arxivId = {2166},
author = {Engelen, A. H. and Olsen, J. L. and Breeman, A. M. and Stam, W. T.},
doi = {10.1109/CIT/IUCC/DASC/PICOM.2015.6},
eprint = {2166},
isbn = {9781509001545},
issn = {00253162},
journal = {Marine Biology},
keywords = {white-tailed deer},
number = {2},
pages = {267--277},
pmid = {2772},
title = {{Genetic differentiation in Sargassum polyceratium (Fucales: Phaeophyceae) around the island of Cura{\c{c}}ao (Netherlands Antilles)}},
url = {http://www.bioone.org/doi/abs/10.1674/0003-0031(2000)144[0419:WTDOVP]2.0.CO;2},
volume = {139},
year = {2001}
}
@article{Dafni2000,
abstract = {The present article reviews the various definitions and terminology of pollen viability and longevity as well as the various tests of its assessment. We compare the advantages and the disadvantages of each method and suggest some practical implications as revealed by the extensive data. We recognize eight main hypotheses concerning the ecology and the evolution of pollen longevity and critically evaluated them according to the literature. The hypotheses are grouped as follows: (1) Desiccation risk-carbohydrate content: (2) Pollen packaging; (3) Pollen competitive ability; (4) Pollinator activity-stigma receptivity duration: (5) Self-pollination chance; (6) Pollen exposure schedule; (7) Pollen travel distance, and (8) Pollen removal chance.},
author = {Dafni, Amots and Firmage, David},
doi = {10.1007/BF00984098},
isbn = {0378-2697},
issn = {03782697},
journal = {Plant Systematics and Evolution},
keywords = {Desiccation,Pollen longevity,Stigma receptivity,Viability},
number = {1-4},
pages = {113--132},
title = {{Pollen viability and longevity: Practical, ecological and evolutionary implications}},
volume = {222},
year = {2000}
}
@article{Guimera2005a,
abstract = {High-throughput techniques are leading to an explosive growth in the size of biological databases and creating the opportunity to revolutionize our understanding of life and disease. Interpretation of these data remains, however, a major scientific challenge. Here, we propose a methodology that enables us to extract and display information contained in complex networks. Specifically, we demonstrate that one can (i) find functional modules in complex networks, and (ii) classify nodes into universal roles according to their pattern of intra- and inter-module connections. The method thus yields a ``cartographic representation'' of complex networks. Metabolic networks are among the most challenging biological networks and, arguably, the ones with more potential for immediate applicability. We use our method to analyze the metabolic networks of twelve organisms from three different super-kingdoms. We find that, typically, 80{\%} of the nodes are only connected to other nodes within their respective modules, and that nodes with different roles are affected by different evolutionary constraints and pressures. Remarkably, we find that low-degree metabolites that connect different modules are more conserved than hubs whose links are mostly within a single module.},
archivePrefix = {arXiv},
arxivId = {q-bio/0502035},
author = {Guimer{\`{a}}, Roger and Amaral, Luis A Nunes},
doi = {10.1038/nature03288},
eprint = {0502035},
isbn = {1476-4687 (Electronic)$\backslash$n0028-0836 (Linking)},
issn = {00280836},
journal = {Nature},
number = {7028},
pages = {895--900},
pmid = {15729348},
primaryClass = {q-bio},
title = {{Functional cartography of complex metabolic networks}},
url = {http://www.nature.com/doifinder/10.1038/nature03288},
volume = {433},
year = {2005}
}
@book{Hassol2004,
abstract = {The Arctic is now experiencing some of the most rapid and severe climate change on earth. Over the next 100 years, climate change is expected to accelerate, contributing to major physical, ecological, social, and economic changes, many of which have already begun. Changes in arctic climate will also affect the rest of the world through increased global warming and rising sea levels. Impacts of a Warming Arctic is a plain language synthesis of the key findings of the Arctic Climate Impact Assessment (ACIA), designed to be accessible to policymakers and the broader public. The ACIA is a comprehensively researched, fully referenced, and independently reviewed evaluation of arctic climate change. It has involved an international effort by hundreds of scientists. This report provides vital information to society as it contemplates its responses to one of the greatest challenges of our time. It is illustrated in full color throughout.},
address = {Cambridge},
archivePrefix = {arXiv},
arxivId = {arXiv:1011.1669v3},
author = {ACIA},
booktitle = {Arctic Climate Impact Assessment (ACIA)},
doi = {10.2277/0521617782},
eprint = {arXiv:1011.1669v3},
isbn = {0521617782},
issn = {1098-6596},
pages = {140},
pmid = {25246403},
publisher = {Cambridge University Press},
title = {{Impacts of a Warming Arctic: Arctic Climate Impact Assessment}},
url = {http://amap.no/acia/},
year = {2004}
}
@article{Steiner2015,
abstract = {Past trends and future projections of key atmospheric, oceanic, sea ice, and biogeochemical variables were assessed to increase our understanding of climate change impacts on Canadian Arctic marine ecosystems. Four subbasins are evaluated: Beaufort Sea, Canadian Arctic Archipelago, Baffin Bay/Davis Strait, and Hudson Bay Complex. Limited observations, especially for ecosystem variables, compromise the trend analyses. Future projections are predominately from global models with few contributions from available marine basin scale models. The assessment indicates a significant increase in air temperature, slight increases in precipitation and snow depth, and appreciable changes in atmospheric circulation patterns. Projections suggest an increase in storm strength and size, leading to enhanced storm surges and coastal erosion, a slight increase in wave heights, increases in gustiness, and small changes in mean wind speed. An Arctic-wide decrease in the extent of multiyear ice and a spatial and temporal increase in ice-free waters in summer have been observed and are projected to continue into the future. Limited observations of ocean properties show local freshening (Beaufort Sea) and summer warming (Baffin Bay). These trends are projected to continue along with localized strengthening in stratification. Increased ocean acidification has been observed and is projected to continue throughout the Canadian Arctic, leading to severely decreased saturation states of calcium carbonate (aragonite and calcite). Qualitative analysis of biological observations indicate large regional differences. Increased primary production and double bloom development is seen in areas of sea ice retreat where nutrient supply is sufficient, and unchanged or reduced production is seen where nutrients are low or suppressed in response to enhanced stratification. Future primary production projections show inconsistent results, with light-dependent increase or nutrient-limited decrease dominating, dependent on the model. For the next decade, natural intradecadal variability is expected to be of similar importance as longer-term trends. To improve our capacity to assess and project climate change adaptation in marine ecosystems, more consistent observations are needed, especially over marine areas and for biogeochemical variables. Higher resolution basin-scale models are also required to provide locally applicable projections relevant for Arctic communities and management units.},
author = {Steiner, Nadja and Azetsu-Scott, Kumiko and Hamilton, Jim and Hedges, Kevin and Hu, Xianmin and Janjua, Muhammad Y. and Lavoie, Diane and Loder, John and Melling, Humfrey and Merzouk, Anissa and Perrie, William and Peterson, Ingrid and Scarratt, Michael and Sou, Tessa and Tallmann, Ross},
doi = {10.1139/er-2014-0066},
isbn = {1181-8700},
issn = {1181-8700},
journal = {Environmental Reviews},
keywords = {Arctic,chemistry,field,ocean acidification},
number = {2},
pages = {191--239},
title = {{Observed trends and climate projections affecting marine ecosystems in the Canadian Arctic}},
url = {http://www.nrcresearchpress.com/doi/10.1139/er-2014-0066},
volume = {23},
year = {2015}
}
@article{Benestad2016,
abstract = {We present an outlook for a number of climate parameters for temperature, precipitation, and storm statistics in the Barents region. Projected temperatures exhibited strongest increase over northern Fennoscandia and the high Arctic, exceeding 7 °C by 2099 for a typical 'warm winter' under the RCP4.5 scenario. More extreme temperatures may be expected with the RCP8.5, with an increase exceeding 18 °C in some places. The magnitude of the day-to-day variability in temperature is likely to decrease with higher temperatures. The skill of the downscaling models was moderate for the wet-day frequency for which the projections indicated both increases and decreases within the range of −5– +10{\%} by 2099. The downscaled results for the wet-day mean precipitation was poor, but for the warming associated with RCP 4.5, it could result in wet-day mean precipitation being intensified by as much as 70{\%} in 2099. The number of synoptic storms over the Barents Sea was found to increase with a warming in the Arctic, however, other climate parameters may not change much, such as the persistence of the temperature and precipitation. These climate change projections were derived using a new strategy for empirical-statistical downscaling, making use of principal component analysis to represent the local climate parameters and large ensembles of global climate model (GCM) simulations to provide information about the large scales. The method and analysis were validated on three different levels: (a) the representativeness of the GCMs, (b) traditional validation of the downscaling method, and (c) assessment of the ensembles of downscaled results in terms of past trends and interannual variability.},
author = {Benestad, Rasmus E. and Parding, Kajsa M. and Isaksen, Ketil and Mezghani, Abdelkader},
doi = {10.1088/1748-9326/11/5/054017},
issn = {17489326},
journal = {Environmental Research Letters},
keywords = {Climate change,Downscaling,Precipitation,Storms,Temperature,The Barents region},
number = {5},
pages = {054017},
publisher = {IOP Publishing},
title = {{Climate change and projections for the Barents region: What is expected to change and what will stay the same?}},
url = {http://stacks.iop.org/1748-9326/11/i=5/a=054017?key=crossref.b19f073c1996a2fe9cd8afedfdd86b18},
volume = {11},
year = {2016}
}
@article{Henle2004,
abstract = {We reviewed empirical data and hypotheses derived from demographic, optimal foraging, life-history, community, and biogeographic theory for predicting the sensitivity of species to habitat fragmentation. We found 12 traits or trait groups that have been suggested as predictors of species sensitivity: population size; population fluctuation and storage effect; dispersal power; reproductive potential; annual survival; sociality; body size; trophic position; ecological specialisation, microhabitat and matrix use; disturbance and competition sensitive traits; rarity; and biogeographic position. For each trait we discuss the theoretical justification for its sensitivity to fragmentation and empirical evidence for and against the suitability of the trait as a predictor of fragmentation sensitivity. Where relevant, we also discuss experimental design problems for testing the underlying hypotheses. There is good empirical support for 6 of the 12 traits as sensitivity predictors: population size; population fluctuation and storage effects; traits associated with competitive ability and disturbance sensitivity in plants; microhabitat specialisation and matrix use; rarity in the form of low abundance within a habitat; and relative biogeographic position. Few clear patterns emerge for the remaining traits from empirical studies if examined in isolation. Consequently, interactions of species traits and environmental conditions must be considered if we want to be able to predict species sensitivity to fragmentation. We develop a classification of fragmentation sensitivity based on specific trait combinations and discuss the implications of the results for ecological theory. },
archivePrefix = {arXiv},
arxivId = {arXiv:1112.2903v1},
author = {Henle, Klaus and Davies, Kendi F. and Kleyer, Michael and Margules, Chris and Settele, Josef},
doi = {10.1023/B:BIOC.0000004319.91643.9e},
eprint = {arXiv:1112.2903v1},
isbn = {0960-3115},
issn = {09603115},
journal = {Biodiversity and Conservation},
keywords = {Biogeographic traits,Demographic traits,Ecological traits,Empirical evidence,Extinction proneness,Habitat fragmentation,Sensitivity indicators,Testing,Theory},
number = {1},
pages = {207--251},
pmid = {445},
title = {{Predictors of species sensitivity to fragmentation}},
volume = {13},
year = {2004}
}
@incollection{Parry2007,
address = {Cambridge},
archivePrefix = {arXiv},
arxivId = {1109.1006v1},
author = {Parry, M L and Canziani, O F and Palutikof, J P and van der Linden, P J and Hanson, C E},
booktitle = {Working Group II Contribution to the Intergovernmental Panel on Climate Change Fourth Assessment Report},
doi = {volume},
editor = {Parry, M. L. and Canziani, O. F. and Palutikof, J. P. and van der Linden, P. J. and Hanson, C. E.},
eprint = {1109.1006v1},
isbn = {9780521880107},
issn = {01480227},
pages = {653--685},
pmid = {1040912},
publisher = {Cambridge University Press},
title = {{Climate Change 2007 – Impacts, Adaptation and Vulnerability: Climate Change 2007 – Impacts, Adaptation and Vulnerability}},
year = {2007}
}
@article{Buisson2008,
abstract = {Stream fish are expected to be significantly influenced by climate change, as they are ectothermic animals whose dispersal is limited within hydrographic networks. Nonetheless, they are also controlled by other physical factors that may prevent them moving to new thermally suitable sites. Using presence-absence records in 655 sites widespread throughout nine French river units, we predicted the potential future distribution of 30 common stream fish species facing temperature warming and change in precipitation regime. We also assessed the potential impacts on fish assemblages' structure and diversity. Only cold-water species, whose diversity is very low in French streams, were predicted to experience a strong reduction in the number of suitable sites. In contrast, most cool-water and warm-water fish species were projected to colonize many newly suitable sites. Considering that cold headwater streams are the most numerous on the Earth's surface, our results suggested that headwater species would undergo a deleterious effect of climate change, whereas downstream species would expand their range by migrating to sites located in intermediate streams or upstream. As a result, local species richness was forecasted to increase greatly and high turnover rates indicated future fundamental changes in assemblages' structure. Changes in assemblage composition were also positively related to the intensity of warming. Overall, these results (1) stressed the importance of accounting for both climatic and topographic factors when assessing the future distribution of riverine fish species and (2) may be viewed as a first estimation of climate change impacts on European freshwater fish assemblages. {\textordfeminine} 2008 The Authors Journal compilation {\textordfeminine} 2008 Blackwell Publishing},
author = {Buisson, La{\"{e}}titia and Thuiller, Wilfried and Lek, Sovan and Lim, Puy and Grenouillet, Ga{\"{e}}l},
doi = {10.1111/j.1365-2486.2008.01657.x},
isbn = {1354-1013},
issn = {13541013},
journal = {Global Change Biology},
keywords = {Climate change,Fish assemblages,GAM,Predictive models,Species distribution,Species turnover,Stream fish,Upstream-downstream gradient},
number = {10},
pages = {2232--2248},
pmid = {7995},
title = {{Climate change hastens the turnover of stream fish assemblages}},
volume = {14},
year = {2008}
}
@article{Flenner2008,
abstract = {1. Climate change affects many ecosystems on earth. If not dying out or migrating, the species affected have to survive the altered conditions, including changes in community structure. It is, however, usually difficult to distinguish changes caused by a changing climate from other factors. 2. Forestry is considered to be the major disturbance factor in Swedish forests. Here, we use forest lake data sets from 1996 and 2006 which include species abundance data for dragonfly larvae, water plant structure, forest age and forestry measures during a period of 25 years: from 1980 to 2005. Hence, we were able to discriminate between forestry effects and changes in species composition driven by recent climate change. 3. We explored effects on regional species composition, species abundance and ecosystem functions, such as changes in niche use, utilising dragonflies (Odonata) as model organisms. 4. Our results show that dragonflies react rapidly to climate change, showing strong responses over such a short time span as 10 years. We observed changes in both species composition and abundance; former rare species have become more frequent and now occur in lakes of a wider quality range, while former widespread species have become more selective in their choice of waters. The new communities harbour about the same number of species as before, but seen from a regional perspective, diversity is reduced. 5. We predict that the altered species composition and abundance might raise new demands in conservation planning as well as altering the ecological functions of the aquatic systems.},
author = {FLENNER, IDA and SAHLN, GRAN},
doi = {10.1111/j.1752-4598.2008.00020.x},
isbn = {1752-4598},
issn = {1752458X},
journal = {Insect Conservation and Diversity},
keywords = {Aquatic,Climate Change,Communities,Dragonflies,Ecosystems,Function,NUMBER,Odonata,Perspective,Recent,Responses,Species Abundance,Structure,abundance,age,change,choice,climate,climate change,community,community structure,composition,conservation,conservation planning,disturbance,diversity,dragonfly,dragonfly larvae,ecology,ecosystem,effects,effects on,forest,forestry,lake,larvae,model,niche,odonata,plant,quality,rare species,regional species pool,species,species composition,system,systems,turnover,water},
number = {3},
pages = {169--179},
pmid = {834},
title = {{Dragonfly community re-organisation in boreal forest lakes: rapid species turnover driven by climate change?}},
url = {http://doi.wiley.com/10.1111/j.1752-4598.2008.00020.x},
volume = {1},
year = {2008}
}
@article{Vanbergen2013,
abstract = {Insect pollinators of crops and wild plants are under threat globally and their decline or loss could have profound economic and environmental consequences. Here, we argue that multiple anthropogenic pressures - including land-use intensification, climate change, and the spread of alien species and diseases - are primarily responsible for insect-pollinator declines. We show that a complex interplay between pressures (eg lack of food sources, diseases, and pesticides) and biological processes (eg species dispersal and interactions) at a range of scales (from genes to ecosystems) underpins the general decline in insect-pollinator populations. Interdisciplinary research on the nature and impacts of these interactions will be needed if human food security and ecosystem function are to be preserved. We highlight key areas that require research focus and outline some practical steps to alleviate the pressures on pollinators and the pollination services they deliver to wild and crop plants.},
archivePrefix = {arXiv},
arxivId = {arXiv:1011.1669v3},
author = {Vanbergen, Adam J. and Garratt, M. P. and Vanbergen, Adam J. and Baude, Mathilde and Biesmeijer, Jacobus C. and Britton, Nicholas F. and Brown, Mark J.F. and Brown, Mike and Bryden, John and Budge, Giles E. and Bull, James C. and Carvell, Claire and Challinor, Andrew J. and Connolly, Christopher N. and Evans, David J. and Feil, Edward J. and Garratt, Mike P. and Greco, Mark K. and Heard, Matthew S. and Jansen, Vincent A.A. and Keeling, Matt J. and Kunin, William E. and Marris, Gay C. and Memmott, Jane and Murray, James T. and Nicolson, Susan W. and Osborne, Juliet L. and Paxton, Robert J. and Pirk, Christian W.W. and Polce, Chiara and Potts, Simon G. and Priest, Nicholas K. and Raine, Nigel E. and Roberts, Stuart and Ryabov, Eugene V. and Shafir, Sharoni and Shirley, Mark D.F. and Simpson, Stephen J. and Stevenson, Philip C. and Stone, Graham N. and Termansen, Mette and Wright, Geraldine A.},
doi = {10.1890/120126},
eprint = {arXiv:1011.1669v3},
isbn = {1540-9295},
issn = {15409309},
journal = {Frontiers in Ecology and the Environment},
number = {5},
pages = {251--259},
pmid = {25246403},
title = {{Threats to an ecosystem service: Pressures on pollinators}},
volume = {11},
year = {2013}
}
@article{Emer2016,
abstract = {Aim Alien species alter interaction networks by disrupting existing interactions, for example between plants and pollinators, and by engaging in new interac- tions. Predicting the effects of an incoming invader can be difficult, although recent work suggests species roles in interaction networks may be conserved across locations. We test whether species roles in plant–pollinator networks dif- fer between their native and alien ranges and whether the former can be used to predict the latter. Location World-wide. Methods We used 64 plant–pollinator networks to search for species occurring in at least one network in its native range and one network in its alien range. We found 17 species meeting these criteria, distributed in 48 plant–pollinator networks. We characterized each species' role by estimating species-level net- work indices: normalized degree, closeness centrality, betweenness centrality and two measures of contribution to modularity (c- and z-scores). Linear mixed models and linear regression models were used to test for differences in species role between native and alien ranges and to predict those roles from the native to the alien range, respectively. Results Species roles varied considerably across species. Nevertheless, although species lost their native mutualists and gained novel interactions in the alien community, their role did not differ significantly between ranges. Consequently, closeness centrality and normalized degree in the alien range were highly pre- dictable from the native range networks. Main conclusions Species with high degree and centrality define the core of nested networks. Our results suggest that core species are likely to establish interactions and be core species in the alien range, whilst species with few interactions in their native range will behave similarly in their alien range. Our results provide new insights into species role conservatism and could help ecol- ogists to predict alien species impact at the community level.},
author = {Emer, Carine and Memmott, Jane and Vaughan, Ian P. and Montoya, Daniel and Tylianakis, Jason M.},
doi = {10.1111/ddi.12458},
file = {::},
isbn = {1472-4642},
issn = {14724642},
journal = {Diversity and Distributions},
keywords = {biological invasions,centrality,conservatism,ecological networks,pollination,predicting invasion},
number = {8},
pages = {841--852},
title = {{Species roles in plant–pollinator communities are conserved across native and alien ranges}},
volume = {22},
year = {2016}
}
@article{Aizen2008,
abstract = {Plant-animal mutualisms are characterized by weak or asymmetric mutual dependences between interacting species, a feature that could increase community stability. If invasive species integrate into mutualistic webs, they may alter web structure, with consequences for species persistence. However, the effect of alien mutualists on the architecture of plant-pollinator webs remains largely unexplored. We analyzed the extent of mutual dependency between interacting species, as a measure of mutualism strength, and the connectivity of 10 paired plant-pollinator webs, eight from forests of the southern Andes and two from oceanic islands, with different incidences of alien species. Highly invaded webs exhibited weaker mutualism than less-invaded webs. This potential increase in network stability was the result of a disproportionate increase in the importance and participation of alien species in the most asymmetric interactions. The integration of alien mutualists did not alter overall network connectivity, but links were transferred from generalist native species to super-generalist alien species during invasion. Therefore, connectivity among native species declined in highly invaded webs. These modifications in the structure of pollination webs, due to dominance of alien mutualists, can leave many native species subject to novel ecological and evolutionary dynamics.},
author = {Aizen, Marcelo A. and Morales, Carolina L. and Morales, Juan M.},
doi = {10.1371/journal.pbio.0060031},
isbn = {1544-9173},
issn = {15449173},
journal = {PLoS Biology},
number = {2},
pages = {0396--0403},
pmid = {18271628},
title = {{Invasive mutualists erode native pollination webs}},
volume = {6},
year = {2008}
}
@article{Leong2015,
abstract = {Urbanization and agricultural intensification of landscapes are important drivers of global change, which in turn have direct impacts on local ecological communities leading to shifts in species distributions and interactions. Here, we illustrate how human-altered landscapes, with novel ornamental and crop plant communities, result not only in changes to local community diversity of floral-dependent species, but also in shifts in seasonal abundance of bee pollinators. Three years of data on the spatio-temporal distributions of 91 bee species show that seasonal patterns of abundance and species richness in human-altered landscapes varied significantly less compared to natural habitats in which floral resources are relatively scarce in the dry summer months. These findings demonstrate that anthropogenic environmental changes in urban and agricultural systems, here mediated through changes in plant resources and water inputs, can alter the temporal dynamics of pollinators that depend on them. Changes in phenology of interactions can be an important, though frequently overlooked, mechanism of global change.},
author = {Leong, Misha and Ponisio, Lauren C. and Kremen, Claire and Thorp, Robbin W. and Roderick, George K.},
doi = {10.1111/gcb.13141},
isbn = {1354-1013},
issn = {13652486},
journal = {Global Change Biology},
keywords = {Agricultural,Bees,Ecosystem services,Land-use change,Phenology,Pollinators,Seasonality,Species distributions,Urban ecology},
number = {3},
pages = {1046--1053},
pmid = {26663622},
title = {{Temporal dynamics influenced by global change: Bee community phenology in urban, agricultural, and natural landscapes}},
volume = {22},
year = {2016}
}
@article{Nilsson2016,
abstract = {Interaction strength (IS) has been theoretically shown to play a major role in governing the stability and dynamics of food webs. Nonetheless, its definition has been varied and problematic, including a range of recent definitions based on biological rates associated with model parameters (e.g., attack rate). Results from food web theory have been used to argue that IS metrics based on energy flux ought to have a clear relationship with stability. Here, we use simple models to elucidate the actual relationship between local stability and a number of common IS metrics (total flux and per capita fluxes) as well as a more recently suggested metric. We find that the classical IS metrics map to stability in a more complex way than suggested by existing food web theory and that the new IS metric has a much clearer, and biologically interpretable, relationship with local stability. The total energy flux metric falls off existing theoretical predictions when the total resource productivity available to the consumer is reduced despite increased consumer attack rates. The density of a consumer can hence decrease when its attack rate increases. This effect, called the paradox of attack rate, is similar to the well-known hydra effect and can even cascade up a food chain to exclude a predator when consumer attack rate is increased.},
author = {Nilsson, Karin A. and McCann, Kevin S.},
doi = {10.1007/s12080-015-0282-8},
issn = {18741746},
journal = {Theoretical Ecology},
keywords = {Hydra effect,Logistic growth,Lotka-Volterra,Paradox of attack rate,Paradox of searching efficiency,Rosenzweig-MacArthur},
number = {1},
pages = {59--71},
title = {{Interaction strength revisited—clarifying the role of energy flux for food web stability}},
volume = {9},
year = {2016}
}
@article{Jordan2007,
abstract = {Conservation biology should focus more on the importance rather than the rarity of species, although the definition and quantification of importance are not easy. One approach involves measuring the positional importance (e.g. centrality) of species in ecological interaction networks to provide a basis for species ranking. However, there are many centrality indices, each reflecting a particular aspect of positional importance and therefore giving a rank order of species different from those provided by alternative formulations. Thus, there is a strong need for comparing the available indices and for examining their relative merits in network analysis. In this paper, we apply 13 centrality indices to the "species" (trophic components) of methodologically comparable trophic flow networks, in order to answer the following questions: (1) What is the disagreement between different indices regarding the rank of a given species in a given network? (2) How is this disagreement in performance influenced by the choice of the network? (3) What is the overall relationship among these indices and, in particular, which are the most similar to degree (the simplest index of all, being equal to the number of links pertaining to a given node)? We compare the 13 indices based on the data of nine networks using metric and rank statistics and multivariate analysis procedures. We conclude that (1) different centrality ranks differ in each network; (2) different webs can be characterized by different relationships between ranks but there is a robust pattern of relationships among the indices, some index pairs behaving very similarly in all networks; and (3) it is the index of closeness centrality which provides a rank most similar to that based on degree. {\textcopyright} 2007 Elsevier B.V. All rights reserved.},
author = {Jord{\'{a}}n, Ferenc and Benedek, Zs{\'{o}}fia and Podani, J{\'{a}}nos},
doi = {10.1016/j.ecolmodel.2007.02.032},
isbn = {0304-3800},
issn = {03043800},
journal = {Ecological Modelling},
keywords = {Centrality,Clustering,Food web,Indirect effect,Keystone species,Network analysis,Ordination,Ranking},
number = {1-2},
pages = {270--275},
pmid = {879},
title = {{Quantifying positional importance in food webs: A comparison of centrality indices}},
volume = {205},
year = {2007}
}
@article{Mello2015,
abstract = {A central issue in ecology is the definition and identification of keystone species, i.e. species that are relatively more important than others for maintaining community structure and ecosystem functioning. Network theory has been pointed out as a robust theoretical framework to enhance the operationality of the keystone species concept. We used the concept of centrality as a proxy for a species ' relative importance for the structure of seed dispersal networks composed of either frugivorous bats or birds and their food-plants. Centrality was expected to be determined mainly by dietary specialization, but also by body mass and geographic range size. Across 15 Neotropical datasets, only specialized frugivore species reached the highest values of centrality. Furthermore, the centrality of specialized frugivores varied widely within and among networks, whereas that of secondary and opportunistic frugivores was consistently low. A mixed-eff ects model showed that centrality was best explained by dietary specialization, but not by body mass or range size. Furthermore, the relationship between centrality and those three ecological correlates diff ered between bat – and bird – fruit networks. Our findings suggest that dietary specialization is key to understand what makes a frugivore species a keystone in seed dispersal networks, and that taxonomic identity also plays a signifi cant role. Specialized frugivores may play a central role in network structuring and ecosystem functioning, which has important implications for conservation and restoration.},
archivePrefix = {arXiv},
arxivId = {arXiv:1011.1669v3},
author = {Mello, Marco Aurelio Ribeiro and Rodrigues, Francisco Aparecido and Costa, Luciano da Fontoura and Kissling, W. Daniel and Şekercioğlu, {\c{C}}ağan H. and Marquitti, Flavia Maria Darcie and Kalko, Elisabeth Klara Viktoria},
doi = {10.1111/oik.01613},
eprint = {arXiv:1011.1669v3},
isbn = {1600-0706},
issn = {16000706},
journal = {Oikos},
number = {8},
pages = {1031--1039},
pmid = {91546240},
title = {{Keystone species in seed dispersal networks are mainly determined by dietary specialization}},
volume = {124},
year = {2015}
}
@article{Bluthgen2006,
abstract = {BACKGROUND: Network analyses of plant-animal interactions hold valuable biological information. They are often used to quantify the degree of specialization between partners, but usually based on qualitative indices such as 'connectance' or number of links. These measures ignore interaction frequencies or sampling intensity, and strongly depend on network size.$\backslash$n$\backslash$nRESULTS: Here we introduce two quantitative indices using interaction frequencies to describe the degree of specialization, based on information theory. The first measure (d') describes the degree of interaction specialization at the species level, while the second measure (H2') characterizes the degree of specialization or partitioning among two parties in the entire network. Both indices are mathematically related and derived from Shannon entropy. The species-level index d' can be used to analyze variation within networks, while H2' as a network-level index is useful for comparisons across different interaction webs. Analyses of two published pollinator networks identified differences and features that have not been detected with previous approaches. For instance, plants and pollinators within a network differed in their average degree of specialization (weighted mean d'), and the correlation between specialization of pollinators and their relative abundance also differed between the webs. Rarefied sampling effort in both networks and null model simulations suggest that H2' is not affected by network size or sampling intensity.$\backslash$n$\backslash$nCONCLUSION: Quantitative analyses reflect properties of interaction networks more appropriately than previous qualitative attempts, and are robust against variation in sampling intensity, network size and symmetry. These measures will improve our understanding of patterns of specialization within and across networks from a broad spectrum of biological interactions.},
author = {Bl{\"{u}}thgen, Nico and Menzel, Florian and Bl{\"{u}}thgen, Nils},
doi = {10.1186/1472-6785-6-9},
file = {:Users/alyssacirtwill/Documents/Papers/Bl{\"{u}}thgen, Menzel, Bl{\"{u}}thgen{\_}2006{\_}BMC Ecology.pdf:pdf},
isbn = {1472-6785},
issn = {14726785},
journal = {BMC Ecology},
keywords = {Adaptation,Animals,Argentina,Bees,Bees: physiology,Biological,Computer Simulation,Ecosystem,Food Chain,Great Britain,Models,Physiological,Plant Physiological Phenomena,Pollen,Population Dynamics,Species Specificity},
number = {1},
pages = {9},
pmid = {16907983},
title = {{Measuring specialization in species interaction networks}},
url = {http://bmcecol.biomedcentral.com/articles/10.1186/1472-6785-6-9},
volume = {6},
year = {2006}
}
@article{Memmott2004,
abstract = {Mutually beneficial interactions between flowering plants and animal pollinators represent a critical 'ecosystem service' under threat of anthropogenic extinction. We explored probable patterns of extinction in two large networks of plants and flower visitors by simulating the removal of pollinators and consequent loss of the plants that depend upon them for reproduction. For each network, we removed pollinators at random, systematically from least-linked (most specialized) to most-linked (most generalized), and systematically from most- to least-linked. Plant species diversity declined most rapidly with preferential removal of the most-linked pollinators, but declines were no worse than linear. This relative tolerance to extinction derives from redundancy in pollinators per plant and from nested topology of the networks. Tolerance in pollination networks contrasts with catastrophic declines reported from standard food webs. The discrepancy may be a result of the method used: previous studies removed species from multiple trophic levels based only on their linkage, whereas our preferential removal of pollinators reflects their greater risk of extinction relative to that of plants. In both pollination networks, the most-linked pollinators were bumble-bees and some solitary bees. These animals should receive special attention in efforts to conserve temperate pollination systems.;Mutually beneficial interactions between flowering plants and animal pollinators represent a critical 'ecosystem service' under threat of anthropogenic extinction. We explored probable patterns of extinction in two large networks of plants and flower visitors by simulating the removal of pollinators and consequent loss of the plants that depend upon them for reproduction. For each network, we removed pollinators at random, systematically from least-linked (most specialized) to most-linked (most generalized), and systematically from most- to least-linked. Plant species diversity declined most rapidly with preferential removal of the most-linked pollinators, but declines were no worse than linear. This relative tolerance to extinction derives from redundancy in pollinators per plant and from nested topology of the networks. Tolerance in pollination networks contrasts with catastrophic declines reported from standard food webs. The discrepancy may be a result of the method used: previous studies removed species from multiple trophic levels based only on their linkage, whereas our preferential removal of pollinators reflects their greater risk of extinction relative to that of plants. In both pollination networks, the most-linked pollinators were bumble-bees and some solitary bees. These animals should receive special attention in efforts to conserve temperate pollination systems.;Mutually beneficial interactions between flowering plants and animal pollinators represent a critical 'ecosystem service' under threat of anthropogenic extinction. We explored probable patterns of extinction in two large networks of plants and flower visitors by simulating the removal of pollinators and consequent loss of the plants that depend upon them for reproduction. For each network, we removed pollinators at random, systematically from least-linked (most specialized) to most-linked (most generalized), and systematically from most- to least-linked. Plant species diversity declined most rapidly with preferential removal of the most-linked pollinators, but declines were no worse than linear. This relative tolerance to extinction derives from redundancy in pollinators per plant and from nested topology of the networks. Tolerance in pollination networks contrasts with catastrophic declines reported from standard food webs. The discrepancy may be a result of the method used: previous studies removed species from multiple trophic levels based only on their linkage, whereas our preferential removal of pollinators reflects their greater risk of extinction relative to that of plants. In both pollination networks, the most-linked pollinators were bumble-bees and some solita y bees. These animals should receive special attention in efforts to conserve temperate pollination systems.;Mutually beneficial interactions between flowering plants and animal pollinators represent a critical `ecosystem service' under threat of anthropogenic extinction. We explored probable patterns of extinction in two large networks of plants and flower visitors by simulating the removal of pollinators and consequent loss of the plants that depend upon them for reproduction. For each network, we removed pollinators at random, systematically from least-linked (most specialized) to most-linked (most generalized), and systematically from most- to least-linked. Plant species diversity declined most rapidly with preferential removal of the most-linked pollinators, but declines were no worse than linear. This relative tolerance to extinction derives from redundancy in pollinators per plant and from nested topology of the networks. Tolerance in pollination networks contrasts with catastrophic declines reported from standard food webs. The discrepancy may be a result of the method used: previous studies removed species from multiple trophic levels based only on their linkage, whereas our preferential removal of pollinators reflects their greater risk of extinction relative to that of plants. In both pollination networks, the most-linked pollinators were bumble-bees and some solitary bees. These animals should receive special attention in efforts to conserve temperate pollination systems.;Mutually beneficial interactions between flowering plants and animal pollinators represent a critical ‘ecosystem service' under threat of anthropogenic extinction. We explored probable patterns of extinction in two large networks of plants and flower visitors by simulating the removal of pollinators and consequent loss of the plants that depend upon them for reproduction. For each network, we removed pollinators at random, systematically from least–linked (most specialized) to most–linked (most generalized), and systematically from most– to least–linked. Plant species diversity declined most rapidly with preferential removal of the most–linked pollinators, but declines were no worse than linear. This relative tolerance to extinction derives from redundancy in pollinators per plant and from nested topology of the networks. Tolerance in pollination networks contrasts with catastrophic declines reported from standard food webs. The discrepancy may be a result of the method used: previous studies removed species from multiple trophic levels based only on their linkage, whereas our preferential removal of pollinators reflects their greater risk of extinction relative to that of plants. In both pollination networks, the most–linked pollinators were bumble–bees and some solitary bees. These animals should receive special attention in efforts to conserve temperate pollination systems.;Mutually beneficial interactions between flowering plants and animal pollinators represent a critical 'ecosystem service' under threat of anthropogenic extinction. We explored probable patterns of extinction in two large networks of plants and flower visitors by simulating the removal of pollinators and consequent loss of the plants that depend upon them for reproduction. For each network, we removed pollinators at random, systematically from least-linked (most specialized) to most-linked (most generalized), and systematically from most- to least-linked. Plant species diversity declined most rapidly with preferential removal of the most-linked pollinators, but declines were no worse than linear. This relative tolerance to extinction derives from redundancy in pollinators per plant and from nested topology of the networks. Tolerance in pollination networks contrasts with catastrophic declines reported from standard food webs. The discrepancy may be a result of the method used: previous studies removed species from multiple trophic levels based only on their linkage, whereas our preferential removal of pollinators reflects their greater risk of extincti n relative to that of plants. In both pollination networks, the most-linked pollinators were bumble-bees and some solitary bees. These animals should receive special attention in efforts to conserve temperate pollination systems.;},
author = {Memmott, Jane and Waser, Nickolas M. and Price, Mary V.},
doi = {10.1098/rspb.2004.2909},
isbn = {0962-8452},
issn = {14712970},
journal = {Proceedings of the Royal Society B: Biological Sciences},
keywords = {Conservation,Food webs,Generalization,Nestedness,Pollination,Redundancy},
number = {1557},
pages = {2605--2611},
pmid = {15615687},
title = {{Tolerance of pollination networks to species extinctions}},
url = {http://murdoch.summon.serialssolutions.com/2.0.0/link/0/eLvHCXMwnV3fb9QwDLYmBBISmtgYUDakPvBT6I4mTZvmYQ9oYuIPGM9R2iRiErTVtZO2{\_}fXYadPtxpAQj9c6d03scz439meAnK-z1R2foEpHPJhOeNxBGptztBNmHILbpvHclpGdIaSp7kDMIqMky61UxXV7{\_}iOkW87rugn91tae4ka6ZI-J{\_}EWh4seu-3nc{\_}wrj3PW},
volume = {271},
year = {2004}
}
@article{Olesen2011a,
abstract = {Nature is organized into complex, dynamical networks of species and their interactions, which may influence diversity and stability. However, network research is, generally, short-term and depict ecological networks as static structures only, devoid of any dynamics. This hampers our understanding of how nature responds to larger disturbances such as changes in climate. In order to remedy this we studied the long-term (12-yrs) dynamics of a flower-visitation network, consisting of flower-visiting butterflies and their nectar plants. Global network properties, i.e. numbers of species and links, as well as connectance, were temporally stable, whereas most species and links showed a strong temporal dynamics. However, species of butterflies and plants varied bimodally in their temporal persistance: Sporadic species, being present only 1-2(-5) years, and stable species, being present (9-)11-12 years, dominated the networks. Temporal persistence and linkage level of species, i.e. number of links to other species, made up two groups of species: Specialists with a highly variable temporal persistence, and temporally stable species with a highly variable linkage level. Turnover of links of specialists was driven by species turnover, whereas turnover of links among generalists took place through rewiring, i.e. by reshuffling existing interactions. However, in spite of this strong internal dynamics of species and links the network appeared overall stable. If this global stability-local instability phenomenon is general, it is a most astonishing feature of ecological networks.},
author = {Olesen, Jens M. and Stefanescu, Constant{\'{i}} and Traveset, Anna},
doi = {10.1371/journal.pone.0026455},
file = {:Users/alyssacirtwill/Documents/Papers/Olesen, Stefanescu, Traveset{\_}2011{\_}PLoS ONE.PDF:PDF},
isbn = {1932-6203},
issn = {19326203},
journal = {PLoS ONE},
number = {11},
pages = {e26455},
pmid = {22125597},
title = {{Strong, Long-Term Temporal Dynamics of an Ecological Network}},
volume = {6},
year = {2011}
}
@article{Milo2004,
abstract = {Complex biological, technological, and sociological networks can be of very different sizes and connectivities, making it difficult to compare their structures. Here we present an approach to systematically study similarity in the local structure of networks, based on the significance profile (SP) of small subgraphs in the network compared to randomized networks. We find several superfamilies of previously unrelated networks with very similar SPs. One superfamily, including transcription networks of microorganisms, represents "rate-limited" information-processing networks strongly constrained by the response time of their components. A distinct superfamily includes protein signaling, developmental genetic networks, and neuronal wiring. Additional superfamilies include power grids, protein-structure networks and geometric networks, World Wide Web links and social networks, and word-adjacency networks from different languages.},
issn = {1095-9203},
journal = {Science},
number = {5663},
pages = {1538--1542},
title = {{Superfamilies of evolved and designed networks.}},
volume = {303},
year = {2004}
}
@article{Hoye2008,
abstract = {The short summers of the High Arctic pose a strong time constraint on the annual cycle of all organisms in this region. Although arctic arthropods can complete their development at very low temperatures, the predicted climatic changes may shift their phenology outside its normal range. Hence, arctic arthropods may become exposed to conditions to which they are not adapted. On the basis of long-term data from several plots of pitfall and window traps at Zackenberg in high-arctic Northeast Greenland, we document that the timing of emergence is closely related to date of snowmelt in nine taxa of common surface-active and flying arthropods. Average air temperature seemed to play a lesser role, although the duration from snowmelt to the date when 50{\%} of the individuals in the season were caught (date50) was negatively related to the average daily air temperature during the same time interval in three of the nine taxa. Since short-term weather fluctuations appeared to have a small effect on capture numbers in pitfall and window traps, we suggest that timing of snowmelt is a good predictor of the phenology of most arthropods in high-arctic Greenland. The spatial synchrony of capture numbers between individual traps within plots was high. However, among pairs of plots, the spatial synchrony varied between taxa and habitats and declined with distance between plots for surface-dwelling taxa and with difference in timing of snowmelt for the most abundant families of Diptera (Muscidae and Chironomidae). Detritus feeders (collembolans, mites and most larvae of Diptera) and predators (spiders of the families Linyphiidae and Lycosidae) were abundant throughout the summer season. In contrast, the abundance of more specialized groups, like butterflies (e.g., Nymphalidae) and parasitoid wasps (e.g., Ichneumonidae), was restricted to a narrow seasonal time window in the warmest part of the summer. Because of their narrow phenological range and their host specialization, these taxa may be most vulnerable to trophic mismatch. Furthermore, snowmelt is predicted to become more variable, and this may affect organisms in areas of late snowmelt most severely. {\textcopyright} 2008 Elsevier Inc. All rights reserved.},
archivePrefix = {arXiv},
arxivId = {arXiv:1011.1669v3},
author = {H{\o}ye, Toke T. and Forchhammer, Mads C.},
doi = {10.1016/S0065-2504(07)00013-X},
eprint = {arXiv:1011.1669v3},
isbn = {9780123736659},
issn = {00652504},
journal = {Advances in Ecological Research},
number = {7},
pages = {299--324},
pmid = {25246403},
title = {{Phenology of High-Arctic Arthropods: Effects of Climate on Spatial, Seasonal, and Inter-Annual Variation}},
volume = {40},
year = {2008}
}
@article{Gaston1991,
author = {Gaston, Kevin J},
journal = {Oikos},
number = {3},
pages = {434--438},
title = {{How large is a species' range?}},
volume = {61},
year = {1991}
}
@article{Odegaard2006,
abstract = {Species diversity, host specificity and species turnover among phytophagous beetles were studied in the canopy of two tropical lowland forests in Panama with the use of canopy cranes. A sharp rainfall gradient occurs between the two sites located 80 km apart. The wetter forest is located in San Lorenzo Protected Area on the Caribbean side of the isthmus, and the drier forest is a part of the Parque Natural Metropolitano close to Panama City on the Pacific slope. Host specificity was measured as effective specialization and recorded by probability methods based on abundance categories and feeding records from a total of 102 species of trees and lianas equally distributed between the two sites. The total material collected included more than 65,000 beetles of 2462 species, of which 306 species were shared between the two sites. The wet forest was 37{\%} more species rich than the dry forest due to more saproxylic species and flower visitors. Saproxylic species and flower visitors were also more host-specific in the wet forest. Leaf chewers showed similar levels of species richness and host specificity in both forests. The effective number of specialized species per plant species was higher in the wet forest. Higher levels of local alpha- and beta-diversity as well as host specificity based on present data from a tropical wet forest, suggests higher number of species at regional levels, a result that may have consequences for ecological estimates of global species richness.},
author = {{\O}degaard, Frode},
doi = {10.1007/s10531-004-3106-5},
isbn = {0960-3115},
issn = {09603115},
journal = {Biodiversity and Conservation},
keywords = {Biodiversity,Canopy,Effective specialization,Insect herbivores,Species turnover},
number = {1},
pages = {83--105},
title = {{Host specificity, alpha- and beta-diversity of phytophagous beetles in two tropical forests in Panama}},
volume = {15},
year = {2006}
}
@article{Mortensen2014,
abstract = {The high arctic is undergoing a faster change in climate than most other regions of the planet, with already observed ecological consequences. Combined with the characteristics of high-arctic ecosystems, such as low species redundancy, high seasonality and weather extremes, shifts in individual species performance and phenology may lead to altered interaction dynamics through trophic mismatch and cascades. An ecosystem approach is therefore desirable in the attempt to understand the multidimensional impacts of climate. Here, we present ecosystem-wide trend analyses of a long-term dataset on terrestrial and limnic biota with focus on the distribution of observed trends and associated variation across the ecosystem. We used 114 time series drawn from 11 abiotic variables, 19 terrestrial and 7 limnic biotic species/taxa and compared temporal trends, changes and abrupt shifts in the variation within and across the two biota. A total of 36 {\%} of the time series analysed showed a significant trend during the study period with a higher frequency of trends occurring within performance variables. Overall, the changes tended to be negative, indicating advances in phenology but reduced species performance. General system variance was also higher in the limnic biota than in the terrestrial biota, both exhibiting increasing variance up through the trophic system. Overall, our results suggest that multiple biotic responses to the climatic changes in this high-arctic ecosystem are not synchronised across trophic levels and may differ qualitatively and quantitatively between terrestrial and limnic biota. {\textcopyright} 2014 Springer-Verlag Berlin Heidelberg.},
author = {Mortensen, Lars O. and Jeppesen, Erik and Schmidt, Niels Martin and Christoffersen, Kirsten S. and Tamstorf, Mikkel P. and Forchhammer, Mads C.},
doi = {10.1007/s00300-014-1501-2},
isbn = {0722-4060},
issn = {07224060},
journal = {Polar Biology},
keywords = {Arctic ecosystems,Greenland,Temporal trends,Temporal variance,Trophic interactions,Zackenberg},
number = {8},
pages = {1073--1082},
pmid = {26099054},
title = {{Temporal trends and variability in a high-arctic ecosystem in Greenland: multidimensional analyses of limnic and terrestrial ecosystems}},
volume = {37},
year = {2014}
}
@article{Iler2013,
abstract = {PREMISE OF THE STUDY: Plants are flowering earlier in response to climate change. However, substantial interannual variation in phenology may make it difficult to discern and compare long-term trends. In addition to providing insight on data requirements for discerning such trends, phenological shifts within subsets of long-term records will provide insight into the mechanisms driving changes in flowering over longer time scales.$\backslash$n$\backslash$nMETHODS: To examine variation in flowering shifts among temporal subsets of long-term records, we used two data sets of flowering phenology from snow-dominated habitats: subalpine meadow in Gothic, Colorado, USA (38 yr), and arctic tundra in Zackenberg, Greenland (16 yr). Shifts in flowering time were calculated as 10-yr moving averages for onset, peak, and end of flowering.$\backslash$n$\backslash$nKEY RESULTS: Flowering advanced over the course of the entire time series at both sites. Flowering shifts at Gothic were variable, with some 10-yr time frames showing significant delays and others significant advancements. Early-flowering species were more responsive than later-flowering species, while the opposite was true at Zackenberg. Flowering shifts at Zackenberg were less variable, with advanced flowering across all 10-yr time frames. At both sites, long-term advancement seemed to be primarily driven by strong advancements in flowering in the 1990s and early 2000s.$\backslash$n$\backslash$nCONCLUSIONS: Analysis of long-term trends can mask substantial variation in phenological shifts through time. This variation in the direction and magnitude of phenological shifts has implications for the evolution of flowering time and for interspecific interactions with flowering plants and can provide more detailed insights into the dynamics of phenological responses to climate change.},
author = {Iler, Amy M. and H{\o}ye, Toke T. and Inouye, David W. and Schmidt, Niels M.},
doi = {10.3732/ajb.1200490},
isbn = {0002-9122},
issn = {00029122},
journal = {American Journal of Botany},
keywords = {Arctic tundra,Climate change,Moving average,Phenology,Rocky Mountain Biological Laboratory,Snowmelt,Subalpine,Subsampling,Variation,Zackenberg},
number = {7},
pages = {1398--1406},
pmid = {23660568},
title = {{Long-term trends mask variation in the direction and magnitude of short-term phenological shifts}},
volume = {100},
year = {2013}
}
@article{Hastings1991,
abstract = {A continuous time model of a food chain incorporating nonlinear functional (and numerical) responses exhibits chaotic dynamics in long-term behaviour when biologically reasonable parameter values are chosen. The appearance of chaos in this model suggests that chaotic dynamics may be common in natural food webs.},
author = {Hastings, A. and Powell, T.},
doi = {10.2307/1940591},
isbn = {00129658},
issn = {00129658},
journal = {Ecology},
number = {3},
pages = {896--903},
pmid = {22465599},
title = {{Chaos in a three-species food chain}},
url = {http://www.jstor.org/stable/1940591},
volume = {72},
year = {1991}
}
@article{Lai2012,
abstract = {Identifying important species for maintaining ecosystem functions is a challenge in ecology. Since species are components of food webs, one way to conceptualize and quantify species importance is from a network perspective. The importance of a species can be quantified by measuring the centrality of its position in a food web, because a central node may have greater influence on others in the network. A species may also be important because it has a unique network position, such that its loss cannot be easily compensated. Therefore, for a food web to be robust, we hypothesize that central species must be functionally redundant in terms of their network position. In this paper, we test our hypothesis by analysing the Prince William Sound ecosystem. We found that species centrality and uniqueness are negatively correlated, and such an observation is also carried over to other food webs.},
author = {Lai, Shu Mei and Liu, Wei Chung and Jord{\'{a}}n, Ferenc},
doi = {10.1098/rsbl.2011.1167},
isbn = {1744-9561},
issn = {1744957X},
journal = {Biology Letters},
keywords = {Centrality,Food web,Species importance,Uniqueness},
number = {4},
pages = {570--573},
pmid = {22357938},
title = {{On the centrality and uniqueness of species from the network perspective}},
volume = {8},
year = {2012}
}
@article{Kondoh2008,
abstract = {Understanding what maintains species and perpetuates their coexistence in a network of feeding relationships (the food web) is of great importance for biodiversity conservation. A food web can be viewed as consisting of a number of simple subunits called trophic modules. Intraguild predation (IGP), in which a prey and its predator compete for the same resource, is one of the best-studied trophic modules. According to theory, there are two ways to yield a large persistent system from such modules: () to use persistent subunits as building blocks or () to arrange the subunits in a way that externally supports the nonpersistent subunits. Here, I show that the complex food web of the Caribbean marine ecosystem is constructed in both ways. I show that IGP modules, which convey internal persistence because of the fact that prey are superior competitors for the resources, are overrepresented in the Caribbean ecosystem. The other modules, consisting of competitively inferior prey, are not persistent in isolation. However, competitively inferior prey in these modules tend to receive more advantage from extra-module interactions, which allows persistence of the IGP module. In addition, those exterior interactions tend to be provided by intrinsically persistent IGP modules to prevent cascading extinction of interacting IGP modules. The food web can be viewed as a set of interacting modules, nonrandomly arranged to enhance the maintenance of biodiversity},
author = {Kondoh, M.},
doi = {10.1073/pnas.0805870105},
isbn = {0805870105},
issn = {0027-8424},
journal = {Proceedings of the National Academy of Sciences},
keywords = {BIODIVERSITY,Coexistence,ECOSYSTEM,EXTINCTION,FOOD-WEB,Isolation,NUMBER,PERSISTENCE,PREDATION,PREY,RESOURCES,conservation},
number = {43},
pages = {16631--16635},
pmid = {18936484},
title = {{Building trophic modules into a persistent food web}},
url = {http://www.pnas.org/cgi/doi/10.1073/pnas.0805870105},
volume = {105},
year = {2008}
}
@article{Holt1987,
abstract = {Interspecific interactions reflect the cumulative consequences of individual behavioral acts. The foraging decisions made by predators influence the way in which predation shapes the structure of prey communities. Alternative prey species co-occurring in a patch embedded in a matrix of many similar patches may interact through a shared mobile predator in two distinct ways. First, the functional response by an individual predator foraging in the patch to one prey species may be affected by the density of a second prey species in the patch (e.g., any time spent handling one prey reduces the time available for capturing other prey). Second, the presence of a second prey species may alter the propensity of predators to aggregate or remain in a given patch. We argue that this aggregative numerical response can in many circumstances generate -, - interactions (apparent competition) between prey species that otherwise would not interact. This is most likely if predators use a simple optimality criterion for prey selectivity within patches and the marginal-value theorem for deciding when to enter and leave patches. By contrast, if predators have suboptimal diets within patches but leave in accord with the marginal-value theorem, alternative prey may experience a +, - interaction; and, if predators use patches independently of prey availability, a +, + interaction between alternative prey can occur. Hence, the qualitative character of the interaction between alternative prey in a patchy environment depends on the degree to which predators do, or do not, match the canonical predictions of optimal foraging theory.},
author = {Holt, Robert D. and Kotler, Burt P.},
doi = {10.1086/284718},
isbn = {0003-0147},
issn = {0003-0147},
journal = {The American Naturalist},
number = {3},
pages = {412--430},
pmid = {1352},
title = {{Short-Term Apparent Competition}},
url = {http://www.journals.uchicago.edu/doi/10.1086/284718},
volume = {130},
year = {1987}
}
@article{Borrelli2015,
abstract = {Much of the focus in evolutionary biology has been on the adaptive differentiation among organisms. It is equally important to understand the processes that result in similarities of structure among systems. Here, we discuss examples of similarities occurring at different ecological scales, from predator-prey relations (attack rates and handling times) through communities (food-web structures) to ecosystem properties. Selection among systemic configurations or patterns that differ in their intrinsic stability should lead generally to increased representation of relatively stable structures. Such nonadaptive, but selective processes that shape ecological communities offer an enticing mechanism for generating widely observed similarities, and have sparked new interest in stability properties. This nonadaptive systemic selection operates not in opposition to, but in parallel with, adaptive evolution.},
author = {Borrelli, Jonathan J. and Allesina, Stefano and Amarasekare, Priyanga and Arditi, Roger and Chase, Ivan and Damuth, John and Holt, Robert D. and Logofet, Dmitrii O. and Novak, Mark and Rohr, Rudolf P. and Rossberg, Axel G. and Spencer, Matthew and Tran, J. Khai and Ginzburg, Lev R.},
doi = {10.1016/j.tree.2015.05.001},
isbn = {0169-5347},
issn = {01695347},
journal = {Trends in Ecology and Evolution},
keywords = {Feasibility,Macroecology,Selection,Stability},
number = {7},
pages = {417--425},
pmid = {26067808},
publisher = {Elsevier Ltd},
title = {{Selection on stability across ecological scales}},
url = {http://dx.doi.org/10.1016/j.tree.2015.05.001},
volume = {30},
year = {2015}
}
@article{Borgatti1993,
abstract = {In this paper we present two algorithms for computing the extent of regular equivalence among pairs of nodes in a network. The first algorithm, REGE, is well known, but has not previously been described in the literature. The second algorithm, CATREGE, is new. Whereas REGE is applicable to quantitative data, CATREGE is used for categorical data. For binary data, either algorithm may be used, though the CATREGE algorithm is significantly faster and its output similarity coefficients have better metric properties. The CATREGE algorithm is also useful pedagogically, because it is easier to grasp. {\textcopyright} 1993.},
author = {Borgatti, Stephen P. and Everett, Martin G.},
doi = {10.1016/0378-8733(93)90012-A},
isbn = {0378-8733},
issn = {03788733},
journal = {Social Networks},
number = {4},
pages = {361--376},
title = {{Two algorithms for computing regular equivalence}},
volume = {15},
year = {1993}
}
@article{Eklof2013,
abstract = {How many dimensions (trait-axes) are required to predict whether two species interact? This unanswered question originated with the idea of ecological niches, and yet bears relevance today for understanding what determines network structure. Here, we analyse a set of 200 ecological networks, including food webs, antagonistic and mutualistic networks, and find that the number of dimensions needed to completely explain all interactions is small ( {\textless} 10), with model selection favouring less than five. Using 18 high-quality webs including several species traits, we identify which traits contribute the most to explaining network structure. We show that accounting for a few traits dramatically improves our understanding of the structure of ecological networks. Matching traits for resources and consumers, for example, fruit size and bill gape, are the most successful combinations. These results link ecologically important species attributes to large-scale community structure.},
author = {Ekl{\"{o}}f, Anna and Jacob, Ute and Kopp, Jason and Bosch, Jordi and Castro-Urgal, Roc{\'{i}}o and Chacoff, Natacha P. and Dalsgaard, Bo and de Sassi, Claudio and Galetti, Mauro and Guimar{\~{a}}es, Paulo R. and Lom{\'{a}}scolo, Silvia Beatriz and {Mart{\'{i}}n Gonz{\'{a}}lez}, Ana M. and Pizo, Marco Aurelio and Rader, Romina and Rodrigo, Anselm and Tylianakis, Jason M. and V{\'{a}}zquez, Diego P. and Allesina, Stefano},
doi = {10.1111/ele.12081},
isbn = {1461-0248},
issn = {1461023X},
journal = {Ecology Letters},
keywords = {Ecological networks,Food web structure,Intervality,Niche space,Scaling,Species traits},
number = {5},
pages = {577--583},
pmid = {23438174},
title = {{The dimensionality of ecological networks}},
volume = {16},
year = {2013}
}
@article{Johnson2001,
author = {Johnson, Jeffrey C and Borgatti, Stephen P and Luczkovich, Joseph J and Everett, Martin G},
journal = {Journal of Social Structure},
number = {3},
pages = {1--32},
title = {{Network Role Analysis in the Study of Food Webs : An Application of Regular Role}},
volume = {2},
year = {2014}
}
@article{Jordan2006,
abstract = {The local extinction or large fluctuation in abundance of a species may seriously affect other species in the community. The effects spread through the community by direct and indirect interactions. The network perspective on ecology can help map the pathways of these effects, for food webs, the pathways of indirect trophic interactions. Indirect interactions typically decay in intensity as they spread. Therefore, there is a conceptual maximum range in topological space beyond which interactions have no effects, even though all species remain connected. Neither the local characteristics of species, nor the global characteristics of entire webs, suitably quantify this range. We therefore apply intermediate scale indices that reflect the limitations imposed by effect damping in networks. We present a complex analysis of the topological positional importance of species in the Chesapeake Bay web. This web is a carbon-flow network that represents trophic interactions. We present several different indices reflecting different properties and discuss which questions the different indices best answer. We look for the best indices for identifying the key players in ecosystem functioning. Our study contributes to the quantification of relative species importance and provides an exact and a priori determination of a class of candidate keystone species that can inform applied and conservation ecology as well as theoretical concerns.},
author = {Jord{\'{a}}n, F and Liu, Wei-chung and Davis, Andrew J},
doi = {10.1111/j.0030-1299.2006.13724.x},
isbn = {0030-1299},
issn = {00301299},
journal = {Oikos},
keywords = {conservation,ecosystem functioning,extinction,food web,indirect interactions,interactions,keystone species,network,trophic interactions},
number = {July 2005},
pages = {535--546},
pmid = {73},
title = {{Topological keystone species: measures of positional importance in food webs}},
volume = {112},
year = {2006}
}
@article{Jordan2002,
author = {Jordan, Ferenc and Pdzmany, P C},
journal = {Oikos},
number = {3},
pages = {607--612},
title = {{FORUM FOR in ecological networks Searching for keystones}},
volume = {99},
year = {2012}
}
@article{Bonacich1972,
abstract = {Counts, mark-recapture estimates of abundance, and simulations were used to assess the population trends of Antipodean wandering albatross (Diomedea antipodensis) and Gibson's wandering albatross (D. gibsoni). Estimates of population size based on mark-recapture analysis had much greater power to detect trends than did annual counts of nests. In fact, nest counts were so variable that signi.cant trends would only be detected when populations had already changed by more than 25{\%}. Population simulation models were constructed using survival and productivity data from the two species, and recruitment data from closely related species. The simulation models were sensitive to variation in recruitment data and suggested that the recruitment of Gibson's wandering albatrosses is signi.cantly lower than that of Antipodean wandering albatrosses. The sensitivity of the models to variation in the surrogate data compromises the usefulness of such models as predictive tools. After large, probably .sheries-induced declines during the 1970s and 1980s, Antipodean wandering albatross populations are now increasing at about 3.1{\%} per annum, while Gibson's wandering albatross populations are static.},
author = {Elliott, Graeme and Walker, Kath},
doi = {10.1080/0022250X.1972.9989806},
isbn = {0022-250X},
issn = {00294470},
journal = {Notornis},
keywords = {Diomedea sp.,Mark-recapture,Population counts,Simulation model},
number = {4},
pages = {215--222},
pmid = {7244},
title = {{Detecting population trends of Gibson's and Antipodean wandering albatrosses}},
volume = {52},
year = {2005}
}
@article{BanasekRichter2009,
abstract = {Food webs depict who eats whom in communities. Ecologists have examined statistical metrics and other properties of food webs, but mainly due to the uneven quality of the data, the results have proved controversial. The qualitative data on which those efforts rested treat trophic interactions as present or absent and disregard potentially huge variation in their magnitude, an approach similar to analyzing traffic without differentiating between highways and side roads. More appropriate data are now available and were used here to analyze the relationship between trophic complexity and diversity in 59 quantitative food webs from seven studies (14-202 species) based on recently developed quantitative descriptors. Our results shed new light on food-web structure. First, webs are much simpler when considered quantitatively, and link density exhibits scale invariance or weak dependence on food-web size. Second, the "constant connectance" hypothesis is not supported: connectance decreases with web size in both qualitative and quantitative data. Complexity has occupied a central role in the discussion of food-web stability, and we explore the implications for this debate. Our findings indicate that larger webs are more richly endowed with the weak trophic interactions that recent theories show to be responsible for food-web stability.},
author = {Bana{\v{s}}ek-Richter, Carolin and Bersier, Louis F{\'{e}}lix and Cattin, Marie France and Baltensperger, Richard and Gabriel, Jean Pierre and Merz, Yves and Ulanowicz, Robert E. and Tavares, Annette F. and Williams, D. Dudley and {De Ruiter}, Peter C. and Winemiller, Kirk O. and Naisbit, Russell E.},
doi = {10.1890/08-2207.1},
isbn = {0012-9658},
issn = {00129658},
journal = {Ecology},
keywords = {Connectance,Diversity,Ecological network,Food-web stability,Scaling,Species diversity,Stability,Trophic complexity},
number = {6},
pages = {1470--1477},
pmid = {19569361},
title = {{Complexity in quantitative food webs}},
volume = {90},
year = {2009}
}
@article{Sanders2013,
abstract = {Species extinctions are biased towards higher trophic levels, and primary extinctions are often followed by unexpected secondary extinctions. Currently, predictions on the vulnerability of ecological communities to extinction cascades are based on models that focus on bottom-up effects, which cannot capture the effects of extinctions at higher trophic levels. We show, in experimental insect communities, that harvesting of single carnivorous parasitoid species led to a significant increase in extinction rate of other parasitoid species, separated by four trophic links. Harvesting resulted in the release of prey from top-down control, leading to increased interspecific competition at the herbivore trophic level. This resulted in increased extinction rates of non-harvested parasitoid species when their host had become rare relative to other herbivores. The results demonstrate a mechanism for horizontal extinction cascades, and illustrate that altering the relationship between a predator and its prey can cause wide-ranging ripple effects through ecosystems, including unexpected extinctions.},
author = {Sanders, Dirk and Sutter, Louis and van Veen, F. J.Frank},
doi = {10.1111/ele.12096},
isbn = {1461-023X},
issn = {1461023X},
journal = {Ecology Letters},
keywords = {Aphids,Community stability,Indirect effects,Non-trophic interactions,Parasitoids,Resource competition,Secondary extinctions,Species loss},
number = {5},
pages = {664--669},
pmid = {23445500},
title = {{The loss of indirect interactions leads to cascading extinctions of carnivores}},
volume = {16},
year = {2013}
}
@article{Wootton2016,
abstract = {Ecological communities consist of generalists who interact with proportionally many species, and specialists who interact with proportionally few. The strength of these interactions also varies, with communities typically exhibiting a few strong links embedded within many weak links. Historically, it has been argued that generalists should interact more weakly with their partners than specialists and, since weak interactions are thought to increase community stability, that this pattern increases the stability of diverse communities. Here, we studied model-generated predator-prey communities to explicitly investigate the validity of this argument. In feasible communities—those which were both locally stable and all species had positive biomass—we indeed found that species with many predators or prey are affected by them more weakly than species with few. This relationship, however, is only part of the story. While species with many predators (or prey) tend to be only weakly affected by each of them, these many weak interactions are balanced by a few strong interactions with prey (or predators). These few strong interactions are large enough that, when the effect of predator and prey interactions are combined, it seems that species with many interactions actually interact more strongly than species with few interactions. Though past research has tended to focus on either the arrangement of species interactions or the strength of those interactions, we show here that the two are in fact inextricably linked. This observation has implications for both the realistic design of theoretical models, and the conservation of ecological communities, especially those in which the strength and arrangement of species' interactions are impacted by biodiversity-loss disturbances such as habitat alteration.},
author = {Wootton, K. L. and Stouffer, D. B.},
doi = {10.1007/s12080-015-0279-3},
isbn = {1208001502793},
issn = {18741746},
journal = {Theoretical Ecology},
keywords = {Community matrix,Food-web structure,Interaction strength,Predator-prey,Stability},
number = {2},
pages = {185--195},
publisher = {Theoretical Ecology},
title = {{Many weak interactions and few strong; food-web feasibility depends on the combination of the strength of species' interactions and their correct arrangement}},
url = {http://dx.doi.org/10.1007/s12080-015-0279-3},
volume = {9},
year = {2016}
}
@article{Gray2002,
abstract = {Biomagnification is the process where xenobiotic substances are transferred from food to an organism resulting in higher concentrations compared with the source. It is widely believed that this is a general phenomenon for marine food webs. An analysis of 148 papers with biomagnification in the title shows that under half show biomagnification. Of studies on metals only organic mercury shows biomagnification and most metals are regulated and excreted and do not biomagnify. Of the studies on organic compounds 67{\%} claimed to show biomagnification. However, bioconcentration (uptake from the surrounding water) is the most usual way that organic compounds are accumulated in organisms from invertebrates to and including fish. Only in sea-birds and marine mammals is food intake the major route and where biomagnification can be clearly shown. Body concentrations of organic compounds vary with lipid content and thus in order to compare across species normalisation to uniform lipid content should be done. Yet often this is not done so data purporting to show biomagnification merely relate to differing lipid content in the different species studied. Finally suggestions are made as to how data can be collected to better interpret the process of biomagnification in marine food webs. {\textcopyright} 2002 Published by Elsevier Science Ltd.},
author = {Gray, John S.},
doi = {10.1016/S0025-326X(01)00323-X},
isbn = {0025-326X},
issn = {0025326X},
journal = {Marine Pollution Bulletin},
keywords = {Bioconcentration,Biomagnification,Food webs,Metals,Organic chemicals},
number = {1-12},
pages = {46--52},
pmid = {12398366},
title = {{Biomagnification in marine systems: The perspective of an ecologist}},
volume = {45},
year = {2002}
}
@article{Coelho2013,
abstract = {The main aim of this study was to ascertain the biomagnification processes in a mercury-contaminated estuary, by clarifying the trophic web structure through stable isotope ratios. For this purpose, primary producers (seagrasses and macroalgae), invertebrates (detritivores and benthic predators) and fish were analysed for total and organic mercury and for stable carbon and nitrogen isotopic signatures. Trophic structure was accurately described by $\delta$15N, while $\delta$13C reflected the carbon source for each species. An increase of mercury levels was observed with trophic level, particularly for organic mercury. Results confirm mercury biomagnification to occur in this estuarine food web, especially in the organic form, both in absolute concentrations and fraction of total mercury load. Age can be considered an important variable in mercury biomagnification studies, and data adjustments to account for the different exposure periods may be necessary for a correct assessment of trophic magnification rates and ecological risk. {\textcopyright} 2013 Elsevier Ltd.},
author = {Coelho, J. P. and Mieiro, C. L. and Pereira, E. and Duarte, A. C. and Pardal, M. A.},
doi = {10.1016/j.marpolbul.2013.01.021},
isbn = {0025-326X},
issn = {0025326X},
journal = {Marine Pollution Bulletin},
keywords = {Biomagnification,Mercury,Organic mercury,Stable isotopes,Trophic level,Trophic magnification factor (TMF)},
number = {1-2},
pages = {110--115},
pmid = {23433553},
publisher = {Elsevier Ltd},
title = {{Mercury biomagnification in a contaminated estuary food web: Effects of age and trophic position using stable isotope analyses}},
url = {http://dx.doi.org/10.1016/j.marpolbul.2013.01.021},
volume = {69},
year = {2013}
}
@article{Tavares2009,
abstract = {Mercury exposure may be linked to several sources of variation related to habitat conditions and species ecology. In generalist birds, feeding habits may change quickly in response to environmental conditions, prey availability and individual requirements. Stable nitrogen isotope ratios (15N) were used as a marker of trophic level, and stable carbon isotope ratios ( 13C) as a marker of carbon sources (terrestrial vs. marine) in food webs involving waterbirds, to infer the effect on mercury exposure due to differences in feeding ecology and the relative dependence on aquatic environments. Four generalist species occupying three different habitats were examined during the breeding season. Habitats: Brackish water - saltwater (saltpans), brackish water - freshwater (ricefields and some saltpans) and terrestrial environment (steppes). Species: Avocet (Recurvirostra avosetta), Black-winged Stilt (Himantopus himantopus), Kentish Plover (Charadrius alexandrinus) and Cattle Egret (Bulbucus ibis). Chick feathers were collected at several locations between 2000 and 2003. Species used resources differently in the environment, and distinct pathways were involved in the mobilization of mercury into food webs. The positive relationship between feather 15N values and mercury levels indicated mercury bioaccumulation and biomagnification. Inter-specific variation in feather 13C values revealed a different relative dependence among species on terrestrial vs. aquatic prey. Intra-specific variation in feather 13C values also indicated differential use of marine inputs within each species, and within saltpans for Avocet chicks. Feather mercury levels and 13C values suggested that the relative use of marine-derived prey influences mercury levels in chicks.},
author = {Tavares, PC and McGill, R and Newton, J and Pereira, E},
doi = {10.1675/063.032.0211},
isbn = {1524-4695},
issn = {1524-4695},
journal = {Waterbirds},
keywords = {2,2009,311-321,and terrestrial habitats,birds,carbon,chicks,generalist birds use a,ing sites in aquatic,mercury,nitrogen,stable isotopes,trophic shift,waterbirds 32,wide range of feed-},
number = {2},
pages = {311--321},
title = {{Relationships Between Carbon Sources, Trophic Level and Mercury Exposure in Generalist Shorebirds Revealed by stable isotope ratios in chicks}},
url = {http://www.bioone.org/doi/pdf/10.1675/063.032.0211},
volume = {32},
year = {2009}
}
@article{Rowan1992,
author = {Rowan, David J. and Rasmussen, Joseph B.},
journal = {Journal of Great Lakes Research},
number = {4},
pages = {724--741},
title = {{RowanRasmussen1992.pdf}},
volume = {18},
year = {1992}
}
@article{Dyer2003,
abstract = {Apex predators and plant resources are both critical for maintaining diversity in biotic communities, but the indirect ('cascading') effects of top-down and bottom-up forces on diversity at different trophic levels are not well resolved in terrestrial systems. Manipulations of predators or resources can cause direct changes of diversity at one trophic level, which in turn can affect diversity at other trophic levels. The indirect diversity effects of resource and consumer variation should be strongest in aquatic systems, moderate in terrestrial systems, and weakest in decomposer food webs. We measured effects of top predators and plant resources on the diversity of endophytic animals in an understorey shrub Piper cenocladum (Piperaceae). Predators and resource availability had significant direct and indirect effects on the diversity of the endophytic animal community, but the effects were not interactive, nor were they consistent between living vs. detrital food webs. The addition of fourth trophic level beetle predators increased diversity of consumers supported by living plant tissue, whereas balanced plant resources (light and nutrients) increased the diversity of primary through tertiary consumers in the detrital resources food web. These results support the hypotheses that top-down and bottom-up diversity cascades occur in terrestrial systems, and that diversity is affected by different factors in living vs. detrital food webs.},
author = {Dyer, Lee A. and Letourneau, Deborah},
doi = {10.1046/j.1461-0248.2003.00398.x},
isbn = {1461-023X},
issn = {1461023X},
journal = {Ecology Letters},
keywords = {Diversity,Food webs,Indirect effects,Piper,Trophic cascades,Tropics},
number = {1},
pages = {60--68},
pmid = {25452575},
title = {{Top-down and bottom-up diversity cascades in detrital vs. living food webs}},
volume = {6},
year = {2003}
}
@article{Hulot2014,
abstract = {We performed a meta-analysis of 31 lake mesocosm experiments to investigate diff erences in the responses of pelagic food chains and food webs to nutrient enrichment and fi sh presence. Trophic levels were divided into size-based functional groups (phytoplankton into highly edible and poorly edible algae, and zooplankton into small herbivores, large herbivores and omnivorous zooplankton) in the food webs. Our meta-analysis shows that 1) nutrient enrichment has a positive eff ect on phytoplankton and zooplankton, while fi sh presence has a positive eff ect on phytoplankton and a negative eff ect on zooplankton in the food chains; 2) nutrient enrichment has a positive eff ect on highly edible algae and small herbivores, but no eff ect on poorly edible algae, large herbivores and omnivorous zooplankton in the food webs. Planktivorous fi sh have a positive eff ect on highly edible algae and small herbivores, a negative eff ect on large herbivores and omnivorous zooplankton, and no eff ect on poorly edible algae. Our meta-analysis confi rms that nutrient enrichment and planktivorous fi sh aff ect functional groups diff erentially within trophic levels, revealing important changes in the functioning of food webs. Th e analysis of fi sh eff ects shows the well-described trophic cascade in the food chain and reveals two trophic cascades in the food web: one transmitted by large herbivores that benefi t highly edible phytoplankton, and one transmitted by omnivorous zooplankton that benefi t small herbivores. Comparison between the responses of food webs and simple food chains also shows consistent biomass compensation between functional groups within trophic levels. After},
archivePrefix = {arXiv},
arxivId = {3025},
author = {Hulot, Florence D. and Lacroix, G{\'{e}}rard and Loreau, Michel},
doi = {10.1111/oik.01116},
eprint = {3025},
isbn = {1600-0706},
issn = {16000706},
journal = {Oikos},
number = {11},
pages = {1291--1300},
title = {{Differential responses of size-based functional groups to bottom-up and top-down perturbations in pelagic food webs: A meta-analysis}},
volume = {123},
year = {2014}
}
@article{Dalton2013,
abstract = {Benthic invertebrates mediate bottom--up and top--down influences in aquatic food webs, and changes in the abundance or traits of invertebrates can alter the strength of top--down effects. Studies assessing the role of invertebrate abundance and behavior as controls on food web structure are rare at the whole ecosystem scale. Here we use a comparative approach to investigate bottom--up and top--down influences on whole anchialine pond ecosystems in coastal Hawai`i. In these ponds, a single species of endemic atyid shrimp (Halocaridina rubra) is believed to structure epilithon communities. Many Hawaiian anchialine ponds and their endemic fauna, however, have been greatly altered by bottom--up (increased nutrient enrichment) and top--down (introduced fish predators) disturbances from human development. We present the results of a survey of dissolved nutrient concentrations, epilithon biomass and composition, and H. rubra abundance and behavior in anchialine ponds with and without invasive predatory fish along a nutrient concentration gradient on the North Kona coast of Hawai`i. We use linear models to assess 1) the effects of nutrient loading and fish introductions on pond food web structure and 2) the role of shrimp density and behavior in effecting that change. We find evidence for bottom--up food web control, in that nutrients were associated with increased epilithon biomass, autotrophy and nutrient content as well as increased abundance and size of H. rubra. We also find evidence for top--down control, as ponds with invasive predatory fish had higher epilithon biomass, productivity, and nutrient content. Top--down effects were transmitted by both altered H. rubra abundance, which changed the biomass of epilithon, and H. rubra behavior, which changed the composition of the epilithon. Our study extends experimental findings on bottom--up and top--down control to the whole ecosystem scale and finds evidence for qualitatively different effects of trait- and density-mediated change in top--down influences.},
author = {Dalton, C. M. and Mokiao-Lee, A. and Sakihara, T. S. and Weber, M. G. and Roco, C. A. and Han, Z. and Dudley, B. and Mackenzie, R. A. and Hairston, N. G.},
doi = {10.1111/j.1600-0706.2012.20696.x},
isbn = {1600-0706},
issn = {00301299},
journal = {Oikos},
number = {5},
pages = {790--800},
title = {{Density- and trait-mediated top-down effects modify bottom-up control of a highly endemic tropical aquatic food web}},
volume = {122},
year = {2013}
}
@article{Wang2005,
abstract = {Uptake, absorption efficiency and elimination of DDT were measured in marine phytoplankton, copepods (Acartia erythraea) and fish (mangrove snappers Lutjanus argentimaculatus). The uptake rate constant of DDT from water decreased with increasing trophic level. The dietary absorption efficiency (AE) of DDT was 10-29{\%} in copepods and 72-99{\%} in fish. Food concentration did not significantly affect the AEs of DDT, but the AEs varied considerably among the different food diets. The elimination rate constants of DDT by the copepods were comparable following uptake from the diet and from the water. Elimination of DDT from the fish was exceedingly low. Both aqueous and dietary uptake are equally important for DDT accumulation in the copepods. In fish, dissolved exposure is a more significant route than intake from the diet. The predicted trophic transfer factors in the copepods and the fish are consistent with the field measurements in marine zooplankton and fish. {\textcopyright} 2005 Elsevier Ltd. All rights reserved.},
author = {Wang, Xinhong and Wang, Wen Xiong},
doi = {10.1016/j.envpol.2005.01.004},
isbn = {0269-7491},
issn = {02697491},
journal = {Environmental Pollution},
keywords = {Biomagnification,Copepods,DDT,Fish,Food chain transfer,Phytoplankton},
number = {3},
pages = {453--464},
pmid = {15862399},
title = {{Uptake, absorption efficiency and elimination of DDT in marine phytoplankton, copepods and fish}},
volume = {136},
year = {2005}
}
@article{Estes2015,
abstract = {The late Pleistocene extinction of so many large-bodied vertebrates has been variously attributed to two general causes: rapid climate change and the effects of humans as they spread from the Old World to previously uninhabited continents and islands. Many large-bodied vertebrates, especially large apex predators, maintain their associated ecosystems through top-down forcing processes, especially trophic cascades, and megaherbivores also exert an array of strong indirect effects on their communities. Thus, a third possibility for at least some of the Pleistocene extinctions is that they occurred through habitat changes resulting from the loss of these other keystone species. Here we explore the plausibility of this mechanism, using information on sea otters, kelp forests, and the recent extinction of Steller's sea cows from the Commander Islands. Large numbers of sea cows occurred in the Commander Islands at the time of their discovery by Europeans in 1741. Although extinction of these last remaining sea cows during early years of the Pacific maritime fur trade is widely thought to be a consequence of direct human overkill, we show that it is also a probable consequence of the loss of sea otters and the co-occurring loss of kelp, even if not a single sea cow had been killed directly by humans. This example supports the hypothesis that the directly caused extinctions of a few large vertebrates in the late Pleistocene may have resulted in the coextinction of numerous other species.},
author = {Estes, James A. and Burdin, Alexander and Doak, Daniel F.},
doi = {10.1073/pnas.1502552112},
isbn = {0027-8424},
issn = {0027-8424},
journal = {Proceedings of the National Academy of Sciences},
keywords = {Commander Islands,Steller's sea cow,extinction,kelp,sea otter},
number = {4},
pages = {880--885},
pmid = {26504217},
title = {{Sea otters, kelp forests, and the extinction of Steller's sea cow}},
url = {http://www.pnas.org/lookup/doi/10.1073/pnas.1502552112},
volume = {113},
year = {2016}
}
@article{Eklof2006,
abstract = {1. The loss of a species from an ecological community can trigger a cascade of secondary extinctions. Here we investigate how the complexity (connectance) of model communities affects their response to species loss. Using dynamic analysis based on a global criterion of persistence (permanence) and topological analysis we investigate the extent of secondary extinctions following the loss of different kinds of species. 2. We show that complex communities are, on average, more resistant to species loss than simple communities: the number of secondary extinctions decreases with increasing connectance. However, complex communities are more vulnerable to loss of top predators than simple communities. 3. The loss of highly connected species (species with many links to other species) and species at low trophic levels triggers, on average, the largest number of secondary extinctions. The effect of the connectivity of a species is strongest in webs with low connectance. 4. Most secondary extinctions are due to direct bottom-up effects: consumers go extinct when their resources are lost. Secondary extinctions due to trophic cascades and disruption of predator-mediated coexistence also occur. Secondary extinctions due to disruption of predator-mediated coexistence are more common in complex communities than in simple communities, while bottom-up and top-down extinction cascades are more common in simple communities. 5. Topological analysis of the response of communities to species loss always predicts a lower number of secondary extinctions than dynamic analysis, especially in food webs with high connectance.},
author = {Ekl{\"{o}}f, Anna and Ebenman, Bo},
doi = {10.1111/j.1365-2656.2006.01041.x},
isbn = {1365-2656},
issn = {00218790},
journal = {Journal of Animal Ecology},
keywords = {Cascading extinction,Connectance,Food web,Keystone species,Resistance},
number = {1},
pages = {239--246},
pmid = {16903061},
title = {{Species loss and secondary extinctions in simple and complex model communities}},
volume = {75},
year = {2006}
}
@article{Boersma2014,
abstract = {Top predator losses affect a wide array of ecological processes, and there is growing evidence that top predators are disproportionately vulnerable to environmental changes. Despite increasing recognition of the fundamental role that top predators play in structuring communities and ecosystems, it remains challenging to predict the consequences of predator extinctions in highly variable environments. Both biotic and abiotic drivers determine community structure, and manipulative experiments are necessary to disentangle the effects of predator loss from other co-occurring environmental changes. To explore the consistency of top predator effects in ecological communities that experience high local environmental variability, we experimentally removed top predators from arid-land stream pool mesocosms in southeastern Arizona, USA, and measured natural background environmental conditions. We inoculated mesocosms with aquatic invertebrates from local streams, removed the top predator Abedus herberti (Hemiptera: Belostomatidae) from half of the mesocosms as a treatment, and measured community divergence at the end of the summer dry season. We repeated the experiment in two consecutive years, which represented two very different biotic and abiotic environments. We found that some of the effects of top predator removal were consistent despite significant differences in environmental conditions, community composition, and colonist sources between years. As in other studies, top predator removal did not affect overall species richness or abundance in either year, and we observed inconsistent effects on community and trophic structure. However, top predator removal consistently affected large-bodied species (those in the top 1{\%} of the community body size distribution) in both years, increasing the abundance of mesopredators and decreasing the abundance of detritivores, even though the identity of these species varied between years. Our findings highlight the vulnerability of large taxa to top predator extirpations and suggest that the consistency of observed ecological patterns may be as important as their magnitude.},
author = {Boersma, Kate S. and Bogan, Michael T. and Henrichs, Brian A. and Lytle, David A.},
doi = {10.1111/oik.00925},
isbn = {1600-0706},
issn = {16000706},
journal = {Oikos},
number = {7},
pages = {807--816},
title = {{Top predator removals have consistent effects on large species despite high environmental variability}},
volume = {123},
year = {2014}
}
@article{Borrvall2006,
abstract = {The large vulnerability of top predators to human-induced disturbances on ecosystems is a matter of growing concern. Because top predators often exert strong influence on their prey populations their extinction can have far-reaching consequences for the structure and functioning of ecosystems. It has, for example, been observed that the local loss of a predator can trigger a cascade of secondary extinctions. However, the time lags involved in such secondary extinctions remain unexplored. Here we show that the loss of a top predator leads to a significantly earlier onset of secondary extinctions in model communities than does the loss of a species from other trophic levels. Moreover, in most cases time to secondary extinction increases with increasing species richness. If local secondary extinctions occur early they are less likely to be balanced by immigration of species from local communities nearby. The implications of these results for community persistence and conservation priorities are discussed.},
author = {Borrvall, Charlotte and Ebenman, Bo},
doi = {10.1111/j.1461-0248.2006.00893.x},
isbn = {1461-0248 (Electronic)$\backslash$n1461-023X (Linking)},
issn = {1461023X},
journal = {Ecology Letters},
keywords = {Ecological community,Food web,Relaxation time,Secondary extinction,Species interactions,Species loss,Species richness,Top predators,Trophic level},
number = {4},
pages = {435--442},
pmid = {16623729},
title = {{Early onset of secondary extinctions in ecological communities following the loss of top predators}},
volume = {9},
year = {2006}
}
@article{RodriguezLozano2015,
abstract = {Top predator loss is a major global problem, with a current trend in biodiversity loss towards high trophic levels that modifies most ecosystems worldwide. Most research in this area is focused on large-bodied predators, despite the high extinction risk of small-bodied freshwater fish that often act as apex consumers. Consequently, it remains unknown if intermittent streams are affected by the consequences of top-predators' extirpations. The aim of our research was to determine how this global problem affects intermittent streams and, in particular, if the loss of a small-bodied top predator (1) leads to a 'mesopredator release', affects primary consumers and changes whole community structures, and (2) triggers a cascade effect modifying the ecosystem function. To address these questions, we studied the top-down effects of a small endangered fish species, Barbus meridionalis (the Mediterranean barbel), conducting an enclosure/exclosure mesocosm experiment in an intermittent stream where B. meridionalis became locally extinct following a wildfire. We found that top predator absence led to 'mesopredator release', and also to 'prey release' despite intraguild predation, which contrasts with traditional food web theory. In addition, B. meridionalis extirpation changed whole macroinvertebrate community composition and increased total macroinvertebrate density. Regarding ecosystem function, periphyton primary production decreased in apex consumer absence. In this study, the apex consumer was functionally irreplaceable; its local extinction led to the loss of an important functional role that resulted in major changes to the ecosystem's structure and function. This study evidences that intermittent streams can be affected by the consequences of apex consumers' extinctions, and that the loss of small-bodied top predators can lead to large ecosystem changes. We recommend the reintroduction of small-bodied apex consumers to systems where they have been extirpated, to restore ecosystem structure and function.},
author = {Rodr{\'{i}}guez-Lozano, Pablo and Verkaik, Iraima and Rieradevall, Maria and Prat, Narc{\'{i}}s},
doi = {10.1371/journal.pone.0117630},
isbn = {1932-6203},
issn = {19326203},
journal = {PLoS ONE},
number = {2},
pages = {1--16},
pmid = {25714337},
title = {{Small but powerful: Top predator local extinction affects ecosystem structure and function in an intermittent stream}},
volume = {10},
year = {2015}
}
@article{Zabalo2012,
abstract = {Intraguild predation, a form of omnivory that can occur in simple food webs when one species preys on and competes for limiting resources with another species, can have either a stabilizing effect (McCann and Hastings in Proc. R. Soc. Lond. B 264:1249-1254, 1997) or a destabilizing effect (Holt and Polis in Am. Nat. 149:745-764, 1997), depending on the assumptions of the system. Another type of behavior that has been observed in simple food web experiments (Murdoch in Ecol. Monogr. 39:335-354, 1969) is prey switching. Prey switching can occur when the predator prefers the most abundant prey. This has also been shown to be capable of having either a stabilizing effect or a destabilizing effect and even possibly lead to predator extinction (VanLeeuwen et al. in Ecology 88:1571-1581, 2007). Therefore, it is clear that incorporating prey switching into an intraguild predation model could lead to unexpected consequences. In this paper, we propose and explore such a model.},
author = {Zabalo, Joaquin},
doi = {10.1007/s11538-012-9740-2},
isbn = {0092-8240},
issn = {00928240},
journal = {Bulletin of Mathematical Biology},
keywords = {Intraguild predation,Omnivory,Permanence,Predator-prey,Prey switching},
number = {9},
pages = {1957--1984},
pmid = {22766924},
title = {{Permanence in an Intraguild Predation Model with Prey Switching}},
volume = {74},
year = {2012}
}
@article{Hammill2015,
abstract = {The strength of interspecific interactions is often proposed to affect food web stability, with weaker interactions increasing the persistence of species, and food webs as a whole. However, the mechanisms that modify interaction strengths, and their effects on food web persistence are not fully understood. Using food webs containing different combinations of predator, prey, and nonprey species, we investigated how predation risk of susceptible prey is affected by the presence of species not directly trophically linked to either predators or prey. We predicted that indirect alterations to the strength of trophic interactions translate to changes in persistence time of extinction-prone species. We assembled interaction webs of protist consumers and turbellarian predators with eight different combinations of prey, predators and nonprey species, and recorded abundances for over 130 prey generations. Persistence of predation-susceptible species was increased by the presence of nonprey. Furthermore, multiple nonprey species acted synergistically to increase prey persistence, such that persistence was greater than would be predicted from the dynamics of simpler food webs. We also found evidence suggesting increased food web complexity may weaken interspecific competition, increasing persistence of poorer competitors. Our results demonstrate that persistence times in complex food webs cannot be predicted from the dynamics of simplified systems, and that species not directly involved in consumptive interactions likely play key roles in maintaining persistence. Global species diversity is currently declining at an unprecedented rate and our findings reveal that concurrent loss of species that modify trophic interactions may have unpredictable consequences for food web stability.},
author = {Hammill, Edd and Kratina, Pavel and Vos, Matthijs and Petchey, Owen L. and Anholt, Bradley R.},
doi = {10.1007/s00442-015-3244-3},
isbn = {0044201532},
issn = {00298549},
journal = {Oecologia},
keywords = {Community persistence,Interaction modifications,Microcosms,Nonprey species,Predation,Trophic interactions},
number = {2},
pages = {549--556},
pmid = {25656586},
publisher = {Springer Berlin Heidelberg},
title = {{Food web persistence is enhanced by non-trophic interactions}},
url = {http://dx.doi.org/10.1007/s00442-015-3244-3},
volume = {178},
year = {2015}
}
@article{Hairston1993,
abstract = {Measurements of the efficiency of energy transfer between trophic levels are consistent with the hypothesis that it is trophic structure that controls the fraction of energy consumed at each trophic level, rather than energetics controlling trophic structure. Moreover, trophic structure is determined by competitive and predator-prey interactions. In freshwater pelagic communities, the collective efficiency of herbivorous plankton in consuming primary producers is up to 10 times as great as is the efficiency of forest herbivores in consuming their food. Conversely, forest predators are about three times as efficient in consuming herbivore production as are zooplanktivorous fish. The presence of an additional level, piscivorous fish, in pelagic communities accounts for the difference. In the aquatic system, herbivorous zooplankton are freed from predation by the effect of piscivorous fish on their predators; in the terrestrial system, green plants are freed from herbivory by predation on the herbivores. We explain the contrast between freshwater pelagic systems and forests and prairies as follows: Pelagic ecosystems have more trophic levels as a result of selection for small rapidly growing primary producers, which cannot hold space in the fluid medium, in contrast to large space-occupying producers in the terrestrial environment. Consumers in pelagic systems are more frequently gape limited in the size range of food they can ingest than are grasping consumers in terrestrial systems. This difference makes for two largely distinct levels of predators in pelagic communities. The energy within the living, nondetrital components is more finely divided between trophic levels in pelagic systems than in terrestrial systems. Ecological efficiencies do not determine trophic structure; rather, they are its product.},
author = {Hairston,, Nelson G. and Hairston,, Nelson G.},
doi = {10.1086/285546},
isbn = {00030147},
issn = {0003-0147},
journal = {The American Naturalist},
number = {3},
pages = {379--411},
pmid = {3965},
title = {{Cause-Effect Relationships in Energy Flow, Trophic Structure, and Interspecific Interactions}},
url = {http://www.journals.uchicago.edu/doi/10.1086/285546},
volume = {142},
year = {1993}
}
@article{Vazquez2005,
abstract = {We evaluate whether species interaction frequency can be used as a surrogate for the total effect of a species on another. Because interaction frequency is easier to estimate than per-interaction effect, using interaction frequency as a surrogate of total effect could facilitate the large-scale analysis of quantitative patterns of species-rich interaction networks. We show mathematically that the correlation between interaction frequency (I) and total effect (T) becomes more strongly positive the greater the variation of I relative to the variation of per-interaction effect (P) and the greater the correlation between I and P. A meta-analysis using data on I, P and T for animal pollinators and seed dispersers visiting plants shows a generally strong, positive relationship between T and I, in spite of no general relationship between P and I. Thus, frequent animal mutualists usually contribute the most to plant reproduction, regardless of their effectiveness on a per-interaction basis.},
author = {V{\'{a}}zquez, Diego P. and Morris, William F. and Jordano, Pedro},
doi = {10.1111/j.1461-0248.2005.00810.x},
isbn = {1461-023X},
issn = {1461023X},
journal = {Ecology Letters},
keywords = {Abundance,Effectiveness,Interaction frequency,Interaction networks,Interaction strength,Mutualism,Plant reproduction,Plant-animal interactions,Pollination,Seed dispersal},
number = {10},
pages = {1088--1094},
pmid = {231737200008},
title = {{Interaction frequency as a surrogate for the total effect of animal mutualists on plants}},
volume = {8},
year = {2005}
}
@article{Holt1997a,
abstract = {Fundamental and objective: Holt-Oram syndrome (HOS) is a heart-hand disease with an autosomal dominant inheritance pattern. About 85{\%} of the affected patients present de novo mutations in the TBX5gene. The aim of this study is to propose a molecular strategy to diagnose patients with clinical suspicion of HOS. Patients and methods: A sequence analysis of 7 patients from exon 2 to exon 8 of the TBX5 gene was performed. MLPAp179 and MLPAp180 were performed in those cases in which no mutation was found. Results: p.Arg270X and p.Ala34Glyfsx27 mutations were identified in 2 cases. These cases fulfilled the strict clinical criteria, had a family history of HOS and had similar clinical features. In other three cases, MLPA results showed deletions of the GLI3 coding region. Conclusions: In order to increase the TBX5 mutation detection rate, an exhaustive physical examination focused on the strict clinical criteria may be necessary to rule out clinical overlapping syndromes. We propose that molecular analysis of GLI3 may be performed in patients with clinical suspicion of HOS without mutations in TBX5. {\&} 2010 Elsevier Espa{\~{n}}a, S.L. All rights reserved.},
author = {Mart{\'{i}}nez-Garc{\'{i}}a, M{\'{o}}nica and Lorda-Sanchez, Isabel and Garc{\'{i}}a-Hoyos, Maria and Ramos, Carmen and Ayuso, Carmen and Trujillo-Tiebas, Mar{\'{i}}a Jos{\'{e}}},
doi = {10.1016/j.medcli.2010.04.013},
isbn = {0003-0147},
issn = {15788989},
journal = {Medicina Clinica},
keywords = {Holt-Oram syndrome TBX5 Gene LI3 gene LPA},
number = {14},
pages = {653--657},
pmid = {21070912},
title = {{S{\'{i}}ndrome de Holt-Oram: descripci{\'{o}}n de 7 casos}},
volume = {135},
year = {2010}
}
@article{Jordano1987,
abstract = {Peer reviewed},
archivePrefix = {arXiv},
arxivId = {http://www.jstor.org/stable/2461728},
author = {Jordano, Pedro},
doi = {10.1086/284665},
eprint = {/www.jstor.org/stable/2461728},
isbn = {00030147},
issn = {0003-0147},
journal = {The American Naturalist},
number = {5},
pages = {657--677},
pmid = {195},
primaryClass = {http:},
title = {{Patterns of Mutualistic Interactions in Pollination and Seed Dispersal: Connectance, Dependence Asymmetries, and Coevolution}},
url = {http://www.journals.uchicago.edu/doi/10.1086/284665},
volume = {129},
year = {1987}
}
@article{Allesina2012a,
abstract = {The 'nested' pattern of mutualistic interactions between plants and their pollinators is thought to promote species coexistence. But the key determinant may instead be the number of partners that species have. See Letter p.227},
author = {Allesina, Stefano},
doi = {10.1038/487175a},
isbn = {0028-0836},
issn = {00280836},
journal = {Nature},
number = {7406},
pages = {175--176},
pmid = {22785307},
title = {{Ecology: The more the merrier}},
url = {http://www.nature.com/doifinder/10.1038/487175a},
volume = {487},
year = {2012}
}
@article{Allesina2015,
abstract = {The stability of ecological systems has been a long-standing focus of ecology. Recently, tools from random matrix theory have identified the main drivers of stability in ecological communities whose network structure is random. However, empirical food webs differ greatly from random graphs. For example, their degree distribution is broader, they contain few trophic cycles, and they are almost interval. Here we derive an approximation for the stability of food webs whose structure is generated by the cascade model, in which 'larger' species consume 'smaller' ones. We predict the stability of these food webs with great accuracy, and our approximation also works well for food webs whose structure is determined empirically or by the niche model. We find that intervality and broad degree distributions tend to stabilize food webs, and that average interaction strength has little influence on stability, compared with the effect of variance and correlation.},
author = {Allesina, Stefano and Grilli, Jacopo and Barab{\'{a}}s, Gy{\"{o}}rgy and Tang, Si and Aljadeff, Johnatan and Maritan, Amos},
doi = {10.1038/ncomms8842},
isbn = {2041-1723 (Electronic)$\backslash$r2041-1723 (Linking)},
issn = {20411723},
journal = {Nature Communications},
pages = {7842},
pmid = {26198207},
title = {{Predicting the stability of large structured food webs}},
url = {http://www.ncbi.nlm.nih.gov/pubmed/26198207},
volume = {6},
year = {2015}
}
@article{Curtsdotter2011,
abstract = {The loss of species from ecological communities can unleash a cascade of secondary extinctions, the risk and extent of which are likely to depend on the traits of the species that are lost from the community. To identify species traits that have the greatest impact on food web robustness to species loss we here subject allometrically scaled, dynamical food web models to several deletion sequences based on species' connectivity, generality, vulnerability or body mass. Further, to evaluate the relative importance of dynamical to topological effects we compare robustness between dynamical and purely topological models. This comparison reveals that the topological approach overestimates robustness in general and for certain sequences in particular. Top-down directed sequences have no or very low impact on robustness in topological analyses, while the dynamical analysis reveals that they may be as important as high-impact bottom-up directed sequences. Moreover, there are no deletion sequences that result, on average, in no or very few secondary extinctions in the dynamical approach. Instead, the least detrimental sequence in the dynamical approach yields an average robustness similar to the most detrimental (non-basal) deletion sequence in the topological approach. Hence, a topological analysis may lead to erroneous conclusions concerning both the relative and the absolute importance of different species traits for robustness. The dynamical sequential deletion analysis shows that food webs are least robust to the loss of species that have many trophic links or that occupy low trophic levels. In contrast to previous studies we can infer, albeit indirectly, that secondary extinctions were triggered by both bottom-up and top-down cascades. {\textcopyright} 2011 Gesellschaft fur Okologie.},
author = {Curtsdotter, Alva and Binzer, Amrei and Brose, Ulrich and de Castro, Francisco and Ebenman, Bo and Ekl{\"{o}}f, Anna and Riede, Jens O. and Thierry, Aaron and Rall, Bj{\"{o}}rn C.},
doi = {10.1016/j.baae.2011.09.008},
isbn = {1439-1791},
issn = {14391791},
journal = {Basic and Applied Ecology},
keywords = {Body size,Bottom-up effect,Extinction cascades,Generality,Keystone species,Species loss,Stability,Top-down effect,Trophic interactions,Vulnerability},
number = {7},
pages = {571--580},
publisher = {Elsevier GmbH},
title = {{Robustness to secondary extinctions: Comparing trait-based sequential deletions in static and dynamic food webs}},
url = {http://dx.doi.org/10.1016/j.baae.2011.09.008},
volume = {12},
year = {2011}
}
@article{Johnson2000a,
author = {Johnson, Jeffrey C and Borgatti, Stephen P and Luczkovich, Joseph J and Everett, Martin G},
journal = {Journal of Social Structure},
pages = {1--32},
title = {{Network Role Analysis in the Study of Food Webs : An Application of Regular Role}},
volume = {2},
year = {2014}
}
@article{Yodzis1999,
abstract = {Aggregations of biological species on the basis of trophic similarity (trophospecies) are the basic units of study in food web and ecosystem research, yet little attention has been devoted to articulating objective protocols for defining such aggregations. This study formulates several possible definitions based on alternative measures of similarity and hierarchical clustering. Twenty-four alternative methods were applied to a food web consisting of 116 original trophic entities (OTEs) from a tropical floodplain, for which the relative magnitude of each trophic link was estimated based on dietary data measured as volumetric proportions. The resulting 24 trophic hierarchies were compared based on cophenetic correlation and matrices of OTE pairwise similarity, and patterns were interpreted based on additional ecological analyses for this system. Similarity measures based on topological food web (presence/absence) data yielded slightly greater cophenetic correlations than did measures based on dietary proportions (flow webs), but, overall, OTE pairwise correlations were not greater for one method relative to the other. The difference between these two approaches is driven by the treatment of weak feeding links; at least for the system considered here, distinctions among dietary generalists were obscured when weak links were weighted lightly. Additively combining the predator and resource aspects of each OTE's trophic role performed better than combining them multiplicatively. In general, there was little correspondence between OTE overlap in resource use and the extent to which predators were shared. Two measures of cluster similarity. one designed by us specifically for food webs (maximum linkage) and the other a standard method (average similarity between clusters), yielded more consistent and ecologically interpretable patterns of aggregation than other measures of cluster similarity considered. In deciding whether two trophospecies should be assigned a trophic link, the maximum linkage convention (in which a link is included if any pair of OTEs, one in each trophospecies. are linked) produced more aggregation than the minimum linkage convention (in which a link is included only if every pair of OTEs, pair members in each of the two trophospecies, is linked). The choice of similarity level for defining trophospecies remained an unresolved issue based on our analysis of this dataset. Perhaps the greatest challenge is posed by sampling bias within empirical datasets, and we ultimately conclude that it is difficult to identify trophospecies in this dataset by strictly objective criteria.},
author = {Yodzis, Peter and Winemiller, Kirk O.},
doi = {10.2307/3546748},
isbn = {0030-1299},
issn = {00301299},
journal = {Oikos},
number = {2},
pages = {327},
pmid = {25041991},
title = {{In Search of Operational Trophospecies in a Tropical Aquatic Food Web}},
url = {http://www.jstor.org/stable/3546748?origin=crossref},
volume = {87},
year = {1999}
}
@book{vegan,
abstract = {Ordination methods, diversity analysis and other functions for community and vegetation ecologists},
annote = {From Duplicate 2 (Community Ecology Package - Oksanen, Author Jari; Blanchet, F Guillaume; Kindt, Roeland; Legen-, Pierre; Minchin, Peter R; Hara, R B O; Simpson, Gavin L; Solymos, Peter; Stevens, M Henry H)

R package version 2.0-5},
archivePrefix = {arXiv},
arxivId = {arXiv:1011.1669v3},
author = {Freeman, Carrie Packwood},
booktitle = {Green Food: An A-to-Z Guide},
doi = {10.4135/9781412971874.n145},
edition = {R package },
eprint = {arXiv:1011.1669v3},
isbn = {{\textless}null{\textgreater}},
issn = {20091007},
pages = {263},
pmid = {17666792},
title = {{Vegan}},
url = {http://sk.sagepub.com/reference/greenfood/n145.xml},
year = {2014}
}
@article{Dunne2013,
abstract = {Comparative research on food web structure has revealed generalities in trophic organization, produced simple models, and allowed assessment of robustness to species loss. These studies have mostly focused on free-living species. Recent research has suggested that inclusion of parasites alters structure. We assess whether such changes in network structure result from unique roles and traits of parasites or from changes to diversity and complexity. We analyzed seven highly resolved food webs that include metazoan parasite data. Our analyses show that adding parasites usually increases link density and connectance (simple measures of complexity), particularly when including concomitant links (links from predators to parasites of their prey). However, we clarify prior claims that parasites "dominate" food web links. Although parasites can be involved in a majority of links, in most cases classic predation links outnumber classic parasitism links. Regarding network structure, observed changes in degree distributions, 14 commonly studied metrics, and link probabilities are consistent with scale-dependent changes in structure associated with changes in diversity and complexity. Parasite and free-living species thus have similar effects on these aspects of structure. However, two changes point to unique roles of parasites. First, adding parasites and concomitant links strongly alters the frequency of most motifs of interactions among three taxa, reflecting parasites' roles as resources for predators of their hosts, driven by trophic intimacy with their hosts. Second, compared to free-living consumers, many parasites' feeding niches appear broader and less contiguous, which may reflect complex life cycles and small body sizes. This study provides new insights about generic versus unique impacts of parasites on food web structure, extends the generality of food web theory, gives a more rigorous framework for assessing the impact of any species on trophic organization, identifies limitations of current food web models, and provides direction for future structural and dynamical models.},
author = {Dunne, Jennifer A. and Lafferty, Kevin D. and Dobson, Andrew P. and Hechinger, Ryan F. and Kuris, Armand M. and Martinez, Neo D. and McLaughlin, John P. and Mouritsen, Kim N. and Poulin, Robert and Reise, Karsten and Stouffer, Daniel B. and Thieltges, David W. and Williams, Richard J. and Zander, Claus Dieter},
doi = {10.1371/journal.pbio.1001579},
editor = {Loreau, Michel},
isbn = {1545-7885},
issn = {15449173},
journal = {PLoS Biology},
month = {jun},
number = {6},
pages = {e1001579},
pmid = {23776404},
title = {{Parasites Affect Food Web Structure Primarily through Increased Diversity and Complexity}},
url = {http://dx.plos.org/10.1371/journal.pbio.1001579},
volume = {11},
year = {2013}
}
@article{Hechinger2011,
abstract = {The objective of this case study was to obtain some first-hand information about the functional consequences of a cosmetic tongue split operation for speech and tongue motility. One male patient who had performed the operation on himself was interviewed and underwent a tongue motility assessment, as well as an ultrasound examination. Tongue motility was mildly reduced as a result of tissue scarring. Speech was rated to be fully intelligible and highly acceptable by 4 raters, although 2 raters noticed slight distortions of the sibilants /s/ and /z/. The 3-dimensional ultrasound demonstrated that the synergy of the 2 sides of the tongue was preserved. A notably deep posterior genioglossus furrow indicated compensation for the reduced length of the tongue blade. It is concluded that the tongue split procedure did not significantly affect the participant's speech intelligibility and tongue motility.},
archivePrefix = {arXiv},
arxivId = {arXiv:1011.1669v3},
author = {Hechinger, Ryan F. and Lafferty, Kevin D. and Dobson, Andy P. and Brown, James H. and Kuris, Armand M.},
doi = {10.1126/science.1204337},
eprint = {arXiv:1011.1669v3},
isbn = {9780874216561},
issn = {00368075},
journal = {Science},
keywords = {Animals,Biodiversity,Biomass,Birds,Birds: metabolism,Birds: physiology,Body Size,Body Temperature,Ecosystem,Energy Metabolism,Fishes,Fishes: metabolism,Fishes: physiology,Food Chain,Invertebrates,Invertebrates: metabolism,Invertebrates: physiology,Linear Models,Parasites,Parasites: metabolism,Parasites: physiology,Population Dynamics,Regression Analysis,Vertebrates,Vertebrates: metabolism,Vertebrates: physiology},
month = {jul},
number = {6041},
pages = {445--448},
pmid = {15003161},
title = {{A common scaling rule for abundance, energetics, and production of parasitic and free-living species}},
url = {http://www.sciencemag.org/cgi/doi/10.1126/science.1204337},
volume = {333},
year = {2011}
}
@article{Wootton2005,
abstract = {Ecologists would like to explain general patterns observed across multi-species communities, such as species-area and abundance-frequency relationships, in terms of the fundamental processes of birth, death and migration underlying the dynamics of all constituent species. The unified neutral theory of biodiversity and related theories based on these fundamental population processes have successfully recreated general species-abundance patterns without accounting for either the variation among species and individuals or resource-releasing processes such as predation and disturbance, long emphasized in ecological theory. If ecological communities can be described adequately without estimating variation in species and their interactions, our understanding of ecological community organization and the predicted consequences of reduced biodiversity and environmental change would shift markedly. Here, I introduce a strong method to test the neutral theory that combines field parameterization of the underlying population dynamics with a field experiment, and apply it to a rocky intertidal community. Although the observed abundance-frequency distribution of the system follows that predicted by the neutral theory, the neutral theory predicts poorly the field experimental results, indicating an essential role for variation in species interactions.},
author = {Wootton, J. Timothy},
doi = {10.1038/nature03211},
isbn = {0028-0836},
issn = {00280836},
journal = {Nature},
number = {7023},
pages = {309--312},
pmid = {15662423},
title = {{Field parameterization and experimental test of the neutral theory of biodiversity}},
volume = {433},
year = {2005}
}
@article{Abrams2000,
abstract = {To describe a predator-prey relationship, it is necessary to specify the rate of prey consumption by an average predator. This functional response largely determines dynamic stability, responses to environmental influences and the nature of indirect effects in the food web containing the predator- prey pair. Nevertheless, measurements of functional responses in nature are quite rare. Recently, much work has been devoted to comparing two idealized forms of the functional response: prey dependent and ratio dependent. Although we agree that predator abundance often affects the consumption rate of individual predators, this phenomenon requires more attention. Disagreement remains over which of the two idealized responses serves as a better starting point in building models when data on predator dependence are absent.},
author = {Abrams, Peter A. and Ginzburg, Lev R.},
doi = {10.1016/S0169-5347(00)01908-X},
isbn = {0169-5347},
issn = {01695347},
journal = {Trends in Ecology and Evolution},
number = {8},
pages = {337--341},
pmid = {10884706},
title = {{The nature of predation: Prey dependent, ratio dependent or neither?}},
volume = {15},
year = {2000}
}
@article{Anderson1978,
abstract = {(1) The functional response of the fish predator Brachydanio rerio Hamilton- Buchanan to changes in the density of two prey species, Daphnia magna L. and the cercarial stage of the ectoparasitic digenean Transversotrema patialense (Soparkar), is shown to be of the type II form where the instantaneous predation rate is unaffected by changes in prey density. (2) A model is developed to describe this functional response, based on the concept of predator satiation. The number of prey items required to create satiation is shown to be dependent on the experimental procedures used to elicit the functional response. (3) The fish predator, Brachydanio rerio, acts in the dual role of predator/host for the cercarial stage of Transversotrema patialense. The concomitant predation and infection processes created by this ecological association are shown to be characterised by constant instantaneous predation and infection rates which appear to be unaffected by changes in prey/parasite density. (4) The infection process is unaffected by density dependent constraints over a wide range of exposure densities and the number of parasites attached per host is shown to be directly proportional to cercarial numbers. (5) Stochastic elements are shown to be important determinants in the dynamics of the infection process and overdispersion in the frequency distribution of the number of parasites attached per host is thought to be generated by heterogeneity between fish and heterogeneity in time created by changes in infective stage density. (6) The relevance of concomitant predation and infection processes to the dynamics of digenean life cycles is discussed.},
author = {Anderson, R M and Whitfield, P J and Dobson, Andy P and Keymer, Anne E},
doi = {10.1177/0884533615572654},
isbn = {0021-8790},
issn = {0884-5336},
journal = {Journal of Animal Ecology},
number = {3},
pages = {891--911},
title = {{Concomitant predation and infection processes: an experimental study}},
url = {http://www.jstor.org/stable/3677{\%}0Ahttp://www.jstor.org/stable/10.2307/3677},
volume = {47},
year = {1978}
}
@book{Hauer2011,
abstract = {This bestselling title has been significantly revised and updated. Now including important information on remote sensing, the relation between stream flow and alluviation, coverage of macrophyte and a new chapter on riparian zones.Providing a complete series of field and laboratory protocols in stream ecology that are idea for teaching or conducting research. The scope of this unique book covers five important areas of stream ecology: Physical Stream Ecology, Material Storage and Transport, Stream Biota, Community Interactions, and Ecosystem Processes.Key Features:*Provides a variety of exercises in each chapter to accomodate both the student and the practicing scientist*Covers all areas of stream ecology in chapters written by leading experts in their respective fields*Includes detailed instructions, illustrations, formulae, and data sheets for conducting stream ecology*Presents taxonomic keys to common stream invertebbrates and algae},
address = {Oxford, UK},
author = {Hauer, F. R. and Lamberti, Gary A.},
booktitle = {Methods in Stream Ecology},
edition = {2nd},
editor = {Hauer, F. Richard and Lamberti, Gary A.},
isbn = {0123329078},
number = {1},
pages = {877},
publisher = {Academic Press},
title = {{Methods in stream ecology (Google eBook)}},
url = {http://books.google.com/books?id=Nr8gdAUHmLEC{\&}pgis=1},
volume = {9},
year = {2006}
}
@article{Poulin2002,
abstract = {Complex life cycles are a hallmark of parasitic trematodes. In several trematode taxa, however, the life cycle is truncated: fewer hosts are used than in a typical three-host cycle, with fewer transmission events. Eliminating one host from the life cycle can be achieved in at least three different ways. Some trematodes show even more extreme forms of life cycle abbreviations, using only a mollusc to complete their cycle, with or without sexual reproduction. The occurrence of these phenomena among trematode families are reviewed here and show that life cycle truncation has evolved independently many times in the phylogeny of trematodes. The hypotheses proposed to account for life-cycle truncation, in addition to the factors preventing the adoption of shorter cycles by all trematodes are also discussed. The study of shorter life cycles offers an opportunity to understand the forces shaping the evolution of life cycles in general.},
author = {Poulin, Robert and Cribb, Thomas H.},
doi = {10.1016/S1471-4922(02)02262-6},
isbn = {1471-4922 (Print)$\backslash$r1471-4922 (Linking)},
issn = {14714922},
journal = {Trends in Parasitology},
number = {4},
pages = {176--183},
pmid = {11998706},
title = {{Trematode life cycles: Short is sweet?}},
volume = {18},
year = {2002}
}
@article{RosasValdez2011,
abstract = {Host specificity plays an essential role in shaping the evolutionary history of host-parasite associations. In this study, an index of host specificity recently proposed was used to test, quantitatively, the hypothesis that some groups of parasites are characteristics of some host fish families along their distribution range. A database with all published records on the helminth parasites of freshwater siluriforms of Mexico was used. The host specificity index was used considering its advantage to measure the taxonomic heterogeneity of the host assemblages and its appropriateness for unequal sampling data. The helminth parasite fauna of freshwater siluriforms in Mexico seems to be specific for different host taxonomic categories. However, a relatively high number of species (47{\%} of the total helminth fauna) is specific to their respective host family. This result provides further corroboration for the biogeographic hypothesis of the core helminth fauna proposed previously. The statistical values for host specificity obtained herein seem to be independent of host range. However, the accurate taxonomic identification of the parasites is fundamental for the evaluation of host specificity and the accurate evolutionary interpretation of this phenomenon.},
author = {Rosas-Valdez, Rogelio and de Le{\'{o}}n, Gerardo P{\'{e}}rez-Ponce},
doi = {10.1645/GE-2541.1},
isbn = {0022-3395},
issn = {0022-3395},
journal = {Journal of Parasitology},
number = {2},
pages = {361--363},
pmid = {21506789},
title = {{Patterns of Host Specificity Among the Helminth Parasite Fauna of Freshwater Siluriforms: Testing the Biogeographical Core Parasite Fauna Hypothesis}},
url = {http://www.bioone.org/doi/abs/10.1645/GE-2541.1},
volume = {97},
year = {2011}
}
@article{Gabriel1988a,
abstract = {Calculates the differences in payoffs between diel vertical migrating and nonmigrating zooplankton. Results depend on the abundance and growth rate of the algae, the filtration rate and density of the zooplankton, the strategy-dependent efficiency of converting food into reproduction, temperature-dependent egg developmental time, day length, and predation pressure. In addition to regions where the 2 strategies exclude each other, it is also possible for a population to reach an evolutionarily stable equilibrium mixture of strategies. At high food concentrations, equilibrium is unlikely because it would require a predation risk increasing with prey abundance; but at low food concentrations, strategy mixtures can be stable for many combinations of realistic model parameters, even with density-independent mortality caused by fish predation. Temporal variation in the selective forces shifts this equilibrium point. Therefore, the model can explain seasonal changes and long-term trends in the pattern of diel vertical migration. -from Authors},
author = {Gabriel, Wilfried and Thomas, Bernhard},
doi = {10.1086/284845},
isbn = {0003-0147},
issn = {0003-0147},
journal = {The American Naturalist},
number = {2},
pages = {199--216},
pmid = {17820973},
title = {{Vertical Migration of Zooplankton as an Evolutionarily Stable Strategy}},
url = {http://www.journals.uchicago.edu/doi/10.1086/284845},
volume = {132},
year = {1988}
}
@article{Huyse2003,
abstract = {Using species-level phylogenies, the speciation mode of Gyrodactylus species infecting a single host genus was evaluated. Eighteen Gyrodactylus species were collected from gobies of the genus Pomatoschistus and sympatric fish species across the distribution range of the hosts. The V4 region of the ssrRNA and the internal transcribed spacers encompassing the 5.8S rRNA gene were sequenced; by including published sequences a total of 30 species representing all subgenera were used in the data analyses. The molecular phylogeny did not support the morphological groupings into subgenera as based on the excretory system, suggesting that the genus needs systematic revisions. Paraphyly of the total Gyrodactylus fauna of the gobies indicates that at least two independent colonisation events were involved, giving rise to two separate groups, belonging to the subgenus Mesonephrotus and Paranephrotus, respectively. The most recent association probably originated from a host switching event from Gyrodactylus arcuatus, which parasitises three-spined stickleback, onto Pomatoschistus gobies. These species are highly host-specific and form a monophyletic group, two possible 'signatures' of co-speciation. Host specificity was lower in the second group. The colonising capacity of these species is illustrated by a host jump from gobiids to another fish order (Anguilliformes), supporting the hypothesis of a European origin of Gyrodactylus anguillae and its intercontinental introduction by the eel trade. Thus, allopatric speciation seems to be the dominant mode of speciation in this host-parasite system, with a possible case of sympatric speciation. {\textcopyright} 2003 Australian Society for Parasitology Inc. Published by Elsevier Ltd. All rights reserved.},
author = {Huyse, Tine and Audenaert, Vanessa and Volckaert, Filip A.M.},
doi = {10.1016/S0020-7519(03)00253-4},
isbn = {3216324575},
issn = {00207519},
journal = {International Journal for Parasitology},
keywords = {Coevolution,Host specificity,Host switching,Host-parasite evolution,Internal transcribed spacers rDNA,Phylogeny,Pomatoschistus},
number = {14},
pages = {1679--1689},
pmid = {14636683},
title = {{Speciation and host-parasite relationships in the parasite genus Gyrodactylus (Monogenea, Platyhelminthes) infecting gobies of the genus Pomatoschistus (Gobiidae, Teleostei)}},
volume = {33},
year = {2003}
}
@book{Sagar1995,
address = {Christchurch, New Zealand},
author = {Sagar, P.M. and Schwarz, A.-M. and Howard-Williams, C.},
isbn = {0478083572},
pages = {40},
publisher = {National Institute of Water and Atmospheric Research Ltd.},
title = {{Review of the ecological role of Black Swan ({\textless}i{\textgreater}Cygnus atratus{\textless}/i{\textgreater})}},
year = {1995}
}
@article{Haney1988,
abstract = {Zooplankton exhibit a variety of daily cycles including vertical and horizontal migrations, changes in feeding behavior and alternating reproductive states. The most popular hypothesis to explain the adaptive advantage of die] vertical migration is predator avoidance, i.e., nocturnal vertical migrations afford protection from visually feeding predators, whereas re- verse (diurnal) migrations result from avoidance of nocturnally migrating, nonvisual pred- ators. Proposed metabolic advantages of vertical migration have received much attention, but relatively little experimental support. Nocturnal migrations may also represent an adap- tive behavior for avoidance of damaging solar radiation. One is struck with the variations in diel behavior patterns and the apparent plasticity of zooplankton in adapting diel behaviors to fit specific environments. Recent studies considering multiple causes of vertical migration show much promise, Problems and improvements in studies of diel zooplankton behavior are discussed.},
author = {Haney, J. F.},
doi = {doi: 10.1016/j.bios.2016.09.034},
isbn = {0007-4977},
issn = {00074977},
journal = {Bulletin of Marine Science},
number = {3},
pages = {583--603},
title = {{Diel patterns of zooplankton behavior}},
volume = {43},
year = {1988}
}
@article{Wakelin2004,
author = {Wakelin, Michael},
issn = {00294470},
journal = {Notornis},
keywords = {Aythya novaeseelandiae,Food,New Zealand dabchick,New Zealand scaup,Poliocephalus rufopectus},
number = {4},
pages = {242--245},
title = {{Foods of New Zealand dabchick (Poliocephalus rufopectus) and New Zealand scaup (Aythya novaeseelandiae)}},
volume = {51},
year = {2004}
}
@article{Wood2012,
abstract = {In lakes, benthic micro-algae and cyanobacteria (periphyton) can contribute significantly to total primary productivity and provide important food sources for benthic invertebrates. Despite recognition of their importance, few studies have explored the diversity of the algal and cyanobacterial composition of periphyton mats in temperate lakes. In this study, we sampled periphyton from three New Zealand lakes: Tikitapu (oligotrophic), Ōkāreka (mesotrophic) and Rotoiti (eutrophic). Statistical analysis of morphological data showed a clear delineation in community structure among lakes and highlighted the importance of cyanobacteria. Automated rRNA intergenic spacer analysis (ARISA) and 16S rRNA gene clone libraries were used to investigate cyanobacterial diversity. Despite the close geographic proximity of the lakes, cyanobacterial species differed markedly. The 16S rRNA gene sequence analysis identified eight cyanobacterial OTUs. A comparison with other known cyanobacterial sequences in GenBank showed relatively low similarities (91-97{\%}). Cyanotoxin analysis identified nodularin in all mats from Lake Tikitapu. ndaF gene sequences from these samples had very low (≤ 89{\%}) homology to sequences in other known nodularin producers. To our knowledge, this is the first detection of nodularin in a freshwater environment in the absence of Nodularia. Six cyanobacteria species were isolated from Lake Tikitapu mats. None were found to produce nodularin. Five of the species shared low ({\textless} 97{\%}) 16S rRNA gene sequence similarities with other cultured cyanobacteria.},
author = {Wood, Susie A. and Kuhajek, Jeannie M. and de Winton, Mary and Phillips, Ngaire R.},
doi = {10.1111/j.1574-6941.2011.01217.x},
isbn = {0168-6496},
issn = {01686496},
journal = {FEMS Microbiology Ecology},
keywords = {Cyanobacteria,Nodularin,Periphyton},
number = {2},
pages = {312--326},
pmid = {22092304},
title = {{Species composition and cyanotoxin production in periphyton mats from three lakes of varying trophic status}},
volume = {79},
year = {2012}
}
@article{Iwasa1982,
abstract = {Many zooplankton species living in a lake or a sea show marked vertical migration over large distances on a daily basis. They usually ascend from deeper waters to surficial strata at dusk and descend at dawn. Zooplankton feeders such as fish and macrocrustacea, also migrate, following their prey. This migration is caused by the behavioral response of the zooplankton to the variation in light intensity (Siebeck 1960; Daan and Ringelberg 1969). On the other hand, the adaptive significance of this movement has been explained by many factors: e.g., the harmfulness of direct solar radiation (Huntsman 1924; Hairston 1976), the effect of shading on the reduction of the photosynthesis by phytoplankton (McAl- lister 1969; Kerfoot 1970), increased genetic exchange (David 1961), optimal phytoplankton harvesting (Conover 1968; Enright 1977) and demographic advan- tages of cold subsurface temperature (McLaren 1974; Enright and Honegger 1977). Here I focus on the hypothesis that a zooplankton migrates to avoid predators that hunt by sight (Murray and Hjort 1912). I propose a habitat-selection game between predator and prey. This simple model explains many characteris- tics of vertical migration and presents several testable predictions.},
author = {Iwasa, Yoh},
doi = {10.1086/283980},
isbn = {00030147},
issn = {0003-0147},
journal = {The American Naturalist},
number = {2},
pages = {171--180},
pmid = {3},
title = {{Vertical Migration of Zooplankton: A Game Between Predator and Prey}},
url = {http://www.journals.uchicago.edu/doi/10.1086/283980},
volume = {120},
year = {1982}
}
@article{Hayes1989,
abstract = {This paper describes the first tests of a fine mesh trap net suitable for sampling shallow- lake fish communities in New Zealand. The catch from 2 shallow lower Waikato lakes of this and 5 other gear types frequently used in freshwater fisheries research in New Zealand are compared. The trap net caught the widest range of fish species and sizes, and was the least selective for species composi tion and relative abundance of any individual gear type. A combination of trap net and gill net caught the widest range of species and sizes for individual or pairwise combinations of gear types, and is recommended as being the best combination for sampling shallow-lake fish communities in New Zealand.},
author = {Hayes, John W.},
doi = {10.1080/00288330.1989.9516368},
issn = {11758805},
journal = {New Zealand Journal of Marine and Freshwater Research},
keywords = {Community composition,Environmental impact assessment,Fish,Selectivity,Shallow water,Species inventory,Trap nets},
number = {3},
pages = {321--324},
title = {{Comparison between a fine mesh trap net and five other fishing gears for sampling shallow‐lake fish communities in new zealand (note)}},
volume = {23},
year = {1989}
}
@article{Rhode2001,
abstract = {The vertical migration of zooplankton into lower and darker water strata by day is generally explained by the avoidance of visually orienting predators, mainly fish; however, it is unclear why daily zooplankton migration has been maintained in fishless areas. In addition to predation, ultraviolet radiation-a hazardous factor for zooplankton in the surface layers of marine and freshwater environments-has been suspected as a possible cause of daytime downward migration. Here we test this hypothesis by studying several Daphnia species, both in a controlled laboratory system and under natural sunlight in an outdoor system. We selected Daphnia species that differed in their pigmentation as both melanin and carotenoids have been shown to protect Daphnia from ultraviolet light. All Daphnia species escaped into significantly deeper water layers under ultraviolet radiation. The extent to which the daphnids responded to this radiation was inversely linked to their pigmentation, which reduced ultraviolet transmission. These results suggest that ultraviolet avoidance is an additional factor in explaining daytime downward migration. Synergistic benefits might have shaped the evolution of this complex behaviour.},
author = {Rhode, S. C. and Pawlowski, M. and Tollrian, R.},
doi = {10.1038/35083567},
isbn = {00280836},
issn = {00280836},
journal = {Nature},
keywords = {Daphnia, crustaceans, ultraviolet, predator-prey i},
number = {6842},
pages = {69--72},
pmid = {11452307},
title = {{The impact of ultraviolet radiation on the vertical distribution of zooplankton of the genus Daphnia}},
volume = {412},
year = {2001}
}
@article{Gabriel1988,
abstract = {Calculates the differences in payoffs between diel vertical migrating and nonmigrating zooplankton. Results depend on the abundance and growth rate of the algae, the filtration rate and density of the zooplankton, the strategy-dependent efficiency of converting food into reproduction, temperature-dependent egg developmental time, day length, and predation pressure. In addition to regions where the 2 strategies exclude each other, it is also possible for a population to reach an evolutionarily stable equilibrium mixture of strategies. At high food concentrations, equilibrium is unlikely because it would require a predation risk increasing with prey abundance; but at low food concentrations, strategy mixtures can be stable for many combinations of realistic model parameters, even with density-independent mortality caused by fish predation. Temporal variation in the selective forces shifts this equilibrium point. Therefore, the model can explain seasonal changes and long-term trends in the pattern of diel vertical migration. -from Authors},
author = {Gabriel, Wilfried and Thomas, Bernhard},
doi = {10.1086/284845},
isbn = {0003-0147},
issn = {0003-0147},
journal = {The American Naturalist},
number = {2},
pages = {199--216},
pmid = {17820973},
title = {{Vertical Migration of Zooplankton as an Evolutionarily Stable Strategy}},
url = {http://www.journals.uchicago.edu/doi/10.1086/284845},
volume = {132},
year = {1988}
}
@article{Hughes2012,
abstract = {The number of sites sampled must be considered when determining the effort necessary for adequately assessing taxa richness in an ecosystem for bioassessment purposes; however, there have been few studies concerning the number of sites necessary for bioassessment of large rivers. We evaluated the effect of sample size (i.e., number of sites) necessary to collect vertebrate (fish and aquatic amphibians), macroinvertebrate, and diatom taxa from seven large rivers in Oregon and Washington, USA during the summers of 2006-2008. We used Monte Carlo simulation to determine the number of sites needed to collect 90-95{\%} of the taxa 75-95{\%} of the time from 20 randomly located sites on each river. The river wetted widths varied from 27.8 to 126.0 m, mean substrate size varied from 1 to 10 cm, and mainstem distances sampled varied from 87 to 254 km. We sampled vertebrates at each site (i.e., 50 times the mean wetted channel width) by nearshore-raft electrofishing. We sampled benthic macroinvertebrates nearshore through the use of a 500-$\mu$m mesh kick net at 11 systematic stations. From each site composite sample, we identified a target of 500 macroinvertebrate individuals to the lowest possible taxon, usually genus. We sampled benthic diatoms nearshore at the same 11 stations from a 12-cm(2) area. At each station, we sucked diatoms from soft substrate into a 60-ml syringe or brushed them off a rock and rinsed them with river water into the same jar. We counted a minimum of 600 valves at 1,000× magnification for each site. We collected 120-211 diatom taxa, 98-128 macroinvertebrate taxa, and 14-33 vertebrate species per river. To collect 90-95{\%} of the taxa 75-95{\%} of the time that were collected at 20 sites, it was necessary to sample 11-16 randomly distributed sites for vertebrates, 13-17 sites for macroinvertebrates, and 16-18 sites for diatoms. We conclude that 12-16 randomly distributed sites are needed for cost-efficient sampling of vertebrate richness in the main stems of our study rivers, but 20 sites markedly underestimates the species richness of benthic macroinvertebrates and diatoms in those rivers.},
author = {Hughes, R M and Herlihy, A T and Gerth, W J and Pan, Y},
doi = {10.1007/s10661-011-2181-9},
isbn = {1573-2959 (Electronic)$\backslash$r0167-6369 (Linking)},
issn = {1573-2959},
journal = {Environmental Monitoring and Assessment},
keywords = {Benthos,Fish,Taxa Richness,macroinvertebrate communities,sampling effort},
number = {5},
pages = {3185--3198},
pmid = {21713475},
title = {{Estimating vertebrate, benthic macroinvertebrate, and diatom taxa richness in raftable Pacific Northwest rivers for bioassessment purposes}},
volume = {184},
year = {2012}
}
@article{Lagrue2011,
abstract = {Parasite infection patterns were compared with the occurrence of their intermediate hosts in the diet of nine sympatric fish species in a New Zealand lake. Stomach contents and infection levels of three gastrointestinal helminth species were examined from the entire fish community. The results highlighted some links between fish host diet and the flow of trophically transmitted helminths. Stomach contents indicated that all but one fish species were exposed to these helminths through their diet. Host feeding behaviour best explained infection patterns of the trematode Coitocaecum parvum among the fish community. Infection levels of the nematode Hedruris spinigera and the acanthocephalan Acanthocephalus galaxii, however, were not correlated with host diets. Host specificity is thus likely to modulate parasite infection patterns. The data indicate that host diet and host-parasite compatibility both contribute to the distribution of helminths in the fish community. Furthermore, the relative influence of encounter (trophic interactions between prey and predator hosts) and compatibility (host suitability) filters on infection levels appeared to vary between host-parasite species associations. Therefore, understanding parasite infection patterns and their potential impacts on fish communities requires determining the relative roles of encounter and compatibility filters within and across all potential host-parasite associations.},
author = {Lagrue, C. and Kelly, D. W. and Hicks, A. and Poulin, R.},
doi = {10.1111/j.1095-8649.2011.03041.x},
isbn = {1095-8649 (Electronic)$\backslash$r0022-1112 (Linking)},
issn = {00221112},
journal = {Journal of Fish Biology},
keywords = {Fish diet,Gastrointestinal helminths,Host specificity,Trophically transmitted parasites},
number = {2},
pages = {466--485},
pmid = {21781103},
title = {{Factors influencing infection patterns of trophically transmitted parasites among a fish community: Host diet, host-parasite compatibility or both?}},
volume = {79},
year = {2011}
}
@article{ODonnell1982,
abstract = {L'analyse des loci microsatellites a {\'{e}}t{\'{e}} utilis{\'{e}}e pour d{\'{e}}finir les rapports de similitude g{\'{e}}n{\'{e}}tique entre 238 cultivars de Vitis vinifera ssp. sativa cultiv{\'{e}}s dans les pays du bassin m{\'{e}}diterran{\'{e}}en central et occidental. L'{\'{e}}laboration statistique des donn{\'{e}}es obtenues, effectu{\'{e}}e par analyse des clusters, a permis de grouper les cultivars en cinq groupes principaux, parmi lesquels, le groupe 1, qui comprend 50 p. cent des {\'{e}}chantillons avec des valeurs de similarit{\'{e}} g{\'{e}}n{\'{e}}tique comprises entre 28 et 87 p. cent et le groupe 3 qui comprend les {\'{e}}chantillons {\`{a}} provenance g{\'{e}}ographique diff{\'{e}}rente. Les autres groupes, avec assez peu d'{\'{e}}chantillons, ont montr{\'{e}} un haut degr{\'{e}} de sp{\'{e}}cificit{\'{e}} g{\'{e}}ographique. L'{\'{e}}laboration statistique, par tests X et par analyse des composants principaux (PCA), a montr{\'{e}} d'importantes relations de similitude g{\'{e}}n{\'{e}}tique entre les cultivars fran{\c{c}}ais et ib{\'{e}}riques et entre les cultivars grecs et ceux de la zone balkanique. Ces donn{\'{e}}es concordent avec la position g{\'{e}}ographique occup{\'{e}}e divers cultivars {\'{e}}tudi{\'{e}}s. Cela fait penser {\`{a}} une origine monocentrique de la vigne cultiv{\'{e}}e et donc d'un pool g{\'{e}}n{\'{e}}tique dominant qui aurait subi une intense circulation vari{\'{e}}tale parmi les pays m{\'{e}}diterran{\'{e}}ens Sur la base des donn{\'{e}}es arch{\'{e}}obotaniques, on a pu d{\'{e}}finir un mod{\`{e}}le d'origine m{\'{e}}diorientale de la vigne permettant de suivre sa diffusion en Europe, de l'orient vers l'occident. Les r{\'{e}}sultats de l'analyse de l'ADN effectu{\'{e}}e au moyen des marqueurs microsatellites sont coh{\'{e}}rents avec les donn{\'{e}}es arch{\'{e}}obotaniques et confirment le mod{\`{e}}le d{\'{e}}crit qui pr{\'{e}}voit une diffusion de la vigne par vagues successives correspondant aussi bien {\`{a}} des d{\'{e}}placements de populations qu'{\`{a}} des {\'{e}}changes culturels.},
author = {Labra, M. and Failla, O. and Forni, G. and Ghiani, A. and Scienza, A. and Sala, F.},
doi = {10.1007/978-4-431-68042-0},
isbn = {978-4-431-68044-4},
issn = {11510285},
journal = {Journal International des Sciences de la Vigne et du Vin},
keywords = {Microsatellite,SSR (Simple Sequence Repeat),Vitis vinifera L.},
number = {1},
pages = {11--20},
title = {{Microsatellite analysis to define genetic diversity of grapevines (Vitis vinifera L.) grown in central and western Mediterranean countries}},
volume = {36},
year = {2002}
}
@incollection{Clayton2006,
address = {Hamilton, New Zealand},
author = {Clayton, John and Edwards, Tracey},
booktitle = {LakeSPI User Manual},
chapter = {Appendix 6},
edition = {Version 2},
pages = {57},
publisher = {National Institute of Water and Atmospheric Research Ltd.},
title = {{Plant identification sheets}},
year = {2006}
}
@book{Chapman2011,
address = {Christchurch, New Zealand},
author = {Chapman, M A and Lewis, M H and Winterbourn, M J},
isbn = {0002169053},
pages = {188},
publisher = {New Zealand Freshwater Sciences Society},
title = {{Guide to the freshwater Crustacea of New Zealand}},
year = {2011}
}
@book{Biggs2000,
abstract = {(Phone: 03 348 8987 Fax: 03 348 5548) for the New Zealand Ministry for the Environment},
address = {Christchurch, New Zealand},
author = {Biggs, B J F and Kilroy, C and NIWA, Christchurch},
booktitle = {Niwa},
isbn = {0478090994},
number = {ISBN 0-478-09099-4},
pages = {222},
publisher = {National Institute of Water and Atmospheric Research Ltd.},
title = {{Stream Periphyton Monitoring Manual. NIWA, Christchurch.}},
year = {2000}
}
@book{Heather1996,
address = {Auckland, New Zealand},
author = {Heather, Barrie and Robertson, Hugh},
pages = {432},
publisher = {Viking},
title = {{The Field Guide to the Birds of New Zealand (revised edition)}},
year = {2005}
}
@book{Moore1997,
address = {Dunedin, New Zealand},
author = {Morley, M S},
booktitle = {New Holland Publishers Ltd, Auckland},
pages = {143},
publisher = {Otago Regional Council},
title = {{A photographic guide to seashells of New Zealand}},
year = {2004}
}
@book{Winterbourn1989,
abstract = {Illustrated keys are provided to the 11 orders of aquatic and water-associated insects inhabiting the three main islands of New Zealand. The life history stages covered are those found in or on water bodies. Where possible insects are identified to genera and species, but sometimes identification is possible only to the family level (many Diptera and Coleoptera). Annotated notes on distribution, habitat, and taxonomic problems are incorporated in the keys, and references to the main .taxonomic and biological studies on New Zealand aquatic insects are given. The section on chironomid larvae is the first comprehensive guide to New Zealand taxa of this family},
address = {Auckland, New Zealand},
author = {Winterbourn, Michael J. and {Gregson, Klatharine}, L.D.},
booktitle = {Bulletin of the Entomological Society of the New Zealand},
edition = {Revised},
isbn = {0477066739},
issn = {0110-4527},
pages = {1--80},
publisher = {Entomological Society of New Zealand},
title = {{Guide to the aquatic insects of New Zealand}},
url = {http://www.ephemeroptera-galactica.com/pubs/pub{\_}w/pubwinterbournm1981p14.pdf},
volume = {5},
year = {1981}
}
@article{Surber1937,
abstract = {1937:. Transactions of the American Fisheries Society 66: 193-202},
author = {Surber, E. W.},
doi = {10.1577/1548-8659(1936)66[193:RTABFP]2.0.CO;2},
isbn = {1548-8659},
issn = {0002-8487},
journal = {Transactions of the American Fisheries Society1},
number = {1},
pages = {193--202},
title = {{Rainbow trout and bottom fauna production in one mile of stream}},
volume = {66},
year = {1937}
}
@incollection{Hubert1996,
address = {Bethesda, Maryland},
author = {Hubert, W A},
booktitle = {Fisheries Techniques},
edition = {2nd},
editor = {Murphy, B. R. and Willis, D. W.},
pages = {95--122},
publisher = {American Fisheries Society},
title = {{Passive Capture Techniques}},
year = {1983}
}
@book{Nielsen1983,
abstract = {BACKGROUND: Dornase alfa is currently used as a mucolytic to treat pulmonary disease (the major cause of morbidity and mortality) in cystic fibrosis. It reduces mucus viscosity in the lungs, promoting improved clearance of secretions. This is an update of a previously published review.$\backslash$n$\backslash$nOBJECTIVES: To determine whether the use of dornase alfa in cystic fibrosis is associated with improved mortality and morbidity compared to placebo or other medications that improve airway clearance, and to identify any adverse events associated with its use.$\backslash$n$\backslash$nSEARCH METHODS: We searched the Cochrane Cystic Fibrosis and Genetic Disorders Group Trials Register which comprises references identified from comprehensive electronic database searches, handsearching relevant journals and abstracts from conferences. Date of the most recent search of the Group's Cystic Fibrosis Register: 30 November 2015.Clinicaltrials.gov was also searched to identify unpublished or ongoing trials. Date of most recent search: 28 November 2015.$\backslash$n$\backslash$nSELECTION CRITERIA: All randomised and quasi-randomised controlled trials comparing dornase alfa to placebo, standard therapy or other medications that improve airway clearance.$\backslash$n$\backslash$nDATA COLLECTION AND ANALYSIS: Authors independently assessed trials against the inclusion criteria; two authors carried out analysis of methodological quality and data extraction.$\backslash$n$\backslash$nMAIN RESULTS: The searches identified 54 trials, of which 19 (including a total of 2565 participants) met our inclusion criteria. Three additional papers examined the healthcare cost from one of the clinical trials. Fifteen trials compared dornase alfa to placebo or no dornase alfa treatment (2447 participants); two compared daily dornase to hypertonic saline (32 participants); one compared daily dornase alfa with hypertonic saline and alternate day dornase alfa (48 participants); one compared dornase alfa to mannitol and the combination of both drugs (38 participants). Trial duration varied from six days to three years.Compared to placebo, forced expiratory volume at one second improved in the intervention groups, with significant differences at one, three, six months and two years. There was also a significant improvement in lung clearance index at one month. There was a decrease in pulmonary exacerbations compared to placebo in trials of longer duration. The quality of the evidence from placebo-controlled trials was moderate to high for outcomes of lung function and pulmonary exacerbations. Limited, low quality evidence was available for changes in quality of life from baseline. One trial that examined the cost of care, including the cost of dornase alfa, found that the cost savings from dornase alfa offset 18{\%} to 38{\%} of the medication costs.The results for trials comparing dornase alfa to other medications that improve airway clearance (hypertonic saline or mannitol) were mixed, with one trial showing a greater improvement in forced expiratory volume at one second for dornase alfa compared to hypertonic saline, and three trials finding no difference between medications. In the only trial to assess the combination of dornase alfa with another medication compared to dornase alone, there was no benefit seen with the combination of dornase alfa and mannitol. Evidence of dornase alfa compared to other medications was limited and the open-label design of the trials may have induced bias, therefore the quality of the evidence was judged to be low.Dornase alfa did not cause significantly more adverse effects, except voice alteration and rash.$\backslash$n$\backslash$nAUTHORS' CONCLUSIONS: There is evidence to show that, compared with placebo, therapy with dornase alfa improves lung function in people with cystic fibrosis in trials lasting one month to two years. There was a decrease in pulmonary exacerbations in trials of six months or longer. Voice alteration and rash appear to be the only adverse events reported with increased frequency in randomised controlled trials. There is not enough evidence to firmly conclude if dornase alfa is superior to hyperosmolar agents in improving lung function.},
address = {Bethesda, Maryland},
author = {Thorogood, John},
booktitle = {Fisheries Research},
doi = {10.1002/14651858.CD001127.pub3},
editor = {Nielsen, L. A. and Johnson, D. L.},
isbn = {09-132-35008},
issn = {1469-493X},
number = {1},
pages = {84--85},
pmid = {27043279},
publisher = {American Fisheries Society},
title = {{Fisheries techniques}},
volume = {4},
year = {1986}
}
@article{Lagrue2015,
abstract = {Theory predicts the bottom–up coupling of resource and consumer densities, and epidemiological models make the same prediction for host–parasite interactions. Empirical evidence that spatial variation in local host density drives parasite population density remains scarce, however. We test the coupling of consumer (parasite) and resource (host) populations using data from 310 populations of metazoan parasites infecting invertebrates and fish in New Zealand lakes, spanning a range of transmission modes. Both parasite density (no. parasites per m2) and intensity of infection (no. parasites per infected hosts) were quantified for each parasite population, and related to host density, spatial variability in host density and transmission mode (egg ingestion, contact transmission or trophic transmission). The results show that dense and temporally stable host populations are exploited by denser and more stable parasite populations. For parasites with multi-host cycles, density of the ‘source' host did not matter: only density of the current host affected parasite density at a given life stage. For contact-transmitted parasites, intensity of infection decreased with increasing host density. Our results support the strong bottom–up coupling of consumer and resource densities, but also suggest that intraspecific competition among parasites may be weaker when hosts are abundant: high host density promotes greater parasite population density, but also reduces the number of conspecific parasites per individual host.},
author = {Lagrue, Cl{\'{e}}ment and Poulin, Robert},
doi = {10.1111/oik.02164},
issn = {16000706},
journal = {Oikos},
number = {12},
pages = {1639--1647},
title = {{Bottom-up regulation of parasite population densities in freshwater ecosystems}},
volume = {124},
year = {2015}
}
@incollection{Newman2010,
abstract = {NIH's Fogarty International Center has provided grants for the development of training programs in international research ethics for low- and middle-income (LMIC) professionals since 2000. Drawing on 12 years of research ethics training experience, a group of Fogarty grantees, trainees, and other ethics experts sought to map the current capacity and need for research ethics in LMICs, analyze the lessons learned about teaching bioethics, and chart a way forward for research ethics training in a rapidly changing health research landscape. This collection of papers is the result.},
address = {Oxford},
archivePrefix = {arXiv},
arxivId = {1212.2425},
author = {Pahwa, S. and Youssef, M. and Scoglio, C.},
booktitle = {Understanding Complex Systems},
doi = {10.1007/978-3-319-03518-5-8},
eprint = {1212.2425},
isbn = {9783319035178},
issn = {18600832},
pages = {163--186},
pmid = {24782067},
publisher = {Oxford University Press},
title = {{Electrical networks: An introduction}},
year = {2014}
}
@article{Chen2008,
abstract = {Parasites are ubiquitous in ecological communities but it is only recently that they have been routinely included in food web studies. Using recently published data and the tool of network analysis, we elucidate features associated with the pattern of parasitism in ecological communities. First we show here that parasitism is non-random in food webs. Second we demonstrate that parasite diversity, the number of parasite species harboured by a host species, is related to the network position of a host species. Specifically, a host species with high parasite diversity tends to have a wide diet range, occupy a network position close to many prey species, or occupy a network position that can better accumulate resources from species at lower trophic levels. Lastly our results also suggest that a host species with higher vulnerability to predators, being at a network position close to many predatory species, or being involved in many different food chains, tends to be important in parasite transmission.},
author = {Bobrov, Leonid A. and Salnikov, Aleksei V.},
doi = {10.1111/j.1600-0706.2008.16607.x},
isbn = {0030-1299},
issn = {20739745},
journal = {Bylye Gody},
keywords = {Borrowing,Conquest,Environment,Flank,Mongols,Nomads,Tulgama},
number = {4},
pages = {786--797},
pmid = {7873},
title = {{Tactic "Tulgama" the art of war the Mongols XIII century}},
volume = {38},
year = {2015}
}
@article{Ness1999,
abstract = {The threespine stickleback fish, Gasterosteus aculeatus, is parasitized by the tapeworm Schistocephalus solidus in freshwater habitats. The competency of the parasite to infect a definitive host, one of many piscivorous birds, is associated with behavioral changes in parasitized stickleback that are likely to increase the chance of transmission to a definitive host. Over a limited geographical range, large tapeworm size is also associated with demelanization, a phenotypic change in the stickleback host that involves a dramatic loss of melanin in the skin and, simultaneously, a darkening of the eye. We demonstrate that stickleback harboring a worm large enough to be infective to a definitive host exhibit behavioral shifts likely to enhance transmission, but that these changes are substantially magnified in demelanized individuals, all of which were infected by large tapeworms. The results were similar whether we used a model of a bird flown over an aquarium, or a preserved trout moved to simulate attack under field conditions. Because changes in the levels of response to both kinds of predators were similar, we infer that behavioral modifications that enhance susceptibility to visually hunting predators that are definitive hosts also enhance susceptibility to visually hunting predators (e.g. trout) that are not. In lakes where predatory fishes are common, the impact on S. solidus transmission could be substantial. Although other studies have suggested that positive buoyancy, forcing infected fish to remain near the surface, is one cause of these behavioral shifts, we detected no differences in water column position among unparasitized and parasitized classes of stickleback. Instead, infected stickleback appeared to move sluggishly and were less likely to respond to simulated attacks than were uninfected fish, and the behavioral shift was most dramatic in demelanized stickleback. CR - Copyright {\&}{\#}169; 1999 Nordic Society Oikos},
author = {Ness, Joshua H J.H. and Foster, Susan A S.A.},
doi = {10.2307/3546798},
isbn = {0030-1299},
issn = {00301299},
journal = {Oikos},
number = {1},
pages = {127--134},
title = {{Parasite-associated phenotype modifications in threespine stickleback}},
url = {http://www.jstor.org/stable/3546798},
volume = {85},
year = {1999}
}
@article{Canard2014,
abstract = {Background A considerable number of patients seen in general outpatient clinics (GOPC) are known to suffer from psychiatric rather than physical disorders. Studies have shown that doctors working in these clinics have difficulty in making accurate ratings of mental health problems in their patients and have poor knowledge of psychiatric diagnosis. Accurate recognition of psychiatric symptoms in a patient is essential for specific diagnosis and successful management. There is a need for the use of an easy tool such as the12-item General Health Questionnaire (GHQ-12) for screening and identification of psychopathologies especially in a busy clinic setting like the GOPC. Aside from psychometric screening tools, patients' sociodemographic characteristics such as gender, age, marital status, occupation, education etc. have been found to be of value in predicting those at risk.Objectives This study seeks to correlate GHQ 'caseness' with sociodemographic factors and to compare physician diagnosis with GHQ diagnosis.Subjects and method Three-hundred and twenty-two respondents were recruited for the study by a systematic random sampling method. Using a cut off score of three on both the English and Efik translation versions of the GHQ-12, 'cases' and 'non-cases' generated were compared with the same classification as identified by the GOPC doctors. Identification rates for both groups were calculated and the coefficients determined using a two-by-two contingency table. Sociodemographic correlates were determined by statistical comparison of the classifications in both groups.Results Statistically significant differences in sociodemographic characteristics of respondents were found for age ($\chi$(2)=48.97; P {\textless}0.05) and education ($\chi$(2)=45.64; P=0.05) using their GHQ-12 scores, and for occupation ($\chi$(2)=37.90; P {\textless}0.05) among those seen by the GOPC doctors. A further comparison of identified 'cases' and 'non-cases' by doctors again revealed significant difference for age ($\chi$(2)=7.151; P {\textless}0.05). Sex as a sociodemographic characteristic showed no statistically significant difference though a greater percentage of females (57.3{\%}) were observed as 'high scorers' as compared to their male counterparts (42.7{\%}). The GHQ-12 identified 46.6{\%} 'cases' while the GOPC doctors identified 6.8{\%} among the attendees with a diagnostic sensitivity of 8{\%} and a specificity of 94{\%}, respectively.Conclusion Belonging to the 18-39 years age group, being employed and having less than 12 years of education were the patients' characteristics that suggested the likelihood of the presence of mental health problems.This study also revealed that despite the high proportion of psychiatric morbidity (46.6{\%}) in the GOPC of the University of Calabar Teaching Hospital (UCTH) rate of detection by the clinic doctors was low (6.8{\%}).It is recommended that primary care doctors should be alerted to the possibility that clinically significant psychiatric morbidity may be present in GOPC attendees. The correlation between patients' sociodemographic parameters and presence of mental health problems could be informative and should be given adequate attention during consultation.},
author = {Asibong, Udeme E. and Udonwa, N. E. and Okokon, I. B. and Gyuse, A. N. and Aluka, T. and Ekpe, E. E.},
doi = {10.1086/675363},
isbn = {1756-8358 (Electronic)$\backslash$r1756-834X (Linking)},
issn = {1756834X},
journal = {Mental Health in Family Medicine},
keywords = {Mental health,Patient characteristics,Primary care clinic},
number = {3},
pages = {169--177},
pmid = {22477938},
title = {{Patient characteristics that may predict the likelihood of the presence of mental health problems in patients attending the general outpatient clinic of a tertiary hospital in South-South Nigeria}},
url = {http://www.ncbi.nlm.nih.gov/pubmed/24642492},
volume = {7},
year = {2010}
}
@article{Choisy2003,
abstract = {Although multihost complex life cycles (CLCs) are common in several distantly related groups of parasites, their evolution remains poorly understood. In this article, we argue that under particular circumstances, adding a second host to a single-host life cycle is likely to enhance transmission (i.e., reaching the target host). For instance, in several situations, the propagules of a parasite exploiting a predator species will achieve a higher host-finding success by encysting in a prey of the target predator than by other dispersal modes. In such a case, selection should favor the transition from a single- to a two-host life cycle that includes the prey species as an intermediate host. We use an optimality model to explore this idea, and we discuss it in relation to dispersal strategies known among free-living species, especially animal dispersal. The model found that selection favored a complex life cycle only if intermediate hosts were more abundant than definitive hosts. The selective value of a complex life cycle increased with predation rates by definitive hosts on intermediate hosts. In exploring trade-offs between transmission strategies, we found that more costly trade-offs made it more difficult to evolve a CLC while less costly trade-offs between traits could favor a mixed strategy.},
author = {Choisy, Marc and Brown, Sam P. and Lafferty, Kevin D. and Thomas, Fr{\'{e}}d{\'{e}}ric},
doi = {10.1086/375681},
isbn = {0003-0147},
issn = {0003-0147},
journal = {The American Naturalist},
keywords = {162,172,181,2003,2003 by the university,am,complex life cycle,corresponding author,dispersal,e-mail,fr,fthomas,ird,life-history strategy,mpl,nat,of chicago,parasites,pp,predation,trophic transmission,vol},
number = {2},
pages = {172--181},
pmid = {12858262},
title = {{Evolution of Trophic Transmission in Parasites: Why Add Intermediate Hosts?}},
url = {http://www.journals.uchicago.edu/doi/10.1086/375681},
volume = {162},
year = {2003}
}
@article{Rossiter2011,
abstract = {The directionality of asymmetric interactions between predators (definitive hosts) and prey (intermediate hosts) should impact trophic transmission in parasites. This study tests the prediction that trophically transmitted parasites are funneled towards asymmetric predator-prey interactions where intermediate hosts have few predators and definitive hosts feed upon many prey ('downward asymmetry'). The distribution of trophically transmitted parasites was examined in four published food webs in relation to mismatch asymmetry of predator-prey interactions. We found that trophically transmitted parasites exploit downwardly asymmetric interactions in a nonrandom manner, and particular predator-prey pairs contain more trophically transmitted parasites than would be expected by random chance alone. These findings suggest that food web topology has great bearing on the ecology of trophically transmitted parasites, and that consideration of parasite life cycles in the context of food web organization can provide insights into the forces affecting the evolution of trophic transmission.},
author = {Rossiter, Wayne and Sukhdeo, Michael V.K.},
doi = {10.1111/j.1600-0706.2010.19077.x},
isbn = {0030-1299},
issn = {00301299},
journal = {Oikos},
number = {4},
pages = {607--614},
title = {{Exploitation of asymmetric predator-prey interactions by trophically transmitted parasites}},
volume = {120},
year = {2011}
}
@article{Valtonen2010,
abstract = {Parasites that are transmitted through predator-prey interactions may be used as indicators of trophic relationships between organisms. Yet, they are rarely used as such in the construction of topological (predator-prey) food webs. We constructed food webs of vertebrate trophic interactions using observed diet alone, trophically transmitted parasites alone, and the combination of the two based on data from 31 species of fish from the Bothnian Bay, Finland. The fish food web contained 530 links derived from observed diet, 724 links inferred from parasitism, and 1,058 links calculated from a combination of both stomach contents and parasites. This sub-web constructed from stomach contents had a mean of 17.1 links per fish species, while that using parasites had 23.4 links per fish. Combining the two diet indicators yielded 34.1 links per fish species, illustrating the complementarity of the two methods. Mean number of prey species per fish species was 12.5 using observed diet items, 15.8 using parasites, and 24.5 using both measures together. Mean number of predators per fish species was 7.4 using observed diet, 11.7 using parasites and 15.0 using both. A positive correlation was found between the mean number of parasites and the number of prey taxa in the diet among the fishes. Omnivorous fish had the highest diversity of both parasite species and prey items, while benthophagous fish had among the lowest. Mean total abundance and mean total prevalence of parasites correlated positively with fish size, with piscivores being the largest with the highest abundance and prevalence, while planktivores and benthivores had the lowest. Trophically transmitted parasites may be used to help construct vertebrate sub-webs and derive information about food web processes. Parasites alone provided equivalent if not more information than observed diet. However, resolution is improved by using parasites and observed diet together.},
author = {Valtonen, E. T. and Marcogliese, David J. and Julkunen, Markku},
doi = {10.1007/s00442-009-1451-5},
isbn = {0029-8549},
issn = {00298549},
journal = {Oecologia},
keywords = {Helminths,Trophic interactions,Trophic transmission},
number = {1},
pages = {139--152},
pmid = {19756761},
title = {{Vertebrate diets derived from trophically transmitted fish parasites in the Bothnian Bay}},
volume = {162},
year = {2010}
}
@article{Lefevre2009,
abstract = {The diversity of ways in which host manipulation by parasites interferes with ecological and evolutionary processes governing biotic interactions has been recently documented, and indicates that manipulative parasites are full participants in the functioning of ecosystems. Phenotypic alterations in parasitised hosts modify host population ecology, apparent competition processes, food web structure and energy and nutrient flow between habitats, as well as favouring habitat creation. As is usually the case in ecology, these phenomena can be greatly amplified by a series of secondary consequences (cascade effects). Here we review the ecological relevance of manipulative parasites in ecosystems and propose directions for further research. {\textcopyright} 2008 Elsevier Ltd. All rights reserved.},
author = {Lef{\`{e}}vre, Thierry and Lebarbenchon, Camille and Gauthier-Clerc, Michel and Miss{\'{e}}, Doroth{\'{e}}e and Poulin, Robert and Thomas, Fr{\'{e}}d{\'{e}}ric},
doi = {10.1016/j.tree.2008.08.007},
isbn = {0169-5347},
issn = {01695347},
journal = {Trends in Ecology and Evolution},
number = {1},
pages = {41--48},
pmid = {19026461},
title = {{The ecological significance of manipulative parasites}},
volume = {24},
year = {2009}
}
@article{Mouritsen2003,
abstract = {Parasites with complex life cycles, relying on trophic transmission to a definitive host, very often induce changes in the behaviour or appearance of their intermediate hosts. Because this usually makes the intermediate host vulnerable to predation by the definitive host, it is generally assumed that the parasite's transmission rate is increased, and that the modification of the host is, therefore, of great adaptive significance to the parasite. However, in the ecological "real world" other predators unsuitable as hosts may just as well take advantage of the facilitation process and significantly erode the benefit of host manipulation. Here we show that the intertidal New Zealand cockle (Austrovenus stutchburyi), manipulated by its echinostome trematode (Curtuteria australis) to rest on the sediment surface fully exposed to predation from the avian definitive host, is also subject to sublethal predation from a benthic feeding fish (Notolabrus celidotus, Labridae). The fish is targeting only the cockle-foot, in which the parasite preferentially encysts, reducing the infection intensity of manipulated cockles to levels comparable with those in non-manipulated, buried cockles. Based on the frequency and intensity of the foot cropping and predation rates on surfaced cockles by avian hosts, it is estimated that 2.5{\%} of the parasite population in manipulated cockles is transmitted successfully whereas 17.1{\%} is lost to fish. We argue that the adaptive significance of manipulation in the present system depends critically on the feeding behaviour of the definitive host. If cockles constitute the majority of prey items, there will be selection against manipulation. If manipulated cockles are taken as an easily accessible supplement to a diet composed mostly of other prey organisms, behavioural manipulation of the cockle host appears a high risk, high profit transmission strategy. Both these feeding behaviours of birds are known to occur in the field. {\textcopyright} 2003 Australian Society for Parasitology Inc. Published by Elsevier Ltd. All rights reserved.},
author = {Mouritsen, Kim N. and Poulin, Robert},
doi = {10.1016/S0020-7519(03)00178-4},
isbn = {0020-7519 (Print)$\backslash$r0020-7519 (Linking)},
issn = {00207519},
journal = {International Journal for Parasitology},
keywords = {Echinostomatidae,Host manipulation,Intertidal bivalve,Intramolluscan metacercariae,Parasitism,Partial predation,Transmission success},
number = {10},
pages = {1043--1050},
pmid = {13129526},
title = {{Parasite-induced trophic facilitation exploited by a non-host predator: A manipulator's nightmare}},
volume = {33},
year = {2003}
}
@article{Miura2006,
abstract = {By modifying the behaviour and morphology of hosts, parasites may strongly impact host individuals, populations and communities. We examined the effects of a common trematode parasite on its snail host, Batillaria cumingi (Batillariidae). This widespread snail is usually the most abundant invertebrate in salt marshes and mudflats of the northeastern coast of Asia. More than half (52.6{\%}, n=1360) of the snails in our study were infected. We found that snails living in the lower intertidal zone were markedly larger and exhibited different shell morphology than those in the upper intertidal zone. The large morphotypes in the lower tidal zone were all infected by the trematode, Cercaria batillariae (Heterophyidae). We used a transplant experiment, a mark-and-recapture experiment and stable carbon isotope ratios to reveal that snails infected by the trematode move to the lower intertidal zone, resume growth after maturation and consume different resources. By simultaneously changing the morphology and behaviour of individual hosts, this parasite alters the demographics and potentially modifies resource use of the snail population. Since trematodes are common and often abundant in marine and freshwater habitats throughout the world, their effects potentially alter food webs in many systems.},
author = {Miura, Osamu and Kuris, Armand M. and Torchin, Mark E. and Hechinger, Ryan F. and Chiba, Satoshi},
doi = {10.1098/rspb.2005.3451},
isbn = {0962-8452},
issn = {14712970},
journal = {Proceedings of the Royal Society B: Biological Sciences},
keywords = {Behavioural modification,Gigantism,Host,Parasite,Stable carbon isotope ratio},
number = {1592},
pages = {1323--1328},
pmid = {16777719},
title = {{Parasites alter host phenotype and may create a new ecological niche for snail hosts}},
url = {http://www.pubmedcentral.nih.gov/articlerender.fcgi?artid=1560305{\&}tool=pmcentrez{\&}rendertype=abstract},
volume = {273},
year = {2006}
}
@article{Cirtwill2016,
abstract = {{\textless}p{\textgreater}Variations in levels of parasitism among individuals in a population of hosts underpin the importance of parasites as an evolutionary or ecological force. Factors influencing parasite richness (number of parasite species) and load (abundance and biomass) at the individual host level ultimately form the basis of parasite infection patterns. In fish, diet range (number of prey taxa consumed) and prey selectivity (proportion of a particular prey taxon in the diet) have been shown to influence parasite infection levels. However, fish diet is most often characterized at the species or fish population level, thus ignoring variation among conspecific individuals and its potential effects on infection patterns among individuals. Here, we examined parasite infections and stomach contents of New Zealand freshwater fish at the individual level. We tested for potential links between the richness, abundance and biomass of helminth parasites and the diet range and prey selectivity of individual fish hosts. There was no obvious link between individual fish host diet and helminth infection levels. Our results were consistent across multiple fish host and parasite species and contrast with those of earlier studies in which fish diet and parasite infection were linked, hinting at a true disconnect between host diet and measures of parasite infections in our study systems. This absence of relationship between host diet and infection levels may be due to the relatively low richness of freshwater helminth parasites in New Zealand and high host–parasite specificity.{\textless}/p{\textgreater}},
author = {Cirtwill, Alyssa R. and Stouffer, Daniel B. and Poulin, Robert and Lagrue, Cl{\'{e}}ment},
doi = {10.1017/S003118201500150X},
isbn = {0031182015001},
issn = {14698161},
journal = {Parasitology},
keywords = {fish diet,helminth parasites,individual host,infection levels,transmission mode},
number = {1},
pages = {75--86},
pmid = {16185695},
title = {{Are parasite richness and abundance linked to prey species richness and individual feeding preferences in fish hosts?}},
url = {http://www.journals.cambridge.org/abstract{\_}S003118201500150X},
volume = {143},
year = {2015}
}
@article{Cirtwill2015SI,
author = {Cirtwill, Alyssa R and Stouffer, Daniel B},
journal = {Journal of Animal Ecology},
pages = {1--19},
title = {{Concomitant predation on parasites is highly variable but constrains ... Supporting Information Materials {\{}{\&}{\}} Methods}},
year = {2015}
}
@book{energy,
author = {Rizzo, Maria L. and Szekely, Gabor L.},
edition = {R package},
title = {{energy: E-statistics (energy statistics)}},
year = {2014}
}
@article{Brose2016,
abstract = {Understanding the consequences of species loss in complex ecological communities is one of the great challenges in current biodiversity research. For a long time, this topic has been addressed by traditional biodiversity experiments. Most of these approaches treat species as trait-free, taxonomic units characterizing communities only by species number without accounting for species traits. However, extinctions do not occur at random as there is a clear correlation between extinction risk and species traits. In this review, we assume that large species will be most threatened by extinction and use novel allometric and size-spectrum concepts that include body mass as a primary species trait at the levels of populations and individuals, respectively, to re-assess three classic debates on the relationships between biodiversity and (i) food-web structural complexity, (ii) community dynamic stability, and (iii) ecosystem functioning. Contrasting current expectations, size-structured approaches suggest that the loss of large species, that typically exploit most resource species, may lead to future food webs that are less interwoven and more structured by chains of interactions and compartments. The disruption of natural body-mass distributions maintaining food-web stability may trigger avalanches of secondary extinctions and strong trophic cascades with expected knock-on effects on the functionality of the ecosystems. Therefore, we argue that it is crucial to take into account body size as a species trait when analysing the consequences of biodiversity loss for natural ecosystems. Applying size-structured approaches provides an integrative ecological concept that enables a better understanding of each species' unique role across communities and the causes and consequences of biodiversity loss.},
author = {Brose, Ulrich and Blanchard, Julia L. and Ekl{\"{o}}f, Anna and Galiana, Nuria and Hartvig, Martin and Hirt, Myriam R. and Kalinkat, Gregor and Nordstr{\"{o}}m, Marie C. and O'gorman, Eoin J. and Rall, Bj{\"{o}}rn C. and Schneider, Florian D. and Th{\'{e}}bault, Elisa and Jacob, Ute},
doi = {10.1111/brv.12250},
isbn = {1469-185X},
issn = {1469185X},
journal = {Biological Reviews},
keywords = {Allometric scaling,Biodiversity,Complexity,Ecosystem functioning,Extinctions,Food webs,Global change,Size spectrum,Stability},
number = {2},
pages = {684--697},
pmid = {26756137},
title = {{Predicting the consequences of species loss using size-structured biodiversity approaches}},
volume = {92},
year = {2017}
}
@article{Murray2000,
abstract = {n.a.},
author = {Murray, Bertram G.},
doi = {10.1034/j.1600-0706.2000.890223.x},
isbn = {1600-0706},
issn = {00301299},
journal = {Oikos},
number = {2},
pages = {403--408},
title = {{Universal laws and predictive theory in ecology and evolution}},
url = {http://doi.wiley.com/10.1034/j.1600-0706.2000.890223.x},
volume = {89},
year = {2000}
}
@article{Lawton1999,
abstract = {The dictionary definition of a law is: "Generalized formulation based on a series of events or processes observed to recur regularly under certain conditions; a widely observable tendency". I argue that ecology has numerous laws in this sense of the word, in the form of widespread, repeatable patterns in nature, but hardly any laws that are universally true. Typically, in other words, ecological patterns and the laws, rules and mechanisms that underpin them are contingent on the organisms involved, and their environment. This contingency is manageable at a relatively simple level of ecological organisation (for example the population dynamics of single and small numbers of species), and re-emerges also in a manageable form in large sets of species, over large spatial scales, or over long time periods, in the form of detail-free statistical patterns - recently called 'macroecology'. The contingency becomes over- whelmingly complicated at intermediate scales, characteristic of community ecology, where there are a large number of case histories, and very little other than weak, fuzzy generalizations. These arguments are illustrated by focusing on examples of typical studies in community ecology, and by way of contrast, on the macroecological relationship that emerges between local species richness and the size of the regional pool of species. The emergent pattern illustrated by local vs regional richness plots is extremely simple, despite the vast number of contingent processes and interactions involved in its generation. To discover general patterns, laws and rules in nature, ecology may need to pay less attention to the 'middle ground' of community ecology, relying less on reductionism and experimental manipulation, but increasing research efforts into macroecolo},
archivePrefix = {arXiv},
arxivId = {arXiv:1011.1669v3},
author = {Poulin, R.},
doi = {10.1017/S0031182006002150},
eprint = {arXiv:1011.1669v3},
isbn = {0031-1820},
issn = {00311820},
journal = {Parasitology},
keywords = {Aggregation,Biomass,Contingency,Interaction networks,Macroecology,Metabolism,Scale,Species richness},
number = {6},
pages = {763--776},
pmid = {17234043},
title = {{Are there general laws in parasite ecology?}},
volume = {134},
year = {2007}
}
@article{Poulin2007,
abstract = {The dictionary definition of a law is: "Generalized formulation based on a series of events or processes observed to recur regularly under certain conditions; a widely observable tendency". I argue that ecology has numerous laws in this sense of the word, in the form of widespread, repeatable patterns in nature, but hardly any laws that are universally true. Typically, in other words, ecological patterns and the laws, rules and mechanisms that underpin them are contingent on the organisms involved, and their environment. This contingency is manageable at a relatively simple level of ecological organisation (for example the population dynamics of single and small numbers of species), and re-emerges also in a manageable form in large sets of species, over large spatial scales, or over long time periods, in the form of detail-free statistical patterns - recently called 'macroecology'. The contingency becomes over- whelmingly complicated at intermediate scales, characteristic of community ecology, where there are a large number of case histories, and very little other than weak, fuzzy generalizations. These arguments are illustrated by focusing on examples of typical studies in community ecology, and by way of contrast, on the macroecological relationship that emerges between local species richness and the size of the regional pool of species. The emergent pattern illustrated by local vs regional richness plots is extremely simple, despite the vast number of contingent processes and interactions involved in its generation. To discover general patterns, laws and rules in nature, ecology may need to pay less attention to the 'middle ground' of community ecology, relying less on reductionism and experimental manipulation, but increasing research efforts into macroecolo},
archivePrefix = {arXiv},
arxivId = {arXiv:1011.1669v3},
author = {Poulin, R.},
doi = {10.1017/S0031182006002150},
eprint = {arXiv:1011.1669v3},
isbn = {0031-1820},
issn = {00311820},
journal = {Parasitology},
keywords = {Aggregation,Biomass,Contingency,Interaction networks,Macroecology,Metabolism,Scale,Species richness},
number = {6},
pages = {763--776},
pmid = {17234043},
title = {{Are there general laws in parasite ecology?}},
volume = {134},
year = {2007}
}
@article{Turchin2001,
abstract = {Turchin, P. 2001. Does population ecology have general laws? – Oikos 94: 17 – 26. There is a widespread opinion among ecologists that ecology lacks general laws. In this paper I argue that this opinion is mistaken. Taking the case of population dynamics, I point out that there are several very general law-like propositions that provide the theoretical basis for most population dynamics models that were devel-oped to address specific issues. Some of these foundational principles, like the law of exponential growth, are logically very similar to certain laws of physics (Newton's law of inertia, for example, is almost a direct analogue of exponential growth). I discuss two other principles (population self-limitation and resource-consumer oscil-lations), as well as the more elementary postulates that underlie them. None of the ''laws'' that I propose for population ecology are new. Collectively ecologists have been using these general principles in guiding development of their models and experiments since the days of Lotka, Volterra, and Gause.},
archivePrefix = {arXiv},
arxivId = {2052},
author = {Turchin, Peter},
doi = {10.1034/j.1600-0706.2001.11310.x},
eprint = {2052},
isbn = {0030-1299},
issn = {00444596},
journal = {Zhurnal Obshchei Biologii},
number = {1},
pages = {13--14},
pmid = {11881213},
title = {{Does Population Ecology Have General Laws?}},
volume = {63},
year = {2002}
}
@article{Nogales2015,
abstract = {  Aim  Mutualistic network parameters, such as modularity and nestedness, show non-random linkage patterns. Both increase network stability in different ways. Modularity hampers extinction cascades, whereas nestedness resists network disassembly. We explore these parameters in seed-dispersal networks in two archipelagos and the significance of life history, habitat, geography and phylogeny as drivers of linkage patterns and the applicability of modules as biogeographical entities.   Location  Canaries (Atlantic Ocean) and Gal{\'{a}}pagos (Pacific Ocean).   Methods  We compiled data on plant–seed disperser interactions from own observations and the literature, estimated network parameters describing interaction patterns (connectance, nestedness and modularity) and constructed a backbone phylogeny for the analyses.   Results  The Canarian network was highly nested but weakly modular, whereas the Gal{\'{a}}pagos network showed the opposite characteristics. Most key network species are native and have a favourable conservation status. Modularity in the Canaries is correlated with habitats (indirectly affected by altitude and orientation), whereas in the Gal{\'{a}}pagos it mainly reflects the functional roles of species.   Main conclusions  The divergent link patterns for the archipelagos imply that the highly nested Canarian network is stable against disassembly, whereas the modular Gal{\'{a}}pagos network may show strong resistance against extinction cascades. This difference may be driven by the specific evolutionary dynamics on the archipelagos. },
author = {Nogales, M. and Heleno, R. and Rumeu, B. and Gonz{\'{a}}lez-Castro, A. and Traveset, A. and Vargas, P. and Olesen, J. M.},
doi = {10.1111/geb.12315},
isbn = {1466-822X},
issn = {14668238},
journal = {Global Ecology and Biogeography},
keywords = {Animal–plant interaction,fleshy fruit,food web,frugivory,insular network,modularity},
number = {7},
pages = {912--922},
title = {{Seed-dispersal networks on the Canaries and the Gal{\'{a}}pagos archipelagos: interaction modules as biogeographical entities}},
volume = {25},
year = {2016}
}
@article{Polis1989,
abstract = {Interactions between species are usually categorized as either competition predation /parasitism mutualism commensalism 0), or amensalism (-0). Intraguild predation (IGP) is a combination of the first two, that is, the killing and eating of species that use similar, often ...},
author = {Nathroy, A. and Jhalani, S. C. and Debnath, C. R.},
doi = {10.1146/annurev.es.20.110189.001501},
isbn = {0066-4162},
issn = {03777537},
journal = {Man-made Textiles in India},
number = {1},
pages = {15--18},
pmid = {247},
title = {{Studies on nonwoven needle-punched dust filter. VII}},
volume = {33},
year = {1990}
}
@incollection{Holt2001,
abstract = {Indirect interactions are almost inevitable in any multi-species community. Understanding the implications of such interactions is a challenging task, in light of the very large number of ways species can be tied together in complex food webs. One approach to this complexity is to focus on strong interactions among a relatively small number (e.g. 3-6) of species interacting in defined configurations: community modules. In recent years, the discipline of community ecology has developed a substantial body of theory focused on such modules. Modules often clearly describe the basic features of empirical systems, particularly in simplified anthropogenic landscapes, and also help to isolate and characterize key processes driving the dynamics of more complex communities. In this chapter, we draw out a number of insights from ecological studies of modules which we believe are relevant to biological control. We emphasize in particular the module of 'shared predation', where a natural enemy attacks two or more species of prey. Theoretical studies suggest a number of 'rules of thumb', including: (i) the greatest risk to non-targets may occur from control agents that are only moderately effective on the target; (ii) targets with a high reproductive capacity can indirectly endanger non-targets; (iii) there can be transient phases of extinction risk for non-targets during the establishment phase of control agents, particularly for species with high attack rates; (iv) at a landscape scale, mobile agents can endanger the fate of non-targets at sites other than the area of control; (v) using specialist natural enemies can pose risks to non-targets, if there are generalist resident predators/ parasitoids which can exploit these introduced agents. The theoretical models help to highlight circumstances when these effects should be particularly strong.},
address = {Oxford, UK},
author = {Holt, Robert D. and Holt, Robert D. and Hochberg, Michael E. and Hochberg, Michael E.},
booktitle = {Evaluating Indirect Ecological Effects of Biological Control},
doi = {10.1670/09-040},
editor = {Waijnberg, E. and Scott, J. K. and Quimby, P. C.},
isbn = {0-85199-453-9},
issn = {0022-1511},
number = {iii},
pages = {13--37},
pmid = {28545514},
publisher = {CAB International},
title = {{Indirect Interactions, Community Modules and Biological Control: a Theoretical Perspective}},
year = {2001}
}
@incollection{Holt1997,
abstract = {17. RD HOLT Department of Systematics and Ecology, Museum of Natural History, Dyche Hall, University of Kansas, Lawrence, Kansas 66045, USA INTRODUCTION Ecological communities are among the most complex entities studied by scientists, not},
address = {Oxford, UK},
author = {R.D., Holt},
booktitle = {Multitrophic Interactions in Terrestrial Systems},
editor = {Gange, A. C. and Brown, V. K.},
pages = {333--350},
publisher = {Blackwell Science},
title = {{Community Modules}},
year = {1997}
}
@article{Brose2008,
abstract = {* 1In natural communities, populations are linked by feeding interactions that make up complex food webs. The stability of these complex networks is critically dependent on the distribution of energy fluxes across these feeding links. * 2In laboratory experiments with predatory beetles and spiders, we studied the allometric scaling (body-mass dependence) of metabolism and per capita consumption at the level of predator individuals and per link energy fluxes at the level of feeding links. * 3Despite clear power-law scaling of the metabolic and per capita consumption rates with predator body mass, the per link predation rates on individual prey followed hump-shaped relationships with the predator–prey body mass ratios. These results contrast with the current metabolic paradigm, and find better support in foraging theory. * 4This suggests that per link energy fluxes from prey populations to predator individuals peak at intermediate body mass ratios, and total energy fluxes from prey to predator populations decrease monotonically with predator and prey mass. Surprisingly, contrary to predictions of metabolic models, this suggests that for any prey species, the per link and total energy fluxes to its largest predators are smaller than those to predators of intermediate body size. * 5An integration of metabolic and foraging theory may enable a quantitative and predictive understanding of energy flux distributions in natural food webs.},
author = {Brose, U. and Ehnes, R. B. and Rall, B. C. and Vucic-Pestic, O. and Berlow, E. L. and Scheu, S.},
doi = {10.1111/j.1365-2656.2008.01408.x},
file = {::},
isbn = {1365-2656},
issn = {00218790},
journal = {Journal of Animal Ecology},
keywords = {Allometric scaling,Biological control,Body size,Food webs,Ingestion rates,Interaction strength,Metabolic theory,Top-down control},
number = {5},
pages = {1072--1078},
pmid = {18540967},
title = {{Foraging theory predicts predator-prey energy fluxes}},
volume = {77},
year = {2008}
}
@article{Yguel2014,
abstract = {Neighboring plants within a local community may be separated by many millions of years of evolutionary history, potentially reducing enemy pressure by insect herbivores. However, it is not known how the evolutionary isolation of a plant affects the fitness of an insect herbivore living on such a plant, especially the herbivore's enemy pressure. Here, we suggest that evolutionary isolation of host plants may operate similarly as spatial isolation and reduce the enemy pressure per insect herbivore. We investigated the effect of the phylogenetic isolation of host trees on the pressure exerted by specialist and generalist enemies (parasitoids and birds) on ectophagous Lepidoptera and galling Hymenoptera. We found that the phylogenetic isolation of host trees decreases pressure by specialist enemies on these insect herbivores. In Lepidoptera, decreasing enemy pressure resulted from the density dependence of enemy attack, a mechanism often observed in herbivores. In contrast, in galling Hymenoptera, enemy pressure declined with the phylogenetic isolation of host trees per se, as well as with the parallel decline in leaf damage by non-galling insects. Our results suggest that plants that leave their phylogenetic ancestral neighborhood can trigger, partly through simple density-dependency, an enemy release and fitness increase of the few insect herbivores that succeed in tracking these plants.},
author = {Yguel, Benjamin and Bailey, Richard Ian and Villemant, Claire and Brault, Amaury and Jactel, Herv{\'{e}} and Prinzing, Andreas},
doi = {10.1007/s00442-014-3026-3},
file = {:Users/alyssacirtwill/Documents/Papers/Yguel et al.{\_}2014{\_}Oecologia.pdf:pdf},
isbn = {0029-8549},
issn = {00298549},
journal = {Oecologia},
keywords = {Community phylogeny,Macroevolution,Parasitism rate,Temperate forest,Trophic chain},
number = {2},
pages = {521--532},
pmid = {25052039},
title = {{Insect herbivores should follow plants escaping their relatives}},
volume = {176},
year = {2014}
}
@article{Yguel2011,
abstract = {Hosts belonging to the same species suffer dramatically different impacts from their natural enemies. This has been explained by host neighbourhood, that is, by surrounding host-species diversity or spatial separation between hosts. However, even spatially neighbouring hosts may be separated by many million years of evolutionary history, potentially reducing the establishment of natural enemies and their impact. We tested whether phylogenetic isolation of oak hosts from neighbouring trees within a forest canopy reduces phytophagy. We found that an increase in phylogenetic isolation by 100 million years corresponded to a 10-fold decline in phytophagy. This was not due to poorer living conditions for phytophages on phylogenetically isolated oaks. Neither species diversity of neighbouring trees nor spatial distance to the closest oak affected phytophagy. We suggest that reduced pressure by natural enemies is a major advantage for individuals within a host species that leave their ancestral niche and grow among distantly related species.},
author = {Yguel, Benjamin and Bailey, Richard and Tosh, N. Denise and Vialatte, Aude and Vasseur, Chlo{\'{e}} and Vitrac, Xavier and Jean, Frederic and Prinzing, Andreas},
doi = {10.1111/j.1461-0248.2011.01680.x},
isbn = {1461-023X},
issn = {1461023X},
journal = {Ecology Letters},
keywords = {Community phylogeny,Forest canopy,Insect herbivory,Intraspecific variation,Lepidoptera,Macroevolution,Plant-insect interactions,Quercus,Temperate forest},
number = {11},
pages = {1117--1124},
pmid = {21923895},
title = {{Phytophagy on phylogenetically isolated trees: Why hosts should escape their relatives}},
volume = {14},
year = {2011}
}
@article{Castagneyrol2014,
abstract = {1. Pest regulation is an important ecosystem service provided by biodiversity, as plants grow- ing in species-rich communities often experience associational resistance to herbivores. How- ever, little is known about the respective influence of the quantity and identity of associated species on herbivory in focal plants. 2. Using a meta-analysis to compare insect herbivory in pure and mixed forests, we specifi- cally tested the effects of the relative abundance of focal tree species and of phylogenetic dis- tance between focal and associated tree species on the magnitude of associational resistance. 3. Overall, insect herbivory was significantly lower in mixed forests, but the outcome varied greatly depending on the phylogenetic relatedness among tree species and the degree of herbi- vore feeding specialization. 4. Specialist herbivore damage or abundance was positively related to relative abundance of their host trees, regardless of the phylogenetic distance between host and associated tree species. 5. By contrast, tree diversity triggered associational resistance to generalist herbivores only when tree mixtures included tree species phylogenetically distant to the focal species. 6. Synthesis and applications. Our study demonstrates that the establishment of mixed forests per se is not sufficient to convey associational resistance to herbivores if the identity of tree species associated in mixtures is not taken into account. As a general rule, mixing phylogenet- ically more distinct tree species, such as mixtures of conifers and broadleaved trees, results in more effective reduction in herbivore damage.},
author = {Castagneyrol, Bastien and Jactel, Herv{\'{e}} and Vacher, Corinne and Brockerhoff, Eckehard G. and Koricheva, Julia},
doi = {10.1111/1365-2664.12175},
file = {:Users/alyssacirtwill/Documents/Papers/Castagneyrol et al.{\_}2014{\_}Journal of Applied Ecology.pdf:pdf},
isbn = {0021-8901},
issn = {00218901},
journal = {Journal of Applied Ecology},
keywords = {Associational resistance,Forest,Herbivores,Insects,Meta-analysis,Pest management,Tree diversity},
number = {1},
pages = {134--141},
title = {{Effects of plant phylogenetic diversity on herbivory depend on herbivore specialization}},
volume = {51},
year = {2014}
}
@article{Nolting2016,
abstract = {When a system has more than one stable state, how can the stability of these states be compared? This deceptively simple question has important consequences for ecosystems, because systems with alternative stable states can undergo dramatic regime shifts. The probability, frequency, duration, and dynamics of these shifts will all depend on the relative stability of the stable states. Unfortunately, the concept of stability in ecology has suffered from substantial confusion and this is particularly problematic for systems where stochastic perturbations can cause shifts between coexisting alternative stable states. A useful way to visualize stable states in stochastic systems is with a ball-in-cup diagram, in which the state of the system is represented as the position of a ball rolling on a surface, and the random perturbations can push the ball from one basin of attraction to another. The surface is determined by a potential function, which provides a natural stability metric. However, systems amenable to this representation, called gradient systems, are quite rare. As a result, the potential function is not widely used and other approaches based on linear stability analysis have become standard. Linear stability analysis is designed for local analysis of deterministic systems and, as we show, can produce a highly misleading picture of how the system will behave under continual, stochastic perturbations. In this paper, we show how the potential function can be generalized so that it can be applied broadly, employing a concept from stochastic analysis called the quasi-potential. Using three classic ecological models, we demonstrate that the quasi-potential provides a useful way to quantify stability in stochastic systems.},
archivePrefix = {arXiv},
arxivId = {1508.02088},
author = {Nolting, Ben C. and Abbott, Karen C.},
doi = {10.1890/15-1047.1},
eprint = {1508.02088},
isbn = {0012-9658},
issn = {00129658},
journal = {Ecology},
keywords = {Alternative stable states,Freidlin-wentzell,Hamilton-jacobi,Quasi-potential,Regime shifts,Resilience,Stochastic differential equations,Stochastic dynamics},
number = {4},
pages = {850--864},
pmid = {7189},
title = {{Balls, cups, and quasi-potentials: Quantifying stability in stochastic systems}},
volume = {97},
year = {2016}
}
@article{Hoye2013,
abstract = {Advancing phenology in response to global warming has been reported across biomes1,2, raising concerns about the temporal uncoupling of trophic interactions3,4. Concurrently, widely reported flower visitor declines have been linked to resource limitations5. Phenological responses in the Arctic have been shown to outpace responses from lower latitudes and recent studies suggest that differences between such responses for plants and their flower visitors could be particularly pronounced in the Arctic1,6. The evidence for phenological uncoupling is scant because relevant data sets are lacking7 or not available at a relevant spatial scale8. Here, we present evidence of a climate-associated shortening of the flowering season and a concomitant decline in flower visitor abundance based on a long-term, spatially replicated (1996–2009) data set from high-Arctic Greenland. A unique feature of the data set is the spatial and temporal overlap of independent observations of plant and insect phenology. The shortening of the flowering season arose through spatial variation in phenological responses to warming. The shorter flowering seasons may have played a role in the observed decline in flower visitor abundance. Our results demonstrate that the dramatic climatic changes currently taking place in the Arctic are strongly affecting individual species and ecological communities, with implications for trophic interactions. Topographic},
author = {H{\o}ye, Toke T. and Post, Eric and Schmidt, Niels M. and Tr{\o}jelsgaard, Kristian and Forchhammer, Mads C.},
doi = {10.1038/nclimate1909},
file = {:Users/alyssacirtwill/Downloads/10.1038@nclimate1909.pdf:pdf},
isbn = {1758-6798},
issn = {1758678X},
journal = {Nature Climate Change},
number = {8},
pages = {759--763},
publisher = {Nature Publishing Group},
title = {{Shorter flowering seasons and declining abundance of flower visitors in a warmer Arctic}},
url = {http://www.nature.com/doifinder/10.1038/nclimate1909},
volume = {3},
year = {2013}
}
@article{Larson2001,
abstract = {{\textless}p{\textgreater} The Diptera are the second most important order among flower-visiting (anthophilous) and flower-pollinating insects worldwide. Their taxonomic diversity ranges from Nematocera to Brachycera, including most families within the suborders. Especially important are Syrphidae, Bombyliidae, and Muscoidea. Other families, especially of small flies, are less appreciated and often overlooked for their associations with flowers. We have compiled records of their flower visitations to show that they may be more prevalent than usually thought. Our knowledge of anthophilous Diptera needs to be enhanced by future research concerning ( {\textless}italic{\textgreater}i{\textless}/italic{\textgreater} ) the significance of nocturnal Nematocera and acalypterate muscoids as pollinators, ( {\textless}italic{\textgreater}ii{\textless}/italic{\textgreater} ) the extent to which the relatively ineffective pollen-carrying ability of some taxa can be compensated by the abundance of individuals, and ( {\textless}italic{\textgreater}iii{\textless}/italic{\textgreater} ) the role of Diptera as pollinators of the first flowering plants (Angiospermae) by using phylogenetic and palaeontological evidence. Specializations in floral relationships involve the morphology of Diptera, especially of their mouthparts, nutritional requirements, and behaviour, as well as concomitant floral attributes. The South African flora has the most highly specialized relations with dipterous pollinators, but in arctic and alpine generalist fly–flower relations are important in pollination and fly nutrition. {\textless}/p{\textgreater}},
archivePrefix = {arXiv},
arxivId = {1508.05882},
author = {Larson, B. M.H. and Kevan, P. G. and Inouye, D. W.},
doi = {10.4039/Ent133439-4},
eprint = {1508.05882},
isbn = {0008-347X},
issn = {19183240},
journal = {Canadian Entomologist},
number = {4},
pages = {439--465},
pmid = {20726002},
title = {{Flies and flowers: Taxonomic diversity of anthophiles and pollinators}},
volume = {133},
year = {2001}
}
@article{Wirta2015,
abstract = {DNA sequences offer powerful tools for describing the members and interactions of natural communities. In this paper, we establish the to-date most comprehensive library of DNA barcodes for a terrestrial site, including all known macroscopic animals and vascular plants of an intensively-studied area of the High Arctic, the Zackenberg Valley in Northeast Greenland. To demonstrate its utility, we apply the library to identify nearly 20 000 arthropod individuals from two Malaise traps, each operated for two summers. Drawing on this material, we estimate the coverage of previous morphology-based species inventories, derive a snapshot of faunal turnover in space and time, and describe the abundance and phenology of species in the rapidly changing arctic environment. Overall, 403 terrestrial animal and 160 vascular plant species were recorded by morphology-based techniques. DNA barcodes (CO1) offered high resolution in discriminating among the local animal taxa, with 92{\%} of morphologically distinguishable taxa assigned to unique Barcode Index Numbers (BINs) and 93{\%} to monophyletic clusters. For vascular plants, resolution was lower, with 54{\%} of species forming monophyletic clusters based on barcode regions rbcLa and ITS2. Malaise catches revealed 122 BINs not detected by previous sampling and DNA barcoding. The insect community was dominated by a few highly abundant taxa. Even closely-related taxa differed in phenology, emphasizing the need for species-level resolution when describing ongoing shifts in arctic communities and ecosystems. The DNA barcode library now established for Zackenberg offers new scope for such explorations, and for the detailed dissection of interspecific interactions throughout the community. This article is protected by copyright. All rights reserved.},
archivePrefix = {arXiv},
arxivId = {arXiv:1011.1669v3},
author = {Wirta, H. and V{\'{a}}rkonyi, G. and Rasmussen, C. and Kaartinen, R. and Schmidt, N. M. and Hebert, P. D.N. and Bart{\'{a}}k, M. and Blagoev, G. and Disney, H. and Ertl, S. and Gjelstrup, P. and Gwiazdowicz, D. J. and Huld{\'{e}}n, L. and Ilmonen, J. and Jakovlev, J. and Jaschhof, M. and Kahanp{\"{a}}{\"{a}}, J. and Kankaanp{\"{a}}{\"{a}}, T. and Krogh, P. H. and Labbee, R. and Lettner, C. and Michelsen, V. and Nielsen, S. A. and Nielsen, T. R. and Paasivirta, L. and Pedersen, S. and Pohjoism{\"{a}}ki, J. and Salmela, J. and Vilkamaa, P. and V{\"{a}}re, H. and von Tschirnhaus, M. and Roslin, T.},
doi = {10.1111/1755-0998.12489},
eprint = {arXiv:1011.1669v3},
isbn = {1755-0998},
issn = {17550998},
journal = {Molecular Ecology Resources},
keywords = {Arthropod,DNA barcode library,Greenland,High arctic,Species diversity},
number = {3},
pages = {809--822},
pmid = {26602739},
title = {{Establishing a community-wide DNA barcode library as a new tool for arctic research}},
volume = {16},
year = {2016}
}
@article{Schmidt2016,
author = {Schmidt, Niels M. and Mosbacher, J. B. and Nielsen, P. S. and Rasmussen, Claus and H{\o}ye, Toke T. and Roslin, Tomas},
doi = {10.1111/oik.02986},
isbn = {1600-0587},
issn = {1600-0587},
journal = {Ecography},
number = {February},
pages = {1--3},
title = {{An ecological function in crisis? - shrinking temporal overlap between plant flowering and pollinator function in a warming Arctic.}},
volume = {39},
year = {2016}
}
@article{Hoye2008a,
abstract = {BACKGROUND: Ninety-one rodent plague epidemics have occurred in Lianghe county, Yunnan Province, China, between 1990 and 2006. This study aimed to identify predictors for the presence and abundance of small mammals in households of villages endemic for rodent plague in Lianghe county. RESULTS: Rattus flavipectus and Suncus murinus were the two species captured in 110 households. Keeping cats decreased the number of captures of R. flavipectus by one to two thirds and the chance of reported small mammal sightings in houses by 60 to 80{\%}. Food availability was associated with fewer captures. Keeping food in sacks decreased the small mammal captures, especially of S. murinus 4- to 8-fold. Vegetables grown around house and maize grown in the village reduced the captures of S. murinus and R. flavipectus by 73 and 45{\%}, respectively. An outside toilet and garbage piles near the house each reduced R. flavipectus captures by 39 and 37{\%}, respectively, while raising dogs and the presence of communal latrines in the village increased R. flavipectus captures by 76 and 110{\%} but were without detectable effect on small mammal sightings. Location adjacent to other houses increased captures 2-fold but reduced the chance of sightings to about half. In addition, raising ducks increased the chance of sighting small mammals 2.7-fold. Even after adjusting for these variables, households of the Dai had higher captures than those of the Han and other ethnic groups. CONCLUSION: Both species captures were reduced by availability of species-specific foods in the environment, whereas other predictors for capture of the two species differed. Other than the beneficial effect of cats, there were also discrepancies between the effects on small mammal captures and those on sightings. These differences should be considered during the implementation and interpretation of small mammal surveys.},
author = {Yin, Jia Xiang and Geater, Alan and Chongsuvivatwong, Virasakdi and Dong, Xing Qi and Du, Chun Hong and Zhong, You Hong and McNeil, Edward},
doi = {10.1186/1472-6785-8-8},
isbn = {1472-6785 (Electronic)},
issn = {14726785},
journal = {BMC Ecology},
pages = {8},
pmid = {19068139},
title = {{Predictors for presence and abundance of small mammals in households of villages endemic for commensal rodent plague in Yunnan Province, China}},
volume = {8},
year = {2008}
}
@article{Aldridge2011,
abstract = {This study compares the development of Kalanchoe brasiliensis and Kalanchoe pinnata, which are medicinal species known as "saiao" and "folha da fortuna" that are used interchangeably by the population for medicinal purposes. The experiment consisted of 20 plots/species planted in plastic bags with homogeneous substrate in a randomized design, which grown under light levels (25{\%}, 50{\%}, 70{\%}, full sunlight) at environment temperature, and a treatment under a plastic with greater temperature range than the external environment. It was obtained for K. pinnata a greater plant height, total length of sprouts, stems, production and dry matter content of leaves than that obtained for K. brasiliensis, which achieved higher averages only for the length of lateral branches. The species showed increases in height, which varied in inverse proportion to the light, and it was observed the influence of temperature in K. pinnata. The production and dry matter content of leaves were proportional to the luminosity; the same occurred in the thickness of leaves for K. brasiliensis. In the swelling index and Brix degree, K. brasiliensis showed higher averages than K. pinnata. In relation to the total content of flavonoids it was not observed significant differences for both species. The analyzed parameters showed the main differences in the agronomic development of the two species.},
author = {Cruz, Bruna P. and Chedier, Luciana M. and Fabri, Rodrigo L. and Pimenta, Daniel S.},
doi = {10.1111/j.1365-2745.2011.01826.x},
isbn = {1678-2690 (Electronic)$\backslash$r0001-3765 (Linking)},
issn = {00013765},
journal = {Anais da Academia Brasileira de Ciencias},
keywords = {Flavonoids,Kalanchoe,Luminosity,Mucilage},
number = {4},
pages = {1435--1441},
pmid = {22146966},
title = {{Chemical and agronomic development of kalanchoe brasiliensis camb. and kalanchoe pinnata (Lamk.) pers under light and temperature levels}},
volume = {83},
year = {2011}
}
@article{Post2008,
abstract = {Temporal advancement of resource availability by warming in seasonal environments can reduce reproductive success of vertebrates if their own reproductive phenology does not also advance with warming. Indirect evidence from large-scale analyses suggests, however, that migratory vertebrates might compensate for this by tracking phenological variation across landscapes. Results from our two-year warming experiment combined with seven years of observations of plant phenology and offspring production by caribou (Rangifer tarandus) in Greenland, however, contradict evidence from large-scale analyses. At spatial scales relevant to the foraging horizon of individual herbivores, spatial variability in plant phenology was reduced--not increased--by both experimental and observed warming. Concurrently, offspring production by female caribou declined with reductions in spatial variability in plant phenology. By highlighting the spatial dimension of trophic mismatch, these results reveal heretofore unexpected adverse consequences of climatic warming for herbivore population ecology.},
author = {Post, Eric and Pedersen, Christian and Wilmers, Christopher C. and Forchhammer, Mads C.},
doi = {10.1098/rspb.2008.0463},
isbn = {0962-8452},
issn = {14712970},
journal = {Proceedings of the Royal Society B: Biological Sciences},
keywords = {Caribou,Climate change,Global warming,Life history,Plant phenology,Population dynamics},
number = {1646},
pages = {2005--2013},
pmid = {18495618},
title = {{Warming, plant phenology and the spatial dimension of trophic mismatch for large herbivores}},
volume = {275},
year = {2008}
}
@incollection{Kudo2006,
abstract = {The time and pattern of flowering strongly influence the reproductive success of animal-pollinated plants by controlling the overlap of flowering with temporally variable abiotic and biotic conditions that affect mating and seed production. Various factors influence the ecology and evolution of flowering phenologies, including features of the mating environment during flowering, herbivory on flowers and developing seeds, the period available for seed development, and seed dispersal conditions. This diversity of influ- ences as well as spatial and temporal variation in flowering conditions complicate selection on flowering phenologies. Nevertheless, many examples demonstrate that flowering phenologies serve as adaptive reproductive strategies. In contrast, environments with brief growing periods may not allow evolutionary adjustment of flowering phenologies. In such cases, restrictions on mating imposed by the flowering phenology influence selection on other reproductive traits. Furthermore, variation in flowering time within and among populations caused by differences in abiotic environments imposes assortative mating, which can create genetic structure among environments. Based on studies of populations along snowmelt gra- dients, I demonstrate that site-specific flowering phenologies influence the quality and quantity of seed production, mating-system evolution, and spatial genetic structure of alpine plants. Finally, I discuss outstanding issues concerning the ecology and evolution of flowering phenologies, including the need to clarify the biological interactions on which selection acts, adequate evaluation of fitness, and experimental approaches that incorporate genetically determined phenological variation},
address = {Oxford},
author = {Kudo, Gaku},
booktitle = {Ecology and Evolution of Flowers},
chapter = {8},
editor = {Harder, L. D. and Barrett, S. C. H.},
isbn = {0198570864 9780198570868 0198570856 9780198570851},
pages = {139--158},
publisher = {Oxford University Press},
title = {{Flowering phenologies of animal- pollinated plants: reproductive strategies and agents of selection}},
url = {http://books.google.com/books?hl=en{\&}lr={\&}id=8uCAr3904yIC{\&}oi=fnd{\&}pg=PA139{\&}dq=Flowering+phenologies+of+animal-+pollinated+plants+:+reproductive+strategies+and+agents+of+selection{\&}ots=Xi1tsJwL9t{\&}sig=uH8tHW2VVBRfC4ytfRu50EftscM},
year = {2006}
}
@article{Memmott2007,
abstract = {Anthropogenic climate change is widely expected to drive species extinct by hampering individual survival and reproduction, by reducing the amount and accessibility of suitable habitat, or by eliminating other organisms that are essential to the species in question. Less well appreciated is the likelihood that climate change will directly disrupt or eliminate mutually beneficial (mutualistic) ecological interactions between species even before extinctions occur. We explored the potential disruption of a ubiquitous mutualistic interaction of terrestrial habitats, that between plants and their animal pollinators, via climate change. We used a highly resolved empirical network of interactions between 1420 pollinator and 429 plant species to simulate consequences of the phenological shifts that can be expected with a doubling of atmospheric CO(2). Depending on model assumptions, phenological shifts reduced the floral resources available to 17-50{\%} of all pollinator species, causing as much as half of the ancestral activity period of the animals to fall at times when no food plants were available. Reduced overlap between plants and pollinators also decreased diet breadth of the pollinators. The predicted result of these disruptions is the extinction of pollinators, plants and their crucial interactions.},
author = {Memmott, Jane and Craze, Paul G. and Waser, Nickolas M. and Price, Mary V.},
doi = {10.1111/j.1461-0248.2007.01061.x},
isbn = {1461-023X},
issn = {1461023X},
journal = {Ecology Letters},
keywords = {Climate,Insects,Network,Phenology,Plants,Pollination},
number = {8},
pages = {710--717},
pmid = {17594426},
title = {{Global warming and the disruption of plant-pollinator interactions}},
volume = {10},
year = {2007}
}
@article{Hulber2010,
abstract = {P{\textgreater} High alpine plants endure a cold climate with short growing seasons entailing severe consequences of an improper timing of development. Hence, their flowering phenology is expected to be rigorously controlled by climatic factors. We studied ten alpine plant species from habitats with early and late melting snow cover for 2 years and compared the synchronizing effect of temperature sums (TS), time of snowmelt (SM) and photoperiod (PH) on their flowering phenology. Intraseasonal and habitat-specific variation in the impact of these factors was analysed by comparing predictions of time-to-event models using linear mixed-effects models. Temperature was the overwhelming trigger of flowering phenology for all species. Its synchronizing effect was strongest at or shortly after flowering indicating the particular importance of phenological control of pollination. To some extent, this pattern masks the common trend of decreasing phenological responses to climatic changes from the beginning to the end of the growing season for lowland species. No carry-over effects were detected. As expected, the impact of photoperiod was weaker for snowbed species than for species inhabiting sites with early melting snow cover, while for temperature the reverse pattern was observed. Our findings provide strong evidence that alpine plants will respond quickly and directly to increasing temperature without considerable compensation due to photoperiodic control of phenology.},
author = {H{\"{u}}lber, Karl and Winkler, Manuela and Grabherr, Georg},
doi = {10.1111/j.1365-2435.2009.01645.x},
isbn = {1365-2435},
issn = {02698463},
journal = {Functional Ecology},
keywords = {Central European Alps,Climate warming,Snow cover duration,Temperature sum},
number = {2},
pages = {245--252},
pmid = {15591899},
title = {{Intraseasonal climate and habitat-specific variability controls the flowering phenology of high alpine plant species}},
volume = {24},
year = {2010}
}
@article{Wipf2009,
abstract = {Snow is an important environmental factor in alpine ecosystems, which influences plant phenology, growth and species composition in various ways. With current climate warming, the snow-to-rain ratio is decreasing, and the timing of snowmelt advancing. In a 2-year field experiment above treeline in the Swiss Alps, we investigated how a substantial decrease in snow depth and an earlier snowmelt affect plant phenology, growth, and reproduction of the four most abundant dwarf-shrub species in an alpine tundra community. By advancing the timing when plants started their growing season and thus lost their winter frost hardiness, earlier snowmelt also changed the number of low-temperature events they experienced while frost sensitive. This seemed to outweigh the positive effects of a longer growing season and hence, aboveground growth was reduced after advanced snowmelt in three of the four species studied. Only Loiseleuria procumbens, a specialist of wind exposed sites with little snow, benefited from an advanced snowmelt. We conclude that changes in the snow cover can have a wide range of species-specific effects on alpine tundra plants. Thus, changes in winter climate and snow cover characteristics should be taken into account when predicting climate change effects on alpine ecosystems.},
author = {Wipf, Sonja and Stoeckli, Veronika and Bebi, Peter},
doi = {10.1007/s10584-009-9546-x},
isbn = {0165-0009},
issn = {01650009},
journal = {Climatic Change},
number = {1-2},
pages = {105--121},
title = {{Winter climate change in alpine tundra: Plant responses to changes in snow depth and snowmelt timing}},
volume = {94},
year = {2009}
}
@article{Potts2010,
abstract = {Pollinators are a key component of global biodiversity, providing vital ecosystem services to crops and wild plants. There is clear evidence of recent declines in both wild and domesticated pollinators, and parallel declines in the plants that rely upon them. Here we describe the nature and extent of reported declines, and review the potential drivers of pollinator loss, including habitat loss and fragmentation, agrochemicals, pathogens, alien species, climate change and the interactions between them. Pollinator declines can result in loss of pollination services which have important negative ecological and economic impacts that could significantly affect the maintenance of wild plant diversity, wider ecosystem stability, crop production, food security and human welfare. {\textcopyright} 2010 Elsevier Ltd.},
archivePrefix = {arXiv},
arxivId = {http://links.jstor.org/sici?sici=0036-8075{\%}2819780324{\%}293{\%}3A199{\%}3A4335{\%}3C1302{\%}3ADITRFA{\%}3E2.0.CO{\%}3B2-2},
author = {Potts, Simon G. and Biesmeijer, Jacobus C. and Kremen, Claire and Neumann, Peter and Schweiger, Oliver and Kunin, William E.},
doi = {10.1016/j.tree.2010.01.007},
eprint = {/links.jstor.org/sici?sici=0036-8075{\%}2819780324{\%}293{\%}3A199{\%}3A4335{\%}3C1302{\%}3ADITRFA{\%}3E2.0.CO{\%}3B2-2},
isbn = {0169-5347},
issn = {01695347},
journal = {Trends in Ecology and Evolution},
number = {6},
pages = {345--353},
pmid = {20188434},
primaryClass = {http:},
publisher = {Elsevier Ltd},
title = {{Global pollinator declines: Trends, impacts and drivers}},
url = {http://dx.doi.org/10.1016/j.tree.2010.01.007},
volume = {25},
year = {2010}
}
@article{Hinkler2008,
abstract = {Snow-cover is of significance not only to climate but also to hydrological and ecological systems through its control of the insulating, reflective and water storage properties at the surface of the Earth. In summer, solar radiation is the most important factor in the arctic energy budget. However, the amount of energy available for driving the soil surface systems depends on the albedo of the surface. The albedo of snow is 60-90{\%}, whereas the albedo in snow-free areas is of the order of 10-20{\%}. The energy budget for a given region is therefore strongly coupled to the fraction of snow-covered ground. By use of different kinds of remote-sensing data (Landsat TM/ETM+, SPOT HRV and Digital Camera Images), the inter- and intra-annual snow-cover distribution in Zackenbergdalen has been monitored. Based on the digital camera data, a model of snow-cover depletion was developed to extend the time series of snow coverage (during melt-off) back in time. The model uses an interpolated data set, which includes information on melt energy and end-of-winter snow amounts, to calculate spatial snow-cover extent during the melting season. The results showed that the date with 50{\%} snow-cover in Zackenbergdalen fluctuated within a margin of approximately ±2 weeks around the mean, which was June 21. During the cold season at high latitudes, short-wave radiation is of minor importance in the energy budget, due to the low sun. Results show that during the winter season, heat advection caused by atmospheric circulation, such as the North Atlantic Oscillation (NAO), may be important in some periods, and that the sea ice conditions in the south-eastern part of the Greenland Sea might be crucial for the winter precipitation budget in Northeast Greenland; that is, significant correlations (R2 = 0.5-0.6) between snow accumulation at Zackenberg and sea ice cover to the east and southeast of Zackenberg was found, showing that reduced amounts of sea ice in the region may be likely to lead to increased snow precipitation. This effect may counteract the expected prolonged average length of the snow-free (growing) season in high-arctic Northeast Greenland in a future climate with higher temperatures. {\textcopyright} 2008 Elsevier Inc. All rights reserved.},
author = {Hinkler, J{\o}rgen and Hansen, Birger U. and Tamstorf, Mikkel P. and Sigsgaard, Charlotte and Petersen, Dorthe},
doi = {10.1016/S0065-2504(07)00008-6},
isbn = {9780123736659},
issn = {00652504},
journal = {Advances in Ecological Research},
number = {07},
pages = {175--195},
title = {{Snow and Snow-Cover in Central Northeast Greenland}},
volume = {40},
year = {2008}
}
@article{MillerRushing2010,
abstract = {Climate change is altering the phenology of species across the world, but what are the consequences of these phenological changes for the demography and population dynamics of species? Time-sensitive relationships, such as migration, breeding and predation, may be disrupted or altered, which may in turn alter the rates of reproduction and survival, leading some populations to decline and others to increase in abundance. However, finding evidence for disrupted relationships, or lack thereof, and their demographic effects, is difficult because the necessary detailed observational data are rare. Moreover, we do not know how sensitive species will generally be to phenological mismatches when they occur. Existing long-term studies provide preliminary data for analysing the phenology and demography of species in several locations. In many instances, though, observational protocols may need to be optimized to characterize timing-based multi-trophic interactions. As a basis for future research, we outline some of the key questions and approaches to improving our understanding of the relationships among phenology, demography and climate in a multi-trophic context. There are many challenges associated with this line of research, not the least of which is the need for detailed, long-term data on many organisms in a single system. However, we identify key questions that can be addressed with data that already exist and propose approaches that could guide future research.},
author = {Miller-Rushing, Abraham J. and H{\o}ye, Toke Thomas and Inouye, David W. and Post, Eric},
doi = {10.1098/rstb.2010.0148},
file = {:Users/alyssacirtwill/Documents/Papers/Miller-Rushing et al.{\_}2010{\_}Philosophical Transactions of the Royal Society B Biological Sciences.pdf:pdf},
isbn = {1471-2970 (Electronic)$\backslash$r0962-8436 (Linking)},
issn = {14712970},
journal = {Philosophical Transactions of the Royal Society B: Biological Sciences},
keywords = {Climate change,Demography,Global warming,Mismatch,Phenology},
number = {1555},
pages = {3177--3186},
pmid = {20819811},
title = {{The effects of phenological mismatches on demography}},
url = {http://europepmc.org/articles/PMC2981949},
volume = {365},
year = {2010}
}
@incollection{Settele2014,
abstract = {The six cores drilled on three ice fields across the Kibo caldera on top of Kilimanjaro range in age from a few hundred to over 11. ka. The longest record shows an early Holocene climate in equatorial East Africa that was warm and humid. Several abrupt climatic events occurred throughout the Holocene, the three largest include an extreme cold-weather event 5.2. ka, and two droughts 8.2 and 4.2. ka. The latter caused a dramatic reduction of what is currently the largest ice field. Like all other mountain glaciers in Africa, the ice fields on Kilimanjaro are rapidly disappearing, and if the current rate of retreat is maintained, the Furtw{\"{a}}ngler Glacier will disappear by 2018 and the remaining ice fields over the following decade, making the top of Kilimanjaro completely ice free for the first time in 11. ka.},
address = {Cambridge, United Kingdom and New York, USA},
archivePrefix = {arXiv},
arxivId = {1109.1006v1},
author = {Villet, Martin H.},
booktitle = {Forensic Entomology: International Dimensions and Frontiers},
chapter = {4},
doi = {10.1201/b18156},
editor = {Field, C.B. and Barros, V.R. and Dokken, D.J. and Mach, K.J. and Mastrandrea, M.D. and Bilir, T.E. and Chatterjee, M. and Ebi, K.L. and Estrada, Y.O. and Genova, R.C. and Girma, B. and Kissel, E.S. and Levy, A.N. and MacCracken, S. and Mastrandrea, P.R. and White, L.L.},
eprint = {1109.1006v1},
isbn = {9781466572416},
issn = {00019909},
pages = {161--171},
pmid = {21243213},
publisher = {Cambridge University Press},
title = {{Africa}},
year = {2015}
}
@article{Jean2016,
abstract = {Aim Our understanding of the ecology and biogeography of microbes, including those that cause human disease, lags behind that for larger species. Despite recent focus on the geographical distribution of viruses and bacteria, the overall environmental distribution of human pathogens and parasites on Earth remains incompletely understood. As islands have long inspired basic ecological insights, we aimed to assess whether the microorganisms that cause human disease in modern times follow patterns common to insular plants and animals. Location Global and regional. Methods Relying on the publically accessible GIDEON database, we use the spatial distribution of nearly 300 human parasites and pathogens across 66 island countries and territories to assess the current predictive value of the ‘equilibrium theory' of island biogeography. The relationships between species richness and (1) island surface area and (2) distance to the nearest mainland were investigated with linear regression, and ANCOVAs were used to test for differences in these relationships with respect to pathogen ecology and taxonomy. Results Pathogen species richness increases with island surface area and decreases with distance to the nearest mainland. The effect of area is more than 10 times lower than that usually reported for macroorganisms, but is greater than the effect of distance. The strongest relationships are for pathogens that are vector-borne, zoonotic (with humans as dead-end hosts) or protozoan. Main conclusion Our results support the basic predictions of the theory: disease diversity is a positive function of island area and a negative function of island isolation. However, differences in the effects of area, distance and pathogen ecology suggest that globalization, probably through human travel and the animal trade, has softened these relationships. Parasites that primarily target non-human species, whose distributions are more constrained by island life than are those restricted to human hosts, drive the island biogeography of human disease.},
author = {Waikhom, Sayanika Devi and Louis, Bengyella and Sharma, Chandradev K. and Kumari, Pushpa and Somkuwar, Bharat G. and Singh, Mohendro W. and Talukdar, Narayan C.},
doi = {10.1111/geb.12393},
isbn = {1466-822X},
issn = {14668238},
journal = {BioMed Research International},
keywords = {Disease ecology,Human,Infectious diseases,Island biogeography,Pathogen diversity,Species-area relationship},
number = {1},
pages = {107--116},
pmid = {24350255},
title = {{Grappling the high altitude for safe edible bamboo shoots with rich nutritional attributes and escaping cyanogenic toxicity}},
volume = {2013},
year = {2013}
}
@article{Parmesan2006,
abstract = {It is important to measure what images, opinions, and at- titudes people have developed toward robots and how they can be changed, from scientific and engineering perspectives. This paper reports results of social research on Japanese peo- ple's attitudes toward robots by using “the Negative Attitudes toward Robots Scale (NARS).” They revealed that attitudes toward robots differ depending on assumptions about robots such as their type and task, and there may be gender differ- ences associated with them. Based on the results, the paper then discusses how people's attitudes toward robots can be altered. Introduction},
author = {Nomura, Tatsuya and Suzuki, T and Kanda, T and Kato, K},
doi = {10.2307/annurev.ecolsys.37.091305.30000024},
isbn = {1577352912},
issn = {1543-592X},
journal = {Proc. AAAI-06 Workshop on Human Implications of Human-Robot Interaction},
keywords = {American Association for Artifici,Copyright {\textcopyright}2006},
number = {Chaplin 1991},
pages = {29--35},
pmid = {243038500023},
title = {{Altered attitudes of people toward robots: Investigation through the Negative Attitudes toward Robots Scale}},
url = {http://www.aaai.org/Papers/Workshops/2006/WS-06-09/WS06-09-006.pdf},
volume = {37},
year = {2006}
}
@article{Hegland2009,
abstract = {Climate warming affects the phenology, local abundance and large-scale distribution of plants and pollinators. Despite this, there is still limited knowledge of how elevated temperatures affect plant-pollinator mutualisms and how changed availability of mutualistic partners influences the persistence of interacting species. Here we review the evidence of climate warming effects on plants and pollinators and discuss how their interactions may be affected by increased temperatures. The onset of flowering in plants and first appearance dates of pollinators in several cases appear to advance linearly in response to recent temperature increases. Phenological responses to climate warming may therefore occur at parallel magnitudes in plants and pollinators, although considerable variation in responses across species should be expected. Despite the overall similarities in responses, a few studies have shown that climate warming may generate temporal mismatches among the mutualistic partners. Mismatches in pollination interactions are still rarely explored and their demographic consequences are largely unknown. Studies on multi-species plant-pollinator assemblages indicate that the overall structure of pollination networks probably are robust against perturbations caused by climate warming. We suggest potential ways of studying warming-caused mismatches and their consequences for plant-pollinator interactions, and highlight the strengths and limitations of such approaches.},
author = {Hegland, Stein Joar and Nielsen, Anders and L{\'{a}}zaro, Amparo and Bjerknes, Anne Line and Totland, {\O}rjan},
doi = {10.1111/j.1461-0248.2008.01269.x},
isbn = {1461-0248},
issn = {1461023X},
journal = {Ecology Letters},
keywords = {Abundance,Climate warming,Distribution,Global change,Interaction,Mismatch,Mutualism,Network,Phenology,Pollination},
number = {2},
pages = {184--195},
pmid = {19049509},
title = {{How does climate warming affect plant-pollinator interactions?}},
volume = {12},
year = {2009}
}
@article{Post2009,
abstract = {This paper presents abrief survey of the research literature on wildfire behavior and effects and assembles formulae and graphical computation aids based on selected theoretical and empirical models. The uses of mathematical fire behavior models are discussed, and the general capabilities and limitations of currently available models are outlined.},
author = {Lin, Da Sen},
doi = {10.1029/2010WR009341.Citation},
isbn = {0894192090303},
issn = {2073753X},
journal = {Journal of Research in Education Sciences},
keywords = {Four-year institutes of technology,Higher education,Multiple college entrance program,Technological and vocational education},
number = {3},
pages = {89--122},
title = {{Factors influencing students' choices in the multiple college entrance program: The case of four-year institutes of technology}},
url = {http://adsabs.harvard.edu/abs/2003AGUFMGC52A..06G},
volume = {55},
year = {2010}
}
@article{Rafferty2013,
abstract = {Ecological networks of two interacting guilds of species, such as flowering plants and pollinators, are common in nature, and studying their structure can yield insights into their resilience to environmental disturbances. Here we develop analytical methods for exploring the strengths of interactions within bipartite networks consisting of two guilds of phylogenetically related species. We then apply these methods to investigate the resilience of a plant-pollinator community to anticipated climate change. The methods allow the statistical assessment of, for example, whether closely related pollinators are more likely to visit plants with similar relative frequencies, and whether closely related pollinators tend to visit closely related plants. The methods can also incorporate trait information, allowing us to identify which plant traits are likely responsible for attracting different pollinators. These questions are important for our study of 14 prairie plants and their 22 insect pollinators. Over the last 70 years, six of the plants have advanced their flowering, while eight have not. When we experimentally forced earlier flowering times, five of the six advanced-flowering species experienced higher pollinator visitation rates, whereas only one of the eight other species had more visits; this network thus appears resilient to climate change, because those species with advanced flowering have ample pollinators earlier in the season. Using the methods developed here, we show that advanced-flowering plants did not have a distinct pollinator community from the other eight species. Furthermore, pollinator phylogeny did not explain pollinator community composition; closely related pollinators were not more likely to visit the same plant species. However, differences among pollinator communities visiting different plants were explained by plant height, floral color, and symmetry. As a result, closely related plants attracted similar numbers of pollinators. By parsing out characteristics that explain why plants share pollinators, we can identify plant species that likely share a common fate in a changing climate.},
author = {Rafferty, Nicole E. and Ives, Anthony R.},
doi = {10.1890/12-1948.1},
isbn = {0012-9658},
issn = {00129658},
journal = {Ecology},
keywords = {Climate change,Interaction network,Linear mixed models,Phenology,Phylogenetic signal,Plant-pollinator interactions},
number = {10},
pages = {2321--2333},
pmid = {24358717},
title = {{Phylogenetic trait-based analyses of ecological networks}},
volume = {94},
year = {2013}
}
@book{IPCC2014,
abstract = {The volume focuses on why climate change matters and is organized into two parts, devoted respectively to human and natural systems and regional aspects, incorporating results from the reports of Working Groups I and III. The volume addresses impacts that have already occurred and risks of future impacts, especially the way those risks change with the amount of climate change that occurs and with investments in adaptation to climate changes that cannot be avoided. For both past and future impacts, a core focus of the assessment is characterizing knowledge about vulnerability, the characteristics and interactions that make some events devastating, while others pass with little notice.},
address = {Cambridge, United Kingdom and New York, USA},
archivePrefix = {arXiv},
arxivId = {arXiv:1011.1669v3},
author = {Corporation, Dow Corning and Road, South Saginaw},
booktitle = {Climate Change 2014: Impacts, Adaptation, and Vulnerability.},
doi = {10.1017/CBO9781107415324.004},
editor = {Field, C. B. and Barros, Vicente R. and Dokken, David Jon and Mach, Katharine J. and Mastrandrea, Michael D. and Bilir, T. Eren and Chatterjee, Monalisa and Ebi, Kristie L. and Estrada, Yuka Otsuki and Genova, Robert C. and Girma, Betelhem and Kissel, Eric S. and Levy, Andrew N. and MacCracken, Sandy and Mastrandrea, Patricia R. and White, Leslie L.},
eprint = {arXiv:1011.1669v3},
isbn = {9781107641655},
issn = {1098-6596},
pages = {1--7},
pmid = {25246403},
publisher = {Cambridge University Press},
title = {{DOW CORNING CORPORATION DOW CORNING ( R ) 3-4241 DIELECTRIC TOUGH GEL KIT ( PART A information is below ) DOW CORNING CORPORATION DOW CORNING ( R ) 3-4241 DIELECTRIC TOUGH GEL KIT ( PART A information is below )}},
url = {https://www.ipcc.ch/pdf/assessment-report/ar5/wg2/WGIIAR5-FrontMatterA{\_}FINAL.pdf},
year = {2009}
}
@article{MillerStruttmann2015,
abstract = {Many coevolved species have precisely matched traits. For example, long-tongued bumblebees are well adapted for obtaining nectar from flowers with long petal tubes. Working at high altitude in Colorado, Miller-Struttmann et al. found that long-tongued bumblebees have decreased in number significantly over the past 40 years. Short-tongued species, which are able to feed on many types of flowers, are replacing them. This shift seems to be a direct result of warming summers reducing flower availability, making generalist bumblebees more successful than specialists and resulting in the disruption of long-held mutualisms.},
archivePrefix = {arXiv},
arxivId = {arXiv:1011.1669v3},
author = {Miller-Struttmann, Nicole E. and Geib, Jennifer C. and Franklin, James D. and Kevan, Peter G. and Holdo, Ricardo M. and Ebert-May, Diane and Lynn, Austin M. and Kettenbach, Jessica A. and Hedrick, Elizabeth and Galen, Candace},
doi = {10.1126/science.aab0868},
eprint = {arXiv:1011.1669v3},
isbn = {0036-8075},
issn = {10959203},
journal = {Science},
number = {6255},
pages = {1541--1544},
pmid = {26404836},
title = {{Functional mismatch in a bumble bee pollination mutualism under climate change}},
url = {http://www.sciencemag.org/cgi/doi/10.1126/science.aab0868},
volume = {349},
year = {2015}
}
@article{Petanidou2014,
abstract = {Recent anthropogenic climate change is strongly associated with average shifts toward earlier seasonal timing of activity (phenology) in temperate-zone species. Shifts in phenology have the potential to alter ecological interactions, to the detriment of one or more interacting species. Recent models predict that detrimental phenological mismatch may increasingly occur between plants and their pollinators. One way to test this prediction is to examine data from ecological communities that experience large annual weather fluctuations. Taking this approach, we analyzed interactions over a four-year period among 132 plant species and 665 pollinating insect species within a Mediterranean community. For each plant species we recorded onset and duration of flowering and number of pollinator species. Flowering onset varied among years, and a year of earlier flowering of a species tended to be a year of fewer species pollinating its flowers. This relationship was attributable principally to early-flowering species, suggesting that shifts toward earlier phenology driven by climate change may reduce pollination services due to phenological mismatch. Earlier flowering onset of a species also was associated with prolonged flowering duration, but it is not certain that this will counterbalance any negative effects of lower pollinator species richness on plant reproductive success. Among plants with different life histories, annuals were more severely affected by flowering-pollinator mismatches than perennials. Specialized plant species (those attracting a smaller number of pollinator species) did not experience disproportionate interannual fluctuations in phenology. Thus they do not appear to be faced with disproportionate fluctuations in pollinator species richness, contrary to the expectation that specialists are at greatest risk of losing mutualistic interactions because of climate change. {\textcopyright} 2014 Elsevier Masson SAS.},
author = {Petanidou, Theodora and Kallimanis, Athanasios S. and Sgardelis, Stefanos P. and Mazaris, Antonios D. and Pantis, John D. and Waser, Nickolas M.},
doi = {10.1016/j.actao.2014.06.001},
isbn = {1146-609X},
issn = {1146609X},
journal = {Acta Oecologica},
keywords = {Bees,Climate change,Flowering duration,Flowering onset,Mediterranean ecosystems,Phenology,Pollination,Specialization},
pages = {104--111},
publisher = {Elsevier Masson SAS},
title = {{Variable flowering phenology and pollinator use in a community suggest future phenological mismatch}},
url = {http://dx.doi.org/10.1016/j.actao.2014.06.001},
volume = {59},
year = {2014}
}
@article{Hoye2007,
abstract = {Despite uncertainties in the magnitude of expected global warming over the next century, one consistent feature of extant and projected changes is that Arctic environments are and will be exposed to the greatest warming [1]. Concomitant with such large abiotic changes, biological responses to warming at high northern latitudes are also expected to outpace those at lower latitudes. One of the clearest and most rapid signals of biological response to rising temperatures across an array of biomes has been shifts in species phenology [2-4], yet to date evidence for phenological responses to climate change has been presented from most biomes except the High Arctic [3]. Given the well-established consequences for population dynamics of shifts in the timing of life history events [5,6], it is essential that the High Arctic be represented in assessments of phenological response to climate change. Using the most comprehensive data set available from this region, we document extremely rapid climate-induced advancement of flowering, emergence and egg-laying in a wide array of species in a high-arctic ecosystem. The strong responses and the large variability within species and taxa illustrate how easily biological interactions may be disrupted by abiotic forcing, and how dramatic responses to climatic changes can be for arctic ecosystems. {\textcopyright} 2007 Elsevier Ltd. All rights reserved.},
author = {H{\o}ye, Toke T. and Post, Eric and Meltofte, Hans and Schmidt, Niels M. and Forchhammer, Mads C.},
doi = {10.1016/j.cub.2007.04.047},
file = {:Users/alyssacirtwill/Documents/Papers/H{\o}ye et al.{\_}2007{\_}Current Biology.pdf:pdf},
isbn = {1422-0067 (Electronic)$\backslash$r1422-0067 (Linking)},
issn = {09609822},
journal = {Current Biology},
number = {12},
pages = {449--451},
pmid = {24776758},
title = {{Rapid advancement of spring in the High Arctic}},
volume = {17},
year = {2007}
}
@article{Forrest2015,
abstract = {Climate change can aff ect plant – pollinator interactions in a variety of ways, but much of the research attention has focused on whether independent shifts in phenology will alter temporal overlap between plants and pollinators. Here I review the research on plant – pollinator mismatch, assessing the potential for observational and experimental approaches to address particular aspects of the problem. Recent, primarily observational studies suggest that phenologies of co-occurring plants and pollinators tend to respond similarly to environmental cues, but that nevertheless, certain pairs of interacting species are showing independent shifts in phenology. Only in a few cases, however, have these independent shifts been shown to aff ect population vital rates (specifi cally, seed production by plants) – but this largely refl ects a lack of research. Compared to the few long-term studies of pollination in natural plant populations, experimental manipulations of phenology have yielded relatively optimistic conclusions about eff ects of phenological shifts on plant reproduction, and I discuss how issues of scale and frequency-dependence in pollinator behaviour aff ect the interpretation of these ‘ temporal transplant ' experiments. Comparable research on the impacts of mismatch on pollinator populations is so far lacking, but both observational studies and focused experiments have the potential to improve our forecasts of pollinator responses to changing phenologies. Finally, while there is now evidence that plant – pollinator mismatch can aff ect seed production by plants, it is still unclear whether this phenological impact will be the primary way in which climate change aff ects plant – pollinator interactions. It would be useful to test the direct eff ects of changing climate on pollinator population persistence, and to compare the importance of phenological mismatch with other threats to pollination.},
author = {Forrest, Jessica R K},
doi = {10.1111/oik.01386},
isbn = {1600-0706},
issn = {16000706},
journal = {Oikos},
number = {1},
pages = {4--13},
pmid = {22349156},
title = {{Plant-pollinator interactions and phenological change: What can we learn about climate impacts from experiments and observations?}},
volume = {124},
year = {2015}
}
@article{Thackeray2010,
abstract = {Recent changes in the seasonal timing (phenology) of familiar biological events have been one of the most conspicuous signs of climate change. However, the lack of a standardized approach to analysing change has hampered assessment of consistency in such changes among different taxa and trophic levels and across freshwater, terrestrial and marine environments. We present a standardized assessment of 25 532 rates of phenological change for 726 UK terrestrial, freshwater and marine taxa. The majority of spring and summer events have advanced, and more rapidly than previously documented. Such consistency is indicative of shared large scale drivers. Furthermore, average rates of change have accelerated in a way that is consistent with observed warming trends. Less coherent patterns in some groups of organisms point to the agency of more local scale processes and multiple drivers. For the first time we show a broad scale signal of differential phenological change among trophic levels; across environments advances in timing were slowest for secondary consumers, thus heightening the potential risk of temporal mismatch in key trophic interactions. If current patterns and rates of phenological change are indicative of future trends, future climate warming may exacerbate trophic mismatching, further disrupting the functioning, persistence and resilience of many ecosystems and having a major impact on ecosystem services.},
author = {Thackeray, Stephen J. and Sparks, Timothy H. and Frederiksen, Morten and Burthe, Sarah and Bacon, Philip J. and Bell, James R. and Botham, Marc S. and Brereton, Tom M. and Bright, Paul W. and Carvalho, Laurence and Clutton-Brock, Tim and Dawson, Alistair and Edwards, Martin and Elliott, J. Malcolm and Harrington, Richard and Johns, David and Jones, Ian D. and Jones, James T. and Leech, David I. and Roy, David B. and Scott, W. Andy and Smith, Matt and Smithers, Richard J. and Winfield, Ian J. and Wanless, Sarah},
doi = {10.1111/j.1365-2486.2010.02165.x},
isbn = {1354-1013},
issn = {13541013},
journal = {Global Change Biology},
keywords = {Climate,Linear mixed effects models,Meta-analysis,Phenology,Traits,Trophic mismatch},
number = {12},
pages = {3304--3313},
title = {{Trophic level asynchrony in rates of phenological change for marine, freshwater and terrestrial environments}},
volume = {16},
year = {2010}
}
@article{Both2006,
abstract = {Dipterocarpaceae is the dominant tree family in the tropical rain$\backslash$nforests of Southeast Asia. Borneo (and Sarawak) is the centre of$\backslash$ndiversity for the dipterocarps. Identification of hotspots is important$\backslash$nfor forest and biodiversity conservation efforts. Species Occurrence$\backslash$nModels (SOMs) were generated for all 247 species of dipterocarps$\backslash$nrecorded in Sarawak using herbarium occurrence data and based on the$\backslash$nbest model selected. The species occurrence density map for each genus$\backslash$nand category (endemic and non endemic) was generated by overlaying the$\backslash$nSOMs of all species in each genus or category. The species occurrence$\backslash$ndensity maps were analyzed with land cover map from Landsat 7-EMT+$\backslash$nimages and protected forest areas for identifying hotspots for$\backslash$nconservation in Sarawak. Overlaying the SOM maps revealed that areas in$\backslash$ncentral Sarawak and the southwest region (northwest Borneo around$\backslash$nKuching) are the main hotspots of dipterocarp diversity in Sarawak while$\backslash$nthe coastal lowland areas in the lower Rejang and Baram River which are$\backslash$nmainly peat swamp forest are poorer in species occurrence density. In$\backslash$nterms of endemism, as with dipterocarp diversity, the mixed diptercarp$\backslash$nforest of central Sarawak is also the most important hotspot. Gap$\backslash$nanalysis revealed that most protected forest areas are in southwest$\backslash$nSarawak (Bako, Kubah, Tanjung Datu and Gunung Gading National Parks) and$\backslash$nin the northern part of Sarawak (Niah, Lambir Hills and Mt Mulu National$\backslash$nParks). This leaves the hotspot in the central part of Sarawak least$\backslash$nprotected. Protected areas only cover between 2 and 4{\%} of the total$\backslash$nareas for the different hotspots (75{\%} species density) while majority$\backslash$nof the hotspots that are still forested are outside the protected areas.},
author = {Teo, Stephen P. and Chai, Paul P.K. and Phua, Mui How},
doi = {10.1038/nature04539},
isbn = {0028-0836},
issn = {01266039},
journal = {Sains Malaysiana},
keywords = {Dipterocarps,Endemic,Non-endemic,Protected areas,Sarawak},
number = {9},
pages = {1237--1246},
pmid = {16672969},
title = {{Conservation gap analysis of dipterocarp hotspots in Sarawak using GIS, remote sensing and herbarium data}},
url = {http://www.nature.com/nature/journal/v441/n7089/abs/nature04539.html},
volume = {42},
year = {2013}
}
@incollection{IPCC2013,
abstract = {IPCC, 2013: Summary for Policymakers. In: Climate Change 2013: The Physical Science Basis. Contribution of Working Group I to the Fifth Assessment Report of the Intergovernmental Panel on Climate Chan ge [Stocker, T.F., D. Qin, G. - K. Plattner, M. Tignor, S.K. Allen, J. Bosc hung, A. Nauels, Y. Xia, V. Bex and P. M. Midgley (eds.)]. Cambridge University Press, Cambridge, United Kingdom and New York, NY, USA , pp. 1 – 30, doi:10.1017/CBO9781107415324.004.},
address = {Cambridge, United Kingdom and New York, USA},
archivePrefix = {arXiv},
arxivId = {arXiv:1011.1669v3},
booktitle = {Climate Change 2013: The Physical Science Basis. Contribution of Working Group I to the Fifth Assessment Report of the Intergovernmental Panel on Climate Change},
doi = {10.1017/CBO9781107415324.004},
editor = {Stocker, T. F. and Qin, D. and Plattner, G.-K. and Tignor, M. and Allen, S. K. and Boschung, J. and Nauels, A. and Xia, Y. and Bex, V. and Midgley, P. M.},
eprint = {arXiv:1011.1669v3},
isbn = {9788578110796},
issn = {01261962},
pages = {1--30},
pmid = {25246403},
publisher = {Cambridge University Press},
title = {{IPCC, 2013: Summary for Policymakers.}},
year = {2013}
}
@article{Dunne2002b,
abstract = {Networks from a wide range of physical, biological, and social systems have been recently described as ''small-world'' and ''scale-free.'' However, studies disagree whether ecological networks called food webs possess the characteristic path lengths, clustering coefficients, and degree distributions required for membership in these classes of networks. Our analysis suggests that the disagree-ments are based on selective use of relatively few food webs, as well as analytical decisions that obscure important variability in the data. We analyze a broad range of 16 high-quality food webs, with 25–172 nodes, from a variety of aquatic and terrestrial ecosystems. Food webs generally have much higher complexity, measured as connectance (the fraction of all possible links that are realized in a network), and much smaller size than other networks studied, which have important implications for network topology. Our results resolve prior conflicts by demonstrating that although some food webs have small-world and scale-free structure, most do not if they exceed a relatively low level of connectance. Although food-web degree distributions do not display a universal func-tional form, observed distributions are systematically related to network connectance and size. Also, although food webs often lack small-world structure because of low clustering, we identify a continuum of real-world networks including food webs whose ratios of observed to random clustering coefficients increase as a power–law function of network size over 7 orders of magnitude. Although food webs are generally not small-world, scale-free networks, food-web topology is consistent with patterns found within those classes of networks.},
archivePrefix = {arXiv},
arxivId = {cond-mat/0101396},
author = {Dunne, J. A. and Williams, R. J. and Martinez, N. D.},
doi = {10.1073/pnas.192407699},
eprint = {0101396},
isbn = {0027-8424 (Print)$\backslash$r0027-8424 (Linking)},
issn = {0027-8424},
journal = {Proceedings of the National Academy of Sciences},
number = {20},
pages = {12917--12922},
pmid = {12235364},
primaryClass = {cond-mat},
title = {{Food-web structure and network theory: The role of connectance and size}},
url = {http://www.pnas.org/cgi/doi/10.1073/pnas.192407699},
volume = {99},
year = {2002}
}
@article{Elias2013,
abstract = {Uncovering the processes that shape the architecture of interaction networks is a major challenge in ecology. Studies have consistently revealed that more closely related taxa tend to show greater overlap in interaction partners, fuelling the idea that interactions are phylogenetically conserved [1-8]. However, local ecological processes such as exploitative or apparent competition (indirect interactions) might instead cause a decrease in overlap in interacting partners. Because of the taxonomic and geographic coarseness of existing studies [2-5, 7], the structuring effect of such processes has been overlooked. Here, we assess the relative importance of phylogeny and ecological processes in a local, highly resolved, four-level antagonistic network. Across all network levels we consistently find that phylogenetic relatedness among resource species is correlated with consumer overlap but that phylogenetic relatedness among consumer species is not or negatively correlated with resource overlap. This pervasive pattern indicates that the antagonistic network has been shaped by both phylogeny on resource range and by exploitative competition limiting resource overlap among closely related consumer species. Intriguingly, the strength of phylogenetic signal varies in a consistent way across the network levels. We discuss the generality of our findings and their implications in a changing world. {\textcopyright} 2013 Elsevier Ltd.},
author = {Elias, Marianne and Fontaine, Colin and van Veen, F. J. Frank},
doi = {10.1016/j.cub.2013.05.066},
isbn = {0960-9822},
issn = {09609822},
journal = {Current Biology},
number = {14},
pages = {1355--1359},
pmid = {23791729},
publisher = {Elsevier Ltd},
title = {{Evolutionary history and ecological processes shape a local multilevel antagonistic network}},
url = {http://linkinghub.elsevier.com/retrieve/pii/S0960982213006921},
volume = {23},
year = {2013}
}
@article{Ackerly2010,
abstract = {The widespread correspondence between phenotypic variation and environmental conditions, the “fit” of organisms to their environment, reflects the adaptive value of plant functional traits. Several processes con-tribute to these patterns: plasticity, ecological sorting, and adaptive evolution. This article addresses the im-portance of ecological sorting processes (community assembly, migration, habitat tracking, etc.) as primary causes of functional trait distributions at the local and landscape level. In relatively saturated communities, plants will establish and regenerate in environments to which they are well adapted, so their distributions, and the distributions of associated functional traits, will reflect the distribution of optimal or near-optimal environmental conditions in space and time. The predicted evolutionary corollary of this process is that traits related to habitat occupancy, e.g., environmental tolerances, will be under stabilizing selection. This process contributes to the widely observed pattern of phylogenetic niche conservatism, i.e., ecological and phenotypic similarities of closely related species. Evidence for niche conservatism in plants is reviewed. Based on Jackson and Overpeck's concept of the realized environment, I propose three scenarios in which a species' distributional responses to environmental conditions will lead to a “mismatch” between its environmental tolerances and the environments it occupies, thus creating opportunities for adaptive evolution: (1) the colonization of “en-vironmental islands” (habitats that are discontinuous in niche space) that require large adaptive shifts in tolerance of one or more environmental factors; (2) the persistence of “trailing-edge” populations in species tracking changing climate, if barriers to dispersal of competitors prevent competitive exclusion in the dete-riorating conditions; and (3) responses to changes in the realized environment in multidimensional niche space, in which species are predicted to track environmental factors for which they exhibit narrow tolerances and exhibit adaptive evolutionary response along axes where they exhibit greater niche breadth. These three scenarios provide a conceptual framework that emphasizes the role of ecological sorting processes and sta-bilizing selection as the context for understanding opportunities for adaptive evolution in heterogeneous and changing environments.},
archivePrefix = {arXiv},
arxivId = {1058-5893},
author = {Ackerly, David D.},
doi = {10.1086/368401},
eprint = {1058-5893},
isbn = {1058-5893},
issn = {1058-5893},
journal = {International Journal of Plant Sciences},
keywords = {California flora,adaptation,climate change,community assembly,environmental tolerances,niche,phylogeny,specific leaf area,stabilizing selection},
number = {S3},
pages = {S165--S184},
pmid = {20102373},
title = {{Community Assembly, Niche Conservatism, and Adaptive Evolution in Changing Environments}},
url = {http://www.journals.uchicago.edu/doi/10.1086/368401},
volume = {164},
year = {2003}
}
@article{Losos2011,
abstract = {The past 30 years have seen a revolution in comparative biology. Before that time, systematics was not at the forefront of the biological sciences, and few scientists considered phylogenetic relationships when investigating evolutionary questions. By contrast, systematic biology is now one of the most vigorous disciplines in biology, and the use of phylogenies not only is requisite in macroevolutionary studies but also has been applied to a wide range of topics and fields that no one could possibly have envisioned 30 years ago. My message is simple: phylogenies are fundamental to comparative biology, but they are not the be-all and end-all. Phylogenies are powerful tools for understanding the past, but like any tool, they have their limitations. In addition, phylogenies are much more informative about pattern than they are about process. The best way to fully understand the past-both pattern and process-is to integrate phylogenies with other types of historical data as well as with direct studies of evolutionary process.},
author = {Losos, Jonathan B.},
doi = {10.1086/660020},
isbn = {1537-5323 (Electronic)$\backslash$n0003-0147 (Linking)},
issn = {0003-0147},
journal = {The American Naturalist},
keywords = {Biological Evolution,Biology,Biology: history,Biology: methods,History, 20th Century,History, 21st Century,Phylogeny},
number = {6},
pages = {709--727},
pmid = {21597249},
title = {{Seeing the Forest for the Trees: The Limitations of Phylogenies in Comparative Biology}},
url = {http://www.journals.uchicago.edu/doi/10.1086/660020},
volume = {177},
year = {2011}
}
@article{Chamberlain2014a,
abstract = {1. Species community composition is known to alter the network of interactions between two tro- phic levels, potentially affecting its functioning (e.g. plant pollination success) and the stability of communities. Phylogenies vary in shape with regard to the rate of evolutionary change across a tree (influencing tree balance) and variation in the timing of branching events (affecting the distribution of node ages in trees), both of which may influence the structure of species interaction networks. Because related species are likely to share many of the traits that regulate interactions, the shape of phylogenetic trees may provide some insights into the distribution of traits within communities, and hence the likelihood of interaction among species. However, little attention has been paid to the potential effects of changes in phylogenetic diversity (PD) on interaction networks. 2. Phylogenetic diversity is influenced by species diversity within a community, but also how dis- tantly-related the constituent species are from one another. Here, we evaluate the relationship between two important measures of phylogenetic diversity (tree shape and age of nodes) and the structure of plant–pollinator interaction networks using empirical and simulated data. Whereas the former allows us to evaluate patterns in real communities, the latter allows us to evaluate more systematically the relationship between tree shape and network structure under three different models of trait evolution. 3. In empirical networks, less balanced plant phylogenies were associated with lower connectance in interaction networks indicating that communities with the descendants of recent radiations are more diverged and specialized in their partnerships. In simulations, tree balance and the distribution of nodes through time were included in the best models for modularity, and the second best models for connectance and nestedness. In models assuming random evolutionary change through time (i.e. Brownian motion), less balanced trees and trees with nodes near the tips exhibited greater modular- ity, whereas in models with an early burst of radiation followed by relative stasis (i.e. early-burst models) more balanced trees and trees with nodes near roots had greater modularity. 4. Synthesis. Overall, these results suggest that the shape of phylogenies can influence the structure of plant–pollinator interaction networks. However, the mismatch between simulations and empirical data indicate that no simple model of trait evolution mimics that observed in real communities.},
author = {Chamberlain, Scott and V{\'{a}}zquez, Diego P. and Carvalheiro, Luisa and Elle, Elizabeth and Vamosi, Jana C.},
doi = {10.1111/1365-2745.12293},
isbn = {1365-2745},
issn = {13652745},
journal = {Journal of Ecology},
keywords = {Connectance,Diversity,Modularity,Nestedness,Network structure,Phylogeny imbalance,Plant population and community dynamics,Plant-pollinator interactions},
number = {5},
pages = {1234--1243},
title = {{Phylogenetic tree shape and the structure of mutualistic networks}},
url = {http://doi.wiley.com/10.1111/1365-2745.12293},
volume = {102},
year = {2014}
}
@article{Mayfield2009,
abstract = {Environmental filtering is a fundamental process in the ecological assembly of communities. Recently developed phylogenetic tools identify patterns associated with environmental filtering across whole communities. Here we introduce a novel method that allows the detection of traits involved in the environmental filtering of species from specific clades in specific habitat types. Our approach identifies nonindependent trait/habitat/clade (THC) associations and also provides a framework for detecting clearly defined two-way trait/clade, trait/habitat, and clade/habitat associations. The THC method relies on exact binomial tests and differentiates THC associations resulting from a three-way interaction from those that are generated by one or more underlying significant two-way interactions. It can also detect THC associations for which there are no significant two-way associations (trait/habitat, trait/clade, clade/habitat). To illustrate the THC method, we examine plant pollination and dispersal traits from six habitat types in a fragmented Costa Rican landscape. Results suggest that these traits are not widely important for the environmental filtering of most clades in this landscape, but animal dispersal and insect pollination are involved in the filtering of monocots and the Piperaceae in rain forest understory.},
author = {Mayfield, Margaret M. and Boni, Maciej F. and Ackerly, David D.},
doi = {10.1086/599293},
isbn = {1537-5323 (Electronic)$\backslash$n0003-0147 (Linking)},
issn = {0003-0147},
journal = {The American Naturalist},
keywords = {community assembly,dispersal,environmental filtering,exact binomial,functional traits,pollination mechanisms,test},
number = {1},
pages = {E1--E22},
pmid = {19463061},
title = {{Traits, Habitats, and Clades: Identifying Traits of Potential Importance to Environmental Filtering}},
url = {http://www.journals.uchicago.edu/doi/10.1086/599293},
volume = {174},
year = {2009}
}
@article{Janz2006,
abstract = {BACKGROUND: Plant-feeding insects make up a large part of earth's total biodiversity. While it has been shown that herbivory has repeatedly led to increased diversification rates in insects, there has been no compelling explanation for how plant-feeding has promoted speciation rates. There is a growing awareness that ecological factors can lead to rapid diversification and, as one of the most prominent features of most insect-plant interactions, specialization onto a diverse resource has often been assumed to be the main process behind this diversification. However, specialization is mainly a pruning process, and is not able to actually generate diversity by itself. Here we investigate the role of host colonizations in generating insect diversity, by testing if insect speciation rate is correlated with resource diversity. RESULTS: By applying a variant of independent contrast analysis, specially tailored for use on questions of species richness (MacroCAIC), we show that species richness is strongly correlated with diversity of host use in the butterfly family Nymphalidae. Furthermore, by comparing the results from reciprocal sister group selection, where sister groups were selected either on the basis of diversity of host use or species richness, we find that it is likely that diversity of host use is driving species richness, rather than vice versa. CONCLUSION: We conclude that resource diversity is correlated with species richness in the Nymphalidae and suggest a scenario based on recurring oscillations between host expansions - the incorporation of new plants into the repertoire - and specialization, as an important driving force behind the diversification of plant-feeding insects.},
author = {Janz, Niklas and Nylin, S{\"{o}}ren and Wahlberg, Niklas},
doi = {10.1186/1471-2148-6-4},
isbn = {1471-2148},
issn = {14712148},
journal = {BMC Evolutionary Biology},
keywords = {Animals,Biodiversity,Butterflies,Butterflies: classification,Butterflies: genetics,Ecosystem,Phylogeny,Plants,Species Specificity},
number = {1},
pages = {4},
pmid = {16420707},
title = {{Diversity begets diversity: Host expansions and the diversification of plant-feeding insects}},
url = {http://www.biomedcentral.com/1471-2148/6/4},
volume = {6},
year = {2006}
}
@article{Linder2008,
abstract = {The spatial and temporal patterns of plant species radiations are largely unknown. I used a nonlinear regression to estimate speciation and extinction rates from all relevant dated clades. Both are surprisingly high. A high species richness can be the result of either little extinction, thus preserving the diversity that dates from older radiations (a 'mature radiation'), or a 'recent and rapid radiation'. The analysis of radiations from different regions (Andes, New Zealand, Australia, southwest Africa, tropics and Eurasia) revealed that the diversity of Australia may be largely the result of mature radiations. This is in sharp contrast to New Zealand, where the flora appears to be largely the result of recent and rapid radiations. Mature radiations are characteristic of regions that have been climatically and geologically stable throughout the Neogene, whereas recent and rapid radiations are more typical of younger (Pliocene) environments. The hyperdiverse Cape and Neotropical floras are the result of the combinations of mature as well as recent and rapid radiations. Both the areas contain stable environments (the Amazon basin and the Cape Fold Mountains) as well as dynamic landscapes (the Andes and the South African west coast). The evolution of diversity can only be understood in the context of the local environment.},
author = {Linder, Hans Peter},
doi = {10.1098/rstb.2008.0075},
isbn = {0962-8436},
issn = {09628436},
journal = {Philosophical Transactions of the Royal Society B: Biological Sciences},
keywords = {Diversification,Dynamic environments,Environmental stability,Extinction rate,Speciation rate,Species richness},
number = {1506},
pages = {3097--3105},
pmid = {18579472},
title = {{Plant species radiations: Where, when, why?}},
url = {http://rstb.royalsocietypublishing.org/cgi/doi/10.1098/rstb.2008.0075},
volume = {363},
year = {2008}
}
@article{Breitkopf2015,
abstract = {Episodes of rapid speciation provide unique insights into evolutionary processes underlying species radiations and patterns of biodiversity. Here we investigated the radiation of sexually deceptive bee orchids (Ophrys). Based on a time-calibrated phylogeny and by means of ancestral character reconstruction and divergence time estimation, we estimated the tempo and mode of this radiation within a state-dependent evolutionary framework. It appears that, in the Pleistocene, the evolution of Ophrys was marked by episodes of rapid diversification coinciding with shifts to different pollinator types: from wasps to Eucera bees to Andrena and other bees. An abrupt increase in net diversification rate was detected in three clades. Among these, two phylogenetically distant lineages switched from Eucera to Andrena and other bees in a parallel fashion and at about the same time in their evolutionary history. Lack of early radiation associated with the evolution of the key innovation of sexual decep- tion suggests that Ophrys diversification was mainly driven by subsequent ecological opportu- nities provided by the exploitation of novel pollinator groups, encompassing many bee species slightly differing in their sex pheromone communication systems, and by spatiotemporal fluc- tuations in the pollinator mosaic.},
author = {Breitkopf, Hendrik and Onstein, Renske E. and Cafasso, Donata and Schl{\"{u}}ter, Philipp M. and Cozzolino, Salvatore},
doi = {10.1111/nph.13219},
isbn = {1469-8137},
issn = {14698137},
journal = {New Phytologist},
keywords = {Andrena,Diversification rates,Eucera,Ophrys,Pollination syndrome,Pollinator shift,Sexual deception (SD),Species radiation},
number = {2},
pages = {377--389},
pmid = {25521237},
title = {{Multiple shifts to different pollinators fuelled rapid diversification in sexually deceptive Ophrys orchids}},
volume = {207},
year = {2015}
}
@article{Davis2014,
abstract = {Many major branches in the Tree of Life are marked by stereotyped body plans that have been maintained over long periods of time. One possible explanation for this stasis is that there are genetic or developmental constraints that restrict the origin of novel body plans. An alternative is that basic body plans are potentially quite labile, but are actively maintained by natural selection. We present evidence that the conserved floral morphology of a species-rich flowering plant clade, Malpighiaceae, has been actively maintained for tens of millions of years via stabilizing selection imposed by their specialist New World oil-bee pollinators. Nine clades that have lost their primary oil-bee pollinators show major evolutionary shifts in specific floral traits associated with oil-bee pollination, demonstrating that developmental constraint is not the primary cause of morphological stasis in Malpighiaceae. Interestingly, Malpighiaceae show a burst in species diversification coinciding with the origin of this plant-pollinator mutualism. One hypothesis to account for radiation despite morphological stasis is that although selection on pollinator efficiency explains the origin of this unique and conserved floral morphology, tight pollinator specificity subsequently permitted greatly enhanced diversification in this system.},
author = {Davis, C. C. and Schaefer, H. and Xi, Z. and Baum, D. A. and Donoghue, M. J. and Harmon, L. J.},
doi = {10.1073/pnas.1403157111},
isbn = {1403157111},
issn = {0027-8424},
journal = {Proceedings of the National Academy of Sciences},
keywords = {Animals,Bees,Bees: physiology,Biological Evolution,Flowers,Flowers: anatomy {\&} histology,Flowers: physiology,Malpighiaceae,Malpighiaceae: anatomy {\&} histology,Malpighiaceae: physiology,Molecular Sequence Data,Phylogeny,Pollination,Pollination: physiology,Species Specificity,Symbiosis,Symbiosis: physiology},
number = {16},
pages = {5914--5919},
pmid = {24706921},
title = {{Long-term morphological stasis maintained by a plant-pollinator mutualism}},
url = {http://www.pnas.org/cgi/doi/10.1073/pnas.1403157111},
volume = {111},
year = {2014}
}
@article{Russell2007,
author = {Lach, Lori and Tillberg, Chadwick V and Suarez, Andrew V},
doi = {10.1007/s10530-010-9703-1},
issn = {0012-9658},
journal = {Biological Invasions},
keywords = {ant mutualisms {\'{a}} extrafloral,enemy release {\'{a}} herbivory,nectaries {\'{a}},{\'{a}} honeydew {\'{a}}},
number = {2},
pages = {3123--3133},
title = {{Author ' s personal copy Contrasting effects of an invasive ant on a native and an invasive plant}},
url = {http://link.springer.com/article/10.1007/s00442-005-0204-3{\%}5Cnhttp://www.esajournals.org/doi/abs/10.1890/0012-9658(2007)88[413:VIHIEO]2.0.CO;2},
volume = {88},
year = {2010}
}
@article{Canto2004,
abstract = {1. The study of the impact of herbivory on plant fitness and its relationship with environmental conditions is essential to an understanding of the ecological and evolutionary routes in different populations of the same plant species. This paper addresses the effects of herbivory and environment on growth (leaf production) and floral display (inflorescence production) in the tropical aroid Anthurium schlechtendalii ssp. schlechtendalii Kunth (Araceae). 2. Defoliation experiments were conducted in three natural A. schlechtendalii populations in Mexico: coastal sand dunes (Telchac); thorny dry forest (Dzemul); and mediumheight subhumid forest (Hobonil). Experiments were also conducted under commongarden conditions using plants transplanted from the three populations. 3. Logistic regression models showed that herbivory did not affect growth negatively, but did affect floral display in A. schlechtendalii. The effect on floral display differed as a function of environment, damage intensity and season. When the plants experienced no resource limitation (common garden), herbivory inhibited inflorescence production in the first season (2001) and at the highest damage level (75{\%}). When plants experienced heterogeneous resource conditions (field experiment), inflorescence production was inhibited in the second season (2002) at intermediate damage intensities. 4. Overall, A. schlechtendalii plants maintained growth regardless of damage intensity, but inflorescence production was reduced, indicating a trade-off between growth and floral display due to herbivory. Results also suggested that plants were constrained in their response by the environmental history of their native site.},
author = {Canto, A. and Parra-Tabla, V. and Garc??a-Franco, J. G.},
doi = {10.1111/j.0269-8463.2004.00886.x},
isbn = {0269-8463},
issn = {02698463},
journal = {Functional Ecology},
keywords = {Binary logistic regression,Common-garden experiment,Heterogeneous environments,Mexico,Trade-off,Tropical perennial plant},
number = {5},
pages = {692--699},
title = {{Variations in leaf production and floral display of Anthurium schlechtendalii (Araceae) in response to herbivory and environment}},
volume = {18},
year = {2004}
}
@article{Adler2006,
abstract = {Correlations between traits may constrain ecological and evolutionary responses to multispecies interactions. Many plants produce defensive compounds in nectar and leaves that could influence interactions with pollinators and herbivores, but the relationship between nectar and leaf defences is entirely unexplored. Correlations between leaf and nectar traits may be mediated by resources and prior damage. We determined the effect of nutrients and leaf herbivory by Manduca sexta on Nicotiana tabacum nectar and leaf alkaloids, floral traits and moth oviposition. We found a positive phenotypic correlation between nectar and leaf alkaloids. Herbivory induced alkaloids in nectar but not in leaves, while nutrients increased alkaloids in both tissues. Moths laid the most eggs on damaged, fertilized plants, suggesting a preference for high alkaloids. Induced nectar alkaloids via leaf herbivory indicate that species interactions involving leaf and floral tissues are linked and should not be treated as independent phenomena in plant ecology or evolution.},
author = {Adler, Lynn S. and Wink, Michael and Distl, Melanie and Lentz, Amanda J.},
doi = {10.1111/j.1461-0248.2006.00944.x},
isbn = {1461-023X},
issn = {1461023X},
journal = {Ecology Letters},
keywords = {Anabasine,Herbivory,Induced defences,Manduca sexta,Nicotiana tabacum,Nicotine,Optimal defence theory,Phenotypic correlation,Pollination,Toxic nectar},
number = {8},
pages = {960--967},
pmid = {16913940},
title = {{Leaf herbivory and nutrients increase nectar alkaloids}},
url = {http://doi.wiley.com/10.1111/j.1461-0248.2006.00944.x},
volume = {9},
year = {2006}
}
@article{McCall2006,
abstract = {Plants interact with many visitors who consume a variety of plant tissues. While the consequences of herbivory to leaves and shoots are well known, the implications of florivory, the consumption of flowers prior to seed coat formation, have received less attention. Herbivory and florivory can yield different plant, population and community outcomes; thus, it is critical to distinguish between these two types of consumption. Here, we consider the ecological and evolutionary consequences of florivory. A growing number of studies recognize that florivory is common in natural systems and in some cases surpasses leaf herbivory in magnitude and impact. Florivores can affect male and female plant fitness via direct trophic effects and through altered pathways of species interactions. In particular, florivory can affect pollination and have consequences for plant mating and floral sexual system evolution. Plants are not defenceless against florivore damage. Concepts of resistance and tolerance can be applied to plant-florivore interactions. Moreover, extant theories of plant chemical defence, including optimal defence theory, growth rate hypothesis and growth differentiation-balance hypothesis, can be used to make testable predictions about when and how plants should defend flowers against florivores. The majority of the predictions remain untested, but they provide a theoretical foundation on which to base future experiments. The approaches to studying florivory that we outline may yield novel insights into floral and defence traits not illuminated by studies of pollination or herbivory alone.},
author = {McCall, Andrew C. and Irwin, Rebecca E.},
doi = {10.1111/j.1461-0248.2006.00975.x},
isbn = {1461-0248 (Electronic)$\backslash$n1461-023X (Linking)},
issn = {1461023X},
journal = {Ecology Letters},
keywords = {Floral herbivory,Florivory,Growth differentiation-balance hypothesis,Growth rate hypothesis,Optimal defence theory,Plant mating system,Pollination,Resistance,Tolerance},
number = {12},
pages = {1351--1365},
pmid = {17118009},
title = {{Florivory: the intersection of pollination and herbivory}},
volume = {9},
year = {2006}
}
@article{Sauve2015,
abstract = {A sequential injection system which consists of a syringe pump, a selector valve, a multi-port valve, a gas-liquid separator and a solenoid valve for the determination of arsenic by hydride generation atomic absorption spectrometry using tetrahydroborate as reductant was developed. The reduction time of sample with tetrahydoborate has increased by keeping the reactant in gas-liquid separator by using the solenoid valve. Various parameters affecting the performance of the sequential injection system were optimized, including reaction-time, carrier gas flow, sample volume, tetrahydroborate volume and concentration. Established sequential injection hydride generation technique was simple and automated operation. A sample throughput of 112/h was achieved with 400 $\mu$L samples with a precision of 2.0{\%} RSD at 4 $\mu$g/L As (n = 10) and a detection limit of 0.09 $\mu$g/L. Good agreement with the certified values was obtained for the determination of arsenic in standard reference materials.},
archivePrefix = {arXiv},
arxivId = {arXiv:1011.1669v3},
author = {Yin, Xuefeng and Zhang, Jianjun and Wang, Xiaofang},
doi = {10.1017/CBO9781107415324.004},
eprint = {arXiv:1011.1669v3},
isbn = {9788578110796},
issn = {02533820},
journal = {Fenxi Huaxue},
keywords = {Arsenic,Hydride generation atomic absorption spectrometry,Sequential injection},
number = {10},
pages = {1365--1367},
pmid = {25246403},
title = {{Sequential injection analysis system for the determination of arsenic by hydride generation atomic absorption spectrometry}},
volume = {32},
year = {2004}
}
@article{Strauss2002,
abstract = {Herbivores can consume significant amounts of plant biomass in many environments. Yet plants are not defenseless against such attack. Although defenses might benefit plants in the presence of herbivores, herbivore attack varies both spatially and temporally, and the expression of plant resistance to herbivores can be costly in the absence of plant enemies. Costs can be described as allocation costs, resource-based tradeoffs between resistance and fitness, or as ecological costs, decreases in fitness resulting from interactions with other species. Here, we update the seminal 1996 Bergelson and Purrington review of resistance costs and find that many more studies have documented costs of resistance (sensu lato) than found during the 1996 survey. Eighty-two percent of studies in which genetic background is controlled, demonstrate significant fitness reductions associated with herbivore resistance. We categorize studies by type of resistance, induced or constitutive, by type of cost, and also by the degree to which investigators controlled for genetic background. Recent work has commonly detected both direct resistance costs, such as resource-based tradeoffs, and ecological costs, which depend on interactions with other species.},
archivePrefix = {arXiv},
arxivId = {arXiv:1002.2562v1},
author = {Strauss, Sharon Y. and Rudgers, Jennifer A. and Lau, Jennifer A. and Irwin, Rebecca E.},
doi = {10.1016/S0169-5347(02)02483-7},
eprint = {arXiv:1002.2562v1},
isbn = {0169-5347},
issn = {01695347},
journal = {Trends in Ecology and Evolution},
number = {6},
pages = {278--285},
pmid = {847},
title = {{Direct and ecological costs of resistance to herbivory}},
volume = {17},
year = {2002}
}
@article{Strauss1997,
author = {Strauss, Sharon Y.},
doi = {doi:10.1890/0012-9658(1997)078[1640:FCLHPA]2.0.CO;2},
journal = {Ecology},
keywords = {floral characters,herbivory,indirect interactions,male plant fitness,nicotiana at-,pollen production,pollination,raphanus raphanistrum,tenuata},
number = {6},
pages = {1640--1645},
title = {{Floral characters link herbivores, pollinators, and plant fitness}},
volume = {78},
year = {1997}
}
@article{Theis2006,
abstract = {The evolution of floral scent as a plant reproductive signal is assumed to be driven by pollinator behavior, with little attention paid to other potential selective forces such as herbivores. I tested 10 out of the 13 compounds emitted by dioecious Cirsium arvense, Canada thistle, including 2-phenylethanol, methyl salicylate, p-anisaldehyde, benzaldehyde, benzyl alcohol, phenylacetaldehyde, linalool, furanoid linalool oxides (E and Z), and dimethyl salicylate. Single compounds (and one isomer) set out in scent-baited water-bowl traps trapped over 10 species of pollinators and 16 species of floral herbivores. The two dominant components of the fragrance blend of C. arvense, benzaldehyde and phenylacetaldehyde, trapped both pollinators and florivores. Other compounds attracted either pollinators or florivores. Florivores of C. arvense appear to use floral scent compounds as kairomones; by advertising to pollinators, C. arvense also attracts its own enemies},
author = {Theis, Nina},
doi = {10.1007/s10886-006-9051-x},
isbn = {0098-0331},
issn = {00980331},
journal = {Journal of Chemical Ecology},
keywords = {Benzaldehyde,Cirsium arvense,Florivores,Fragrance,Herbivores,Phenylacetaldehyde,Pollinators,Scent,Trapping,Volatiles},
number = {5},
pages = {917--927},
pmid = {16739013},
title = {{Fragrance of Canada thistle (Cirsium arvense) attracts both floral herbivores and pollinators}},
url = {http://link.springer.com/10.1007/s10886-006-9051-x},
volume = {32},
year = {2006}
}
@article{Sauve2014,
abstract = {The relationship between the structure of ecological networks and community stability has been studied for decades. Recent developments highlighted that this relationship depended on whether interactions were antagonistic or mutualistic. Different structures promoting stability in different types of ecological networks, i.e. mutualistic or antagonistic, have been pointed out. However, these findings come from studies considering mutualistic and antagonistic interactions separately whereas we know that species are part of both types of networks simultaneously. Understanding the relationship between network structure and community stability, when mutualistic and antagonistic interactions are merged in a single network, thus appears as the next challenge to improve our understanding of the dynamics of natural communities. Using a theoretical approach, we test whether the structural characteristics known to promote stability in networks made of a single interaction type still hold for network merging mutualistic and antagonistic interactions. We show that the effects of diversity and connectance remain unchanged. But the effects of nestedness and modularity are strongly weakened in networks combining mutualistic and antagonistic interactions. By challenging the stabilizing mechanisms proposed for networks with a single interaction type, our study calls for new measures of structure for networks that integrate the diversity of interaction.},
author = {Sauve, Alix M C and Fontaine, Colin and Th{\'{e}}bault, Elisa},
doi = {10.1111/j.1600-0706.2013.00743.x},
isbn = {1600-0706},
issn = {00301299},
journal = {Oikos},
number = {3},
pages = {378--384},
title = {{Structure-stability relationships in networks combining mutualistic and antagonistic interactions}},
volume = {123},
year = {2014}
}
@article{Ollerton2011,
abstract = {It is clear that the majority of flowering plants are pollinated by insects and other animals, with a minority utilising abiotic pollen vectors, mainly wind. However there is no accurate published calculation of the proportion of the ca 352 000 species of angiosperms that interact with pollinators. Widely cited figures range from 67{\%} to 96{\%} but these have not been based on firm data. We estimated the number and proportion of flowering plants that are pollinated by animals using published and unpublished community-level surveys of plant pollination systems that recorded whether each species present was pollinated by animals or wind. The proportion of animal-pollinated species rises from a mean of 78{\%} in temperate-zone communities to 94{\%} in tropical communities. By correcting for the latitudinal diversity trend in flowering plants, we estimate the global number and proportion of animal pollinated angiosperms as 308 006, which is 87.5{\%} of the estimated species-level diversity of flowering plants. Given current concerns about the decline in pollinators and the possible resulting impacts on both natural communities and agricultural crops, such estimates are vital to both ecologists and policy makers. Further research is required to assess in detail the absolute dependency of these plants on their pollinators, and how this varies with latitude and community type, but there is no doubt that plant-pollinator interactions play a significant role in maintaining the functional integrity of most terrestrial ecosystems},
author = {Ollerton, Jeff and Winfree, Rachael and Tarrant, Sam},
doi = {10.1111/j.1600-0706.2010.18644.x},
isbn = {0030-1299},
issn = {00301299},
journal = {Oikos},
number = {3},
pages = {321--326},
pmid = {22209078},
title = {{How many flowering plants are pollinated by animals?}},
volume = {120},
year = {2011}
}
@article{Adler2004,
abstract = {Traits that are attractive to mutualists may also attract antagonists, resulting in conflicting selection pressures. Here we develop the idea that increased floral nectar production can, in some cases, increase herbivory. In these situations, selection for increased nectar production to attract pollinators may be constrained by a linked cost of herbivore attraction. In support of this hypothesis, we report that experimentally supplementing nectar rewards in Datura stramonium led to increased oviposition by Manduca sexta, a sphingid moth that pollinates flowers, but whose larvae feed on leaf tissue. We speculate that nectar composition may provide information about plant nutritional status or defense that floral visitors could use to make oviposition decisions. Thus, selection by floral visitors and leaf herbivores may be inextricably intertwined, and herbivores may represent a relatively un- explored agent of selection on nectar traits.},
author = {Adler, Lynn S. and Bronstein, Judith L.},
doi = {10.1890/03-0409},
isbn = {0012-9658},
issn = {00129658},
journal = {Ecology},
keywords = {Datura stramonium,Herbivory,Manduca sexta,Mutualism,Natural selection,Nectar composition,Nectar reward,Oviposition,Pollination},
number = {6},
pages = {1519--1526},
title = {{Attracting antagonists: Does floral nectar increase leaf herbivory?}},
url = {http://www.esajournals.org/doi/abs/10.1890/03-0409},
volume = {85},
year = {2004}
}
@article{Fontaine2015,
abstract = {Conservatism in species interaction, meaning that related species tend to interact with similar partners, is an important feature of ecological interactions. Studies at community scale highlight variations in conservatism strength depending on the characteristics of the ecological interaction studied. However, the heterogeneity of datasets and methods used prevent to compare results between mutualistic and antagonistic networks. Here we perform such a comparison by taking plant--insect communities as a study case, with data on plant--herbivore and plant--pollinator networks. Our analysis reveals that plants acting as resources for herbivores exhibit the strongest conservatism in species interaction among the four interacting groups. Conservatism levels are similar for insect pollinators, insect herbivores and plants as interacting partners of pollinators, although insect pollinators tend to have a slightly higher conservatism than the two others. Our results thus clearly support the current view that within antagonistic networks, conservatism is stronger for species as resources than for species as consumer. Although the pattern tends to be opposite for plant--pollinator networks, our results suggest that asymmetry in conservatism is much less pronounced between the pollinators and the plant they interact with. We discuss these differences in conservatism strength in relation with the processes structuring plant--insect communities.},
author = {Fontaine, Colin and Th{\'{e}}bault, Elisa},
doi = {10.1007/s10144-014-0473-y},
file = {:Users/alyssacirtwill/Documents/Papers/Fontaine, Th{\'{e}}bault{\_}2015{\_}Population Ecology.pdf:pdf},
issn = {14383896},
journal = {Population Ecology},
keywords = {Antagonistic,Communities,Interaction network,Mutualistic,Phylogenetic signal},
number = {1},
pages = {29--36},
title = {{Comparing the conservatism of ecological interactions in plant–pollinator and plant–herbivore networks}},
url = {http://link.springer.com/10.1007/s10144-014-0473-y},
volume = {57},
year = {2015}
}
@article{Mitchell2009,
abstract = {BACKGROUND: Co-flowering plant species frequently share pollinators. Pollinator sharing is often detrimental to one or more of these species, leading to competition for pollination. Perhaps because it offers an intriguing juxtaposition of ecological opposites - mutualism and competition - within one relatively tractable system, competition for pollination has captured the interest of ecologists for over a century. SCOPE: Our intent is to contemplate exciting areas for further work on competition for pollination, rather than to exhaustively review past studies. After a brief historical summary, we present a conceptual framework that incorporates many aspects of competition for pollination, involving both the quantity and quality of pollination services, and both female and male sex functions of flowers. Using this framework, we contemplate a relatively subtle mechanism of competition involving pollen loss, and consider how competition might affect plant mating systems, overall reproductive success and multi-species interactions. We next consider how competition for pollination might be altered by several emerging consequences of a changing planet, including the spread of alien species, climate change and pollinator declines. Most of these topics represent new frontiers whose exploration has just begun. CONCLUSIONS: Competition for pollination has served as a model for the integration of ecological and evolutionary perspectives in the study of species interactions. Its study has elucidated both obvious and more subtle mechanisms, and has documented a range of outcomes. However, the potential for this interaction to inform our understanding of both pure and applied aspects of pollination biology has only begun to be realized.},
author = {Mitchell, Randall J. and Flanagan, Rebecca J. and Brown, Beverly J. and Waser, Nickolas M. and Karron, Jeffrey D.},
doi = {10.1093/aob/mcp062},
isbn = {0305-7364},
issn = {10958290},
journal = {Annals of Botany},
keywords = {Alien plants,Lythrum,Mimulus,climate change,competition for pollination,facilitation,invasive species,mating system,mechanism,pollen loss,pollinator visitation,sexual function},
number = {9},
pages = {1403--1413},
pmid = {19304814},
title = {{New frontiers in competition for pollination}},
url = {http://aob.oxfordjournals.org/cgi/doi/10.1093/aob/mcp062},
volume = {103},
year = {2009}
}
@article{Andersson2002,
abstract = {The study explores whether or not there are convergent patterns in floral scent composition among plant species that completely or partially rely on butterflies for pollination. Floral scent compounds were analysed from 22 flowering butterfly-pollinated plant species, representing 13 families which originate mainly from temperate North Europe but also from tropical and temperate America. Scents were collected using the dynamic headspace adsorption method and identified with coupled gas chromatography and mass spectrometry (GC-MS). In total, 217 floral scent compounds were identified, with the number per species ranging from 8 to 65. The major emerging pattern is the occurrence of certain compounds emitted exclusively by the flowers of many of the investigated species in major amounts - the benzenoids phenylacetaldehyde and 2-phenylethanol, the monoterpenes linalool and linalool oxide (furanoid) I and II and the irregular terpene oxoisophorone. It is likely that these compounds serve as a signal to attract pollinating butterflies, and may have evolved in conjunction with the sensory capabilities of butterflies as a specific group of pollinators. While there is convergence in terms of the compounds sharing this function there has been a geographical divergence in terms of their relative abundance. The predominance (in terms of both numbers and relative amount) of benzenoids in many of the scent blends of the European temperate species and of linalool and its derivatives in those of the American species constitute two discernible groups among these plants. (C) 2002. The Linnean Society of London},
author = {Andersson, Susanna and Nilsson, L. A.Anders and Groth, Inga and Bergstr{\"{o}}m, Gunnar},
doi = {10.1046/j.1095-8339.2002.00068.x},
isbn = {1095-8339},
issn = {00244074},
journal = {Botanical Journal of the Linnean Society},
keywords = {Dynamic headspace,GC-MS,Lepidoptera,Nectar-plants},
number = {2},
pages = {129--153},
pmid = {629},
title = {{Floral scents in butterfly-pollinated plants: Possible convergence in chemical composition}},
volume = {140},
year = {2002}
}
@article{Jarne1993,
abstract = {Selfing, the fusion of male and female gametes from a single genetic individual or colony, is possible in many plants and also in hermaphrodite animals.We review the occurrence of selfing and mechanisms for its avoidance, in functionally hermaphrodite animal and plants. We discuss means by which selfing can be detected and briefly review techniques for estimation of selfing frequencies in natural populations. Although many functionally hemaphrodite species are probably almost complete outcrossers or inbreeders, mixed mating systems are also found in both plant and animal populations. We review therories for the advantages and disadvantages of selfing, and for the maintenance of mixed mating systems, together with empirical data showing that at least some of the factos involved in the theories (for instanc,e reproductive assurance, cost of mating, and inbreeding depression) are detectable in actually or potentially selfing organisms. More work is still needed on animal selfing and selfing avoidance and, for both animals and plants, on the evolutionary origins of selfing and on the effects of selfing on genetic diversity.},
author = {Jarne, Philippe and Charlesworth, Deborah},
doi = {10.1146/annurev.es.24.110193.002301},
isbn = {0066-4162},
issn = {0066-4162},
journal = {Source: Annual Review of Ecology and Systematics},
keywords = {hermaphrodite,inbreeding depression,selfing},
number = {1},
pages = {441--466},
pmid = {111},
title = {{The Evolution of the Selfing Rate in Functionally Hermaphrodite Plants and Animals}},
volume = {24},
year = {1993}
}
@article{Schemske1981,
abstract = {The hypothesis that the understory herbs Costus allenii and C. laevis (Zingiberaceae) have converged in floral characteristics to use the same pollinator was investigated in central Panama. Observations and experiments indicated that these species (1) occupy the same habitats, (2) flower synchronously, (3) are identical in flower color, morphology, and nectar secretion patterns, (4) share the same pollinator, the bee Euglossa imperialis, (5) are self-compatible, but not autogamous, and (6) have strong barriers to hybridization. Both grow in low density along streamsides and produce a single flower per day for an extended period (up to 4 mo). Flower density is depressed through extensive predation by the weevil Cholus cinctus, which damaged 31{\%} of all C. allenii and 60{\%} of all C. laevis inflorescences. Direct observation of foraging bees indicated that individuals regularly visit both plant species. An experimental analysis of interspecific pollen transfer using powered paint as a marker verified these results; 97{\%} of the flowers checked had received heterospecific visits. The high probability of interspecific pollination did not affect fruiting success. I suggest that low flower density, exaggerated by extreme floral predation, has selected for floral similarity and pollinator sharing in these species. Floral convergence increases effective flower density and nectar supplies, and probably increases the regularity and rate of pollinator visitation.},
author = {Schemske, Douglas W.},
doi = {10.2307/1936993},
isbn = {00129658},
issn = {00129658},
journal = {Ecology},
number = {4},
pages = {946--954},
pmid = {2468},
title = {{Floral Convergence and Pollinator Sharing in Two Bee-Pollinated Tropical Herbs}},
url = {http://doi.wiley.com/10.2307/1936993},
volume = {62},
year = {1981}
}
@article{Ehrlich1964,
abstract = {JSTOR is a not-for-profit service that helps scholars, researchers, and students discover, use, and build upon a wide range of content in a trusted digital archive. We use information technology and tools to increase productivity and facilitate new forms of scholarship. For more information about JSTOR, please contact support@jstor.org.},
archivePrefix = {arXiv},
arxivId = {0710.4428v1},
author = {Ehrlich, Paul R. and Raven, Peter H.},
doi = {10.2307/2406212},
eprint = {0710.4428v1},
isbn = {00143820},
issn = {00143820},
journal = {Evolution},
number = {4},
pages = {586--608},
pmid = {197},
title = {{Butterflies and Plants: a Study in Coevolution}},
url = {http://doi.wiley.com/10.1111/j.1558-5646.1964.tb01674.x},
volume = {18},
year = {1964}
}
@article{MartenRodriguez2015,
abstract = {* The evolution of self-pollination has long been considered an adaptive strategy to cope with low or variable pollinator service; however, alternative reproductive strategies, such as generalized pollination ({\textgreater}1 pollinator functional group), may also ensure plant reproductive success in environments with inadequate pollinator visitation. * Island–mainland systems provide ideal settings to assess the interaction between pollination and breeding systems in response to pollinator visitation because islands are often pollinator-depauperate. This study compared 28 insular and 26 mainland species of Caribbean Gesneriaceae to test the hypothesis that low diversity and possibly low pollinator service on islands would lead to a greater frequency of generalized plant–pollinator interactions and/or a higher potential for autonomous self-pollination in insular than in mainland species. We also assessed the hypothesis that epiphytic species should have greater autofertility than species occurring in other habitats. * Pollinator observations conducted in the field from 2004 to 2014 revealed bat, bee, butterfly, hummingbird, moth, and generalized pollination systems. Functional specialization in pollination systems was high in insular (71{\%} of the species) and mainland sites (all species), but generalized and bat-pollinated species were more common on islands. Overall, pollinator visitation rates did not differ between island and mainland; however, for hummingbird-pollinated species, visitation rate was on average three times higher in mainland than island species. Autofertility indices (fruit set of bagged/outcross flowers) ranged from 0 to 1 and did not differ between island and mainland species. Species growing on rocks (rupiculous) and trees (epiphytic) had on average higher autofertility than terrestrial species. * Synthesis. This study revealed that alternative reproductive strategies are used in pollinator-depauperate environments. Pollinator visitation is lower in insular hummingbird-pollinated species (the ancestral pollination system of insular Gesneriaceae); therefore, generalized pollination may be considered a reproductive assurance mechanism evolved primarily on island environments. Contrary to the long-standing tenet, however, autonomous self-pollination was similar in island and mainland Gesneriaceae suggesting that: (i) generalized pollination provides a viable alternative to selfing in pollinator-depauperate environments, (ii) autofertility as a reproductive assurance mechanism may be frequent in plant species from mainland regions in environments with unpredictable pollinator visitation and resource availability.},
author = {Mart{\'{e}}n-Rodr{\'{i}}guez, Silvana and Quesada, Mauricio and Castro, Abel Almarales and Lopezaraiza-Mikel, Martha and Fenster, Charles B.},
doi = {10.1111/1365-2745.12457},
isbn = {1365-2745},
issn = {13652745},
journal = {Journal of Ecology},
keywords = {Caribbean,Gesneriaceae,Island,Mainland,Pollination,Reproductive ecology,Self-pollination,Specialization},
number = {5},
pages = {1190--1204},
title = {{A comparison of reproductive strategies between island and mainland Caribbean Gesneriaceae}},
url = {http://dx.doi.org/10.1111/1365-2745.12457},
volume = {103},
year = {2015}
}
@article{Borba2001,
abstract = {We studied the floral biology of 12 populations of five rupicolous Pleurothallis (Orchidaceae) species occurring in campo rupestre vegetation at nine localities in Brazil. All of these species are pollinated by flies belonging to the families Chloropidae and Phoridae. In the five Pleurothallis species studied, all conspecific populations attracted the same pollinator species. All pollinators were females; they laid eggs in flowers of the two nectarless species, but never in the flowers of nectar-presenting species. The two pairs of Pleurothallis species with similar flower morphologies and odours attracted the same pollinators: P. johannensis-P. fabiobarrosii, pollinated by Tricimba sp. (Chloropidae) and P. teres-P. ochreata pollinated by Megaselia spp. (Phoridae). There was no overlap in the distribution of the Pleurothallis species that shared pollinators. Despite similarities in floral morphology and odour, genetic data show that these species pairs are not each other's closest relatives. We hypothesize that these similarities are due to convergence in allopatric species that evolved similar pollination mechanisms. Conversely, there are reasons to believe that adaptation to different pollination mechanisms occurred in the closely related species P. johannensis and P. teres. {\textcopyright} 2001 Annals of Botany Company.},
author = {Borba, Eduardo L. and Semir, Jo{\~{a}}o},
doi = {10.1006/anbo.2001.1434},
isbn = {0305-7364},
issn = {03057364},
journal = {Annals of Botany},
keywords = {Campo rupestre,Chloropidae,Floral biology,Fly-pollination,Orchidaceae,Phoridae,Pleurothallis,Pollinator specificity},
number = {1},
pages = {75--88},
pmid = {775},
title = {{Pollinator specificity and convergence in fly-pollinated Pleurothallis (Orchidaceae) species: A multiple population approach}},
url = {http://aob.oxfordjournals.org/cgi/doi/10.1006/anbo.2001.1434},
volume = {88},
year = {2001}
}
@article{Brown1979,
abstract = {We studied the pollination ecology of nine species of red, tubular flowers which bloom together in different combinations in the White Mountains of Arizona, USA. All species were strik-ingly convergent in floral color, size, and shape. Hummingbirds, the primary pollinators, usually did not visit flower species selectively, and individual birds often simultaneously carried four or more species of pollen. Flowers may have competed interspecifically for these shared pollinators, but competition was reduced because character displacement in orientation of anthers and stigma resulted in some species using different parts of the bird to transport their pollen. Most flower species secreted nectar at similar rates, particularly when they bloomed together in mixed stands. A population of Lobelia cardinalis secreted no nectar; it attracted hummingbirds by mimicing more abundant, nectar-producing species. This temperate flower community, which resembles some associations of conver-gent Mullerian and Batesian mimics, appears to have evolved its characteristic convergent structure because the advantages of using similar signals and rewards to share the same hummingbird pollinators outweigh the advantages of diverging to reduce interspecific competition.},
author = {Brown, James H. and Kodric-Brown, Astrid},
doi = {10.2307/1936870},
isbn = {0012-9658},
issn = {00129658},
journal = {Ecology},
keywords = {birds,coevolution,coexistence,community,competition,convergence,flowers,humming-,mimicry,nectar,pollination},
number = {5},
pages = {1022--1035},
title = {{Convergence, Competition, and Mimicry in a Temperate Community of Hummingbird-Pollinated Flowers}},
url = {http://doi.wiley.com/10.2307/1936870},
volume = {60},
year = {1979}
}
@article{Bascompte2006,
abstract = {The mutualistic interactions between plants and their pollinators or seed dispersers have played a major role in the maintenance of Earth's biodiversity. To investigate how coevolutionary interactions are shaped within species-rich communities, we characterized the architecture of an array of quantitative, mutualistic networks spanning a broad geographic range. These coevolutionary networks are highly asymmetric, so that if a plant species depends strongly on an animal species, the animal depends weakly on the plant. By using a simple dynamical model, we showed that asymmetries inherent in coevolutionary networks may enhance long-term coexistence and facilitate biodiversity maintenance.},
archivePrefix = {arXiv},
arxivId = {arXiv:1207.6416},
author = {Bascompte, Jordi and Jordano, Pedro and Olesen, Jens M.},
doi = {10.1126/science.1123412},
eprint = {arXiv:1207.6416},
file = {:Users/alyssacirtwill/Downloads/bascompte2006.pdf:pdf},
isbn = {1095-9203},
issn = {00368075},
journal = {Science},
keywords = {Animals,Biodiversity,Biological,Biological Evolution,Ecosystem,Mathematics,Models,Plant Physiological Phenomena,Pollen,Symbiosis},
number = {5772},
pages = {431--433},
pmid = {16627742},
title = {{Asymmetric coevolutionary networks facilitate biodiversity maintenance}},
url = {http://www.ncbi.nlm.nih.gov/pubmed/16627742},
volume = {312},
year = {2006}
}
@article{Johnson2000,
abstract = {The long-standing notion that most angiosperm flowers are specialized for pollination by particular animal types, such as birds or bees, has been challenged recently on the basis of apparent widespread generalization in pollination systems. At the same time, biologists working mainly in the tropics and the species-rich temperate floras of the Southern hemisphere are documenting pollination systems that are remarkably specialized, often involving a single pollinator species. Current studies are aimed at understanding: (1) the ecological forces that have favoured either generalization or specialization in particular lineages and regions; (2) the implications for selection on floral traits and divergence of populations; and (3) the risk of collapse in plant-pollinator mutualisms of varying specificity.},
author = {Johnson, Steven D. and Steiner, Kim E.},
doi = {10.1016/S0169-5347(99)01811-X},
isbn = {0169-5347},
issn = {01695347},
journal = {Trends in Ecology and Evolution},
number = {4},
pages = {140--143},
pmid = {10717682},
title = {{Generalization versus specialization in plant pollination systems}},
volume = {15},
year = {2000}
}
@article{Ollerton1996,
abstract = {To what extent do studies of the ecology of mutualistic interactions inform us about the evolution of such relationships? As I will show below, the evolution of floral diversity seems to be based upon specialized relationships with pollinators, yet (with some obvious exceptions) the majority of angiosperms appear to be promiscuously pollinated by a range of texa},
author = {Ollerton, Jeff},
doi = {10.2307/2261338},
isbn = {0022-0477},
issn = {00220477},
journal = {The Journal of Ecology},
number = {5},
pages = {767--769},
title = {{Reconciling ecological processes with phylogenetic patterns: the apparent paradox of plant--pollinator systems}},
url = {http://www.jstor.org/stable/2261338?origin=crossref},
volume = {84},
year = {1996}
}
@article{Armbruster1997,
abstract = {To evaluate possible evolutionary links between plant–herbivore and plant–pollinator relationships, defense and reward characteristics and pollination ecology were mapped onto a morphologically estimated phylogeny of 42 species of Dalechampia. This procedure generated detailed hypotheses about the evolution of anti-herbivore defense and pollination systems. These hypotheses were tested using the results of chemical analyses and bioassays of foliar and floral secretions. Multiple lines of defense appear to have evolved in sequence in Dalechampia. The first defense system to originate was deployment of triterpene resins to defend the staminate flowers. This feature was a preaptation (preadaptation) that allowed the evolution of a resin-based, pollinator-reward system. Thus pollination by resin-collecting bees originated as a “transfer exaptation” (a new function replaced the old). This hypothesis is supported by anti-herbivore activities of floral resins and by chemical similarity of floral defense and reward resins. After the resin defense of flowers was lost by conversion into a reward system, there followed (in evolutionary time) a sequence of defensive innovations. These included nocturnal closure of large, involucral bracts to protect both staminate and pistillate flowers. Phylogenetic analysis showed that the large bracts themselves probably originated as a floral advertisement system, and the bracts assumed a defensive function through “addition exaptation” (a new function was added to the old). Additional lines of defense to evolve were deployment of resin to defend the developing ovaries and seeds (an addition exaptation), deployment of sharp, detaching trichomes on enveloping sepals to defend developing seeds (apparent adaptations), closure of involucral bracts around the developing fruits and seeds (an addition exaptation), and deployment of resin to defend the leaves and growing shoot tips (also an addition exaptation). Support for this scenario also derives from the chemical similarities of sepal, foliar, and floral resins, and the anti-herbivore properties of foliar resins. It appears that at least one pollinator-reward system originated by modification of a defense system, and several defense systems originated by modification of pollinator reward and advertisement systems. Thus exaptations have been important in the origin of new defense and pollination systems, and each system has significantly influenced the evolution of the other on several occasions.},
author = {Armbruster, W Scott},
doi = {10.1890/0012-9658(1997)078[1661:ELEOPH]2.0.CO;2},
isbn = {0012-9658},
issn = {00129658},
journal = {Ecology},
keywords = {Dalechempia,Euphorbiaceae,evolution,exaptation,herbivory,phylogeny,plant defenses,plant-animal interactions,pollination,preadaptation,preaptation.},
number = {6},
pages = {1661--1672},
title = {{Exaptations Link Evolution of Plant-Herbivore and Plant-Pollinator Interactions : A Phylogenetic Inquiry Published by : Ecological Society of America EXAPTATIONS LINK EVOLUTION OF PLANT-HERBIVORE AND PLANT-POLLINATOR INTERACTIONS : A PHYLOGENETIC INQUIRY}},
url = {http://www.esajournals.org/doi/abs/10.1890/0012-9658(1997)078[1661:ELEOPH]2.0.CO;2},
volume = {78},
year = {1997}
}
@article{Kursar2009,
abstract = {Plants and their herbivores constitute more than half of the organisms in tropical forests. Therefore, a better understanding of the evolution of plant defenses against their herbivores may be central for our understanding of tropical biodiversity. Here, we address the evolution of antiherbivore defenses and their possible contribution to coexistence in the Neotropical tree genus Inga (Fabaceae). Inga has {\textgreater}300 species, has radiated recently, and is frequently one of the most diverse and abundant genera at a given site. For 37 species from Panama and Peru we characterized developmental, ant, and chemical defenses against herbivores. We found extensive variation in defenses, but little evidence of phylogenetic signal. Furthermore, in a multivariate analysis, developmental, ant, and chemical defenses varied independently (were orthogonal) and appear to have evolved independently of each other. Our results are consistent with strong selection for divergent defensive traits, presumably mediated by herbivores. In an analysis of community assembly, we found that Inga species co-occurring as neighbors are more different in antiherbivore defenses than random, suggesting that possessing a rare defense phenotype increases fitness. These results imply that interactions with herbivores may be an important axis of niche differentiation that permits the coexistence of many species of Inga within a single site. Interactions between plants and their herbivores likely play a key role in the generation and maintenance of the conspicuously high plant diversity in the tropics.},
author = {Kursar, T. A. and Dexter, K. G. and Lokvam, J. and Pennington, R. T. and Richardson, J. E. and Weber, M. G. and Murakami, E. T. and Drake, C. and McGregor, R. and Coley, P. D.},
doi = {10.1073/pnas.0904786106},
file = {:Users/alyssacirtwill/Documents/Papers/Kursar et al.{\_}2009{\_}Proceedings of the National Academy of Sciences.pdf:pdf},
isbn = {0027-8424},
issn = {0027-8424},
journal = {Proceedings of the National Academy of Sciences},
number = {43},
pages = {18073--18078},
pmid = {19805183},
title = {{The evolution of antiherbivore defenses and their contribution to species coexistence in the tropical tree genus Inga}},
url = {http://www.pnas.org/cgi/doi/10.1073/pnas.0904786106},
volume = {106},
year = {2009}
}
@article{Lankau2007,
abstract = {Plant defense traits often show high levels of genetic variation, despite clear impacts on plant fitness. This variation may be partly maintained by trade-offs in the defense against multiple herbivore species, for example between generalists and coevolved specialists. Despite a long-standing discussion in the literature on the subject, no study to date has specifically manipulated specialist and generalist herbivores independently of one another to determine whether the two guilds exert opposing selection pressures on specific defensive traits. In two separate experiments, the dominant specialist and generalist herbivores of Brassica nigra were independently manipulated to test whether the composition of the herbivore community altered the direction of selection on a major defensive trait of the plant, sinigrin concentration. It was found that generalist damage was negatively correlated but specialist loads were positively correlated with increasing sinigrin concentrations; and sinigrin concentration was favored when specialists were removed, disfavored (past an intermediate point) when generalists were removed and selectively neutral when plants faced both generalists and specialists.},
author = {Lankau, Richard A.},
doi = {10.1111/j.1469-8137.2007.02090.x},
isbn = {0028-646X},
issn = {0028646X},
journal = {New Phytologist},
keywords = {Brassica nigra,Brevicoryne brassicae,Chemical defense,Generalist,Glucosinolates,Herbivore,Selection,Specialist},
number = {1},
pages = {176--184},
pmid = {17547677},
title = {{Specialist and generalist herbivores exert opposing selection on a chemical defense}},
url = {https://nph.onlinelibrary.wiley.com/doi/full/10.1111/j.1469-8137.2007.02090.x},
volume = {175},
year = {2007}
}
@article{Gomez2010,
abstract = {Ecological interactions are crucial to understanding both the ecology and the evolution of organisms. Because the phenotypic traits regulating species interactions are largely a legacy of their ancestors, it is widely assumed that ecological interactions are phylogenetically conserved, with closely related species interacting with similar partners. However, the existing empirical evidence is inadequate to appropriately evaluate the hypothesis of phylogenetic conservatism in ecological interactions, because it is both ecologically and taxonomically biased. In fact, most studies on the evolution of ecological interactions have focused on specialized organisms, such as some parasites or insect herbivores, belonging to a limited subset of the overall tree of life. Here we study the evolution of host use in a large and diverse group of interactions comprising both specialist and generalist acellular, unicellular and multicellular organisms. We show that, as previously found for specialized interactions, generalized interactions can be evolutionarily conserved. Significant phylogenetic conservatism of interaction patterns was equally likely to occur in symbiotic and non-symbiotic interactions, as well as in mutualistic and antagonistic interactions. Host-use differentiation among species was higher in phylogenetically conserved clades, irrespective of their generalization degree and taxonomic position within the tree of life. Our findings strongly suggest a shared pattern in the organization of biological systems through evolutionary time, mediated by marked conservatism of ecological interactions among taxa.},
author = {G{\'{o}}mez, Jos{\'{e}} M. and Verd{\'{u}}, Miguel and Perfectti, Francisco},
doi = {10.1038/nature09113},
isbn = {1476-4687 (Electronic)$\backslash$n0028-0836 (Linking)},
issn = {00280836},
journal = {Nature},
keywords = {Animals,Biological Evolution,Ecosystem,Host-Parasite Interactions,Phylogeny,Symbiosis,Symbiosis: physiology},
number = {7300},
pages = {918--921},
pmid = {20520609},
title = {{Ecological interactions are evolutionarily conserved across the entire tree of life}},
url = {http://www.ncbi.nlm.nih.gov/pubmed/20520609},
volume = {465},
year = {2010}
}
@article{Rasmussen2013,
abstract = {Most ecological networks are analysed as static structures, where all observed species and links are present simultaneously. However, this is over-simplified, because networks are temporally dynamical. We resolved an arctic, entire-season plant-flower visitor network into a temporal series of 1-day networks and compared the properties with its static equivalent based on data pooled over the entire season. Several properties differed. The nested link pattern in the static network was blurred in the dynamical version, because the characteristic long nestedness tail of flower-visitor specialists got stunted in the dynamical networks. This tail comprised a small food web of pollinators, parasitoids and hyper-parasitoids. The dynamical network had strong time delays in the transmission of direct and indirect effects among species. Twenty percent of all indirect links were impossible in the dynamical network. Consequently, properties and thus also robustness of ecological networks cannot be deduced from the static topology alone.},
author = {Rasmussen, Claus and Dupont, Yoko L. and Mosbacher, Jesper B. and Trj{\o}elsgaard, Kristian and Olesen, Jens M.},
doi = {10.1371/journal.pone.0081694},
isbn = {1932-6203},
issn = {19326203},
journal = {PLoS ONE},
number = {12},
pages = {e81694},
pmid = {24324718},
title = {{Strong impact of temporal resolution on the structure of an ecological network}},
url = {http://dx.plos.org/10.1371/journal.pone.0081694},
volume = {8},
year = {2013}
}
@article{Nuismer2013,
abstract = {Although coevolution is widely recognized as an important evolutionary process for pairs of reciprocally specialized species, its importance within species-rich communities of generalized species has been questioned. Here we develop and analyze mathematical models of mutualistic communities, such as those between plants and pollinators or plants and seed-dispersers to evaluate the importance of coevolutionary selection within complex communities. Our analyses reveal that coevolutionary selection can drive significant changes in trait distributions with important consequences for the network structure of mutualistic communities. One such consequence is greater connectance caused by an almost invariable increase in the rate of mutualistic interaction within the community. Another important consequence is altered patterns of nestedness. Specifically, interactions mediated by a mechanism of phenotype matching tend to be antinested when coevolutionary selection is weak and even more strongly antinested as increasing coevolutionary selection favors the emergence of reciprocal specialization. In contrast, interactions mediated by a mechanism of phenotype differences tend to be nested when coevolutionary selection is weak, but less nested as increasing coevolutionary selection favors greater levels of generalization in both plants and animals. Taken together, our results show that coevolutionary selection can be an important force within mutualistic communities, driving changes in trait distributions, interaction rates, and even network structure.},
archivePrefix = {arXiv},
arxivId = {arXiv:1011.1669v3},
author = {Nuismer, Scott L. and Jordano, Pedro and Bascompte, Jordi},
doi = {10.1111/j.1558-5646.2012.01801.x},
eprint = {arXiv:1011.1669v3},
isbn = {1558-5646},
issn = {00143820},
journal = {Evolution},
keywords = {Dispersal,Mutualism,Nestedness,Network structure,Pollination},
number = {2},
pages = {338--354},
pmid = {23356608},
title = {{Coevolution and the architecture of mutualistic networks}},
volume = {67},
year = {2013}
}
@article{Guimaraes2011,
abstract = {Ecology Letters (2011) 14: 877-885 ABSTRACT: A major current challenge in evolutionary biology is to understand how networks of interacting species shape the coevolutionary process. We combined a model for trait evolution with data for twenty plant-animal assemblages to explore coevolution in mutualistic networks. The results revealed three fundamental aspects of coevolution in species-rich mutualisms. First, coevolution shapes species traits throughout mutualistic networks by speeding up the overall rate of evolution. Second, coevolution results in higher trait complementarity in interacting partners and trait convergence in species in the same trophic level. Third, convergence is higher in the presence of super-generalists, which are species that interact with multiple groups of species. We predict that worldwide shifts in the occurrence of super-generalists will alter how coevolution shapes webs of interacting species. Introduced species such as honeybees will favour trait convergence in invaded communities, whereas the loss of large frugivores will lead to increased trait dissimilarity in tropical ecosystems.},
archivePrefix = {arXiv},
arxivId = {arXiv:1012.5461v2},
author = {Guimar{\~{a}}es, Paulo R. and Jordano, Pedro and Thompson, John N.},
doi = {10.1111/j.1461-0248.2011.01649.x},
eprint = {arXiv:1012.5461v2},
isbn = {1461-023X},
issn = {14610248},
journal = {Ecology Letters},
keywords = {Coevolution,Complementarity,Convergence,Ecological networks,Evolutionary cascades,Generalists,Mutualisms,Pollination,Seed dispersal,Small-world networks},
number = {9},
pages = {877--885},
pmid = {21749596},
title = {{Evolution and coevolution in mutualistic networks}},
url = {http://doi.wiley.com/10.1111/j.1461-0248.2011.01649.x},
volume = {14},
year = {2011}
}
@article{Poisot2015,
abstract = {Community ecology is tasked with the considerable challenge of predicting the structure, and properties, of emerging ecosystems. It requires the ability to understand how and why species interact, as this will allow the development of mechanism-based predictive models, and as such to better characterize how ecological mechanisms act locally on the existence of inter-specific interactions. Here we argue that the current conceptualization of species interaction networks is ill-suited for this task. Instead, we propose that future research must start to account for the intrinsic variability of species interactions, then scale up from here onto complex networks. This can be accomplished simply by recognizing that there exists intra-specific variability, in traits or properties related to the establishment of species interactions. By shifting the scale towards population-based processes, we show that this new approach will improve our predictive ability and mechanistic understanding of how species interact over large spatial or temporal scales. $\backslash$nSynthesis$\backslash$nAlthough species interactions are the backbone of ecological communities, we have little insights on how (and why) they vary through space and time. In this article, we build on existing empirical literature to show that the same species may happen to interact in different ways when their local abundances vary, their trait distribution changes, or when the environment affects either of these factors. We discuss how these findings can be integrated in existing frameworks for the analysis and simulation of species interactions.},
archivePrefix = {arXiv},
arxivId = {10.1101/001677},
author = {Poisot, Timoth{\'{e}}e and Stouffer, Daniel B. and Gravel, Dominique},
doi = {10.1111/oik.01719},
eprint = {001677},
file = {:Users/alyssacirtwill/Documents/Papers/Poisot, Stouffer, Gravel{\_}2015{\_}Oikos.pdf:pdf},
isbn = {0000000207355},
journal = {Oikos},
number = {3},
pages = {243--251},
pmid = {22382098},
primaryClass = {10.1101},
title = {{Beyond species: Why ecological interaction networks vary through space and time}},
url = {http://doi.wiley.com/10.1111/oik.01719},
volume = {124},
year = {2015}
}
@article{Thompson2006a,
abstract = {Quantitative analysis of a network of plant- animal interactions reveal new organizing prin- ciples, including how asymmetric relations sta- bilize the coevolution of the whole network.},
author = {Thompson, John N.},
doi = {10.1126/science.1126904},
isbn = {0036-8075},
issn = {00368075},
journal = {Science},
number = {5772},
pages = {372--373},
pmid = {16627726},
title = {{Mutualistic webs of species}},
volume = {312},
year = {2006}
}
@article{APG2009,
abstract = {Clostridium clariflavum is an anaerobic, thermophilic, Gram-positive bacterium, capable of growth on crystalline cellulose as a single carbon source. The genome of C. clariflavum has been sequenced to completion, and numerous cellulosomal genes were identified, including putative scaffoldin and enzyme subunits.},
author = {Bremer, B. and Bremer, K. and Chase, M. W. and Fay, M. F. and Reveal, J. L. and Bailey, L. H. and Soltis, D. E. and Soltis, P. S. and Stevens, P. F.},
doi = {10.1111/j.1095-8339.2009.00996.x},
isbn = {1095-8339},
issn = {00244074},
journal = {Botanical Journal of the Linnean Society},
keywords = {APG system,Angiosperm classification,Classification system,DNA phylogenetics,Phylogenetic classification},
number = {2},
pages = {105--121},
pmid = {26413154},
title = {{An update of the Angiosperm Phylogeny Group classification for the orders and families of flowering plants: APG III}},
volume = {161},
year = {2009}
}
@article{Webb2000,
abstract = {Because of the correlation expected between the phylogenetic relatedness of two taxa and their net ecological similarity, a measure of the overall phylogenetic relatedness of a community of interacting organisms can be used to investigate the contemporary ecological processes that structure community composition. I describe two indices that use the number of nodes that separate taxa on a phylogeny as a measure of their phylogenetic relatedness. As an example of the use of these indices in community analysis, I compared the mean observed net relatedness of trees ({\textgreater}/=10 cm diameter at breast height) in each of 28 plots (each 0.16 ha) in a Bornean rain forest with the net relatedness expected if species were drawn randomly from the species pool (of the 324 species in the 28 plots), using a supertree that I assembled from published sources. I found that the species in plots were more phylogenetically related than expected by chance, a result that was insensitive to various modifications to the basic methodology. I tentatively infer that variation in habitat among plots causes ecologically more similar species to co-occur within plots. Finally, I suggest a range of applications for phylogenetic relatedness measures in community analysis.},
archivePrefix = {arXiv},
arxivId = {arXiv:1011.1669v3},
author = {Webb, Campbell O.},
doi = {10.1086/303378},
eprint = {arXiv:1011.1669v3},
isbn = {00030147},
issn = {0003-0147},
journal = {The American Naturalist},
keywords = {and,dynamics of local communities,in the species composition,net ecological similarity,of interacting organisms,phylogenetic conservatism,supertree,taxonomic diversity,the search for patterns,tropical rain forest},
number = {2},
pages = {145--155},
pmid = {10856198},
title = {{Exploring the Phylogenetic Structure of Ecological Communities: An Example for Rain Forest Trees}},
url = {http://www.journals.uchicago.edu/doi/10.1086/303378},
volume = {156},
year = {2000}
}
@article{Siepielski2010,
abstract = {The high levels of species diversity observed within many biological communities are captivating, yet the mechanisms that may maintain such diversity remain elusive. Many of the phenotypic differences observed among species cause interspecific tradeoffs that ultimately act to maintain diversity through niche-based coexistence. In contrast, neutral community theory argues that phenotypic differences among species do not contribute to maintaining species diversity because species are ecologically equivalent. Here we provide experimental and observational field evidence that two phylogenetically very distant Enallagma species appear to be ecologically equivalent to one another. Experimental abundance manipulations showed that each species gains no demographic advantage at low relative abundance, whereas manipulations of total Enallagma abundance resulted in large increases in per capita mortality and large decreases in growth for both species. Moreover, demographic rates and relative abundances of multiple Enallagma species were uncorrelated with major environmental gradients in an observational study of 20 natural lakes. These are the expected patterns if species are ecologically equivalent. However, these results do not imply that all damselflies in these lakes are ecologically identical. Previous experimental results have demonstrated the operation of strong coexistence mechanisms maintaining Enallagma and its sister-genus Ischnura in these littoral food webs. Combined with a simple theoretical model we present, these results taken together show how both neutral and niche dynamics can jointly structure communities.},
author = {Siepielski, Adam M. and Hung, Keng Lou and Bein, Eben E. B. and McPeek, Mark A.},
doi = {10.1890/09-0609.1},
isbn = {0012-9658},
issn = {00129658},
journal = {Ecology},
keywords = {Coexistence,Community structure,Damselflies,Enallagma spp,Littoral zones,Neutral community dynamics,New hampshire,Niche,Species diversity,USA},
number = {3},
pages = {847--857},
pmid = {20426342},
title = {{Experimental evidence for neutral community dynamics governing an insect assemblage}},
volume = {91},
year = {2010}
}
@article{ade4,
abstract = {Multivariate analyses are well known and widely used to identify and understand structures of ecological communities. The ade4 package for the R statistical environment proposes a great number of multivariate methods. Its implementation follows the tradition of the French school of ”Analyse des Donn´ees” and is based on the use of the duality diagram. We present the theory of the duality diagram and discuss its implementation in ade4. Classes and main functions are presented. An example is given to illustrate the ade4 philosophy},
author = {Swanson, K. S. and Suchodolski, J. S. and Turnbaugh, P. J.},
doi = {10.2527/jas.2011-3873},
isbn = {1525-3163 (Electronic)$\backslash$n0021-8812 (Linking)},
issn = {00218812},
journal = {Journal of Animal Science},
keywords = {ade4,duality diagram,ecological data,multivariate analysis,ordination},
number = {5},
pages = {1496--1497},
pmid = {21521817},
title = {{Companion animals symposium: Microbes and health}},
url = {http://www.jstatsoft.org/},
volume = {89},
year = {2011}
}
@article{Johnson2014,
abstract = {Plant species vary greatly in defenses against herbivores, but existing theory has struggled to explain this variation. Here, we test how phylogenetic relatedness, tradeoffs, trait syndromes, and sexual reproduction affect the macroevolution of defense. To examine the macroevolution of defenses, we studied 26 Oenothera (Onagraceae) species, combining chemistry, comparative phylogenetics and experimental assays of resistance against generalist and specialist herbivores. We detected dozens of phenolic metabolites within leaves, including ellagitannins (ETs), flavonoids, and caffeic acid derivatives (CAs). The concentration and composition of phenolics exhibited low to moderate phylogenetic signal. There were clear negative correlations between multiple traits, supporting the prediction of allocation tradeoffs. There were also positively covarying suites of traits, but these suites did not strongly predict resistance to herbivores and thus did not act as defensive syndromes. By contrast, specific metabolites did correlate with the performance of generalist and specialist herbivores. Finally, that repeated losses of sex in Oenothera was associated with the evolution of increased flavonoid diversity and altered phenolic composition. These results show that secondary chemistry has evolved rapidly during the diversification of Oenothera. This evolution has been marked by allocation tradeoffs between traits, some of which are related to herbivore performance. The repeated loss of sex appears also to have constrained the evolution of plant secondary chemistry, which may help to explain variation in defense among plants.},
author = {Johnson, Marc T. J. and Ives, Anthony R. and Ahern, Jeffrey and Salminen, Juha Pekka},
doi = {10.1111/nph.12763},
isbn = {1469-8137 (Electronic)$\backslash$r0028-646X (Linking)},
issn = {14698137},
journal = {New Phytologist},
keywords = {Apomixis,Coevolution,Evening primrose,Phenolics,Phylogenetic generalized least squares,Plant defense,Plant-herbivore interaction,Tannins},
number = {1},
pages = {267--279},
pmid = {24634986},
title = {{Macroevolution of plant defenses against herbivores in the evening primroses}},
volume = {203},
year = {2014}
}
@article{taxize1,
abstract = {All species are hierarchically related to one another, and we use taxonomic names to label the nodes in this hierarchy. Taxonomic data is becoming increasingly available on the web, but scientists need a way to access it in a programmatic fashion that's easy and reproducible. We have developed taxize, an open-source software package (freely available from http://cran.r-project.org/web/packages/taxize/index.html) for the R language. taxize provides simple, programmatic access to taxonomic data for 13 data sources around the web. We discuss the need for a taxonomic toolbelt in R, and outline a suite of use cases for which taxize is ideally suited (including a full workflow as an appendix). The taxize package facilitates open and reproducible science by allowing taxonomic data collection to be done in the open-source R platform.},
author = {Chamberlain, Scott A. and Szocs, Eduard},
doi = {10.12688/f1000research.2-191.v2},
issn = {2046-1402},
journal = {F1000Research},
pages = {1--25},
pmid = {24555091},
title = {{taxize - taxonomic search and retrieval in R}},
url = {http://f1000research.com/articles/2-191/v1},
year = {2013}
}
@article{Smirnov1961,
abstract = {The submersed sphagnum is populated by Odonata and Chironomidae larvae, Hemiptera, Cladocera, Cyclopoida, Rotatoria and Rhizopoda. Sphagnum-eaters might be found among Nematocera larvae. To find them the food cymposition of this group was studied. The dissections demonstrated that of 9 species of Nematocera larvae Sphagnum was present in the food only of Psectrocladius ex gr. psilopterus, the quantity of Sphagnum being 0.16 of the food volume. The main food of non predating Nematocera larvae living in the submersed Sphagnum consists of the algae and the detritus. The emersed Sphagnum is also eaten but little. The springtails dominating here feed mainly on the fungi growing on the decomposing Sphagnum. Our data and the data by other authors gave a possibility to respresent the main food interrelations in the sphagnous bogs as shown in fig. 1.},
author = {Smirnov, N. N.},
doi = {10.1007/BF00040419},
issn = {00188158},
journal = {Hydrobiologia},
number = {1-2},
pages = {175--182},
title = {{Food cycles in sphagnous bogs}},
volume = {17},
year = {1961}
}
@article{Brown1964,
author = {Brown, A},
issn = {0038-2353},
journal = {South African Journal of Science},
number = {2},
pages = {35--41},
title = {{Food-relationships on the intertidal sandy beaches of the Cape Peninsula}},
volume = {60},
year = {1964}
}
@article{ape,
abstract = {Description Functions for reading, writing, plotting, and manipulating phylogenetic trees, analyses of comparative data in a phylogenetic framework, ancestral character analyses, analyses of diversification and macroevolution, computing distances from allelic and ... $\backslash$n},
author = {Paradis, E and Blomberg, S and Bolker, B and Claude, J},
doi = {10.1093/bioinformatics/btg412},
isbn = {9781461417422},
journal = {Bioinformatics},
number = {2},
pages = {289--290},
title = {{APE: analyses of phylogenetics and evolution in R language}},
url = {http://cran.itam.mx/web/packages/ape/ape.pdf{\%}5Cnpapers3://publication/uuid/B383216F-AD44-488B-835D-87C4CF0D7B88},
volume = {20},
year = {2012}
}
@article{Webb2008,
abstract = {MOTIVATION: The increasing availability of phylogenetic and trait data for communities of co-occurring species has created a need for software that integrates ecological and evolutionary analyses. Capabilities: Phylocom calculates numerous metrics of phylogenetic community structure and trait similarity within communities. Hypothesis testing is implemented using several null models. Within the same framework, it measures phylogenetic signal and correlated evolution for species traits. A range of utility functions allow community and phylogenetic data manipulation, tree and trait generation, and integration into scientific workflows. Availability: Open source at: http://phylodiversity.net/phylocom/.},
author = {Webb, Campbell O. and Ackerly, David D. and Kembel, Steven W.},
doi = {10.1093/bioinformatics/btn358},
isbn = {1367-4811 (Electronic)$\backslash$r1367-4803 (Linking)},
issn = {13674803},
journal = {Bioinformatics},
number = {18},
pages = {2098--2100},
pmid = {18678590},
title = {{Phylocom: Software for the analysis of phylogenetic community structure and trait evolution}},
volume = {24},
year = {2008}
}
@article{DallaRiva2015,
abstract = {Increasing evidence suggests that an appropriate model for food webs, the network of feeding links in a community of species, should take into account the inherent variability of ecological interactions. Harnessing this variability, we will show that it is useful to interpret empirically observed food webs as realisations of a family of stochastic processes, namely random dot-product graph models. These models provide an ideal extension of food-web models beyond the limitations of current deterministic or partially probabilistic models. As an additional benefit, our RDPG framework enables us to identify the pairwise distance structure given by species' functional food-web traits: this allows for the natural emergence of ecologically meaningful species groups. Lastly, our results suggest the notion that the evolutionary signature in food webs is already detectable in their stochastic backbones, while the contribution of their fine wiring is arguable.},
author = {{Dalla Riva}, Giulio V. and Stouffer, Daniel B.},
doi = {10.1111/oik.02305},
isbn = {00301299},
issn = {16000706},
journal = {Oikos},
number = {4},
pages = {446--456},
title = {{Exploring the evolutionary signature of food webs' backbones using functional traits}},
volume = {125},
year = {2016}
}
@book{taxize2,
abstract = {Interacts with a suite of web 'APIs' for taxonomic tasks, such as verifying species names, and getting taxonomic 'hierarchies'.},
address = {R package version 0.3.0},
author = {Chamberlain, Scott and Szoecs, Eduard and Foster, Zachary and Boettiger, Carl and Ram, Karthik and Bartomeus, Ignasi and Baumgartner, John and O'Donnell, James},
title = {{taxize: Taxonomic Information from Around the Web}},
url = {https://cran.r-project.org/web/packages/taxize/index.html},
year = {2016}
}
@article{Tilzer1988,
abstract = {Annual phytoplankton productivity in Lake Constance is about 300 g C m−2, a value typical for mesoeutrophic lakes. Seasonal variations in phytoplankton biomass and productivity are exceptionally great because of a sequence of factors controlling the production process. During winter productivity is controlled by low energy inputs and high respiratory losses due to deep water column mixing. Biomass is low and water transparancy high. The spring phytoplankton growth is triggered by the thermal stabilization of the water column. The summer phytoplankton biomass maximum mainly depends on phosphorus availability. However, biomass yields comprise only 15–20{\%} of values to be expected from the Redfield ratio because large proportions of POM are detritus and non-algal biota. Moreover, sedimentation during the second half of the year removes biomass from the euphotic zone. Water transparency and thus vertical distribution of algal photosynthesis is highly dependent on phytoplankton biomass. Self-shading causes considerably smaller seasonal variations in areal biomass and photosynthetic rates than in volume-based values. By light-shade adaptation effects of seasonal fluctuations in mean daily surface radiance fluxes on algal photosynthesis can to a significant extent be compensated for. At any given level of biomass daylength is the major determinant of daily production rates.},
author = {Tilzer, Max M. and Beese, B{\"{a}}rbel},
doi = {10.1007/BF02538370},
isbn = {0036-7842},
issn = {00367842},
journal = {Swiss Journal of Hydrology},
number = {1},
pages = {1--39},
title = {{The seasonal productivity cycle of phytoplankton and controlling factors in Lake Constance}},
volume = {50},
year = {1988}
}
@article{Robertson1999,
abstract = {The ability of flower visitors to monitor returns when collecting pollen from flowers has been seldom studied despite the importance of pollen as a food resource, particularly for bees. Californian populations of Mimulus guttatus are polymorphic for pollen quality: many plants produce a high proportion of cytoplasmless pollen grains that render the grains incapable of fertilizing ovules or of supporting bees nutritionally. We found that different genotypes maintained a consistent proportion of inviable pollen within a genotype and over time. The number of pollen grains per flower was also consistent within a plant at each date but declined over time. We studied the ability of British bumble bees (Bombus spp.) to discriminate among plants of Mimulus guttatus on the basis of pollen quality and quantity at three scales: indoors with choices of two genotypes, in outdoor plots of several genotypes that varied in pollen quality, and outdoors at a whole-patch scale where two patches of plants differed in quality. We found that bees could discriminate among plants on the basis of pollen quality provided that flowers still retained most of the pollen. In the two-genotype trials, bees chose genotypes primarily on the quantity of viable pollen, and nectar was much less important. Similarly, where patches of low- and high-pollen quality plants were established, bees responded by visiting the high-quality patch more often and by visiting more flowers within the patch. However, the results from the outdoor plots that contained genotypes of varying phenotypes were inconsistent. A meta-analysis of a large number of separate plots showed that the overall correlation between visitation rate and pollen quality was significant, but variation among plots was also significant. A possible explanation for this inconsistency was suggested in a greenhouse trial in which we showed that, when foraging density was high, depletion of the standing crop of pollen happened quickly, and this reduced the ability of the foragers to choose the higher-quality genotypes. The results have implications for the evolution of pollen production in Mimulus guttatus and reward production in other plants.},
author = {Robertson, Alastair W and Mountjoy, Claire and Faulkner, Brian E and Roberts, Matthew V and Macnair, Mark R},
doi = {10.1890/0012-9658(1999)080[2594:BBSOMG]2.0.CO;2},
isbn = {0012-9658},
issn = {00129658},
journal = {Ecology},
keywords = {Bombus,Bumble bees,Mimulus guttatus,Pollen number,Pollen quality,Pollinator behavior,Pollinator-mediated selection,Reward depletion},
number = {8},
pages = {2594--2606},
title = {{Bumble bee selection of $\backslash$emph{\{}Mimulus guttatus{\}} flowers: the effects of pollen quality and reward depletion}},
volume = {80},
year = {2008}
}
@article{Siepielski2009,
abstract = {The raw material for evolution is variation. Consequently, identifying the factors that generate, maintain, and erode phenotypic and genetic variation in ecologically important traits within and among populations is important. Although persistent directional or stabilizing selection can deplete variation, spatial variation in conflicting directional selection can enhance variation. Here, we present evidence that phenotypic variation in limber pine (Pinus flexilis) cone structure is enhanced by conflicting selection pressures exerted by its mutualistic seed disperser (Clark's nutcracker Nucifraga columbiana) and an antagonistic seed predator (pine squirrel Tamiasciurus spp.). Phenotypic variation in cone structure was bimodal and about two times greater where both agents of selection co-occurred than where one (the seed predator) was absent. Within the region where both agents of selection co-occurred, bimodality in cone structure was pronounced where there appears to be a mosaic of habitats with some persistent habitats supporting only the seed disperser. These results indicate that conflicting selection stemming from spatial variation in community diversity can enhance phenotypic variation in ecologically important traits.},
author = {Siepielski, Adam M. and Benkman, Craig W.},
doi = {10.1111/j.1558-5646.2009.00867.x},
isbn = {0014-3820},
issn = {00143820},
journal = {Evolution},
keywords = {Bimodality,Nucifraga columbiana,Phenotypic selection,Phenotypic variation,Pinus flexilis,Seed dispersal,Seed predation,Tamiasciurus},
number = {4},
pages = {1120--1128},
pmid = {19817846},
title = {{Conflicting selection from an antagonist and a mutualist enhances phenotypic variation in a plant}},
url = {http://doi.wiley.com/10.1111/j.1558-5646.2009.00867.x},
volume = {64},
year = {2010}
}
@article{Schemske1999,
abstract = {A paradigm of evolutionary biology is that adaptation and reproductive isolation are caused by a nearly infinite number of mutations of individually small effect. Here, we test this hypothesis by investigating the genetic basis of pollinator discrimination in two closely related species of monkeyflowers that differ in their major pollinators. This system provides a unique opportunity to investigate the genetic architecture of adaptation and speciation because floral traits that confer pollinator specificity also contribute to premating reproductive isolation. We asked: (i) What floral traits cause pollinator discrimination among plant species? and (ii) What is the genetic basis of these traits? We examined these questions by using data obtained from a large-scale field experiment where genetic markers were employed to determine the genetic basis of pollinator visitation. Observations of F2 hybrids produced by crossing bee-pollinated Mimulus lewisii with hummingbird-pollinated Mimulus cardinalis revealed that bees preferred large flowers low in anthocyanin and carotenoid pigments, whereas hummingbirds favored nectar-rich flowers high in anthocyanins. An allele that increases petal carotenoid concentration reduced bee visitation by 80{\%}, whereas an allele that increases nectar production doubled hummingbird visitation. These results suggest that genes of large effect on pollinator preference have contributed to floral evolution and premating reproductive isolation in these monkeyflowers. This work contributes to growing evidence that adaptation and reproductive isolation may often involve major genes.},
author = {Schemske, D. W. and Bradshaw, H. D.},
doi = {10.1073/pnas.96.21.11910},
isbn = {0027-8424},
issn = {0027-8424},
journal = {Proceedings of the National Academy of Sciences},
number = {21},
pages = {11910--11915},
pmid = {10518550},
title = {{Pollinator preference and the evolution of floral traits in monkeyflowers (Mimulus)}},
url = {http://www.pnas.org/cgi/doi/10.1073/pnas.96.21.11910},
volume = {96},
year = {1999}
}
@article{Pearse2013a,
abstract = {* Ecologists increasingly wish to use phylogenies, but are hampered by the technical challenge of phylogeny estimation. * We present phyloGenerator, an open-source, stand-alone Python program, that makes use of pre-existing sequence data and taxonomic information to largely automate the estimation of phylogenies. * phyloGenerator allows nonspecialists to quickly and easily produce robust, repeatable, and defensible phylogenies without requiring an extensive knowledge of phylogenetics. Experienced phylogeneticists may also find it useful as a tool to conduct exploratory analyses. * phyloGenerator performs a number of ‘sanity checks' on users' output, but users should still check their outputs carefully; we give some advice on how to do so. * By linking a number of tools in a common framework, phyloGenerator is a step towards an open, reproducible phylogenetic workflow. * Bundled downloads for Windows and Mac OSX, along with the source code and an install script for Linux, can be found at http://willpearse.github.io/phyloGenerator (note the capital ‘G').},
author = {Pearse, William D. and Purvis, Andy},
doi = {10.1111/2041-210X.12055},
isbn = {2041-210X},
issn = {2041210X},
journal = {Methods in Ecology and Evolution},
keywords = {Community phylogenetics,Comparative analysis,Phylogenetics,Phylogeny construction,Sequence alignment},
number = {7},
pages = {692--698},
title = {{phyloGenerator: An automated phylogeny generation tool for ecologists}},
url = {http://doi.wiley.com/10.1111/2041-210X.12055},
volume = {4},
year = {2013}
}
@article{Tilzer1988a,
abstract = {Annual phytoplankton productivity in Lake Constance is about 300 g C m−2, a value typical for mesoeutrophic lakes. Seasonal variations in phytoplankton biomass and productivity are exceptionally great because of a sequence of factors controlling the production process. During winter productivity is controlled by low energy inputs and high respiratory losses due to deep water column mixing. Biomass is low and water transparancy high. The spring phytoplankton growth is triggered by the thermal stabilization of the water column. The summer phytoplankton biomass maximum mainly depends on phosphorus availability. However, biomass yields comprise only 15–20{\%} of values to be expected from the Redfield ratio because large proportions of POM are detritus and non-algal biota. Moreover, sedimentation during the second half of the year removes biomass from the euphotic zone. Water transparency and thus vertical distribution of algal photosynthesis is highly dependent on phytoplankton biomass. Self-shading causes considerably smaller seasonal variations in areal biomass and photosynthetic rates than in volume-based values. By light-shade adaptation effects of seasonal fluctuations in mean daily surface radiance fluxes on algal photosynthesis can to a significant extent be compensated for. At any given level of biomass daylength is the major determinant of daily production rates.},
author = {Tilzer, Max M. and Beese, B{\"{a}}rbel},
doi = {10.1007/BF02538370},
isbn = {0036-7842},
issn = {00367842},
journal = {Swiss Journal of Hydrology},
number = {1},
pages = {1--39},
title = {{The seasonal productivity cycle of phytoplankton and controlling factors in Lake Constance}},
volume = {50},
year = {1988}
}
@article{Zeng2008,
abstract = {Particulate organic matter (POM), dissolved organic matter (DOM), bacteria and cladoceran were sampled seasonally at Zhihugang Estuary and Lake Center in Taihu Lake. The $\delta$13C of the four organic matter fractions showed consistent temporal variation, with heaviest values in summer and lower at other times of the year. The cladoceran $\delta$13C showed a significant correlation with that of POM, reflecting a heavy dietary dependence on POM during the study period. The bacteria became enriched in 13C compared with that of DOM throughout the sampling dates, although no significant relationship was found between the two fractions. $\delta$13C values of POM, cladoceran and bacteria were all negative significantly correlated with oxidation and reduction potential (ORP), and specific conductivity (SpCond). As for $\delta$15N, the seasonal pattern of food web components was variable. The POM $\delta$15N signature exhibited the most enriched isotope ratios during the summer months when dissolved inorganic nitrogen (DIN) nutrients were at their lowest concentrations. The consumption of DIN in summer can explain in part the progressive accumulation of heavy nitrogen isotopes during this period. Spatially, $\delta$13C and $\delta$15N of the food web components were all slightly depleted at Estuary than that at Lake Center during the study period, possibly due to large allochthonous inputs at Zhihugang Estuary. Relatively wide ranges of stable isotopic values from both sites suggest that seasonality should be considered when attempting to establish food web structures in a eutrophic lake.},
author = {Zeng, Qingfei and Kong, Fan Xiang and Zhang, En Lou and Tan, Xiao and Wu, Xiao Dong},
doi = {10.1051/limn:2008019},
isbn = {0003-4088},
issn = {0003-4088},
journal = {Annales de Limnologie-International Journal of Limnology},
keywords = {DOM,Lake Taihu (China),POM,bacteria,cladoceran (Bosmina spp.),cladoceran (Ceriodaphnia spp.),cladoceran (Daphnia spp.),d13C,d15N,eutrophic ecosystem,pelagic food web,seasonal cycle,shallow lake,spatial variation},
number = {1},
pages = {1--6},
title = {{Seasonality of stable carbon and nitrogen isotopes within the pelagic food web of Taihu Lake}},
url = {http://journals.cambridge.org/action/displayAbstract?fromPage=online{\&}aid=8146403},
volume = {44},
year = {2008}
}
@article{Liem1990,
author = {Modes, Aquatic Versus Terrestrial Feeding and Impacts, Possible and Ecology, Trophic},
journal = {American Zoologist},
keywords = {DIET,ECOMORPHOLOGY,FEEDING,FUNCTIONAL MORPHOLOGY,OBTER,VERTEBRATES},
pages = {1--15},
title = {{Seminars on Science: Diversity of Fishes}},
volume = {30},
year = {1990}
}
@article{Meding2001,
abstract = {We compiled data for 23 North American temperate zone lakes to assess three alternative winter O2 depletion models for estimating O2 dynamics from freezing to thawing. Dissolved O2 concentrations were constant or declined slightly for an average of 40 days after freezing and then declined rapidly. Once O2 concentrations reached 1– 3 mg{\textperiodcentered}L–1, consumption slowed. No model that we fit captured O2 dynamics shortly after freezing. The best fit was a one-pool exponential decay model after one to four initial data were removed. Photosynthesis and freeze-out estimates suggest that O2 inputs are more important in shallow than in deep lakes. Oxygen decay rates (k) correlated with morphometry in shallow lakes and chlorophyll a, Secchi depth, and the sediment surface area to volume ratio in deep lakes. We hypothesize that the failure of chlorophyll a to correlate with k in shallow lakes is because macrophytes are the primary source of decaying organic matter but have not been included in assessments of winter O2 depletion. Thus, some processes in deep lakes cannot simply be scaled to smaller scales in shallow lakes.},
author = {Meding, Marianne E. and Jackson, Leland J.},
doi = {10.1139/cjfas-58-9-1727},
isbn = {0706-652X},
issn = {12057533},
journal = {Canadian Journal of Fisheries and Aquatic Sciences},
number = {9},
pages = {1727--1736},
title = {{Biological implications of empirical models of winter oxygen depletion}},
url = {http://www.nrc.ca/cgi-bin/cisti/journals/rp/rp2{\_}abst{\_}e?cjfas{\_}f01-109{\_}58{\_}ns{\_}nf{\_}cjfas58-01},
volume = {58},
year = {2001}
}
@incollection{Winterbourn1997,
abstract = {In this chapter, the nature of stream invertebrate communities in the South Island, New Zealand is considered in relation to concepts of disturbance and stability. Disturbance is widely regarded as a primary organizing factor in physically "harsh" stream environments, but the definition of what constitutes a disturbance is not always clear. Both environmental and faunal stability can also be difficult to define and measure, and are influenced by the spatial and temporal scales of interest, and historical factors. To illustrate the relationships between stability of stream faunas and the role of disturbance as a regulatory force, four case studies are presented. Two studies consider the persistence and resilience of invertebrate faunas in streams of contrasting and similar physical stability and flow variability. The other two studies examine faunal responses to catchment-scale disturbances (logging and the development of land for forestry) and the temporal stability of stream faunas following afforestation with exotic conifers. These and other studies indicate that the faunas of many South Island mountain streams are dominated by the same widely distributed species characterized by life history flexibility, lack of habitat specificity, and strong colonizing abilities. They also tend to exhibit a high degree of stability in species composition over time despite many streams having physically unstable beds, variable and unpredictable discharge patterns, and changing vegetational settings. It is hypothesized that the resilience of New Zealand mountain stream faunas, and the lack of strong habitat specialization of many species, can be explained in terms of their evolutionary history in changeable environments.},
address = {Basel},
author = {Winterbourn, M J},
booktitle = {Evolutionary ecology of freshwater animals},
doi = {10.1007/978-3-0348-8880-6_2},
editor = {Streit, B. and Stadler, T. and Lively, C. M.},
isbn = {978-3-0348-9812-6},
keywords = {community,community structure,development,discharge,disturbance,ecology,environment,environmental effects,equilibrium,fauna,flexibility,flow variability,freshwater,habitat,history,invertebrate community,life history,mountain stream,pattern,persistence,physical stressors,relationships,resilience,response,scales,spatial scale,specialization,species,species composition,stability,stream community,streams,temporal scale,variability},
pages = {31--54},
publisher = {Birkhauser Verlag},
title = {{New Zealand mountain stream communities: stable yet disturbed?}},
year = {1997}
}
@article{Lennon2004,
abstract = {Subsidies are donor-controlled inputs of nutrients and energy that can affect ecosystem-level processes in a recipient environment. Lake ecosystems receive large inputs of terrestrial carbon (C) in the form of dissolved organic matter (DOM). DOM inputs may energetically subsidize heterotrophic bacteria and determine whether lakes function as sources or sinks of atmospheric CO(2). I experimentally tested this hypothesis using a series of mesocosm experiments in New England lakes. In the first experiment, I observed that CO(2) flux increased by 160{\%} 4 days following a 1,000 microM C addition in the form of DOM. However, this response was relatively short lived, as there was no effect of DOM enrichment on CO(2) flux beyond 8 days. In a second experiment, I demonstrated that peak CO(2) flux from mesocosms in two lakes increased linearly over a broad DOM gradient (slope for both lakes=0.02+/-0.001 mM CO(2).m(-2) day(-1) per microM DOC, mean+/-SE). Concomitant changes in bacterial productivity and dissolved oxygen strengthen the inference that increasing CO(2) flux resulted from the metabolism of DOM. I conducted two additional studies to test whether DOM-correlated attributes were responsible for the observed change in plankton metabolism along the subsidy gradient. First, terrestrial DOM reduced light transmittance, but experimental shading revealed that this was not responsible for the observed patterns of CO(2) flux. Second, organically bound nitrogen (N) and phosphorus (P) accompanied DOM inputs, but experimental nutrient additions (without organic C) caused mesocosms to be saturated with CO(2). Together, these results suggest that C content of terrestrial DOM may be an important subsidy for freshwater bacteria that can influence whether recipient aquatic ecosystems are sources or sinks of atmospheric CO(2).},
author = {Lennon, Jay T.},
doi = {10.1007/s00442-003-1459-1},
isbn = {0029-8549},
issn = {00298549},
journal = {Oecologia},
keywords = {Bacteria,DOC,DOM,Plankton,Subsidy},
number = {4},
pages = {584--591},
pmid = {14689297},
title = {{Experimental evidence that terrestrial carbon subsidies increase CO 2 flux from lake ecosystems}},
volume = {138},
year = {2004}
}
@article{Winterbourn1981,
abstract = {New Zealand stream ecosystems differ from many of their North American counterparts, on which general stream ecosystem models are based, in several ways. In New Zealand, large particle detritivores (shredders) are poorly rep-resented, and the dominant invertebrates are browsers which feed on fine particulate organic matter and stone-surface organic layers. In contrast with the river continuum concept of Vannote et al. (Canadian Journal of Fisheries and Aquatic Sciences 37: 130-137, 1980), representation of functional feeding groups shows little change downstream and a temporal continuum of synchronous species replacements is not found. Many common benthic invertebrates are ecologically flexible species with poorly synchronised life histories. These differences appear to be associated with the non-retentive, climatically unpredictable nature of the stream environment. The idea that stream communities are highly structured entities is questioned, as is the generality of the river continuum concept.},
author = {Winterbourn, M. J. and Rounick, J. S. and Rounick, J. S.},
doi = {10.1080/00288330.1981.9515927},
isbn = {0028-8330},
issn = {11758805},
journal = {New Zealand Journal of Marine and Freshwater Research},
keywords = {Ecology,Functional feeding groups,Organic layers,River continuum concept,Stream ecosystems},
number = {3},
pages = {321--328},
title = {{Are New Zealand stream ecosystems really different?}},
url = {http://www.tandfonline.com/loi/tnzm20{\%}5Cnhttp://dx.doi.org/10.1080/00288330.1981.9515927{\%}5Cnhttp://www.tandfonline.com/},
volume = {15},
year = {1981}
}
@article{Nakano2001,
abstract = {Mutual trophic interactions between contiguous habitats have$\backslash$n remained poorly understood despite their potential significance for$\backslash$n community maintenance in ecological landscapes. In a deciduous forest$\backslash$n and stream ecotone, aquatic insect emergence peaked around spring, when$\backslash$n terrestrial invertebrate biomass was low. In contrast, terrestrial$\backslash$n invertebrate input to the stream occurred primarily during summer, when$\backslash$n aquatic invertebrate biomass was nearly at its lowest. Such reciprocal,$\backslash$n across-habitat prey flux alternately subsidized both forest birds and$\backslash$n stream fishes, accounting for 25.6{\%} and 44.0{\%} of the annual total$\backslash$n energy budget of the bird and fish assemblages, respectively. Seasonal$\backslash$n contrasts between allochthonous prey supply and in situ$\backslash$n prey biomass determine the importance of reciprocal subsidies. },
author = {Sobolewski, M. and Wlodarczyk, R. and Wojtas, G. and Twarkowski, P. and Jahnz-Rozyk, K. and Hussein, N.},
doi = {10.1073/pnas.98.1.166},
isbn = {0027-8424},
issn = {01377183},
journal = {Polski Przeglad Radiologii},
keywords = {Cancer,Fine-needle biopsy,Tumor of lung},
number = {4},
pages = {340--342},
pmid = {11136253},
title = {{Biopsja aspiracyjna cienkoiglowa guzow pluc pod kontrola rentgenoskopii}},
url = {wos:000166222600034{\%}5Cnhttp://www.pubmedcentral.nih.gov/articlerender.fcgi?artid=14562{\&}tool=pmcentrez{\&}rendertype=abstract},
volume = {64},
year = {1999}
}
@article{Wang2008,
abstract = {Generally, the curse of dimensionality leads to great bias of NNC. However, in this paper, multiple real-valued NNCs based on different feature subsets are combined by fuzzy integral so that the bias of NNC in high dimensionality is minimized, which is called FI-MRNNC. In FI-MRNNC, the feature set is partitioned into several low dimensionality feature subsets, where fuzzy measure is used to measure the importance of each feature subset and the interaction between feature subsets in its decision making process. According to the FI-MRNNC's classification accuracy, GA not only partitions the feature set into several feature subsets but oho defines a density value for the corresponding feature subset. Experimental results on some UCI databases illustrate that FI-MRNNC can reduce the bias of NNC, especially in high dimensionality. ICIC International {\textcopyright} 2008.},
archivePrefix = {arXiv},
arxivId = {arXiv:1501.06893v1},
author = {Wang, Li Juan and Wang, Xiao Long and Liu, Yuan Chao},
doi = {10.1111/j.2008.0030-1299.16644.x},
eprint = {arXiv:1501.06893v1},
isbn = {3338933673},
issn = {13494198},
journal = {International Journal of Innovative Computing, Information and Control},
keywords = {Feature subset partition,Fuzzy integral,Fuzzy mea-sure,GA,Multiple-classifier combination,Real-valued nearest neighbor classifier},
number = {2},
pages = {369--379},
title = {{Combination of multiple real-valued nearest neighbor classifiers based on different feature subsets with fuzzy integral}},
volume = {4},
year = {2008}
}
@article{Wise2013,
abstract = {Although plants are generally attacked by a community of several species of herbivores, relatively little is known about the strength of natural selection for resistance in multiple-herbivore communities-particularly how the strength of selection differs among herbivores that feed on different plant organs or how strongly genetic correlations in resistance affect the evolutionary responses of the plant. Here, we report on a field study measuring natural selection for resistance in a diverse community of herbivores of Solanum carolinense. Using linear phenotypic-selection analyses, we found that directional selection acted to increase resistance to seven species. Selection was strongest to increase resistance to fruit feeders, followed by flower feeders, then leaf feeders. Selection favored a decrease in resistance to a stem borer. Bootstrapping analyses showed that the plant population contained significant genetic variation for each of 14 measured resistance traits and significant covariances in one-third of the pairwise combinations of resistance traits. These genetic covariances reduced the plant's overall predicted evolutionary response for resistance against the herbivore community by about 60{\%}. Diffuse (co)evolution was widespread in this community, and the diffuse interactions had an overwhelmingly constraining (rather than facilitative) effect on the plant's evolution of resistance.},
author = {Wise, Michael J. and Rausher, Mark D.},
doi = {10.1111/evo.12061},
isbn = {0014-3820},
issn = {00143820},
journal = {Evolution},
keywords = {Diffuse coevolution,Directional selection,G-matrix,Horsenettle,Multiple herbivores,Resistance to herbivory,Selection analysis,Solanum carolinense},
number = {6},
pages = {1767--1779},
pmid = {23730768},
title = {{Evolution of resistance to a multiple-herbivore community: genetic correlations, diffuse coevolution, and constraints on the plant's response to selection}},
volume = {67},
year = {2013}
}
@article{Ollerton2009,
abstract = {Background and Aims ‘Pollination syndromes' are suites of phenotypic traits hypothesized to reflect convergent adaptations of flowers for pollination by specific types of animals. They were first developed in the 1870s and honed during the mid 20th Century. In spite of this long history and their central role in organizing research on plant–pollinator interactions, the pollination syndromes have rarely been subjected to test. The syndromes were tested here by asking whether they successfully capture patterns of covariance of floral traits and predict the most common pollinators of flowers.Methods Flowers in six communities from three continents were scored for expression of floral traits used in published descriptions of the pollination syndromes, and simultaneously the pollinators of as many species as possible were characterized.Key Results Ordination of flowers in a multivariate ‘phenotype space' defined by the syndromes showed that almost no plant species fall within the discrete syndrome clusters. Furthermore, in approximately two-thirds of plant species, the most common pollinator could not be successfully predicted by assuming that each plant species belongs to the syndrome closest to it in phenotype space.Conclusions The pollination syndrome hypothesis as usually articulated does not successfully describe the diversity of floral phenotypes or predict the pollinators of most plant species. Caution is suggested when using pollination syndromes for organizing floral diversity, or for inferring agents of floral adaptation. A fresh look at how traits of flowers and pollinators relate to visitation and pollen transfer is recommended, in order to determine whether axes can be identified that describe floral functional diversity more successfully than the traditional syndromes.},
author = {Ollerton, Jeff and Alarcon, Ruben and Waser, Nickolas M. and Price, Mary V. and Watts, Stella and Cranmer, Louise and Hingston, Andrew and Peter, Craig I. and Rotenberry, John},
doi = {10.1093/aob/mcp031},
file = {:Users/alyssacirtwill/Documents/Papers/Ollerton et al.{\_}2009{\_}Annals of Botany.pdf:pdf},
isbn = {0305-7364},
issn = {10958290},
journal = {Annals of Botany},
keywords = {Convergent evolution,floral traits,global,montane meadow,multidimensional scaling,mutualism,phenotype space,pollination syndromes,temperate grassland,test,tropical forest,tropical mountains},
number = {9},
pages = {1471--1480},
pmid = {19218577},
title = {{A global test of the pollination syndrome hypothesis}},
volume = {103},
year = {2009}
}
@article{Levin1970,
archivePrefix = {arXiv},
arxivId = {http://www.jstor.org/stable/2462300},
author = {Levin, Donald A. and Anderson, Wyatt W.},
doi = {10.2307/2678832},
eprint = {/www.jstor.org/stable/2462300},
isbn = {1630130044},
issn = {13494198},
journal = {The American Naturalist},
keywords = {Clustering,Distance measure,Impulse noise,Pixels,Switching filters,Vector filter},
number = {939},
pages = {455--467},
pmid = {17891731},
primaryClass = {http:},
title = {{Competition for pollinators between simultaneously flowering species}},
url = {http://www.jstor.org/stable/2678832?origin=crossref},
volume = {104},
year = {1970}
}
@article{Lind2015,
abstract = {Contemporary animal-plant interactions such as herbivory are widely understood to be shaped by evolutionary history. Yet questions remain about the role of plant phylogenetic diversity in generating and maintaining herbivore diversity, and whether evolutionary relatedness of producers might predict the composition of consumer communities. We tested for evidence of evolutionary associations among arthropods and the plants on which they were found, using phylogenetic analysis of naturally occurring arthropod assemblages sampled from a plant-diversity manipulation experiment. Considering phylogenetic relationships among more than 900 arthropod consumer taxa and 29 plant species in the experiment, we addressed several interrelated questions. First, our results support the hypothesis that arthropod functional traits such as body size and trophic role are phylogenetically conserved in community ecological samples. Second, herbivores tended to cooccur with closer phylogenetic relatives than would be expected at random, whereas predators and parasitoids did not show phylogenetic association patterns. Consumer specialization, as measured by association through time with monocultures of particular host plant species, showed significant phylogenetic signal, although the strength of this association varied among plant species. Polycultures of phylogenetically dissimilar plant species supported more phylogenetically dissimilar consumer communities than did phylogenetically similar polycultures. Finally, we separated the effects of plant species richness and relatedness in predicting the phylogenetic distribution of the arthropod assemblages in this experiment. The phylogenetic diversity of plant communities predicted the phylogenetic diversity of herbivore communities even after accounting for plant species richness. The phylogenetic diversity of secondary consumers differed by guild, with predator phylogenetic diversity responding to herbivore relatedness, while parasitoid phylogenetic diversity was driven by plant relatedness. Evolutionary associations between plants and their consumers are apparent in plots only meters apart in a single field, indicating a strong role for host-plant phylogenetic diversity in sustaining landscape consumer biodiversity.},
author = {Lind, Eric M. and Vincent, John B. and Weiblen, George D. and Cavender-Bares, Jeannine and Borer, Elizabeth T.},
doi = {10.1890/14-0784.1.sm},
isbn = {0012-9658},
issn = {0012-9658},
journal = {Ecology},
keywords = {Arthropod phylogeny,Cedar Creek LTER Minnesota USA,Community phylogenetics,Diversity,Grassland,Herbivores,Parasitoids,Predators},
number = {4},
pages = {998--1009},
pmid = {26230020},
title = {{Trophic phylogenetics: evolutionary influences on body size, feeding, and species associations in grassland arthropods}},
url = {https://www.scopus.com/inward/record.uri?partnerID=HzOxMe3b{\&}scp=84929222824{\&}origin=inward},
volume = {96},
year = {2015}
}
@article{Fenster2004,
abstract = {Floral evolution has often been associated with differences in pollina- tion syndromes. Recently, this conceptual structure has been criticized on the grounds that flowers attract a broader spectrum of visitors than one might expect based on their syndromes and that flowers often diverge without excluding one type of pollinator in favorof another. Despite these criticisms, we showthat pollination syndromes provide great utility in understanding the mechanisms of floral diversification. Our conclusions are based on the importance of organizing pollinators into functional groups according to presumed similarities in the selection pressures they exert. Furthermore, functional groups vary widely in their effectiveness as pollinators for particular plant species. Thus, although a plant may be visited by several functional groups, the relative se- lective pressures they exert will likely be very different.We discuss various methods of documenting selection on floral traits. Our review of the literature indicates over- whelming evidence that functional groups exert different selection pressures on floral traits.We also discuss the gaps in our knowledge of the mechanisms that underlie the evolution of pollination syndromes. In particular, we need more information about the relative importance of specific traits in pollination shifts, about what selective factors favor shifts between functional groups, about whether selection acts on traits inde- pendently or in combination, and about the role of history in pollination-syndrome evolution.},
author = {Fenster, Charles B. and Armbruster, W. Scott and Wilson, Paul and Dudash, Michele R. and Thomson, James D.},
doi = {10.1146/annurev.ecolsys.34.011802.132347},
isbn = {1543592X},
issn = {1543-592X},
journal = {Annual Review of Ecology, Evolution, and Systematics},
keywords = {floral evolution,mutualism,plant-animal interaction,pollinator},
number = {1},
pages = {375--403},
pmid = {15876374},
title = {{Pollination syndromes and floral specialization}},
url = {http://www.annualreviews.org/doi/10.1146/annurev.ecolsys.34.011802.132347},
volume = {35},
year = {2004}
}
@article{Pichersky2000,
abstract = {The evolution of new genes to make novel secondary compounds in plants is an ongoing process and might account for most of the differences in gene function among plant genomes. Although there are many substrates and products in plant secondary metabolism, there are only a few types of reactions. Repeated evolution is a special form of convergent evolution in which new enzymes with the same function evolve independently in separate plant lineages from a shared pool of related enzymes with similar but not identical functions. This appears to be common in secondary metabolism and might confound the assignment of gene function based on sequence information alone.},
author = {Pichersky, Eran and Gang, David R.},
doi = {10.1016/S1360-1385(00)01741-6},
isbn = {1360-1385},
issn = {13601385},
journal = {Trends in Plant Science},
number = {10},
pages = {439--445},
pmid = {11044721},
title = {{Genetics and biochemistry of secondary metabolites in plants: An evolutionary perspective}},
volume = {5},
year = {2000}
}
@article{Bell2005,
abstract = {Abstract. Sympatric plant species with similar flowering phenologies and floral mor- phologies may compete for pollination, and as a consequence potentially influence each other's reproductive success and mating system. Two likely competitors are Mimulus ringens and Lobelia siphilitica, which co-occur in wet meadows of central and eastern North Amer- ica, produce blue zygomorphic flowers, and share several species of bumble bee pollinators. To test for effects of competition for pollination, we planted experimental arrays of Mimulus ringens, each consisting of genets with unique combinations of homozygous marker ge- notypes. In two arrays we planted mixtures of Mimulus and Lobelia, and in two additional arrays we planted Mimulus without a competitor for pollination. Bumble bee pollinators frequently moved between Mimulus and Lobelia flowers in the mixed-species arrays, with 42{\%} of plant-to-plant movements being interspecific transitions. Pollinator movements between species were associated with a reduction in the amount of conspecific pollen arriving on Mimulus stigmas. The presence of Lobelia led to a significant 37{\%} reduction in the mean number of Mimulus seeds per fruit. In addition, Mimulus had a significantly lower rate of outcrossing in the mixed-species arrays (0.43) than in the ‘‘pure'' arrays (0.63). This is the first study to demonstrate that competition for pollination directly in- fluences outcrossing rates. Our work suggests that in self-compatible populations with genetic load, competition for pollination may not only reduce seed quantity, but may also lower seed quality.},
author = {Bell, John M. and Karron, Jeffrey D. and Mitchell, Randall J.},
doi = {10.1890/04-0694},
isbn = {0012-9658},
issn = {00129658},
journal = {Ecology},
keywords = {Bombus fervidus,Competition for pollination,Field experiment,Improper pollen transfer,Lobelia siphilitica,Mating system,Mimulus ringens,Outcrossing rate,Pollen loss,Seed set,Seeds per fruit,Visitation rate},
number = {3},
pages = {762--771},
pmid = {17746751},
title = {{Interspecific competition for pollination lowers seed production and outcrossing in Mimulus ringens}},
volume = {86},
year = {2005}
}
@article{Kraft2010,
abstract = {Despite a long history of the study of tropical forests, uncertainty about the importance of different ecological processes in shaping tropical tree species distributions persists. Trait- and phylogenetic-based...},
author = {Kraft, Nathan J.B. and Ackerly, David D.},
doi = {10.1890/09-1672.1},
isbn = {0012-9615},
issn = {00129615},
journal = {Ecological Monographs},
keywords = {Coexistence theory,Functional equivalence,Functional traits,Habitat filtering,Neutral theory,Null models,Phylogenetic community structure,Power analysis,Simulation modeling,Yasun{\'{i}} Forest Dynamics Plot, ecuador},
number = {3},
pages = {401--422},
pmid = {18948539},
title = {{Functional trait and phylogenetic tests of community assembly across spatial scales in an Amazonian forest}},
volume = {80},
year = {2010}
}
@article{Jordano2003c,
abstract = {Plant-animal mutualistic networks are interaction webs consisting of two sets of entities, plant and animal species, whose evolutionary dynamics are deeply influenced by the outcomes of the interactions, yielding a diverse array of coevolutionary processes. These networks are two-mode networks sharing many common properties with others such as food webs, social, and abiotic networks. Here we describe generalized patterns in the topology of 29 plant-pollinator and 24 plant-frugivore networks in natural communities. Scale-free properties have been described for a number of biological, social, and abiotic networks; in contrast, most of the plant-animal mutualistic networks (65.6{\%}) show species connectivity distributions (number of links per species) with a power-law regime but decaying as a marked cut-off, i.e. truncated power-law or broad-scale networks and few (22.2{\%}) show scale-invariance. We hypothesize that plant-animal mutualistic networks follow a build-up process similar to complex abiotic nets, based on the preferential attachment of species. However, constraints in the addition of links such as morphological mismatching or phenological uncoupling between mutualistic partners, restrict the number of interactions established, causing deviations from scale-invariance. This reveals generalized topological patterns characteristic of self-organized complex systems. Relative to scale-invariant networks, such constraints may confer higher robustness to the loss of keystone species that are the backbone of these webs.},
author = {Jordano, Pedro and Bascompte, Jordi and Olesen, Jens M.},
doi = {10.1046/j.1461-0248.2003.00403.x},
isbn = {1461-023X},
issn = {1461023X},
journal = {Ecology Letters},
keywords = {Biodiversity,Coevolution,Ecological networks,Food web structure,Pollination,Seed-dispersal},
number = {1},
pages = {69--81},
pmid = {966},
title = {{Invariant properties in coevolutionary networks of plant-animal interactions}},
volume = {6},
year = {2003}
}
@article{Rohr2014a,
abstract = {Recent studies have shown a phylogenetic signal in the structure of ecological networks, making the point that evolutionary history is important in explaining network architecture. However, this previous work has focused on either antagonistic (i.e., predator-prey) or mutualistic networks and has used different methodologies. Thus, a comparative assessment of both the frequency and the strength of phylogenetic signal across network types and components of network structure has been precluded. Here, we address this issue using a data set comprising 60 antagonistic and mutualistic networks. By quantifying simultaneously the matching and centrality components of network architecture-capturing the modular and nested structure, respectively-we test the presence and quantify the strength of phylogenetic signal across network types, species sets, and components of network structure. We find contrasting differences across such groups. First, phylogenetic signal is stronger in antagonistic webs than in mutualistic webs. Second, resources are more strongly constrained than consumers in food webs, while animals show more constraints than plants in mutualistic networks. Third, phylogenetic constraints are stronger for the matching component than for the centrality component of network structure. These results can shed light on the contrasting evolutionary constraints shaping network structure across interaction types and species sets.},
author = {Rohr, Rudolf P. and Bascompte, Jordi},
doi = {10.1086/678234},
isbn = {00030147},
issn = {0003-0147},
journal = {The American Naturalist},
keywords = {food webs,plant-frugivore net-,plant-pollinator networks},
number = {5},
pages = {556--564},
pmid = {25325741},
title = {{Components of Phylogenetic Signal in Antagonistic and Mutualistic Networks}},
url = {http://www.journals.uchicago.edu/doi/10.1086/678234},
volume = {184},
year = {2014}
}
@article{Rezende2007b,
abstract = {Apoptotic cells release 'find-me' signals at the earliest stages of death to recruit phagocytes. The nucleotides ATP and UTP represent one class of find-me signals, but their mechanism of release is not known. Here, we identify the plasma membrane channel pannexin 1 (PANX1) as a mediator of find-me signal/nucleotide release from apoptotic cells. Pharmacological inhibition and siRNA-mediated knockdown of PANX1 led to decreased nucleotide release and monocyte recruitment by apoptotic cells. Conversely, PANX1 overexpression enhanced nucleotide release from apoptotic cells and phagocyte recruitment. Patch-clamp recordings showed that PANX1 was basally inactive, and that induction of PANX1 currents occurred only during apoptosis. Mechanistically, PANX1 itself was a target of effector caspases (caspases 3 and 7), and a specific caspase-cleavage site within PANX1 was essential for PANX1 function during apoptosis. Expression of truncated PANX1 (at the putative caspase cleavage site) resulted in a constitutively open channel. PANX1 was also important for the 'selective' plasma membrane permeability of early apoptotic cells to specific dyes. Collectively, these data identify PANX1 as a plasma membrane channel mediating the regulated release of find-me signals and selective plasma membrane permeability during apoptosis, and a new mechanism of PANX1 activation by caspases.},
archivePrefix = {arXiv},
arxivId = {Figures, S., 2010. Supplementary information. Nature, 1(c), pp.1–7. Available at: http://www.pubmedcentral.nih.gov/articlerender.fcgi?artid=3006164{\&}tool=pmcentrez{\&}rendertype=abstract.},
author = {Tsuchida, Hiroyuki and Tsukagoshi, Hideo and Inoue, Asako},
doi = {10.1038/nature0},
eprint = {/www.pubmedcentral.nih.gov/articlerender.fcgi?artid=3006164{\&}tool=pmcentrez{\&}rendertype=abstract.},
isbn = {4936551015},
issn = {03853667},
journal = {Japanese Journal of Chest Diseases},
number = {8},
pages = {727},
pmid = {20944749},
primaryClass = {Figures, S., 2010. Supplementary information. Nature, 1(c), pp.1–7. Available at: http:},
title = {{A Case with Bilateral Pleuritis as a Presenting Symptom of Recurrent Sarcoidosis after 16 Years Remission Period}},
url = {http://www.pubmedcentral.nih.gov/articlerender.fcgi?artid=3006164{\&}tool=pmcentrez{\&}rendertype=abstract},
volume = {61},
year = {2002}
}
@article{Rezende2007,
abstract = {Recent studies have started to unravel the structure of mutualistic networks, although few functional explanations underlying such structure have been explored. We used computer simulations to test whether complementarity between phenotypic traits of plants and animals can explain the pervasive tendency of specialists to interact with proper subsets of species that generalists interact with (nested interactions), and how phylogeny affects such interaction patterns. Simultaneously, we assessed whether complementarity and phylogenetic structure were associated with patterns of interaction in a real mutualistic community. Simulation results support that highly nested networks can emerge from phenotypic complementarity, particularly when several traits are involved. The hierarchical structure of phylogenetic relations can also contribute to increased nestedness because traits determining complementarity are then inherited in a correlated fashion. Phylogenetic effects on resulting generalization levels are often low, but can be detected. Results from the empirical network support a relevant role of complementarity and phylogenetic history on interaction patterns. Our results demonstrate that these factors can contribute to nestedness, which emphasize the necessity of considering evolutionary mechanisms in studies of community structure.},
author = {Rezende, Enrico L. and Jordano, Pedro and Bascompte, Jordi},
doi = {10.1111/j.0030-1299.2007.16029.x},
isbn = {0030-1299},
issn = {16000706},
journal = {Oikos},
number = {11},
pages = {1919--1929},
pmid = {251278700014},
title = {{Effects of phenotypic complementarity and phylogeny on the nested structure of mutualistic networks}},
volume = {116},
year = {2007}
}
@article{Rezende2007a,
abstract = {The interactions between plants and their animal pollinators and seed dispersers have moulded much of Earth's biodiversity(1-3). Recently, it has been shown that these mutually beneficial interactions form complex networks with a well-defined architecture that may contribute to biodiversity persistence(4-8). Little is known, however, about which ecological and evolutionary processes generate these network patterns(3,9). Here we use phylogenetic methods(10,11) to show that the phylogenetic relationships of species predict the number of interactions they exhibit in more than one-third of the networks, and the identity of the species with which they interact in about half of the networks. As a consequence of the phylogenetic effects on interaction patterns, simulated extinction events tend to trigger coextinction cascades of related species. This results in a non-random pruning of the evolutionary tree(12,13) and a more pronounced loss of taxonomic diversity than expected in the absence of a phylogenetic signal. Our results emphasize how the simultaneous consideration of phylogenetic information and network architecture can contribute to our understanding of the structure and fate of species-rich communities.},
archivePrefix = {arXiv},
arxivId = {Figures, S., 2010. Supplementary information. Nature, 1(c), pp.1–7. Available at: http://www.pubmedcentral.nih.gov/articlerender.fcgi?artid=3006164{\&}tool=pmcentrez{\&}rendertype=abstract.},
author = {Rezende, Enrico L. and Lavabre, Jessica E. and Guimar{\~{a}}es, Paulo R. and Jordano, Pedro and Bascompte, Jordi},
doi = {10.1038/nature05956},
eprint = {/www.pubmedcentral.nih.gov/articlerender.fcgi?artid=3006164{\&}tool=pmcentrez{\&}rendertype=abstract.},
isbn = {0028-0836},
issn = {14764687},
journal = {Nature},
number = {7156},
pages = {925--928},
pmid = {17713534},
primaryClass = {Figures, S., 2010. Supplementary information. Nature, 1(c), pp.1–7. Available at: http:},
title = {{Non-random coextinctions in phylogenetically structured mutualistic networks}},
volume = {448},
year = {2007}
}
@article{Rohr2014,
abstract = {In theoretical ecology, traditional studies based on dynamical stability and numerical simulations have not found a unified answer to the effect of network architecture on community persistence. Here, we introduce a mathematical framework based on the concept of structural stability to explain such a disparity of results.We investigated the range of conditions necessary for the stable coexistence of all species in mutualistic systems.We show that the apparently contradictory conclusions reached by previous studies arise as a consequence of overseeing either the necessary conditions for persistence or its dependence onmodel parameterization.We show that observed network architectures maximize the range of conditions for species coexistence.We discuss the applicability of structural stability to study other types of interspecific interactions.},
author = {Rohr, Rudolf P. and Saavedra, Serguei and Bascompte, Jordi},
doi = {10.1126/science.1253497},
isbn = {0036-8075, 1095-9203},
issn = {10959203},
journal = {Science},
number = {6195},
pages = {1253497},
pmid = {25061214},
title = {{On the structural stability of mutualistic systems}},
url = {http://www.sciencemag.org/cgi/doi/10.1126/science.1253497},
volume = {345},
year = {2014}
}
@misc{GlobalWeb,
author = {Caffrey, Laura and Thompson, Ross},
booktitle = {University of Canberra},
title = {{GlobalWeb: An online collection of food webs}},
url = {globalwebdb.com},
year = {2015}
}
@article{Hurlbert1972,
abstract = {The organophosphorus insecticide Selecron [O-(4-bromo-2-chlorophenyl) O-ethyl S-n-propyl-phosphorotioate] at 10 and 50 ppm significantly decreased respiration, mycelial protein, extracellular protein and mycelial dry weight of Aspergillus fumigatus, A. terreus and Myceliophthora thermophila when grown at 45°C. Cxand C1cellulases of tested fungi were significantly decreased. However, C1cellulase of A. fumigatus was slightly increased. {\textcopyright} 1993.},
author = {Omar, S. A. and Moharram, A. M. and Abd-Alla, M. H.},
doi = {10.1016/0964-8305(93)90025-W},
issn = {09648305},
journal = {International Biodeterioration and Biodegradation},
number = {4},
pages = {305--310},
title = {{Effects of an organophosphorus insecticide on the growth and cellulolytic activity of fungi}},
volume = {31},
year = {1993}
}
@book{Elton1927,
abstract = {Global positioning system (GPS) telemetry technology allows us to monitor and to map the details of animal movement, securing vast quantities of such data even for highly cryptic organisms. We envision an exciting synergy between animal ecology and GPS-based radiotelemetry, as for other examples of new technologies stimulating rapid conceptual advances, where research opportunities have been paralleled by technical and analytical challenges. Animal positions provide the elemental unit of movement paths and show where individuals interact with the ecosystems around them. We discuss how knowing where animals go can help scientists in their search for a mechanistic understanding of key concepts of animal ecology, including resource use, home range and dispersal, and population dynamics. It is probable that in the not-so-distant future, intense sampling of movements coupled with detailed information on habitat features at a variety of scales will allow us to represent an animal's cognitive map of its environment, and the intimate relationship between behaviour and fitness. An extended use of these data over long periods of time and over large spatial scales can provide robust inferences for complex, multi-factorial phenomena, such as meta-analyses of the effects of climate change on animal behaviour and distribution.},
address = {New York},
archivePrefix = {arXiv},
arxivId = {arXiv:1011.1669v3},
author = {Cagnacci, Francesca and Boitani, Luigi and Powell, Roger A. and Boyce, Mark S.},
booktitle = {Philosophical Transactions of the Royal Society B: Biological Sciences},
doi = {10.1098/rstb.2010.0107},
eprint = {arXiv:1011.1669v3},
isbn = {0226206394},
issn = {14712970},
keywords = {Animal movement,Autocorrelation,Biotelemetry,Fitness,Global positioning system technology,Mechanistic models},
number = {1550},
pages = {2157--2162},
pmid = {20566493},
publisher = {Macmillan Co.},
title = {{Animal ecology meets GPS-based radiotelemetry: A perfect storm of opportunities and challenges}},
url = {http://www.cabdirect.org/abstracts/19632204195.html},
volume = {365},
year = {2010}
}
@incollection{Tsuda1972,
address = {Warsaw},
author = {Tsuda, M},
booktitle = {Productivity Problems of Freshwaters Pages},
editor = {Kajak, Z. and Hillbricht-Ilkowska, A.},
keywords = {2PROD,INCOMPLETE,INVERT,LOTIC},
pages = {827--841},
publisher = {Polish Scientific},
title = {{Interim results of the Yoshino River productivity survey, especially on benthic animals}},
year = {1972}
}
@incollection{Sorokin1972,
address = {Warsaw},
author = {Sorokin, Y I},
booktitle = {Productivity problems of freshwaters},
editor = {Kajak, Z. and Hillbricht-Ilkowska, A.},
pages = {493--503},
publisher = {Polish Scientific},
title = {{Biological productivity of the Rybinsk reservoir}},
year = {1972}
}
@article{Clarke1967,
author = {Clarke, Thomas A. and Flechsig, Arthur O. and Grigg, Richard W.},
doi = {10.1126/science.157.3795.1381},
issn = {00368075},
journal = {Science},
number = {3795},
pages = {1381--1389},
pmid = {4382569},
title = {{Ecological studies during Project Sealab II}},
volume = {157},
year = {1967}
}
@book{Minckley1963,
abstract = {This book critically interrogates the work of David Harvey, one of the worlds most influential geographers, and one of its best known Marxists. Considers the entire range of Harveys oeuvre, from the nature of urbanism to environmental issues. Written by contributors from across the human sciences, operating with a range of critical theories. Focuses on key themes in Harveys work. Contains a consolidated bibliography of Harveys writings.},
address = {Washington},
archivePrefix = {arXiv},
arxivId = {arXiv:1011.1669v3},
author = {Castree, Noel and Gregory, Derek},
booktitle = {David Harvey: A Critical Reader},
doi = {10.1002/9780470773581},
eprint = {arXiv:1011.1669v3},
isbn = {9780631235095},
issn = {0004-5608},
number = {11},
pages = {1--324},
pmid = {37},
publisher = {Wildlife Monographs, The Wildlife Society},
title = {{David Harvey: A Critical Reader}},
url = {http://www.jstor.org/stable/3830530},
year = {2008}
}
@article{Rosenthal1974,
author = {Rosenthal, R J and Clarke, W D and Dayton, P K},
journal = {California. Fish. Bull. U.S.},
number = {3},
pages = {670--684},
title = {{Ecology and natural history of a stand of giant kelp, Macrocystis pyrifera}},
volume = {72},
year = {1974}
}
@incollection{Varley1970,
abstract = {The alteration of landscapes by human activities worldwide has forced many animal species to persist in remnants of natural habitats. One of the best examples of this process occurs in central-east Argentina, where the original prairies were almost completely replaced by agricultural land, cattle production areas and increasing urbanization. We used nine microsatellite loci to analyze the population genetic structure of the Sigmodontine rodent Calomys musculinus in two anthropically altered habitats: an agroecosystem and a city. Rodents inhabiting urban vacant lots showed higher levels of relatedness and genetic differentiation than rodents inhabiting the agroecosystem. Urban rodents presented a pattern of isolation by distance; in the rural habitat this pattern was present only along the border of a secondary road, but not over the entire area surveyed. In the city, a spatially limited but buffered environment, populations would be small, demographically stable, and dispersal would be restricted. On the contrary, in agroecosystems populations would experience a high rate of turn over: local demes would originate each year by a mixture of overwintering individuals, and dispersal would occur preferentially along the weedy borders of fields. {\textcopyright} 2010 Deutsche Gesellschaft f{\"{u}}r S{\"{a}}ugetierkunde.},
address = {Oxford},
author = {Chiappero, Marina B. and Panzetta-Dutari, Graciela M. and G{\'{o}}mez, Daniela and Castillo, Ernesto and Polop, Jaime J. and Gardenal, Cristina N.},
booktitle = {Mammalian Biology},
doi = {10.1016/j.mambio.2010.02.003},
editor = {Watson, A.},
isbn = {1616-5047},
issn = {16165047},
keywords = {Agroecosystems,Calomys musculinus,Genetic structure,Microsatellites,Urban populations},
number = {1},
pages = {41--50},
pmid = {3200},
publisher = {Blackwell Scientific},
title = {{Contrasting genetic structure of urban and rural populations of the wild rodent Calomys musculinus (Cricetidae, Sigmodontinae)}},
volume = {76},
year = {2011}
}
@article{Sarvala1974,
author = {Sarvala, J},
journal = {Luonnon Tutkija},
number = {4},
pages = {181--190},
title = {{Paarjarven energiatalous}},
volume = {78},
year = {1974}
}
@incollection{Schiemer1979,
address = {The Hague},
author = {Schiemer, F},
booktitle = {Monograohiae Biologicae},
editor = {Loffler, H.},
keywords = {Chironomidae},
pages = {337--384},
publisher = {Dr. W. Junk Publishers},
title = {{The benthic community of the open lake}},
volume = {1979; 37},
year = {1979}
}
@article{Society2012,
author = {Hurlbert, Stuart H. and Mulla, Mir S. and Willson, Harold R.},
journal = {Ecological Monographs},
number = {3},
pages = {269--299},
title = {{Effects of an Organophosphorus Insecticide on the Phytoplankton , Zooplankton , and Insect Populations of Fresh-Water Ponds Author ( s ): Stuart H . Hurlbert , Mir S . Mulla and Harold R . Willson Reviewed work ( s ): Published by : Ecological Society of}},
volume = {42},
year = {1972}
}
@article{Nixon1973,
abstract = {Measurements of the abundance of major populations, their metabolism, and the seasonal patterns of total system metabolism throughout a year were used to develop energy-flow diagrams for a New England salt-marsh embayment. The annual ecological energy budget for the embayment indicates that consumption exceeds production, so that the system must depend on inputs of organic detritus from marsh grasses. Gross production ranged from almost zero in winter to about 5 g O"2 m{\^{}}-{\^{}}2 day{\^{}}-{\^{}}1 in summer. Respiration values were similar, but slightly higher, with the maximum difference observed in fall. Populations of shrimp and fish were largest in fall, with a much smaller peak in spring. Few animals were present in the embayment from May to July, but fall populations of shrimp ranged from 250 to 800 m{\^{}}-{\^{}}2 and fish averaged over 10 m{\^{}}-{\^{}}2. Birds were most abundant in winter and spring. In spite of high numbers, no evidence was found that the marsh embayment exported large amounts of shrimp or fish to the estuary. Production of aboveground emergent grasses on the marsh equaled 840 g m{\^{}}-{\^{}}2 for tall Spartina alterniflora, 432 g m{\^{}}-{\^{}}2 for short S. alterniflora, and 430 g m{\^{}}-{\^{}}2 for S. patens. These values are similar to those for New York marshes, but substantially lower than the southern marsh types. The efficiency of production of marsh grasses in the New England marsh was lower than reported for southern areas. A simulation model based on the laboratory and field metabolism and biomass measurements of parts of the embayment system was developed to predict diurnal patterns of dissolved oxygen in the marsh. The model was verified with field measurements of diurnal oxygen curves. The model indicated the importance of the timing of high tides in determining oxygen levels and was used to explore simulated additions of sewage BOD and increases in temperature.},
author = {Nixon, Scott W. and Oviatt, Candace A.},
doi = {10.2307/1942303},
isbn = {00129615},
issn = {00129615},
journal = {Ecological Monographs},
number = {4},
pages = {463--498},
pmid = {487},
title = {{Ecology of a New England Salt Marsh}},
url = {http://doi.wiley.com/10.2307/1942303},
volume = {43},
year = {1973}
}
@article{Ruzicka2012,
abstract = {The Northern California Current (NCC) is a seasonally productive and open ecosystem. It is home to both a diverse endemic community and to seasonally transient species. Productivity and food web structure vary seasonally, interannually, and decadally due to variability in coastal upwelling, climate-scale physical processes, and the migratory species entering the system. The composition of the pelagic community varies between years, including changes to mid-trophic level groups that represent alternate energy-transfer pathways between lower and upper trophic levels (forage fishes, euphausiids, jellyfish). Multiple data sets, including annual spring and summer mesoscale surveys of the zooplankton, pelagic fish, and seabird communities, were used to infer NCC trophic network arrangements and develop end-to-end models for each of the 2003-2007 upwelling seasons. Each model was used to quantify the interannual variability in energy-transfer efficiency from bottom to top trophic levels. When each model was driven under an identical nutrient input rate, substantial differences in the energy available to each functional group were evident. Scenario analyses were used to examine the roles of forage fishes, euphausiids, and jellyfish (small gelatinous zooplankton and large carnivorous jellyfish) as alternate energy transfer pathways. Euphausiids were the more important energy transfer pathway; a large proportion of the lower trophic production consumed was transferred to higher trophic levels. In contrast, jellyfish acted as a production loss pathway; little of the production consumed was passed upwards. Analysis of the range of ecosystem states observed interannually and understanding system sensitivity to variability among key trophic groups improves our ability to predict NCC ecosystem response to short- and long-term environmental change. {\textcopyright} 2012 Elsevier Ltd.},
author = {Ruzicka, James J. and Brodeur, Richard D. and Emmett, Robert L. and Steele, John H. and Zamon, Jeannette E. and Morgan, Cheryl A. and Thomas, Andrew C. and Wainwright, Thomas C.},
doi = {10.1016/j.pocean.2012.02.002},
isbn = {00796611},
issn = {00796611},
journal = {Progress in Oceanography},
pages = {19--41},
publisher = {Elsevier Ltd},
title = {{Interannual variability in the Northern California Current food web structure: Changes in energy flow pathways and the role of forage fish, euphausiids, and jellyfish}},
url = {http://dx.doi.org/10.1016/j.pocean.2012.02.002},
volume = {102},
year = {2012}
}
@article{Hiatt1960,
abstract = {See full-text article at JSTOR},
archivePrefix = {arXiv},
arxivId = {1503.02727},
author = {Hiatt, Robert W. and Strasburg, Donald W.},
doi = {10.2307/1942181},
eprint = {1503.02727},
isbn = {00129615},
issn = {00129615},
journal = {Ecological Monographs},
number = {1},
pages = {65--127},
pmid = {946},
title = {{Ecological Relationships of the Fish Fauna on Coral Reefs of the Marshall Islands}},
url = {http://doi.wiley.com/10.2307/1942181},
volume = {30},
year = {1960}
}
@article{Schoenly1983,
abstract = {It is reported that cationic liposomes are capable of transfecting embryos in unincubated fertile chicken eggs and that the cationic liposome, TransfectAce, has superior properties to Lipofectin. In order to determine the duration of expression of genes introduced in this way, embryos were transfected with an expression vector encoding the firefly luciferase cDNA under the control of the Rous sarcoma virus long terminal repeat (LTR). Luciferase activity could be observed consistently in day 3 embryos and activity was detectable up to day 8 of incubation. The relative expression of luciferase under the control of different viral promoters was compared in transfected chicken embryo fibroblasts and day 3 embryos. The cytomegalovirus immediate early promoter and the SV40 early promoter directed the highest amount of expression in fibroblasts while the Rous sarcoma virus LTR caused the highest amount of expression in embryos. Chicken embryo fibroblasts were transfected with the luciferase vector in order to examine duration of reporter gene expression in vitro. Luciferase expression was decreased exponentially over a 24-day period after which point luciferase activity could no longer be detected. These data suggest that stable integration of transfected DNA using liposomes is a rare event. Nevertheless, liposome-mediated transfection of embryos is suitable for the examination of promoter activity in vivo and may be a useful method to transfect genes to study embryonic development.},
author = {Schoenly, Kenneth G.},
issn = {0013-8746},
journal = {Annals Entomological Society of America},
keywords = {Conomyrma bicolor,Dolichoderinae,Formicidae,Formicinae,Iridomyrmex pruinosus,Myrmecocystus romainei,Myrmicinae,North America,Pogonomyrmex maricopa,Texas,USA,ant,bait traps,community structure,dung reducers,scientific,succession,trophic relationships},
pages = {790--796},
title = {{Arthropods associated with bovine and equine dung in ungrazed Chihuahuan desert ecosystem}},
volume = {76},
year = {1983}
}
@article{Ratsirarson1996,
abstract = {The pitchers of Nepenthes madagascariensis are visited by a wide variety of insects attracted to the bright co the pitcher, the nectar secreted around its opening and the odor of the fluid. The insect visitors appear to become disoriented over time and with increasing likelihood fall into the pitcher and drown. The pitchers form a temporary habitat functional for about three months. Several specialized arthropods including mosquito larvae (Uranotaenia bosseri, U. belkini), mites (Creutzeria sp.), and frit fly larvae (Chloropidae) complete their life cycle in the pitcher and depend directly or indirectly on the drowned insects falling into the pitcher. Colonization by these arthropods varies spatially and temporally between the two forms of pitchers (juvenile, rosette form and adult form). Mites (Creutzeria sp.) show a phoretic relationship in colonizing new pitchers by clinging to adult frit flies that emerge from the pitcher's fluid. Food web interactions in the Nepenthes madagascariensis pitcher are more complex than those reported by Beaver (1985). RESUME Les urnes de Nepenthes madagascariensis sont visitees par de differents types d'insectes attires par la couleur de l'ur par le nectar secr{\&}e autour de son ouverture et par l'odeur du liquide. Les insectes visiteurs semblent etre de plus en plus desorientes, et leur chance de tomber dans l'urne augmente en fonction du temps. Les urnes forment un habitat temporaire et fonctionnel pour trois mois. Plusieurs arthropodes sp6cialises, y compri des larves de moustiques (Uranotaenia bosseri, U. belkini), des mites (Creutzeria sp.), des larves de moucherons (Chloropidae) vivent a l'interieur de l'urne et dependent directement ou indirectement des insectes tombant dans l'urne. La colonisation de ces arthropodes varie dans le temps et l'espace entre les deux types d'urnes (forme juvenile et forme adulte). Les mites (Creutzeria sp.) montrent une association phoretique pour coloniser l'urne en s'accrochant sur les mouches adultes (Chloropidae) qui emergent du liquide. Les interactions entre les organismes associes au Nepenthes madagascariensis semblent plus complexes que celles publiees par Beaver (1985).},
author = {Ratsirarson, Joelisoa and Silander, John A.},
doi = {10.2307/2389076},
issn = {00063606},
journal = {Biotropica},
keywords = {community structure,food-webs,pitcher plant,tropical},
number = {2},
pages = {218},
title = {{Structure and Dynamics in Nepenthes madagascariensis Pitcher Plant Micro-Communities}},
url = {https://www.jstor.org/stable/2389076?origin=crossref},
volume = {28},
year = {1996}
}
@article{Schoenly1983a,
abstract = {Species numbers, guild, and trophic structure of carrion arthropod communities were examined for 18 mammalian carcasses of varying mass in the northern Chihuahuan desert from May to September 1980. Community structure was described for five carcass weight classes using seven guilds and three trophic levels. Twenty-three carrion-associated species were collected from the carcasses. Large carcasses attracted more species and guilds than small ones. Biomass of necrophagous taxa was higher than predator biomass in all carcass sizes. Linear regression analyses revealed a significant positive relationship between arthropod species richness and carcass mass. },
author = {Schoenly, K and Reid, W},
issn = {0140-1963},
journal = {Journal of Arid Environments},
keywords = {Arthropoda,Chihuahuan Desert,Texas,USA,carrion,community structure,desert,trophic relationships},
number = {3},
pages = {253--263},
title = {{Community structure of carrion arthropods in the Chihuahuan Desert.}},
volume = {6},
year = {1983}
}
@article{Schneider1997,
abstract = {The food webs of 7 temporary pond communities in Vilas County, Wisconsin, USA, were studied in 1984-90. Food web statistics showed a complex relationship with measures of habitat variability in the ponds. Connectance was highest in short-duration, highly variable habitats, and lowest in habitats of intermediate duration and variability. The number of links and links/taxon increased with increasing duration. Much of the variation in the food web statistics could be explained by a strong linear relationship between number of taxa and number of links/taxon and a quadratic relationship of taxa number with the number of links. However, after accounting for this variation, there remained a relationship of duration with links and links/taxon. The relationship between the food web statistics and duration corresponded to experimental evaluations of predation in these habitats that showed an increasing importance of predation in long-duration habitats. The food web statistics, however, missed threshold effects in the relationship between predation and habitat duration. Differences in food web statistics before and after a regional drought could be explained by a decrease in taxa number after the drought. Connectance was the most robust statistic in relation to taxa number, but was also the least sensitive to changes in habitat characteristics. Among the many taxa examined were Diptera (including Aedes, Culex, Anopheles, Chaoborus and Chironomidae), Coleoptera, Hemiptera, Odonata, Crustacea, Gastropoda and frogs.},
author = {Schneider, Daniel W.},
doi = {10.1007/s004420050197},
isbn = {0029-8549},
issn = {00298549},
journal = {Oecologia},
keywords = {Connectance,Food webs,Habitat variability,Predation,Temporary ponds},
number = {4},
pages = {567--575},
pmid = {1931},
title = {{Predation and food web structure along a habitat duration gradient}},
volume = {110},
year = {1997}
}
@article{Seifert1976,
abstract = {First to use the dynamic regression analysis. They used a community matrix approach.},
author = {Seifert, Richard P. and Seifert, Florence Hammett},
doi = {10.1086/283080},
isbn = {1630130044},
issn = {0003-0147},
journal = {The American Naturalist},
keywords = {coexistence,competition,dispersal,erogeneity,kernels,on biological,regional-scale ecological dynamics depend,spatial het-,spatial scale},
number = {973},
pages = {461--483},
title = {{A Community Matrix Analysis of Heliconia Insect Communities}},
url = {https://www.journals.uchicago.edu/doi/10.1086/283080},
volume = {110},
year = {1976}
}
@article{Ricker1935,
author = {Ricker, W E},
journal = {Publications of the Ontario Fisheries Research Laboratory},
number = {10},
pages = {1--114},
title = {{An ecological classification of certain Ontario streams}},
volume = {49},
year = {1934}
}
@article{Dexter1947,
author = {Dexter, Ralph W},
journal = {Ecological Monographs},
number = {3},
pages = {261--294},
title = {{The marine communities of a tidal inlet at Cape Ann, Massachusetts: A study in bio-ecology. Ecological Monographs 17(3):261-294. [Nudibranchia p. 274]. [Aeolis sp., Onchidoris bilamellata p. 291}},
volume = {17},
year = {1947}
}
@article{Johnston1956,
abstract = {"The Short-cared Owl is a common winter visitant to the salt marshes around San Francisco Bay. Between four and ten owls live in the winter on the study plot of some 200 acres on the San Pablo salt marsh. The owls forage mainly at night there. Of 638 items found in pellets, 75 per cent were mammals, 20 per cent birds, and 5 per cent insects. Mammals were responsible for about 90 per cent of the mass consumed, Microtus and Rattus being the most important kinds. The relationship of Short-eared Owl predation to the community food web is indicated by means of a diagram."},
author = {Johnston, R. F.},
doi = {10.2307/4158481},
isbn = {ISSN: 0043-5643},
issn = {00435643},
journal = {Wilson Bulletin},
number = {2},
pages = {92--102},
pmid = {3051293},
title = {{Predation by short-eared owls on a Salicornia salt marsh}},
volume = {68},
year = {1956}
}
@article{Chapman1955,
abstract = {During a university entomology course held at the Juniper Hall Field Centre, Dorking, in 1951, daily investigations were made for two weeks by a group of students- on the larger invertebrate fauna of three rabbit carcasses. In the evening of 4 July three freshly killed rabbits were put out in places separated by approximately 30 and 40 m. Carcass (A) was placed just inside a small shrubbery with no undergrowth and amongst partly rotted leaves. This habitat was generally dry and in deep shade. (B) was placed on cinders in loose sward under a large plane tree j this habitat was more subject to early morning dew than (A) and was exposed to light. (C) was situated in thick grass near the open verge of a meadow. Each carcass was covered with wire-netting pegged down at the comers to prevent disturbance by foxes and dogs. During examination this netting was removed and the carcass in each case was disturbed. Daily observations were made between 0800 and 0830 h G.M.T., 5-18 July inclusive. In many cases it was necessary to collect specimens for identification. In only a few instances did specimens escape unrecorded, and these represent only a fraction of the fauna present on examination. Not all specimens seen or caught have been fully identified, but specific determinations have been undertaken where possible. Grateful acknowledgement is made to the members of the staff of the Brit. Mus. (Nat. Hist.) who came to our aid with the more difficult species},
author = {Chapman, R F and Sankey, J H P and Journal, The and Nov, No and Chapman, B Y R F},
isbn = {00218790},
journal = {Journal of Animal Ecology},
number = {2},
pages = {395--402},
title = {{The Larger Invertebrate Fauna of Three Rabbit Carcasses T H E LARGER INVERTEBRATE FAUNA O F THREE RABBIT CARCASSES}},
url = {http://www.jstor.org/stable/1720},
volume = {24},
year = {2008}
}
@article{Paine1980,
abstract = {A review paper of Paine's previous work with emphasis on strong and weak species' interactions. The paper states that the strength or importance of a trophic relationship cannot be assumed equivalent for all web species. Some species will be a strong interactor and will have pronounced effects on the community. An example is Pisaster ochraceus' predation on Mytilus californianus. Other species will have little or no effect on the community upon removal. Therefore just enumerating links of a food web provides little information on species' importance or role in the community. What counts instead is the interaction strength of the predator or the competitive stature of the preferred prey. Paine suggests that the next generation of food web ecologists be more sensitive to interaction strength and less so to trophic complexity.},
archivePrefix = {arXiv},
arxivId = {arXiv:1011.1669v3},
author = {Paine, R. T.},
doi = {10.2307/4220},
eprint = {arXiv:1011.1669v3},
isbn = {616569},
issn = {00218790},
journal = {The Journal of Animal Ecology},
number = {3},
pages = {666},
pmid = {231},
title = {{Food Webs: Linkage, Interaction Strength and Community Infrastructure}},
url = {http://www.jstor.org/stable/4220?origin=crossref},
volume = {49},
year = {1980}
}
@article{Preston2012,
abstract = {This data set presents a comprehensive food web for Quick Pond, a northern California pond ecosystem. The web includes organisms from all regions of the pond (i.e., littoral, limnetic, profundal, and benthic zones) as well as terrestrial organisms that interact with the aquatic community or have aquatic life-stages. The food web has three attributes that are often omitted from freshwater food webs: inclusion of (1) parasites and other infectious agents, (2) ontogenetic stages of most animals with complex life cycles, and (3) biomass information for many animals. Data on species presence was obtained over three years using field sampling techniques (i.e., seine- and D-nets, stove-pipe samplers, and visual encounter surveys) and laboratory examinations of free-living organisms for infectious agents (primarily metazoan parasites, but also some microbes). We collected body size and biomass data for abundant aquatic animals.1mm and for trematode parasites, which were the most abundant parasitic group. Information on trophic interactions was obtained from direct observations and published literature sources. Within the food-web data we include supporting information for each node on taxonomy, lifestyle, and residency; and for each link we include information on type of interaction and the source of evidence (e.g., direct observation, literature, or inferred). The food web contains 113 nodes, 1905 links, and 63 species. To facilitate comparisons between food webs from different ecosystems we present the data in a system- neutral format.},
author = {Preston, D L and Orlofske, S A and McLaughlin, J P and Johnson, P T J},
doi = {10.1890/11-2194.1},
isbn = {0012-9658},
issn = {0012-9658},
journal = {Ecology},
keywords = {Quick Pond,biomass,complex life cycles,food webs,freshwater,infectious agents,parasites,pond,trophic interactions,wetland.},
number = {March},
pages = {2012},
title = {{Food web including infectious agents for a California freshwater pond}},
volume = {93},
year = {2012}
}
@article{Badcock1949,
author = {Badcock, Ruth M and Badcock, B Y Ruth M},
journal = {Journal of Animal Ecology},
number = {2},
pages = {193--208},
title = {{Studies in Stream Life in Tributaries of the Welsh Dee STUDIES IN STREAM LIFE IN TRIBUTARIES OF THE WELSH DEE}},
volume = {18},
year = {2014}
}
@article{Stagliano2002,
author = {Stagliano, David M and Whiles, Matt R and Journal, Source and American, North and Society, Benthological and March, No},
journal = {Journal of the North American Benthological Society},
keywords = {american prairie streams are,aquatic macroinvertebrates,benthic invertebrate communities in,energy flow,food web,konza prairie biological station,north,organic matter,poorly studied,prairie stream,secondary production,structure,trophic},
number = {1},
pages = {97--113},
title = {{Macroinvertebrate production and trophic structure in a tallgrass prairie headwater stream Macroinvertebrate production and trophic structure in a tallgrass prairie headwater stream}},
volume = {21},
year = {2014}
}
@article{Zetina-Rejon2003,
abstract = {The Huizache-Caimanero coastal lagoon complex on the Pacific coast of Mexico supports an important shrimp fishery and is one of the most productive systems in catch per unit area of this resource. Four other less important fish groups are also exploited. In this study, we integrated the available information of the system into a mass-balance trophic model to describe the ecosystem structure and flows of energy using the Ecopath approach. The model includes 26 functional groups consisting of 15 fish groups, seven invertebrate groups, macrophytes, phytoplankton, and a detritus group. The resulting model was consistent as indicated by the output parameters. According to the overall pedigree index (0.75), which measures the quality of the input data on a scale from 0 to 1, it is a high quality model. Results indicate that zooplankton, microcrustaceans, and polychaetes are the principal link between trophic level (TL) one (primary producers and detritus) and consumers of higher TLs. Most production from macrophytes flows to detritus, and phytoplankton production is incorporated into the food web by zooplankton. Half of the flow from TL one to the next level come from detritus, which is an important energy source not only for several groups in the ecosystem but also for fisheries, as shown by mixed trophic impacts. The Huizache-Caimanero complex has the typical structure of tropical coastal lagoons and estuaries. The TL of consumers ranges from 2.0 to 3.6 because most groups are composed of juveniles, which use the lagoons as a nursery or protection area. Most energy flows were found in the lower part of the trophic web. {\textcopyright} 2003 Elsevier Ltd. All rights reserved.},
author = {Zetina-Rej{\'{o}}n, Manuel J. and Arregu{\'{i}}n-S{\'{a}}nchez, Francisco and Ch{\'{a}}vez, Ernesto A.},
doi = {10.1016/S0272-7714(02)00410-9},
isbn = {02727714},
issn = {02727714},
journal = {Estuarine, Coastal and Shelf Science},
keywords = {Ecopath,Ecosystem structure,Huizache-Caimanero,Mexico,Network analysis,Trophic model},
number = {5-6},
pages = {803--815},
title = {{Trophic structure and flows of energy in the Huizache-Caimanero lagoon complex on the Pacific coast of Mexico}},
volume = {57},
year = {2003}
}
@article{Bird1930,
author = {Bird, Ralph D.},
doi = {10.2307/1930270},
isbn = {0012-9658},
issn = {00129658},
journal = {Ecology},
number = {2},
pages = {356--442},
title = {{Biotic Communities of the Aspen Parkland of Central Canada}},
url = {http://doi.wiley.com/10.2307/1930270},
volume = {11},
year = {1930}
}
@article{Stewart2011,
abstract = {Replicate mass-balanced solutions to Ecopath models describing carbon-based trophic structures and flows were developed for the Lake Ontario offshore food web before and after invasion-induced disruption. The food webs link two pathways of energy and matter flow: the grazing chain (phytoplankton-zooplankton-fish) and the microbial loop (bacteria-protozoans) and include 19 species-groups and three detrital groups. Mass-balance was achieved by using constrained optimization techniques to randomly vary initial estimates of biomass and diet composition. After the invasion, production declined for all trophic levels and species-groups except Chinook salmon. The trophic level (TL) increased for smelt, adult sculpin, adult alewife and Chinook salmon. Changes to ecotrophic efficiencies indicate a reduction in phytoplankton grazing, increased predation pressure on Mysis, adult smelt and alewife and decreased predation pressure on protozoans. Specific resource to consumer TTE changed; increasing for protozoans (8.0-11.5{\%}), Mysis (0.6-1.0{\%}), and Chinook salmon (1.0-2.3{\%}) and other salmonines (0.4-0.5{\%}) and decreasing for zooplankton (20.2-15.1{\%}), prey-fish (9.7-8.8{\%}), and benthos (1.7-0.6{\%}). Direct trophic influences of recent invasive species were low. The synchrony of the decline in PP and species-group production indicates strong bottom-up influence. Mass balance required an increase of two to threefold in lower trophic level biomass and production, confirming a previously observed paradoxical deficit in lower trophic level production. Analysis of food web changes suggest hypotheses that may apply to other similar large pelagic systems including, (1) as pelagic primary productivity declines, overgrazing of zooplankton results in an increase in protozoan production and a loss of trophic transfer efficiency, (2) habitat and food web changes increased Mysis predation on Diporeia and contributed to their recent decline, and (3) production of Chinook salmon, the primary piscivore, was uncoupled from pelagic production processes. This study demonstrates the value of food web models to better understand the impact of invasive species and to develop novel hypotheses concerning trophic influences. {\textcopyright} 2010 Elsevier B.V.},
author = {Stewart, Thomas J. and Sprules, W. Gary},
doi = {10.1016/j.ecolmodel.2010.10.024},
isbn = {0304-3800},
issn = {03043800},
journal = {Ecological Modelling},
keywords = {Carbon fluxes,Ecopath,Food web,Invasive species,Lake Ontario,Mass-balance},
number = {3},
pages = {692--708},
pmid = {17675488},
publisher = {Elsevier B.V.},
title = {{Carbon-based balanced trophic structure and flows in the offshore Lake Ontario food web before (1987-1991) and after (2001-2005) invasion-induced ecosystem change}},
url = {http://dx.doi.org/10.1016/j.ecolmodel.2010.10.024},
volume = {222},
year = {2011}
}
@article{Smith2007,
abstract = {The continental shelf of the Ross Sea is one of the Antarctic's most intensively studied regions. We review the available data on the region's physical characteristics (currents and ice concentrations) and their spatial variations, as well as components of the neritic food web, including lower and middle levels (phytoplankton, zooplankton, krill, fishes), the upper trophic levels (seals, penguins, pelagic birds, whales) and benthic fauna. A hypothetical food web is presented. Biotic interactions, such as the role of Euphausia crystallorophias and Pleuragramma antarcticum as grazers of lower levels and food for higher trophic levels, are suggested as being critical. The neritic food web contrasts dramatically with others in the Antarctic that appear to be structured around the keystone species Euphausia superba. Similarly, we suggest that benthic-pelagic coupling is stronger in the Ross Sea than in most other Antarctic regions. We also highlight many of the unknowns within the food web, and discuss the impacts of a changing Ross Sea habitat on the ecosystem.},
author = {Smith, Walker O. and Ainley, David G. and Cattaneo-Vietti, Riccardo},
doi = {10.1098/rstb.2006.1956},
isbn = {0962-8436},
issn = {09628436},
journal = {Philosophical Transactions of the Royal Society B: Biological Sciences},
keywords = {Bio-physical coupling,Ecosystem function,Ecosystem structure,Neritic food web,Pelagic-benthic coupling,Ross Sea},
number = {1477},
pages = {95--111},
pmid = {17405209},
title = {{Trophic interactions within the Ross Sea continental shelf ecosystem}},
volume = {362},
year = {2007}
}
@article{Schroter2003,
abstract = {Belowground processes are essential for the overall carbon and nitrogen fluxes in forests. Neither the functioning of the soil food web mediating these fluxes, nor its modulation by environmental factors is sufficiently understood. In this study the belowground carbon and nitrogen mineralisation of four European coniferous forest sites (northern Sweden to north-east France) with different climate and N depositional inputs was analysed by investigating the soil food webs using field observations and modelling. The soil fauna directly contributed 7-13{\%} to C mineralisation, among which the testate amoebae (Protozoa) made the largest contribution. Microbial grazing was suggested to have an important indirect effect by stimulating bacterial turnover. Due to relatively high C:N ratios of their substrate, bacteria immobilized N, while the fauna i.e. testate amoebae, nematodes, microarthropods and enchytraeids. counteracted this N immobilisation. Despite similar food web biomass, the sites differed with respect to food web structure and C and N flows. Model calculations suggested a significant influence of food web structure on soil ecosystem processes in addition to environmental factors and resource quality. Mineralisation rates were lowest at the low N input boreal site with a food web dominated by fungal pathways. Further south, as N availability increased. bacterial pathways became more important and the cycling of C and N was faster. The bioavailability of degradable C sources is suggested to be a limiting factor for microbial activity and overall mineralisation rates. In this respect, above- and belowground interactions e.g. transfers of labile C sources from the vegetation to the decomposer system deserve further attention. Our study revealed the combined effects of climate and nutrient inputs to ecosystems and the subsequent changes in the structure and functioning of the systems. If decomposition, and therefore carbon loss, is stimulated as a consequence of structural and/or nutritional changes, resulting for example from continuous industrial N emission, the storage capacity of forest ecosystems Could be altered.},
author = {Schr{\"{o}}ter, Dagmar and Wolters, Volkmar and {De Ruiter}, Peter C.},
doi = {10.1034/j.1600-0579.2003.12064.x},
isbn = {0030-1299},
issn = {00301299},
journal = {Oikos},
number = {2},
pages = {294--308},
title = {{C and N mineralisation in the decomposer food webs of a European forest transect}},
volume = {102},
year = {2003}
}
@incollection{Petipa1979,
address = {Cambridge},
author = {Petipa, T S},
booktitle = {Mar. Production Mechanisms},
editor = {Dunbar, M.},
pages = {233--250},
publisher = {Cambridge University Press},
title = {{Trophic relationships in communities and the functioning of marine ecosystems.}},
year = {1979}
}
@article{Pattie1966,
abstract = {This study was conducted in the alpine regions of the Beartooth Plateau on the border of Wyoming and Montana during the summers of 1961 to 1964 and for shorter periods in the summers of 1958 to 1960. A general description of the geology and topography of the plateau is given. Six vegetation stand types were recognized, and the location, dominant plants, and their importance to the avifauna were described. Sixty-one species of birds were observed, and the status of each is described},
author = {Pattie, Donald L.},
doi = {10.2307/1365715},
issn = {00105422},
journal = {The Condor},
number = {2},
pages = {167--176},
title = {{Alpine Birds of the Beartooth Mountains}},
url = {https://www.jstor.org/stable/info/10.2307/1365715},
volume = {68},
year = {1966}
}
@article{Motta2005,
abstract = {Abstract This study investigated the structure and properties of a tropical stream food web in a small spatial scale, characterizing its planktonic, epiphytic and benthic compartments. The study was carried out in the Potreirinho Creek, a second-order stream located in the south-east of Brazil. Some attributes of the three subwebs and of the conglomerate food web, composed by the trophic links of the three compartments plus the fish species, were determined. Among compartments, the food webs showed considerable variation in structure. The epiphytic food web was consistently more complex than the planktonic and benthic webs. The values of number of species, number of links and maximum food chain length were significantly higher in the epiphytic compartment than in the other two. Otherwise, the connectance was significantly lower in epiphyton. The significant differences of most food web parameters were determined by the increase in the number of trophic species, represented mainly by basal and intermediate species. High species richness, detritus-based system and high degree of omnivory characterized the stream food web studied. The aquatic macrophytes probably provide a substratum more stable and structurally complex than the sediment. We suggest that the greater species richness and trophic complexity in the epiphytic subweb might be due to the higher degree of habitat complexity supported by macrophyte substrate. Despite differences observed in the structure of the three subwebs, they are highly connected by trophic interactions, mainly by fishes. The high degree of fish omnivory associated with their movements at different spatial scales suggests that these animals have a significant role in the food web dynamic of Potreirinho Creek. This interface between macrophytes and the interconnections resultant from fish foraging, diluted the compartmentalization of the Potreirinho food web.},
author = {Motta, Rosin{\^{e}}s L. and Uieda, Virginia S.},
doi = {10.1111/j.1442-9993.2005.01424.x},
isbn = {1442-9985},
issn = {14429985},
journal = {Austral Ecology},
keywords = {Compartmentalized food web,Food chain length,Food web,Omnivory,Pyramid of number,Spatial variation,Taxonomic resolution,Tropical stream},
number = {1},
pages = {58--73},
title = {{Food web structure in a tropical stream ecosystem}},
volume = {30},
year = {2005}
}
@article{Hartley1948,
abstract = {1. A short stretch of the upper reaches of the River Cam and one of its tributaries, the Shepreth Brook, contained a community of eleven species of fish. The area is defined and briefly described. 2. An investigation was made of the food of the members of the fish community, and of a few aquatic birds from the same area. It is shown that, on a basis of food, the fish may be divided into four groups: (a) Specialist predators upon a single prey. One species. (b) Fish taking a wide variety of foods. Two species. (c) Fish taking a diet in which insects and plants predominate. Three species. (d) Fish taking a diet in which insects and crustaceans predominate. Four species. 3. It is shown that between no two species is there complete identity of feeding habit, but that there is much general competition between all the fish of the community for certain staple foods.},
author = {Hartley, P. H. T.},
doi = {10.2307/1604},
issn = {00218790},
journal = {Journal of Animal Ecology},
number = {1},
pages = {1--14},
title = {{Food and feeding relationships in a community of fresh-water fishes}},
volume = {17},
year = {1948}
}
@article{Parker2006,
abstract = {1. We studied the effect of substratum movement on the communities of adjacent mountain and spring tributaries of the Ivishak River in arctic Alaska (69 degrees 1'N, 147 degrees 43'W). We expected the mountain stream to have significant bed movement during summer because of storm flows and the spring stream to have negligible bed movement because of constant discharge. 2. We predicted that the mountain stream would be inhabited only by taxa able to cope with frequent bed movement. Therefore, we anticipated that the mountain stream would have lower macroinvertebrate species richness and biomass and a food web with fewer trophic levels and lower connectance than the spring stream. 3. Substrata marked in situ indicated that 57-66{\%} of the bed moved during summer in the mountain stream and 4-20{\%} moved in the spring stream. 4. Macroinvertebrate taxon richness was greater in the spring (25 taxa) than in the mountain stream (20 taxa). Mean macroinvertebrate biomass was also greater in the spring (4617 mg dry mass m(-2)) than in the mountain stream (635 mg dry mass m(-2)). Predators contributed 25{\%} to this biomass in the spring stream, but only 7{\%} in the mountain stream. 5. Bryophyte biomass was {\textgreater} 1000 times greater in the spring stream (88.4 g ash-free dry mass m(-2)) than the mountain stream (0.08 g ash-free dry mass m(-2)). We attributed this to differences in substratum stability between streams. The difference in extent of bryophyte cover between streams probably explains the high macroinvertebrate biomass in the spring stream. 6. Mean food-web connectance was similar between streams, ranging from 0.18 in the spring stream to 0.20 in the mountain stream. Mean food chain length was 3.04 in the spring stream and 1.83 in the mountain stream. Dolly Varden char (Salvelinus malma) was the top predator in the mountain stream and the American dipper (Cinclus mexicanus) was the top predator in the spring stream. The difference in mean food chain length between streams was due largely to the presence of C. mexicanus at the spring stream. 7. Structural differences between the food webs of the spring and mountain streams were relatively minor. The difference in the proportion of macroinvertebrate biomass contributing to different trophic levels was major, however, indicating significant differences in the volume of material and energy flow between food-web nodes (i.e. food web function).},
author = {Parker, Stephanie M. and Huryn, Alexander D.},
doi = {10.1111/j.1365-2427.2006.01567.x},
isbn = {0046-5070},
issn = {00465070},
journal = {Freshwater Biology},
keywords = {Alaska,Bryophytes,Cinclus,Macroinvertebrates,Salvelinus,Spring streams,Substratum stability},
number = {7},
pages = {1249--1263},
title = {{Food web structure and function in two arctic streams with contrasting disturbance regimes}},
volume = {51},
year = {2006}
}
@incollection{Pechlaner1972,
abstract = {The sprawling patterns of land development common to metropolitan areas of the US have been blamed for high levels of automobile travel, and thus for air quality problems. In response, smart growth programs - designed to counter sprawl - have gained popularity in the US. Studies show that, all else equal, residents of neighborhoods with higher levels of density, land-use mix, transit accessibility, and pedestrian friendliness drive less than residents of neighborhoods with lower levels of these characteristics. These studies have shed little light, however, on the underlying direction of causality - in particular, whether neighborhood design influences travel behavior or whether travel preferences influence the choice of neighborhood. The evidence thus leaves a key question largely unanswered: if cities use land use policies to bring residents closer to destinations and provide viable alternatives to driving, will people drive less and thereby reduce emissions? Here a quasi-longitudinal design is used to investigate the relationship between neighborhood characteristics and travel behavior while taking into account the role of travel preferences and neighborhood preferences in explaining this relationship. A multivariate analysis of cross-sectional data shows that differences in travel behavior between suburban and traditional neighborhoods are largely explained by attitudes. However, a quasi-longitudinal analysis of changes in travel behavior and changes in the built environment shows significant associations, even when attitudes have been accounted for, providing support for a causal relationship. {\textcopyright} 2005 Elsevier Ltd. All rights reserved.},
address = {Warsaw},
archivePrefix = {arXiv},
arxivId = {NIHMS150003},
author = {Handy, Susan and Cao, Xinyu and Mokhtarian, Patricia},
booktitle = {Transportation Research Part D: Transport and Environment},
doi = {10.1016/j.trd.2005.05.002},
editor = {Kajak, Z. and Hillbricht-Ilkowska, A.},
eprint = {NIHMS150003},
isbn = {13619209 (ISSN)},
issn = {13619209},
keywords = {Built environment,Land-use,Smart growth,Travel behavior},
number = {6},
pages = {427--444},
pmid = {4332},
publisher = {Polish Scientific},
title = {{Correlation or causality between the built environment and travel behavior? Evidence from Northern California}},
volume = {10},
year = {2005}
}
@article{Richards1926,
author = {Richards, O. W.},
doi = {10.2307/2256021},
isbn = {00220477},
issn = {00220477},
journal = {The Journal of Ecology},
number = {2},
pages = {244},
title = {{Studies on the Ecology of English Heaths: III. Animal Communities of the Felling and Burn Successions at Oxshott Heath, Surrey}},
url = {https://www.jstor.org/stable/2256021?origin=crossref},
volume = {14},
year = {1926}
}
@article{Collins1976,
abstract = {Low temperature ({\textless}40C) alkaline thermal spring effluents in Yellowstone National Park support a benthic algal-bacterial mat fed upon by a single herbivorous brine fly (Diptera: Ephydridae), which in turn is consumed by a number of arthropod predators (water mites, spiders, and a predaceous fly). A census of an entire spring ecosystem provided a framework upon which to integrate present knowledge of this system.},
author = {Collins, N C and Mitchell, R and Wiegert, R G},
doi = {10.2307/1935046},
isbn = {0012-9658},
issn = {0012-9658},
journal = {Ecology (Washington D C)},
keywords = {01500, Evolution,07506, Ecology: environmental biology - Plant,07508, Ecology: environmental biology - Animal,07514, Ecology:,07517, Ecology: environmental,10012, Biochemistry -,10060, Biochemistry studies - General,10069, Biochemistry,10614, External effects - Temperature as a primary,12100, Movement,13003, Metabolism - Energy and respiratory,13202, Nutrition - General studies, nutritional st,16504, Reproductive system - Physiology and bioche,25508, Development and Embryology - Morphogenesis,31000, Physiology,51504, Plant physiology - Nutrition,51510, Plant physiology - Growth, differentiation,64060,,64076, Invertebrata: comparative,,Acarina [75403],Animals,,Animals, Arthropods, Insects, Invertebrates,Arachnida,Arthropods, Chelicerates, Invertebrates,Bacteria,Bacteria, Cyanobacteria, Eubacteria, Microorganism,Bacteria, Eubacteria, Microorganisms,Cyanobacteria [09200]/Insecta, Arthropoda, Inverte,Diptera,Ecology (Environmental Sciences),Environmental Sciences),Freshwater Ecology (Ecology,,Gases,Invertebrata: comparative, experimental morphology,Microorganisms,Nutrition,Physiology,[05000]/Oxygenic Photosynthetic Bacteria, Eubacter,[75314]/Chelicerata, Arthropoda, Invertebrata, Ani,[75402]/Chelicerata, Arthropoda, Invertebrata, Ani,and biochemistry of bacteria,biology - Water research and fishery biology,environmental biology - Limnology,experimental morphology, physiology and pathology,metabolism,methods,pathology - Arthropoda: chelicerata,studies - Minerals,variable},
number = {6},
pages = {1221--1232},
title = {{Functional Analysis of a Thermal Spring Ecosystem With an Evaluation of the Role of Consumers}},
volume = {57},
year = {1976}
}
@incollection{Petipa1970,
abstract = {Black Sea; phytoplankton and zooplankton only; consumption and expenditure of matter by communities (epi- and bathyplankton); specific rates of uptake, energy accumulation and energy metabolism for each trophic level of each community; production of each trophic level; value and specific rates of energy flow through each trophic level; ecological efficiency of each level},
address = {Edinburgh},
author = {Petipa, T S and Pavlova, E V and Mironov, G N},
booktitle = {Marine Food Chains},
editor = {Steele, J. H.},
keywords = {community ecology,no copy,notes,trophic structure},
pages = {142--167},
publisher = {Oliver and Boyd},
title = {{The food web structure, utilization and transport of energy by trophic levels in the plnaktonic communities}},
volume = {-},
year = {1970}
}
@article{Larsson1978,
abstract = {The Norwegian subalpine lake, {\O}vre Heimdalsvatn, has a surface area of {\$}0.78\backslash {\{}\backslashrm km{\}}{\^{}}{\{}2{\}}{\$} and a maximum depth of 13 m. It is ice-covered for 7.5-8 months, has a marked spring spate and a mean annual renewal period of about two months. The water is poor in electrolytes. Intensive studies have been made by an interdisciplinary team of the lake's physical and chemical properties, primary production and secondary production under the auspices of IBP/PF from 1969 to 1973. Allochthonous material accounted for 1/3 of utilized plant input. The major lake predator, the brown trout, fed largely on benthic organisms and did not exploit the zooplankton biomass. On account of the long period of ice cover and the rapid rise in temperature after ice break, many organisms (both planktonic and benthic) showed synchronous development. Abiotic conditions, such as the nature of the spring spate and the temperature rise, strongly affect species and community development.},
author = {Larsson, Petter and Brittain, John E. and Lien, Leif and Lillehammer, Albert and Tangen, Karl},
doi = {10.1111/j.1600-0587.1978.tb00965.x},
isbn = {01059327},
issn = {16000587},
journal = {Ecography},
keywords = {{\O}},
number = {2-3},
pages = {304--320},
title = {{The lake ecosystem of {\O}vre Heimdalsvatn}},
url = {http://www.jstor.org/stable/3682696},
volume = {1},
year = {1978}
}
@book{Parin1970,
address = {Springfield, VA},
author = {Parin, N V and Collection, Moved T O Mlml Library},
booktitle = {U.S. Dept. of the Interior and the National Science Foundation, Washington, D.C.},
pages = {206 pp.},
publisher = {Israel Program for Scientific Translations (available from the U.S. Department of Commerce Clearinghouse for Federal Scientific and Technical Information)},
title = {{Ichthyofauna of the epipelagic zone.}},
year = {1968}
}
@incollection{Patten1975,
address = {New York},
author = {Patten, B C and Egloff, D A and Richardson, T H},
booktitle = {System Analysis and Simulation in Ecology.},
doi = {http://dx.doi.org/10.1016/B978-0-12-547203-6.50018-9},
editor = {Patten, B. C.},
isbn = {978-0-12-547203-6},
issn = {1483262731},
pages = {206--423.},
publisher = {Academic Press},
title = {{Total ecosystem model for a cove in Lake Texoma.}},
url = {http://www.sciencedirect.com/science/article/pii/B9780125472036500189},
volume = {III},
year = {1975}
}
@incollection{Patten1979,
abstract = {Describes Flow Analysis in Detail; But Nothing Special},
address = {New York},
author = {Patten, Bernard C. and Finn, John T.},
booktitle = {Theoretical Systems Ecology},
doi = {10.1016/B978-0-12-318750-5.50014-3},
editor = {Halfon, E.},
isbn = {0123187508},
keywords = {Compartmental Analysis,Cycling Index,Ecology,Food Webs,Methods,Theory},
pages = {183--212},
publisher = {Academic Press},
title = {{Systems Approach To Continental Shelf Ecosystems}},
url = {https://linkinghub.elsevier.com/retrieve/pii/B9780123187505500143},
year = {1979}
}
@article{Poepperl2003,
abstract = {Trophic interactions and cycling of organic carbon within the macroinvertebrate community of a Northern German lowland stream were analyzed based on a compartment model. The network model describes the structure of the food web quantifying biomass, production, and consumption of their ele- ments, of the entire system and between trophic levels. System primary production is 153.7 g C m–2 yr–1 and invertebrate production 53.3 g C m–2 yr–1. Invertebrate consumption amounts to 702.6 g C m–2 yr–1. Main flows are identified between trophic level 1 and 2 and are connected with highly productive compartments. ‘Anodonta and Pseudanodonta' and Dreissena polymorpha show the highest consump- tion of all groups with 269.9 g C m–2 yr–1 and 114.1 g C m–2 yr–1, respectively. System consumption is highest on the import from the upstream lake with 532.5 g C m–2 yr–1, sediment detritus with 135.5 g C m–2 yr–1, and primary producers with 25.7 g C m–2 yr–1. The lowest predation pressure is observed for Bivalvia with an ecotrophic efficiency of {\textless}10{\%} and highest for Chironomidae with 91{\%}. Approxi- mately 20{\%} of organic matter entering the detritus pool are recycled to the living groups of the system. Transfer efficiencies between discrete trophic levels are generally low except for transfer of detrital material between level I and II.},
author = {Poepperl, Rainer},
doi = {10.1002/iroh.200310666},
isbn = {1434-2944},
issn = {14342944},
journal = {International Review of Hydrobiology},
keywords = {Connectance,Consumption,Mass-balance,Network analysis,Trophic interactions},
number = {5},
pages = {433--452},
title = {{A Quantitative Food Web Model for the Macroinvertebrate Community of a Northern German Lowland Stream}},
volume = {88},
year = {2003}
}
@article{Morgan1972,
author = {Morgan, N C and McLusky, D S},
journal = {Proceedings of the Royal Society of Edinburgh: B},
pages = {407--416},
title = {{A summary of the Loch Leven IBP results in relation to lake management and future research}},
volume = {74},
year = {1974}
}
@incollection{Odum1975,
address = {New York},
author = {Odum, E P and Heald, E J},
booktitle = {Estuarine Research},
editor = {Cronin, L. Eugene},
isbn = {0121975010},
issn = {0323142702},
pages = {265--286},
publisher = {Academic Press},
title = {{The detritus bases food web of an estuarine mangrove community}},
year = {1975}
}
@incollection{Qazim1970,
address = {Edinburgh},
author = {Qazim, S Z},
booktitle = {Marine Food Chains},
editor = {Steele, J. H.},
keywords = {doc,estuaries,phytoplankton,primary production,theory,trophic levels,zooplankton},
number = {;},
pages = {45--51},
publisher = {Oliver and Boyd},
title = {{Some problems related to the food chain in a tropical estuary}},
year = {1971}
}
@book{Nybakken1982,
address = {New York},
author = {Nybakken, Jw and Bertness, M D},
pages = {592},
publisher = {Harper and Row},
title = {{Marine biology: an ecological approach}},
url = {http://www.citeulike.org/group/16475/article/10562819},
year = {2004}
}
@incollection{Mizuno1982,
abstract = {Artificial O-2-carrying hemoprotein composed of human serum albumin including tetrakis(o-amidophenyl)-porphinatoiron(II) (Fe4P or Fe3P) [HSA-FeXP] has been modified by maleimide- or succinimide-terminated poly(ethylene alycol) (PEG), and the formed PEG bioconjugates have been physicochemically characterized. 2-Iminothiolane (IMT) reacted with the amino groups of Lys to create active thiol groups, which bind to alpha-maleimide-omega-methoxy PEG [Mw: 2-kDa (PEG(M2)), 5-kDa (PEG(M5))]. On the other hand, alpha-succinimidyl-omega-methoxy PEG [Mw: 2-kDa (PEG(S2)), 5-kDa (PEG(S5))] directly binds to Lys residues. MALDI-TOF MS of the PEG-conjugated HSA-FeXP showed distinct molecular ion peaks, which provide an accurate number of the PEG chains. In the case of PEG(MY)(HSA-FeXP), the spectroscopic assay of the thiol groups also provided the mean of the binding numbers of the polymers, and the degree of the modification was controlled by the ratio of [IMT]/[HSA]. The viscosity and colloid osmotic pressures of the 2-kDa PEG conjugates (phosphate-buffered saline solution, [HSA] = 5 g dL(-1)) were almost the same as that of the nonmodified one, whereas the 5-kDa PEG binding increased the theological parameters. The presence of flexible polymers on the HSA surface retarded the association reaction of O-2 to FeXP and stabilized the oxygenated complex. Furthermore, PEG(MY)(HSA-FeXP) exhibited a long circulation lifetime of FeXP in rats (13-16 h). On the basis of these results, it can be concluded that the surface modification of HSA-FeXP by PEG has improved its comprehensive O-2-transporting ability. In particular the PEG(MY)(HSA-FeXP) solution could be a promising material for entirely synthetic O-2-carrying plasma expander as a red cell substitute.},
address = {The Hague},
author = {Mizuno, N. and Furtado, J. I.},
booktitle = {Ecology},
doi = {10.1007/978-94-009-7980-2_7},
editor = {Furtado, J. I. and Mori, S.},
isbn = {978-94-009-7980-2},
number = {1},
pages = {321--354},
pmid = {236226200020},
publisher = {Dr. W. Junk Publishers},
title = {{Ecological notes on fishes}},
url = {http://www.springerlink.com/index/10.1007/978-94-009-7980-2{\_}7},
volume = {42},
year = {1982}
}
@article{Beaver1985,
abstract = {Abstract.  1. Relative to Nepenthes species in West Malaysia near the evolutionary centre of the genus, outlying species of Nepenthes in the Seychelles, Sri Lanka and Madagascar have fewer species of both prey and predator living in them, fewer and smaller guilds of species, much apparently empty niche space, less complex food webs, and a greater connectance. The ratios of prey to predators, and of connectance (C1) to the total number of trophic types present remain approximately constant. 2. Differences between the food webs appear to be related in a complex way to the size of the country and its degree of spatial and temporal isolation, the size of the local species pool capable of colonizing the pitchers, and the number of Nepenthes species present. However, the maximal length of food chains in the richest and most complex food webs is probably limited by energetic constraints or environmental predictability. 3. The data may illustrate how food webs change to become more complex, both by the addition of new guilds of species and the addition of species to existing guilds, while at the same time certain properties of the food web are kept approximately constant.},
author = {BEAVER, R. A.},
doi = {10.1111/j.1365-2311.1985.tb00720.x},
issn = {13652311},
journal = {Ecological Entomology},
keywords = {Food webs,Nepenthes,connectance,food chain lengths,geographical variation,predator‐prey ratios,species guilds},
number = {3},
pages = {241--248},
pmid = {12200004606},
title = {{Geographical variation in food web structure in Nepenthes pitcher plants}},
url = {http://onlinelibrary.wiley.com/doi/10.1111/j.1365-2311.1985.tb00720.x/abstract},
volume = {10},
year = {1985}
}
@article{Koepcke1952,
author = {Koepcke, Hans-Wilhelm and Koepcke, Maria},
journal = {Revista de Ciencias},
number = {479-480},
pages = {5--29},
title = {{Sobre el proceso de transformaci{\'{o}}n de la materia org{\'{a}}nica en las playas arenosas marinas del Per{\'{u}}}},
volume = {54},
year = {1952}
}
@incollection{Kitching1967,
abstract = {This chapter reviews on how the distribution of marine organisms in a limited coastal area is determined, how they react with each other and with their inanimate environment, and how their numbers are controlled. The chapter involves the assessment of physical and chemical conditions of the environment, observations of distribution and abundance of plants and animals, field experiments in which it interfere either with the inanimate environment or with its occupants, observations of behavior, and experiments in the laboratory. The chapter starts with a brief description of the topography of Lough Ine, and with some comments on those geological characteristics which throw light on the origin of the Lough or determine the present features of the shore. It proceeds to the hydrographical features - the water currents set up by the tides and the physical and chemical conditions of the water, and the distribution of sediment. It then describes the distribution of some important and conspicuous inhabitants of the rocky shores, relating their distribution to current, wave action, exposure to air, and other inanimate conditions, and biotic factors such as predation. Systematic problems and problems of natural selection arose incidentally during the course of the work.},
address = {London},
author = {Kitching, J. A. and Ebling, F. J.},
booktitle = {Advances in Ecological Research},
doi = {10.1016/S0065-2504(08)60322-0},
editor = {Cragg, J. B.},
isbn = {0065-2504},
issn = {00652504},
number = {C},
pages = {197--291},
publisher = {Academic Press},
title = {{Ecological Studies at Lough Ine}},
volume = {4},
year = {1967}
}
@book{Kremer1978,
address = {Berlin},
author = {Silvert, William},
booktitle = {Journal of the Fisheries Research Board of Canada},
doi = {10.1139/f79-087},
isbn = {0387083650},
issn = {0015-296X},
number = {5},
pages = {597--598},
publisher = {Springer-Verlag},
title = {{A Coastal Marine Ecosystem. Simulation and Analysis.}},
url = {http://www.nrcresearchpress.com/doi/10.1139/f79-087},
volume = {36},
year = {1979}
}
@incollection{Walsh1967,
address = {Washington},
author = {Walsh, G E},
booktitle = {Dissertation},
editor = {Lauff, G. H.},
keywords = {ADDED REFERENCE / POLYCHAETA},
pages = {219},
publisher = {American Association for the Advancement of Science, Publication 83},
title = {{An ecological study of the Heeia mangrove swamp. University of Hawaii, Honolulu}},
url = {http://scholarspace.manoa.hawaii.edu/handle/10125/11982},
year = {1963}
}
@incollection{Milne1972,
abstract = {Work currently being carried out on the Ythan estuary by members of staff, research students and undergraduates of the Zoology Department, University of Aberdeen, is part of a long-term programme planned in 1964. The aim is to quantify the seasonably variable exploitation of the invertebrate fauna of the estuary by the wide variety of predators present, mainly birds and fish. In the past seven years, three staff members, ten Ph.D. students and about ten undergraduates have participated actively in research projects. Some of the results have already been published but many data are still in unpublished theses. This paper is the first attempted synthesis of the work. Whereas we are primarily concerned here with the interrelations between populations of animals within the estuary, this particular estuarine ecosystem also offered the opportunity to make detailed population studies per se, in a relatively natural situation.  Our studies have been deliberately confined to the upper trophic levels of the ecosystem; we believe that a useful contribution can be made to the understanding of estuarine food chains, and to predatorprey interrelations in particular, without becoming involved, initially at least, in studies of the food and feeding of the prey. Bird predators are easily observed and quantitative data on their numbers, distribution, feeding behaviour, feeding rates, and diet can readily be obtained, but the fish species are much more difficult to study in this context, and we have to rely on various catching methods to provide much of the data. The invertebrate prey species are readily accessible, in intertidal areas exposed at low tide, allowing studies of distribution, dispersion pattern, standing crop numbers and biomass, and productivity to be made.  The project is recognised as part of the UK's supporting programme to the International Biological Programme (PM Section).},
address = {Edinburgh},
author = {Milne, H and Dunnet, G M},
booktitle = {The estuarine environment},
editor = {Barnes, R. S. K. and Green, J.},
keywords = {PRODUCAO CADEIA TROFICA BIOMASSA ESTU},
pages = {86----106.},
publisher = {Applied Science Publications},
title = {{Standing crop, productivity and trophic relations of the fauna of the Ythan estuary.}},
year = {1972}
}
@article{Massana1996,
abstract = {We studied the planktonic community established in Lake Ciso (Girona, Spain) during summer stratification, with the aim of describing the food web of a system as completely as possible. The lake was sampled 19 times during 1990 and 1991. We first determined which populations contributed significantly to total summer biomass. Then, we determined the trophic role of these populations by several independent approaches, and aggregated the community into functional groups. The binary food web obtained indicated that Me structure of Me food web in Lake Ciso was similar to that found in other systems. Finally, we quantified Me trophic fluxes among populations using a simple algorithm which considers the vertical distribution of organisms and the functional responses of the different predators. The trophic food web obtained revealed 2 interesting properties. First, the compartments with larger biomass were relatively stable during stratification and presented slow growth and low predatory losses. Second, there was a very inefficient transfer of organic matter from the lower levels (bacteria, algae and protozoans) to the higher levels (rotifers and zooplankton) of the food web. Both properties could be explained by the fact that most biomass of the system accumulated in the metalimnion, along opposite gradients of oxygen and sulfide, which determined an environment with reduced competition and predation. We postulate that metalimnetic communities above anaerobic hypolimnia can be regarded as sinks of organic matter off the epilimnion.},
author = {Massana, Ramon and Garc{\'{i}}a-Cantizano, Josefina and Pedr{\'{o}}s-Ali{\'{o}}, Carlos},
doi = {10.3354/ame011279},
isbn = {0948-3055},
issn = {09483055},
journal = {Aquatic Microbial Ecology},
keywords = {Binary food web,Lake Cis{\'{o}},Prey refuge,Trophic food web},
number = {3},
pages = {279--288},
pmid = {7070},
title = {{Components, structure and fluxes of the microbial food web in a small, stratified lake}},
volume = {11},
year = {1996}
}
@incollection{Mackintosh1964,
address = {Paris},
author = {Mackintosh, N a},
booktitle = {Biologie Antarctique},
editor = {Carrick, R. and Holdgate, M. and Prevost, J.},
pages = {29--38},
publisher = {Hermann},
title = {{A survey of Antarctic biology up to 1945}},
year = {1964}
}
@incollection{Mann1972,
address = {New York},
author = {Mann, K.H.},
booktitle = {River Ecology and Man},
doi = {10.1016/B978-0-12-524450-3.50018-6},
editor = {Oglesby, R. T. and Carlson, C. A. and McCann, J. A.},
number = {Figure 1},
pages = {215--232},
publisher = {Academic Press},
title = {{Case History: the River Thames}},
url = {http://linkinghub.elsevier.com/retrieve/pii/B9780125244503500186},
year = {1972}
}
@incollection{Mann1972a,
abstract = {The trophic model for the River Thames, England, developed by the International Biological Programme (IBP) is probably the most complete ever constructed for a riverine ecosystem. A Mark 2 model is presented here, constructed using ECOPATH II. The model reinforces many of the conclusions of the earlier study and exposes certain weaknesses. In particular, the role of the main fish populations and of detritus is revised. Certain improvments that could be made to Mark 2 model are identified, relating to the inclusion of incoming solar energy and to the efficiency of the community in converting solar energy to animal and plant tissue.},
address = {Warsaw},
author = {Mathews, C.P.},
booktitle = {ICLARM conference procedings},
editor = {Kajak, Z. and Hillbricht-Ilkowska, A.},
isbn = {971-1022-84-2},
pages = {161--171},
publisher = {Polish Scientific},
title = {{Productivity and energy flows at all trophic levels in the river Thames, England: Mark 2}},
volume = {26},
year = {1993}
}
@incollection{Marshall1982,
address = {The Hague},
author = {Merriam, C H},
booktitle = {Lake McIlwaine: the eutrophication and recovery of a tropical man-made lake. Monographia Biologicae, Vol. 49},
editor = {Thornton, J. A.},
keywords = {WALLEYE},
pages = {287--288},
publisher = {D. W. Junk Publishers},
title = {{The fish of Lake Champlain}},
volume = {4},
year = {1884}
}
@article{Torres2013,
abstract = {The Gulf of Cadiz (North-eastern Atlantic, Spain) is an exploited ecosystem characterized by high marine biodiversity and productivity. Over the last decade, the landings of fish stocks such as anchovy (Engraulis encrasicolus), sardine (Sardina pilchardus) and hake (Merluccius merluccius) have been declining and currently remain low. A food-web model of the Gulf of Cadiz has been developed by means of a mass balance approach using the software EwE 6 to provide a snapshot of the ecosystem in 2009. The goals of this study were to: (1) characterize the food-web structure and functioning, (2) identify the main keystone groups of the ecosystem, (3) assess the impact of fishing to the Gulf of Cadiz compared to that in other essential marine ecosystems in the coastal area of Spain: Cantabrian Sea (North-eastern Atlantic) and Southern Catalan Sea (Mediterranean Sea), and (4) examine the limitations and weaknesses of the Gulf of Cadiz model for improvements and future research directions. The model consists of 43 functional groups, including the main trophic components of the system with emphasis target and non-target fish species. The main trophic flows are determined by the interaction between detritus, phytoplankton and micro- and mesozooplankton. Rose shrimp (Parapenaeus longirostris), cephalopods and dolphins present important overall effects as keystone species on the rest of the groups. The exploitation of fisheries composed mainly of trawlers, purse seiners and artisanal boats is intensive in the Gulf of Cadiz with all fleets exerting high impacts on most living groups of the ecosystem. The findings highlighted that the Gulf of Cadiz is a notably stressed ecosystem, displaying characteristics of a heavily exploited area. The comparative approach highlights that the three ecosystems display similarities with regard to structure and functioning such as the dominance of the pelagic fraction, a strong benthic-pelagic coupling, the important role of detritus, and the high impact of fishery exploitation. {\textcopyright} 2013 Elsevier B.V.},
author = {Torres, Mar{\'{i}}a {\'{A}}ngeles and Coll, Marta and Heymans, Johanna Jacomina and Christensen, Villy and Sobrino, Ignacio},
doi = {10.1016/j.ecolmodel.2013.05.019},
isbn = {0304-3800},
issn = {03043800},
journal = {Ecological Modelling},
keywords = {Ecopath,Ecosystem approach,Fishing impacts,Food-web model,Gulf of Cadiz,Information theory,Trophic network analysis},
pages = {26--44},
publisher = {Elsevier B.V.},
title = {{Food-web structure of and fishing impacts on the Gulf of Cadiz ecosystem (South-western Spain)}},
url = {http://dx.doi.org/10.1016/j.ecolmodel.2013.05.019},
volume = {265},
year = {2013}
}
@article{Zaret1973,
abstract = {-Introduction of Cichla ocellaris to Lake Gatun, Panama, had devastating effects on the ecosystem - all trphic levels were impacted either by direct or indirect effects -Same fish as introduced to Dade Cty canals for "better fishing" (same as in Lake Gutan)-One species of native fish increased its numbers dramitically -authors speculate this is a result of the natural predators on this species fry were removed from the community by Cichla -the response of P. latiipinna to C. uroph. may be very similar; after all P. lat. has a unique breeding pattern compared to all the othe native species which lay eggs - a likely target for an oppertunist such as C.uroph.- TS - ProCite 5},
author = {Zaret, Thomas M. and Paine, R. T.},
doi = {10.1126/science.182.4111.449},
isbn = {0036-8075},
issn = {00368075},
journal = {Science},
keywords = {Freshwater,Q1 01482 Ecosystems and energetics},
number = {4111},
pages = {449--455},
pmid = {17832455},
title = {{Species introduction in a tropical lake}},
volume = {182},
year = {1973}
}
@article{Yodzis1998,
abstract = {1. A method for finding the consequences of long-term generalized press perturbations in multispecies ecological communities, with relatively modest requirements for data, is explicated. The approach uses energetic and allometric reasoning to set some parameter values for which data are not available. The remaining unknown parameters are treated as random variables, enabling the calculation of probability distributions for the outcomes that are of interest. 2. The method is used to investigate the effect of a cull of fur seals on fisheries in the Benguela ecosystem, using a 29-species foodweb for that system. In the case of Cape fur seals treated here, it is found that a cull of seals is more likely to be detrimental to total yields from all exploited species than it is to be beneficial. 3. The influence of weak links on the effects of a cull is investigated. Using both consumption by each species and consumption of each species to define link strength, a clear threshold in link strength is found, indicating that 44{\%} of all links could be deleted from the foodweb without affecting the predictions significantly. Even using a criterion based on consumption by each species alone (conventional dietary proportion data), about the same number of links can be deleted without seriously affecting the predictions of the model. This is a very helpful (and encouraging) result for the design of an observational protocol for systematic efforts to gather data for multispecies modelling.},
author = {Yodzis, Peter},
doi = {10.1046/j.1365-2656.1998.00224.x},
isbn = {00218790},
issn = {00218790},
journal = {Journal of Animal Ecology},
keywords = {Food web,Indirect effect,Mathematical model,Perturbation experiment},
number = {4},
pages = {635--658},
pmid = {2013522265},
title = {{Local trophodynamics and the interaction of marine mammals and fisheries in the Benguela ecosystem}},
url = {http://doi.wiley.com/10.1046/j.1365-2656.1998.00224.x},
volume = {67},
year = {1998}
}
@article{Bradstreet1982,
abstract = {This multivariate statistical analysis is designed to isolate the effect of size from other variables that influence the underlying cost structure of municipal solid waste incinerators. Hence, the regression models for both construction and operating costs investigate several inherent factors, including design capacity, material and/or energy recovery, air pollution control, ownership, as well as actual and expected operating hours for different technologies. Several econometric model structures are evaluated systemmatically to find the most appropriate functional form for both construction costs and operating costs. The tested functional forms cover multiplicative single equation models, multiplicative recursive equation systems, just or over identified simultaneous systems, nonlinear seeming unrelated regression (SUR) systems. IML and ETS subroutines in SAS package have been applied for solving these models. Nonlinear SUR is selected as the representative model in this paper. The major estimation issues examined are heteroscedasticity, collinearity and the influence of outliers. But the scale economy, cost elasticity, and the interaction between construction and operating costs are also of significance economically. Based on nearly 150 observations, normalizations for different geographic regions and time are conducted and facilities are classified as mass burn waterwall, modular, refuse-derived-fuel (RDF), refractory incinerators and rotary combustors. The results show that it is important to model individual technologies separately to avoid exaggerating the importance of scale economy by overall observations. {\textcopyright} 1993.},
author = {Chang, Ni Bin and Mount, Timothy D. and Schuler, Richard E.},
doi = {10.1016/0266-9838(93)90013-8},
isbn = {0004-0843},
issn = {02669838},
journal = {Environmental Software},
keywords = {cost prediction,econometric analysis,incinerator planning,solid waste management},
number = {3},
pages = {173--186},
pmid = {39},
title = {{Econometric analysis of the construction and operating costs of municipal solid waste incinerators}},
volume = {8},
year = {1993}
}
@article{Dunbar1953,
abstract = {Article},
author = {Dunbar, M. J.},
doi = {10.2307/40506631},
issn = {1923-1245},
journal = {Arctic},
pages = {75--90},
title = {{Arctic and subarctic marine ecology: Immediate problems}},
url = {http://pubs.aina.ucalgary.ca/arctic/Arctic6-2-75.pdf{\%}0Ahttps://www.jstor.org/stable/pdf/40506631.pdf},
volume = {6},
year = {1953}
}
@article{Summerhayes1928,
author = {Summerhayes, V S},
doi = {10.2307/2255796},
issn = {00220477},
journal = {Journal of Ecology},
number = {2},
pages = {201----212 CR ---- Copyright {\{}{\&}{\}}{\{}{\#}{\}}169; 1928 British E},
title = {{North-East Land and Hinlopen Strait}},
url = {http://www.jstor.org/stable/2255796},
volume = {16},
year = {1928}
}
@article{Summerhayes1923,
author = {Summerhayes, V. S. and Elton, Charles S.},
doi = {10.2307/2255864},
isbn = {00220477},
issn = {0022-0477},
journal = {Journal of Ecology},
number = {2},
pages = {216--233},
title = {{Bear Island}},
volume = {11},
year = {1923}
}
@article{Whittaker1984,
abstract = {JSTOR is a not-for-profit service that helps scholars, researchers, and students discover, use, and build upon a wide range of content in a trusted digital archive. We use information technology and tools to increase productivity and facilitate new forms of scholarship. For more information about JSTOR, please contact support@jstor.org. Southwestern Association of Naturalists is collaborating with JSTOR to digitize, preserve and extend access to The Southwestern Naturalist.},
author = {Smith, Michael H},
doi = {10.1894/0038-4909-55.2.310},
isbn = {978-1-4673-0867-0},
issn = {0038-4909},
journal = {The Southwestern Naturalist},
number = {2},
pages = {310--310},
pmid = {21519884},
title = {{Southwestern Association of Naturalists}},
url = {http://www.bioone.org/doi/abs/10.1894/0038-4909-55.2.310},
volume = {55},
year = {2010}
}
@article{Landry1977,
abstract = {The physical environment has an important influence on the size composition of primary producers in plankton communities. This effect is transmitted through the trophic structure by size selective feeding processes at each level.},
author = {Landry, M. R.},
doi = {10.1007/BF02207821},
isbn = {0017-9957},
issn = {00179957},
journal = {Helgol{\"{a}}nder Wissenschaftliche Meeresuntersuchungen},
number = {1-4},
pages = {8--17},
pmid = {24801165},
title = {{A review of important concepts in the trophic organization of pelagic ecosystems}},
volume = {30},
year = {1977}
}
@article{Hewatt1937,
author = {Hewatt, Willis G},
doi = {10.2307/2420496},
isbn = {19490F},
issn = {00030031},
journal = {The American Midland Naturalist},
keywords = {ASTEROIDEA,CALIFORNIA,COMMUNITIES,DESCRIPTIONS,EASTERN CENTRAL PACIFIC,EASTERN PACIFIC,ECHINODERMATA,ECOLOGY,INTERTIDAL,INVERTEBRATES,LIFE HISTORY,MARINE,NORTH AMERICA,PACIFIC,SURVEY},
number = {2},
pages = {161--206},
title = {{Ecological studies on selected marine intertidal communities of Monterey Bay, California}},
url = {http://www.jstor.org/stable/2420496{\%}0Ahttp://www.jstor.org/stable/10.2307/2420496},
volume = {18},
year = {1937}
}
@article{Vinogradov1978,
abstract = {The principal trophic levels, each subdivided into groups of organismic elements, are distinguished in the planktonic communities of the Eastern Equatorial and the Peruvian upwellings. Production intensity or metabolism have been determined experimentally for all elements. A scheme is suggested for computing production from data on metabolism for all the elements of a community, as well as for computing net and real production and other functional characteristics for definite trophic levels and the community as a whole. Based on the quantitative estimation of the efficiency of primary production and other functional characteristics, the development of communities is divided into production and destruction periods; they are, in turn, subdivided into steps associated with a certain degree of water trophicity. The balance of net production of the communities in the Peruvian upwelling indicates that the excess production of a community above the shelf is utilized completely in the narrow (1OO to 150 sea miles) band of off-shore water. This paper describes an attempt to trace the changes taking place in the functional characteristics of plankton communities and to compare them with the changes observed in the communities of the Peruvian and East-Equatorial upwellings.},
author = {Vinogradov, M. E. and Shushkina, E. A.},
doi = {10.1007/BF00391640},
issn = {00253162},
journal = {Marine Biology},
number = {4},
pages = {357--366},
title = {{Some development patterns of plankton communities in the upwelling areas of the Pacific Ocean}},
volume = {48},
year = {1978}
}
@article{Rejmanek1979,
abstract = {Nature},
author = {Rejm{\'{a}}nek, M. and Star{\'{y}}, P.},
doi = {10.1038/280311a0},
isbn = {0028-0836},
issn = {00280836},
journal = {Nature},
number = {5720},
pages = {311--313},
pmid = {581},
title = {{Connectance in real biotic communities and critical values for stability of model ecosystems [9]}},
volume = {280},
year = {1979}
}
@article{Lin2006,
abstract = {Tapong Bay, a eutrophic and poorly flushed tropical lagoon, supports intensive oyster culture. Using the Ecopath approach and network analysis, a mass-balanced trophic model was constructed to analyze the structure and matter flows within the food web. The lagoon model is comprised of 18 compartments with the highest trophic level of 3.2 for piscivorous fish. The high pedigree index (0.82) reveals the model to be of high quality. The most-prominent living compartment in terms of matter flow and biomass in the lagoon is cultured oysters and bivalves, respectively. The mixed trophic impacts indicate that phytoplankton and periphyton are the most-influential living compartments in the lagoon. Comparative analyses with the eutrophic and well-flushed Chiku Lagoon and non-eutrophic tropical lagoons show that high nutrient loadings might stimulate the growth and accumulation of phytoplankton and periphyton and therefore support high fishery yields. However, net primary production, total biomass, fishery yields per unit area, and mean transfer efficiency of Tapong Bay were remarkably lower than those of Chiku Lagoon. The lower transfer efficiency likely results from the low mortality of cultured oysters and invasive bivalves from predation or the lower density of benthic feeders constrained by the hypoxic bottom water as a result of poor flushing. This might therefore result in a great proportion of flows to detritus. However, the hypoxic bottom water might further reduce the recycling of the entering detritus back into the food web. In contrast to many estuaries and tropical lagoons, poor flushing of this eutrophic tropical lagoon might induce a shift from detritivory to herbivory in the food web. {\textcopyright} 2006 Elsevier Ltd. All rights reserved.},
author = {Lin, Hsing Juh and Dai, Xiao Xun and Shao, Kwang Tsao and Su, Huei Meei and Lo, Wen Tseng and Hsieh, Hwey Lian and Fang, Lee Shing and Hung, Jia Jang},
doi = {10.1016/j.marenvres.2006.03.003},
isbn = {0141-1136},
issn = {01411136},
journal = {Marine Environmental Research},
keywords = {Ecopath,Eutrophication,Flushing,Network analysis,Tapong Bay,The black striped mussel},
number = {1},
pages = {61--82},
pmid = {16626801},
title = {{Trophic structure and functioning in a eutrophic and poorly flushed lagoon in southwestern Taiwan}},
volume = {62},
year = {2006}
}
@article{Liu2007,
abstract = {An Ecopath model was constructed to describe the ecosystem of Lake Qiandaohu, a stock-enhanced large deep Chinese reservoir with silver carp (Hypophthalmichthys molitrix) and bighead carp (Aristichthys nobilis) dominated in its pelagic community. The food web structure and ecosystem property of the reservoir were analyzed and evaluated. The results showed that there were seven trophic levels (TLs) in the system, with the trophic flows primarily occurring through the first four TLs. The food web structure of this ecosystem was characterized with a bulged intermediate trophic level, which was contrary to the wasp-waist food web structure occurred in most natural aquatic ecosystems. The corresponding trophic flow pattern showing by transfer efficiencies (TEs) between TLs indicated that the trophic flows primarily went through from TL I to II with a high TE (of over 50{\%}) and through a flow loop or short cut between detritus and TL II but greatly reduced from TL II to III with a lowest TE of 2.5{\%} due to the bulged biomass at TL II. The trophic flow loop greatly increased the throughput recycled, which, together with high connectance index (CI), system omnivory index (SOI), Finn's cycled index (FCI) and Finn's mean path length (FML), might be beneficial to the maintaining of ecosystem stability. Finally, ecosystem property indices showed that this reservoir had a high value of Pp/R and Pp/B, indicating this ecosystem of short history was immature, but highly productive. This silver carp and bighead carp dominated deep reservoir ecosystem had both the characteristics of high productivity of an immature ecosystem and the feature of high stability of a mature ecosystem. {\textcopyright} 2006 Elsevier B.V. All rights reserved.},
author = {Liu, Qi Gen and Chen, Yong and Li, Jia Le and Chen, Li Qiao},
doi = {10.1016/j.ecolmodel.2006.11.028},
isbn = {0304-3800},
issn = {03043800},
journal = {Ecological Modelling},
keywords = {Aristichthys nobilis,Ecopath model,Ecosystem properties,Hypophthalmichthys molitrix,Lake Qiandaohu,Stocking,Throughput,Trophic flow,Trophic level},
number = {3-4},
pages = {279--289},
title = {{The food web structure and ecosystem properties of a filter-feeding carps dominated deep reservoir ecosystem}},
volume = {203},
year = {2007}
}
@article{Mayse1978,
abstract = {Numbers of individuals of arthropod species were sampled by direct observation at four sites in each of three different east central Illinois soybean fields at weekly intervals from plant emergence until harvest. Soybean plant development at each field was monitored throughout the season. The number of species detected in a field was greater at the edge (site A) than in the middle (site D). Site A vs. site D differences in numbers of herbivore species were greater than for predator/parasitoid species. Habitat space development (i.e. plant growth) was correlated with the pattern of soybean field colonization by arthropods: a relatively constant number of species per habitat space existed throughout most of the season in all three fields. The mean number of species per habitat space was higher at site A than at site D for both herbivores and predators/parasitoids. A windbreak at the edge of a field concentrated certain aerially dispersed herbivores at the leeward edge early in the season. Arthropod food webs during most of the season were very complex compared to early season trophic relationships. Suggestions for further study, including investigation of the effects of planting time, row-width, and interplanting on seasonal development of the soybean arthropod community are discussed. {\textcopyright} 1978.},
author = {Mayse, Mark A. and Price, Peter W.},
doi = {10.1016/0304-3746(78)90004-5},
issn = {03043746},
journal = {Agro-Ecosystems},
number = {3},
pages = {387--405},
title = {{Seasonal development of soybean arthropod communities in east central Illinois}},
volume = {4},
year = {1978}
}
@article{McKinnerney1978,
abstract = {Carrion community composition was examined in 40 rabbit carcasses at three northern Chihuahuan Desert study areas in Texas and New Mexico from May to August, 1976. Four seral stages of decomposition are described for the carcasses: Fresh, Active, Advanced Decay, and Dried. Of 80 arthropod species collected, 63 are identified as participants in the carrion community. Six vertebrate forms were participants. Probable feeding roles of arthropod and vertebrate taxa are presented. Diversity and concentration of dominance were calculated for seral stages, and relationships to resource diversity and food chain complexity are discussed. Removal efficiency of vertebrates and colonization efficiency of arthropods are correlated to describe a probable abbreviation of arthropod carrion communities in small carcasses; possible implication for arthropod adaptations are presented. No direct correlation appeared between carrion community composition and area meteorological conditions.},
author = {McKinnerney, M},
isbn = {00384909},
journal = {The Southwestern Naturalist},
number = {4},
pages = {563--576},
title = {{Carrion communities in the Northern Chihuahan Desert}},
url = {http://www.jstor.org/stable/3671178},
volume = {23},
year = {1978}
}
@article{Jiron1981,
abstract = {Observations were made on the decomposition of a dead dog during the dry season of 1977 in the Central Valley of Costa Rica, Central America. The terrain is classified as premontane humid forest and the observations were made in a secondary forest. The general pattern of decomposition was basically the same as has been described by authors in other latitudes, but different in the ecological complexity and the insect fauna involved. The classification used for human cadavers by forensic pathologists in Costa Rica and other countries in the American tropics was utilized in this study. It includes the following stages: discoloration, emphysematic (bloated), liquefaction and skeletal remains. The succession of different species appeared to depend on their specific feeding preferences, interspecific competition, and the microclimate provided by the substratum. Marked changes in the activity of populations during crepuscular periods coincided with an increase in relative humidity and a decline in temperature in the macroenvironment of the surrounding forest. Included among the principal insect consumers of the remains were the calliphorid dipterans Phaenicia eximia Wiedemann and Hemilucilia segmentaria Fabricius, the piophilid dipteran Prochyliza azteca McAlpine and the coleopteran Dermestes carnivorus Fabricius. The most important predators were the histerids Euspilotus aenicollis Marshall, Hister punctiger Paykal and Geomysaprinus (Priscosaprinus) belioculus Marshall. Some of these species have also been associated with a similar type of substratum in the tropical rain forest and tropical dry forest in Costa Rica.},
author = {Jir{\'{o}}n, Luis Fernando and Cart{\'{i}}n, V{\'{i}}ctor M.},
isbn = {00287199},
issn = {0028-7199},
journal = {Journal of the New York Entomological Society},
keywords = {http://www.archive.org/details/journalofnewyork121},
number = {3},
pages = {158--165},
title = {{Insect succession in the decomposition of a mammal in Costa Rica}},
url = {http://www.jstor.org/stable/25009256},
volume = {89},
year = {1981}
}
@article{Lancraft1991,
abstract = {Fifty-seven species of oceanic micronekton and macrozooplankton were collected under pack ice during the winter in the vicinity of the Weddell-Scotia Confluence with a modified opening-closing Tucker trawl. The majority of the 57 species did not vertically migrate and lived deeper during the winter than during the spring or fall. However, despite the short day length, several of the most common mesopelagic fish and crustaceans did migrate. Fish moved into shallower depths at night but apparently most did not continue into the near-freezing upper mixed layer, leaving that zone to the migratory crustaceans. In the upper 1000 m, the dominant species were, in order of decreasing biomass, Euphausia superba, the cnidarian Atolla wyvillei, the ctenophore Beroe sp., and the mesopelagic fish Electrona antarctica, Bathylagus antarcticus and Gymnoscopelus braueri. Thysanoessa macrura and Salpa thompsoni were biomass subdominants. The majority of the dominant species showed little seasonal differences in biomass. However, the biomass of gelatinous species varied considerably with A. wyvillei and Beroe sp. being most abundant and S. thompsoni least abundant during the winter. Incidence of food in the stomachs in several important species was low, suggesting a low impact on their zooplankton prey. Specimens of S. thompsoni had high quantities of food in their guts but this species was uncommon so its net impact would also have been low. Euphausia superba and the three common mesopelagic fish had significantly lower stomach fullness ratings during the winter than during the fall, suggesting an overall decrease in feeding activity of dominant species during the winter.},
author = {Lancraft, Thomas M. and Hopkins, Thomas L. and Torres, Joseph J. and Donnelly, Joseph},
doi = {10.1007/BF00240204},
isbn = {0722-4060},
issn = {07224060},
journal = {Polar Biology},
number = {3},
pages = {157--167},
pmid = {199192062615},
title = {{Oceanic micronektonic/macrozooplanktonic community structure and feeding in ice covered Antarctic waters during the winter (AMERIEZ 1988)}},
volume = {11},
year = {1991}
}
@article{Swan1961,
abstract = {swan paper},
author = {Swan, L.W.},
doi = {10.1038/scientificamerican1061-68},
issn = {0036-8733},
journal = {Sci. Am.},
number = {4},
pages = {68--78},
title = {{The ecology of the high Himalayas}},
volume = {205},
year = {1961}
}
@article{Tilly1968,
author = {Stegmuller, Wolfgang and Wohlhueter, William},
journal = {Ecological Monographs},
keywords = {WissTheorie},
number = {2},
pages = {1--22},
title = {{The structure and dynamics of theories}},
url = {http://www.worldcat.org/search?qt=worldcat{\_}org{\_}all{\&}q=structure+dynamics+theories{\%}5Cnpapers2://publication/uuid/F520691A-C996-4E72-A040-D9C8F4545469},
volume = {38},
year = {1976}
}
@article{Kuusela1980,
author = {Kuusela, K},
journal = {Acta-Universitatis-Ouluensis-Series-Biologica},
keywords = {Chironomidae},
number = {6},
pages = {1--130},
title = {{Early summer ecology and community structure of the macrozoobenthos on stones in the Javajankoski rapids on the River Lestijoki, Finland}},
volume = {1980}
}
@article{Lewis2002,
abstract = {1 Quantitative host–parasitoid food webs are descriptions of community structure that include data on the abundance of hosts and parasitoids, and the frequency of links between them, all expressed in the same units. 2 Quantitative host–parasitoid food webs were constructed describing the community of leaf-mining insects (Diptera, Coleoptera and Lepidoptera) and their parasitoids (Hymenoptera) in an 8500-m 2 area of moist tropical forest in Belize, Central America, over a 1-year period. 3 The summary food web, containing data for the whole year, is we believe the most diverse quantitative host–parasitoid web yet described. It contains 93 species of leaf-miner, 84 species of parasitoid and 196 links between hosts and parasitoids. 4 Most parasitoids were generalists, with individual parasitoid species recorded as parasitizing up to 21 host species. In contrast, most leaf-miners were specialists, with 114 links documented between leaf-miners and their host plants. 5 A robustness analysis was used to reveal the effects of different sampling intensities on food web statistics. The results suggest that the sampling had revealed most of the species of host and parasitoid in the community, but further interactions among species would continue to be detected with additional sampling. Measures of the ratio of hosts to parasitoids and of realized connectance were relatively insensitive to sampling intensity. 6 Three seasonal webs were constructed, revealing temporal changes in the structure of the community. There was greater turnover in host species composition than parasitoid species composition among seasons, but most web statistics remained relatively constant across seasons. 7 Both the summary web and the seasonal webs show low levels of compartmentalization, suggesting that the host–parasitoid community is not divided into relatively discrete subwebs with largely independent dynamics. 8 The extent of potential indirect interactions between pairs of hosts was assessed by constructing quantitative parasitoid overlap graphs. These suggest that abundant species are likely to have greater indirect effects on less abundant species than vice versa , and that species in the same taxonomic order are more likely to interact indirectly. The graphs do not support the hypothesis that species sharing the same host plant are more likely to interact via shared parasitoids.},
author = {Lewis, Owen T. and Memmott, Jane and Lasalle, John and Lyal, Chris H.C. and Whitefoord, Caroline and Godfray, H. Charles J.},
doi = {10.1046/j.1365-2656.2002.00651.x},
isbn = {0021-8790},
issn = {00218790},
journal = {Journal of Animal Ecology},
keywords = {Apparent competition,Biodiversity,Food web,Leaf miner,Rain forest},
number = {5},
pages = {855--873},
pmid = {178187900014},
title = {{Structure of a diverse tropical forest insect-parasitoid community}},
volume = {71},
year = {2002}
}
@article{Kitching1987,
author = {Variation, Temporal and Webs, Food and Author, Water-filled Treeholes and Source, Kitching and Publishing, Blackwell and Society, Nordic and Stable, Oikos},
journal = {Oikos},
number = {3},
pages = {280--288},
title = {{Spatial and temporal variation in food webs in water-filled treeholes}},
volume = {48},
year = {2009}
}
@article{Snow1958,
abstract = {The rhyodacite of Ruiz Peak Volcano (New Mexico, USA) is an exceptional rock because it contains both long period and short period polytypes of mica. Our petrographic study shows that this rhyodacite is characterized by numerous disequilibrium textures of phenocrysts (mica, amphibole, clinopyroxene, olivine and plagioclase) contained within both dark-grey and reddish coloured groundmass. The presence of two groundmasses, as well as of disequilibrium textures (reaction rims, resorption, dendritic, skeletal morphologies, etc.) suggests a complex magmatic history. These two types of groundmass are not due to a mixing of magmas but result from a degassing process during the magma ascent in the conduit. The disequilibrium textures are interpreted to be the result of small, short-lived convection cells in the magmatic chamber, which may allow crystal-crystal, crystal-spiral and spiral-spiral interactions to occur, leading to the formation of long period polytypes of mica. For the first time, the relationships between the crystallographic features of mica and the host-rock formation are underlined in this study. It follows that long period polytypes of mica can be considered markers of the complex history of magmas.},
author = {Pignatelli, I. and Faure, F. and Mosser-Ruck, R.},
doi = {10.1016/j.lithos.2016.10.024},
isbn = {0012-9658},
issn = {18726143},
journal = {Lithos},
keywords = {Convective cells,Disequilibrium textures,Long period polytypes,Mica,Self-mixing magma},
number = {1},
pages = {332--347},
title = {{Self-mixing magma in the Ruiz Peak rhyodacite (New Mexico, USA): A mechanism explaining the formation of long period polytypes of mica}},
volume = {266-267},
year = {2016}
}
@incollection{Knox1970,
address = {New York},
author = {Knox, G. a.},
booktitle = {Antarctic ecology},
editor = {Holdgate, M. W.},
pages = {69--96},
publisher = {Academic Press},
title = {{Antactic marine ecosystems}},
year = {1970}
}
@incollection{Kemp1977,
address = {Boulder},
author = {Kemp, W M and Smith, W H B and McKellar, H N and Lehman, M E and Homer, M and Young, D L and Odum, H T},
booktitle = {Ecosystem Modeling in Theory and Practice: An introduction with case histories},
editor = {Hall, Charles A. S. and {Day, Jr.}, John W.},
pages = {684},
publisher = {University Press of Colorado},
title = {{Energy cost-benefit analysis applied to power plants near Crystal River, Florida}},
year = {1977}
}
@incollection{Kitching1983,
address = {Medford, NJ},
author = {Kitching, R L and Frank, J H and Lounibos, L P},
booktitle = {Phytotelmata: terrestrial plants as hosts for aquatic insect communities.},
editor = {Frank, J. H. and Lounibos, L. P.},
pages = {205--222},
publisher = {Plexus Publishsing},
title = {{Community structure in water-filled treeholes in Europe and Australia - comparisons and speculations}},
year = {1983}
}
@article{Hopkins1993,
abstract = {The structure of the food web was investigated in open waters adjacent to the marginal ice zone in the southern Scotia Sea in spring 1983. Diets were defined for dominant zooplankton, micronekton, and flying seabird species and then aggregated by cluster analysis into feeding groups. Most zooplankton were omnivorous, feeding on phytoplankton, protozoans, and in some cases, small metazoans (copepods). Only two speeies were found to be exclusively herbivorous: Calanoides acutus and Rhincalanus gigas. Micronekton were earnivores with eo- pepods being the dominant prey in all their diets. The midwater fish Electrona antarctica was the dominant food item in seven of the nine seabird speeies examined. Cepha- lopods, midwater decapod shrimps and carrion were also important in the diets of a few seabird species. Com- parison (eluster analysis) of diets in spring with other seasons (winter, fall) indicated that over half the speeies examined (18 of 31) had similar diets in all seasons tested. The significant intraspecifie shifts in diet that did occur were attributable to regional, seasonal, and interannual effects. A scheme is presented that describes the major energetic pathways through the open water ecosystem from phytoplankton to apex predators. At the base are phytoplankton and protozoans which are the principal food resource for the biomass copepods and krill. Krill and the biomass copepods are the principal forage of the midwater fish Electrona antarctica which, in turn, is the central diet component of flying seabirds as well as im- portant food for the Antarctie fur seal and cephalopods. Krill area major diet element for the fur seal and eephalo- pods, and the principal food of the minke whale.},
author = {Hopkins, Thomas L. and Ainley, David G. and Torres, Jos{\'{e}} J. and Lancraft, Thomas M.},
doi = {10.1007/BF01681980},
isbn = {0722-4060},
issn = {07224060},
journal = {Polar Biology},
number = {6},
pages = {389--397},
title = {{Trophic structure in open waters of the marginal ice zone in the Scotia-Weddell confluence region during spring (1983)}},
volume = {13},
year = {1993}
}
@article{Huang2008,
abstract = {Because few detailed invertebrate food webs on riverine ecosystems have been reported in Asia, we undertook a dietary study on Heizhuchong Stream, a second order stream in Hubei Province, China. Dietary information was obtained from foregut content analysis of the dominant benthic macroinvertebrates from June 2003 to June 2004. Allochthonous inputs were the most important food sources for the macroinvertebrate community; amorphous detritus comprised 38.3-98.8{\%} of their diets.},
author = {Huang, Ying and Yan, Yunjun and Li, Xiaoyu},
doi = {10.1080/02705060.2008.9664219},
issn = {21566941},
journal = {Journal of Freshwater Ecology},
number = {3},
pages = {421--427},
title = {{Food web structure of benthic macrolnvertebrates in a second order stream of the hanjiang river basin in middle China}},
volume = {23},
year = {2008}
}
@article{Teal1962,
abstract = {JSTOR is a not-for-profit service that helps scholars, researchers, and students discover, use, and build upon a wide range of content in a trusted digital archive. We use information technology and tools to increase productivity and facilitate new forms of scholarship. For more information about JSTOR, please contact support@jstor.org. Wiley is collaborating with JSTOR to digitize, preserve and extend access to Ecology This content downloaded from 138.23.141.99 on Thu, 09 Feb 2017 16:36:52 UTC All use subject to http://about.jstor.org/terms 614 JOHN M. TEAL Ecology, Vol. 43, No. 4 aln(l their (raillage I)asilis; hydrophiysical approach to q(uanititative miiorphology. Bull. Geol. Soc. Amer. 56: 2175-37(1. Hubbs, C., R. A. Kuehne and J. C. Ball. 1953. The Oishies of the upper Guadalupe River, Texas. Tex. Jouir. Sci. 5: 216-244. Klugh, A. B. 1923. A comimolin systemi of classification ini plalit anidI animiial ecology. Ecology 4: 366-377. Kuehne, R. A. and R. M. Bailey. 1901. Streanm cap-ture ind the (listributioln of the)ercid fishi, l]'thco-Sloiil(I .Oi(Jitt(i, ithi geologic ani(l taxoniomic conisid-eratioln.s. (Col)eia 1961: 1-8. Margalef, Ramon. 1960. Ideas for a syinthietic aipproach to the ecology of runinlilig waters.},
author = {Teal, John M.},
doi = {10.2307/1933451},
isbn = {0012-9658},
issn = {00129658},
journal = {Ecology},
keywords = {energy flow,salt marsh},
number = {4},
pages = {614--624},
pmid = {772},
title = {{Energy Flow in the Salt Marsh Ecosystem of Georgia}},
url = {http://doi.wiley.com/10.2307/1933451},
volume = {43},
year = {1962}
}
@article{Kaiser-Bunbury2011,
abstract = {1. Invasive alien plant species pose a severe threat to native plant communities world-wide, especially on islands. While many studies focus on the direct impact of alien plants on native systems, indirect effects of plant invaders on co-flowering natives, for example through competition for pollination services, are less well studied and the results are variable.  2. We used six temporally and taxonomically highly resolved plant–pollinator networks from the island of Mah{\'{e}}, Seychelles, to investigate the indirect impact of invasive alien plant species on remnant native plant communities mediated by shared pollinators. We employed fully quantitative network parameters and information on plant reproductive success, and pollinator diversity and behaviour, to detect changes in plant–pollinator networks along an invasion gradient.  3. The number of visits to and fruit set of native plants did not change with invasion intensity. Weighted plant linkage and interaction evenness, however, was lower at invaded sites than at less invaded sites. These patterns were primarily driven by shifts in interactions of the most common pollinator, the introduced honey bee Apis mellifera, while weak interactions and strong native interactions remained unchanged.  4.Synthesis. The implications of these findings are twofold: first, quantitative network parameters are important tools for detecting underlying biological patterns. Secondly, alien plants and pollinators may play a greater role in shaping network structure at high than low levels of invasion. We could not show, however, whether the presence of invasive plants result in a simplification of plant–pollinator networks that is detrimental to native plants and pollinators alike.},
author = {Kaiser-Bunbury, Christopher N. and Valentin, Terence and Mougal, James and Matatiken, Denis and Ghazoul, Jaboury},
doi = {10.1111/j.1365-2745.2010.01732.x},
isbn = {1365-2745},
issn = {00220477},
journal = {Journal of Ecology},
keywords = {Indian ocean,Indirect interactions,Inselbergs,Interaction connectance,Invasion ecology,Invasive alien species,Plant communities,Pollination webs,Seychelles Islands},
number = {1},
pages = {202--213},
title = {{The tolerance of island plant-pollinator networks to alien plants}},
volume = {99},
year = {2011}
}
@article{Jones1949,
abstract = {1. This paper deals with the Sawdde and its tributary, the Clydach. These are swift, stony trout streams flowing on the northern side of a range of hills, and have cool, unpolluted, alkaline, moderately calcareous water, are very constant in flow and have a very stable bed. 2. The fauna of the Clydach has been described in a previous study. The fauna of the Sawdde, examined in July-September 1948, included about 75 species; dominant animals are Atractides brevirostris, Perla carlukiana, Leuctra fusciventris, Ephemerella notata, Baetis rhodani, Glossosoma boltoni and Rhyacophila obliterata. 3. The flora of the Sawdde and the Clydach is described. Fontinalis antipyretica is widely distributed and abundant at some points. Ulothrix is the dominant green alga. There is a rich diatom flora; characteristic species are Diatoma vulgare, Ceratoneis arcus, Achnanthes spp., Cocconeis placentula and Navicula viridula. 4. The food of the dominant insect species has been studied and about 600 specimens were examined for gut contents. Perla carlukiana, Dinocras cephalotes, Rhyacophila obliterata and Plectrocnemia conspersa are the chief predatory species; Baetis, Ephemerella, Simulium and chironomid larvae are their most frequent prey. The common mayflies are entirely herbivorous; Baetis feeds upon Ulothrix and other green algae, detritus and diatoms; Ecdyonurus mainly upon detritus; Siphlonurus upon the leaves of higher plants and detritus; Ephemerella upon Ulothrix and Fontinalis. Ecdyonurus venosus and certain other common animals in the stream system are apparently able to escape being eaten by the predatory Plecoptera and Trichoptera. Of the common, larger insects Philopotamus montanus is the only species mainly dependent on diatoms.},
author = {Jones, J R Erichsen},
doi = {10.2307/1596},
isbn = {00218790},
issn = {00218790},
journal = {The Journal of Animal Ecology},
number = {2},
pages = {142--159},
title = {{A Further Ecological Study of Calcareous Streams in theBlack Mountain'District of South Wales}},
url = {http://www.jstor.org/stable/1596},
volume = {18},
year = {1949}
}
@article{Robinson1953,
author = {Robinson, I},
journal = {The Journal of Animal Ecology},
number = {1},
pages = {149--153},
title = {{On the Fauna of a Brown Flux on an Elm Tree, Ulmus procera Salisb.}},
volume = {22},
year = {1953}
}
@article{Kelleway2010,
abstract = {Riverine food webs are often laterally disconnected (i.e. between watercourses) in regulated floodplain wetlands for prolonged periods. We compared the trophic structure of benthic resources and consumers (crustaceans and fish) of the three watercourses in a regulated floodplain wetland (the Gwydir Wetlands, Australia) that shared the same source water but were laterally disconnected. The crustaceans Cherax destructor (yabby), Macrobrachium australiense (freshwater prawn), the exotic fish Cyprinus carpio (European carp) and Carassius auratus (goldfish) showed significantly different d13C values among the watercourses, suggesting spatial differences in primary carbon sources. Trophic positions were estimated by using d15N values of benthic organic matter as the base of the food web in each watercourse. The estimated trophic positions and gut contents showed differences in trophic positions and feeding behaviours of consumers between watercourses, in particular for Melanotaenia fluviatilis (Murray–Darling rainbowfish) and M. australiense. Our findings suggest that the observed spatial variation in trophic structure appears to be largely related to the spatial differences in the extent and type of riparian vegetation (i.e. allochthonous carbon source) across the floodplain that most likely constituted part of the benthic resources},
author = {Kelleway, Jeff and Mazumder, Debashish and Wilson, G. Glenn and Saintilan, Neil and Knowles, Lisa and Iles, Jordan and Kobayashi, Tsuyoshi},
doi = {10.1071/MF09113},
isbn = {1323-1650},
issn = {13231650},
journal = {Marine and Freshwater Research},
keywords = {Allochthonous carbon source,Food webs,Riparian vegetation,River regulation,Spatial segregation,Stable isotopes},
number = {4},
pages = {430--440},
title = {{Trophic structure of benthic resources and consumers varies across a regulated floodplain wetland}},
volume = {61},
year = {2010}
}
@article{Kaiser-Bunbury2009,
abstract = {Pollination webs have recently deepened our understanding of complex ecosystem functions and the susceptibility of biotic networks to anthropogenic disturbances. Extensive mutualistic networks from tropical species-rich communities, however, are extremely scarce. We present fully quantitative pollination webs of two plant-pollinator communities of natural heathland sites, one of which was in the process of being restored, on the oceanic island of Mauritius. The web interaction data cover a full flowering season from September 2003 to March 2004 and include all flowering plant and their pollinator species. Pollination webs at both sites were dominated by a few super-abundant, disproportionately well-connected species, and many rare and specialised species. The webs differed greatly in size, reflecting higher plant and pollinator species richness and abundance at the restored site. About one fifth of plant species at the smaller community received {\textless}3 visits. The main pollinators were insects from diverse taxonomic groups, while the few vertebrate pollinator species were abundant and highly linked. The difference in plant community composition between sites appeared to strongly affect the associated pollinator community and interactions with native plant species. Low visitation rate to introduced plant species suggested little indirect competition for pollinators with native plant species. Overall, our results indicated that the community structure was highly complex in comparison to temperate heathland communities. We discuss the observed differences in plant linkage and pollinator diversity and abundance between the sites with respect to habitat restoration management and its influence on pollination web structure and complexity. For habitat restoration to be successful in the long term, practitioners should aim to maintain structural diversity to support a species-rich and abundant pollinator assemblage which ensures native plant reproduction. {\textcopyright} 2009 R{\"{u}}bel Foundation, ETH Z{\"{u}}rich.},
author = {Kaiser-Bunbury, Christopher N. and Memmott, Jane and M{\"{u}}ller, Christine B.},
doi = {10.1016/j.ppees.2009.04.001},
isbn = {1433-8319},
issn = {14338319},
journal = {Perspectives in Plant Ecology, Evolution and Systematics},
keywords = {Alien invasive species,Complex mutualistic network,Oceanic island,Plant-animal interaction,Restoration},
number = {4},
pages = {241--254},
title = {{Community structure of pollination webs of Mauritian heathland habitats}},
volume = {11},
year = {2009}
}
@article{Allan1982,
author = {Allan, J David},
journal = {Ecology1},
keywords = {aquatic insects,benthos,drift,field experiment,predator,prey,salvelinus fontinalis},
number = {5},
pages = {1444--1455},
title = {{The Effects of Reduction in Trout Density on the Invertebrate Community of a Mountain Stream Author ( s ): J . David Allan Reviewed work ( s ): Published by : Ecological Society of America Stable URL : http://www.jstor.org/stable/1938871 . THE EFFECTS OF}},
volume = {63},
year = {1982}
}
@article{Cummins1966,
abstract = {Many challenging problems could be better solved by exploiting
crowdsourcing platforms than traditional machine-based methods. However,
data quality in crowdsourcing applications has become a crucial aspect
since crowdsourcing workers may have different capabilities. In this
paper, we propose a novel weighted aggregation rule (WAR) to improve the
result accuracy in crowdsourcing systems. According to the agreement of
answers given by the workers, we classify all the tasks into the
high-agreement tasks and low-agreement tasks. For the high-agreement
tasks, we use simple majority voting to select the correct answer while
ensuring the result accuracy. For the low-agreement tasks, we adopt
weighted majority voting strategy, which assigns a weight for each
worker according to his performance on the high-agreement tasks. We
evaluate the effectiveness of our proposed method using three real-world
datasets on AMT. The experimental results show that our method achieves
excellent result accuracy.},
author = {{Cummins, K.W., Coffman, W.P., Roff}, P.A.},
doi = {10.1109/DASC.2014.54},
isbn = {978-1-4799-5079-9},
journal = {Proceedings of the International Association of Theoretical and Applied Limnology},
pages = {627 -- 638},
title = {{Trophic relations in a small woodland stream.pdf}},
volume = {16},
year = {1996}
}
@article{Jones1950,
abstract = {Calcium carbonate was preserved more than 1 km below the modern calcite compensation depth (CCD) at ODP Site 1179 over several short intervals around 2.5-2.4, 1.65 and 0.9-0.7 Ma. This anomalous preservation resulted from a combination of increased production of planktonic foraminiferal tests at the sea-surface and increased rate of sedimentation to the sea floor. The abundance of dinoflagellate cysts in calcareous sediments records intense plankton blooms, and the preservation of oxidation-susceptible round brown Brigantedinium cysts in foraminifer-rich samples supports the theory of rapid burial. The rise in sea-surface productivity was driven by enhanced flux of continentally derived limiting nutrients, consistent with the pollen evidence of continental aridification, cooling, and an increase in wind strength. The abundant pollen, dominated by steppe herb and montane-boreal conifer taxa, contrasts with lower pollen concentrations dominated by temperate-subtropical deciduous tree and Taxodium-type pollen in non-calcareous sediments. {\textcopyright} 2004 Elsevier B.V. All rights reserved.},
author = {McCarthy, Francine M.G. and Findlay, Duncan J. and Little, Martin L.},
doi = {10.1016/S0031-0182(04)00402-X},
isbn = {0031-0182},
issn = {00310182},
journal = {Palaeogeography, Palaeoclimatology, Palaeoecology},
keywords = {Carbonate preservation,Late Cenozoic palynology,Matuyama Chron,Planktonic foraminifera,Productivity,Terrigenous flux},
number = {1-2},
pages = {1--15},
pmid = {129},
title = {{The micropaleontological character of anomalous calcareous sediments of late Pliocene through early Pleistocene age below the CCD in the northwestern North Pacific Ocean}},
volume = {215},
year = {2004}
}
@article{Harrison1962,
author = {Harrison, Author J L},
journal = {Journal of Animal Ecology1},
number = {1},
pages = {53--63},
title = {{the Distribution of Feeding Habits Among}},
volume = {31},
year = {2009}
}
@article{Valiela1969,
abstract = {Larvae of M. autumnalis reared under optimal conditions in the laboratory showed a 76.2{\%} survivorship, plus or minus 1.9{\%}.  In the field, face fly larvae suffered a larger mortality.  For 3 sets of field measurements only 44.9{\%}, plus or minus 22.4{\%}, 36.9{\%}, plus or minus 6.9{\%}, and 12.0{\%}, plus or minus 6.0{\%} reached the pupal stage.  Most of this field mortality took place very early in the larval development.  Potential mortality factors were then independently evaluated in an attempt to account for the field mortality.  Temperature appeared to act as a lethal factor only in the high temperature range.  A consideration of natural history and physical properties led to the assumption that high temperatures as found on the top surface of droppings could exert their effect only on the egg and first instar stages.  The temperature regimes of each of the field trials led to predicting heat mortalities only in 1 (21.8{\%}) out of 3 trials.  moisture content and pH of dung were quite constant.  An examination of the biological properties of face fly larvae showed that these variables were dubious contributors to the mortality budget.  Competition for food led to a reduction in pupal weight when larval densities were high, regardless of whether the competitors were other face fly larvae or other species.  At lower densities, pupal weights also decreased, but in this case mortality increased, perhaps because a critical minimal density is needed for the proper liquefaction of the substrate to make available enough food for the filter-feeding larvae.  This effect of undercrowding was not large enough to cause significant mortality in the field trials.  Parasitism was unimportant as far as face fly larvae were concerned.  Predators (Philonthus cruentatus) in the field experiments caused 38.4{\%}, plus or minus 7.7{\%} larval mortality if burrowing beetles (Sphaeridium scarabaeoides) were present.  Predatory mortality was reduce if the burrowers were absent, indicating some sort of synergistic interaction.  A rougher second estimate based on prey consumption rates suggest that predators could have reduced face fly populations by 51.4, 47.9 and 45.7 percent in the field trials.  Laboratory experiments showed that the bulk of predation occurred during the egg and first instar stages.  The laboratory predation experiments provided a third corroborative mortality prediction of 42.3{\%} with burrowers present.  Predatory actions by birds, though possibly important, were discounted since they were outside the scope of this study.  When an estimate of physiological death at each of 4 time intervals was removed from field survival, the remainders estimated true field mortality for each trial.  These remainders for the 3 trials can be shown to be similar.  Further, the overall mortalities for the field trials were quite close to the results of both field and laboratory predation experiments.  The addition of all the mortality components furnished a predicted set of mortalities that tended to be somewhat larger than the observed values.  However, the overall fit between predicted and observed survival rates was good.  The independent experimental evaluations of potential mortality factors furnished an adequate manner with which to judge the relative importance of each factor and yielded a good overall mortality prediction.  Interactions and overlaps between mortality factors were not accounted for in this study.  A study of these factors could provide better and more realistic prediction and models.  However, the variability intrinsic to the system and the errors of measurement may limit the detail of the analysis even if proper experimental design is used.  Better evaluation of these sources of error is needed before further field work is attempted.},
author = {Monographs, Ecological and Spring, No},
isbn = {00129615},
journal = {Ecological Monographs},
keywords = {--},
number = {2},
pages = {199--225},
title = {{An Experimental Study of the Mortality Factors of Larval Musca autumnalis DeGeer Ivan Valiela AN EXPERIMENTAL STUDY OF THE MORTALITY FACTORS OF LARVAL M U S C A A U T U M N A L I S DEGEER {\~{}}}},
volume = {39},
year = {1969}
}
@article{Valiela1974,
abstract = {Arthropod species invade fresh dung in an orderly pattern and the number of taxa and the complexity of the food web increase as succession takes place. Short-term, local changes in the environment during early succession seem to have a more pervasive effect on species abundance than seasonal changes. The interactions between burrowing and predatory beetles, and large and small diptera larvae are the core of the food webs in dung succession. Burrowing beetles arrive early and remain throughout succession. Their tunnels riddle the dung and are used by the air-breathing fly larvae to reach the interior of the dropping. Predatory beetles lack adaptations for burrowing but may use burrows already excavated to reach the otherwise unavailable fly larvae. The presence of burrowers, therefore, facilitates predation. The predators involved can feed only on small prey 0.5-6.0 mm long. The larvae of larger flies found early in dung succession are able quickly to outgrow this size range, and escape predation. There is a second group of small fly larvae that then enters this size range and remains exposed to predation for the duration of succession. Competition for dung is not likely to be a major limitation for dung feeder populations since excess dung seems available. Similarly, estimates of prey needed by predators are smaller than the standing crops of prey available. Predation does not appear to be limiting to prey populations. Similarly, predators themselves are unlikely to be prey-limited. Local, short-term changes in dung and in the immediate environment may be too fast and too erratic to permit fuller use of dung as a resource. The frequent occurrence of introduced species in dung composition may be related to the lack of competition and predation pressure in dung-inhabiting arthropods. The initial stages of dung succession, as in most other new environments, are largely determined by factors other than competition and predation.},
author = {Dame, Notre},
doi = {10.2307/2424302},
issn = {00030031},
journal = {The American Midland Naturalist},
keywords = {Diptera,arthropod,assembly,beetles,competition,dung,dung beetle,food limitation,food webs,succession},
number = {2},
pages = {370--385},
title = {{Composition, Food Webs and Population Limitation in Dung Arthropod Communities During Invasion and Succession}},
volume = {92},
year = {1974}
}
@book{Howes1954,
address = {New York},
author = {Howes, P G},
booktitle = {The giant cactus forest and its world. [S.W. Arizona].},
keywords = {Biology,Chordates,Land zones,Reptiles,Vertebrates},
pages = {i--xxx 1--258},
publisher = {Duell, Sloan, and Pearce},
title = {{The giant cactus forest and its world. [S.W. Arizona]}},
year = {1954}
}
@incollection{Hatanaka1977,
address = {Tokyo},
author = {Hatanaka, M a.},
booktitle = {Productivity of biocenoses in coastal regions of Japan},
editor = {Hogetsu, K. and Horanaka, M. and Hatanaka, T. and Kawamura, T.},
pages = {173--221},
publisher = {Japanese Committee for the International Biological Program Synthesis},
title = {{Sendai Bay}},
year = {1977}
}
@techreport{Harris1972,
address = {Fort Collins, Colorado},
author = {Harris, L. Dale and Paur, Leonard Francis},
booktitle = {Technical report (US International Biological Program Grassland Biome)},
institution = {Colorado State University},
pages = {11--21},
title = {{A quantitative food web analysis of a shortgrass community}},
year = {1972}
}
@article{Hopkins1984,
author = {Hopkins, M J G},
journal = {Entomologist's Monthly Magazine},
pages = {187--192},
title = {{The parasite complex associated with stem-boring Apion (Col., Curculionidae) feeding on Rumex species (Polygonaceae)}},
volume = {120},
year = {1984}
}
@incollection{Harris1980,
address = {Cambridge},
author = {Harris, L D and Bowman, G B and Breymeyer, a I and {Van Dyne}, G M},
booktitle = {Grasslands, system analysis and man: International Biological Programme Series, no. 19},
editor = {Breymeyer, A. I. and {Van Dyne}, G. M.},
pages = {591--607},
publisher = {Cambridge University Press},
title = {{Vertebrate predator subsystem}},
year = {1980}
}
@article{Holm1980,
abstract = {*[Composition {\&} patterns of the main biotic {\&} abiotic components or the system are outlined. Notes on biology {\&} ecology of 127 recorded animal species are given, with special attention to the apterous arthropods which dominate the system. The results of a one-year trapping survey are used to illustrate habitat preferences, diet activity cycles, {\&} seasonal occurrences of the majority of the species.]},
author = {Holm, E and Scholtz, C H},
doi = {http://reference.sabinet.co.za/document/AJA10115498_226},
issn = {1011-5498},
journal = {Madoqua},
keywords = {Africa,Camponotus detritus,Coleoptera,Formicidae,Formicinae,Histeridae,Namaqualand,Namibia,ant,climate,coprophages,detritivores,distribution,dunes,ecology,endemism,fossorial activity,habitat,invertebrates,myrmecophile,nests,omnivores,parasitoids,physical environment,plains,predators,savanna,scavengers,scientific,space-time niches,substrate,symbiont,trapping,vegetation,vertebrates},
number = {1},
pages = {3--39},
title = {{Structure and pattern of the Namib Desert dune ecosystem at Gobabeb}},
url = {https://journals.co.za/content/madoqua/12/1/AJA10115498{\_}226},
volume = {12},
year = {1980}
}
@article{Hodkinson2004,
author = {Nitsche, Christoph and Scheithauer, Guntram and Terno, Johannes},
doi = {10.1007/s001860050015},
issn = {14322994},
journal = {Mathematical Methods of Operations Research},
keywords = {Cutting stock problem,Linear relaxation,Modified integer roundup property},
number = {1},
pages = {105--115},
title = {{New cases of the cutting stock problem having MIRUP}},
volume = {48},
year = {1998}
}
@incollection{Hogetsu1979,
address = {Cambridge},
author = {Hogetsu, K},
booktitle = {Marine Production Mechanisms},
editor = {Dunbar, M.},
pages = {71--87},
publisher = {Cambridge University Press},
title = {{Biological productivity of some coastal regions of Japan}},
year = {1979}
}
@article{Gontikaki2011,
abstract = {The benthic food web of the deep Faroe-Shetland Channel (FSC) was modelled by using the linear inverse modelling methodology. The reconstruction of carbon pathways by inverse analysis was based on benthic oxygen uptake rates, biomass data and transfer of labile carbon through the food web as revealed by a pulse-chase experiment. Carbon deposition was estimated at 2.2mmolCm-2d-1. Approximately 69{\%} of the deposited carbon was respired by the benthic community with bacteria being responsible for 70{\%} of the total respiration. The major fraction of the labile detritus flux was recycled within the microbial loop leaving merely 2{\%} of the deposited labile phytodetritus available for metazoan consumption. Bacteria assimilated carbon at high efficiency (0.55) but only 24{\%} of bacterial production was grazed by metazoans; the remaining returned to the dissolved organic matter pool due to viral lysis. Refractory detritus was the basal food resource for nematodes covering ∼99{\%} of their carbon requirements. On the contrary, macrofauna seemed to obtain the major part of their metabolic needs from bacteria (49{\%} of macrofaunal consumption). Labile detritus transfer was well-constrained, based on the data from the pulse-chase experiment, but appeared to be of limited importance to the diet of the examined benthic organisms ({\textless}1{\%} and 5{\%} of carbon requirements of nematodes and macrofauna respectively). Predation on nematodes was generally low with the exception of sub-surface deposit-feeding polychaetes that obtained 35{\%} of their energy requirements from nematode ingestion. Carnivorous polychaetes also covered 35{\%} of their carbon demand through predation although the preferred prey, in this case, was other macrofaunal animals rather than nematodes. Bacteria and detritus contributed 53{\%} and 12{\%} to the total carbon ingestion of carnivorous polychaetes suggesting a high degree of omnivory among higher consumers in the FSC benthic food web. Overall, this study provided a unique insight into the functioning of a deep-sea benthic community and demonstrated how conventional data can be exploited further when combined with state-of-the-art modelling approaches. {\textcopyright} 2010 Elsevier Ltd.},
author = {Gontikaki, E. and van Oevelen, D. and Soetaert, K. and Witte, U.},
doi = {10.1016/j.pocean.2010.12.014},
isbn = {0079-6611},
issn = {00796611},
journal = {Progress in Oceanography},
number = {3},
pages = {245--259},
publisher = {Elsevier Ltd},
title = {{Food web flows through a sub-arctic deep-sea benthic community}},
url = {http://dx.doi.org/10.1016/j.pocean.2010.12.014},
volume = {91},
year = {2011}
}
@article{Hampton2011,
abstract = {In large deep oligotrophic lakes, multiple lines of evidence suggest that the shallow nearshore water provides disproportionately important feeding and breeding habitat for the whole-lake food web. We examined the trophic importance of the nearshore environment, human impacts nearshore, and several approaches to disturbance detection in a deep (190 m) oligotrophic lake with relatively modest residential development. In Lake Crescent, on the Olympic Peninsula of Washington (USA), stable isotope analysis demonstrated that apex salmonid predators derived more than 50{\%}of their carbon from nearshore waters, even though this nearshore water accounted for only 2.5{\%} of total lake volume. Unfortunately, it is this land–water interface that is initially degraded as shorelines are developed. We hypothesised that under these conditions of relatively modest disturbance, the effects of residential development would be strongly localised near to shore. Indeed,wefound striking differences between developed and undeveloped sites in periphyton and associated organic matter, though there were no offshore signals of human impact in water nutrient analysis or paleolimnological investigations. Together, these results suggest that nearshore biological monitoring should be integrated in lake management plans to provide ‘early warning' of potential food-web repercussions before pollution problems are evident in open water and comparatively intractable.},
author = {Hampton, Stephanie E. and Fradkin, Steven C. and Leavitt, Peter R. and Rosenberger, Elizabeth E.},
doi = {10.1071/MF10229},
isbn = {1323-1650},
issn = {13231650},
journal = {Marine and Freshwater Research},
keywords = {Oncorhynchus clarkii,Oncorhynchus mykiss,habitat coupling,littoral zone,recreational fisheries,septic systems},
number = {4},
pages = {350--358},
title = {{Disproportionate importance of nearshore habitat for the food web of a deep oligotrophic lake}},
volume = {62},
year = {2011}
}
@article{Savely1939,
abstract = {1. During the period from September, 1936 to February, 1938 about 280 logs, 155 pines and 125 oaks, were examined in the Duke Forest, Durham, North Carolina. 2. There is a succession of animals in pine and oak logs from the time they are cut until they disintegrate. 3. Insects feeding in the phloem of pine and oak logs during the first year after they are cut prepare the way for the entrance of fungi and a subcortical fauna including mycetophagous and predacious species. A subcortical fauna in oaks may enter spaces formed between the bark and wood by the shrinkage of sapwood. The environmental changes in logs leading to a succession of wood-feeding animals appear to be largely those caused by fungi. 4. Subcortical temperatures under the loose bark of pine logs exposed to sunlight in summer rose as much as 11°C. above air temperature when the latter was 33°C. 5. Temperatures under the bark and at various depths in sapwood of logs fluctuate with air temperatures in all seasons of the year with a lag depending in part on the depth in the log at which temperatures are taken. Animals under bark probably receive little protection from low temperatures in winter in the Duke Forest. It is suggested that for many animals logs are more important as hibernacula because of their moisture content rather than the protection they afford from low temperatures, though logs do moderate extreme and rapid changes. 6. Temperatures in rotting logs were not perceptibly affected by heat that might have been produced by fungi or other organisms, or by possible auto- oxidative processes within the wood. 7. Samples of air taken from under bark and from rotting wood of logs contained as much as 5.53 per cent (volume) carbon dioxide, and oxygen concentration as low as 15.53 per cent (volume). 8. The highest temperature of two hours duration tolerable to larvae of Chrysobothris sp. (probably femorata Oliv.) taken from oak logs, was 52°C. regardless of the relative humidity in which they were exposed. Toleration of Chrysobothris sp. to such high temperature may be correlated with the fact that it lives on the top side of oak logs exposed to the sun. 9. The highest temperature of two hours duration tolerable to Romaleum atomarium, Monochammus titillator, and Acanthocinus nodosus was 49°, 50°, and 46° respectively, when exposed in air with a relative humidity of approximately 10-15 per cent. With relative humidity of 100 per cent they could only tolerate temperatures of 44°, 44°, and 43°C., respectively. The explanation offered for this phenomenon is that the cooling effect of evaporation of water from a larva in dry air undoubtedly lowers its body temperature below that of its surroundings. 10. The highest temperature of two hours duration tolerable to larvae of Dendroides bicolor was 41°C. in humidities of both 10-15 and 90-95 per cent. It is considered that their small size prevents them from being able to lose enough water by evaporation to maintain a lower body temperature than their surroundings. 11. Loss of water from larvae tested for toleration to high temperatures was proportional to the temperatures at which they were exposed, when relative humidity was 10-15 per cent. 12. Of the larvae tested, those of Chrysobothris sp. were most resistant to desiccation, and those of Dendroides bicolor least resistant. The percentage of body weight lost by evaporation from larvae of Romaleum atomarium, Monochammus titillator, and Acanthocinus nodosus was intermediate between that of Chrysobothris sp. and Dendroides bicolor. The difference in the larvae investigated with respect to their resistance to desiccation may be correlated with the conditions of moisture in the logs in which they live. 13. The average dry weight of the phloem covering one square centimeter of area on shortleaf pines was found to be 0.0555 +/- 0.0021 grams. 14. The amount of starch in three samples of living phloem collected from shortleaf pine in November was 5.02 per cent of the dry weight of the sample. 15. The larvae of Callidium antennatum and Chrysobothris sp. feed on the phloem of pine poles containing as little as 26 per cent water. They eat little, if any of the sapwood which they dig out in making their burrows. In poles which contained larvae the sapwood contained no starch, but the phloem contained enough starch to be turned blue-black by iodine. 16. Larvae of Callidium antennatum and Chrysobothris sp. remove starch from the phloem they eat. 17. The areas of the burrows that had been formed by ten larvae of Callidium antennatum and four larvae of Chrysobothris sp. up until they had formed pupal chambers, were determined. The areas of the burrows measured were proportional to the dry weights of the larvae that had made them. From the average dry weight of the phloem covering one square centimeter of surface on shortleaf pine, the amount of phloem that had probably been consumed by larvae of both species was calculated. The dry weights of phloem probably consumed were proportional to the dry weights of the larvae. The average amount of phloem (dry weight) that had been eaten by a larva (dry weight); for Chrysobothris sp. it was 77.1 +/- 3.1 grams. 18. The gut contents of five larvae feeding on rotting wood and five larvae feeding on phloem were tested for the presence of a cellulase with negative results. Cellulase was found in the gut contents of the larvae of Derobrachus brunneus, which feeds in solid and soft rotting wood. 19. Larvae of beetles feeding on phloem of logs less than one year old are probably able to derive all their essential food from products stored in the cells of the phloem without digesting cellulose. 20. The gut contents of 22 species of insects of doubtful food habits were examined. Some were found to be predacious, others fed on fungus hyphae and spores, rotting wood, or both.},
author = {Savgely, H. E. Jr.},
doi = {10.2307/1943233},
isbn = {0012-9615},
issn = {00129615},
journal = {Ecological Monographs},
number = {3},
pages = {321--385},
title = {{Ecological relations of certain animals in dead pine and oak logs}},
volume = {9},
year = {1939}
}
@article{Wilbur1972,
abstract = {biological tractability of current concepts of the organization of natural communities. Experimental communities with a known composition of mature eggs were introduced into screen enclosures in a pond to assay the importance of competition and predation to the ecology of amphibian larvae in temporary ponds. The competitive ability of each population was measured by its survivorship, mean length of its larval period, and mean weight at metamorphosis. Three simultaneous experiments (requiring 70 enclosures and 137 populations) were replicated in a randomized complete-block design for variance analysis. The assumptions of the classical Lotka-Volterra model of competition were tested by raising Ambystoma laterale, Ambystoma tremblayi, and Ambystoma maculatum in all combinations of three initial densities (0, 32, and 64). All three measures of competitive ability were affected by competition with other species. Higher-order interactions decreased the variance of the outcomes of the experiments as species were added to the communities. The statistical effects of these higher-order interactions between the densities of competing species often exceeded the simple effects of competition. The increase in community stability with the addition of species to the community is not predicted by the classical models of community ecology. The second experiment tested the effects of adjacent trophic levels on the structure of the three-species community. Eggs of Ambystoma tigrinum, a predator, and Rana sylvatica, an alternate prey of Ambystoma tigrinum, were added singly and together into systems with 16 eggs of species in the Maculatum species-group. Ambystoma tigrinum was a predator if it acquired an initial size advantage by preying on Rana sylvatica tadpoles; otherwise it was principally a competitor. Rana sylvatica adversely affected the Maculatum group by competing with invertebrate prey for periphyton and photoplankton. The three species in the Maculatum group had nearly the same response to the addition of both A. tigrinum and R. sylvatica. Ambystoma texanum, which occurs sporadically in southern Michigan at the northern limit of its range but not on the study area, was introduced as a test for community saturation. Ambystoma texanum was successfully raised alone. When mixed with the Maculatum group, Ambystoma texanum had a low survivorship, a small body size, and a long larval period. The native species were affected equally by the introduction of Ambystoma texanum, demonstrating the complexity of the food web and the ecological pliability of salamander larvae. The uncertainty of the temporary pond environment precludes extreme ecological specialization among these species of salamanders. Coexistence is a consequence of the relative advantages of the species in different years and the long adult life spans. The complexity of the food web and "predator switching" are probably important elements of the densitydependent interactions that contribute to the stability of communities within seasons.},
author = {Wilbur, Henry M.},
doi = {10.2307/1935707},
isbn = {00129658},
issn = {00129658},
journal = {Ecology},
number = {1},
pages = {3--21},
pmid = {17937997},
title = {{Competition, Predation, and the Structure of the Ambystoma-Rana Sylvatica Community}},
url = {http://doi.wiley.com/10.2307/1935707},
volume = {53},
year = {1972}
}
@article{Gaedke1994,
abstract = {The trophic transfer efficiencies in the planktonic food web of large, deep, and mesoeutrophic Lake Constance were derived independently from biomass size distributions and from mass-balanced carbon flow diagrams based on comprehensive data for biomass, production, and food web structure. The main emphasis was on the transfer of primary production to herbivores since this process dominates the flow of matter within the food web. Biomass size distributions offer an ecosystem approach which relies only on measurements of biomass and a few general assumptions, whereas network analysis is predominantly based on production estimates and requires more detailed knowledge of the ecosystem. Despite these differences, both approaches give consistent results for both the absolute values of the transfer efficiencies and seasonal trends. Estimates of the seasonally averaged transfer efficiency (dominated by the utilization of primary production by herbivores) range from 0.20 to 0.27. They are considerably lower in late winter and spring (0.05 to 0.21) than in summer and autumn (0.25 to 0.38, extreme values: 0.20 and 0.42). {\textcopyright} 1994.},
author = {Gaedke, Ursula and Straile, Dietmar},
doi = {10.1016/0304-3800(94)90038-8},
isbn = {0304-3800},
issn = {03043800},
journal = {Ecological Modelling},
keywords = {Biomass,Food webs,Network analysis,Plankton,Seasonality},
number = {C},
pages = {435--445},
title = {{Seasonal changes of trophic transfer efficiencies in a plankton food web derived from biomass size distributions and network analysis}},
volume = {75-76},
year = {1994}
}
@article{MacGinitie1935,
abstract = {Includes the following:  Introduction to Elkhorn Slough Materials Used Location and Physical Features Tides, Bottom Soil Plant Life Temperature,  Oxygen, Salinity, Light, Pollution Food supply Association, Zoning and Numbers, Abundance of Organisms, New species and Extension of Ranges Parasitism and commensalism Tropisms in General Nocturnal Activities Concerning Migrations from Ocean to Land Geological Indications Phyletic Catalog of Animals Bibliography},
author = {MacGinitie, George Eber},
isbn = {0003-0031},
journal = {American Midland Naturalist},
keywords = {ECOLOGY / POLYCHAETA},
number = {5},
pages = {629--765},
title = {{Ecological aspects of a California marine estuary. American Midland Naturalist, 16(5): 629-765, 21 figures.}},
volume = {16},
year = {1935}
}
@article{Filgueira2011,
abstract = {Based on the stable isotope composition in 15N and 13C of different potential sources of organic matter and consumers of an intertidal Zostera marina meadow located in San Sim{\'{o}}n Bay (R{\'{i}}a de Vigo, NW of Spain), a simplified food web of this community was reconstructed. For this purpose, some alternatives in different steps of the most used methodology of stable isotope dietary analysis were developed that cope with some of the limitations associated to the interpretation of isotopic signals for food web analysis, those of uncertainty on the fractionation value, mathematical model to use for the diet resolution and shortage of the isotope number for discriminating many food sources. The application of this protocol to the studied community reported similar results to those from other studies based on similar trophic webs, emphasizing the importance of local primary producers, especially microphytobenthos, which could be available for several primary consumers through resuspension forced by tidal hydrodynamic. The good agreement with previous results suggests that the proposed protocol is a feasible alternative to elucidate the most plausible trophic relationships in complex trophic webs using stable isotopes analysis. {\textcopyright} 2010 Elsevier Ltd.},
author = {Filgueira, R. and Castro, B. G.},
doi = {10.1016/j.csr.2010.10.010},
isbn = {0278-4343},
issn = {02784343},
journal = {Continental Shelf Research},
keywords = {Diet analysis,Food web,R{\'{i}}a de Vigo,Seagrass,Stable isotope},
number = {5},
pages = {476--487},
publisher = {Elsevier},
title = {{Study of the trophic web of San Sim{\'{o}}n Bay (R{\'{i}}a de Vigo) by using stable isotopes}},
url = {http://dx.doi.org/10.1016/j.csr.2010.10.010},
volume = {31},
year = {2011}
}
@article{Douglass2011,
abstract = {Changes in seagrass food-web structure can shift the competitive balance between seagrass and algae, and may alter the flow of energy from lower trophic levels to commercially important fish and crustaceans. Yet, trophic relationships in many seagrass systems remain poorly resolved. We estimated the food web linkages among small predators, invertebrate mesograzers, and primary producers in a Chesapeake Bay eelgrass (Zostera marina) bed by analyzing gut contents and stable C and N isotope ratios. Though trophic levels were relatively distinct, predators varied in the proportion of mesograzers consumed relative to alternative prey, and some mesograzers consumed macrophytes or exhibited intra-guild predation in addition to feeding on periphyton and detritus. These findings corroborate conclusions from lab and mesocosm studies that the ecological impacts of mesograzers vary widely among species, and they emphasize the need for taxonomic resolution and ecological information within seagrass epifaunal communities. {\^{A}}{\textcopyright} 2011 Coastal and Estuarine Research Federation.},
author = {Douglass, James G. and {Emmett Duffy}, J. and Canuel, Elizabeth A.},
doi = {10.1007/s12237-010-9356-4},
isbn = {1559-2723},
issn = {15592723},
journal = {Estuaries and Coasts},
keywords = {Diet,Food web,Mesograzer,Omnivory,Seagrass,Stable isotope},
number = {4},
pages = {701--711},
title = {{Food Web Structure in a Chesapeake Bay Eelgrass Bed as Determined through Gut Contents and13C and15N Isotope Analysis}},
volume = {34},
year = {2011}
}
@article{Force1974,
abstract = {Although conclusive evidence is lacking for its establishment, the thesis that complexity adds stability to communities is probably accepted by the majority of ecologists. I believe this attitude found its origins in the indisputable fact that there are latitudinal and altitudinal changes in community complexity. As one progresses northward or southward from the equator, or higher in altitude in most parts of the world, one cannot help but notice that communities tend to become simpler, that is, there are fewer species per community. At the same time, these communities appear to become less stable. But perhaps this change in stability is in appearance only; they appear to be less stable because of the relatively greater number of individuals comprising each species population in temperate areas. Each population, because of its greater numbers, is therefore conspicuous, and changes in these numbers are noticed. We are particularly aware of such changes because populations in these areas of the world have been comparatively well studied. Many of the most studied populations include species of economic importance where changes in population numbers are vital to agricultural or forestry practices. Equatorial populations, on the other hand, contain smaller numbers of individuals of each species because of the greater number of species present. Number changes are simply not as noticeable because the population itself is not as obvious among the other populations. It may be that when (if ever) we have as much data on equatorial populations as we have on those of temperate climates, we will find fluctuations of equal relative magnitude (but not of equal numbers, of course). If, on the other hand, we really do find a correlation between complexity and stability, the suggestion by May (12) that stability permits complexity may be well worth investigating. Because of its organization and physical setting, the Rhopalomyia community I have studied might be expected to have considerable stability. In fact, however, it does not. Each of the populations in the community fluctuates greatly and irregularly in both percentages and numbers, and these populations apparently become locally extinct occasionally, because they sometimes cannot be found even in extensive collections. After studying several of the more important parasitoid species, it is evident to me that there is little or nothing about their interactions that might induce greater community stability. Each species seems to have evolved into the community with no higher purpose than simply to usurp what it can from some other member, and it does this by concentrating its energies on better competitive mechanisms rather than higher reproductive capacities. There are never empty niches to be filled by organisms having the "correct specifications" because new niches are created out of parts of older, broader niches which were occupied by other, more r-selected organisms. Thus, perhaps we have read too much into community organization. Perhaps the "ifiling of niches" is essentially nothing more than the haphazard result of competitive jostling among species; and that as communities develop, they are not necessarily programmed for such things as greater stability or better energy utilizationthe species merely become more closely packed.},
author = {Force, Don C.},
doi = {10.1126/science.184.4137.624},
isbn = {0036-8075},
issn = {00368075},
journal = {Science},
number = {4137},
pages = {624--632},
pmid = {4820846},
title = {{Ecology of insect host-parasitoid communities}},
volume = {184},
year = {1974}
}
@article{VanEs1977,
author = {van Es, F. B.},
doi = {10.1007/BF02207842},
issn = {00179957},
journal = {Helgol{\"{a}}nder Wissenschaftliche Meeresuntersuchungen},
number = {1-4},
pages = {283--294},
title = {{A preliminary carbon budget for a part of the Ems estuary: The Dollard}},
volume = {30},
year = {1977}
}
@article{Minshall1967,
abstract = {The community trophic structure of Morgan's Creek, Meade County, Kentucky was analyzed through regular measurement of standing crops of the chief potential sources of plant materials available to the animals (suspended particulate, attached particulate, and allochthonous leaf materials) and an examination of their gut contents. The most important food was allochthonous leaf materials, which occurred as suspended material in the water, as a component of materials attached to the streambed, and as whole leaves and fragments. Diatoms were the only other important source of plant materials and constituted the greatest proportion of the attached organic fraction. Mean standing crop measurements of potential foods for five sampling stations ranged from 0.6 to 1.0 kcal/m3 for suspended particulate organic matter; 12 to 19 kcal/m2 for attached particulate organic matter; and 4.7 to 13 kcal/m2 for allochthonous leaf materials. Comparison of standing crop data with previous findings indicates that the values generally are within the known ranges for flowing waters. Analysis of gut contents and determination of the principal pathways of energy flow in the stream indicate that imported organic matter in the form of allochthonous leaf materials provides the main source of energy for the primary consumers and, indirectly, for the entire benthic community of Morgan's Creek. Of the 37 taxa of animals studied, 24 were herbivores, 5 omnivores, and 8 carnivores. In general, detritus made up from 50 to 100{\%} of all the materials ingested by both the herbivores and omnivores. The total number of benthic animals was comprised of 14{\%} herbivores, 83{\%} omnivores, and 3{\%} carnivores. Gammarus minus was the single most important member of the fauna. It contributed 81{\%} of the total number of invertebrates, and well over 90{\%} of its diet consisted of allochthonous leaf detritus.},
author = {Minshall, G W},
doi = {10.2307/1933425},
isbn = {0012-9658},
issn = {00129658},
journal = {Ecology},
number = {1},
pages = {139--149},
pmid = {205},
title = {{Role of Allochthonous Detritus in the Trophic Structure of a Woodland Springbrook Community}},
url = {http://www.jstor.org/stable/1933425{\%}0Ahttp://www.jstor.org/page/info/about/policies/terms.jsp{\%}0Ahttp://www.jstor.org},
volume = {48},
year = {1967}
}
@article{Percival1929,
abstract = {OBJECTIVE: The protein capsule of Yersinia pestis, known as Fraction 1 or F1, is a protective immunogen and is an assumed, but not proven, virulence factor. Our objectives were to determine if inhaled F1-negative and/or F1-positive strains of Y pestis were virulent in the African green monkey and, if so, to differentiate F1-negative from F1-positive monkeys. Because F1-negative strains have been isolated from natural sources and have caused experimental fatal disease, we felt that this information was crucial to the development of future vaccines and diagnostic tests. MATERIALS AND METHODS: Adult African green monkeys were exposed by aerosol to F1-positive (CO92, n=15) or F1-negative (CO92-C12, n=6; Java-9, n=2) Y pestis strains. RESULTS: All monkeys died 4 to 10 days postexposure and had lesions consistent with primary pneumonic plague. Antibodies to F1 antigen and other Y pestis antigens allowed us to differentiate F1-positive from F1-negative Y pestis strains in fixed tissues. CONCLUSIONS: In this study, F1 antigen was not a required virulence factor. Therefore, there may be a need for vaccines and diagnostic assays that are not solely based on the F1 antigen.},
author = {Percival, Author E and Whitehead, H},
doi = {10.2307/2256044},
isbn = {00220477},
issn = {00220477},
journal = {Journal of Ecology},
number = {2},
pages = {282--314},
pmid = {8712895},
title = {{A Quantitative Study of the Fauna of Some Types of Stream-Bed}},
volume = {17},
year = {1929}
}
@article{Glynn1965,
abstract = {Studies of the community composition, structure and species interrelationships of the Endocladia-Balanus association were carried out on the rocky shores at the Hopkins Marine Station, Pacific Grove, California, over the period 1959—1961. The organisms making up this biotic association form a horizontal band approximately two feet wide on intertidal rocks. The center of this belt averaged 4.6 ft in surveyed height above tidal datum, but field observations showed that the periods of exposure and submersion actually experienced under a variety of conditions at this level are those predicted for a level of about 3.8 ft above tidal datum. Selected aspects of the relatively mild marine and terrestrial climates were studied in relation to the high and low water periods. The composition of the association was determined from quadrat samples, qualitative collections and field observations. A total of at least 93 benthic and transient species was enumerated, belonging to 15 major groups. Samples taken near the center of the association in different areas and at different seasons demonstrated great similarity and stability in species composition. Thirty-five of the species present in the assemblage were of common occurrence, and where feasible the following aspects of the biology of each were studied: (a) habitat niche, (b) activity patterns, (c) organic matter content of the body (based on the dry weight and nitrogen content), (d) seasonal occurrence, (e) population structure, (f) reproductive activity, (g) food relationships, and (h) growth rate. The most characteristic species were: Rhodophyta-Endocladia muricata, Gigartina agardhii; Platyhelminthes-Notoplana acticola; Nemertea- Emplectonema gracile, Nemertopsis gracilis; Annelida-Syllis spenceri, S. vittata, S. armillaris, Nereis grubei, Perinereis monterea; Bryozoa- Filicrisia franciscana; Mollusca (8 gastropods, 1 chiton, and 3 bivalves)- Acmaea digitalis, A. scabra, A. pelta, Littorina scutulata, L. planaxis, Acanthina spirata, Thais emarginata, Tegula funebralis, Cyanoplax dentiens, Lasaea cistula, Mytilus californianus, Musculus sp.; Arthropoda (7 crustaceans, 2 insects, and 3 mites)- Balanus glandula, Chthamalus dalli, C. microtretus, Dynamenella glabra, Allorchestes ptilocerus, Hyale sp., Pachygrapsus crassipes, Diaulota densissima, Limonia marmorata, Agauopsis sp., a mesostigmatid mite, and Suidasia sp. A study of space relations shows that a multitude of species live among the holdfast branchlets and along the blades of E. muricata, and in the spaces formed by the tests of both living and dead B. glandula. Food studies of the 34 commonest animal species show that seven are filter feeders, ten are herbivorous browsers and scrapers feeding on encrusting forms, six are herbivores feeding on the large benthic algae, five are omnivores and scavengers, and six are carnivores which prey mostly on filter feeders and scraping herbivores. Pelagic materials present during high tide periods, in order of decreasing absolute volume, were: (a) large plant fragments, (b) phytoplankton, (c) other organic detritus, (d) zooplankton, and (e) inorganic detritus. Approximately 570 mg dry weight was available per m³ of sea water throughout the year. For the center of the Endocladia-Balanus zone the mean number of individuals above microscopic size of all species present at any time was 210,000/ m², the mean dry weight biomass was 2,640 g/m², and the mean nitrogen content was 25 g/m². In quantitative as well as qualitative terms, the Endocladia-Balanus association shows considerable similarity with the Gloiopeltis- Chthamalus association in the Sea of Japan and with the Chthamalus-Pygmaea zone on English rocky shores. Analysis of the association in terms of the protein content of the standing crop for the major trophic groups shows the following: filter feeding animals, 84 g/m²; larger red algae and their attached epiphytes, 58 g/m²; resident herbivores, 12 g/m²; a transient herbivore, 1.2 g/m²; resident carnivores, 0.9 g/m²; transient carnivores, 0.8 g/m²; omnivores and scavengers, 1.3 g/m². Although the scraping and grazing herbivores feed on algae produced in the zone, much of the food consumed in the association is derived by import of suspended detritus and plankton at high water.},
author = {Glynn, Peter W},
issn = {0067-4745},
journal = {Beaufortia},
number = {148},
pages = {1--198},
pmid = {23872633},
title = {{Community composition, structure, and interrelationships in the marine intertidal Endocladia muricata – Balanus glandula association in Monterey Bay, California}},
url = {http://www.repository.naturalis.nl/record/504945{\%}5Cnhttp://www.repository.naturalis.nl/document/548602},
volume = {12},
year = {1965}
}
@incollection{Day1967,
address = {Washington},
author = {Day, J H},
booktitle = {Estuaries},
editor = {Lauff, G. H.},
pages = {397--407},
publisher = {American Association for the Advancement of Science},
title = {{The biology of the Knysna estuary, South Africa}},
year = {1967}
}
@article{Angelini2006,
abstract = {Recently, there is an increasing perception that the ecosystem approach gives important insights to support fisheries stock assessment and management. This paper aims to quantify energy flows in the Itaipu Reservoir (Brazil) and to simulate increase of the fishing effort of some species, using Ecopath with Ecosim software, which could allow inferences on stability. Therefore, two steady-state Itaipu models were built (1983-87 and 1988-92). Results showed that: a) there are no differences between models, and results on aging trends do not vary over time indicating that fishery does not alter the ecosystem as a whole; b) results of fisheries simulations are approximate to mono-specific stock assessment for the same species and periods; c) many authors believe that tropical ecosystems are environments where biotic and abiotic oscillations are annual and sometimes unexpected, but the results found for the Itaipu Reservoir indicate that stability was met after 16 years.},
archivePrefix = {arXiv},
arxivId = {0811.3031},
author = {Angelini, Ronaldo and Agostinho, Angelo Antonio and Gomes, Luiz Carlos},
doi = {10.1590/S1679-62252006000200011},
eprint = {0811.3031},
isbn = {1679-6225},
issn = {16796225},
journal = {Neotropical Ichthyology},
keywords = {Ecopath,Ecosystem,Fisheries,Food web,Simulation},
number = {2},
pages = {253--260},
title = {{Modeling energy flow in a large Neotropical reservoir: A tool do evaluate fishing and stability}},
volume = {4},
year = {2006}
}
@incollection{Harrison1995,
address = {Cambridge},
author = {Harrison, S},
booktitle = {Global Biodiversity Assessment, United Nations Environmental Programme},
chapter = {5.3.2},
editor = {Heywood, V. H. and Watson, R. T.},
keywords = {BIODIVERSITY,BODY SIZE,DISPERSAL,DOMINANCE,FRAGMENTATION,METAPOPULATION,POLLINATION},
pages = {Ch. 5.2.3},
publisher = {Cambridge University Press},
title = {{Effects of spatial structure on ecosystem functioning}},
year = {1995}
}
@article{Rasmussen1941,
abstract = {An introduction is given, but it is too long to type in here.},
author = {Rasmussen, Irvin},
isbn = {00129615},
journal = {Ecological Monographs},
keywords = {Arizona,DESERT,Juniper,PLANT COMMUNITY,WESTERN UNITED-STATES,biotic community,ecology,pinyon},
number = {3},
pages = {229--275},
title = {{Biotic Communities of Kaibab Plateau, Arizona Author(s):}},
url = {http://www.jstor.org/stable/1943204},
volume = {11},
year = {1941}
}
@article{Angelini2005,
abstract = {To describe the Upper Paran{\'{a}} River Floodplain (the last non-dammed stretch of the Paran{\'{a}} River, Brazil) a food web model was quantified using ECOPATH. The modeled ecosystem showed maturity because of the total primary production/total respiration ratio (close to 2), Finn's cycling index (7{\%}) and overhead (65{\%}). The first model elaborated had 40 compartments/groups, but its transfer efficiencies among trophic levels did not reduce in despite the trophic level increasing. To solve this, the effect of two grouping methods on system-level information and other ecosystem attributes was investigated. The first series tested, named "classic" (researcher intuitive way and by food preferences) also did not reduce transfer efficiencies. In the second series, named "by pathways", the first species grouping were those with higher number of input pathways and longest mean length of pathways. Thereby, the news groups from aggregation decreased the number of components and system's richness, but stability (measured by overhead) did not change, including the model with only eight compartments. The great number of the ten compartments that showed these characteristics was piscivores, increasing the redundancy within highest trophic level. The use of pathways (number and length) can be useful to lumping species since it reduces compartments and do compromise neither maturity nor stability, diminishing grouping subjectivity. {\textcopyright} 2004 Elsevier B.V. All rights reserved.},
author = {Angelini, Ronaldo and Agostinho, Angelo Antonio},
doi = {10.1016/j.ecolmodel.2004.06.025},
isbn = {0304-3800},
issn = {03043800},
journal = {Ecological Modelling},
keywords = {Aquatic ecosystem,Ecopath,Floodplain,Food web,Mathematical models},
number = {2-3},
pages = {109--121},
title = {{Food web model of the Upper Paran{\'{a}} River Floodplain: Description and aggregation effects}},
volume = {181},
year = {2005}
}
@article{Amundsen2013,
abstract = {Introduced species can alter the topology of food webs. For instance, an introduction can aid the arrival of free-living consumers using the new species as a resource, while new parasites may also arrive with the introduced species. Food-web responses to species additions can thus be far more complex than anticipated. In a subarctic pelagic food web with free-living and parasitic species, two fish species (arctic charr Salvelinus alpinus and three-spined stickleback Gasterosteus aculeatus) have known histories as deliberate introductions. The effects of these introductions on the food web were explored by comparing the current pelagic web with a heuristic reconstruction of the pre-introduction web. Extinctions caused by these introductions could not be evaluated by this approach. The introduced fish species have become important hubs in the trophic network, interacting with numerous parasites, predators and prey. In particular, five parasite species and four predatory bird species depend on the two introduced species as obligate trophic resources in the pelagic web and could therefore not have been present in the pre-introduction network. The presence of the two introduced fish species and the arrival of their associated parasites and predators increased biodiversity, mean trophic level, linkage density, and nestedness; altering both the network structure and functioning of the pelagic web. Parasites, in particular trophically transmitted species, had a prominent role in the network alterations that followed the introductions.},
author = {Amundsen, Per Arne and Lafferty, Kevin D. and Knudsen, Rune and Primicerio, Raul and Kristoffersen, Roar and Klemetsen, Anders and Kuris, Armand M.},
doi = {10.1007/s00442-012-2461-2},
isbn = {0044201224612},
issn = {00298549},
journal = {Oecologia},
keywords = {Non-native species,Pelagic community,Species additions,Topology,Trophic interactions},
number = {4},
pages = {993--1002},
pmid = {23053223},
title = {{New parasites and predators follow the introduction of two fish species to a subarctic lake: Implications for food-web structure and functioning}},
volume = {171},
year = {2013}
}
@article{Twomey1945,
abstract = {See full-text article at JSTOR},
author = {Twomey, Arthur C},
issn = {00129615},
journal = {Ecological Monographs},
number = {2},
pages = {173--205},
title = {{The bird population of an Elm-Maple forest with special reference to aspection, territorialism, and coactions}},
volume = {15},
year = {1945}
}
@article{Alcorlo2001,
abstract = {Energetic and dynamic constraints have been proposed as rival factors in determining food-web structure. Food- web length might be controlled either by the amount of energy entering the web (energetic constraints) or by time span between consecutive disturbances relative to time needed to build up a population (dynamic constraints). Dynamic constraints are identified with processes functioning at a regional scale such as climate, lithology and hydrogeology. Energetic constraints are related with processes operating both at a regional and a local scale. We studied the contribution of energetic constraints to food-web organization in two temporary saline lakes with similar dynamic constraints. Lakes were sampled fortnightly during two hydroperiods (1994/1995 and 1995/1996). Differences in energetic constraints between lakes result in divergent assemblages of primary producers. Consumer assemblages in both lakes, however, are similar in species composition although differ in total biomass and species abundances. Food-webs are short with a high proportion of omnivores. To simulate an increase in the energy input entering to these systems, an addition of nutrients (to a final concentration of 100 µg{\textperiodcentered}l −1 P-PO4 3−) was done in mesocosms placed within the lakes in order to obtain an increase in the phytoplankton biomass. No significant response to nutrient enrichment was found in food-web structure (composition, density or biomass)},
author = {Alcorlo, Paloma and Baltan{\'{a}}s, Angel and Montes, Carlos},
doi = {10.1023/A:1014594408119},
isbn = {0018-8158},
issn = {00188158},
journal = {Hydrobiologia},
keywords = {Connectance,Ecosystem functioning,Food web patterns,Trophic interactions},
pages = {307--316},
title = {{Food-web structure in two shallow salt lakes in Los Monegros (NE Spain): Energetic vs dynamic constraints}},
volume = {466},
year = {2001}
}
@article{Angelini2011,
abstract = {Information on the mean trophic level of fishery landings in Angola and the output from a preliminary Ecopath with Ecosim (EwE) model were used to examine the dynamics of the Angolan marine ecosystem. Results were compared with the nearby Namibian and South African ecosystems, which share some of the exploited fish populations. The results show that: (i) The mean trophic level of Angola's fish landings has not decreased over the years; (ii) There are significant correlations between the landings of Angola, Namibia and South Africa; (iii) The ecosystem attributes calculated by the EwE models for the three ecosystems were similar, and the main differences were related to the magnitude of flows and biomass; (iv) The similarity among ecosystem trends for Namibia, South Africa and Angola re-emphasizes the need to continue collaborative regional studies on the fish stocks and their ecosystems. To improve the Angolan model it is necessary to gain a better understanding of plankton dynamics because plankton are essential for Sardinella spp. An expanded analysis of the gut contents of the fish species occupying Angola's coastline is also necessary.},
author = {Angelini, Ronaldo and Vaz-Velho, Filomena},
doi = {10.3989/scimar.2011.75n2309},
isbn = {0214-8358},
issn = {1886-8134},
journal = {Scientia Marina},
keywords = {Ecopath con Ecosim,Ecopath with Ecosim,Sardinella,an{\'{a}}lisis tr{\'{o}}fico de desembarques,con,con los ecosistemas de,de angola,din{\'{a}}mica de la pesquer{\'{i}}a,ecopath with ecosim,ecosim,ecosistema marino,el modelo ecopath con,el nivel tr{\'{o}}-,estructura del ecosistema y,ewe,fico de los desembarques,fishery management,fueron utilizados para examinar,gesti{\'{o}}n pesquera,la,los resultados fueron comparados,marine ecosystem,namibia y de sud{\'{a}}frica,nivel tr{\'{o}}fico,pesqueros de angola y,pesqueros en angola,resumen,sardinella,trophic level},
number = {2},
pages = {309--319},
title = {{Ecosystem structure and trophic analysis of Angolan fishery landings}},
url = {http://scientiamarina.revistas.csic.es/index.php/scientiamarina/article/view/1254/1324},
volume = {75},
year = {2010}
}
@article{Riede2011,
abstract = {Despite growing awareness of the significance of body-size and predator-prey body-mass ratios for the stability of ecological networks, our understanding of their distribution within ecosystems is incomplete. Here, we study the relationships between predator and prey size, body-mass ratios and predator trophic levels using body-mass estimates of 1313 predators (invertebrates, ectotherm and endotherm vertebrates) from 35 food-webs (marine, stream, lake and terrestrial). Across all ecosystem and predator types, except for streams (which appear to have a different size structure in their predator-prey interactions), we find that (1) geometric mean prey mass increases with predator mass with a power-law exponent greater than unity and (2) predator size increases with trophic level. Consistent with our theoretical derivations, we show that the quantitative nature of these relationships implies systematic decreases in predator-prey body-mass ratios with the trophic level of the predator. Thus, predators are, on an average, more similar in size to their prey at the top of food-webs than that closer to the base. These findings contradict the traditional Eltonian paradigm and have implications for our understanding of body-mass constraints on food-web topology, community dynamics and stability.},
author = {Riede, Jens O. and Brose, Ulrich and Ebenman, Bo and Jacob, Ute and Thompson, Ross and Townsend, Colin R. and Jonsson, Tomas},
doi = {10.1111/j.1461-0248.2010.01568.x},
isbn = {1461-0248},
issn = {14610248},
journal = {Ecology Letters},
keywords = {Allometry,Body-size ratio,Ecological networks,Food-webs,Predation,Predator-prey interactions},
number = {2},
pages = {169--178},
pmid = {21199248},
title = {{Stepping in Elton's footprints: A general scaling model for body masses and trophic levels across ecosystems}},
volume = {14},
year = {2011}
}
@article{Rayner2010,
abstract = {In Australia's Wet Tropics rivers, perennial base flows punctuated by wet season floods drive instream responses across a range of spatial and temporal scales. We combined gut-content and stable-isotope analyses to produce preliminary webs depicting trophic links between fish, their main prey items and basal productivity sources. We then used these webs to test the applicability of general food web principles developed in other tropical systems. Although a range of sources appeared to underpin fish productivity, a large portion of total energy transfer occurred through a subset of trophic links. Variability in food web structure was negatively correlated with spatial scale, being seasonally stable at river reaches and variable at smaller scales. Wet Tropics rivers are similar to those in other tropical areas, but exhibit some unique characteristics. Their high degree of channel incision improves longitudinal connectivity, thereby allowing fish to move between mesohabitats and target their preferred prey items, rather than shifting their diet as resources fluctuate. However, this also inhibits lateral connectivity and limits terrestrial energy inputs from beyond the littoral zone.},
author = {Rayner, Thomas S. and Pusey, Bradley J. and Pearson, Richard G. and Godfrey, Paul C.},
doi = {10.1071/MF09202},
isbn = {1323-1650},
issn = {13231650},
journal = {Marine and Freshwater Research},
keywords = {connectance,disturbance,flood,movement,pulse,scaling,seasonality},
number = {8},
pages = {909--917},
title = {{Food web dynamics in an Australian Wet Tropics river}},
volume = {61},
year = {2010}
}
@article{Edwards1982,
author = {Edwards, D Craig and Conover, David and Iii, Frederick Sutter},
journal = {Ecology},
number = {August},
pages = {1175--1180},
title = {{MOBILE PREDATORS AND THE STRUCTURE OF MARINE INTER TIDAL COMMUNITIES ' at Canoe Beach Cov}},
volume = {63},
year = {1982}
}
@article{Hildrew1985,
abstract = {1. Three species of Tanypodinae (Chironomidae) were found in an acid and iron?rich stream in southern England. Maximum abundance was achieved in summer and they were sparse at other times. Individuals were aggregated on the stream bed and were overrepresented in accumulations of leaf litter. 2. The diets of all three species consisted of a mixture of prey (prominently detritivorous chironomid larvae) and detritus. More detritus and fewer prey were taken in winter than in summer. 3. When comparing large tanypod species with small and, intraspecifically, late instars with early, the proportion of guts containing prey increased with increasing body size. 4. Stonefly larvae were more prominent in the diet of Zavrelimyia barbatipes (Kieffer) in summer than in winter but for the other two species the reverse was true. A bigger proportion of Trissopelopia longimana (Staeger) guts contained prey in early summer than in August whereas more Macropelopia goetghebueri (Kieffer) guts contained prey in August. This was apparently a consequence of seasonal differences in the distribution of body size among the populations of these two species. 5. The stream contains two further common predators, Plectrocnemia conspersa (Curtis) and Sialis fuliginosa Pict. These are important predators of tanypod larvae but might also compete with them since they severely deplete populations of prey taken in common. 6. Analysis of the food?web in Broadstone Stream reveals remarkably high values of connectance (C and Cmax) and of species richness times connectance (SCmax). Such characteristics are theoretically associated with fragile and dynamically unstable food webs, and may be found in ?constant? environments. There is also an apparently unusual prevalence of omnivory in the community.},
author = {HILDREW, ALAN G. and TOWNSEND, COLIN R. and HASHAM, AZIM},
doi = {10.1111/j.1365-2311.1985.tb00738.x},
isbn = {1365-2311},
issn = {13652311},
journal = {Ecological Entomology},
keywords = {Tanypodinae,acid,ecology,feeding,food webs,predators,streams},
number = {4},
pages = {403--413},
pmid = {700},
title = {{The predatory Chironomidae of an iron‐rich stream: feeding ecology and food web structure}},
url = {http://doi.wiley.com/10.1111/j.1365-2311.1985.tb00738.x},
volume = {10},
year = {1985}
}
@article{Angelini2013,
abstract = {Flood pulse and biotic interrelationships control the food web dynamics of river floodplain systems. The Pantanal Plain in the Paraguay River Basin (Brazil) occupies 140,000km2of periodically flooded areas and is divided into 12 subregions with different characteristics related to the flood pulse duration, the vegetation, the type of soil, and the resources used in activities, particularly fishing. In this study, we used Ecopath with Ecosim (EwE) to model three oxbow lakes in the South Pantanal Plain, where there is no fishing activity, to test the similarity of the ecosystems, to identify the keystone species and the types of food web controls, and to determine whether these environments can support moderated fishing pressure. We found that the food webs of the oxbow lakes are similar to each other because, although they depend mainly on the presence or absence of predators, flood pulses similarly homogenize the lakes. The results highlight the importance of detritus in these food webs. In addition, the highest values of the keystoneness species index in the three models highlight the role of top predators (Hoplias malabaricus, Serrasalmus spp., Pseudoplatystoma reticulatum, birds, and mammals). Therefore, we suggest that the food webs in the three systems are subjected to an alternated control process: detritus controls the food web during the flood season and by the top predators during the dry season. The simulation outputs indicate that these oxbow lakes can sustain only moderate fishing because increasing the fishing pressure reduces the biodiversity and can negatively impact the top predators. {\textcopyright} 2013 Elsevier B.V.},
author = {Angelini, Ronaldo and de Morais, Ronny Jos{\'{e}} and Catella, Agostinho Carlos and Resende, Emiko Kawakami and Libralato, Simone},
doi = {10.1016/j.ecolmodel.2013.01.001},
isbn = {0304-3800},
issn = {03043800},
journal = {Ecological Modelling},
keywords = {Flood pulse, Top predator,Food web control,Kempton index,Keystone species,L index},
pages = {82--96},
publisher = {Elsevier B.V.},
title = {{Aquatic food webs of the oxbow lakes in the Pantanal: A new site for fisheries guaranteed by alternated control?}},
url = {http://dx.doi.org/10.1016/j.ecolmodel.2013.01.001},
volume = {253},
year = {2013}
}
@phdthesis{Corker1984,
author = {Corker, Barbara},
pages = {343},
school = {University of Hong Kong},
title = {{The ecology of the pitcher plant Nepethes mirabilis and its associated fauna in Hong Kong}},
year = {1984}
}
@article{Cruz-Escalona2007,
abstract = {Alvarado is one of the most productive estuary-lagoon systems in the Mexican Gulf of Mexico. It has great economic and ecological importance due to high fisheries productivity and because it serves as a nursery, feeding, and reproduction area for numerous populations of fishes and crustaceans. Because of this, extensive studies have focused on biology, ecology, fisheries (e.g. shrimp, oysters) and other biological components of the system during the last few decades. This study presents a mass-balanced trophic model for Laguna Alvarado to determine it's structure and functional form, and to compare it with similar coastal systems of the Gulf of Mexico and Mexican Pacific coast. The model, based on the software Ecopath with Ecosim, consists of eighteen fish groups, seven invertebrate groups, and one group each of sharks and rays, marine mammals, phytoplankton, sea grasses and detritus. The acceptability of the model is indicated by the pedigree index (0.5) which range from 0 to 1 based on the quality of input data. The highest trophic level was 3.6 for marine mammals and snappers. Total system throughput reached 2680 t km-2 year-1, of which total consumption made up 47{\%}, respiratory flows made up 37{\%} and flows to detritus made up 16{\%}. The total system production was higher than consumption, and net primary production higher than respiration. The mean transfer efficiency was 13.8{\%}. The mean trophic level of the catch was 2.3 and the primary production required to sustain the catch was estimated in 31 t km-2 yr-1. Ecosystem overhead was 2.4 times the ascendancy. Results suggest a balance between primary production and consumption. In contrast with other Mexican coastal lagoons, Laguna Alvarado differs strongly in relation to the primary source of energy; here the primary producers (seagrasses) are more important than detritus pathways. This fact can be interpreted a response to mangrove deforest, overfishing, etc. Future work might include the compilation of fishing and biomass time trends to develop historical verification and fitting of temporal simulations. {\textcopyright} 2007 Elsevier Ltd. All rights reserved.},
author = {Cruz-Escalona, V. H. and Arregu{\'{i}}n-S{\'{a}}nchez, F. and Zetina-Rej{\'{o}}n, M.},
doi = {10.1016/j.ecss.2006.10.013},
isbn = {0272-7714},
issn = {02727714},
journal = {Estuarine, Coastal and Shelf Science},
keywords = {Alvarado Lagoon,Ecopath model,Mexico,coastal lagoon,energy flows,trophic structure},
number = {1-2},
pages = {155--167},
title = {{Analysis of the ecosystem structure of Laguna Alvarado, western Gulf of Mexico, by means of a mass balance model}},
volume = {72},
year = {2007}
}
@article{Cornejo-Donoso2008,
abstract = {Species, biomasses, production rates, distribution and other aspects of the community structure have been studied in the Antarctic Ecosystem; however there are no integrated models that explain mass transfer at the spatial mesoscale. Even though the Antarctic Ecosystem as a whole has been identified as a functional unit, subsystems could be identified and characterized, among which, the Antarctic Peninsula stands out for its particular geography, oceanography and trophic web. The aim of this work is to construct a mass balanced model describing the main trophic interactions of this community. The model is built using the software Ecopath with Ecosim 5.0, which yields a representation of the trophic web and estimations of global ecosystem properties. Phytoplankton, zooplankton and krill accounted for most of the mass flow. Flows to the trophic level II (TL II; detritivores and herbivores) were attributed to zooplanktonic and benthic organisms mainly. Flows to the TL III were explained by fish, birds (flying birds and penguins), Balaenoptera acutorostrata and baleen whales. Flows to the TL IV were dominated by some fish, birds (flying birds and penguins) and mammals. Finally, in TL VI, the flows were dominated by Orcinus orca. O. orca was the top predator in the ecosystem with a TL of 4.88, followed by Physeter catodon (4.63) and Hydrurga leptonyx (4.62). Krill was found at the intermediate TL (2.33). Resulting ecosystem indexes (e.g. total transfer efficiency, connectance index, etc.) were consistent and characteristic of ecosystems of high temporal and spatial variability. The model gives a comprehensive description of the food web dominated by phytoplankton-krill-top predators chain, and complemented with alternative food pathways (e.g. through Electrona antarctica), which together gives an enhanced complexity to the system. Despite the limitations of the model in data gaps, particularly for winter season, grouping of functional groups, steady state assumptions, etc., it improves the description of the trophic structure and ecosystem functioning of the Antarctic Peninsula and highlights gaps in knowledge. {\textcopyright} 2008 Elsevier B.V. All rights reserved.},
author = {Cornejo-Donoso, Jorge and Antezana, Tarsicio},
doi = {10.1016/j.ecolmodel.2008.06.011},
isbn = {0304-3800},
issn = {03043800},
journal = {Ecological Modelling},
keywords = {Antarctic Peninsula,CCAMLR 48.1,Ecopath,Euphausia superba,Krill,Krill fishery,Mass balance,Trophic model},
number = {1-2},
pages = {1--17},
title = {{Preliminary trophic model of the Antarctic Peninsula Ecosystem (Sub-area CCAMLR 48.1)}},
volume = {218},
year = {2008}
}
@article{Angelini2010,
abstract = {Understanding the mechanisms which regulate aquatic food webs dynamics have been an important focus inquiry since ecological research started to emphasize ecosystems' structure and functioning. The objective of this study was verify if species composition were the same at three different habitats in Corrente River. If species composition were the same it will allow to model the food web by a single model, otherwise each portion should be modeled independently. Ecopath model and the keystoneness index (KSi) were used in order to evaluate the entire food web and to comprehend the system control. Five surveys in the period from June of 2003 to until June of 2005 were carried out in three different habitats in the river. Results showed that: i) there are seven fish species; ii) habitats are very similar; iii) all species reproduced in the rainy season with exception of Brycon nattereri. A single Ecopath model indicated low resilience and stability. Keystoneness rank values showed a mixed system control mechanism, where one predator, Salminus hilarii, one intermediate consumer, Astyanax altiparanae and one consumer of trophic level two (terrestrial invertebrates) had respectively, the highest keystoneness index values.},
author = {Angelini, Ronaldo and Alo{\'{i}}sio, Gustavo Ribeiro and Carvalho, Adriana Rosa},
issn = {18099009},
journal = {Pan-American Journal of Aquatic Sciences},
keywords = {Corrente River,Feeding behavior,Food web mechanisms,Paran{\'{a}} River,Reproduction},
number = {3},
pages = {421--431},
title = {{Mixed food web control and stability in a Cerrado river (Brazil)}},
volume = {5},
year = {2011}
}
@article{Castilla1981,
author = {Santelices, Bernab{\'{e}}},
issn = {0304-8764},
journal = {Medio Ambiente},
keywords = {ACORU},
number = {1/2},
pages = {175--189},
title = {{Perspectivas de investigaci{\'{o}}n en estructura y din{\'{a}}mica de comunidades intermareales rocosas de Chile Central. I. Cinturones de macroalgas}},
url = {http://agris.fao.org/agris-search/search.do?recordID=CL19840084603},
volume = {5},
year = {1981}
}
@article{Mohr1943,
abstract = {JSTOR is a not-for-profit service that helps scholars, researchers, and students discover, use, and build upon a wide range of content in a trusted digital archive. We use information technology and tools to increase productivity and facilitate new forms of scholarship.},
author = {Society, Ecological and Monographs, Ecological},
doi = {10.2307/1943223},
isbn = {00129615},
issn = {00129615},
journal = {Ecological Monographs},
number = {3},
pages = {275--298},
title = {{Cattle Droppings as Ecological Units}},
url = {http://www.jstor.org/stable/1943223 http://www.jstor.org/page/info/about/policies/terms.jsp http://www.jstor.org},
volume = {13},
year = {1943}
}
@article{Peterson1979,
abstract = {Community organization was studied by experiment and observa-tion from October 1972-October 1974 in the marine epifaunal assemblages at each end of Barnegat Inlet, New Jersey. The rock jetty at the wave-exposed eastern end of the inlet possesses an intertidal community with the following attributes: (1) a high intertidal zone dominated by the barnacle, Balanus balanoides, but also occupied by the blue mussel, Mytilus edulis, in rock crevices, (2) a mid and low intertidal zone with usually {\textless} 10{\%} free space and extreme numerical dominance by Mytilus edulis (usually {\textgreater} 85{\%} cover) during summer and fall, and (3) almost no intertidal predators or herbivores. The predatory seastar, Asterias forbesi, is abundant subtidally. Controlled removal experiments indicate that in the mid and low intertidal underlying barnacles perish as a consequence of the establishment of extensive secondary cover by Mytilus, probably because Mytilus outcompetes Balanus through suffocation or starvation. Mytilus transplants demonstrate that the mussels do not survive outside of crevices in the high intertidal, which thus may represent for Balanus a refuge from competition by Mytilus. The pilings on docks at the protected western end of Barnegat Inlet possess an intertidal epifaunal community with the following characteristics: (1) a high intertidal zone that includes Balanus balanoides, a second barnacle, Balanus eburneus, and an herbivorous gastropod, Littorina littorea, (2) a mid and low intertidal zone with usually {\textgreater} 40{\%} free space in the summer and fall and the remaining area covered by several abundant species with no extreme dominant, and (3) abundant predators, chiefly the oyster drill, Urosalpinx cinerea, the blue crab, Callinectes sapidus, and a mud crab, Neopa-hope texana sayi. Asteriasforbesi, while abundant subtidally, is also occasio-nally present on intertidal surfaces. Controlled exclusion of predators by caging several replicate pilings at the western end of the inlet reveals that},
author = {Peterson, Charles H.},
doi = {10.1007/BF00345993},
isbn = {0029-8549},
issn = {00298549},
journal = {Oecologia},
number = {1},
pages = {1--24},
title = {{The importance of predation and competition in organizing the intertidal epifaunal communities of Barnegat Inlet, New Jersey}},
volume = {39},
year = {1979}
}
@article{Cohen2003,
abstract = {Measuring the numerical abundance and average body size of individuals of each species in an ecological community's food web reveals new patterns and illuminates old ones. This approach is illustrated using data from the pelagic community of a small lake: Tuesday Lake, Michigan, United States. Body mass varies almost 12 orders of magnitude. Numerical abundance varies almost 10 orders of magnitude. Biomass abundance (average body mass times numerical abundance) varies only 5 orders of magnitude. A new food web graph, which plots species and trophic links in the plane spanned by body mass and numerical abundance, illustrates the nearly inverse relationship between body mass and numerical abundance, as well as the pattern of energy flow in the community. Species with small average body mass occur low in the food web of Tuesday Lake and are numerically abundant. Larger-bodied species occur higher in the food web and are numerically rarer. Average body size explains more of the variation in numerical abundance than does trophic height. The trivariate description of an ecological community by using the food web, average body sizes, and numerical abundance includes many well studied bivariate and univariate relationships based on subsets of these three variables. We are not aware of any single community for which all of these relationships have been analyzed simultaneously. Our approach demonstrates the connectedness of ecological patterns traditionally treated as independent. Moreover, knowing the food web gives new insight into the disputed form of the allometric relationship between body mass and abundance.},
author = {Cohen, J. E. and Jonsson, T. and Carpenter, S. R.},
doi = {10.1073/pnas.232715699},
isbn = {0027-8424},
issn = {0027-8424},
journal = {Proceedings of the National Academy of Sciences},
keywords = {Body Constitution,Food Chain,Species Specificity},
number = {4},
pages = {1781--1786},
pmid = {12547915},
title = {{Ecological community description using the food web, species abundance, and body size}},
url = {http://www.pnas.org/cgi/doi/10.1073/pnas.232715699},
volume = {100},
year = {2003}
}
@article{Closs1994,
abstract = {Trophic interactions between benthic invertebrates in a large freshwater pond were established using analyses of gut contents, laboratory feeding trials and published information. The web was detritus based and contained 36 "species". Spatial and temporal variation in food web structure was assessed by partitioning the overall food web into subwebs drawn up for two areas of the pond on each of five sampling dates over the course of a season. Substantial variation occurred between webs from the open water benthos and the pond margin areas, both within and between sampling dates. Webs became more complex (species rich) over the season and, within, the webs from each area, species composition and interactions varied due to body size and life history effects. In relation to published data the webs had high average connectance, high proportions of intermediate species (and links among intermediate species) and moderately high predator: prey ratios. Other food web statistics varied considerably, but most fell within the ranges of values from previous analyses. Omnivory was extensive and, due to size dependent predation, cannibalism and trophic loops occurred. The potential effects of spatial and temporal variation in the web on the dynamics of trophic interactions suggest that Cohen and Newman's "cascade model", which imposes simple, non-dynamic constraints on the distribution of trophic interactions, may be an appropriate explanation for web structure. However, simulated webs generated by the cascade model, using parameters derived from the webs in this study, indicated the model's sensitivity to connectance, and suggested that, in its present form, the model adequately accounts for the proportions of basal, top and intermediate species, but may not be a sufficient explanation for observed food chain lengths.},
author = {Warren, Philip H.},
doi = {10.2307/3565588},
isbn = {00301299},
issn = {00301299},
journal = {Oikos},
keywords = {connectance},
number = {3},
pages = {299},
pmid = {14},
title = {{Spatial and Temporal Variation in the Structure of a Freshwater Food Web}},
url = {http://www.jstor.org/stable/3565588?origin=crossref},
volume = {55},
year = {1989}
}
@incollection{Copeland1974,
address = {Washington},
author = {Copeland, B J and Tenore, K R and Horton, D B},
booktitle = {Coastal ecological systems of the United States, Vol. 2},
editor = {Odum, Howard Thomas and Copeland, B. J. and McMahan, E. A.},
pages = {315--357},
publisher = {Conservation Foundation, National Oceanic and Atmospheric Administration},
title = {{Oligohaline regime}},
year = {1974}
}
@article{Carlson1968,
abstract = {Over 1400 benthos collections were made from eight sampling areas near the Illinois shore of the Mississippi River above Dam 19 in the summers of 1960 and 1961.  No significant difference was found in numbers of macroscopic organisms collected after sifting Ekman dredge contents through 20- and 40-mesh screens.  Sphaerium transversum was the most abundant organism and was the only organism collected at every sampling plot on each collection date.  Hexagenia naiads were the most abundant insects at all sampling areas in 1960.  Coelotanypus, Tendipes plumosus, Stenochironomus, oligochaetes, Campeloma, and Lioplax subcarinata were also abundant and frequently collected.  Somatogyrus depressus and Oecetis sp. b were abundant during certain time intervals.  Tendipes plumosus became the most abundant insect in 1961 at the part of the study area nearest Dam 19.  Several organisms were more abundant in 1961, when mayfly population densities were low, than in 1960.  As the most abundant insects emerged during each summer, other elements of the benthos increased in abundance.  An average of 2,924 organisms/m² was collected in the study area.  Major elements of the benthos seem to have changed little in the last 30 years.  The study area had a climax community characteristic of mature streams and showed no evidence of serious pollution.},
author = {Carlson, Clarence a},
journal = {Ecology},
keywords = {--},
number = {1},
pages = {162--169},
title = {{Summer Bottom Fauna of the Mississippi}},
volume = {49},
year = {1968}
}
@article{Bulman2001,
abstract = {A total of 8200 stomach samples was collected from 102 fish species caught by trawl or gillnet during research surveys on the south-eastern Australian shelf from 1993 to 1996. Diet compositions were analysed based on percentages of wet weight of prey. Of the total fish examined, 70 species had sufficient stomach samples (i.e. {\&}{\#}62;10) for further analysis. Ten trophic guilds were identified from cluster analysis. Benthic prey dominated the diets. However, analysis on a subset of 28 abundant species that were commercially and ecologically important, showed that pelagic prey was dominant, particularly for 12 quota species. This suggests that pelagic production contributes significantly to the trawl fishery production. Further analysis on the diets of these 28 species found that although fish was more important than invertebrate prey, there was no evidence of significant predation on commercially important species (quota species)by other fish species. A food web diagram was constructed, mostly based on the diet compositions, guild structure and relative abundance of commercially and ecologically important fish species, to show major trophic interactions of the shelf ecosystem.{\textless}/p{\textgreater}},
author = {Bulman, C. and Althaus, F. and He, X. and Bax, N. J. and Williams, A.},
doi = {10.1071/MF99152},
isbn = {1323-1650},
issn = {13231650},
journal = {Marine and Freshwater Research},
number = {4},
pages = {537--548},
title = {{Diets and trophic guilds of demersal fishes of the south-eastern Australian shelf}},
volume = {52},
year = {2001}
}
@article{Brown1971,
author = {Brown, J},
journal = {US Tundra Biome Program Report},
pages = {26},
title = {{The structure and function of the tundra ecosystem. Final summer activity report.}},
volume = {6},
year = {1971}
}
@article{Simenstad1978,
abstract = {Reexamination of stratified faunal components of a prehistoric Aleut midden excavated on Amchitka Island, Alaska [USA] indicates that Aleut prey items changed dramatically during 2500 yr of aboriginal occupation. Recent ecological studies in the Aleutian Islands have shown the concurrent existence of 2 alternate stable nearshore communities, one dominated by macroalgae, the other by epibenthic herbivores, which are respectively maintained by the presence or absence of dense sea otter [Enhydra lutris] populations. Rather than cultural shifts in food preference, the changes in Aleut prey were probably the result of local overexploitation of sea otters by aboriginal Aleuts.},
author = {Simenstad, Charles A. and Estes, James A. and Kenyon, Karl W.},
doi = {10.1126/science.200.4340.403},
isbn = {0036-8075},
issn = {00368075},
journal = {Science},
number = {4340},
pages = {403--411},
pmid = {197866047964},
title = {{Aleuts, sea otters, and alternate stable-state communities}},
volume = {200},
year = {1978}
}
@article{Boit2012,
abstract = {Mechanistic understanding of consumer-resource dynamics is critical to predicting the effects of global change on ecosystem structure, function and services. Such understanding is severely limited by mechanistic models' inability to reproduce the dynamics of multiple populations interacting in the field. We surpass this limitation here by extending general consumer-resource network theory to the complex dynamics of a specific ecosystem comprised by the seasonal biomass and production patterns in a pelagic food web of a large, well-studied lake. We parameterised our allometric trophic network model of 24 guilds and 107 feeding relationships using the lake's food web structure, initial spring biomasses and body-masses. Adding activity respiration, the detrital loop, minimal abiotic forcing, prey resistance and several empirically observed rates substantially increased the model's fit to the observed seasonal dynamics and the size-abundance distribution. This process illuminates a promising approach towards improving food-web theory and dynamic models of specific habitats.},
author = {Boit, Alice and Martinez, Neo D. and Williams, Richard J. and Gaedke, Ursula},
doi = {10.1111/j.1461-0248.2012.01777.x},
isbn = {1461-023X},
issn = {1461023X},
journal = {Ecology Letters},
keywords = {Allometric Trophic Network model,Community ecology,Food web,Multi-trophic dynamics,Seasonal plankton succession},
number = {6},
pages = {594--602},
pmid = {22513046},
title = {{Mechanistic theory and modelling of complex food-web dynamics in Lake Constance}},
volume = {15},
year = {2012}
}
@incollection{Bradshaw1983,
abstract = {Summary$\backslash$n3- {\{}"Intra-{\}} and interspecific encounter of overwintering mosquitoes, Wyeomyia smithii, and midges, Metriocnemus knabi, from Florida to Canada along with competition experiments in the field and laboratory indicate that both species are now more self- than other-regulated. Consequently, they should be able to coexist indefinitely."$\backslash$n5- {\{}"Gas{\}} and nitrogen production and consumption by insects and leaves suggest that the interaction between the plant and its inhabitants is mutualistic."},
address = {Medford, NJ},
author = {Bradshaw, W E},
booktitle = {International Congress of Entomology Proceedings},
editor = {Frank, J. H. and Lounibos, L. P.},
keywords = {Chironomidae,Metriocnemus,Orthocladiinae,ecology},
pages = {123},
publisher = {Plexus Publishsing},
title = {{Interaction of the mosquito, Wyeomyia smithii, and the midge, Metriocnemus knabi, with their carnivorous host, Sarracenia purpurea}},
volume = {16},
year = {1980}
}
@incollection{Brooks1963,
abstract = {Noninvasive prenatal testing performed with the use of massively parallel sequencing of cell-free {\{}DNA{\}} ({\{}cfDNA{\}} testing) in maternal plasma came into use in clinical prenatal care in the United States in late 2011. This transition occurred after multiple clinical validation studies all showed high sensitivities, specificities, and negative predictive values for detection of the most common autosomal aneuploidies.1?9 Plasma samples for the validation studies were either acquired retrospectively from populations with known karyotypes or collected prospectively from high-risk populations to ensure an adequate enrichment of aneuploid fetal samples for testing. The results of these studies were sufficiently robust to . . .},
address = {Madison, WI},
author = {Leone, L. and Germiglio, C. and Monteforte, R. and Trovato, G.},
booktitle = {Minerva Cardioangiologica},
doi = {10.1056/NEJMoa1311037},
editor = {Frey, D. G.},
isbn = {1533-4406 (Electronic)$\backslash$r0028-4793 (Linking)},
issn = {00264725},
keywords = {Aneurysms,Aortic aneurysm, abdominal,Surgery},
number = {4},
pages = {507--520},
pmid = {24571752},
publisher = {University of Wisconsin Press},
title = {{Aneurisma dell'aorta sottorenale associati a patologia viscerale addominale}},
volume = {54},
year = {2006}
}
@book{Brown1975,
address = {Anchorage},
author = {Brown, J},
booktitle = {Univ. of AK},
editor = {Brown, Jerry},
keywords = {ALASKA WETLAND ECOLOGY VEGETATION},
pages = {215},
publisher = {University of Alaska},
title = {{Biological papers of the ecological investigations of the tundra biome in the Prudhoe Bay Region, Alaska:}},
year = {1975}
}
@article{Brodeur1992,
abstract = {The dietary compositions of 18 species of pelagic nekton were examined$\backslash$nfrom purse seine collections made during 4 (1981 to 1984)$\backslash$noceanographically contrasting summers in the coastal upwelling zone off$\backslash$nOregon and Washington, USA. Euphausiids, hyperiid amphipods, decapod$\backslash$nlarvae, pteropods, and larval and juvenile fishes were the dominant prey$\backslash$ncategories consumed during all years by this assemblage of nekton,$\backslash$nalthough their relative proportions varied among the years. Considerable$\backslash$ndifferences were observed in food habits, diet overlap, and food web$\backslash$nstructure depending upon the prevailing oceanographic conditions. The$\backslash$nmoderate to low upwelling year of 1981 showed a generally high overall$\backslash$ntrophic diversity and a low level of diet overlap and food web$\backslash$ncomplexity. During the relatively strong upwelling year of 1982,$\backslash$neuphausiids were the dominant food consumed, resulting in high dietary$\backslash$noverlap and low trophic diversity and food web complexity. During the$\backslash$nwarm and low productivity El Nino year of 1983, marked changes were$\backslash$nobserved in the taxonomic composition of the diet of many species. The$\backslash$ndiets contained many species of southern origin leading to a high$\backslash$ndiversity of prey and low overall dietary overlap. A late occurrence of$\backslash$nstrong upwelling in 1984 resulted in a trophic diversity and overlap$\backslash$nthat were intermediate to the other years. Although some species preyed$\backslash$non species of lower trophic level during strong upwelling conditions,$\backslash$nthe overall trophic level was lowest during 1983 and 1984 due to the$\backslash$ninflux of large numbers of pelagic zooplanktivores.},
author = {Brodeur, R. D. and Pearcy, W. G.},
doi = {10.3354/meps084101},
isbn = {0171-8630},
issn = {01718630},
journal = {Marine Ecology Progress Series},
number = {2},
pages = {101--119},
pmid = {184},
title = {{Effects of environmental variability on trophic interactions and food web structure in a pelagic upwelling ecosystem}},
volume = {84},
year = {1992}
}
@phdthesis{Baril1983,
author = {Baril, Alain},
pages = {143},
school = {University of Ottawa},
title = {{The effect of the water mite, Piona constricta, on planktonic community structure.}},
year = {1983}
}
@article{Cornaby1974,
abstract = {The causes of suicidal behaviour are not fully understood; however, this behaviour clearly results from the complex interaction of many factors. Although many risk factors have been identified, they mostly do not account for why people try to end their lives. In this Review, we describe key recent developments in theoretical, clinical, and empirical psychological science about the emergence of suicidal thoughts and behaviours, and emphasise the central importance of psychological factors. Personality and individual differences, cognitive factors, social aspects, and negative life events are key contributors to suicidal behaviour. Most people struggling with suicidal thoughts and behaviours do not receive treatment. Some evidence suggests that different forms of cognitive and behavioural therapies can reduce the risk of suicide reattempt, but hardly any evidence about factors that protect against suicide is available. The development of innovative psychological and psychosocial treatments needs urgent attention.},
author = {O'Connor, Rory C. and Nock, Matthew K.},
doi = {10.1016/S2215-0366(14)70222-6},
isbn = {978-0-470-84959-0},
issn = {22150366},
journal = {The Lancet Psychiatry},
keywords = {Facebook groups,Higher education,Social networks,Structured learning,Web 2.0},
number = {1},
pages = {73--85},
pmid = {530},
title = {{The psychology of suicidal behaviour}},
volume = {1},
year = {2014}
}
@article{Baeta2011,
abstract = {Human-mediated and natural disturbances such as nutrient enrichment, habitat modification, and flood events often result in significant shifts in species composition and abundance that translate into changes in the food web structure. Six mass-balanced models were developed using the " Ecopath with Ecosim" software package to assess changes in benthic food web properties in the Mondego estuarine ecosystem (Portugal). Field, laboratory and literature information were used to construct the models. The main study objective was to assess at 2 sites (a Zostera meadow and a bare sediment area) the effects of: (1) a period of anthropogenic enrichment, which led to excessive production of organic matter in the form of algal blooms (1993/1994); (2) the implementation of mitigation measures, following a long period of eutrophication (1999/2000); and (3) a centenary flood (winter 2000/2001). Different numbers of compartments were identified at each site and in each time period. In general, the Zostera site, due to its complex community, showed a higher number of compartments and a higher level of system activity (i.e. sum of consumptions, respiration, flow to detritus, production, total system throughput, net primary production and system omnivory index). The differences at the two sites in the three time periods in the breakdown of throughput were mainly due to differences in the biomass of the primary producers (higher primary production at the Zostera site). Consumption, respiration and flow to detritus were dominated by the grazers Hydrobia ulvae and Scrobicularia plana at the Zostera and bare sediment sites respectively. At both sites, after recovery measures were implemented there was an increase in S. plana and Hediste diversicolor biomass, consumption, respiration and flows to detritus, and a decrease in H. ulvae biomass and associated flows, which increased again after the flood event. The mass-balanced models showed that the trophic structure of the benthic communities in Mondego estuary was affected differently by each disturbance event. Interestingly, in our study a high system throughput seems to be associated with higher stress levels, which contradicts the idea that higher system activity is always a sign of healthier conditions. {\textcopyright} 2010 Elsevier B.V.},
author = {Baeta, Alexandra and Niquil, Nathalie and Marques, Jo{\~{a}}o C. and Patr{\'{i}}cio, Joana},
doi = {10.1016/j.ecolmodel.2010.12.010},
isbn = {0304-3800},
issn = {03043800},
journal = {Ecological Modelling},
keywords = {Ecological model,Ecopath,Eutrophication,Flood,Food web,Management,Mondego estuary,Portugal},
number = {6},
pages = {1209--1221},
publisher = {Elsevier B.V.},
title = {{Modelling the effects of eutrophication, mitigation measures and an extreme flood event on estuarine benthic food webs}},
url = {http://dx.doi.org/10.1016/j.ecolmodel.2010.12.010},
volume = {222},
year = {2011}
}
@article{Beaver1972,
author = {Beaver, R A},
journal = {The Entomologist},
pages = {41--53},
title = {{Ecological studies on Diptera breeding in dead snails. 1. Biology of the species found in Cepaea nemoralis (L.)}},
volume = {105},
year = {1972}
}
@article{Beaver1979,
author = {Beaver, R.A.},
journal = {Malayan Nature Journal},
keywords = {Araneae,FAUNA,IN,Malaysia,Plant,West,foodweb,plants,spider},
number = {1},
pages = {1--10},
pmid = {11600004170},
title = {{Fauna and foodwebs of pitcher plants in west Malaysia}},
volume = {33},
year = {1979}
}
@incollection{Bindloss1972,
address = {Warsaw},
author = {Bindloss, M E and Holden, a. V and Bailey-Watts, a. E and Smith, I R},
booktitle = {Productivity problems of freshwaters},
editor = {Kajak, Z. and Hillbricht-Ilkowska, A.},
pages = {918},
publisher = {Polish Scientific},
title = {{Phytoplankton production, chemical and physical conditions in Loch Leven}},
year = {1972}
}
@article{Hawkins1984,
abstract = {Abstract. 1. Atriplex canescens (Pursh) Nuttall and A.polycarpa (Torrey) Watson (Chenopodiaceae) support twelve morphologically distinct gall types in southern California. Thirty-seven common species of parasitoids, predators and inquilines are associated with these galls.2. The galls incited by eight members of the Asphondylia atriplicis Cockerell (Diptera: Cecidomyiidae) species complex are linked into a single, interacting community through shared hymenopterous parasitoids and inquilines.3. Cluster analysis (UPGMA) grouped the fifteen most common species of Chalcidoidea into three host guilds of five species each: (1) specialists in tumour stem and blister leaf galls on A.canescens, (2) specialists in woolly stem galls on A.poiycarpa, and (3) generalists that attack all galls. Guild 1 dominated the galls with which it was primarily associated, while guild 3 dominated the remainder.4. The abundances of the parasitoids of the tumour stem and blister leaf galls were negatively correlated with the abundances of two organizer species, a gall-forming inquiline, Tetrastichus cecidobroter Gordh and Hawkins, and an internal, larval—pupal parasitoid, Tetrastichus sp. B. The abundances of nine of the twelve most common chalcidoids were not correlated with the abundances of all coaccurring species in six other galls.5. Host seasonality partly determines parasitoid population dynamics and guild structure. Parasitoid dominance increased with gall duration, suggesting that parasitoid competition depends on resource stability. The two continuously available galls were dominated by their specialist guild, while all seasonal galls were dominated by generalists. The subdominant specialists of woolly stem galls may represent competitively inferior species that utilize those galls opportunistically, because of the gall's widespread distribution and 9–10 month yearly availability.6. Sites in the Colorado Desert and chaparral that supported several gall types showed stable relative abundances of the major parasitoid species, whereas sites in the Mojave Desert that supported only woolly stem galls had unpredictable parasitoid species assemblages.7. The competitive success of Atriplex gall parasitoids may depend primarily on voltinism (multivoltine species dominated univoltine species) and mode of feeding (phytophagous, mixed entomophagous—phytophagous and facultatively hyperparasitic species in general dominated strict primary parasitoids).},
author = {HAWKINS, BRADFORD A. and GOEDEN, RICHARD D.},
doi = {10.1111/j.1365-2311.1984.tb00851.x},
isbn = {0307-6946},
issn = {13652311},
journal = {Ecological Entomology},
keywords = {Atriplex,community structure,gall parasitoids,guilds,phenology,resource stability.},
number = {3},
pages = {271--292},
title = {{Organization of a parasitoid community associated with a complex of galls on Atriplex spp. in southern California}},
url = {http://dx.doi.org/10.1111/j.1365-2311.1984.tb00851.x},
volume = {9},
year = {1984}
}
@incollection{Askew1975,
address = {New York},
author = {Askew, R R},
booktitle = {Evolutionary strategies of parasitic insects and mites},
doi = {10.1007/978-1-4615-8732-3_7},
editor = {Price, Peter W.},
keywords = {Parasitoid,host,parasitoid community},
pages = {130--153},
publisher = {Springer US},
title = {{The organization of chalcid-dominated parasitoid communities centered upon endophytic hosts}},
year = {1975}
}
@book{Cohen1990,
address = {Berlin},
author = {Steele, J. H.},
booktitle = {Science},
isbn = {0387511296},
issn = {00678821},
number = {4994},
pages = {686--687},
publisher = {Springer-Verlag},
title = {{Community Food Webs - Data and Theory - Cohen,Je, Briand,F, Newman,Cm}},
url = {http://books.google.com/books?id=gJJ4MQEACAAJ},
volume = {251},
year = {1991}
}
@article{Askew1961,
author = {Askew, Richard R},
doi = {Torymidae T},
journal = {Transactions of the Society for British entomology},
pages = {237--268},
title = {{On the biology of the inhabitants of oak galls of Cynipidae (Hymenoptera) in Britain}},
volume = {14},
year = {1961}
}
@article{Seifert1979,
abstract = {Abstract. An experimental study was designed to evaluate the importance of first-order species interactions, higher-order species interactions and habitat (flower bract) age on the survivorship of 4 species of insects living in the water-filled floral bracts of Heliconia bihai L. in a Venezuelan cloud forest. Only 3 out of 16 first-order species interactions were statistically significant and they included both competitive and symbiotic effects. A higher-order effect was found for only 1 of 4 species while habitat age was found to influence 3 of 4 species. The experimental results indicate that for 3 species survival is greater in the older habitats. These results correspond to data from field studies on non- experimental inflorescences in which insects were found most frequently in mature floral bracts. The results of this study are similar to those of an earlier study (Seifert and Seifert 1976a) on 2 species of Heliconia from lowland Costa Rica. We propose that Heliconia insect communities in general show low levels of 1st-order species interactions, some of which are symbiotic, and that higher-order species interactions are not a general component of these communities.},
author = {Seifert, Richard P. and Seifert, Florence Hammett},
doi = {10.2307/1936064},
isbn = {0012-9658},
issn = {00129658},
journal = {Ecology},
keywords = {competition,habitat age effects,heliconia,higher-,insects,jirst-order species interactions,order species interactions,symbiosis,venezuela},
number = {3},
pages = {462--467},
title = {{A Heliconia Insect Community in a Venezuelan Cloud Forest}},
url = {http://doi.wiley.com/10.2307/1936064},
volume = {60},
year = {1979}
}
@book{Yanez1978,
abstract = {1. Las lagunas costeras de Guerrero presentan un ciclo de fisiolog{\'{i}}a ambiental con tres periodos ecol{\'{o}}gicos anuales: Periodo 1 (normal, salinidades entre 15 y 34‰), agosto a noviembre, las lagunas se encuentran en contacto con el mar a trav{\'{e}}s de una boca en la barrera arenosa existiendo un intercambio biol{\'{o}}gico, f{\'{i}}sico y qu{\'{i}}mico. Periodo 2 (hipersalino, salinidades mayores de 35‰ noviembre a mayo, las lagunas se encuentran aisladas del mar y la evaporaci{\'{o}}n excede a los aportes de aguas dulces; m{\'{i}}nimo volumen de agua en las lagunas, Periodo 3 (hipo salino, salinidades menores de 15‰ mayo a agosto, las lagunas se encuentran aisladas del mar y los aportes de aguas dulces exceden a la tasa de evaporaci{\'{o}}n; m{\'{a}}ximo volumen de agua en las lagunas. 2. La vegetaci{\'{o}}n de las lagunas se compone de los siguientes elementos: a) palmar, b) lignetum pernnifolio de manglar, c) semiacu{\'{a}}tica, y d) acu{\'{a}}tica. Los principales productores primarios son los manglares y fitoplancton. 3. Seg{\'{u}}n estudios de otros autores la biomasa fitoplanct{\'{o}}nica es alta comparada con otros ecosistemas lagunares de M{\'{e}}xico. 4. Los grupos del zooplancton m{\'{a}}s comunes han sido cop{\'{e}}podos, quetognatos, larvas de gastr{\'{o}}podos, larvas de bivalvos, larvas de cirripedios, larvas de poliquetos, larvas de dec{\'{a}}podos, y algunas larvas y huevecillos de peces. Estudios de otros autores discuten la diversidad espec{\'{i}}fica y abundancia del zooplancton en las lagunas de Guerrero. 5. La macrofauna bent{\'{o}}nica tiene representantes de Porifera, Cnidaria, Annelida, Mollusca (bivalos y gastr{\'{o}}podos) en fondos blandos y duros, Arthropoda (Scalpellidae, Balanidae, Penaeidae, Palaemonidae, Callianasidae, Diogenidae, Coenobitidae, Portunidae, Xanthidae, Pinnotheridae, Grapsidae, Gecarcinidae, Ocypodidae, Isopoda y Tanaidacea) distribuidos en: a) facie de playa, b) facie de manglar, c) facie bent{\'{o}}nico pel{\'{a}}gico, y d) facie bent{\'{o}}nico de fondos arenosos y fangosos. En general el bentos est{\'{a}} pobremente representado y no caracteriza a las lagunas en particular. 6. Examinados 15 905 espec{\'{i}}menes de peces fueron determinados: 2 clases, 2 divisiones, 6 super{\'{o}}rdenes, 13 {\'{o}}rdenes, 22 sub{\'{o}}rdenes, 37 familias, 67 g{\'{e}}neros y 105 especies. Las familias mejor representadas en diversidad han sido Carangidae (11 especies), Sciaenidae (8), Gobidae (8), Gerridae (7), Urolophidae (6), Engraulidae (5), Poecilidae (5), y Pomadasyidae (4). 7. Zoogeogr{\'{a}}ficamente el sistema lagunar estudiado (37 familias, 67 g{\'{e}}neros y 105 especies) presenta una afinidad de "cero" especies con Canad{\'{a}}, 15{\%} con California, 31{\%}, con el Golfo de California, 45{\%} con Huizache-Caimanero Sinaloa, 48{\%}, con Agua Brava Nayarit, 60{\%} con el litoral de Guerrero, 67{\%} con el complejo tropical de Panam{\'{a}}-Colombia-Ecuador, 60{\%}, con el litoral de Per{\'{u}}, 47, con el litoral del norte de Chile, 16{\%} con las islas Gal{\'{a}}pagos, 15{\%} con la costa del Sur del Golfo de M{\'{e}}xico y el Caribe, 5{\%} con la ictiofauna continental de Sudam{\'{e}}rica, 2{\%} de especies introducidas y el 10{\%} son end{\'{e}}micas del Pac{\'{i}}fico de M{\'{e}}xico. Estos antecedentes refuerzan la hip{\'{o}}tesis que la ictiofauna costera del Pac{\'{i}}fico Oriental ha seguido ciertos patrones de radiaci{\'{o}}n limitados m{\'{a}}s bien por el avance en el grado de evoluci{\'{o}}n y adaptaci{\'{o}}n a las masas de aguas fr{\'{i}}o-temperadas, que por una capacidad de desplazamiento de los peces. 8. La diversidad var{\'{i}}a en el espacio y en el tiempo y muy pocas especies se encuentran presentes todo el a{\~{n}}o y a trav{\'{e}}s de todo el sistema lagunar, estas especies fueron 9 (8.5{\%}) y en n{\'{u}}mero de individuos 10 152 (64{\%}) de las colectas totales, i. e., Galeichthys caerulescens (3 396), Mugil curema (2 825), Diapterus peruvianus (1831), Lile stolifera (478), Dormitator latifrons (425) Mugil cephalus (402), Cichlasoma trimaculatum (384), Gerres cinereus (267) y Gobionellus microdon (145). 9. La din{\'{a}}mica ecol{\'{o}}gica del ambiente se refleja en la composici{\'{o}}n cuali y cuantitativa de las comunidades ictiofaun{\'{i}}sticas, las cuales, en todo el sistema lagunar, est{\'{a}}n constituidas por un 14{\%}, de peces dulceacu{\'{i}}colas, 6{\%} de peces marinos propiamente estuarinos, 28{\%} de peces marinos que utilizan el estuario como {\'{a}}reas de crianza, 31{\%} de peces marinos que utilizan el estuario como adultos y para alimentarse, y 21{\%} de peces marinos visitantes ocasionales de periodicidad ac{\'{i}}clica. Esta proporci{\'{o}}n var{\'{i}}a en las diferentes lagunas en su composici{\'{o}}n y abundancia relativa de especies de acuerdo a: a) las condiciones hidrol{\'{o}}gicas del sistema ecol{\'{o}}gico, consecuencia de b) la {\'{e}}poca del a{\~{n}}o y el per{\'{i}}odo ecol{\'{o}}gico de las lagunas, c) la localidad dentro del estuario y sus gradientes de salinidades, y d) la disponibilidad del alimento. 10. Examinados cuantitativamente 2 372 est{\'{o}}magos de los peces de importancia comercial (i. e., Elops affinis, Galeichthys caerulescens, Centropomus robalito, C. nigrescens, Caranx hippos, Lutjanus argentiventris, L. novemfasciatus, Gerres cinereus, Eugerres lineatus, Diapterus peruvianus, Mugil curema, Cichlasoma trimaculatum y Dormitator latifrons) y determinada cualitativamente la alimentaci{\'{o}}n del resto de la fauna ictiol{\'{o}}gica, se concluye que dentro de la trama tr{\'{o}}fica de los ecosistemas lagunares existen 3 categor{\'{i}}as ictiotr{\'{o}}ficas. 11. Las 3 categor{\'{i}}as ictiotr{\'{o}}ficas corresponden a: I. Consumidores Primarios donde se incluye a los peces a) Planct{\'{o}}fagos - fito y/o zooplanct{\'{o}}fagos-, b) Detrit{\'{i}}voros, y c) Omn{\'{i}}voros que se alimentan de peque{\~{n}}os organismos animales, detritus y vegetales. 2. Consumidores Secundarios donde se incluye a los peces predominantemente carn{\'{i}}voros que eventualmente pueden incorporar en su dieta algunos vegetales y detritus pero sin mucha significaci{\'{o}}n cuantitativa. 3. Consumidores de Tercer Orden donde se incluye a los peces exclusivamente carn{\'{i}}voros donde los vegetales y el detritus son un alimento accidental. 12. El espectro tr{\'{o}}fico de los peces puede sufrir modificaciones, dentro de un patr{\'{o}}n general, debido a: a) la disponibilidad del alimento, b) la edad del pez, c) la {\'{e}}poca del a{\~{n}}o y el periodo ecol{\'{o}}gico en que se encuentra la laguna, y d) el {\'{a}}rea particular dentro de la laguna. 13. Por afinidad ecol{\'{o}}gica las lagunas pueden ser reunidas en dos tipos. Grupo A en las cuales el ciclo de fisiolog{\'{i}}a ambiental afecta a toda la laguna, profundidades medias de 1 m, temperaturas de 29 a 35° C, salinidades de 2 a 125‰ alta biomasa fitoplanct{\'{o}}nica, variable cantidad de detritus, pocos manglares, variable biomasa macrobent{\'{o}}nica, estructura tr{\'{o}}fica y comunidades nect{\'{o}}nicas complejas en diversidad durante el periodo 1 simplific{\'{a}}ndose durante los periodos 2 y 3; s{\'{o}}lo un 15{\%} de peces presentes durante todo el a{\~{n}}o demuestra lo inestable del ambiente. Grupo B en las cuales el ciclo afecta s{\'{o}}lo una parte limitada de las lagunas, profundidades medias de 2 m, temperaturas de 29 a 33°C, salinidades de 0 a 4‰, muy alta biomasa fitoplanct{\'{o}}nica, grandes cantidades de detritus, numerosos manglares, casi ausencia de biomasa macrobent{\'{o}}nica, estructura tr{\'{o}}fica y comunidades nect{\'{o}}nicas de complejidad relativa durante el periodo 1 en la zona de influencia. marina y simples en el resto de la laguna durante ese periodo, as{\'{i}} como tambi{\'{e}}n en toda la superficie lagunar durante los periodos 2 y 3; un 55{\%} de peces presentes durante todo el a{\~{n}}o demuestra lo estable del ambiente. 14. Los manglares son los productores primarios m{\'{a}}s importantes en ambos grupos de lagunas, agregando adem{\'{a}}s al fitoplancton en las lagunas del Grupo B. A partir de ellos existen al menos cuatro v{\'{i}}as del flujo energ{\'{e}}tico hacia los heter{\'{o}}trofos: 1. Hojas de manglar y fitoplancton que proveen sustancias org{\'{a}}nicas disueltas --{\textgreater} microorganismos --{\textgreater} consumidores. 2. Sustancias org{\'{a}}nicas disueltas --{\textgreater} absorci{\'{o}}n por sedimentos y por detritus particulado ya existente --{\textgreater} consumidores. 3. Hojas y restos de hojas --{\textgreater} consumidores, 4. Hojas y restos de hojas --{\textgreater} bacterias y hongos --{\textgreater} consumidores. Esta {\'{u}}ltima es la m{\'{a}}s importante. 15. La diversidad ictiofaun{\'{i}}stica, la productividad pesquera de las lagunas, y la complejidad de la trama tr{\'{o}}fica, est{\'{a}}n en relaci{\'{o}}n directa con la influencia marina que reciben las lagunas durante el periodo 1. En las lagunas del Grupo A, durante el Periodo 1. La diversidad puede llegar a 70 especies, la densidad es de 12 ejem./19.5 m² y el "standing crop" de 66.7 g/m²; durante el periodo 2 la diversidad es de 18 especies, la densidad es de 8 ejem./m² y el "standing crop" de 44.3 g/m²; durante el periodo 3 la diversidad es de 22 especies, la densidad es de 7 ejem./19.5 m² y el "standing crop" de 38.8 g/m². En las lagunas del Grupo B, durante el periodo 1, la diversidad puede llegar a 26 especies, la densidad es de 14 ejem./19.5 m² y el "standing crop" de 43.1 g/m²; durante el periodo 2 la diversidad es de 19 especies la densidad es de 7 ejem./19.5 m² y el "standing crop" de 21.5 g/m²; durante el periodo 3 la diversidad es de 26 especies, la densidad es de 8 ejem./19.5 m² y el "standing crop" de 24.6 g/m². Las condiciones de salinidad determinan un reemplazo de especies dentro de los diferentes niveles tr{\'{o}}ficos, tanto intra como inter lagunar. 16. La diversidad ictiofaun{\'{i}}stica, en t{\'{e}}rminos de n{\'{u}}mero de especies (riqueza o variabilidad) y de {\'{i}}ndice (H'), est{\'{a}} regulada principalmente por los siguientes factores: 1. variedad de nichos, 2. tama{\~{n}}o o sobreposici{\'{o}}n de los nichos, 3. estabilidad del medioambiente, 4. rigurosidad del medioambiente, 5. sucesi{\'{o}}n, 6. productividad, 7. acumulaci{\'{o}}n de biomasa, 8. competencia, 9. espacio, 10. tama{\~{n}}o de los organismos, y 11. longitud de las cadenas alimenticias. Estos factores tienen un efecto particular en el medioambiente lagunar estuarino y se discuten en detalle. 17. El corolario ecol{\'{o}}gico es que, en los estuarios y lagunas costeras, los peces transforman energ{\'{i}}a desde fuentes primarias, conducen energ{\'{i}}a activamente a trav{\'{e}}s de la trama tr{\'{o}}fica, intercambian energ{\'{i}}a con ecosistemas vecinos a trav{\'{e}}s de importaci{\'{o}}n y exportaci{\'{o}}n de ella, representan una forma de almacenamiento de energ{\'{i}}a dentro del ecosistema y, finalmente, constituyen un agente de regulac},
address = {Ciudad Universitaria, México, D.F.},
author = {Y{\'{a}}{\~{n}}ez-Arancibia, Alejandro},
booktitle = {Publicaciones especiales - Instituto de Ciencias del Mar y Limnolog{\'{i}}a},
edition = {1st},
keywords = {lagunas costeras,peces},
pages = {225},
publisher = {Universidad Nacional Autónoma de México, Centro de Ciencias del Mar y Limnología},
title = {{Taxonom{\'{i}}a, ecolog{\'{i}}a y estructura de las comunidades de peces en lagunas costeras con bocas ef{\'{i}}meras del Pac{\'{i}}fico de M{\'{e}}xico: Taxonomy, ecology and structure of fish communities in coastal lagoons with ephemeral inlets on the Pacific Coast of Mexico}},
volume = {2},
year = {1980}
}
@article{Fetahi2011,
abstract = {We generated a mass-balance model to figure out the food web structure and trophic interactions of the major functional groups of the Ethiopian highland Lake Hayq. Moreover, the study lay down a baseline data for future ecosystem-based investigations and management activities. Extensive data collection has been taken place between October 2007 and May 2009. Ecotrophic efficiency (EE) of several functional groups including phytoplankton (0.8) and detritus (0.85) was high indicating the utilization of the groups within the system. However, the EE of Mesocyclops (0.03) and Thermocyclops (0.30) was very low implying these resources were rather a 'sink' in the trophic hierarchy. Flows based on aggregated trophic level sensu Lindeman revealed the importance of both phytoplankton and detritus to higher trophic levels. The computed average transfer efficiency of 11.5{\%} for the first four trophic levels was within the range for highly efficient African lakes. The primary production to respiration (P/. R) ratio (1.05) of Lake Hayq indicates the maturity of the ecosystem. We also modeled the food-web by excluding Tilapia and reduced phytoplankton biomass to get insight into the mass balance before Tilapia was introduced. The analysis resulted in a lower system omnivory index (SOI. = 0.016) and a reduced P/. R ratio (0.13) that described the lake as immature ecosystem, suggesting the introduction of Tilapia might have contributed to the maturity of the lake. Tilapia in Lake Hayq filled an ecological empty niche of pelagic planktivores, and contributed for the better transfer efficiency observed from primary production to fish yield. {\textcopyright} 2010 Elsevier B.V.},
author = {Fetahi, Tadesse and Schagerl, Michael and Mengistou, Seyoum and Libralato, Simone},
doi = {10.1016/j.ecolmodel.2010.09.038},
isbn = {0304-3800},
issn = {03043800},
journal = {Ecological Modelling},
keywords = {Ecopath,Food web,Mass balance,Trophic efficiency,Tropical lake},
number = {3},
pages = {804--813},
pmid = {2445625},
publisher = {Elsevier B.V.},
title = {{Food web structure and trophic interactions of the tropical highland lake Hayq, Ethiopia}},
url = {http://dx.doi.org/10.1016/j.ecolmodel.2010.09.038},
volume = {222},
year = {2011}
}
@article{Khan2009,
abstract = {A network model of trophic interactions in a tropical reservoir in India was developed with the objective to quantify matter and energy flows between system components and to study the impact of invasive fishes on the ecosystem. Structure of flows and their distribution within and between trophic levels were analysed by aggregating single flows into combined flows for discrete trophic levels. The trophic flows primarily occurred in the first four trophic level (TL) and the food web structure in this reservoir ecosystem was characterized by the dominance of low TL organisms, with the highest TL of only 3.57 for the top predator. Highest system omnivory index (SOI) was observed for indigenous catfishes (0.422), followed by the exotic fish Mozambique Tilapia (0.402). Nile Tilapia and Pearl spots show the highest niche overlap which suggests high competition for similar resources. The mixed trophic impact routine reveals that an increase in the abundance of the African catfish would negatively impact almost all fish groups such as Indian major carps, Pearl spots, indigenous catfishes and Tilapines. The other invasive fish Mozambique Tilapia adversely affects the indigenous catfishes. The most interesting observation in this study is that the most dominant invasive fish in this reservoir, the Nile Tilapia does not negatively impact any of the fish groups. In fact it positively impacts the Indian major carps. The direct and indirect effects of predation between system components (i.e. fish, invertebrates, phytoplankton and detritus) are quantitatively described and the possible influence and role in the ecosystem's functioning of the invasive fish species are discussed. {\textcopyright} 2009 Elsevier B.V. All rights reserved.},
author = {{Feroz Khan}, M. and Panikkar, Preetha},
doi = {10.1016/j.ecolmodel.2009.05.020},
isbn = {0304-3800},
issn = {03043800},
journal = {Ecological Modelling},
keywords = {Ascendancy,Invasive fishes,Mixed trophic impact,Modeling,Reservoir ecosystem},
number = {18},
pages = {2281--2290},
title = {{Assessment of impacts of invasive fishes on the food web structure and ecosystem properties of a tropical reservoir in India}},
volume = {220},
year = {2009}
}
@incollection{Burgis1972,
address = {Warsaw},
author = {Burgis, M J and Dunn, I G and Ganf, G G and McGowan, L M and Viner, a. B},
booktitle = {Productivity problems of freshwaters},
editor = {Kajak, Z. and Hillbricht-Ilkowska, A.},
pages = {918},
publisher = {Polish Scientific},
title = {{Lake George: Uganda: Studies on a tropical freshwater ecosystem.}},
year = {1972}
}
@article{Niering1963a,
author = {Niering, W A},
doi = {10.5479/si.00775630.49.1},
issn = {00129615},
journal = {Ecological Monographs},
keywords = {CAROLINE ISLANDS,ISLANDS,VEGETATION},
number = {2},
pages = {131--160},
title = {{Terrestrial ecology of Kapingamarangi Atoll, Caroline Islands}},
url = {http://www.jstor.org/stable/1948559},
volume = {33},
year = {1963}
}
@article{Moriarty1973,
abstract = {Experiments comparing television and print news have shown that children learn most from television, whereas adults learn most from print. An experiment was conducted in which both 96 children (5th and 6th graders) and 96 adults (university students) mere presented with a sequence of five news stories, either in their original televised form or in a printed version. Half of the participants were presented with stories taken from a children's news program (high audiovisual redundancy), whereas the other participants were exposed to corresponding stories adopted from an adult news program (low audiovisual redundancy). Results indicated that both children and adults learned most from television stories when presented in a children's news format, whereas the recall advantage of television disappeared when adult news stories were involved. The results suggest that the correspondence between verbal and visual content of television stories is decisive for the relative effectiveness of television and print.},
author = {Moriarty, D. J. W. and Darlington, J. P. E. C. and Dunn, I. G. and Moriarty, C. M. and Tevlin, M. P.},
doi = {10.1098/rspb.1973.0050},
isbn = {0962-8452},
issn = {0962-8452},
journal = {Proceedings of the Royal Society B: Biological Sciences},
number = {1076},
pages = {299--319},
pmid = {741417094},
title = {{Feeding and Grazing in Lake George, Uganda}},
url = {http://rspb.royalsocietypublishing.org/cgi/doi/10.1098/rspb.1973.0050},
volume = {184},
year = {1973}
}
@incollection{Shushkina1979,
address = {Cambridge},
author = {Petipa, T S},
booktitle = {Mar. Production Mechanisms},
chapter = {12},
editor = {Dunbar, M. J.},
pages = {233--250},
publisher = {Cambridge University Press},
title = {{Trophic relationships in communities and the functioning of marine ecosystems.}},
year = {1979}
}
@incollection{Winemiller2008,
abstract = {This chapter emphasizes the ecological responses of fishes to spatial and temporal variation in tropical stream habitats. It also discusses the steps to alleviate human impacts and restore degraded streams to protect the species of the fish. It reveals that freshwater fishes comprise 25{\%} of the vertebrate species on earth. The South and Central American rivers and streams contain the greatest number of species on Earth, with recent estimates ranging as high as 8000 and 25{\%} of global fish species richness. The highest fish species richness in the Neotropics is within the Amazon Basin. The ultimate environmental driver for variations in habitat quality and quantity, and hence breeding events, is probably stream hydrology. Tropical stream fishes occupy almost the entire spectrum of trophic niches that can occur in aquatic communities. Aquatic invertebrates are important food sources for stream fishes throughout the tropics. Terrestrial plants tissues, especially flowers, fruits, and seeds are very important resources for fishes in tropical streams. The primary threat to the ecological integrity of tropical streams and the long. -term survival of their fish faunas is degradation of watersheds by a variety of human activities. The largest impact is from deforestation and conversion of land to agriculture besides over-fishing, dam construction, use of pesticides and herbicides, and introductions of exotic species, etc. Protection of vegetation cover, maintenance of the integrity of riparian zones, and reductions in point. - and non. -point. -source pollution are essential components of conservation and management strategies for stream fishes. {\textcopyright} 2008 Elsevier Inc. All rights reserved.},
address = {London},
archivePrefix = {arXiv},
arxivId = {arXiv:0901.0931},
author = {Winemiller, Kirk O. and Agostinho, Angelo A. and Caramaschi, {\'{E}}rica Pellegrini},
booktitle = {Tropical Stream Ecology},
chapter = {5},
doi = {10.1016/B978-012088449-0.50007-8},
edition = {First},
editor = {Dudgeon, David},
eprint = {arXiv:0901.0931},
isbn = {9780120884490},
issn = {1573-4595},
pages = {107--146},
pmid = {18926558},
publisher = {Elsevier Inc.},
title = {{Fish ecology in tropical streams}},
url = {www.elsevier.com},
year = {2008}
}
@article{Carpenter1987,
author = {Carpenter, Author S R and Kitchell, James F and Hodgson, J R and Cochran, P a and Elser, J J and Lodge, D M and Kretchmer, D and He, X and Ende, C N Von and Ecology, Source and Dec, No},
journal = {Ecology},
keywords = {fish,food web,herbivory},
number = {6},
pages = {1863--1876},
title = {{Regulation of Lake Primary Productivity by Food Web Structure Published by : Ecological Society of America REGULATION OF LAKE PRIMARY PRODUCTIVITY BY FOOD WEB STRUCTURE '}},
volume = {68},
year = {1987}
}
@article{Hillebrand2004,
abstract = {Spatial coexistence depends on a variety of biological and physical processes, and the relative scales of these processes may promote or suppress coexistence. We model plant competition in a spatially varying environment to show how shifting scales of dis- persal, competition, and environmental heterogeneity affect coexis- tence. Spatial coexistence mechanisms are partitioned into three types: the storage effect, nonlinear competitive variance, and growth- density covariance. We first describe how the strength of each of these mechanisms depends on covariances between population den- sities and between population densities and the environment, and we then explain how changes in the scales of dispersal, competition, and environmental heterogeneity should affect these covariances.Our quantitative approach allows us to show how changes in the scales of biological and physical processes can shift the relative importance of different classes of spatial coexistence mechanisms and gives us a more complete understanding of how environmental heterogeneity can enable coexistence. For example, we show how environmental heterogeneity can promote coexistence even when competing species have identical responses to the environment. Keywords:},
archivePrefix = {arXiv},
arxivId = {http://www.jstor.org/stable/2462300},
author = {Weyori, Benjamin Asubam and Boateng, Kwame Osei and Twumasi-Ankrah, Sampson},
doi = {10.2307/2678832},
eprint = {/www.jstor.org/stable/2462300},
isbn = {1630130044},
issn = {13494198},
journal = {International Journal of Innovative Computing, Information and Control},
keywords = {Clustering,Distance measure,Impulse noise,Pixels,Switching filters,Vector filter},
number = {6},
pages = {1965--1980},
pmid = {17891731},
primaryClass = {http:},
title = {{Efficient novel vector median filter design for impulse noise suppression in color images}},
url = {http://www.jstor.org/stable/2678832?origin=crossref},
volume = {13},
year = {2017}
}
@article{Downing2002,
abstract = {Resolving current concerns about the role of biodiversity on ecosystems calls for understanding the separate roles of changes in species numbers and of composition. Recent work shows that primary productivity often, but not always, saturates with species richness within single trophic levels. However, any interpretation of such patterns must consider that variation in biodiversity is necessarily associated with changes in species composition (identity), and that changes in biodiversity often occur across multiple trophic levels. Here we present results from a mesocosm experiment in which we independently manipulated species richness and species composition across multiple trophic levels in pond food webs. In contrast to previous studies that focused on single trophic levels, we found that productivity is either idiosyncratic or increases with respect to species richness, and that richness influences trophic structure. However, the composition of species within richness levels can have equally or more marked effects on ecosystems than average effects of richness per se. Indirect evidence suggests that richness and associated changes in species composition affect ecosystem attributes through indirect effects and trophic interactions among species, features that are highly characteristic of natural, complex ecosystems.},
author = {Downing, Amy L. and Leibold, Mathew A.},
doi = {10.1038/416837a},
isbn = {0028-0836},
issn = {00280836},
journal = {Nature},
number = {6883},
pages = {837--841},
pmid = {11976680},
title = {{Ecosystem consequences of species richness and composition in pond food webs}},
volume = {416},
year = {2002}
}
@article{Magalhaes1993,
abstract = {Food resource use by seven cyprinids from an Iberian stream was analysed over 9 months. Differences in food resource use were found both between species and within species between seasons. Plant material was a more important food for carp, nase, goldfish and barbel than for gudgeon, roach and chub, irrespective of the season. Chironomid larvae were the staple animal food for the former five species throughout the year. Roach and chub, especially the latter, displayed large seasonal variations in prey use, with chironomid larvae only being important during autumn. Ephemerellid nymphs and ephemeropteran images dominated the animal diet of chub during spring and summer, respectively. Dipteran adults and Formicidae were the most important prey for roach during spring and summer, with other common prey being ephemerellid nymphs and hydropsychid larvae. Food resource overlap among the three dominant species (roach, barbel and chub) displayed a large seasonal variation. High overlaps were observed during autumn when these species used the same resources. During summer overlaps were much lower with each species specialising on different prey. The remaining less abundant species had large diet overlaps amongst themselves and with barbel, over all seasons. It is suggested that morphological constraints, habitat partitioning and temporal changes in food resource limitation may be involved in producing these patterns of food resource use.},
author = {Magalh{\~{a}}es, M. F.},
doi = {10.1007/BF00317739},
isbn = {0029-8549},
issn = {00298549},
journal = {Oecologia},
keywords = {Cyprinids,Feeding ecology,Iberian stream,Resource partitioning,Seasonal variability},
number = {2},
pages = {253--260},
pmid = {1406},
title = {{Feeding of an Iberian stream cyprinid assemblage: seasonality of resource use in a highly variable environment}},
volume = {96},
year = {1993}
}
@article{Winemiller2001,
abstract = {Three theories on food web patterns currently exist: 1. link-species scaling: describes an inverse relationship between community species richness and web connectance. Also two predator-prey interactions per species across all values of S is predicted (i.e. L/S is constant across values of S) 2. constant connectance: constant connectance C (=L/S{\^{}}2, L=observed links, S=possible links) and consequently an increasing number of predator-prey interactions with increasing number of species (S) (i.e. L/S increases with S) 3. species packing: average number of predator-prey interactions reduces with increasing number of species (i.e. L/S decreases with S)  Six independent datasets allow serious testing of number of species with C and diet breadth (relative importance of different food sources as determined from gut contents).   The relationships with species richness and mean diet breadth of the six cases show contrasting results. One case weakly support constant connectance, one support sprecies packing and four support none of the three theories. Taking a closer look at the fits, they conclude that in fact no theory is supported by the examples. Why?   In four of the 6 cases the L/S relations vary widely across S values and only two significant linear relations could be observed. The slope was both positive and as such support the constant connectance theory, but the number of link per consumer is uniformely underestimated. Aside from this, all L/S values greetly exceed the value of 2 as predicted by the link-species scaling law. The species packing rule is not supported.  The lack of congruence between theory and field data is probably the result of intersite and temporal variability.},
author = {Winemiller and Pianka and Vitt and Joern},
doi = {10.2307/3079163},
isbn = {0003-0147},
issn = {00030147},
journal = {The American Naturalist},
keywords = {assemblage,connectance,niche,predator-prey interaction},
number = {2},
pages = {193},
pmid = {18707347},
title = {{Food Web Laws or Niche Theory? Six Independent Empirical Tests}},
url = {http://www.jstor.org/stable/10.2307/3079163},
volume = {158},
year = {2001}
}
@article{Brown2004,
author = {Brown, James H. and Gillooly, James F. and Allen, Andrew P. and Savage, Van M. and West, Geoffrey B.},
doi = {10.1017/CBO9781107415324.004},
journal = {Climate Change 2013 - The Physical Science Basis},
keywords = {allometry,biogeochemical cycles,body size,development,ecological interactions,ecological theory,metabolism,population growth,production,stoichiometry,temperature,trophic},
number = {7},
pages = {1--30},
title = {{Summary for Policymakers}},
url = {https://www.cambridge.org/core/product/identifier/CBO9781107415324A009/type/book{\_}part},
volume = {85},
year = {2004}
}
@article{Davies2007,
abstract = {At small scales, areas with high native diversity are often resistant to invasion, while at large scales, areas with more native species harbor more exotic species, suggesting that different processes control the relationship between native and exotic species diversity at different spatial scales. Although the small-scale negative relationship between native and exotic diversity has a satisfactory explanation, we lack a mechanistic explanation for the change in relationship to positive at large scales. We investigated the native–exotic diversity relationship at three scales (range: 1–4000 km2) in California serpentine, a system with a wide range in the productivity of sites from harsh to lush. Native and exotic diversity were positively correlated at all three scales; it is rarer to detect a positive relationship at the small scales within which interactions between individuals occur. However, although positively correlated on average, the small-scale relationship between native and exotic diversity was p...},
author = {Davies, Kendi F. and Harrison, Susan and Safford, Hugh D. and Viers, Joshua H.},
doi = {10.1890/06-1907.1},
isbn = {0012-9658},
issn = {00129658},
journal = {Ecology},
keywords = {California grassland,Coexistence mechanisms,Competitive exclusion,Diversity-invasibility paradox,Heterogeneity,Invasion,Niche partitioning,Productivity,Serpentine},
number = {8},
pages = {1940--1947},
pmid = {17824424},
title = {{Productivity alters the scale dependence of the diversity-invasibility relationship}},
volume = {88},
year = {2007}
}
@article{Stegen2009,
abstract = {A component of metabolic scaling theory has worked towards understanding the influence of metabolism over the generation and maintenance of biodiversity. Specific models within this 'metabolic theory of biodiversity' (MTB) have addressed temperature gradients in speciation rate and species richness, but the scope of MTB has been questioned because of empirical departures from model predictions. In this study, we first show that a generalized MTB is not inconsistent with empirical patterns and subsequently implement an eco-evolutionary MTB which has thus far only been discussed qualitatively. More specifically, we combine a functional trait (body mass) approach and an environmental gradient (temperature) with a dynamic eco-evolutionary model that builds on the current MTB. Our approach uniquely accounts for feedbacks between ecological interactions (size-dependent competition and predation) and evolutionary rates (speciation and extinction). We investigate a simple example in which temperature influences mutation rate, and show that this single effect leads to dynamic temperature gradients in macroevolutionary rates and community structure. Early in community evolution, temperature strongly influences speciation and both speciation and extinction strongly influence species richness. Through time, niche structure evolves, speciation and extinction rates fall, and species richness becomes increasingly independent of temperature. However, significant temperature-richness gradients may persist within emergent functional (trophic) groups, especially when niche breadths are wide. Thus, there is a strong signal of both history and ecological interactions on patterns of species richness across temperature gradients. More generally, the successful implementation of an eco-evolutionary MTB opens the perspective that a process-based MTB can continue to emerge through further development of metabolic models that are explicit in terms of functional traits and environmental gradients.},
author = {Stegen, James C. and Enquist, Brian J. and Ferriere, Regis},
doi = {10.1111/j.1461-0248.2009.01358.x},
isbn = {1461-0248},
issn = {1461023X},
journal = {Ecology Letters},
keywords = {Adaptive dynamics,Body size,Community assembly,Diversification,Eco-evolutionary feedbacks,Food webs,Interaction networks,Metabolism,Mutation rate},
number = {10},
pages = {1001--1015},
pmid = {19747180},
title = {{Advancing the metabolic theory of biodiversity}},
volume = {12},
year = {2009}
}
@article{Field1998,
abstract = {Integrating conceptually similar models of the growth of marine and terrestrial primary producers yielded an estimated global net primary production (NPP) of 104.9 petagrams of carbon per year, with roughly equal contributions from land and oceans. Approaches based on satellite indices of absorbed solar radiation indicate marked heterogeneity in NPP for both land and oceans, reflecting the influence of physical and ecological processes. The spatial and temporal distributions of ocean NPP are consistent with primary limitation by light, nutrients, and temperature. On land, water limitation imposes additional constraints. On land and ocean, progressive changes in NPP can result in altered carbon storage, although contrasts in mechanisms of carbon storage and rates of organic matter turnover result in a range of relations between carbon storage and changes in NPP.},
archivePrefix = {arXiv},
arxivId = {1011.1669},
author = {Field, Christopher B. and Behrenfeld, Michael J. and Randerson, James T. and Falkowski, Paul},
doi = {10.1126/science.281.5374.237},
eprint = {1011.1669},
isbn = {00368075},
issn = {00368075},
journal = {Science},
number = {5374},
pages = {237--240},
pmid = {9657713},
title = {{Primary production of the biosphere: Integrating terrestrial and oceanic components}},
volume = {281},
year = {1998}
}
@article{Jacobs2015,
archivePrefix = {arXiv},
arxivId = {arXiv:1505.04741v1},
author = {Jacobs, Abigail Z and Dunne, Jennifer a and Moore, Cris and Clauset, Aaron},
eprint = {arXiv:1505.04741v1},
journal = {bioRxiv},
pages = {1--17},
title = {{Untangling the roles of parasites in food webs with generative network models}},
year = {2015}
}
@misc{GoogleEarth,
abstract = {Google Earth 7.1.2.2041 Data da compila{\c{c}}{\~{a}}o 10/7/2013},
author = {GoogleEarth{\textcopyright}},
isbn = {9780470095287},
issn = {2329549X},
publisher = {Google Inc.},
title = {{Google Earth}},
url = {https://www.google.co.uk/maps/place/cyprus},
year = {2013}
}
@article{Marra1997,
abstract = {--For almost three decades, structural habitat complexity has been regarded as a primary ecological factor responsible for maintaining high species diversity of trop-ical bird communities. However, differences in habitat complexity between temperate and tropical forests have not been documented sufficiently. Differences between tem-perate and tropical forests were quantified by measuring 36 variables related to under-story habitat structure. Structural components of temperate and tropical habitats differed significantly in several features. However, no differences were found in overall habitat heterogeneity (i.e. complexity) between temperate and tropical forests. To search for "bird-related" factors important in maintaining high tropical species diversity, we com-pared habitat selection between foliage-gleaning insectivorous birds in temperate and tropical forest understories. The tropical species were more specialized in horizontal and vertical habitat selection, and had lower "niche breadths" in foraging substrate and in foraging height. Tropical species also showed less interspecific overlap in most foraging variables than did temperate species. Therefore, higher species diversity, at least within this guild of birds in tropical and temperate forest understories, can be attributed at the proximate level to greater specialization and "tighter species packing," and may be more independent of greater habitat complexity than previously thought. RESUMEN.--Por casi tres decadas, la complejidad estmctural del habitat se ha consi-derado como el principal factor ecologico responsable de mantener la alta variedad de especies encontrada en comunidades de pajaros tropicales. Sin embargo, la diferencia en complejidad del habitat entre bosques templados y bosques tropicales no ha sido sufi-cientemente quantificada. Nosotros quantificamos diferencias entre bosques templados y bosques tropicales midiendo 36 variables relacionadas a la estructura del habitat bajo el dosel y encontramos varias diferencias significativas en componentes estructurales entre habitats tropicales y templados. Sin embargo, no encontramos diferencias en la hetero-geneidad del habitat (complejidad) entre bosques templados y tropicales. Buscando fac-tores que rueran importantes en mantenet la alta diversidad de las especies de pajaros tropicales, comparamos la seleccion del habitat entre pajaros insectivoros que usan el follaje bajo el dosel en bosques templados y tropicales. Las especies tropicales estaban mas especializadas en su seleccion de habitats al nivel horizontal y vertical, teniendo pequenos nichos en el substrato y altitud del area donde forrajeaban. En la mayoria de las variables relacionadas al forraje, las especies tropicales tambien demonstraron menos traslapo entre-especies que las especies templadas. Concluimos pues que la alta variedad de especies, pot lo menos dentro de este grupo de pajaros que viven bajo el dosel en bosques templados y tropicales, se puede atribuir inicialmente a mayor especializacion y concentracion de especies y puede que sea mas independiente de la total complejidad del habitat de lo que actualmente se cree.},
author = {Marra, Peter P. and Remsen, James Van Jr.},
doi = {10.2307/40157547},
isbn = {0078-6594},
issn = {00786594},
journal = {Ornithological Monographs},
number = {48},
pages = {445--483},
title = {{Insights Into the Maintenance of High Species Diversity in the Neotropics: Habitat Selection and Foraging Behavior in Understory Birds of Tropical and Temperate Forests}},
volume = {48},
year = {1997}
}
@article{OConnor2009,
abstract = {Climate change disrupts ecological systems in many ways. Many documented responses depend on species' life histories, contributing to the view that climate change effects are important but difficult to characterize generally. However, systematic variation in metabolic effects of temperature across trophic levels suggests that warming may lead to predictable shifts in food web structure and productivity. We experimentally tested the effects of warming on food web structure and productivity under two resource supply scenarios. Consistent with predictions based on universal metabolic responses to temperature, we found that warming strengthened consumer control of primary production when resources were augmented. Warming shifted food web structure and reduced total biomass despite increases in primary productivity in a marine food web. In contrast, at lower resource levels, food web production was constrained at all temperatures. These results demonstrate that small temperature changes could dramatically shift food web dynamics and provide a general, species-independent mechanism for ecological response to environmental temperature change.},
author = {O'Connor, Mary I. and Piehler, Michael F. and Leech, Dina M. and Anton, Andrea and Bruno, John F.},
doi = {10.1371/journal.pbio.1000178},
isbn = {1545-7885},
issn = {15449173},
journal = {PLoS Biology},
number = {8},
pages = {3--8},
pmid = {19707271},
title = {{Warming and resource availability shift food web structure and metabolism}},
volume = {7},
year = {2009}
}
@article{Wilhelm1999,
abstract = {We investigated the seasonal diet of a native, undisturbed population$\backslash$nof bull trout Salvelinus confluentus in an alpine lake to examine$\backslash$npredation patterns between fish size-classes and in relation to available$\backslash$ninvertebrate prey. The diets of small (ltoreq250 mm in fork length,$\backslash$nFL) and large ({\textgreater}250 mm FL) bull trout were similar. Bull trout fed$\backslash$non seasonally abundant prey species. After ice-out in July, the diet$\backslash$nwas dominated by chironomid pupae. Daphnia pulex var. and the amphipod$\backslash$nGammarus lacustris dominated the diet in August and September. Both$\backslash$nDaphnia and Gammarus reproduced before bull trout switched to preying$\backslash$non them in early August. Bull trout fed size-selectively on large$\backslash$nindividuals of both Daphnia and Gammarus. Large bull trout preyed$\backslash$non larger Daphnia than did small bull trout. Fish of both size-classes$\backslash$nconsumed large Gammarus. Bull trout were spatially segregated; small$\backslash$nfish occupied shallow water ({\textless}1 m deep), while large fish occupied$\backslash$nthe profundal offshore zone. Spatial segregation prevented small$\backslash$nbull trout from cropping small immature Daphnia in offshore areas.$\backslash$nAverage total food volume in stomachs of small fish increased between$\backslash$nJuly and September whereas it decreased in large fish. The latter$\backslash$nwere frequently emaciated, indicating that large individuals may$\backslash$nbe food limited for much of the open-water period. Our data and observations$\backslash$nsuggest that prey switching, timing of prey reproduction, and spatial$\backslash$nsegregation of the fish population by size are tightly coupled and$\backslash$ncontribute to the survival of the key prey species. The survival$\backslash$nof a variety of invertebrate species, including large Gammarus, in$\backslash$nthe presence of bull trout suggests that stocks of this fish species$\backslash$ncould be increased by stocking small mountain lakes without severely$\backslash$naffecting the native invertebrate fauna.},
author = {Wilhelm, Frank M. and Parker, Brian R. and Schindler, David W. and Donald, David B.},
doi = {10.1577/1548-8659(1999)128<1176:SFHOBT>2.0.CO;2},
isbn = {0002-8487},
issn = {0002-8487},
journal = {Transactions of the American Fisheries Society},
number = {6},
pages = {1176--1192},
title = {{Seasonal Food Habits of Bull Trout from a Small Alpine Lake in the Canadian Rocky Mountains}},
volume = {128},
year = {1999}
}
@article{Currie2004,
abstract = {Broad-scale variation in taxonomic richness is strongly correlated with climate. Many mechanisms have been hypothesized to explain these patterns; however, testable predictions that would distinguish among them have rarely been derived. Here, we examine several prominent hypotheses for climate–richness relationships, deriving and testing predictions based on their hypothesized mechanisms. The ?energy–richness hypothesis? (also called the ?more individuals hypothesis? ) postulates that more productive areas have more individuals and therefore more species. More productive areas do often have more species, but extant data are not consistent with the expected causal relationship from energy to numbers of individuals to numbers of species. We reject the energy–richness hypothesis in its standard form and consider some proposed modifications. The ?physiological tolerance hypothesis? postulates that richness varies according to the tolerances of individual species for different sets of climatic conditions. This hypothesis predicts that more combinations of physiological parameters can survive under warm and wet than cold or dry conditions. Data are qualitatively consistent with this prediction, but are inconsistent with the prediction that species should fill climatically suitable areas. Finally, the ?speciation rate hypothesis? postulates that speciation rates should vary with climate, due either to faster evolutionary rates or stronger biotic interactions increasing the opportunity for evolutionary diversification in some regions. The biotic interactions mechanism also has the potential to amplify shallower, underlying gradients in richness. Tests of speciation rate hypotheses are few (to date), and their results are mixed.},
author = {Currie, David J. and Mittelbach, Gary G. and Cornell, Howard V. and Field, Richard and Gu{\'{e}}gan, Jean Francois and Hawkins, Bradford a. and Kaufman, Dawn M. and Kerr, Jeremy T. and Oberdorff, Thierry and O'Brien, Eileen and Turner, J. R G},
doi = {10.1111/j.1461-0248.2004.00671.x},
isbn = {1461-023X},
issn = {1461023X},
journal = {Ecology Letters},
keywords = {Climatic gradients,Latitudinal gradients,Productivity,Speciation,Species richness,Species-energy theory},
number = {12},
pages = {1121--1134},
pmid = {231},
title = {{Predictions and tests of climate-based hypotheses of broad-scale variation in taxonomic richness}},
volume = {7},
year = {2004}
}
@article{White2007,
abstract = {A classic example of ecophysiological adaptation is the observation that animals from hot arid environments have lower basal metabolic rates (BMRs, ml O2min-1) than those from non-arid (luxuriant) ones. However, the term 'arid' conceals within it a multitude of characteristics including extreme ambient temperatures (Ta, degrees C) and low annual net primary productivities (NPPs, gCm-2), both of which have been shown to correlate with BMR. To assess the relationship between environmental characteristics and metabolic rate in birds, we collated BMR measurements for 92 populations representing 90 wild-caught species and examined the relationships between BMR and NPP, Ta, annual temperature range (Tr), precipitation and intra-annual coefficient of variation of precipitation (PCV). Using conventional non-phylogenetic and phylogenetic generalized least-squares approaches, we found no support for a relationship between BMR and NPP, despite including species captured throughout the world in environments spanning a 35-fold range in NPP. Instead, BMR was negatively associated with Ta and Tr, and positively associated with PCV.},
author = {White, Craig R and Blackburn, Tim M and Martin, Graham R and Butler, Patrick J},
doi = {10.1098/rspb.2006.3727},
isbn = {0962-8452 (Print)$\backslash$r0962-8452 (Linking)},
issn = {0962-8452},
journal = {Proceedings. Biological sciences / The Royal Society},
keywords = {allometry,aridity,basal metabolic rate,precipitation,productivity,temperature},
number = {1607},
pages = {287--293},
pmid = {17148258},
title = {{Basal metabolic rate of birds is associated with habitat temperature and precipitation, not primary productivity.}},
volume = {274},
year = {2007}
}
@article{Dyer2007,
abstract = {For numerous taxa, species richness is much higher in tropical than in temperate zone habitats. A major challenge in community ecology and evolutionary biogeography is to reveal the mechanisms underlying these differences. For herbivorous insects, one such mechanism leading to an increased number of species in a given locale could be increased ecological specialization, resulting in a greater proportion of insect species occupying narrow niches within a community. We tested this hypothesis by comparing host specialization in larval Lepidoptera (moths and butterflies) at eight different New World forest sites ranging in latitude from 15 degrees S to 55 degrees N. Here we show that larval diets of tropical Lepidoptera are more specialized than those of their temperate forest counterparts: tropical species on average feed on fewer plant species, genera and families than do temperate caterpillars. This result holds true whether calculated per lepidopteran family or for a caterpillar assemblage as a whole. As a result, there is greater turnover in caterpillar species composition (greater beta diversity) between tree species in tropical faunas than in temperate faunas. We suggest that greater specialization in tropical faunas is the result of differences in trophic interactions; for example, there are more distinct plant secondary chemical profiles from one tree species to the next in tropical forests than in temperate forests as well as more diverse and chronic pressures from natural enemy communities.},
author = {Dyer, L a and Singer, M S and Lill, J T and Stireman, J O and Gentry, G L and Marquis, R J and Ricklefs, R E and Greeney, H F and Wagner, D L and Morais, H C and Diniz, I R and Kursar, T a and Coley, P D},
doi = {10.1038/nature05884},
isbn = {0028-0836},
issn = {0028-0836},
journal = {Nature},
number = {7154},
pages = {696--699},
pmid = {17687325},
title = {{Host specificity of Lepidoptera in tropical and temperate forests.}},
volume = {448},
year = {2007}
}
@article{Isaac2012,
abstract = {Mysis diluviana is an important prey item to the Lake Superior fish community as found through a recent diet study. We further evaluated this by relating the quantity of prey found in fish diets to the quantity of prey available to fish, providing insight into feeding behavior and prey preferences. We describe the seasonal prey selection of major fish species collected across 18 stations in Lake Superior in spring, summer, and fall of 2005. Of the major nearshore fish species, bloater (. Coregonus hoyi), rainbow smelt (. Osmerus mordax), and lake whitefish (. Coregonus clupeaformis) consumed . Mysis, and strongly selected . Mysis over other prey items each season. However, lake whitefish also selected . Bythotrephes in the fall when . Bythotrephes were numerous. Cisco (. Coregonus artedi), a major nearshore and offshore species, fed largely on calanoid copepods, and selected calanoid copepods (spring) and . Bythotrephes (summer and fall). Cisco also targeted prey similarly across bathymetric depths. Other major offshore fish species such as kiyi (. Coregonus kiyi) and deepwater sculpin (. Myoxocephalus thompsoni) fed largely on . Mysis, with kiyi targeting . Mysis exclusively while deepwater sculpin did not prefer any single prey organism. The major offshore predator siscowet lake trout (. Salvelinus namaycush siscowet) consumed deepwater sculpin and coregonines, but selected deepwater sculpin and . Mysis each season, with juveniles having a higher selection for . Mysis than adults. Our results suggest that . Mysis is not only a commonly consumed prey item, but a highly preferred prey item for pelagic, benthic, and piscivorous fishes in nearshore and offshore waters of Lake Superior. {\textcopyright} 2012 Elsevier B.V.},
author = {Isaac, Edmund J. and Hrabik, Thomas R. and Stockwell, Jason D. and Gamble, Allison E.},
doi = {10.1016/j.jglr.2012.02.017},
issn = {03801330},
journal = {Journal of Great Lakes Research},
keywords = {Fish community,Food web,Lake Superior,Mysis,Prey selection},
number = {2},
pages = {326--335},
publisher = {Elsevier B.V.},
title = {{Prey selection by the Lake Superior fish community}},
url = {http://dx.doi.org/10.1016/j.jglr.2012.02.017},
volume = {38},
year = {2012}
}
@article{Baker2015,
abstract = {Ecological communities are composed of many species and an intricate network of interactions between them. Because of their overall complexity, an intriguing approach to understanding network structure is by breaking it down into the structural roles of its constituent species. The structural role of a species can be directly measured based on how it appears in network motifs - the basic building blocks of complex networks. Here, we study the distribution of species' roles at three distinct spatio-temporal scales (i.e. species, network, and temporal) in host-parasitoid networks collected across 22 sites over two years within a fragmented landscape of oaks in southern Finland. We found that species' roles for hosts and parasitoids were heterogeneously distributed across the study system but that roles are strongly conserved over spatial scales. In addition, we found that species' roles were remarkably consistent between years even in the presence of disturbances (e.g. species turnover). Overall, our results suggest that species' roles are an intrinsic property of species that may be predictable over spatial and temporal scales.},
author = {Baker, Nick J. and Kaartinen, Riikka and Roslin, Tomas and Stouffer, Daniel B.},
doi = {10.1111/ecog.00913},
file = {:Users/alyssacirtwill/Documents/Papers/Baker et al.{\_}2015{\_}Ecography.pdf:pdf},
isbn = {1600-0587},
issn = {16000587},
journal = {Ecography},
number = {2},
pages = {130--139},
title = {{Species' roles in food webs show fidelity across a highly variable oak forest}},
url = {http://doi.wiley.com/10.1111/ecog.00913},
volume = {38},
year = {2015}
}
@article{Stanley1999,
abstract = {This brief overview is designed to introduce some of the advances that have occurred in our understanding of phase transitions and critical phenomena. The presentation is organized around three simple questions: (i) What are the basic phenomena under consideration? (ii) Why do we care? (iii) What do we actually do? To answer the third question, the author shall briefly review scaling, universality, and renormalization, three of the many important themes which have served to provide the framework of much of our current understanding of critical phenomena. The style is that of a colloquium, not that of a mini-review article. [S0034-6861 (99)02902-5].},
author = {Stanley, H.Eugene},
doi = {10.1103/RevModPhys.71.S358},
isbn = {0034-6861},
issn = {0034-6861},
journal = {Reviews of Modern Physics},
number = {2},
pages = {S358--S366},
pmid = {78900800042},
title = {{Scaling, universality, and renormalization: Three pillars of modern critical phenomena}},
url = {http://link.aps.org/doi/10.1103/RevModPhys.71.S358},
volume = {71},
year = {1999}
}
@article{Morelli2015,
abstract = {Traditionally, the niche of a species is described as a hypothetical 3D space, constituted by well-known biotic interactions (e.g. predation, competition, trophic relationships, resource–consumer interactions, etc.) and various abiotic environmental factors. Species distribution models (SDMs), also called “niche models” and often used to predict wildlife distribution at landscape scale, are typically constructed using abiotic factors with biotic interactions generally been ignored. Here, we compared the goodness of fit of SDMs for red-backed shrike Lanius collurio in farmlands of Western Poland, using both the classical approach (modeled only on environmental variables) and the approach which included also other potentially associated bird species. The potential associations among species were derived from the relevant ecological literature and by a correlation matrix of occurrences. Our findings highlight the importance of including heterospecific interactions in improving our understanding of niche occupation for bird species. We suggest that suite of measures currently used to quantify realized species niches could be improved by also considering the occurrence of certain associated species. Then, an hypothetical “species 1” can use the occurrence of a successfully established individual of “species 2” as indicator or “trace” of the location of available suitable habitat to breed. We hypothesize this kind of biotic interaction as the “heterospecific trace effect” (HTE): an interaction based on the availability and use of “public information” provided by individuals from different species. Finally, we discuss about the incomes of biotic interactions for enhancing the predictive capacities on species distribution models.},
author = {Morelli, Federico and Tryjanowski, Piotr},
doi = {10.1002/ece3.1387},
isbn = {2045-7758},
issn = {20457758},
journal = {Ecology and Evolution},
keywords = {Avian niche,Biotic interactions,Red-backed shrike,Species association,Species distribution models},
month = {jan},
number = {3},
pages = {759--768},
title = {{No species is an island: Testing the effects of biotic interactions on models of avian niche occupation}},
url = {http://doi.wiley.com/10.1002/ece3.1387},
volume = {5},
year = {2015}
}
@article{Nakagawa2013,
abstract = {* The use of both linear and generalized linear mixed-effects models (LMMs and GLMMs) has become popular not only in social and medical sciences, but also in biological sciences, especially in the field of ecology and evolution. Information criteria, such as Akaike Information Criterion (AIC), are usually presented as model comparison tools for mixed-effects models. * The presentation of ‘variance explained' (R2) as a relevant summarizing statistic of mixed-effects models, however, is rare, even though R2 is routinely reported for linear models (LMs) and also generalized linear models (GLMs). R2 has the extremely useful property of providing an absolute value for the goodness-of-fit of a model, which cannot be given by the information criteria. As a summary statistic that describes the amount of variance explained, R2 can also be a quantity of biological interest. * One reason for the under-appreciation of R2 for mixed-effects models lies in the fact that R2 can be defined in a number of ways. Furthermore, most definitions of R2 for mixed-effects have theoretical problems (e.g. decreased or negative R2 values in larger models) and/or their use is hindered by practical difficulties (e.g. implementation). * Here, we make a case for the importance of reporting R2 for mixed-effects models. We first provide the common definitions of R2 for LMs and GLMs and discuss the key problems associated with calculating R2 for mixed-effects models. We then recommend a general and simple method for calculating two types of R2 (marginal and conditional R2) for both LMMs and GLMMs, which are less susceptible to common problems. * This method is illustrated by examples and can be widely employed by researchers in any fields of research, regardless of software packages used for fitting mixed-effects models. The proposed method has the potential to facilitate the presentation of R2 for a wide range of circumstances.},
archivePrefix = {arXiv},
arxivId = {2746},
author = {Nakagawa, Shinichi and Schielzeth, Holger},
doi = {10.1111/j.2041-210x.2012.00261.x},
editor = {O'Hara, Robert B.},
eprint = {2746},
isbn = {2041-210X},
issn = {2041210X},
journal = {Methods in Ecology and Evolution},
keywords = {Coefficient of determination,Goodness-of-fit,Heritability,Information criteria,Intra-class correlation,Linear models,Model fit,Repeatability,Variance explained},
month = {feb},
number = {2},
pages = {133--142},
title = {{A general and simple method for obtaining R2 from generalized linear mixed-effects models}},
url = {http://doi.wiley.com/10.1111/j.2041-210x.2012.00261.x},
volume = {4},
year = {2013}
}
@incollection{Futuyama1983,
address = {Sunderland},
author = {Futuyama, D. J.},
booktitle = {Coevolution},
chapter = {10},
editor = {Futuyama, Douglas J. and Slatkin, Montgomery},
keywords = {Herbivory Defence-substance Secondary-substance},
pages = {207--231},
publisher = {Sinauer Associates Inc.},
title = {{Evolutionary interactions among herbivorous insects and plants.}},
year = {1983}
}
@incollection{Feinsinger1983,
address = {Sunderland},
author = {Feinsinger, P.},
booktitle = {Coevolution},
chapter = {13},
editor = {Futuyama, Douglas J. and Slatkin, Montgomery},
pages = {282--310},
publisher = {Sinauer Associates Inc.},
title = {{Coevolution and pollination}},
url = {http://www.citeulike.org/group/894/article/521587},
year = {1983}
}
@incollection{Mitter1983,
abstract = {The human being is not only characterized by the ability of abstract thinking, upright walking and speaking, but also by a strong and differentiate functional asymmetry. The great majority are right-handers, showing a left-cerebrally dominance for language and praxis. It is very important to know the period when the human laterality of functions is evolved. Human beings of all races and cultures appear to have been predominantly right-handed since the earliest records were made. The time when right-handedness emerged in human evolution is a matter of some dispute. Because of poor record-keeping in prehistoric times, statements about the evolution of laterality in human beings must be based largely on speculation. The present review summarizes the prehistoric data concerning handedness and cerebral laterality. It is confirmed that laterality is not evolved in the last thousand years but it seems that lateralization of functions is the result of a protracted process elapsing perhaps several million years ago. Moreover this paper discusses the genesis and reasons of laterality.},
address = {Sunderland},
author = {Reiss, M},
booktitle = {Anthropol Anz},
chapter = {4},
editor = {Futuyama, Douglas J. and Slatkin, Montgomery},
keywords = {*Phylogeny,Animal,Anthropology, Physical/history,English Abstract,History of Medicine, 20th Cent.,Human,Laterality/*physiology},
number = {1},
pages = {81--90},
pmid = {9569983},
publisher = {Sinauer Associates Inc.},
title = {{[Phylogenetic aspects of laterality]}},
url = {http://www.ncbi.nlm.nih.gov/entrez/query.fcgi?cmd=Retrieve{\&}amp{\%}5Cndb=PubMed{\&}amp{\%}5Cndopt=Citation{\&}amp{\%}5Cnlist{\_}uids=9569983},
volume = {56},
year = {1998}
}
@book{Mayr2001,
abstract = {At once a spirited defense of Darwinian explanations of biology and an elegant primer on evolution for the general reader, What Evolution Is poses the questions at the heart of evolutionary theory and considers how our improved understanding of evolution has affected the viewpoints and values of modern man. Science Masters Series},
address = {New York},
author = {Mayr, Ernst},
isbn = {0465044263},
keywords = {Coevolution - p.210-211},
mendeley-tags = {Coevolution - p.210-211},
pages = {318},
pmid = {12318604},
publisher = {Basic Books},
title = {{What Evolution is}},
url = {https://books.google.com/books?id=i8jx-ZyRRkkC{\&}pgis=1},
year = {2001}
}
@article{Guimera2005,
abstract = {Integrative approaches to the study of complex systems demand that one knows the manner in which the parts comprising the system are connected. The structure of the complex network defining the interactions provides insight into the function and evolution of the components of the system. Unfortunately, the large size and intricacy of these networks implies that such insight is usually difficult to extract. Here, we propose a method that allows one to systematically extract and display information contained in complex networks. Specifically, we demonstrate that one can (i) find modules in complex networks and (ii) classify nodes into universal roles according to their pattern of within- and between-module connections. The method thus yields a 'cartographic representation' of complex networks.},
author = {Guimer{\`{a}}, Roger and Amaral, Lu{\'{i}}s A. Nunes},
doi = {10.1088/1742-5468/2005/02/P02001},
isbn = {1742-5468},
issn = {1742-5468},
journal = {Journal of Statistical Mechanics},
month = {feb},
pages = {P02001},
pmid = {18159217},
title = {{Cartography of complex networks: modules and universal roles.}},
url = {http://www.pubmedcentral.nih.gov/articlerender.fcgi?artid=2151742{\&}tool=pmcentrez{\&}rendertype=abstract},
volume = {2},
year = {2005}
}
@article{Wikstrom2001,
abstract = {Growing evidence of morphological diversity in angiosperm flowers, seeds and pollen from the mid Cretaceous and the presence of derived lineages from increasingly older geological deposits both imply that the timing of early angiosperm cladogenesis is older than fossil-based estimates have indicated. An alternative to fossils for calibrating the phylogeny comes from divergence in DNA sequence data. Here, angiosperm divergence times are estimated using non-parametric rate smoothing and a three-gene dataset covering ca. 75{\%} of all angiosperm families recognized in recent classifications. The results provide an initial hypothesis of angiosperm diversification times. Using an internal calibration point, an independent evaluation of angiosperm and eudicot origins is performed. The origin of the crown group of extant angiosperms is indicated to be Early to Middle Jurassic (179-158 Myr), and the origin of eudicots is resolved as Late Jurassic to mid Cretaceous (147-131 Myr). Both estimates, despite a conservative calibration point, are older than current fossil-based estimates.},
author = {Wikstr{\"{o}}m, N and Savolainen, V and Chase, M W},
doi = {10.1098/rspb.2001.1782},
isbn = {0962-8452},
issn = {0962-8452},
journal = {Proceedings. Biological sciences / The Royal Society},
keywords = {Angiosperms,Angiosperms: classification,Angiosperms: genetics,Calibration,Evolution, Molecular,RNA, Ribosomal, 18S,Ribulose-Bisphosphate Carboxylase,Ribulose-Bisphosphate Carboxylase: genetics},
month = {nov},
number = {1482},
pages = {2211--20},
pmid = {11674868},
title = {{Evolution of the angiosperms: calibrating the family tree.}},
url = {http://rspb.royalsocietypublishing.org/content/268/1482/2211},
volume = {268},
year = {2001}
}
@article{Helmus2012,
abstract = {Phylogenetic diversity-area curves are analogous to species-area curves and quantify the relationship between the phylogenetic diversity of species assemblages and the area over which assemblages are sampled. Here, we developed theoretical expectations of these curves under different ecological and macroevolutionary processes. We first used simulations to generate curves expected under three ecological community assembly processes: species sorting, where species have distinct environmental preferences; random placement, where species have no environmental preference but vary in their prevalence across communities; and limited dispersal, where species have no environmental preference but vary in their ability to disperse. Second, we simulated curves expected across regions (e.g., across oceanic islands) that are derived from colonization among regions, within-region speciation, and extinction. We also computed curves for two data sets, one on forest plots along an elevation gradient and the other on Caribbean island Anolis lizards. Of the three ecological processes, only species sorting produced strong relationships between phylogenetic diversity and area. The forest plot curves matched the species-sorting expectation, but only when phylogenetic repulsion (that caused closely related species to be found in similar habitats but not in the same plots) was also included in the simulation. Strong relationships between regional phylogenetic diversity and area were simulated if species were derived only from within-region speciation; colonizations among regions obscured the pattern. Similarly, larger Caribbean islands had more within-island speciation and contained more Anolis phylogenetic diversity than smaller islands, but colonizations among islands obscured this relationship. This work furthers our understanding of the processes that govern the phylogenetic diversity of ecological communities and biogeographic regions.},
author = {Helmus, Matthew R. and Ives, Anthony R.},
doi = {10.1890/11-0435.1},
isbn = {0012-9658},
issn = {00129658},
journal = {Ecology},
keywords = {Anolis,Elevation,Environmental gradient,Island biogeography,Mt. Hood, Oregon, USA,Phylogenetic community structure,Phylogeny,Random placement,Spatial scale,Species area relationship,Species richness,Species sorting},
number = {8 SPEC. ISSUE},
pages = {S31--S43},
title = {{Phylogenetic diversity-area curves}},
url = {http://www.esajournals.org/doi/abs/10.1890/11-0435.1},
volume = {93},
year = {2012}
}
@article{Gastauer2013,
abstract = {There is an increased interest in phylogenetic approaches for conservation biology and community analysis. Many of these analyses are carried out using the Phylocom 4.2 package. With this computational tool, already existing trees are pruned to species from community to be studied. For plant communities, a variety of megatrees including all angiosperm families are available for phylogenetic community analysis. Using the bladj algorithm, internal nodes of community trees derived from these megatrees are calibrated on time scales from fossil or molecular data provided in an ages file. The higher precision of tree calibration, the better is the ecological interpretation if we assume that the closest related species have the most superposed set of traits, the highest probability of co-occurrence in case of environmental filter effects and the highest ratio of competitive exclusion. Together with the Phylocom 4.2 package comes an ages file based on Wikstrom's dating of angiosperm families (wikstrom.ages). But there are inconsistencies in syntax and/or nomenclature between internal node names of trees and the ages file from phylocom that influence the tree calibration and the subsequent analysis. To avoid that, we classified all online available megatrees according to their syntax and nomenclatureof internal nodes. For each of the four classes we provide a new, fully compatible ages file in the supplement material. Each online available megatree, pruned to the species from an example community from the Atlantic Rainforest, was calibrated twice running the bladj algorithm using once the original wikstrom.ages file and, additionally, the new ages file prepared for that tree class. Outcomes from trees calibrated by different methods have been compared. To avoid inconsistencies that push results beyond the realistic, we recommend a strict application of the four ages files provided as supplementary files. {\textcopyright} 2013 Elsevier B.V.},
author = {Gastauer, Markus and Meira-Neto, Jo{\~{a}}o Augusto Alves},
doi = {10.1016/j.ecoinf.2013.03.005},
isbn = {1574-9541},
issn = {15749541},
journal = {Ecological Informatics},
keywords = {Mean pairwise distance,Nearest taxon index,Net relatedness index,Phylogenetic diversity,Tree dating,Wikstrom ages},
month = {may},
pages = {85--90},
publisher = {Elsevier B.V.},
title = {{Avoiding inaccuracies in tree calibration and phylogenetic community analysis using Phylocom 4.2}},
url = {http://linkinghub.elsevier.com/retrieve/pii/S1574954113000265},
volume = {15},
year = {2013}
}
@article{Milo2003,
abstract = {Random graphs with prescribed degree sequences have been widely used as a model of complex networks. Comparing an observed network to an ensemble of such graphs allows one to detect deviations from randomness in network properties. Here we briefly review two existing methods for the generation of random graphs with arbitrary degree sequences, which we call the ``switching'' and ``matching'' methods, and present a new method based on the ``go with the winners'' Monte Carlo method. The matching method may suffer from nonuniform sampling, while the switching method has no general theoretical bound on its mixing time. The ``go with the winners'' method has neither of these drawbacks, but is slow. It can however be used to evaluate the reliability of the other two methods and, by doing this, we demonstrate that the deviations of the switching and matching algorithms under realistic conditions are small compared to the ``go with the winners'' algorithm. Because of its combination of speed and accuracy we recommend the use of the switching method for most calculations.},
archivePrefix = {arXiv},
arxivId = {cond-mat/0312028},
author = {Milo, R and Kashtan, N and Itzkovitz, S and Newman, M E J and Alon, U},
eprint = {0312028},
journal = {Arxiv preprint condmat0312028},
month = {dec},
pages = {1--4},
primaryClass = {cond-mat},
title = {{Uniform generation of random graphs with arbitrary degree sequences}},
url = {http://arxiv.org/abs/cond-mat/0312028},
volume = {cond-mat/0},
year = {2003}
}
@article{Blomberg2003,
abstract = {The primary rationale for the use of phylogenetically based statistical methods is that phylogenetic signal, the tendency for related species to resemble each other, is ubiquitous. Whether this assertion is true for a given trait in a given lineage is an empirical question, but general tools for detecting and quantifying phylogenetic signal are inadequately developed. We present new methods for continuous-valued characters that can be implemented with either phylogenetically independent contrasts or generalized least-squares models. First, a simple randomization procedure allows one to test the null hypothesis of no pattern of similarity among relatives. The test demonstrates correct Type I error rate at a nominal alpha = 0.05 and good power (0.8) for simulated datasets with 20 or more species. Second, we derive a descriptive statistic, K, which allows valid comparisons of the amount of phylogenetic signal across traits and trees. Third, we provide two biologically motivated branch-length transformations, one based on the Ornstein-Uhlenbeck (OU) model of stabilizing selection, the other based on a new model in which character evolution can accelerate or decelerate (ACDC) in rate (e.g., as may occur during or after an adaptive radiation). Maximum likelihood estimation of the OU (d) and ACDC (g) parameters can serve as tests for phylogenetic signal because an estimate of d or g near zero implies that a phylogeny with little hierarchical structure (a star) offers a good fit to the data. Transformations that improve the fit of a tree to comparative data will increase power to detect phylogenetic signal and may also be preferable for further comparative analyses, such as of correlated character evolution. Application of the methods to data from the literature revealed that, for trees with 20 or more species, 92{\%} of traits exhibited significant phylogenetic signal (randomization test), including behavioral and ecological ones that are thought to be relatively evolutionarily malleable (e.g., highly adaptive) and/or subject to relatively strong environmental (nongenetic) effects or high levels of measurement error. Irrespective of sample size, most traits (but not body size, on average) showed less signal than expected given the topology, branch lengths, and a Brownian motion model of evolution (i.e., K was less than one), which may be attributed to adaptation and/or measurement error in the broad sense (including errors in estimates of phenotypes, branch lengths, and topology). Analysis of variance of log K for all 121 traits (from 35 trees) indicated that behavioral traits exhibit lower signal than body size, morphological, life-history, or physiological traits. In addition, physiological traits (corrected for body size) showed less signal than did body size itself. For trees with 20 or more species, the estimated OU (25{\%} of traits) and/or ACDC (40{\%}) transformation parameter differed significantly from both zero and unity, indicating that a hierarchical tree with less (or occasionally more) structure than the original better fit the data and so could be preferred for comparative analyses.},
author = {Blomberg, Simon P and Jr, Theodore Garland and Ives, Anthony R},
doi = {10.1111/j.0014-3820.2003.tb00285.x},
isbn = {1558-5646},
issn = {0014-3820},
journal = {Evolution; international journal of organic evolution},
keywords = {2002,accepted november 27,adaptation,behavior,body size,branch lengths,comparative method,constraint,physiology,received march 29},
number = {4},
pages = {717--745},
pmid = {12778543},
title = {{TESTING FOR PHYLOGENETIC SIGNAL IN COMPARATIVE DATA : BEHAVIORAL TRAITS ARE MORE LABILE Published By : The Society for the Study of Evolution TESTING FOR PHYLOGENETIC SIGNAL IN COMPARATIVE DATA : BEHAVIORAL TRAITS ARE MORE LABILE}},
url = {http://onlinelibrary.wiley.com/doi/10.1111/j.0014-3820.2003.tb00285.x/abstract},
volume = {57},
year = {2003}
}
@article{Bush1997,
abstract = {We consider 27 population and community terms used frequently by parasitologists when describing the ecology of parasites. We provide suggestions for various terms in an attempt to foster consistent use and to make terms used in parasite ecology easier to interpret for those who study free-living organisms. We suggest strongly that authors, whether they agree or disagree with us, provide complete and unambiguous definitions for all parameters of their studies.},
author = {Bush, Albert O. and Lafferty, Kevin D. and Lotz, Jeffrey M. and Shostak, Allen W.},
doi = {10.2307/3284227},
isbn = {00223395},
issn = {0022-3395},
journal = {The Journal of parasitology},
number = {4},
pages = {575--583},
pmid = {9267395},
title = {{Parasitology meets ecology on its own terms: Margolis et al. revisited.}},
url = {http://www.jstor.org/stable/3284227},
volume = {83},
year = {1997}
}
@article{Chamberlain2014,
abstract = {Interaction webs, or networks, define how the members of two or more trophic levels interact. However, the traits that mediate network structure have not been widely investigated. Generally, the mechanism that determines plant-pollinator partnerships is thought to involve the matching of a suite of species traits (such as abundance, phenology, morphology) between trophic levels. These traits are often unknown or hard to measure, but may reflect phylogenetic history. We asked whether morphological traits or phylogenetic history were more important in mediating network structure in mutualistic plant-pollinator interaction networks from Western Canada. At the plant species level, sexual system, growth form, and flower symmetry were the most important traits. For example species with radially symmetrical flowers had more connections within their modules (a subset of species that interact more among one another than outside of the module) than species with bilaterally symmetrical flowers. At the pollinator species level, social species had more connections within and among modules. In addition, larger pollinators tended to be more specialized. As traits mediate interactions and have a phylogenetic signal, we found that phylogenetically close species tend to interact with a similar set of species. At the network level, patterns were weak, but we found increasing functional trait and phylogenetic diversity of plants associated with increased weighted nestedness. These results provide evidence that both specific traits and phylogenetic history can contribute to the nature of mutualistic interactions within networks, but they explain less variation between networks.},
author = {Chamberlain, Scott A. and Cartar, Ralph V. and Worley, Anne C. and Semmler, Sarah J. and Gielens, Grahame and Elwell, Sherri and Evans, Megan E. and Vamosi, Jana C. and Elle, Elizabeth},
doi = {10.1007/s00442-014-3035-2},
isbn = {9789078146131},
issn = {00298549},
journal = {Oecologia},
keywords = {Functional trait,Interaction webs,Morphological trait,Mutualism,Trophic levels},
month = {oct},
number = {2},
pages = {545--556},
pmid = {25142045},
title = {{Traits and phylogenetic history contribute to network structure across Canadian plant--pollinator communities}},
url = {http://www.ncbi.nlm.nih.gov/pubmed/25142045},
volume = {176},
year = {2014}
}
@incollection{Terborgh2010,
address = {Princeton},
author = {Terborgh, John},
booktitle = {The theory of island biogeography revisited},
editor = {Losos, J. B. and Ricklefs, R. E.},
isbn = {140083192X},
pages = {116--142},
publisher = {Princeton University Press},
title = {{The trophic cascade on islands}},
year = {2009}
}
@incollection{Hanski2010,
address = {Princeton},
author = {Hanski, Ilkka},
booktitle = {The Theory of Island Biogeography Revisited},
editor = {Losos, J. B. and Ricklefs, E. R.},
isbn = {9780691136523},
pages = {186--213},
publisher = {Princeton University Press},
title = {{The Theories of Island Biogeography and Metapopulation Dynamics}},
url = {http://medcontent.metapress.com/index/A65RM03P4874243N.pdf{\%}5Cnhttp://books.google.com/books?hl=en{\&}lr={\&}id=slwedU4I4JMC{\&}oi=fnd{\&}pg=PR3{\&}dq=The+Theory+of+Island+Biogeography+Revisited{\&}ots=CKcKB5Cslt{\&}sig=SGVIiFNwQ7tWaP-9K3tJnX9q0ss},
year = {2010}
}
@book{MacArthur1967,
abstract = {The Theory of Island Biogeography establishes the conditions for the attainment and maintenance of equilibrium species numbers on islands and frag- mented habitats. It employs mathematical models to estimate rates of colonization and turnover, as well as differences in species diversity among islands. (The SCI indicates that this book has been cited in over 1,830 publications.)},
address = {Princeton, NJ},
archivePrefix = {arXiv},
arxivId = {arXiv:1011.1669v3},
author = {Hamilton, T. H.},
booktitle = {Science},
doi = {10.1126/science.159.3810.71},
edition = {Princeton},
eprint = {arXiv:1011.1669v3},
isbn = {9780691088365},
issn = {0036-8075},
keywords = {biogeography,island models,species richness},
number = {3810},
pages = {71--72},
pmid = {250558},
publisher = {Princeton University Press},
title = {{The Theory of Island Biogeography. Robert H. MacArthur and Edward O. Wilson. Princeton University Press, Princeton, N.J., 1967. 215 pp., illus. Cloth, {\$}8; paper, {\$}3.95. Monographs in Population Biology, No. 1}},
url = {http://www.sciencemag.org/cgi/doi/10.1126/science.159.3810.71},
volume = {159},
year = {1968}
}
@book{Harte2011,
address = {Oxford},
author = {Harte, J},
booktitle = {Oikos},
isbn = {ISBN: 9780199593422},
issn = {0717-6163},
pages = {257},
publisher = {Oxford University Press},
title = {{Maximum entropy and ecology}},
url = {http://owenpetchey.staff.shef.ac.uk/assets/Petchey-2010-Oikos.pdf},
year = {2011}
}
@article{Taylor1987,
abstract = {Taylor, R.J.},
author = {Taylor, Robert J.},
doi = {10.2307/3565859},
issn = {00301299},
journal = {Oikos},
number = {2},
pages = {225},
title = {{The Geometry of Colonization: 1. Islands}},
url = {http://www.jstor.org/stable/3565859?origin=crossref},
volume = {48},
year = {1987}
}
@article{Kotiaho2007,
abstract = {The three most important ecological factors affecting the success of island invasions are the area of the island, isolation of the island and occurrence of predators on the island. Traditionally, invasion success has been studied on natural islands, which partly explains the rarity of controlled and replicated experiments. Here we report results from a field experiment investigating the influence of the above three factors in artificial islands. As an experimental system, we used predatory mites and a nematode community occurring naturally in boreal coniferous forests. We found that all three factors had an effect on invasion success, but surprisingly, that there were no interaction effects. ?? 2006 Elsevier B.V. All rights reserved.},
author = {Kotiaho, Janne S. and Sulkava, Pekka},
doi = {10.1016/j.apsoil.2006.05.003},
issn = {09291393},
journal = {Applied Soil Ecology},
keywords = {Area effect,Dispersal,Distance effect,Nematode,Predatory mites},
month = {jan},
number = {1},
pages = {256--259},
title = {{Effects of isolation, area and predators on invasion: A field experiment with artificial islands}},
url = {http://linkinghub.elsevier.com/retrieve/pii/S0929139306001181},
volume = {35},
year = {2007}
}
@article{Baker2014,
abstract = {Ecological communities are composed of many species and an intricate network of interactions between them. Because of their overall complexity, an intriguing approach to understanding network structure is by breaking it down into the structural roles of its constituent species. The structural role of a species can be directly measured based on how it appears in network motifs - the basic building blocks of complex networks. Here, we study the distribution of species' roles at three distinct spatio-temporal scales (i.e. species, network, and temporal) in host-parasitoid networks collected across 22 sites over two years within a fragmented landscape of oaks in southern Finland. We found that species' roles for hosts and parasitoids were heterogeneously distributed across the study system but that roles are strongly conserved over spatial scales. In addition, we found that species' roles were remarkably consistent between years even in the presence of disturbances (e.g. species turnover). Overall, our results suggest that species' roles are an intrinsic property of species that may be predictable over spatial and temporal scales.},
author = {Baker, Nick J. and Kaartinen, Riikka and Roslin, Tomas and Stouffer, Daniel B.},
file = {:Users/alyssacirtwill/Documents/Papers/Baker et al.{\_}2015{\_}Ecography.pdf:pdf},
isbn = {1600-0587},
issn = {16000587},
journal = {Ecography},
number = {2},
pages = {130--139},
title = {{Species' roles in food webs show fidelity across a highly variable oak forest}},
url = {http://doi.wiley.com/10.1111/ecog.00913},
volume = {38},
year = {2015}
}
@article{Heleno2014,
abstract = {In recent years, the analysis of interaction networks has grown popular as a framework to explore ecological processes and the relationships between community structure and its functioning. The field has rapidly grown from its infancy to a vibrant youth, as reflected in the variety and quality of the discussions held at the first international symposium on Ecological Networks in Coimbra-Portugal (23-25 October 2013). The meeting gathered 170 scientists from 22 countries, who presented data from a broad geographical range, and covering all stages of network analyses, from sampling strategies to effective ways of communicating results, presenting new analytical tools, incorporation of temporal and spatial dynamics, new applications and visualization tools.(1) During the meeting it became evident that while many of the caveats diagnosed in early network studies are successfully being tackled, new challenges arise, attesting to the health of the discipline.},
author = {Heleno, Ruben and Garcia, Cristina and Jordano, Pedro and Traveset, Anna and G{\'{o}}mez, Jos{\'{e}} Maria and Bl{\"{u}}thgen, Nico and Memmott, Jane and Moora, Mari and Cerdeira, Jorge and Rodr{\'{i}}guez-Echeverr{\'{i}}a, Susana and Freitas, Helena and Olesen, Jens M},
doi = {10.1098/rsbl.2013.1000},
isbn = {1744-957X (Electronic)$\backslash$r1744-9561 (Linking)},
issn = {1744-957X},
journal = {Biology letters},
keywords = {bioinformatics,ecology,environmental science,plant science,systems biology},
number = {1},
pages = {20131000},
pmid = {24402718},
title = {{Ecological networks: delving into the architecture of biodiversity.}},
url = {http://www.pubmedcentral.nih.gov/articlerender.fcgi?artid=3917341{\&}tool=pmcentrez{\&}rendertype=abstract},
volume = {10},
year = {2014}
}
@article{White2013,
abstract = {Sharing data is increasingly considered to be an important part of the scientific process. Making your data publicly available allows original results to be reproduced and new analyses to be conducted. While sharing your data is the first step in allowing reuse, it is also important that the data be easy to understand and use. We describe nine simple ways to make it easy to reuse the data that you share and also make it easier to work with it yourself. Our recommendations focus on making your data understandable, easy to analyze, and readily available to the wider community of scientists.},
author = {White, Ethan P and Baldridge, Elita and Brym, Zachary T and Locey, Kenneth J and McGlinn, Daniel J and Supp, Sarah R},
doi = {10.7287/peerj.preprints.7v2},
isbn = {2167-9843},
issn = {2167-9843},
journal = {PeerJ PrePrints},
keywords = {data format,data reuse,data sharing,license,repository},
number = {April},
pages = {e7v2},
pmid = {1000164289},
title = {{Nine simple ways to make it easier to (re)use your data}},
url = {http://dx.doi.org/10.7287/peerj.preprints.7v2{\%}5Cnhttps://peerj.com/preprints/7v2.pdf},
volume = {1},
year = {2013}
}
@article{McCann2005,
abstract = {The dynamics of ecological systems include a bewildering number of biotic interactions that unfold over a vast range of spatial scales. Here, employing simple and general empirical arguments concerning the nature of movement, trophic position and behaviour we outline a general theory concerning the role of space and food web structure on food web stability. We argue that consumers link food webs in space and that this spatial structure combined with relatively rapid behavioural responses by consumers can strongly influence the dynamics of food webs. Employing simple spatially implicit food web models, we show that large mobile consumers are inordinately important in determining the stability, or lack of it, in ecosystems. More specifically, this theory suggests that mobile higher order organisms are potent stabilizers when embedded in a variable, and expansive spatial structure. However, when space is compressed and higher order consumers strongly couple local habitats then mobile consumers can have an inordinate destabilizing effect. Preliminary empirical arguments show consistency with this general theory.},
author = {McCann, K. S. and Rasmussen, J. B. and Umbanhowar, J.},
doi = {10.1111/j.1461-0248.2005.00742.x},
isbn = {1461-023X},
issn = {1461023X},
journal = {Ecology Letters},
keywords = {Compartment,Consumer-resource interaction,Food web,Foraging,Scale,Space,Stability},
month = {may},
number = {5},
pages = {513--523},
pmid = {21352455},
title = {{The dynamics of spatially coupled food webs}},
url = {http://www.ncbi.nlm.nih.gov/pubmed/21352455},
volume = {8},
year = {2005}
}
@article{Holt1996,
abstract = {All ecologists are familiar with graphical portrayals of food webs such as that shown in Figure 29.1a—tinkertoy constructions of nodes (e.g., species) connected by lines (feeding relations). This depiction of food webs (or, more formally, its matrix equivalent) has without question helped articulate many important questions in community ecology (e.g., Pimm (1982), Pimm et al. (1991), and Cohen et al (1990)). Yet, as with any powerful conceptual schemata in science, this characterization of community organization both liberates—organizing one's thoughts in fruitful directions—and enslaves—subtly constraining the questions one tends to ask. In particular, most descriptions of, and models about, food web structure make no explicit reference to space. But all ecological interactions, including trophic relations, are necessarily played out in a spatial arena. For some purposes this observation may well be irrelevant. However, it is becoming increasingly clear that the resolution of many classical problems in community ecology, from the coexistence of competitors (e.g., Hanski (1983)), to the stabilization of predator-prey interactions (e.g. Hassell et al. (1991), to the interpretation of species richness patterns (Cornell and Lawton, 1992), requires a consideration of spatial dynamics. Food web ecology, too, should profit from an explicit incorporation of spatial perspectives.},
author = {{Frank van Veen}, F. J.},
doi = {10.1016/j.cub.2009.01.026},
isbn = {0960-9822},
issn = {09609822},
journal = {Current Biology},
number = {7},
pages = {313--323},
pmid = {19368868},
title = {{Food webs}},
url = {http://people.biology.ufl.edu/rdholt/holtpublications/056.PDF},
volume = {19},
year = {2009}
}
@article{Gripenberg2007,
abstract = {Why is the World green - what keeps herbivores, and herbivorous insects in particular, from consuming all of their food? When this question was first posed, the relative importance of top-down and bottom-up effects was hotly disputed. While modern ecologists may agree that impacts from several different directions will affect local insect densities, the bottom-up vs top-down jargon seems to be stuck in a unidimensional world. Here, we argue that the strength of almost every bottom-up and top-down force is likely to vary in space, and that in itself, spatial structure invokes new processes which defy classification in the traditional bottom-up top-down scheme. To understand the relative importance of different forces keeping herbivore numbers in check, we feel that we need a fresh synthesis between the novel paradigm of spatial ecology and the classical paradigms of top-down and bottom-up studies. This synthesis requires a consideration of forces beyond the standard framework of top-down vs bottom-up effects, and should be based on comparing the relative strength of such forces at several sites in a spatially explicit framework. Overall, we should switch our focus from whether the relative strength of top-down and bottom-up factors vary in space to why there is variation, how much variation there is, and at what spatial scale(s) it occurs.},
author = {Gripenberg, Sofia and Roslin, Tomas},
doi = {10.1111/j.2006.0030-1299.15266.x},
isbn = {0030-1299},
issn = {00301299},
journal = {Oikos},
month = {feb},
number = {2},
pages = {181--188},
pmid = {875},
title = {{Up or down in space? Uniting the bottom-up versus top-down paradigm and spatial ecology}},
url = {http://doi.wiley.com/10.1111/j.2006.0030-1299.15266.x},
volume = {116},
year = {2007}
}
@article{Hastings2004,
abstract = {Ecological theory has been dominated by a focus on long-term or asymptotic behavior as a way to understand natural systems. Yet experiments are done on much shorter timescales, and the relevant timescales for ecological systems can also be relatively short. Thus, there is a mismatch between the timescales of most experiments and the timescales of many theoretical investigations. However, recent work has emphasized the importance of transient dynamics rather than long-term behavior in ecological systems, enabling the examination of forces that allow coexistence on ecological timescales. Through an examination of what leads to transients in ecological systems, a deeper appreciation of the forces leading to persistence or coexistence in ecological systems emerges, as well as a general understanding of how population levels can change through time.},
author = {Hastings, Alan},
doi = {10.1016/j.tree.2003.09.007},
isbn = {0169-5347 (Print)$\backslash$n0169-5347 (Linking)},
issn = {01695347},
journal = {Trends in Ecology and Evolution},
month = {jan},
number = {1},
pages = {39--45},
pmid = {16701224},
title = {{Transients: The key to long-term ecological understanding?}},
url = {http://www.ncbi.nlm.nih.gov/pubmed/16701224},
volume = {19},
year = {2004}
}
@article{Araujo2006,
abstract = {To study the abundance and occurrence of herbivore insects on plants it is important to consider plant characteristics that may control the insects. In this study the following hypotheses were tested: (i) an increase of plant architecture increases species richness and abundance of gall-inducing insects and (ii) plant architecture increases gall survival and decreases parasitism. Two hundred and forty plants of Baccharis pseudomyriocephala Teodoro (Asteraceae) were sampled, estimating the number of shoots, branches and their biomass. Species richness and abundance of galling insects were estimated per module, and mortality of the galls was assessed. Plant architecture influenced positively species richness, abundance and survival of galls. However, mortality of galling insects by parasitoids was low (13.26{\%}) and was not affected by plant architecture, thus suggesting that other plant characteristics (a bottom-up pressure) might influence gall-inducing insect communities more than parasitism (a top-down pressure). The opposite effect of herbivore insects on plant characteristics must also be considered, and such effects may only be assessed through manipulative experiments.},
author = {Ara{\'{u}}jo, Ana Paula Albano and {De Paula}, Joana D Arc and Carneiro, Marco Antonio Alves and Schoereder, Jos{\'{e}} Henrique},
doi = {10.1111/j.1442-9993.2006.01563.x},
isbn = {1442-9985},
issn = {14429985},
journal = {Austral Ecology},
keywords = {Abundance,Baccharis pseudomyriocephala,Galling insect,Richness,Structural complexity},
month = {may},
number = {3},
pages = {343--348},
pmid = {546},
title = {{Effects of host plant architecture on colonization by galling insects}},
url = {http://doi.wiley.com/10.1111/j.1442-9993.2006.01563.x},
volume = {31},
year = {2006}
}
@article{Denno2002,
abstract = {We employed a combination of factorial experiments in the field and laboratory to investigate the relative magnitude and degree of interaction of bottom-up factors (two levels each of host-plant nutrition and vegetation complexity) and top-down forces (two levels of wolf-spider predation) on the population growth of Prokelisia planthoppers (P. dolus and P. marginata), the dominant insect herbivores on Spartina cordgrass throughout the intertidal marshes of North America. Treatments were designed to mimic combinations of plant characteristics and predator densities that occur naturally across habitats in the field.There were complex interactive effects between plant resources and spider predation on the population growth of planthoppers. The degree that spiders suppressed planthoppers depended on both plant nutrition and vegetation complexity, an interaction that was demonstrated both in the field and laboratory. Laboratory results showed that spiders checked planthopper populations most effectively on poor-quality Spartina with an associated matrix of thatch, all characteristics of high-marsh meadow habitats. It was also this combination of plant resources in concert with spiders that promoted the smallest populations of planthoppers in our field experiment. Planthopper populations were most likely to escape the suppressing effects of predation on nutritious plants without thatch, a combination of factors associated with observed planthopper outbreaks in low-marsh habitats in the field. Thus, there is important spatial variation in the relative strength of forces with bottom-up factors dominating under low-marsh conditions and top-down forces increasing in strength at higher elevations on the marsh.Enhancing host-plant biomass and nutrition did not strengthen top-down effects on planthoppers, even though nitro-en-rich plants supported higher densities of wolf spiders and other invertebrate predators in the field. Rather, planthopper populations, particularly those of Prokelisia marginata, escaped predator restraint on high-quality plants, a result we attribute to its mobile life history, enhanced colonizing ability, and rapid growth rate. Thus, our results for Prokelisia planthoppers suggest that the life history strategy of a species is an important mediator of top-down and bottom-up impacts.In laboratory mesocosms, enhancing plant biomass and nutrition resulted in increased spider reproduction, a,cascading effect associated with planthopper increases on high-quality plants. Although the adverse effects of spider predation on planthoppers cascaded down and fostered increased plant biomass in laboratory mesocosms, this result did not occur in the field where top-down effects attenuated. We attributed this outcome in part to the intraguild predation of other planthopper predators by wolf spiders. Overall, the general paradigm in this system is for bottom-up forces to dominate, and when predators do exert a significant suppressing effect on planthoppers, their impact is generally legislated by vegetation characteristics.},
author = {Denno, Robert F. and Gratton, Claudio and Peterson, Merrill a. and Langellotto, Gail a. and Finke, Deborah L. and Huberty, Andrea F.},
doi = {10.1890/0012-9658(2002)083[1443:BUFMNE]2.0.CO;2},
isbn = {0012-9658},
issn = {00129658},
journal = {Ecology},
keywords = {Bottom-up vs. top-down impact,Habitat complexity,Intertidal wetlands,Intraguild predation,Multitrophic interactions,Phytophagous insect community,Plant nutrition,Planthopper,Prokelisia spp.,Spartina alterniflora,Trophic cascades,Vegetation structure},
month = {may},
number = {5},
pages = {1443--1458},
pmid = {5323},
title = {{Bottom-up forces mediate natural-enemy impact in a phytophagous insect community}},
url = {http://www.esajournals.org/doi/abs/10.1890/0012-9658(2002)083[1443:BUFMNE]2.0.CO;2},
volume = {83},
year = {2002}
}
@article{Gratton2003,
abstract = {Although many studies now examine how multiple factors influence the dynamics of herbivore populations, few studies explicitly attempt to document where and when each is important and how they vary and interact. In fact, how temporal variation in top-down (natural enemies) and bottom-up (host plant resources) factors affect herbivore dynamics has been suggested as a particularly important yet poorly understood feature of terrestrial food webs. In this study we examined how temporal changes in predator density (wolf spiders, sheet-web builders, and mirid egg predators) and host-plant resources (plant quality and structural complexity) influ-ence the population dynamics of the dominant phytoph-agous insects on Atlantic-coast salt marshes, namely Prokelisia planthoppers (Homoptera: Delphacidae). We designed a factorial experiment in meadows of Spartina alterniflora to mimic natural variation in vegetation quality and structure by establishing two levels of plant nutrition (leaf nitrogen content) by fertilization, and two levels of habitat complexity by adding leaf litter (thatch). We then assessed seasonal changes in the strength of bottom-up (plant quality) and top-down (predator) im-pacts on planthopper populations. Planthopper popula-tions responded positively to increased plant quality treatments in late summer. Despite the greater number of planthopper adults colonizing fertilized Spartina plots compared to unfertilized controls, the offspring of these colonists were much less abundant at the end of the season in fertilized plots, particularly those with thatch. The initial colonization effect was later erased because arthropod predators selectively accumulated in fertilized plots where they inflicted significant mortality on all stages of planthoppers. Predators rapidly colonized fertilized plots and reached high densities well in advance of planthopper colonization, a response we attribute to their rapid aggregation in complex-structured habitats with readily available alternative prey. Our results suggest that plant resources not only mediate the strength of predator impacts on herbivore populations, but they also promote the coupling of predator and prey populations and thus influence when enemy impacts are realized.},
author = {Gratton, Claudio and Denno, Robert F},
doi = {10.1007/s00442-002-1137-8},
isbn = {0044200211378},
issn = {0029-8549},
journal = {Oecologia},
keywords = {Alternate prey {\textperiodcentered},Habitat complexity {\textperiodcentered},Multichannel omnivory {\textperiodcentered},Predator-prey {\textperiodcentered},Temporal variation},
month = {mar},
number = {4},
pages = {487--495},
pmid = {12647120},
title = {{Seasonal shift from bottom-up to top-down impact in phytophagous insect populations}},
url = {http://www.ncbi.nlm.nih.gov/pubmed/12647120},
volume = {134},
year = {2003}
}
@article{Araujo2006a,
abstract = {To study the abundance and occurrence of herbivore insects on plants it is important to consider plant characteristics that may control the insects. In this study the following hypotheses were tested: (i) an increase of plant architecture increases species richness and abundance of gall-inducing insects and (ii) plant architecture increases gall survival and decreases parasitism. Two hundred and forty plants of Baccharis pseudomyriocephala Teodoro (Asteraceae) were sampled, estimating the number of shoots, branches and their biomass. Species richness and abundance of galling insects were estimated per module, and mortality of the galls was assessed. Plant architecture influenced positively species richness, abundance and survival of galls. However, mortality of galling insects by parasitoids was low (13.26{\%}) and was not affected by plant architecture, thus suggesting that other plant characteristics (a bottom-up pressure) might influence gall-inducing insect communities more than parasitism (a top-down pressure). The opposite effect of herbivore insects on plant characteristics must also be considered, and such effects may only be assessed through manipulative experiments.},
author = {Ara{\'{u}}jo, Ana Paula Albano and {De Paula}, Joana D Arc and Carneiro, Marco Antonio Alves and Schoereder, Jos{\'{e}} Henrique},
doi = {10.1111/j.1442-9993.2006.01563.x},
isbn = {1442-9985},
issn = {14429985},
journal = {Austral Ecology},
keywords = {Abundance,Baccharis pseudomyriocephala,Galling insect,Richness,Structural complexity},
number = {3},
pages = {343--348},
pmid = {546},
title = {{Effects of host plant architecture on colonization by galling insects}},
url = {http://onlinelibrary.wiley.com/doi/10.1111/j.1442-9993.2006.01563.x/full},
volume = {31},
year = {2006}
}
@article{Karnatak2014,
abstract = {We study the dynamics of two predator-prey systems that are coupled via cross-predation, in which each predator consumes also the other prey. This setup constitutes a model system in which conjugate coupling emerges naturally and denotes the transition from two separate food chains to a food web. We show that cross-predation of a certain strength leads to amplitude death stabilizing the food web in a new equilibrium. {\textcopyright} 2014 Elsevier Ltd. All rights reserved.},
author = {Karnatak, Rajat and Ramaswamy, Ram and Feudel, Ulrike},
doi = {10.1016/j.chaos.2014.07.003},
issn = {09600779},
journal = {Chaos, Solitons and Fractals},
month = {nov},
pages = {48--57},
publisher = {Elsevier Ltd},
title = {{Conjugate coupling in ecosystems: Cross-predation stabilizes food webs}},
url = {http://linkinghub.elsevier.com/retrieve/pii/S0960077914001192},
volume = {68},
year = {2014}
}
@article{Simberloff1976a,
author = {Simberloff, D and Simberloff, D},
journal = {Ecology},
keywords = {1972,action,analyzing the colo-,arthropods,colonization,community,florida,heatwole and levins,insect,insects,inter-,islands,species,tropic structure},
number = {2},
pages = {395--398},
title = {{Trophic Structure Determination and Equilibrium in an Arthropod Community}},
volume = {57},
year = {1976}
}
@article{Simberloff1970a,
author = {Simberloff, D S and Wilson, E O},
journal = {Ecology},
number = {5},
pages = {934--937},
title = {{Simberloff {\&} Wilson 1970 - Experimental zoogeography of islands-two-year record of colonization.pdf}},
url = {http://www.jstor.org/stable/1933995},
volume = {51},
year = {1970}
}
@article{Heatwole1972,
author = {Heatwole, Harold and Levins, Richard},
doi = {10.2307/1934248},
isbn = {0012-9658},
issn = {0012-9658},
journal = {Ecology},
number = {3},
pages = {531--534},
title = {{Trophic Structure Stability and Faunal Change during Recolonization}},
url = {http://www.jstor.org/stable/1934248},
volume = {53},
year = {1972}
}
@article{Whittaker2000,
abstract = {Aim Few data sets exist that allow measurement of long-term extinction and turnover rates for islands of the size of the three main islands of the Krakatau group. We test the reliability of previous estimates of plant species extinction and examine structure within the extinction data. Location The data analysed are for the three older Krakatau islands: Rakata, Sertung and Panjang in the Sunda Strait, Indonesia. Methods Our analysis is based on a comprehensive database incorporating all species records for each island since recolonization began after the 1883 sterilization, plus attributes such as distribution, phylogeny, population status and dispersal mechanism for each species. We employ a combination of univariate and multi-term analyses in analysing structure, and derive Minimal Adequate Models using binary logistic analyses of variance and covariance. We compare the 1883-1934 data set with the contemporary nora as represented by (1) 1979-83 records las used in previous analyses) and (2) 1979-94 data (original). Results The improved data for the contemporary flora reduces the number of missing species by one-third. We show that a variety of estimates of extinction rate can be produced depending on what assumptions are made concerning the status of particular species groups. Structural features in the extinction data persist despite the reduction in overall numbers of losses. Losses relate to: (Ij the number of islands on which a species originally occurred, (2) the primary dispersal mode, and (3) the original abundance of a species (e.g. whether it was known to have established a successful resident population, and whether it was in decline or increasing in c. 1930). The 'best' descriptive model employs the variables denoted under (3). A high proportion of losses comprised species introduced by people and rare or ephemeral species. Losses of 'residents' that had colonized naturally could largely be accounted for by reference to (1) successional loss of habitat and, to a lesser degree, (2) other habitat disturbance or loss. Main conclusions Previous analyses, based on a more limited data set, have significantly over-estimated extinction from the Krakatau flora. Few naturally colonizing and established species have become extinct. The findings indicate that caution is necessary in interpreting 'headline' island ecological rates, and in analysing and modelling such data. Examination of structural features of the data appear to be valuable both in providing ecological insights in their own right, and in enabling refinements to estimates of extinction and thus turnover.},
author = {Whittaker, Robert J. and Field, Richard and Partomihardjo, Tukirin},
doi = {10.1046/j.1365-2699.2000.00487.x},
isbn = {0305-0270},
issn = {03050270},
journal = {Journal of Biogeography},
keywords = {Extinction estimates,Island biogeography,Krakatau islands,Plant dispersal,Sampling problems,Species abundance,Species turnover rates,Succession},
number = {5},
pages = {1049--1064},
title = {{How to go extinct: Lessons from the lost plants of Krakatau}},
volume = {27},
year = {2000}
}
@article{Ryberg2012,
abstract = {Although predator effects on the number of locally coexisting species are well understood, there are few formal predictions of how these local predator effects influence patterns of prey diversity at larger spatial scales. Building on the theory of island biogeography, we develop a simple model that describes how predators can alter the scaling of diversity in prey metacommunities and compares the effects of generalist and specialist predators on regional prey diversity. Generalist predators, which consume prey randomly with respect to species identity, are predicted to reduce $\alpha$-diversity and increase $\beta$-diversity thereby maintaining regional diversity ($\gamma$-diversity). Alternatively, specialist predators, which filter out prey species intolerant of predators, are predicted to reduce both$\alpha$-diversity and$\beta$-diversity by causing the same prey species to be extirpated in each locality, resulting in regional prey species extinctions and lower $\gamma$-diversity. These distinct effects of generalist and specialist predators on prey diversity at different spatial scales are uniquely shaped by the extent of predation within those metacommunities. Overall, our model results make general predictions for how different types of predators can differentially affect prey diversity across spatial scales, allowing a more complete understanding of the possible implications of predator eradications or introductions for biodiversity.},
author = {Ryberg, Wade A. and Smith, Kevin G. and Chase, Jonathan M.},
doi = {10.1111/j.1600-0706.2012.19620.x},
isbn = {00301299},
issn = {00301299},
journal = {Oikos},
month = {dec},
number = {12},
pages = {1995--2000},
title = {{Predators alter the scaling of diversity in prey metacommunities}},
url = {http://doi.wiley.com/10.1111/j.1600-0706.2012.19620.x},
volume = {121},
year = {2012}
}
@article{Slove2010,
abstract = {2. The evidence for such a gradient is, however, ambiguous, and the results have varied as much as the methods. Several studies have considered the non-independence of species, but few have performed explicit phylogenetic analyses.3. In the present study, we tested for a correlation between diet breadth and latitude of distribution in Nymphalinae butterflies using generalised estimating equations (GEE) and accounting for phylogenetic independence.4. Using a simple model with only latitude of distribution as a predictor variable revealed a significant positive relationship with diet breadth. Previous studies, however, have shown that diet breadth is also correlated with butterfly range size, and in turn, that range size may be correlated with latitude of distribution. Including geographical range size in the model also turned out to have a profound effect on the results - to the extent that the relationship between latitude of distribution and diet breadth was effectively reversed.5. We conclude that, at least for this group of butterflies, there is no evidence for a positive correlation between latitude of species distribution and diet breadth when controlling for range size, and that the effect may actually even be reversed.},
author = {Slove, Jessica and Janz, Niklas},
doi = {10.1111/j.1365-2311.2010.01238.x},
isbn = {0307-6946},
issn = {03076946},
journal = {Ecological Entomology},
keywords = {Diversification,Generalisation,Host range,Latitude,Polyphagy,Specialisation},
month = {dec},
number = {6},
pages = {768--774},
title = {{Phylogenetic analysis of the latitude-niche breadth hypothesis in the butterfly subfamily Nymphalinae}},
url = {http://doi.wiley.com/10.1111/j.1365-2311.2010.01238.x},
volume = {35},
year = {2010}
}
@article{Krasnov2008,
abstract = {Aim We searched for relationships between latitude and both the geographic range size and host specificity of fleas parasitic on small mammals. This provided a test for the hypothesis that specialization is lower, and thus niche breadth is wider, in high-latitude species than in their counterparts at lower latitudes. Location We used data on the host specificity and geographic range size of 120 Palaearctic flea species (Siphonaptera) parasitic on small mammals (Soricomorpha, Lagomorpha and Rodentia). Data on host specificity were taken from 33 regions, whereas data on geographic ranges covered the entire distribution of the 120 species. Methods Our analyses controlled for the potentially confounding effects of phylogenetic relationships among flea species by means of the independentcontrasts method. We used regressions and structural equation modelling to determine whether the latitudinal position of the geographic range of a flea covaried with either the size of its range or its host specificity. The latter was measured as the number of host species used, as well as by an index providing the average (and variance in) taxonomic distinctness among the host species used by a flea. Results Geographic range size was positively correlated with the position of the centre of the range; in other words, fleas with more northerly distributions had larger geographic ranges. Although the number of host species used by a flea did not vary with latitude, both the mean taxonomic distinctness among host species used and its variance increased significantly towards higher latitudes. Main conclusions The results indicate that niche breadth in fleas, measured in terms of both its spatial (geographic range size) and biological (host specificity) components, increases at higher latitudes. These findings are compatible with the predictions of recent hypotheses about latitudinal gradients.},
author = {Krasnov, Boris R. and Shenbrot, Georgy I. and Khokhlova, Irina S. and Mouillot, David and Poulin, Robert},
doi = {10.1111/j.1365-2699.2007.01800.x},
isbn = {0305-0270},
issn = {03050270},
journal = {Journal of Biogeography},
keywords = {Fleas,Geographic range,Host specificity,Latitude,Niche breadth,Rapoport's rule,Siphonaptera,Small mammals},
month = {apr},
number = {4},
pages = {592--601},
title = {{Latitudinal gradients in niche breadth: Empirical evidence from haematophagous ectoparasites}},
url = {http://doi.wiley.com/10.1111/j.1365-2699.2007.01800.x},
volume = {35},
year = {2008}
}
@article{Flores2014,
abstract = {The statistical analysis of the structure of bipartite ecological networks has increased in importance in recent years. Yet, both algorithms and software packages for the analysis of network structure focus on properties of unipartite networks. In response, we describe BiMAT, an object-oriented MATLAB package for the study of the structure of bipartite ecological networks. BiMAT can analyze the structure of networks, including features such as modularity and nestedness, using a selection of widely-adopted algorithms. BiMAT also includes a variety of null models for evaluating the statistical significance of network properties. BiMAT is capable of performing multi-scale analysis of structure - a potential (and under-examined) feature of many biological networks. Finally, BiMAT relies on the graphics capabilities of MATLAB to enable the visualization of the statistical structure of bipartite networks in either matrix or graph layout representations. BiMAT is available as an open-source package at http://ecotheory.biology.gatech.edu/cflores.},
archivePrefix = {arXiv},
arxivId = {arXiv:1406.6732v1},
author = {Flores, Cesar O. and Poisot, Timoth{\'{e}}e and Valverde, Sergi and Weitz, Joshua S.},
doi = {10.1111/2041-210X.12458},
eprint = {arXiv:1406.6732v1},
isbn = {2041-210X},
issn = {2041210X},
journal = {Methods in Ecology and Evolution},
keywords = {Community ecology,Complex networks,Ecological networks,Host-parasite interactions,Microbial ecology},
title = {{BiMat: A MATLAB package to facilitate the analysis of bipartite networks}},
url = {http://arxiv.org/abs/1406.6732},
year = {2015}
}
@article{Thompson2004c,
abstract = {We used standardised techniques to assemble 18 food-webs in streams subject to four land uses; exotic pasture, native tussock, native forest, and pine plantation. There were clear differences in the algal productivity and standing crops of organic matter between the forested (native and pine) and grassland (tussock and pasture), but not within each grouping. Algal productivity was more than twice as high in the grassland sites, whereas the converse was true for organic matter standing crop. These differences in energy resources were correlated with differences in community composition and food-web structure. Although all streams had a generalist core of species, certain species of algae and invertebrates were predictably associated with either forested or grassland sites. Food-web structure in the forested and grassland sites was also distinct. Grassland foodwebs were complex, highly internally connected, and typified by a “triangular” shape. Forested food-webs in contrast were less highly connected, tended to have fewer trophic levels, and were “squarer” in shape. These results provide some support for the concept that energy supply may be an important contributing factor influencing stream community structure. In terms of riparian management, the results emphasise the importance of protecting representative vegetation around streams to protect stream communities.},
author = {Thompson, R M and Townsend, C R},
doi = {10.1080/00288330.2004.9517265},
isbn = {0028-8330},
issn = {0028-8330},
journal = {New Zealand Journal of Marine and Freshwater Research},
keywords = {Land use,Nitrogen,algae,ammonia,chemical analysis,crayfish,fish,food chains,pH,phosphorus,riparian},
number = {December 2011},
pages = {595--608},
title = {{Land-use influences on New Zealand stream communities: effects on species composition, functional organisation, and food web structure.}},
url = {{\%}5C{\%}5Cniwa-ham{\%}5Cshared{\%}5CProjects{\%}5CHRC08201{\%}5CWorking{\%}5CBibliography{\%}5CLiterature database (EKW updated 180908){\%}5CAotearoa relevant{\%}5CFoodwebs{\%}5CThompson {\&} Townsend 2004 stream communities {\&} foodwebs.pdf},
volume = {38},
year = {2004}
}
@article{Harrison2004,
abstract = {1. Many lowland rivers in Western Europe have been substantially modified to aid land drainage and support the intensification of agriculture. Although there have been many attempts at rehabilitation, few have been systematically evaluated on ecological criteria. 2. Macroinvertebrates were assessed in 13 UK lowland rivers containing instream rehabilitation structures, seven with artificial riffles (intended to mimic natural gravel riffles) and six with flow deflectors (intended to increase flow, depth and substrate heterogeneity within the channel). In each river, invertebrates were compared between stretches of river with and without rehabilitation structures. 3. Rehabilitated and reference stretches were subdivided into different benthic and macrophyte habitats. Three macroinvertebrate samples were taken once in July/August 1999 from each habitat across all schemes and rivers. Current velocity, depth and substratum particle size were recorded at the same time from each habitat. 4. Artificial riffle benthos had faster current, a coarser substratum and was shallower than reference benthos. Depth and substratum particle size differed little between flow deflector and reference benthos, although velocity downstream of the deflector tip was greater, and velocity in the lee of the deflector lower, than reference benthos. At a habitat scale, the benthos of artificial riffles, but not flow deflectors, had higher abundance, taxon richness and diversity than reference benthos. The impact of artificial riffles was most marked for benthic rheophilic taxa. 5. In all rivers, macroinvertebrate diversity was highest in marginal macrophytes and abundance highest in instream macrophytes. Although invertebrate communities were distinct between artificial riffle (but not flow deflector) and reference benthos, these differences were negligible in comparison to those between benthic and macrophyte habitats. 6. Neither artificial riffles nor flow deflectors had any significant impact on the taxon richness of the benthos or of the rehabilitated stretch of the river as a whole. Invertebrate diversity of rehabilitated stretches related closely to that of reference stretches, indicating that larger scale factors constrained any impact of rehabilitation. 7. Synthesis and applications. Local rehabilitation structures appeared to have minor biological effects in lowland rivers. We suggest that post-project appraisal should be more rigorously applied to rehabilitation schemes, measuring success against more clearly defined goals. We also advocate a greater emphasis on large-scale riparian, floodplain and catchment rehabilitation, rather than small-scale channel rehabilitation. Such a change in approach needs more effective cooperation and collaboration between all catchment users.},
author = {Harrison, S. S C and Pretty, J. L. and Shepherd, D. and Hildrew, a. G. and Smith, C. and Hey, R. D.},
doi = {10.1111/j.0021-8901.2004.00958.x},
isbn = {0021-8901},
issn = {00218901},
journal = {Journal of Applied Ecology},
keywords = {Artificial riffles,Benthos,Flow deflectors,Macrophytes,Restoration,River restoration},
month = {dec},
number = {6},
pages = {1140--1154},
pmid = {3247},
title = {{The effect of instream rehabilitation structures on macroinvertebrates in lowland rivers}},
url = {http://doi.wiley.com/10.1111/j.0021-8901.2004.00958.x},
volume = {41},
year = {2004}
}
@article{Brose2005,
abstract = {Trophic information -who eats whom- ;and species' body sizes are two of the most basic descriptions necessary to understand community structure as well as ecological and evolutionary dynamics. Consumer-resource body size ratios between predators and their prey, and parasitoids and their hosts, have recently gained increasing attention due to their important implications for species' interaction strengths and dynamical population stability. This data set documents body sizes of consumers and their resources. We gathered body size data for the food webs of Skipwith Pond, a parasitoid community of grass-feeding chalcid wasps in British grasslands; the pelagic community of the Benguela system, a source web based on broom in the United Kingdom; Broadstone Stream, UK; the Grand Cari{\c{c}}aie marsh at Lake Neuch{\^{a}}tel, Switzerland; Tuesday Lake, USA; alpine lakes in the Sierra Nevada of California; Mill Stream, UK; and the eastern Weddell Sea Shelf, Antarctica. Further consumer-resource body size data are included for planktonic predators, predatory nematodes, parasitoids, marine fish predators, freshwater invertebrates, Australian terrestrial consumers, and aphid parasitoids. Containing 16 807 records, this is the largest data set ever compiled for body sizes of consumers and their resources. In addition to body sizes, the data set includes information on consumer and resource taxonomy, the geographic location of the study, the habitat studied, the type of the feeding interaction (e.g., predacious, parasitic) and the metabolic categories of the species (e.g., invertebrate, ectotherm vertebrate). The present data set was gathered with the intent to stimulate research on effects of consumer-resource body size patterns on food-web structure, interaction-strength distributions, population dynamics, and community stability. The use of a common data set may facilitate cross-study comparisons and understanding of the relationships between different scientific approaches and models.},
author = {Brose, Ulrich and Cushing, Lara and Berlow, Eric L. and Jonsson, Tomas and Banasek-Richter, Carolin and Bersier, Louis-Felix and Blanchard, Julia L. and Brey, Thomas and Carpenter, Stephen R. and Blandenier, Marie-France Cattin and Cohen, Joel E. and Dawah, Hassan Ali and Dell, Tony and Edwards, Francois and Harper-Smith, Sarah and Jacob, Ute and Knapp, Roland a. and Ledger, Mark E. and Memmott, Jane and Mintenbeck, Katja and Pinnegar, John K. and Rall, Bj{\"{o}}rn C. and Rayner, Tom and Ruess, Liliane and Ulrich, Werner and Warren, Philip and Williams, Rich J. and Woodward, Guy and Yodzis, Peter and Martinez, Neo D.},
doi = {10.1890/05-0379},
isbn = {00129658},
issn = {0012-9658},
journal = {Ecology},
number = {9},
pages = {2545--2545},
title = {{Body Sizes of Consumers and Their Resources}},
url = {http://doi.wiley.com/10.1890/05-0379},
volume = {86},
year = {2005}
}
@article{Knapp2000,
abstract = {One of the most puzzling aspects of the worldwide decline of amphibians is their disappearance from within protected areas. Because these areas are ostensibly undisturbed, habitat alterations are generally perceived as unlikely causes. The introduction of non-native fishes into protected areas, however, is a common practice throughout the world and may exert an important influence on amphibian distributions. We quantified the role of introduced fishes (several species of trout) in the decline of the mountain yellow-legged frog (Rana muscosa) in California's Sierra Nevada through surveys of {\textgreater}1700 sites in two adjacent and historically fishless protected areas that differed primarily in the distribution of introduced fish. Negative effects of fishes on the distribution of frogs were evident at three spatial scales. At the landscape scale, comparisons between the two protected areas indicated that fish distribution was strongly negatively correlated with the distribution of frogs. At the watershed scale, the percentage of total water-body surface area occupied by fishes was a highly significant predictor of the percentage of total water-body surface area occupied by frogs. At the scale of individual water bodies, frogs were three times more likely to be found and six times more abundant in fishless than in fish-containing waterbodies, after habitat effects were accounted for. The strong effect of introduced fishes on mountain yellow-legged frogs appears to result from the unique life history of this amphibian which frequently restricts larvae to deeper water bodies, the same habitats into which fishes have most frequently been introduced. Because fish populations in at least some Sierra Nevada lakes can be removed with minimal effort, our results suggest that the decline of the mountain yellow-legged frog might be relatively easy to reverse.},
author = {Knapp, Roland a. and Matthews, Kathleen R.},
doi = {10.1046/j.1523-1739.2000.99099.x},
isbn = {0888-8892},
issn = {08888892},
journal = {Conservation Biology},
month = {apr},
number = {2},
pages = {428--438},
title = {{Non-native fish introductions and the decline of the mountain yellow-legged frog from within protected areas}},
url = {http://doi.wiley.com/10.1046/j.1523-1739.2000.99099.x},
volume = {14},
year = {2000}
}
@article{Niering1963,
author = {Niering, W a},
doi = {10.5479/si.00775630.49.1},
issn = {00775630},
journal = {Atoll Research Bulletin},
number = {2},
pages = {74},
title = {{Bioecology of Kapingmarangi Atoll, Caroline Islands: Terrestrial aspects}},
url = {http://www.jstor.org/stable/1948559},
volume = {49},
year = {1956}
}
@article{Woodwell1967,
author = {Woodwell, G M},
doi = {10.1038/scientificamerican0367-24},
issn = {0036-8733},
journal = {Scientific American},
number = {3},
pages = {24--31},
pmid = {6041701},
title = {{Toxic substances and ecological cycles.}},
url = {http://www.cfr.washington.edu/classes.esrm.456/Woodwell.1967.pdf},
volume = {216},
year = {1967}
}
@phdthesis{CattinBlandenier2004,
author = {{Cattin Blandenier}, Marie-France},
pages = {1--127},
school = {Universite de Neuchatel},
title = {{Food web ecology: models and application to conservation}},
url = {http://grande-caricaie.ch/spip/IMG/pdf/these{\_}MFCattin.pdf},
year = {2004}
}
@article{Patricio2006,
abstract = {Three Ecopath with Ecosim models were constructed to represent the eutrophication gradient along the south arm of the Mondego estuary (Portugal). Sampling was conducted in three areas representative of different environmental situations along the gradient: (a) a non-eutrophic area (Zostera noltii meadows), (b) an intermediate eutrophic area (macrophyte absent, although residual roots can still be found in the sediment, and the occasional formation of abundant macroalgae mats) and (c) a strongly eutrophic area (macrophyte community totally absent for at least a decade and strong, regularly occurring, blooms of Ulva spp.). Field, laboratory and literature information were used to construct the models, as well as empirical ecological knowledge gained from years of work on this system. Approximately 76 trophic groups (e.g. Phytoplankton and Zooplankton species), species and genera were included. These species were grouped into 43, 36 and 34 model groups for Zostera sp. meadows, intermediate eutrophic area and strongly eutrophic area, respectively. The groups were arranged by trophic similarity and habitat preferences; special distinction is given to macrofauna. Biomass, production, consumption, and diet are among the parameters used to describe each group. The sum of consumptions, exports, respiration, production, flow to detritus, total system throughput and annual rate of net primary production was always higher in the Zostera sp. meadows, followed by the strongly eutrophic area and, finally, by the intermediate eutrophic area. ?? 2006 Elsevier B.V. All rights reserved.},
author = {Patr??cio, Joana and Marques, Jo??o Carlos},
doi = {10.1016/j.ecolmodel.2006.03.008},
isbn = {0304-3800},
issn = {03043800},
journal = {Ecological Modelling},
keywords = {Ecological model,Ecopath,Estuary,Eutrophication,Trophic structure},
month = {aug},
number = {1-2},
pages = {21--34},
title = {{Mass balanced models of the food web in three areas along a gradient of eutrophication symptoms in the south arm of the Mondego estuary (Portugal)}},
url = {http://linkinghub.elsevier.com/retrieve/pii/S0304380006001189},
volume = {197},
year = {2006}
}
@article{Warren1989,
author = {Warren, P H and Warren, P H},
journal = {Oikos},
keywords = {connectance},
number = {1},
pages = {299--311},
title = {{Spatial and temporal variation in the structure of a freshwater food web}},
url = {http://www.jstor.org/stable/3565588},
volume = {55},
year = {1989}
}
@article{Townsend1998,
abstract = {We studied food webs comprising fish, macroinvertebrates, and algae (identified to species or morphospecies) in small streams using a consistent methodology at the same spatial and temporal scales. Our aim was to test apriori hypotheses derived from dynamic-demographic and energetics models concerning the effects of disturbance and resource availability on food-web attributes. The regime of bed disturbance affecting the organisms in the webs was measured in 10 streams. We also derived measures of the supply of resources for animals in the webs in terms of algal primary productivity and detritus standing crop. Both web size and number of links per species were significantly negatively related to mean intensity of bed disturbance. Mean chain length had a significant positive relationship with algal primary productivity but not disturbance, No food-web attribute was related to detritus standing crop.},
author = {Townsend, Colin R and M, Ross and Mcintosh, Angus R},
doi = {10.1046/j.1461-0248.1998.00039.x},
isbn = {1461-023X},
issn = {1461023X (ISSN)},
journal = {Ecology Letters},
keywords = {1 ross m,colin r,connectance,detritus,disturbance,epilithon,fish,food chain length,food webs,macroinvertebrates,productivity,streams,townsend},
number = {3},
pages = {200--209},
pmid = {5284975},
title = {{Disturbance, resource supply, and food-web architecture in streams}},
url = {http://doi.wiley.com/10.1046/j.1461-0248.1998.00039.x},
volume = {1},
year = {1998}
}
@article{Polis1991a,
abstract = {Food webs in the real world are much more complex than food-web literature would have us believe.  This is illustrated by the web of the sand community in the Coachella Valley desert.  The biota include 174 species of vascular plants, 138 species of vertebrates, more than 55 species of arachnids, and an unknown (but great) number of microorganisms, insects (2,000-3,000 estimated species), acari, and nematodes.  Trophic relations are presented in a series of nested subwebs and delineations of the community.  Complexity arises from the large number of interactive species, the frequency of omnivory, age structure, looping, the lack of compartmentalization, and the complexity of the arthropod and soil faunas.  Web features found in the Coachella also characterize other communities and should produce equivalently complex webs.  If anything, diversity and complexity in most nondesert habitats are greater than those in deserts.  Patterns from the Coachella web are compared with theoretical predictions and "empirical generalizations" derived from catalogs of published webs.  The Coachella web differs greatly:  chains are longer, omnivory and loops are not rare, connectivity is greater (species interact with many more predators and prey), top predators are rare or nonexistent, and prey-to-predator ratios are greater than 1.0.  The evidence argues that actual community food webs are extraordinarily more complex than those webs cataloged by theorists.  I argue that most cataloged webs are oversimplified caricatures of actual communities.  That cataloged webs depict so few species, absurdly low ratios of predators on prey and prey eaten by predators, so few links, so little omnivory, a veritable absence of looping, and such a high proportion of top predators argues strongly that they poorly represent real biological communities.  Consequently, the practice of abstracting empirical regularities from such catalogs yields an inaccurate and artifactual view of trophic interactions within communities.  Contrary to strong assertions by many theorists, patterns from food webs of real communities generally do not support predictions arising from dynamic and graphic models of food-web structure.},
author = {Polis, Gary a.},
doi = {10.1086/285208},
isbn = {0003-0147},
issn = {0003-0147},
journal = {The American Naturalist},
number = {1},
pages = {123--155},
pmid = {2445},
title = {{Complex Trophic Interactions in Deserts: An Empirical Critique of Food-Web Theory}},
url = {http://www.journals.uchicago.edu/doi/10.1086/285208},
volume = {138},
year = {1991}
}
@article{Woodward2005,
abstract = {Body size determines a host of species traits that can affect the structure and dynamics of food webs, and other ecological networks, across multiple scales of organization. Measuring body size provides a relatively simple means of encapsulating and condensing a large amount of the biological information embedded within an ecological network. Recently, important advances have been made by incorporating body size into theoretical models that explore food web stability, the patterning of energy fluxes, and responses to perturbations. Because metabolic constraints underpin body-size scaling relationships, metabolic theory offers a potentially useful new framework within which to develop novel models to describe the structure and functioning of ecological networks and to assess the probable consequences of biodiversity change. ?? 2005 Elsevier Ltd. All rights reserved.},
author = {Woodward, Guy and Ebenman, Bo and Emmerson, Mark and Montoya, Jose M. and Olesen, Jens M. and Valido, Alfredo and Warren, Philip H.},
doi = {10.1016/j.tree.2005.04.005},
file = {:Users/alyssacirtwill/Documents/Papers/Woodward et al.{\_}2005{\_}Trends in Ecology and Evolution.pdf:pdf},
isbn = {0169-5347},
issn = {01695347},
journal = {Trends in Ecology and Evolution},
month = {jul},
number = {7},
pages = {402--409},
pmid = {16701403},
title = {{Body size in ecological networks}},
url = {http://www.ncbi.nlm.nih.gov/pubmed/16701403},
volume = {20},
year = {2005}
}
@article{Woodward2008,
author = {Woodward, Guy and Papantoniou, Georgia and Edwards, Francois and Lauridsen, Rasmus B},
doi = {10.1111/j.2008.0030-1299.16500.x},
isbn = {0521844460},
issn = {0030-1299},
journal = {Oikos},
pages = {683--692},
title = {{Trophic trickles and cascades in a complex food web: impacts of a keystone predator on stream community structure and ecosystem processes}},
volume = {117},
year = {2008}
}
@article{Williams2002,
abstract = {Feeding relationships can cause invasions, extirpations, and population fluctuations of a species to dramatically affect other species within a variety of natural habitats. Empirical evidence suggests that such strong effects rarely propagate through food webs more than three links away from the initial perturbation. However, the size of these spheres of potential influence within complex communities is generally unknown. Here, we show for that species within large communities from a variety of aquatic and terrestrial ecosystems are on average two links apart, with {\textgreater}95{\%} of species typically within three links of each other. Species are drawn even closer as network complexity and, more unexpectedly, species richness increase. Our findings are based on seven of the largest and most complex food webs available as well as a food-web model that extends the generality of the empirical results. These results indicate that the dynamics of species within ecosystems may be more highly interconnected and that biodiversity loss and species invasions may affect more species than previously thought.},
author = {Williams, Richard J and Berlow, Eric L. and Dunne, Jennifer a and a.-L. Barabasi and Martinez, Neo D.},
doi = {10.1073/pnas.192448799},
isbn = {0027-8424},
issn = {0027-8424},
journal = {Proceedings of the National Academy of Sciences},
keywords = {Animals,Ecology,Ecosystem,Feeding Behavior,Food Chain,Species Specificity},
month = {oct},
number = {20},
pages = {12913--12916},
pmid = {12235367},
title = {{Two degrees of separation in complex food webs}},
url = {http://www.pubmedcentral.nih.gov/articlerender.fcgi?artid=130559{\&}tool=pmcentrez{\&}rendertype=abstract{\%}5Cnhttp://www.pnas.org/cgi/doi/10.1073/pnas.192448799},
volume = {99},
year = {2002}
}
@article{Xiao2011,
abstract = {Power-law relationships are among the most well-studied functional relationships in biology. Recently the common practice of fitting power laws using linear regression (LR) on log-transformed data has been criticized, calling into question the conclusions of hundreds of studies. It has been suggested that nonlinear regression (NLR) is preferable, but no rigorous comparison of these two methods has been conducted. Using Monte Carlo simulations, we demonstrate that the error distribution determines which method performs better, with NLR better characterizing data with additive, homoscedastic, normal error and LR better characterizing data with multiplicative, heteroscedastic, lognormal error. Analysis of 471 biological power laws shows that both forms of error occur in nature. While previous analyses based on log-transformation appear to be generally valid, future analyses should choose methods based on a combination of biological plausibility and analysis of the error distribution. We provide detailed guidelines and associated computer code for doing so, including a model averaging approach for cases where the error structure is uncertain.},
author = {Packard, Gary C.},
doi = {10.1111/bij.12396},
isbn = {0012-9658},
issn = {10958312},
journal = {Biological Journal of the Linnean Society},
keywords = {Allometry,Scaling},
month = {oct},
number = {4},
pages = {1167--1178},
pmid = {22073779},
title = {{On the use of log-transformation versus nonlinear regression for analyzing biological power laws}},
url = {http://www.ncbi.nlm.nih.gov/pubmed/22073779},
volume = {113},
year = {2014}
}
@incollection{Harper-Smith2005,
abstract = {This chapter covers the concept of food web as a means to intuitively and synthetically visualize and communicate about complex interconnections among species in natural communities. Given this pervasive use of structural food webs as a teaching tool, it is perhaps surprising that they are rarely used in resource management to communicate about system-level impacts of human activities. Despite calls for more holistic approaches to "ecosystem management," a single species focus remains the norm. Broader considerations may include physical or biotic "habitat" protection, but rarely do they explicitly incorporate fundamental species interactions, such as feeding relations. In addition to qualitatively describing multivariate community changes in an intuitively tractable form, quantitative changes in web structural properties may offer novel insights into potential ecological consequences of these changes. Basic binary structural food webs provided a compact, visually accessible description of the dramatic, multivariate, community-level changes and recovery of alpine lakes in response to introduced trout. Stocked lakes with trout present had dramatically simplified food webs compared to lakes that never had fish. It has been well documented that species dynamics are difficult to infer from a simple knowledge of binary link structure alone. Similarly, nontrophic processes such as recruitment, abiotic conditions, disturbance, facilitations, and interference competition often regulate the presence and relative abundances of species in a community. ?? 2006 Elsevier Inc. All rights reserved.},
address = {London},
author = {Harper-Smith, Sarah and Berlow, Eric L. and Knapp, Roland a. and Williams, Richard J. and Martinez, Neo D.},
booktitle = {Dynamic Food Webs},
chapter = {Communicat},
doi = {10.1016/B978-012088458-2/50038-2},
editor = {de Ruiter, Peter C and Wolters, Volkmar and Moore, John C.},
isbn = {9780120884582},
pages = {407--423},
publisher = {Elsevier},
title = {{Communicating Ecology through Food Webs: Visualizing and Quantifying the Effects of Stocking Alpine Lakes with Trout}},
url = {http://www.wmrs.edu/people/BIOs/ericberlow/EricPubs/Harper et al 2005 visualizing effects of introduced trout sierra nevada.doc},
year = {2006}
}
@article{Tang2014,
abstract = {Food webs have markedly non-random network structure. Ecologists maintain that this non-random structure is key for stability, since large random ecological networks would invariably be unstable and thus should not be observed empirically. Here we show that a simple yet overlooked feature of natural food webs, the correlation between the effects of consumers on resources and those of resources on consumers, substantially accounts for their stability. Remarkably, random food webs built by preserving just the distribution and correlation of interaction strengths have stability properties similar to those of the corresponding empirical systems. Surprisingly, we find that the effect of topological network structure on stability, which has been the focus of countless studies, is small compared to that of correlation. Hence, any study of the effects of network structure on stability must first take into account the distribution and correlation of interaction strengths.},
author = {Tang, Si and Pawar, Samraat and Allesina, Stefano},
doi = {10.1111/ele.12312},
editor = {Holyoak, Marcel},
isbn = {1461-0248},
issn = {14610248},
journal = {Ecology Letters},
keywords = {Complexity,Food webs,Pairwise correlation,Population dynamics,Stability},
month = {jun},
number = {9},
pages = {1094--1100},
pmid = {24946877},
title = {{Correlation between interaction strengths drives stability in large ecological networks}},
url = {http://doi.wiley.com/10.1111/ele.12312},
volume = {17},
year = {2014}
}
@article{Albrecht2014,
abstract = {Compartmentalization - the organization of ecological interaction networks into subsets of species that do not interact with other subsets (true compartments) - has been identified as a key property for the functioning, stability and evolution of ecological communities. Invasions by entomophilous invasive plants may profoundly alter the way interaction networks are compartmentalized. We analysed a comprehensive dataset of 40 paired plant-pollinator networks (invaded versus uninvaded) to test this hypothesis. We show that invasive plants have higher generalization levels with respect to their pollinators than natives. The consequences for network topology are that - rather than displacing native species from the network - plant invaders attracting pollinators into invaded modules tend to play new important topological roles (i.e. network hubs, module hubs and connectors) and cause role shifts in native species, creating larger modules that are more connected among each other. While the number of true compartments was lower in invaded compared with uninvaded networks, the effect of invasion on modularity was contingent on the study system. Interestingly, the generalization level of invasive plants partially explains this pattern, with more generalized invaders contributing to a lower modularity. Our findings indicate that the altered interaction structure of invaded networks makes them more robust against simulated random secondary species extinctions, but more vulnerable when the typically highly connected invasive plants go extinct first. The consequences and pathways by which biological invasions alter the interaction structure of plant-pollinator communities highlighted in this study may have important dynamical and functional implications, for example, by influencing multi-species reciprocal selection regimes and coevolutionary processes.},
author = {Albrecht, Matthias and Padr{\'{o}}n, Benigno and Bartomeus, Ignasi and Traveset, Anna},
doi = {10.1098/rspb.2014.0773},
isbn = {1471-2954 (Electronic)$\backslash$r0962-8452 (Linking)},
issn = {1471-2954},
journal = {Proceedings of the Royal Society - B},
number = {June},
pages = {20140773},
pmid = {24943368},
title = {{Consequences of plant invasions on compartmentalization and species' roles in plant--pollinator networks}},
url = {http://rspb.royalsocietypublishing.org/content/281/1788/20140773.short},
volume = {281},
year = {2014}
}
@article{Heymans2002,
abstract = {Network analysis (NA) is used to compare two ecosystems with different spatial extents to understand the different patterns and dynamics that arise. NA allows one to study the system at different scales: At the level of bilateral interactions, input-output structure matrices are calculated to look at the direct and indirect effects that one flow has on another; at the functional level, the food web is mapped into a concatenated trophic chain, and all simple, directed biogeochemical cycles are identified and separated from the supporting dissipative flows; and at the system's level global variables describe the state of development of the total network. The systems in question are the Everglades graminoid marsh and the adjacent cypress swamp. The graminoid marsh is essentially a two-dimensional system, with reduced diversity of primary producers, and a more focussed dependency of higher trophic levels on one particular primary producer, the periphyton. Although the cypress swamp system contains most of the same flora and fauna as the graminoids, it extends into a third dimension, and contains additional forms of terrestrial vegetation that increase the diversity of primary production, and thereby the resilience of the ecosystem. The importance of detritus to both systems is marked, although recycling within detritus is far more important in the graminoids than in the cypress. The linkages to higher trophic levels are relatively fewer in the graminoids, and the diversity of interactions between the detritus and higher trophic levels is much higher in the cypress. Overall, the presence of a third dimension imparts diversity and resilience to the cypress system, although the faster turnover rates of the graminoids make them more productive. ?? 2002 Elsevier Science B.V. All rights reserved.},
author = {Heymans, J. J. and Ulanowicz, R. E. and Bondavalli, C.},
doi = {10.1016/S0304-3800(01)00511-7},
isbn = {0304-3800},
issn = {03043800},
journal = {Ecological Modelling},
keywords = {Cypress swamp,Ecosystem,Everglades,Graminoids,Network analysis,Scale},
month = {mar},
number = {1-2},
pages = {5--23},
title = {{Network analysis of the South Florida Everglades graminoid marshes and comparison with nearby cypress ecosystems}},
url = {http://linkinghub.elsevier.com/retrieve/pii/S0304380001005117},
volume = {149},
year = {2002}
}
@article{Baird1989,
abstract = {The full suite of carbon exchanges among the 36 most important components of the Chesapeake Bay mesohaline ecosystem is estimated to examine the seasonal trends in energy flow and the trophic dynamics of the ecosystem. The networks provide infor- mation on the rates of energy transfer between the trophic components in a system wherein autochthonous production is dominated by phytoplankton production. A key seasonal feature of the system is that the summer grazing of primary producers by zooplankton is greatly reduced due to top-down control of zooplankton by ctenophores and sea nettles. Some of the ungrazed phytoplankton is left to fuel the activities of the pelagic microbial community, and the remainder falls to the bottom where it augments the deposit-feeding assemblage of polychaetes, amphipods, and blue crabs. There is a dominant seasonal cycle in the activities of all subcommunities, which is greatest in the summer and least in the cold season. However, the overall topology of the ecosystem does not appear to change substantially from season to season. Matrix operations can be employed to assess the various direct and indirect pathways by which each trophic group obtains energy. Often, indirect linkages reveal interesting differences. For example, although the bluefish and striped bass are both piscivorous pred- ators, 63{\%} of bluefish intake depends indirectly on benthic organisms, whereas striped bass depends mainly on planktonic organisms. Nearly all higher trophic species exhibit signif- icant indirect dependencies upon the upper components of the microbial loop, especially during summer. The complicated trophic network can be mapped into an eight-level trophic chain in the sense of Lindeman. Such analysis reveals that detritivory is about 10 times greater than herbivorous grazing in the Chesapeake system and that 70{\%} of detritus results from internal recycle. Annual efficiencies of trophic levels decrease as one ascends the chain. Major seasonal shifts in trophic efficiencies at higher levels appear to be modulated by how effectively microscopic zooplankton (mostly ciliates) are cropped by their predators. Av- erage trophic efficiency is 9.6{\%}. Despite the existence of eight trophic levels, the average level at which each species feeds always remains below 5. One "pest" species (the coelen- terate Chrysaora quinquecirrha) feeds rather high on the trophic pyramid and may exert a heretofore unappreciated level of control on the planktonic food chain. The number of cycles present in the network is surprisingly few, despite the fact that a relatively large and seemingly constant amount (23.2{\%}) of total system activity is devoted to recycling. This combination of factors possibly indicates a stressed ecosystem. A study of the rate-limiting links in the seasonal networks of recycling of material within the plankton reconfirms the shift of predator control from crustaceous zooplankton in spring- time to the sea nettle (Chrysaora quinquecirrha) during summer months. The collection of cycles present in the system is disjoint; there is no overlap between the cycles among the planktonic community and the circulations among the deposit feeders and nekton. The filter-feeding benthos and fish do not participate in any cycling, but serve rather as bridges to shift carbon and energy from the planktonic community into the benthic-nektonic subsystems. Neither do most of the members of the microbial loop engage in any recycle of carbon, functioning instead as a dissipative shunt of energy out of the system.},
author = {Baird, Daniel and Ulanowicz, Robert E.},
doi = {10.2307/1943071},
isbn = {0012-9615},
issn = {00129615},
journal = {Ecological Monographs},
keywords = {ascendency,chesapeake bay,chrysaora quinquecirrha,dependency coefficients,ergetics,food,input-output analysis,microbial loop,nexus of cycles,population and community en-,predator control,recycling,seasonal activity levels,seasonal energy flows,structure of recycling,trophic pyramids,trophic structure,web structure},
number = {4},
pages = {329--364},
title = {{The seasonal dynamics of the Chesapeake Bay ecosystem}},
url = {http://www.esajournals.org/doi/abs/10.2307/1943071},
volume = {59},
year = {1989}
}
@article{Schleuning2012,
abstract = {Species-rich tropical communities are expected to be more specialized than their temperate counterparts [1-3]. Several studies have reported increasing biotic specialization toward the tropics [4-7], whereas others have not found latitudinal trends once accounting for sampling bias [8, 9] or differences in plant diversity [10, 11]. Thus, the direction of the latitudinal specialization gradient remains contentious. With an unprecedented global data set, we investigated how biotic specialization between plants and animal pollinators or seed dispersers is associated with latitude, past and contemporary climate, and plant diversity. We show that in contrast to expectation, biotic specialization of mutualistic networks is significantly lower at tropical than at temperate latitudes. Specialization was more closely related to contemporary climate than to past climate stability, suggesting that current conditions have a stronger effect on biotic specialization than historical community stability. Biotic specialization decreased with increasing local and regional plant diversity. This suggests that high specialization of mutualistic interactions is a response of pollinators and seed dispersers to low plant diversity. This could explain why the latitudinal specialization gradient is reversed relative to the latitudinal diversity gradient. Low mutualistic network specialization in the tropics suggests higher tolerance against extinctions in tropical than in temperate communities. {\textcopyright} 2012 Elsevier Ltd.},
author = {Schleuning, Matthias and Fr{\"{u}}nd, Jochen and Klein, Alexandra Maria and Abrahamczyk, Stefan and Alarc{\'{o}}n, Ruben and Albrecht, Matthias and Andersson, Georg K S and Bazarian, Simone and B{\"{o}}hning-Gaese, Katrin and Bommarco, Riccardo and Dalsgaard, Bo and Dehling, D. Matthias and Gotlieb, Ariella and Hagen, Melanie and Hickler, Thomas and Holzschuh, Andrea and Kaiser-Bunbury, Christopher N. and Kreft, Holger and Morris, Rebecca J. and Sandel, Brody and Sutherland, William J. and Svenning, Jens Christian and Tscharntke, Teja and Watts, Stella and Weiner, Christiane N. and Werner, Michael and Williams, Neal M. and Winqvist, Camilla and Dormann, Carsten F. and Bl{\"{u}}thgen, Nico},
doi = {10.1016/j.cub.2012.08.015},
isbn = {0960-9822},
issn = {09609822},
journal = {Current Biology},
keywords = {Animals,Biodiversity,Ecosystem,Genetic Variation,Plants,Plants: genetics,Pollination,Seed Dispersal,Symbiosis,Tropical Climate},
month = {oct},
number = {20},
pages = {1925--1931},
pmid = {22981771},
title = {{Specialization of mutualistic interaction networks decreases toward tropical latitudes}},
url = {http://www.ncbi.nlm.nih.gov/pubmed/22981771},
volume = {22},
year = {2012}
}
@article{Baiser2012,
abstract = {Aim The network structure of food webs plays an important role in the maintenance of diversity and ecosystem functioning in ecological communities. Previous research has found that ecosystem size, resource availability, assembly history and biotic interactions can potentially drive food web structure. However, the relative influence of climatic variables that drive broad-scale biogeographic patterns of species richness and composition has not been explored for food web structure. In this study, we assess the influence of broad-scale climatic variables in addition to known drivers of food web structure on replicate observations of a single aquatic food web, sampled from the leaves of the pitcher plant (Sarracenia purpurea), at different geographic sites across a broad latitudinal and climatic range. Location Using standardized sampling methods, we conducted an extensive `snapshot' survey of 780 replicated aquatic food webs collected from the leaves of the pitcher plant S. purpurea at 39 sites from northern Florida to Newfoundland and westward to eastern British Columbia. Methods We examined correlations of 15 measures of food web structure at the pitcher and site scales with geographic variation in temperature and precipitation, concentrations of nutrients from atmospheric nitrogen deposition, resource availability, ecosystem size and the abundance of the pitcher plant mosquito (Wyeomyia smithii), a potential keystone species. Results At the scale of a single pitcher plant leaf, linkage density, species richness, measures of chain length and the proportion of omnivores in a web all increased with pitcher volume. Linkage density and species richness were greater at high-latitude sites, which experience low mean temperatures and precipitation and high annual variation in both of these variables. At the site scale, variation in 8 of the 15 food web metrics decreased at higher latitudes, and variation in measures of chain length increased with the abundance of mosquitoes. Main conclusions Ecosystem size and climatic variables related to latitude were most strongly correlated with network structure of the Sarracenia food web. However, in spite of large sample sizes, thorough standardized sampling and the large geographic extent of the survey, even the best-fitting models explained less than 40{\%} of the variation in food web structure. In contrast to biogeographic patterns of species richness, food web structure was largely independent of broad-scale climatic variables. The large proportion of unexplained variance in our analyses suggests that stochastic assembly may be an important determinant of local food web structure.},
author = {Baiser, Benjamin and Gotelli, Nicholas J. and Buckley, Hannah L. and Miller, Thomas E. and Ellison, Aaron M.},
doi = {10.1111/j.1466-8238.2011.00705.x},
file = {:Users/alyssacirtwill/Documents/Papers/Baiser et al.{\_}2012{\_}Global Ecology and Biogeography.pdf:pdf},
isbn = {1978756615},
issn = {1466822X},
journal = {Global Ecology and Biogeography},
keywords = {Chain length,Climate,Food web,Keystone predation,Network structure,North America,Pitcher plant,Sarracenia purpurea},
month = {may},
number = {5},
pages = {579--591},
title = {{Geographic variation in network structure of a nearctic aquatic food web}},
url = {http://doi.wiley.com/10.1111/j.1466-8238.2011.00705.x},
volume = {21},
year = {2012}
}
@article{Cardillo2005,
author = {Park, Silwood and Park, Regents and Park, Silwood},
journal = {America},
keywords = {diversification,extinction,latitudinal diversity gradient,new world birds,speciation},
number = {9},
pages = {2278--2287},
title = {{TESTING FOR LATITUDINAL BIAS IN DIVERSIFICATION RATES : AN EXAMPLE USING NEW WORLD BIRDS Special Feature}},
url = {http://www.esajournals.org/doi/abs/10.1890/05-0112},
volume = {86},
year = {2005}
}
@article{Hawkins2004,
author = {Hawkins, Bradford A and Diniz‐Filho, Jose Alexandre Felizola},
file = {:Users/alyssacirtwill/Documents/Papers/Hawkins, Diniz‐Filho{\_}2004{\_}Ecography.pdf:pdf},
issn = {16000587},
journal = {Ecography},
number = {2},
pages = {268--272},
title = {{'Latitude' and geographic patterns in species richness}},
url = {http://onlinelibrary.wiley.com/doi/10.1111/j.0906-7590.2004.03883.x/full},
volume = {27},
year = {2004}
}
@article{Lappalainen2006,
abstract = {The latitudinal gradient in diversity is widely acknowledged, but the mechanisms contributing to this pattern are still poorly known. Given that the species have environmental optima, a central issue is how species' niche parameters, i.e. niche breadth and niche position, vary along the latitudinal gradient. In this study, we examined the determinants of fish distribution and the variability in species' niche breadth and position along latitudinal gradient using a regional data set of boreal lakes. Results of the Outlying Mean Index analysis showed that the fish community structure was jointly controlled by a number of environmental factors, ranging from water chemistry and temperature to local physical factors such as lake area and depth. Corroborating the number of earlier findings, the regional occupancy of species was more strongly governed by the niche position than the niche breadth, although both showed a significant relationship with the regional distribution. When the latitudinal variability in niche parameters of the main taxonomic groups was analysed, both percids and cyprinids, being cool water species, showed significant decrease in niche breadth northwards as we predicted. By contrast, the niche position and latitude were non-significantly correlated in percids and salmonids, and negatively correlated in cyprinids, the latter showing the opposite pattern as we predicted. However, even if only a part of our predictions was supported, the results generally implied that the examination of latitudinal variability in the niche properties is potentially highly rewarding, not only in estimation of present community structure in lakes but also for predictions of species' responses to climate change.},
author = {Lappalainen, Jyrki and Soininen, Janne},
doi = {10.1007/s00114-006-0093-2},
isbn = {0028-1042},
issn = {00281042},
journal = {Naturwissenschaften},
keywords = {Animals,Environment,Finland,Fishes,Fishes: anatomy {\&} histology,Fresh Water,Population Density,Temperature,Water,Water: analysis},
month = {may},
number = {5},
pages = {246--250},
pmid = {16538374},
title = {{Latitudinal gradients in niche breadth and position - Regional patterns in freshwater fish}},
url = {http://www.ncbi.nlm.nih.gov/pubmed/16538374},
volume = {93},
year = {2006}
}
@article{Vazquez2004b,
abstract = {We examine Robert MacArthur's hypothesis that niche breadth is positively associated with latitude (the latitude-niche breadth hypothesis). This idea has been influential and long standing, yet no studies have evaluated its generality or the validity of its assumptions. We review the theoretical arguments suggesting a positive relationship between niche breadth and latitude. We also use available evidence to evaluate the assumptions and predictions of MacArthur's latitude-niche breadth hypothesis. We find that neither the assumptions nor the predictions of the hypothesis are supported by data. We propose an alternative hypothesis linking latitude with niche breadth. Unlike previous ideas, our conceptual framework does not require equilibrial assumptions and is based on recently uncovered patterns of species interactions.},
author = {V{\'{a}}zquez, D P and Stevens, R D},
doi = {10.1086/421445},
isbn = {0003-0147},
issn = {0003-0147},
journal = {American Naturalist},
keywords = {latitudinal gradient,niche breadth},
month = {jul},
number = {1},
pages = {E1--E19},
pmid = {15266376},
title = {{The latitudinal gradient in niche breadth: concepts and evidence}},
url = {http://www.ncbi.nlm.nih.gov/pubmed/15266376},
volume = {164},
year = {2004}
}
@article{Letcher1994,
abstract = {Rapoport's rule, the tendency for species' geographical range sizes to increase from the equator to the poles, holds for Palearctic mammals under four methods of analysis, each of which makes different statistical assumptions. The climatic variability hypothesis for explaining Rapoport's rule is tested. In support of the hypothesis, annual temperature range is a good predictor of range size. However, latitude is an even better predictor of range size. Because meridians do not cause ecological patterns, some other factor, for which latitude is a surrogate measure, must also determine range size; habitat size may be one such factor.},
author = {Letcher, Andy J. and Harvey, Paul H.},
doi = {10.1086/285659},
isbn = {0003-0147},
issn = {0003-0147},
journal = {The American Naturalist},
number = {1},
pages = {30},
title = {{Variation in Geographical Range Size Among Mammals of the Palearctic}},
url = {http://cat.inist.fr/?aModele=afficheN{\&}cpsidt=4132643},
volume = {144},
year = {1994}
}
@article{Fernandez2005,
abstract = {Aim One of the mechanisms proposed to explain the tendency for geographical range size to increase from the equator to the poles, known as the Rapoport effect, is the climatic variability hypothesis. It states that, towards higher latitudes, greater seasonal climatic variability is the most important pressure that selectively promotes greater general climatic tolerance of species, and therefore also more extensive species ranges. In order to test this hypothesis, we explore the influence of climate, area and biome diversity on the latitudinal gradient of climatic specialization. Location The study used the large mammal assemblage from Africa. Methods The degree of climatic specialization of African large mammals (Primates, Carnivora, Proboscidea, Perissodactyla, Hyracoidea, Tubulidentata, Artiodactyla and Pholidota) is investigated using the biomic specialization index (BSI) for each mammal species, based on the number of biomes it inhabits. We studied the influence of 11 climatic and biogeographical predictors in the latitudinal pattern of biomic specialization. Stepwise multiple regressions were used to identify the strongest predictors of biomic specialization in Africa and, separately, in both continental hemispheres. We also studied differences among taxonomical groups (primates, carnivores and artiodactyls). We used correlograms generated using Moran's I coefficients to control for spatial autocorrelation in all these analyses. Results Average BSI values for successive 1 degrees-latitude bands generally decline towards the equator and temperature variability emerged as the most predictive factor in the regression model for the whole continent, thus supporting the climatic variability hypothesis. Nevertheless, there are differences between hemispheres and among taxa. While temperature variability is the most important predictor of latitudinal variability in biomic specialization in most of the regression models for the northern hemisphere, continental area for each latitudinal band is the best predictor in all the regression models in the southern hemisphere. Main conclusions It appears that similar patterns in latitudinal variation in average BSI may be caused by different factors in the two hemispheres. We suggest that the strong north-south geographical asymmetry of Africa, which influences its biogeographical structure, and the presence of land connections with Eurasia in the northern hemisphere are responsible for the observed patterns. Our data illustrate the influence of continental biogeographical structure and history on macroecological patterns.},
author = {Fern{\'{a}}ndez, Manuel Hern{\'{a}}ndez and Vrba, Elisabeth S.},
doi = {10.1111/j.1365-2699.2004.01188.x},
isbn = {0305-0270},
issn = {03050270},
journal = {Journal of Biogeography},
keywords = {Artiodactyla,Bioclimatology,Biogeography,Biome,Carnivora,Ecological pattern,Ecological specialization,Macroecology,Mammalia,Primates},
month = {apr},
number = {5},
pages = {903--918},
pmid = {6397},
title = {{Rapoport effect and biomic specialization in African mammals: Revisiting the climatic variability hypothesis}},
url = {http://doi.wiley.com/10.1111/j.1365-2699.2004.01188.x},
volume = {32},
year = {2005}
}
@article{Vazquez2004a,
abstract = {We examine Robert MacArthur's hypothesis that niche breadth is positively associated with latitude (the latitude-niche breadth hypothesis). This idea has been influential and long standing, yet no studies have evaluated its generality or the validity of its assumptions. We review the theoretical arguments suggesting a positive relationship between niche breadth and latitude. We also use available evidence to evaluate the assumptions and predictions of MacArthur's latitude-niche breadth hypothesis. We find that neither the assumptions nor the predictions of the hypothesis are supported by data. We propose an alternative hypothesis linking latitude with niche breadth. Unlike previous ideas, our conceptual framework does not require equilibrial assumptions and is based on recently uncovered patterns of species interactions.},
author = {V{\'{a}}zquez, D P and Stevens, R D},
doi = {10.1086/421445},
isbn = {0003-0147},
issn = {0003-0147},
journal = {American Naturalist},
keywords = {latitudinal gradient,niche breadth},
month = {jul},
number = {1},
pages = {E1--E19},
pmid = {15266376},
title = {{The latitudinal gradient in niche breadth: concepts and evidence}},
url = {http://www.ncbi.nlm.nih.gov/pubmed/15266376},
volume = {164},
year = {2004}
}
@article{Romanuk2006a,
abstract = {Using a combination of stable isotope analysis of delta13C and delta15N and long-term census data on population abundances for meiofauna in tropical aquatic rock pools, we provide evidence that species which exhibit greater variation in delta13C, an indication of a greater range of distinct carbon sources in their diet, have more stable populations than species with lower variation in delta13C. This link between increased isotope variability and reduced population variability, however, did not hold for delta15N. This suggests that increases in population stability were due to non-omnivorous feeding on multiple carbon sources within a trophic level rather than omnivorous feeding on multiple carbon sources across trophic levels. Our findings corroborate MacArthur's original hypothesis that populations that can access a greater range of resources are more stable than those which consume a more restricted range of resources.},
author = {Romanuk, Tamara N and Beisner, Beatrix E and Martinez, Neo D and Kolasa, Jurek},
doi = {10.1098/rsbl.2006.0464},
isbn = {1744-9561},
issn = {1744-9561},
journal = {Biology letters},
keywords = {2002,an increase in species,generality,omnivory,population dynamics,richness could,rock pools,second,stable isotope analysis,zooplankton},
month = {sep},
number = {3},
pages = {374--377},
pmid = {17148407},
title = {{Non-omnivorous generality promotes population stability.}},
url = {http://www.pubmedcentral.nih.gov/articlerender.fcgi?artid=1686196{\&}tool=pmcentrez{\&}rendertype=abstract},
volume = {2},
year = {2006}
}
@article{Taylor1980,
author = {{Taylor J D., Morris N J.}, Taylor C N.},
journal = {Paleontology},
pages = {375--409},
title = {{Taylor et al. 1980.pdf Food Specialization and the evolution of predatory prosobranch gastropods}},
url = {http://scholar.google.com/scholar?hl=en{\&}btnG=Search{\&}q=intitle:Food+specialization+and+the+evolution+of+predatory+prosobranch+gastropods{\#}0},
volume = {23},
year = {1980}
}
@article{Gittleman1985,
abstract = {Variation in body size (weight) is examined across the order Carnivora in relation to taxonomy (phylo- geny), latitude, habitat, zonation, activity cycle, diet, prey size, and prey diversity. Significant differences in body weight are observed with respect to family membership. Some of these differences may be explained by phylogenetic history and/or dietary effects. Body weight is not correlated with habitat, zonation, activity cycle or latitudinal gra- dients. Significant differences in body weight are found among insectivorous, herbivorous and carnivorous species, and some of these differences may relate to energetic con- straints. Among predatory carnivores, prey size and diver- sity increases with body weight. The adaptive significance, both intra- and inter-specifically, of prey characteristics (size, availability, diversity) and carnivore body weight qua- lities (strength, endurance, hunting technique) is discussed.},
author = {Gittleman, John L.},
doi = {10.1007/BF00790026},
isbn = {00298549},
issn = {00298549},
journal = {Oecologia},
number = {4},
pages = {540--554},
pmid = {6549},
title = {{Carnivore body size: Ecological and taxonomic correlates}},
url = {http://link.springer.com/10.1007/BF00790026},
volume = {67},
year = {1985}
}
@article{Olsson2000,
abstract = {This simulation study demonstrates how the choice of estimation method affects indexes of fit and parameter bias for different sample sizes when nested models vary in terms of specification error and the data demonstrate different levels of kurtosis. Using a fully crossed design, data were generated for 11 conditions of peakedness, 3 conditions of misspecification, and 5 different sample sizes. Three estimation methods (maximum likelihood [ML], generalized least squares [GLS], and weighted least squares [WLS]) were compared in terms of overall fit and the discrepancy between estimated parameter values and the true parameter values used to generate the data. Consistent with earlier findings, the results show that ML compared to GLS under conditions of misspecification provides more realistic indexes of overall fit and less biased parameter values for paths that overlap with the true model. However, despite recommendations found in the literature that WLS should be used when data are not normally distributed, we find that WLS under no conditions was preferable to the 2 other estimation procedures in terms of parameter bias and fit. In fact, only for large sample sizes (N = 1,000 and 2,000) and mildly misspecified models did WLS provide estimates and fit indexes close to the ones obtained for ML and GLS. For wrongly specified models WLS tended to give unreliable estimates and over-optimistic values of fit.},
author = {Olsson, Ulf Henning and Foss, Tron and Troye, Sigurd V. and Howell, Roy D.},
doi = {10.1207/S15328007SEM0704_3},
isbn = {1070-5511},
issn = {1070-5511},
journal = {Structural Equation Modeling: A Multidisciplinary Journal},
month = {oct},
number = {4},
pages = {557--595},
pmid = {680},
title = {{The Performance of ML, GLS, and WLS Estimation in Structural Equation Modeling Under Conditions of Misspecification and Nonnormality}},
url = {http://www.tandfonline.com/doi/abs/10.1207/S15328007SEM0704{\_}3},
volume = {7},
year = {2000}
}
@article{Hallett2014,
abstract = {Understanding how biotic mechanisms confer stability in variable environments is a fundamental quest in ecology, and one that is becoming increasingly urgent with global change. Several mechanisms, notably a portfolio effect associated with species richness, compensatory dynamics generated by negative species covariance and selection for stable dominant species populations can increase the stability of the overall community. While the importance of these mechanisms is debated, few studies have contrasted their importance in an environmental context. We analyzed nine long-term data sets of grassland species composition to investigate how two key environmental factors, precipitation amount and variability, may directly influence community stability and how they may indirectly influence stability via biotic mechanisms. We found that the importance of stability mechanisms varied along the environmental gradient: strong negative species covariance occurred in sites characterized by high precipitation variability, whereas portfolio effects increased in sites with high mean annual precipitation. Instead of questioning whether compensatory dynamics are important in nature, our findings suggest that debate should widen to include several stability mechanisms and how these mechanisms vary in importance across environmental gradients.},
author = {Hallett, Lauren M. and Hsu, Joanna S. and Cleland, Elsa E. and Collins, Scott L. and Dickson, Timothy L. and Farrer, Emily C. and Gherardi, Laureano a. and Gross, Katherine L. and Hobbs, Richard J. and Turnbull, Laura and Suding, Katharine N.},
doi = {10.1890/13-0895.1},
isbn = {0012-9658},
issn = {00129658},
journal = {Ecology},
keywords = {Compensatory dynamics,Dominant species,LTER,Mean-variance scaling,Negative covariance,Portfolio effect,Taylor's power law},
number = {6},
pages = {1693--1700},
pmid = {25039233},
title = {{Biotic mechanisms of community stability shift along a precipitation gradient}},
url = {http://www.esajournals.org/doi/abs/10.1890/13-0895.1},
volume = {95},
year = {2014}
}
@article{Santos2010,
abstract = {We conducted a comparative analysis of bee-plant and wasp-plant interaction networks, aiming at the identification of similarities and differences between networks of flower-visiting groups with direct or indirect mutualism with plants. We measured for each network: number of social bees and social wasps, number of plants visited (P), degree of nestedness, number of observed (I) and possible interactions, connectance (C), and interaction density (D). The network formed by pooling together social bees and social wasps exhibited 25 species (12 bees and 13 wasps) and 49 visited plants, with a connectance of 15.34{\%}. The wasp-plant network had higher connectance (C = 21.24) than the bee-plant network (C = 15.79). Both the social wasp-plant and the social bee-plant network were significantly nested, they presented structure more nested than all randomly generated matrices (n = 1000). Both interaction networks have similar topologies and are nested, asymmetrical and modular structures. {\textcopyright} 2010 INRA/DIB-AGIB/EDP Sciences.},
author = {Santos, Gilberto M. De Mendon{\c{c}}a and Aguiar, C{\^{a}}ndida M. Lima and a.R. Mello, Marco},
doi = {10.1051/apido/2009081},
isbn = {0044-8435},
issn = {0044-8435},
journal = {Apidologie},
month = {jan},
number = {4},
pages = {466--475},
title = {{Flower-visiting guild associated with the Caatinga flora: trophic interaction networks formed by social bees and social wasps with plants}},
url = {http://link.springer.com/10.1051/apido/2009081},
volume = {41},
year = {2010}
}
@article{Hanna2014a,
abstract = {Plant-pollinator mutualisms are disrupted by a variety of competitive interactions between introduced and native floral visitors. The invasive western yellowjacket wasp, Vespula pensylvanica, is an aggressive nectar thief of the dominant endemic Hawaiian tree species, Metrosideros polymorpha. We conducted a large-scale, multiyear manipulative experiment to investigate the impacts of V. pensylvanica on the structure and behavior of the M. polymorpha pollinator community, including competitive mechanisms related to resource availability. Our results demonstrate that V. pensylvanica, through both superior exploitative and interference competition, influences resource partitioning and displaces native and nonnative M. polymorpha pollinators. Furthermore, the restructuring of the pollinator community due to V. pensylvanica competition and predation results in a significant decrease in the overall pollinator effectiveness and fruit set of M. polymorpha. This research highlights both the competitive mechanisms and contrasting effects of social insect invaders on plant-pollinator mutualisms and the role of competition in pollinator community structure.},
author = {Hanna, Cause and Foote, David and Kremen, Claire},
doi = {10.1890/13-1276.1},
isbn = {0012-9658},
issn = {00129658},
journal = {Ecology},
keywords = {Apis mellifera,Bees,Community structure,Competition,Hawaii,Honey bee,Hylaeus spp.,Invasive species,Metrosideros polymorpha,Mutualism,Pollination,Resource partitioning,Vespula pensylvanica},
number = {6},
pages = {1622--1632},
pmid = {25039226},
title = {{Competitive impacts of an invasive nectar thief on plant-pollinator mutualisms}},
volume = {95},
year = {2014}
}
@article{Lau2014,
abstract = {Dystrophic lakes are widespread in temperate regions and intimately interact with surrounding terrestrial ecosystems in energy and nutrient dynamics, yet the relative importance of autochthonous and allochthonous resources to consumer production in dystrophic lakes remains controversial. We argue that allochthonous organic matter quantitatively dominates over photosynthetic autotrophs in dystrophic lakes, but that autotrophs are higher in diet quality and more important for consumers as they contain essential polyunsaturated fatty acids (PUFA). In a field study, we tested the hypotheses that (1) autochthonous primary production is the main driver for consumer production, despite being limited by light availability and low nutrient supplies, and greater supply of allochthonous carbon, (2) the relative contribution of autotrophs to consumers is directly related to their tissue PUFA concentrations, and (3) methane-oxidizing bacteria (MOB) provide an energy alternative for consumers. Pelagic and benthic consumer taxa representing different trophic levels were sampled from five dystrophic lakes: isopod Asellus aquaticus, megalopteran Sialis lutaria, dipteran Chaoborus flavicans, and perch Perca fluviatilis. Based on carbon and nitrogen stable isotopes, the relative contributions of autochthonous (biofilms and seston) and allochthonous (coarse particulate and dissolved organic matter) resources and MOB to these taxa were 47–79{\%}, 9–44{\%} and 7–12{\%} respectively. Results from fatty acid (FA) analyses show that the relative $\omega$3-FA and PUFA concentrations increased with trophic level (Asellus {\textless} Sialis and Chaoborus {\textless} Perca). Also, eicosapentaenoic-acid (EPA), $\omega$3-FA and PUFA concentrations increased with the autochthonous contribution in consumers, i.e., a 47–79{\%} biofilm and/or seston diet resulted in tissue EPA of 4.2–18.4, $\omega$3 FAs of 11.6–37.0 and PUFA of 21.6–61.0 mg/g dry mass. The results indicate that consumers in dystrophic lakes predominantly rely on energy from autotrophs and that their PUFA concentrations are dependent on the relative contribution of these autochthonous resources. The limited energy support from MOB suggests they are not negligible and are potentially an integral part of the food webs. Our findings show that autochthonous resources are the main driver of secondary production even in dystrophic lakes and offer new insights into the functioning of these ecosystems.},
author = {Lau, Danny C P and Sundh, Ingvar and Vrede, Tobias and Pickova, Jana and Goedkoop, Willem},
doi = {10.1890/13-1141.1},
isbn = {0012-9658},
issn = {00129658},
journal = {Ecology},
keywords = {Algae,Aquatic food webs,Fatty acids,Fish,Invertebrates,IsoSource,SIAR,Stable isotopes,Trophic transfer},
number = {6},
pages = {1506--1519},
pmid = {25039216},
title = {{Autochthonous resources are the main driver of consumer production in dystrophic boreal lakes}},
volume = {95},
year = {2014}
}
@article{Scharnweber2014,
abstract = {Lake ecosystems are strongly linked to their terrestrial surroundings by material and energy fluxes across ecosystem boundaries. However, the contribution of terrestrial particulate organic carbon (tPOC) from annual leaf fall to lake food webs has not yet been adequately traced and quantified. In this study, we conducted whole-lake experiments to trace artificially added tPOC through the food webs of two shallow lakes of similar eutrophic status, but featuring alternative stable regimes (macrophyte rich vs. phytoplankton dominated). Lakes were divided with a curtain, and maize (Zea mays) leaves were added, as an isotopically distinct tPOC source, into one half of each lake. To estimate the balance between autochthonous carbon fixation and allochthonous carbon input, primary production and tPOC and tDOC (terrestrial dissolved organic carbon) influx were calculated for the treatment sides. We measured the stable isotope ratios of carbon ($\delta$13C) of about 800 samples from all trophic consumer levels and compared them between lake sides, lakes, and three seasons. Leaf litter bag experiments showed that added maize leaves were processed at rates similar to those observed for leaves from shoreline plants, supporting the suitability of maize leaves as a tracer. The lake-wide carbon influx estimates confirmed that autochthonous carbon fixation by primary production was the dominant carbon source for consumers in the lakes. Nevertheless, carbon isotope values of benthic macroinvertebrates were significantly higher with maize additions compared to the reference side of each lake. Carbon isotope values of omnivorous and piscivorous fish were significantly affected by maize additions only in the macrophyte-dominated lake and $\delta$13C of zooplankton and planktivorous fish remained unaffected in both lakes. In summary, our results experimentally demonstrate that tPOC in form of autumnal litterfall is rapidly processed during the subsequent months in the food web of shallow lakes and is channeled to secondary and tertiary consumers predominantly via the benthic pathways. A more intense processing of tPOC seems to be connected to a higher structural complexity in littoral zones, and hence may differ between shallow lakes of alternative stable states.},
author = {Scharnweber, K. and Syv??ranta, J. and Hilt, S. and Brauns, M. and Vanni, M. J. and Brothers, S. and K??hler, J. and Kne??ev??c-Jar??c, J. and Mehner, T.},
doi = {10.1890/13-0390.1},
isbn = {0012-9658},
issn = {00129658},
journal = {Ecology},
keywords = {Allochthony,Omnivorous fish,Shallow lakes,Stable isotope analysis,Terrestrial carbon,Whole-lake experiment},
number = {6},
pages = {1496--1505},
pmid = {25039215},
title = {{Whole-lake experiments reveal the fate of terrestrial particulate organic carbon in benthic food webs of shallow lakes}},
volume = {95},
year = {2014}
}
@article{Adams2014,
abstract = {Phylogenetic signal is the tendency for closely related species to display similar trait values due to their common ancestry. Several methods have been developed for quantifying phylogenetic signal in univariate traits and for sets of traits treated simultaneously, and the statistical properties of these approaches have been extensively studied. However, methods for assessing phylogenetic signal in high-dimensional multivariate traits like shape are less well developed, and their statistical performance is not well characterized. In this article, I describe a generalization of the K statistic of Blomberg et al. (2003) that is useful for quantifying and evaluating phylogenetic signal in highly-dimensional multivariate data. The method (Kmult) is found from the equivalency between statistical methods based on covariance matrices and those based on distance matrices. Using computer simulations based on Brownian motion, I demonstrate that the expected value of Kmult remains at 1.0 as trait variation among species is increased or decreased, and as the number of trait dimensions is increased. By contrast, estimates of phylogenetic signal found with a squared-change parsimony procedure for multivariate data change with increasing trait variation among species and with increasing numbers of trait dimensions, confounding biological interpretations. I also evaluate the statistical performance of hypothesis testing procedures based on Kmult and find that the method displays appropriate Type I error and high statistical power for detecting phylogenetic signal in high-dimensional data. Statistical properties of Kmult were consistent for simulations using bifurcating and random phylogenies, for simulations using different numbers of species, for simulations that varied the number of trait dimensions, and for different underlying models of trait covariance structure. Overall these findings demonstrate that Kmult provides a useful means of evaluating phylogenetic signal in high-dimensional multivariate traits. Finally, I illustrate the utility of the new approach by evaluating the strength of phylogenetic signal for head shape in a lineage of Plethodon salamanders.},
author = {Adams, Dean C.},
doi = {10.1093/sysbio/syu030},
isbn = {1063-5157},
issn = {1076836X},
journal = {Systematic Biology},
keywords = {Geometric morphometrics,Macroevolution,Morphological evolution,Phylogenetic comparative method},
number = {5},
pages = {685--697},
pmid = {24789073},
title = {{A generalized K statistic for estimating phylogenetic signal from shape and other high-dimensional multivariate data}},
url = {http://www.ncbi.nlm.nih.gov/pubmed/24789073},
volume = {63},
year = {2014}
}
@article{Wilks1938,
abstract = {Project Euclid - mathematics and statistics online},
author = {Wilks, S. S.},
doi = {10.1214/aoms/1177732360},
isbn = {0003-4851},
issn = {0003-4851, 2168-8990},
journal = {The Annals of Mathematical Statistics},
number = {1},
pages = {60--62},
title = {{The Large-Sample Distribution of the Likelihood Ratio for Testing Composite Hypotheses}},
url = {http://projecteuclid.org/euclid.aoms/1177732360{\%}5Cnhttps://projecteuclid.org/euclid.aoms/1177732360},
volume = {9},
year = {1938}
}
@article{Searls2012,
abstract = {The success of online courseware such as that offered by the Massachusetts Institute of Technology (MIT) (http:// ocw.mit.edu) and now by many other institutions, together with a plethora of recent announcements of major new initiatives in this arena such as Coursera (https://www.coursera.org), Udacity (http://www.udacity.com), and the Har- vard-MIT partnership edX (http://www. edxonline.org), have made it clear that online learning has reached a tipping point. Many signs point to the possibility in the near future of getting a quality, university-level education at a distance, and for free. As exciting as this prospect may be, it behooves online students to follow a few simple rules for getting the most out of the experience, while being realistic in their expectations, as outlined below.},
author = {Searls, David B.},
doi = {10.1371/journal.pcbi.1002631},
isbn = {1553-7358 (Electronic)$\backslash$r1553-734X (Linking)},
issn = {1553734X},
journal = {PLoS Computational Biology},
keywords = {Algorithms,Computational Biology,Computational Biology: education,Computer-Assisted Instruction,Computer-Assisted Instruction: methods,Curriculum,Internet,Online Systems,Teaching,Teaching: methods},
month = {jan},
number = {9},
pages = {e1002631},
pmid = {23028268},
title = {{Ten Simple Rules for Online Learning}},
url = {http://www.pubmedcentral.nih.gov/articlerender.fcgi?artid=3441493{\&}tool=pmcentrez{\&}rendertype=abstract},
volume = {8},
year = {2012}
}
@article{Fischer2014,
abstract = {There is increasing evidence that rapid phenotypic evolution can strongly influence population dynamics, but how are such eco-evolutionary dynamics influenced by the source of trait variation (i.e., genetic variation or phenotypic plasticity)? To investigate this, we used rotifer?algae microcosm experiments to test how the phenotypic and genetic composition of prey populations affect predator?prey population dynamics. We chose four genetically distinct strains of the green alga Chlamydomonas reinhardtii that varied in their growth rate, standing levels of defense, and inducible defense. To additionally test for strain specificity of plasticity responses, we quantified protein expression of each strain in the presence and absence of rotifer predators (Brachionus calyciflorus). We then tested how different strain combinations influenced the outcome of pairwise competition trials with and without rotifer predation. We tracked individual strain frequencies using quantitative polymerase chain reaction (qPCR), and compared the observed dynamics to a suite of eco-evolutionary models of varying complexity. We found that variation in trade-offs between growth and defense between algal strains strongly influenced the outcome of competition and the overall predator?prey dynamics. Our purely ecological model of the observed dynamics, which allowed for the presence of phenotypic plasticity but no trait variation between strains, never outperformed any of our eco-evolutionary models in which strains could have different trait values. Our best fitting eco-evolutionary model allowed strains to differ in an inducible defense trait. Overall, our results provide some of the first experimental evidence that variation in phenotypically plastic responses among prey genotypes can be an important component of eco-evolutionary dynamics in a predator?prey system.},
author = {Fischer, Beat B. and Kwiatkowski, Marek and Ackermann, Martin and Krismer, Jasmin and Roffler, Severin and Suter, Marc J F and Eggen, Rik I L and Matthews, Blake},
doi = {10.1890/14-0116.1},
isbn = {0012-9658},
issn = {00129658},
journal = {Ecology},
keywords = {Coexistence,Competition,Eco-evolutionary dynamics and feedbacks,Induced defense,Intrapopulation diversity,Phenotypic plasticity,Predator-prey systems,Proteomics,Rapid evolution,Tradeoffs},
month = {may},
number = {11},
pages = {3080--3092},
title = {{Phenotypic plasticity influences the eco-evolutionary dynamics of a predator-prey system}},
url = {http://www.esajournals.org/doi/abs/10.1890/14-0116.1},
volume = {95},
year = {2014}
}
@article{Wolkovich,
abstract = {Recent advances in food web ecology highlight that most real food webs (1) represent an interplay between producer and detritus-based webs and, (2) are governed by consumers which are rampant omnivores-feeding on varied prey across trophic levels and resource channels. A possible avenue to unify these advances comes from models demonstrating that predators feeding on distinctly different channels may stabilize food webs. Empirical studies suggest many consumers engage in such behavior by feeding on prey items from both living-autotroph (green) and detritus-based (brown) webs-what we term 'multi-channel feeding'-yet we know little about how common such feeding is across systems and trophic levels, or its effect on system stability. Considering 23 empirical webs, we find that multi-channel feeding is equally common across terrestrial, freshwater and marine systems, with {\textgreater}50{\%} of consumers classified as multi-channel consumers. Multi-channel feeding occurred most often at the first consumer level, indicating ...},
author = {Wolkovich, Elizabeth M. and Allesina, Stefano and Cottingham, Kathryn L. and Moore, John C. and Sandin, Stuart a. and {De Mazancourt}, Claire},
doi = {10.1890/13-1721.1.sm},
isbn = {0012-9658},
issn = {00129658},
journal = {Ecology},
keywords = {Attack rates,Brown world,Detritus,Food webs,Green world,Multichannel,Stability},
number = {12},
pages = {3376--3386},
title = {{Linking the green and brown worlds: The prevalence and effect of multichannel feeding in food webs}},
url = {http://www.esajournals.org/doi/abs/10.1890/13-1721.1},
volume = {95},
year = {2014}
}
@article{Dibble,
abstract = {Intraspecific variation may shape colonization of new habitat patches through a variety of mechanisms. In particular, trait variation among colonizing individuals can produce intraspecific priority effects (IPEs), where early arrivers of a single species affect the establishment or growth of later conspecifics. While we have some evidence for the importance of IPEs, we lack a general understanding of factors affecting their presence or magnitude across a landscape. Specifically, IPEs should depend strongly on success of colonizers in the new habitat patch. This success hinges on interactions between colonizer traits and local selective pressures, but such context dependence remains unexplored experimentally. We addressed this gap by looking for the dynamical signature of IPEs in environments with and without a selective (parasite) pressure. We tested whether IPEs affected the population dynamics of a zooplankton host species (Daphnia dentifera) collected from two populations showing a tradeoff between gro...},
author = {Dibble, Christopher J. and Hall, Spencer R. and Rudolf, Volker H W},
doi = {10.1890/13-1958.1.sm},
issn = {00129658},
journal = {Ecology},
keywords = {Colonization,Daphnia,Host-parasite dynamics,Intraspecific variation,Metapopulation dynamics,Metschnikowia bicuspidata,Spatial structure},
number = {12},
pages = {3354--3363},
title = {{Intraspecific priority effects and disease interact to alter population growth}},
volume = {95},
year = {2014}
}
@article{Ives2007a,
abstract = {Most phylogenetically based statistical methods for the analysis of quantitative or continuously varying phenotypic traits assume that variation within species is absent or at least negligible, which is unrealistic for many traits. Within-species variation has several components. Differences among populations of the same species may represent either phylogenetic divergence or direct effects of environmental factors that differ among populations (phenotypic plasticity). Within-population variation also contributes to within-species variation and includes sampling variation, instrument-related error, low repeatability caused by fluctuations in behavioral or physiological state, variation related to age, sex, season, or time of day, and individual variation within such categories. Here we develop techniques for analyzing phylogenetically correlated data to include within-species variation, or "measurement error" as it is often termed in the statistical literature. We derive methods for (i) univariate analyses, including measurement of "phylogenetic signal," (ii) correlation and principal components analysis for multiple traits, (iii) multiple regression, and (iv) inference of "functional relations," such as reduced major axis (RMA) regression. The methods are capable of incorporating measurement error that differs for each data point (mean value for a species or population), but they can be modified for special cases in which less is known about measurement error (e.g., when one is willing to assume something about the ratio of measurement error in two traits). We show that failure to incorporate measurement error can lead to both biased and imprecise (unduly uncertain) parameter estimates. Even previous methods that are thought to account for measurement error, such as conventional RMA regression, can be improved by explicitly incorporating measurement error and phylogenetic correlation. We illustrate these methods with examples and simulations and provide Matlab programs.},
author = {Ives, Anthony R and Midford, Peter E and Garland, Theodore},
doi = {10.1080/10635150701313830},
isbn = {1063-5157},
issn = {1063-5157},
journal = {Systematic biology},
keywords = {Analysis of Variance,Animals,Computational Biology,Computational Biology: methods,Computer Simulation,Genetic Variation,Lizards,Lizards: genetics,Phenotype,Phylogeny,Regression Analysis,Software},
month = {apr},
number = {2},
pages = {252--270},
pmid = {17464881},
title = {{Within-species variation and measurement error in phylogenetic comparative methods.}},
url = {http://www.ncbi.nlm.nih.gov/pubmed/17464881},
volume = {56},
year = {2007}
}
@article{Pfeifer2014,
abstract = {Habitat fragmentation studies have produced complex results that are challenging to synthesize. Inconsistencies among studies may result from variation in the choice of landscape metrics and response variables, which is often compounded by a lack of key statistical or methodological information. Collating primary datasets on biodiversity responses to fragmentation in a consistent and flexible database permits simple data retrieval for subsequent analyses. We present a relational database that links such field data to taxonomic nomenclature, spatial and temporal plot attributes, and environmental characteristics. Field assessments include measurements of the response(s) (e.g., presence, abundance, ground cover) of one or more species linked to plots in fragments within a partially forested landscape. The database currently holds 9830 unique species recorded in plots of 58 unique landscapes in six of eight realms: mammals 315, birds 1286, herptiles 460, insects 4521, spiders 204, other arthropods 85, gastropods 70, annelids 8, platyhelminthes 4, Onychophora 2, vascular plants 2112, nonvascular plants and lichens 320, and fungi 449. Three landscapes were sampled as long-term time series ({\textgreater}10 years). Seven hundred and eleven species are found in two or more landscapes. Consolidating the substantial amount of primary data available on biodiversity responses to fragmentation in the context of land-use change and natural disturbances is an essential part of understanding the effects of increasing anthropogenic pressures on land. The consistent format of this database facilitates testing of generalizations concerning biologic responses to fragmentation across diverse systems and taxa. It also allows the re-examination of existing datasets with alternative landscape metrics and robust statistical methods, for example, helping to address pseudo-replication problems. The database can thus help researchers in producing broad syntheses of the effects of land use. The database is dynamic and inclusive, and contributions from individual and large-scale data-collection efforts are welcome.},
author = {Pfeifer, Marion and Lefebvre, Veronique and Gardner, Toby a. and Arroyo-Rodriguez, Victor and Baeten, Lander and Banks-Leite, Cristina and Barlow, Jos and Betts, Matthew G. and Brunet, Joerg and Cerezo, Alexis and Cisneros, Laura M. and Collard, Stuart and D'Cruze, Neil and {da Silva Motta}, Catarina and Duguay, Stephanie and Eggermont, Hilde and Eigenbrod, Felix and Hadley, Adam S. and Hanson, Thor R. and Hawes, Joseph E. and {Heartsill Scalley}, Tamara and Klingbeil, Brian T. and Kolb, Annette and Kormann, Urs and Kumar, Sunil and Lachat, Thibault and {Lakeman Fraser}, Poppy and Lantschner, Victoria and Laurance, William F. and Leal, Inara R. and Lens, Luc and Marsh, Charles J. and Medina-Rangel, Guido F. and Melles, Stephanie and Mezger, Dirk and Oldekop, Johan a. and Overal, William L. and Owen, Charlotte and Peres, Carlos a. and Phalan, Ben and Pidgeon, Anna M. and Pilia, Oriana and Possingham, Hugh P. and Possingham, Max L. and Raheem, Dinarzarde C. and Ribeiro, Danilo B. and {Ribeiro Neto}, Jose D. and {Douglas Robinson}, W. and Robinson, Richard and Rytwinski, Trina and Scherber, Christoph and Slade, Eleanor M. and Somarriba, Eduardo and Stouffer, Philip C. and Struebig, Matthew J. and Tylianakis, Jason M. and Tscharntke, Teja and Tyre, Andrew J. and {Urbina Cardona}, Jose N. and Vasconcelos, Heraldo L. and Wearn, Oliver and Wells, Konstans and Willig, Michael R. and Wood, Eric and Young, Richard P. and Bradley, Andrew V. and Ewers, Robert M.},
doi = {10.1002/ece3.1036},
isbn = {20457758},
issn = {20457758},
journal = {Ecology and Evolution},
keywords = {Bioinformatics,Data sharing,Database,Edge effects,Forest fragmentation,Global change,Landscape metrics,Matrix contrast,Species turnover},
month = {may},
number = {9},
pages = {1524--1537},
pmid = {24967073},
title = {{BIOFRAG - a new database for analyzing BIOdiversity responses to forest FRAGmentation}},
url = {http://doi.wiley.com/10.1002/ece3.1036},
volume = {4},
year = {2014}
}
@article{Rossman2014,
abstract = {This study examines resource use (diet, habitat use, and trophic level) within and among demographic groups (males, females, and juveniles) of bottlenose dolphins (Tursiops truncatus). We analyzed the d13Cand d15N values of 15 prey species consti- tuting 84{\%} of the species found in stomach contents. We used these data to estab- lish a trophic enrichment factor (TEF) to inform dietary analysis using a Bayesian isotope mixing model. We document a TEF of 0{\&} and 2.0{\&} for d13Cand d15N, respectively. The dietary results showed that all demographic groups relied heavily on low trophic level seagrass-associated prey. Bayesian standard ellipse areas (SEAb) were calculated to assess diversity in resource use. The SEAb of females was nearly four times larger than that of males indicating varied resource use, likely a conse- quence of small home ranges and habitat specialization. Juveniles possessed an inter- mediate SEAb, generally feeding at a lower trophic level compared to females, potentially an effect of natal philopatry and immature foraging skills. The small SEAb of males reflects a high degree of specialization on seagrass associated prey. Pat- terns in resource use by the demographic groups are likely linked to differences in the relative importance of social and ecological factors.},
author = {Rossman, Sam and {Berens Mccabe}, Elizabeth and Barros, N??lio B. and Gandhi, Hasand and Ostrom, Peggy H. and Stricker, Craig a. and Wells, Randall S.},
doi = {10.1111/mms.12143},
isbn = {0824-0469},
issn = {17487692},
journal = {Marine Mammal Science},
keywords = {Bottlenose dolphin,Diet,Foraging ecology,Generalist,Habitat use,Individual specialization,Sarasota Bay,Stable isotopes,Tursiops truncatus},
month = {may},
number = {1},
pages = {155--168},
title = {{Foraging habits in a generalist predator: Sex and age influence habitat selection and resource use among bottlenose dolphins (Tursiops truncatus)}},
url = {http://doi.wiley.com/10.1111/mms.12143},
volume = {31},
year = {2015}
}
@article{Baraloto2012,
abstract = {1. Niche theory proposes that species differences underlie both coexistence within communities and the differentiation in species composition among communities via limiting similarity and environmental filtering. However, it has been difficult to extend niche theory to species-rich communities because of the empirical challenge of quantifying niches for many species. This has motivated the development of functional and phylogeny-based approaches in community ecology, which represent two different means of approximating niche attributes. 2. Here, we assess the utility of plant functional traits and phylogenetic relationships in predicting community assembly processes using the largest trait and phylogenetic data base to date for any set of species-rich communities. 3. We measured 17 functional traits for all 4672 individuals of 668 tree species co-occurring in nine tropical rain forest plots in French Guiana. Trait variation was summarized into two ordination axes that reflect species niche overlap. 4. We also generated a dated molecular phylogenetic tree based onDNA sequencing of two plastid loci (rbcL and matK) comprising 97{\%}of the individuals and 91{\%}of the species in the plots. 5. We found that, on average, co-occurring species had greater functional and, to a lesser extent, phylogenetic similarity than expected by chance. 6. We also found that functional traits and their ordination loadings showed significant, albeit weak, phylogenetic signal, suggesting that phylogenetic distance provides pertinent information on niche overlap in tropical tree communities. 7. Synthesis. We provide the most comprehensive examination to date of the relative importance of environmental filtering and limiting similarity in structuring tropical tree communities. Our results confirm that environmental filtering is the overriding influence on community assembly in these species-rich systems.},
author = {Baraloto, Christopher and Hardy, Olivier J. and Paine, C. E Timothy and Dexter, Kyle G. and Cruaud, Corinne and Dunning, Luke T. and Gonzalez, Mailyn Adriana and Molino, Jean Fran{\c{c}}ois and Sabatier, Daniel and Savolainen, Vincent and Chave, Jerome},
doi = {10.1111/j.1365-2745.2012.01966.x},
isbn = {0022-0477},
issn = {00220477},
journal = {Journal of Ecology},
keywords = {Competition,Determinants of plant community diversity and stru,Environmental filtering,French Guiana,Functional traits,Limiting similarity,Niche,Phylogenetic signal,Tropical forests},
month = {may},
number = {3},
pages = {690--701},
title = {{Using functional traits and phylogenetic trees to examine the assembly of tropical tree communities}},
url = {http://doi.wiley.com/10.1111/j.1365-2745.2012.01966.x},
volume = {100},
year = {2012}
}
@article{Vamosi2014,
abstract = {Understanding the evolution of specialization in host plant use by pollinators is often complicated by variability in the ecological context of specialization. Flowering communities offer their pollinators varying numbers and proportions of floral resources, and the uniformity observed in these floral resources is, to some degree, due to shared ancestry. Here, we find that pollinators visit related plant species more so than expected by chance throughout 29 plant–pollinator networks of varying sizes, with “clade specialization” increasing with commu- nity size. As predicted, less versatile pollinators showed more clade specializa- tion overall. We then asked whether this clade specialization varied with the ratio of pollinator species to plant species such that pollinators were changing their behavior when there was increased competition (and presumably a forced narrowing of the realized niche) by examining pollinators that were present in at least three of the networks. Surprisingly, we found little evidence that varia- tion in clade specialization is caused by pollinator species changing their behav- ior in different community contexts, suggesting that clade specialization is observed when pollinators are either restricted in their floral choices due to morphological constraints or innate preferences. The resulting pollinator shar- ing between closely related plant species could result in selection for greater pollinator specialization.},
author = {Vamosi, Jana C. and Moray, Clea M. and Garcha, Navdeep K. and Chamberlain, Scott A. and Mooers, Arne {\O}},
doi = {10.1002/ece3.1051},
isbn = {2045-7758},
issn = {20457758},
journal = {Ecology and Evolution},
keywords = {Competition,Linkage rules,Phylogenetic community ecology,Phylogenetic signal,Plant-pollinator networks},
month = {apr},
number = {12},
pages = {2303--2315},
pmid = {25360269},
title = {{Pollinators visit related plant species across 29 plant-pollinator networks}},
url = {http://doi.wiley.com/10.1002/ece3.1051},
volume = {4},
year = {2014}
}
@article{Revell2008,
abstract = {A recent advance in the phylogenetic comparative analysis of continuous traits has been explicit, model-based measurement of "phylogenetic signal" in data sets composed of observations collected from species related by a phylogenetic tree. Phylogenetic signal is a measure of the statistical dependence among species' trait values due to their phylogenetic relationships. Although phylogenetic signal is a measure of pattern (statistical dependence), there has nonetheless been a widespread propensity in the literature to attribute this pattern to aspects of the evolutionary process or rate. This may be due, in part, to the perception that high evolutionary rate necessarily results in low phylogenetic signal; and, conversely, that low evolutionary rate or stabilizing selection results in high phylogenetic signal (due to the resulting high resemblance between related species). In this study, we use individual-based numerical simulations on stochastic phylogenetic trees to clarify the relationship between phylogenetic signal, rate, and evolutionary process. Under the simplest model for quantitative trait evolution, homogeneous rate genetic drift, there is no relation between evolutionary rate and phylogenetic signal. For other circumstances, such as functional constraint, fluctuating selection, niche conservatism, and evolutionary heterogeneity, the relationship between process, rate, and phylogenetic signal is complex. For these reasons, we recommend against interpretations of evolutionary process or rate based on estimates of phylogenetic signal.},
author = {Revell, Liam and Harmon, Luke and Collar, David},
doi = {10.1080/10635150802302427},
isbn = {1063-5157},
issn = {1063-5157},
journal = {Systematic Biology},
keywords = {Computer Simulation,Evolution,Genetic,Genetic Drift,Models,Molecular,Mutation,Phylogeny},
month = {aug},
number = {4},
pages = {591--601},
pmid = {18709597},
title = {{Phylogenetic Signal, Evolutionary Process, and Rate}},
url = {http://sysbio.oxfordjournals.org/cgi/doi/10.1080/10635150802302427},
volume = {57},
year = {2008}
}
@article{Murdoch1969,
abstract = { Switching" in predators which attack several prey species potentially can stabilize the numbers in prey populations. In switching, the number of attacks upon a species is disproportionately large when the species is abundant relative to other prey, and disproportionately small when the species is relatively rare. The null case for two prey species can be written: P1/P2 = cN1/N2, where P1/P2 is the ratio of the two prey expected in the diet, N1/N2 is the ratio given and c is a proportionality constant. Predators were sea—shore snails and prey were mussels and barnacles. Experiments in the laboratory modelled aspects of various natural situations. When the predator had a strong preference (c) between prey the data and the "null case" model were in good agreement. Preference could not altered by subjecting predators to training regimens. When preference was weak the data did not fit the model replicates were variable. Predators could be trained easily to one or other prey species. From a number of experiments it was concluded that in the weak—preference case no switch would occur in nature except where there is an opportunity for predators to become trained to the abundant species. A patchy distribution of the abundant prey could provide this opportunity. Given one prey species, snails caused a decreasing percentage mortality as prey numbers increased. This occurred also with 2 prey species present when preference was strong. When preference was weak the form of the response was unclear. When switching occurred the percentage prey mortality increased with prey density, giving potentially stabilizing mortality. The consequences of these conclusions for prey population regulation and for diversity are discussed.   Read More: http://www.esajournals.org/doi/abs/10.2307/1942352},
author = {Murdoch, W W},
journal = {Ecological Monographs},
number = {4},
pages = {335--354},
title = {{Switching in general predators: expreiments on predators specificity and suitability of prey populations}},
url = {http://www.esajournals.org/doi/abs/10.2307/1942352},
volume = {39},
year = {1969}
}
@article{Bull2010,
abstract = {BACKGROUND: Spatial structure across fragmented landscapes can enhance regional population persistence by promoting local "rescue effects." In small, vulnerable populations, where chance or random events between individuals may have disproportionately large effects on species interactions, such local processes are particularly important. However, existing theory often only describes the dynamics of metapopulations at regional scales, neglecting the role of multispecies population dynamics within habitat patches.$\backslash$n$\backslash$nFINDINGS: By coupling analysis across spatial scales we quantified the interaction between local scale population regulation, regional dispersal and noise processes in the dynamics of experimental host-parasitoid metapopulations. We find that increasing community complexity increases negative correlation between local population dynamics. A potential mechanism underpinning this finding was explored using a simple population dynamic model.$\backslash$n$\backslash$nCONCLUSIONS: Our results suggest a paradox: parasitism, whilst clearly damaging to hosts at the individual level, reduces extinction risk at the population level.},
author = {Bull, James C. and Bonsall, Michael B.},
doi = {10.1371/journal.pone.0011635},
isbn = {1932-6203},
issn = {19326203},
journal = {PLoS ONE},
keywords = {Animals,Beetles,Beetles: growth {\&} development,Beetles: physiology,Fabaceae,Fabaceae: growth {\&} development,Fabaceae: parasitology,Host-Parasite Interactions,Host-Parasite Interactions: physiology,Models,Population Dynamics,Theoretical,Weevils,Weevils: growth {\&} development,Weevils: physiology},
month = {jan},
number = {7},
pages = {e11635},
pmid = {20657767},
title = {{Predators Reduce Extinction Risk in Noisy Metapopulations}},
url = {http://www.pubmedcentral.nih.gov/articlerender.fcgi?artid=2908118{\&}tool=pmcentrez{\&}rendertype=abstract},
volume = {5},
year = {2010}
}
@article{Coll2002,
abstract = {Many terrestrial communities include omnivorous arthropods that feed on both prey and plant resources. In this review we first discuss some unique morphological, physiological, and behavioral traits that enable omnivores to exploit such dissimilar foods, and we explore possible evolutionary pathways to omnivory. We then examine possible benefits and costs of omnivory, describe the relationships between omnivory and other high-order complex trophic interactions, and consider the stability level of communities with closed-loop omnivory. Finally, we explore some of the implications of omnivory for crop damage and for biological, chemical, and cultural control practices. We conclude that the growing realization of the ubiquity of omnivory in nature may require a change in our view of the structure and function of ecological systems.},
author = {Coll, Moshe and Guershon, Moshe},
doi = {10.1146/annurev.ento.47.091201.145209},
isbn = {0066-4170},
issn = {0066-4170},
journal = {Annual Review of Entomology},
keywords = {community organization,tri-trophic interactions,zoophytophagy},
month = {jan},
pages = {267--297},
pmid = {11729076},
title = {{O MNIVORY IN T ERRESTRIAL A RTHROPODS : Mixing Plant and Prey Diets}},
url = {http://www.ncbi.nlm.nih.gov/pubmed/11729076},
volume = {47},
year = {2002}
}
@article{Bozdogan1987,
abstract = {During the last fifteen years, Akaike's entropy-based Information Criterion (AIC) has had a fundamental impact in statistical model evaluation problems. This paper studies the general theory of the AIC procedure and provides its analytical extensions in two ways without violating Akaike's main principles. These extensions make AIC asymptotically consistent and penalize overparameterization more stringently to pick only the simplest of the “true” models. These selection criteria are called CAIC and CAICF. Asymptotic properties of AIC and its extensions are investigated, and empirical performances of these criteria are studied in choosing the correct degree of a polynomial model in two different Monte Carlo experiments under different conditions.},
author = {Bozdogan, Hamparsum},
doi = {10.1007/BF02294361},
isbn = {0033-3123$\backslash$r1860-0980},
issn = {00333123},
journal = {Psychometrika},
keywords = {AIC,Akaike's information criterion,CAIC,CAICF,asymptotic properties,model selection},
number = {3},
pages = {345--370},
pmid = {1222},
title = {{Model selection and Akaike's Information Criterion (AIC): The general theory and its analytical extensions}},
url = {http://link.springer.com/article/10.1007/BF02294361},
volume = {52},
year = {1987}
}
@article{Shimodaira1989,
abstract = {A fixed blood lactate value of 4 mM was commonly used to calculate workload at maximal lactate steady state (MLSS) in kayaking. Our purpose was to measure the actual blood lactate value at MLSS and workload at MLSS in kayaking and assess the validity of using a fixed blood lactate value to calculate the workload at MLSS. 8 junior kayakers (15.1±1.2 years; 179.9±7.3 cm; 72.3±4.9 kg) participated in an incremental workload test and 4-6 sub-maximal constant workload tests (duration of 30 min) on a kayaking ergometer. Blood lactate was measured to calculate the blood lactate value and workload at MLSS. The blood lactate value at MLSS in kayaking was 5.4±0.7 mM. The measured workload at MLSS (112±22 watts) was significantly greater than the calculated workload using a lactate value of 4 mM (104±18 watts, p=0.016). The measured MLSS workload was not significantly different from the calculated workload using a fixed lactate value of 5.4 mM (115±19 watts, p=0.16) or 5.0 mM (113±19 watts, p=0.78) in the incremental tests. A fixed blood lactate value of 5 mM instead of 4 mM might be a better estimate in kayaking given the incremental workload test used in this study.},
archivePrefix = {arXiv},
arxivId = {arXiv:1212.1685v1},
author = {Li, Y. and Margot, N. and Chen, X. and Hartmann, U.},
doi = {10.1055/s-0035-1548759},
eprint = {arXiv:1212.1685v1},
isbn = {0825-8597},
issn = {14393964},
journal = {International Journal of Sports Medicine},
keywords = {2422,anaerobic threshold,ergometer,formation,incremental test,individual sources,iras16293,ism,jets and outflows,lactate,molecules,stars,workload},
number = {4},
pages = {339},
pmid = {16558619},
title = {{Responses from authors}},
url = {http://www.ncbi.nlm.nih.gov/pubmed/24886924{\%}5Cnhttp://www.ncbi.nlm.nih.gov/pubmed/24174569{\%}5Cnhttp://www.thieme-connect.de/DOI/DOI?10.1055/s-0035-1548759},
volume = {36},
year = {2015}
}
@article{Ostfeld2000,
abstract = {Many terrestrial ecosystems are characterized by intermittent production of abundant resources for consumers, such as mast seeding and pulses of primary production following unusually heavy rains. Recent research is revealing patterns in the ways that consumer communities respond to these pulsed resources. Studies of the ramifying effects of pulsed resources on consumer communities integrate 'top-down' and 'bottom-up' approaches to community dynamics, and illustrate how the strength of species interactions can change dramatically through time.},
author = {Ostfeld, Richard S. and Keesing, Felicia},
doi = {10.1016/S0169-5347(00)01862-0},
isbn = {0169-5347},
issn = {01695347},
journal = {Trends in Ecology and Evolution},
number = {6},
pages = {232--237},
pmid = {10802548},
title = {{Pulsed resources and community dynamics of consumers in terrestrial ecosystems}},
url = {http://www.sciencedirect.com/science/article/pii/S0169534700018620},
volume = {15},
year = {2000}
}
@article{Vuong1989,
abstract = {Describes a likelihood ratio test that allows the use of nested or non-nested models. This is particularly useful when comparing Zero Inflated Poisson to regular Poisson models (either negative binomial or not). A V statistic {\textgreater}1.96 is strong evidence for Poisson... A V statistic {\textless}-1.96 is strong evidence for ZIP... Anything between those two},
archivePrefix = {arXiv},
arxivId = {1007.2958v1},
author = {Vuong, Quanq H.},
doi = {10.2307/1912557},
eprint = {1007.2958v1},
isbn = {0012-9682},
issn = {00129682},
journal = {Econometrica},
keywords = {likelihood ratio tests,misspecified,model selection,models,non-nested hypotheses,weighted sums of chi-squares},
number = {2},
pages = {307--333},
title = {{Likelihood Ratio Tests for Model Selection and Non-Nested Hypotheses}},
url = {http://www.jstor.org/stable/1912557},
volume = {57},
year = {1989}
}
@article{Amarasekare2008,
abstract = {Foodwebs are important units of biodiversity, and yet, our knowledge of their spatial dynamics is sketchy at best. Here I attempt to synthesize existing knowledge into a framework that can both identify crucial gaps in the theory as well as facilitate empirical investigations. The synthesis is based on two major axes, foodweb complexity and type of movement, and considers two types of spatial effects, foodweb persistence via a reduction in local extinction and foodweb diversity via an increase in species coexistence. It highlights both invariant properties that are robust to increasing foodweb complexity and emergent properties that result from the interplay between foodweb dynamics and type of movement. It underscores the need for a comparative theoretical framework that can yield testable predictions},
author = {Amarasekare, Priyanga},
doi = {10.1146/annurev.ecolsys.39.110707.173434},
isbn = {1543-592X},
issn = {1543-592X},
journal = {Annual Review of Ecology, Evolution, and Systematics},
keywords = {BIODIVERSITY,Coexistence,DIVERSITY,DYNAMICS,EXTINCTION,FRAMEWORK,GAPS,PERSISTENCE,Spatial effects,Species,movement,prediction},
month = {dec},
number = {1},
pages = {479--500},
pmid = {57},
title = {{Spatial Dynamics of Foodwebs}},
url = {http://dx.doi.org/10.1146/annurev.ecolsys.39.110707.173434{\%}5Cnfile://c/Documents and Settings/Cristina/Meus documentos/My Dropbox/Meu Documentos/Papers/2008/Amarasekare 2008 AnnRevEcolSyst.pdf},
volume = {39},
year = {2008}
}
@article{Ryberg2007,
abstract = {In addition to having a positive effect on species richness (species-area relationships [SARs]), habitat area can influence the presence of predators, which can indirectly influence prey richness. While these direct and indirect effects of area on richness occur simultaneously, no research has examined how predation might contribute to SAR variation. We extend MacArthur and Wilson's equilibrium theory of island biogeography by including predation-induced shifts in prey extinction and predict that predators will reduce slopes of prey SARs. We provide support for this with data from two insular ecosystems: orthopteran richness in Ozark glades (rocky herbaceous communities within a forested matrix) with and without insectivorous lizards and zooplankton richness in freshwater ponds with and without zooplanktivorous fishes. Our results emphasize that anthropogenic activities yield simultaneous changes in processes altering diversity and that it is critical that we understand how these components of anthropogenic change interact to impact diversity.},
author = {Ryberg, Wade A. and Chase, Jonathan M.},
doi = {10.1086/521228},
isbn = {0003-0147},
issn = {0003-0147},
journal = {The American Naturalist},
keywords = {biodiversity,habitat loss,island biogeography,neutral},
month = {oct},
number = {4},
pages = {636--642},
pmid = {17891741},
title = {{Predator‐Dependent Species‐Area Relationships}},
url = {http://www.journals.uchicago.edu/doi/10.1086/521228},
volume = {170},
year = {2007}
}
@article{Holt2002,
abstract = {Ecologists increasingly recognize that a consideration of spatial dynamics is essential for resolving many classical problems in community ecology. In the present paper, I argue that understanding how trophic interactions influence population stability can have important implications for the expression of spatial processes. I use two examples to illustrate this point. The first example has to do with spatial determinants of food chain length. Prior theoretical and empirical work has suggested that colonization-extinction dynamics can influence food chain length, at least for specialist consumers. I briefly review evidence and prior theory that food chain length is sensitive to area. A metacommunity scenario, in which each of various patches can have a food chain varying in length (but in which a consumer is not present on a patch unless its required resource is also present), shows that alternative landscape states are possible. This possibility arises if top predators moderate unstable interactions between intermediate predators and basal resources. The second example has to do with the impact of recurrent immigration on the stability of persistent populations. Immigration can either stabilize or destabilize local population dynamics. Moreover, an increase in immigration can decrease average population size for unstable populations with direct density-dependence, or in predator-prey systems with saturating functional responses. These theoretical models suggest that the interplay of temporal variation and spatial fluxes can lead to novel qualitative phenomena.},
author = {Holt, Robert D.},
doi = {10.1046/j.1440-1703.2002.00485.x},
isbn = {0912-3814},
issn = {09123814},
journal = {Ecological Research},
keywords = {Food chain length,Metacommunity,Spatial flows,Spatial food web ecology},
month = {mar},
number = {2},
pages = {261--273},
pmid = {34},
title = {{Food webs in space: On the interplay of dynamic instability and spatial processes}},
url = {http://link.springer.com/10.1046/j.1440-1703.2002.00485.x},
volume = {17},
year = {2002}
}
@article{Holt1999,
abstract = {The species-area relationship may be the strongest empirical generalization in community ecology. We explore the effect of trophic rank upon the "strength" of the species-area relationship, as measured by z, the slope of a log(species) vs. log(area) plot. We present a simple model for communities closed to immigration, composed of "stacked specialist" food chains (where each plant species supports a specialist herbivore, which in turn sustains a specialist carnivore, etc.), that predicts z should increase with trophic rank; the model brings out some of the spatial implications of sequential dependencies among species. We discuss empirical examples in which the z values of taxa differing in trophic rank were reported and lament the shortage of well-documented examples in the ecological literature. Several examples fit the expected pattern, but others do not. We outline several additional reasons why z values might increase with trophic rank, even for generalists. If the qualitative assumptions of the model are relaxed, the predicted effect of trophic rank on z should weaken or even be reversed. Trophic rank may not have a systematic effect on the species-area relationship if (1) there are strong top-down interactions leading to prey extinctions; (2) communities are open, with recurrent immigration, particularly at higher trophic ranks; (3) consumers are facultative generalists, able to exist on a wide range of resource species; or (4) systems are far from equilibrium. Our aim in this thought piece is to stimulate community ecologists to link theoretical and empirical studies of food web structure with analyses of spatial dynamics and landscape ecology, and to encourage em- pirical studies of the species-area relationship focused on comparisons across taxa varying in trophic rank.},
author = {Holt, Robert D. and Lawton, John H. and Polis, Gary a. and Martinez, Neo D.},
doi = {10.1890/0012-9658(1999)080[1495:TRATSA]2.0.CO;2},
isbn = {0012-9658},
issn = {00129658},
journal = {Ecology},
keywords = {Distribution,Generalist,Island,Patch,Specialist,Specialist vs. generalist food webs,Species-area,Trophic rank and species-area relationship},
number = {5},
pages = {1495--1504},
pmid = {1496},
title = {{Trophic rank and the species-area relationship}},
url = {http://www.esajournals.org/doi/abs/10.1890/0012-9658(1999)080[1495:TRATSA]2.0.CO;2},
volume = {80},
year = {1999}
}
@book{NumericalEcology,
abstract = {Simple beta-lactams and their hydrolysis products, the beta-amino acids, react with TpZn-OH under deprotonation. The latter become semibidentate carboxylate ligands with a NH...O hydrogen bond, and the former become N-bound beta-lactamide ligands. Likewise the antibiotic derivatives 6-aminopenicillanic acid and 7-aminocephalosporanic acid are incorporated as carboxylate ligands. beta-Lactams bearing nitrophenyl or acyl substituents at the nitrogen atoms are opened hydrolytically by TpZn-OH, and the resulting N-substituted beta-amino acids are attached to zinc by their carboxylate functions. Only with trifluoroacetyl as the N-substituent does the hydrolytic cleavage occur at the external amide bond, yielding the free beta-lactam and TpZn-trifluoroacetate. The kinetic investigation of the opening reactions has shown them to be of second order like all other TpZn-OH-induced hydrolytic cleavages, thereby supporting the four-center mechanism for the monozinc beta-lactamases.},
address = {Sydney},
archivePrefix = {arXiv},
arxivId = {0-444-89250-8},
author = {Legendre, Pierre and Legendre, Louis},
booktitle = {Numerical Ecology},
doi = {10.1021/ic050220j},
edition = {3rd Englis},
eprint = {0-444-89250-8},
isbn = {0-444-89250-8},
issn = {00201669},
pmid = {15934776},
publisher = {Elsevier},
title = {{Reactions of pyrazolylborate-zinc-hydroxide complexes related to $\beta$-lactamase activity}},
url = {http://books.google.com/books?hl=en{\&}lr={\&}id=6ZBOA-iDviQC{\&}oi=fnd{\&}pg=PP2{\&}dq=Numerical+ecology{\&}ots=uwaj1-VaWk{\&}sig=NDdt44YeBnqA4m3ZT18JYWNAfG4{\%}5Cnhttp://books.google.com/books?hl=en{\&}lr={\&}id=KBoHuoNRO5MC{\&}oi=fnd{\&}pg=PP1{\&}dq=Numerical+Ecology{\&}o},
year = {2012}
}
@article{Rosenfeld2002,
abstract = {Multiple studies have shown that biodiversity loss can impair ecosystem processes, providing a sound basis for the general application of a precautionary approach to managing biodiversity. However, mechanistic details of species loss effects and the generality of impacts across ecosystem types are poorly understood. The functional niche is a useful conceptual tool for understanding redundancy, where the functional niche is defined as the area occupied by a species in an n-dimensional functional space. Experiments to assess redundancy based on a single functional attribute are biased towards finding redundancy, because species are more likely to have non-overlapping functional niches in a multi-dimensional functional space. The effect of species loss in any particular ecosystem will depend on i) the range of function and diversity of species within a functional group, ii) the relative partitioning of variance in functional space between and within functional groups, and iii) the potential for functional compensation (degree of functional niche overlap) of the species within a functional group. Future research on functional impairment with species loss should focus on identifying which species, functional groups, and ecosystems are most vulnerable to functional impairment from species loss, so that these can be prioritized for management activities directed at maintaining ecosystem function. This will require a better understanding of how the organization of diversity into discrete functional groups differs between different communities and ecosystems.},
author = {Rosenfeld, J},
doi = {10.1034/j.1600-0706.2002.980116.x},
isbn = {0030-1299},
issn = {1600-0706},
journal = {Oikos},
keywords = {AQUATIC INSECTS,BIODIVERSITY,COMMUNITY,DIVERSITY,DYNAMICS,ECOSYSTEM PROCESSES,FOOD-WEB COMPLEXITY,PLANT,PRODUCTIVITY,SERVICES},
number = {1},
pages = {156--162},
pmid = {10},
title = {{Functional redundancy in ecology and conservation}},
volume = {98},
year = {2002}
}
@article{Petchey2008b,
abstract = {Understanding what structures ecological communities is vital to answering questions about extinctions, environmental change, trophic cascades, and ecosystem functioning. Optimal foraging theory was conceived to increase such understanding by providing a framework with which to predict species interactions and resulting community structure. Here, we use an optimal foraging model and allometries of foraging variables to predict the structure of real food webs. The qualitative structure of the resulting model provides a more mechanistic basis for the phenomenological rules of previous models. Quantitative analyses show that the model predicts up to 65{\%} of the links in real food webs. The deterministic nature of the model allows analysis of the model's successes and failures in predicting particular interactions. Predacious and herbivorous feeding interactions are better predicted than pathogenic, parasitoid, and parasitic interactions. Results also indicate that accurate prediction and modeling of some food webs will require incorporating traits other than body size and diet choice models specific to different types of feeding interaction. The model results support the hypothesis that individual behavior, subject to natural selection, determines individual diets and that food web structure is the sum of these individual decisions.},
author = {Petchey, Owen L and Beckerman, Andrew P and Riede, Jens O and Warren, Philip H},
doi = {10.1073/pnas.0710672105},
file = {:Users/alyssacirtwill/Documents/Papers/Petchey et al.{\_}2008{\_}Proceedings of the National Academy of Sciences of the United States of America.pdf:pdf},
isbn = {0027-8424},
issn = {1091-6490},
journal = {Proceedings of the National Academy of Sciences of the United States of America},
keywords = {Animals,Food Chain,Models, Biological},
month = {mar},
number = {11},
pages = {4191--4196},
pmid = {18337512},
title = {{Size, foraging, and food web structure}},
url = {http://www.pubmedcentral.nih.gov/articlerender.fcgi?artid=2393804{\&}tool=pmcentrez{\&}rendertype=abstract},
volume = {105},
year = {2008}
}
@article{Petchey2008a,
abstract = {Understanding which species might become extinct and the consequences of such loss is critical. One consequence is a cascade of further, secondary extinctions. While a significant amount is known about the types of communities and species that suffer secondary extinctions, little is known about the consequences of secondary extinctions for biodiversity. Here we examine the effect of these secondary extinctions on trophic diversity, the range of trophic roles played by the species in a community. Our analyses of natural and model food webs show that secondary extinctions cause loss of trophic diversity greater than that expected from chance, a result that is robust to variation in food web structure, distribution of interactions strengths, functional response, and adaptive foraging. Greater than expected loss of trophic diversity occurs because more trophically unique species are more vulnerable to secondary extinction. This is not a straightforward consequence of these species having few links with others but is a complex function of how direct and indirect interactions affect species persistence. A positive correlation between a species' extinction probability and the importance of its loss defines high-risk species and should make their conservation a priority.},
author = {Petchey, O L and Eklof, Anna and Borrvall, Charlotte and Ebenman, Bo},
doi = {10.1086/587068},
isbn = {0003-0147},
issn = {0003-0147},
journal = {The American Naturalist},
keywords = {EXTINCTION,Species},
month = {may},
number = {5},
pages = {568--579},
pmid = {18419567},
title = {{Trophically unique species are vulnerable to cascading extinction}},
url = {http://www.journals.uchicago.edu/doi/abs/10.1086/587068{\%}5Cnfile://c/Documents and Settings/Cristina/Meus documentos/My Dropbox/Meu Documentos/Papers/2008/Petchey etal 2008 AmericNatur.pdf},
volume = {171},
year = {2008}
}
@article{Cornell1992,
abstract = {Summary 1. Are local ecological communities ever saturated with species? That is, do they ever reach a point where species from the regional pool are unable to invade the local habitat because of exclusion by resident species? 2. We review the theoretical evidence for saturation in various community models and find that non-interactive models predict the absence of saturation as expected, but that interactive models do not uniformly predict saturation. 3. Instead, models where coexistence is based on niche space heterogeneity predict saturation, whereas those where coexistence is based upon spatial heterogeneity yield mixed predictions. 4. Thus, theory says that species interactions are a necessary but not sufficient condition for local saturation in ecological time. 5. We then argue that unsaturated (Type I) assemblages are likely to be ubiquitous in nature and that even saturated (Type II) assemblages may not show hard limits to richness over evolutionary time-scales. 6. If local richness is not often saturated, then regional richness is freed from local constraint, and other limits on regional richness (which, in turn, limit local rich- ness) become important, including phylogenetic diversification over evolutionary time-scales. 7. Our speculations inevitably suggest that the principal direction of control for species richness is from regional to local. If correct, then the key to community structure may lie in extrinsic biogeography rather than in intrinsic local processes, making community ecology a more historical science.},
author = {Cornell, H. V. and Lawton, J. H.},
doi = {10.2307/5503},
isbn = {00218790},
issn = {00218790},
journal = {Journal of Animal Ecology},
keywords = {community classification,ecological community,local richness,regional richness,saturation,species richness},
number = {1},
pages = {1--12},
pmid = {156},
title = {{Species interactions , local and regional processes , and limits to the richness of ecological communities : a theoretical perspective}},
url = {http://www.jstor.org/stable/5503},
volume = {61},
year = {1992}
}
@book{May2001,
abstract = {What makes populations stabilize? What makes them fluctuate? Are populations$\backslash$nin complex ecosystems more stable than populations in simple ecosystems?$\backslash$nIn 1973, Robert May addressed these questions in this classic book.$\backslash$nMay investigated the mathematical roots of population dynamics and$\backslash$nargued-counter to most current biological thinking-that complex ecosystems$\backslash$nin themselves do not lead to population stability. {\_}Stability and$\backslash$nComplexity in Model Ecosystems{\_} played a key role in introducing$\backslash$nnonlinear mathematical models and the study of deterministic chaos$\backslash$ninto ecology, a role chronicled in James Gleick's book {\_}Chaos{\_}.$\backslash$nIn the quarter century since its first publication, the book's message$\backslash$nhas grown in power. Nonlinear models are now at the center of ecological$\backslash$nthinking, and current threats to biodiversity have made questions$\backslash$nabout the role of ecosystem complexity more crucial than ever. In$\backslash$na new introduction, the author addresses some of the changes that$\backslash$nhave swept biology and the biological world since the book's first$\backslash$npublication.},
address = {Princeton, NJ},
author = {May, R M},
booktitle = {Princeton University Press},
doi = {10.1109/TSMC.1978.4309856},
edition = {Princeton },
isbn = {9780691088617},
issn = {0018-9472},
pages = {1--235},
pmid = {4723571},
publisher = {Princeton University Press},
title = {{Stability and Complexity in Model Ecosystems (Princeton Landmarks In Biology)}},
volume = {6},
year = {2001}
}
@article{Saavedra2011,
abstract = {The architecture of mutualistic networks facilitates coexistence of individual participants by minimizing competition relative to facilitation(1,2). However, it is not known whether this benefit is received by each participant node in proportion to its overall contribution to network persistence. This issue is critical to understanding the trade-offs faced by individual nodes in a network(3-5). We address this question by applying a suite of structural and dynamic methods to an ensemble of flowering plant/insect pollinator networks. Here we report two main results. First, nodes contribute heterogeneously to the overall nested architecture of the network. From simulations, we confirm that the removal of a strong contributor tends to decrease overall network persistence more than the removal of a weak contributor. Second, strong contributors to collective persistence do not gain individual survival benefits but are in fact the nodes most vulnerable to extinction. We explore the generality of these results to other cooperative networks by analysing a 15-year time series of the interactions between designer and contractor firms in the New York City garment industry. As with the ecological networks, a firm's survival probability decreases as its individual nestedness contribution increases. Our results, therefore, introduce a new paradox into the study of the persistence of cooperative networks, and potentially address questions about the impact of invasive species in ecological systems and new competitors in economic systems.},
author = {Saavedra, S and Stouffer, D B and Uzzi, B and Bascompte, J},
doi = {10.1038/nature10433},
isbn = {0028-0836},
issn = {0028-0836},
journal = {Nature},
keywords = {ANIMAL MUTUALISTIC NETWORKS,ARCHITECTURE,BIODIVERSITY,Multidisciplinary Sciences,SYSTEMS},
month = {sep},
number = {7368},
pages = {233--235},
pmid = {21918515},
publisher = {Nature Publishing Group},
title = {{Strong contributors to network persistence are the most vulnerable to extinction}},
volume = {478},
year = {2011}
}
@book{MuMIn,
author = {Barto$\backslash$'{\{}n{\}}, Kamil},
booktitle = {R package version 1.15.6},
title = {{MuMIn: Multi-Model Inference}},
url = {https://cran.r-project.org/web/packages/MuMIn/MuMIn.pdf},
year = {2015}
}
@article{lme4,
abstract = {Fit linear and generalized linear mixed-effects models. The models and their components are represented using S4 classes and methods. The core computational algorithms are implemented using the Eigen C++ library for numerical linear algebra and RcppEigen ``glue''.},
address = {R package version 1.1-7},
archivePrefix = {arXiv},
arxivId = {http://arxiv.org/abs/1406.5823},
author = {Bates, D and M$\backslash$"{\{}a{\}}chler, M and Bolker, B and Walker, S},
doi = {http://lme4.r-forge.r-project.org/},
eprint = {/arxiv.org/abs/1406.5823},
isbn = {{\%}(},
issn = {13514180},
journal = {Journal of Statistical Software},
keywords = {GLMM,LMM,Linear mixed-models,R},
number = {1},
pages = {1--48},
pmid = {15991970},
primaryClass = {http:},
title = {{Fitting linear mixed-effects models using lme4}},
url = {http://cran.r-project.org/package=lme4{\%}5Cnhttp://arxiv.org/abs/1406.5823},
volume = {67},
year = {2015}
}
@article{Yachi1999,
abstract = {Although the effect of biodiversity on ecosystem functioning has become a major focus in ecology, its significance in a fluctuating environment is still poorly understood. According to the insurance hypothesis, biodiversity insures ecosystems against declines in their functioning because many species provide greater guarantees that some will maintain functioning even if others fail. Here we examine this hypothesis theoretically. We develop a general stochastic dynamic model to assess the effects of species richness on the expected temporal mean and variance of ecosystem processes such as productivity, based on individual species' productivity responses to environmental fluctuations. Our model shows two major insurance effects of species richness on ecosystem productivity: (i) a buffering effect, i.e., a reduction in the temporal variance of productivity, and (ii) a performance-enhancing effect, i.e., an increase in the temporal mean of productivity. The strength of these insurance effects is determined by three factors: (i) the way ecosystem productivity is determined by individual species responses to environmental fluctuations, (ii) the degree of asynchronicity of these responses, and (iii) the detailed form of these responses. In particular, the greater the variance of the species responses, the lower the species richness at which the temporal mean of the ecosystem process saturates and the ecosystem becomes redundant. These results provide a strong theoretical foundation for the insurance hypothesis, which proves to be a fundamental principle for understanding the long-term effects of biodiversity on ecosystem processes.},
author = {Yachi, S and Loreau, M},
doi = {10.1073/pnas.96.4.1463},
isbn = {0027-8424},
issn = {0027-8424},
journal = {Proceedings of the National Academy of Sciences of the United States of America},
number = {4},
pages = {1463--1468},
pmid = {9990046},
title = {{Biodiversity and ecosystem productivity in a fluctuating environment: the insurance hypothesis.}},
url = {http://www.pnas.org/content/96/4/1463.short},
volume = {96},
year = {1999}
}
@article{Naeem1998,
abstract = {The concept of species redundancy in ecosystem processes is troublesome because it appears to contradict the traditional emphasis in ecology on species singularity. When species richness is high, however, ecosystem processes seem clearly insensitive to considerable variation in biodiversity. Some elementary principles from reliability engineering where engineering redundancy is a valued part of systems design, suggest that we should rethink our stance on species redundancy. For example, a central tenet of reliability engineering is that reliability always increases as redundant components are added to a system, a principle that directly supports redundant species as guarantors of reliable ecosystem functioning. I argue that we should embrace species redundancy and perceive redundancy as a critical feature of ecosystems which must be preserved if ecosystems are to function reliably and provide us with goods and services. My argument is derived from basic principles of reliability engineering which demonstrate that the probability of reliable system performance is closely tied to the level of engineered redundancy in its design. Empirical demonstrations of the value of species redundancy in ecosystem reliability would provide new insights into the ecology of communities and the value of species conservation.},
author = {Naeem, Shahid},
doi = {10.1046/j.1523-1739.1998.96379.x},
isbn = {0888-8892},
issn = {08888892},
journal = {Conservation Biology},
month = {jul},
number = {1},
pages = {39--45},
pmid = {844},
title = {{Species redundancy and ecosystem reliability}},
url = {http://doi.wiley.com/10.1111/j.1523-1739.1998.96379.x},
volume = {12},
year = {1998}
}
@book{car,
abstract = {This is a broad introduction to the R statistical computing environment in the context of applied regression analysis. It is a thoroughly updated edition of John Fox's bestselling text An R and S-Plus Companion to Applied Regression (SAGE, 2002). The Second Edition is intended as a companion to any course on modern applied regression analysis. The authors provide a step-by-step guide to using the high-quality free statistical software R, an emphasis on integrating statistical computing in R with the practice of data analysis, coverage of generalized linear models, enhanced coverage of R graphics and programming, and substantial web-based support materials.},
address = {Thousand Oaks, CA},
archivePrefix = {arXiv},
arxivId = {arXiv:1011.1669v3},
author = {Healy, K.},
booktitle = {Sociological Methods {\&} Research},
doi = {10.1177/0049124105277200},
edition = {Second},
eprint = {arXiv:1011.1669v3},
isbn = {1852338822},
issn = {0049-1241},
number = {1},
pages = {137--140},
pmid = {13623851},
publisher = {Sage},
title = {{Book Review: An R and S-PLUS Companion to Applied Regression}},
url = {http://smr.sagepub.com/cgi/doi/10.1177/0049124105277200},
volume = {34},
year = {2005}
}
@article{Caswell1978,
abstract = {Spatial coexistence depends on a variety of biological and physical processes, and the relative scales of these processes may promote or suppress coexistence. We model plant competition in a spatially varying environment to show how shifting scales of dis- persal, competition, and environmental heterogeneity affect coexis- tence. Spatial coexistence mechanisms are partitioned into three types: the storage effect, nonlinear competitive variance, and growth- density covariance. We first describe how the strength of each of these mechanisms depends on covariances between population den- sities and between population densities and the environment, and we then explain how changes in the scales of dispersal, competition, and environmental heterogeneity should affect these covariances.Our quantitative approach allows us to show how changes in the scales of biological and physical processes can shift the relative importance of different classes of spatial coexistence mechanisms and gives us a more complete understanding of how environmental heterogeneity can enable coexistence. For example, we show how environmental heterogeneity can promote coexistence even when competing species have identical responses to the environment. Keywords:},
author = {Caswell, Hal},
journal = {The American Naturalist},
keywords = {coexistence,competition,dispersal,erogeneity,kernels,on biological,regional-scale ecological dynamics depend,spatial het-,spatial scale},
number = {983},
pages = {127--154},
title = {{Predator-mediated coexistence: a nonequilibrium model}},
url = {http://www.jstor.org/stable/2678832?origin=crossref},
volume = {12},
year = {1978}
}
@article{Gilpin1976,
abstract = {Quantitative models of the species-area-distance relation, based on equilibria between immigration and extinction rates, have been tested against data for birds on 52 Solomon islands. Biologically reasonable models account for 98{\%} of the variance in species number. The data are adequate to permit determination of immigration and extinction curves and the values of seven associated parameters. The resulting curves are very concave. Extinction rates vary almost exactly as the reciprocal of area, but the effect of area on immigration rates is slight. Recognition of major differences among species in immigration and extinction rates and in dispersal distances proves essential to a successful model.},
author = {Gilpin, M E and Diamond, J M},
doi = {10.1073/pnas.73.11.4130},
isbn = {0027-8424},
issn = {0027-8424},
journal = {Proceedings of the National Academy of Sciences of the United States of America},
month = {nov},
number = {11},
pages = {4130--4134},
pmid = {16592364},
title = {{Calculation of immigration and extinction curves from the species-area-distance relation.}},
url = {http://www.pubmedcentral.nih.gov/articlerender.fcgi?artid=431355{\&}tool=pmcentrez{\&}rendertype=abstract},
volume = {73},
year = {1976}
}
@article{Clune2013,
abstract = {A central biological question is how natural organisms are so evolvable (capable of quickly adapting to new environments). A key driver of evolvability is the widespread modularity of biological networks--their organization as functional, sparsely connected subunits--but there is no consensus regarding why modularity itself evolved. Although most hypotheses assume indirect selection for evolvability, here we demonstrate that the ubiquitous, direct selection pressure to reduce the cost of connections between network nodes causes the emergence of modular networks. Computational evolution experiments with selection pressures to maximize network performance and minimize connection costs yield networks that are significantly more modular and more evolvable than control experiments that only select for performance. These results will catalyse research in numerous disciplines, such as neuroscience and genetics, and enhance our ability to harness evolution for engineering purposes.},
archivePrefix = {arXiv},
arxivId = {1207.2743v1},
author = {Clune, Jeff and Mouret, Jean-Baptiste and Lipson, Hod},
doi = {10.1098/rspb.2012.2863},
eprint = {1207.2743v1},
isbn = {1471-2954 (Electronic)$\backslash$n0962-8452 (Linking)},
issn = {1471-2954},
journal = {Proceedings. Biological sciences / The Royal Society},
keywords = {Biological Evolution,Biometry,Computer Simulation,Models, Genetic,Phenotype,Selection, Genetic},
number = {1755},
pages = {20122863},
pmid = {23363632},
title = {{The evolutionary origins of modularity.}},
url = {http://rspb.royalsocietypublishing.org/content/280/1755/20122863},
volume = {280},
year = {2013}
}
@article{Diamond1969,
abstract = {Insular species diversities, and their dependence on island size and isolation, have been postulated to represent a dynamic equilibrium between species immigration rates and species extinction rates. This interpretation has been tested by determining the land and freshwater birds breeding on the nine Channel Islands off southern California in 1968 and comparing the results with a similar survey for the years up to 1917. Most of the islands were found to be in equilibrium as to number of species, but between 17 and 62 per cent of the 1917 breeding species had disappeared by 1968, and an approximately equal number of new immigrant species had become established. Percentage turnover rates vary inversely as insular species diversities, with no effect of distance apparent.},
author = {Diamond, J. M.},
doi = {10.1073/pnas.64.1.57},
isbn = {0027-8424},
issn = {0027-8424},
journal = {Proceedings of the National Academy of Sciences},
month = {sep},
number = {1},
pages = {57--63},
pmid = {16591783},
title = {{Avifaunal equilibria and species turnover rates on the channel islands of California}},
url = {http://www.pnas.org/cgi/doi/10.1073/pnas.64.1.57},
volume = {64},
year = {1969}
}
@article{Case1975,
abstract = {The distribution of lizard species and numbers on islands in the Gulf of Cali- fornia was examined in light of island biogeographic theory. Using linear multiple regression techniques with the percent completeness of the island's lizard fauna as the dependent variable (LS), the relationship and predictive power of several island independent variables (in original and transformed forms) were explored. The results of this analysis differed sharply for dif- ferent subsets of islands. The independent variable accounting for the most variance in LS for oceanic islands was island area; their distance from the mainland explained most of the residual variance. For land-bridge islands, maximum island elevation was the most predictive single variable, and island distance was least important. In both these island groups and considering all the islands together, island area, elevation, and the number of perennial plant species were highly correlated and explained little of each others residual variance. On 12 islands, diversity of shrub volume was also measured and was found to have less predictive power than the other variables considered. Life history attributes of each lizard species were gleaned from the literature and com- pared with the respective colonizing abilities of the species. In general, species occurring on many Gulf islands (e.g., Uta stansburiana) may reach high mainland population densities, have potentially high birth rates and high death rates, and may be habitat generalists. Species absent from most islands and most likely to go extinct on land-bridge islands (e.g., Xantusia 'igilis) have the converse of these attributes. The reasons for the divergence of these results and those expected from predictions is discussed. Lizard fauna relaxation times were calculated for land-bridge islands and were found to be inversely related to island area as predicted from island biogeographic theory. Extinction rates for lizards are about one half as high as those calculated for birds and mammals. Timed searches for lizards were also performed on most of the islands and the adjacent Baja California peninsula. These revealed that the numbers of Uta and total lizard numbers were both inversely related to the number of lizard species occupying an island. The probable reasons for such excess density compensation were examined, and it was concluded that differences in predation intensity, insect productivity, and possibly competitive interference account for the extremely high number of lizards on some small isolated islands.},
author = {Case, Ted J.},
doi = {10.1525/vs.2006.1.1-2.485},
isbn = {0012-9658},
issn = {1559372X},
journal = {Ecology},
keywords = {Baja,colonization,competition,ecology islands,lizards,predation.},
number = {1},
pages = {3--18},
title = {{Species numbers, density compensation, and colonizing ability of lizards on islands in the Gulf of California}},
url = {http://vs.ucpress.edu/cgi/doi/10.1525/vs.2006.1.1-2.485},
volume = {56},
year = {1975}
}
@article{Macarthur2014,
abstract = {See full-text article at JSTOR},
author = {MacArthur, Robert},
doi = {10.2307/1929601},
isbn = {0012-9658},
issn = {00129658},
journal = {Ecology},
number = {3},
pages = {533},
pmid = {566},
title = {{Fluctuations of Animal Populations and a Measure of Community Stability}},
url = {http://www.esajournals.org/doi/abs/10.2307/1929601},
volume = {36},
year = {1955}
}
@article{May1974,
abstract = {Some of the simplest nonlinear difference equations describing the growth of biological populations with nonoverlapping generations can exhibit a remarkable spectrum of dynamical behavior, from stable equilibrium points, to stable cyclic oscillations between 2 population points, to stable cycles with 4, 8, 16, . . . points, through to a chaotic regime in which (depending on the initial population value) cycles of any period, or even totally aperiodic but boundedpopulation fluctuations, can occur. This rich dynamical structure is overlooked in conventional linearized analyses; its existence in such fully deterministic nonlinear difference equations is a fact of considerable mathematical and ecological interest.},
archivePrefix = {arXiv},
arxivId = {arXiv:1011.1669v3},
author = {May, Robert M},
doi = {10.1126/science.186.4164.645},
eprint = {arXiv:1011.1669v3},
isbn = {2014075522785},
issn = {0036-8075},
journal = {Science},
keywords = {Ecology,Mathematics,Models, Biological,Population Density,Population Growth,Species Specificity},
month = {nov},
number = {4164},
pages = {645--647},
pmid = {4412202},
title = {{Biological Populations with Nonoverlapping Generations: Stable Points, Stable Cycles, and Chaos}},
url = {http://science.sciencemag.org/content/186/4164/645.abstract},
volume = {186},
year = {1974}
}
@article{Ives2007,
abstract = {Understanding the relationship between diversity and stability requires a knowledge of how species interact with each other and how each is affected by the environment. The relationship is also complex, because the concept of stability is multifaceted; different types of stability describing different properties of ecosystems lead to multiple diversity-stability relationships. A growing number of empirical studies demonstrate positive diversity-stability relationships. These studies, however, have emphasized only a few types of stability, and they rarely uncover the mechanisms responsible for stability. Because anthropogenic changes often affect stability and diversity simultaneously, diversity-stability relationships cannot be understood outside the context of the environmental drivers affecting both. This shifts attention away from diversity-stability relationships toward the multiple factors, including diversity, that dictate the stability of ecosystems.},
author = {Ives, Anthony R and Carpenter, Stephen R},
doi = {10.1126/science.1133258},
isbn = {1095-9203 (Electronic)$\backslash$n0036-8075 (Linking)},
issn = {0036-8075},
journal = {Science (New York, N.Y.)},
keywords = {Animals,Biodiversity,Ecosystem,Environment,Extinction, Biological,Models, Biological,Population Dynamics},
month = {jul},
number = {5834},
pages = {58--62},
pmid = {17615333},
title = {{Stability and diversity of ecosystems.}},
url = {http://www.ncbi.nlm.nih.gov/pubmed/17615333},
volume = {317},
year = {2007}
}
@article{Levine2000,
abstract = {In a California riparian system, the most diverse natural assemblages are the most invaded by exotic plants. A direct in situ manipulation of local diversity and a seed addition experiment showed that these patterns emerge despite the intrinsic negative effects of diversity on invasions. The results suggest that species loss at small scales may reduce invasion resistance. At community-wide scales, the overwhelming effects of ecological factors spatially covarying with diversity, such as propagule supply, make the most diverse communities most likely to be invaded.},
author = {Levine, J M},
doi = {10.1126/science.288.5467.852},
isbn = {0036-8075},
issn = {00368075},
journal = {Science},
month = {may},
number = {5467},
pages = {852--854},
pmid = {10797006},
title = {{Species diversity and biological invasions: Relating local process to community pattern}},
volume = {288},
year = {2000}
}
@article{Torchin2003,
abstract = {Damage caused by introduced species results from the high population densities and large body sizes that they attain in their new location. Escape from the effects of natural enemies is a frequent explanation given for the success of introduced species. Because some parasites can reduce host density and decrease body size, an invader that leaves parasites behind and encounters few new parasites can experience a demographic release and become a pest. To test whether introduced species are less parasitized, we have compared the parasites of exotic species in their native and introduced ranges, using 26 host species of molluscs, crustaceans, fishes, birds, mammals, amphibians and reptiles. Here we report that the number of parasite species found in native populations is twice that found in exotic populations. In addition, introduced populations are less heavily parasitized (in terms of percentage infected) than are native populations. Reduced parasitization of introduced species has several causes, including reduced probability of the introduction of parasites with exotic species (or early extinction after host establishment), absence of other required hosts in the new location, and the host-specific limitations of native parasites adapting to new hosts.},
author = {Torchin, Mark E and Lafferty, Kevin D and Dobson, Andrew P and McKenzie, Valerie J and Kuris, Armand M},
doi = {10.1038/nature01346},
isbn = {0028-0836},
issn = {0028-0836},
journal = {Nature},
number = {6923},
pages = {628--630},
pmid = {12571595},
title = {{Introduced species and their missing parasites.}},
url = {http://www.nature.com/nature/journal/v421/n6923/abs/nature01346.html},
volume = {421},
year = {2003}
}
@article{Stachowicz1999,
abstract = {Each copy of any part of a JSTOR transmission must contain the same copyright notice that appears on the screen or printed page of such transmission. JSTOR is a not-for-profit organization founded in 1995 to build trusted digital archives for scholarship. We work with the scholarly community to preserve their work and the materials they rely upon, and to build a common research platform that promotes the discovery and use of these resources. For more information about JSTOR, please contact support@jstor.org. REPORTS Syst. App!. Microbiol. 11, 128 (1989); J. A. Eisen,J. Mo!. Evol. 41,1105 (1995). 13. Searches were done as in (7). 14. A. C. Ferreira et a!., Int. J. Syst. Bacteriol. 47, 939 (1997). 15. The distribution of the 64 trinucleotides was computed for the megaplasmid and the two chromosomes. The compositions of the megaplasmid and chromosome 11 were compared with the x2 test, with each other, and with similarly sized samples of chromosome I chosen uniformly at random. These tests indicate that all three elements have significantly different trinucleotide com-position (P {\textless} 0.01). Although the size of the plasmid is too small to obtain statistically meaningful compari-sons of trinucleotide composition, its GC content is significantly different from the remaining genome (56{\%} compared with 67{\%}) (Tables 1 to 5). 16. J. G. Lawrence and H. Ochman,J. Mo!. Evo!. 44, 383 (1997). 17. S. Tirgari and B. E. B. Moseley,J. Gen. Microbio!. 119, 287 (1980). 18. A. Tanaka, H. Hirano, M. Kikuchi, S. Kitayama, H. Watanabe, Radiat. Environ. Biophys. 35, 95 (1996).},
author = {Stachowicz, John J and Whitlatch, Robert B and Osman, Richard W},
doi = {10.1126/science.286.5444.1577},
isbn = {0036-8075},
issn = {00368075},
journal = {Source: Science, New Series},
month = {nov},
number = {19},
pages = {1577--1579},
pmid = {10567267},
title = {{Species Diversity and Invasion Resistance in a Marine Ecosystem}},
url = {http://www.jstor.org/stable/2899829{\%}5Cnhttp://www.jstor.org/{\%}5Cnhttp://www.jstor.org/action/showPublisher?publisherCode=aaas.},
volume = {286},
year = {1999}
}
@article{Knops1999,
abstract = {Declining biodiversity represents one of the most dramatic and irreversible aspects of anthropogenic global change, yet the ecological implications of this change are poorly understood. Recent studies have shown that biodiversity loss of basal species, such as autotrophs or plants, affects fundamental ecosystem processes such as nutrient dynamics and autotrophic production. Ecological theory predicts that changes induced by the loss of biodiversity at the base of an ecosystem should impact the entire system. Here we show that experimental reductions in grassland plant richness increase ecosystem vulnerability to invasions by plant species, enhance the spread of plant fungal diseases, acid alter the richness and structure of insect communities. These results suggest that the loss of basal species may have profound effects on the integrity and functioning of ecosystems.},
author = {Knops, Johannes M H and Tilman, David and Haddad, Nick M. and Naeem, Shahid and Mitchell, Charles E. and Haarstad, John and Ritchie, Mark E. and Howe, Katherine M. and Reich, Peter B. and Siemann, Evan and Groth, James},
doi = {10.1046/j.1461-0248.1999.00083.x},
isbn = {1461-023X},
issn = {1461023X},
journal = {Ecology Letters},
keywords = {Biodiversity,Biological invasions,Ecosystem functioning,Insect abundance,Insect diversity,Plant diseases,Plant pathogens},
number = {5},
pages = {286--293},
pmid = {21884563},
title = {{Effects of plant species richness on invasion dynamics, disease outbreaks, insect abundances and diversity}},
url = {http://onlinelibrary.wiley.com/doi/10.1046/j.1461-0248.1999.00083.x/full},
volume = {2},
year = {1999}
}
@article{Simberloff1982,
abstract = {Cole's theoretical conclusion that one large site generally contains more species than several small ones of equal area is falsified by data in the literature, as is his contention that exceptions will only occur when the species in the sites are but a small fraction of those in the species pool. For a variety of taxa, for a number of different habitat types, and for a wide range of sizes of the biota as a fraction of the pool, either there is no clear best strategy, or several small sites are better than one large site. Since there are numerous idiosyncratic biological considerations, plus a number of nonbiological ones that bear heavily on refuge design, it is unlikely that a general reductionist model can generate useful predictions or advice on this matter},
author = {Simberloff, Daniel and Abele, Lawrence G.},
doi = {10.1086/283968},
isbn = {00030147},
issn = {0003-0147},
journal = {The American Naturalist},
number = {1},
pages = {41},
pmid = {349},
title = {{Refuge Design and Island Biogeographic Theory: Effects of Fragmentation}},
url = {http://www.jstor.org/stable/2461084},
volume = {120},
year = {1982}
}
@article{Eadie1986,
abstract = {Fish species richness in 82 lakes in Ontario, Canada was significantly correlated with surface area. In this region, latitude explained only a small amount of the variation in fish species richness. Thus, the study provides a clear demonstration of the relation between fish species richness and lake area without the confounding effects of latitude and physiography inherent in analyses from broader geographic regions. By comparison with the species-area relationship obtained, the authors show that acidification clearly depressed the number of fish species in 66 acid-stressed lakes in Ontario. Fish species richness was also significantly correlated with both drainage and surface areas of 21 Ontario rivers. Slopes of species-area regressions of lakes and rivers did not differ significantly, suggesting that species are added to these habitats at similar rates.},
author = {Eadie, John McA and Hurly, T. Andre and Montgomerie, Robert D. and Teather, Kevin L.},
doi = {10.1007/BF00005423},
isbn = {0378-1909},
issn = {03781909},
journal = {Environmental Biology of Fishes},
keywords = {Acid rain,Fish habitat,Island biogeography,Species-area curves},
number = {2},
pages = {81--89},
title = {{Lakes and rivers as islands: species-area relationships in the fish faunas of Ontario}},
url = {http://link.springer.com/article/10.1007/BF00005423},
volume = {15},
year = {1986}
}
@article{Courchamp2003,
abstract = {The invasion of ecosystems by exotic species is currently viewed as one of the most important sources of biodiversity loss. The largest part of this loss occurs on islands, where indigenous species have often evolved in the absence of strong competition, herbivory, parasitism or predation. As a result, introduced species thrive in those optimal insular ecosystems affecting their plant food, competitors or animal prey. As islands are characterised by a high rate of endemism, the impacted populations often correspond to local subspecies or even unique species. One of the most important taxa concerning biological invasions on islands is mammals. A small number of mammal species responsible for most the damage to invaded insular ecosystems: rats, cats, goats, rabbits, pigs and a few others. The effect of alien invasive species may be simple or very complex, especially since a large array of invasive species, mammals and others, can be present simultaneously and interact among themselves as well as with the indigenous species. In most cases, introduced species generally have a strong impact and they often are responsible for the impoverishment of the local flora and fauna. The best response to these effects is almost always to control the alien population, either by regularly reducing their numbers, or better still, by eradicating the population as a whole from the island. Several types of methods are currently used: physical (trapping, shooting), chemical (poisoning) and biological (e.g. directed use of diseases). Each has its own set of advantages and disadvantages, depending on the mammal species targeted. The best strategy is almost always to combine several methods. Whatever the strategy used, its long-term success is critically dependent on solid support from several different areas, including financial support, staff commitment, and public support, to name only a few. In many cases, the elimination of the alien invasive species is followed by a rapid and often spectacular recovery of the impacted local populations. However, in other cases, the removal of the alien is not sufficient for the damaged ecosystem to revert to its former state, and complementary actions, such as species re-introduction, are required. A third situation may be widespread: the sudden removal of the alien species may generate a further disequilibrium, resulting in further or greater damage to the ecosystern. Given the numerous and complex population interactions among island species, it is difficult to predict the outcome of the removal of key species, such as a top predator. This justifies careful pre-control study and preparation prior to initiating the eradication of an alien species, in order to avoid an ecological catastrophe. In addition, long-term monitoring of the post-eradication ecosystem is crucial to assess success and prevent reinvasion.},
author = {Courchamp, Franck and Chapuis, Jean-Louis and Pascal, Michel},
doi = {10.1017/S1464793102006061},
isbn = {1464-7931},
issn = {14647931},
journal = {Biological Reviews},
keywords = {Animals,BIOLOGICAL-CONTROL,Biodiversity,CATS FELIS-CATUS,Ecosystem,FERAL HOUSE CATS,Food Chain,Geography,KERGUELEN ARCHIPELAGO,LADY-ALICE-ISLAND,Mammals,Mammals: physiology,NEW-ZEALAND FOREST,Population Control,Population Density,Population Dynamics,Predatory Behavior,QUEEN-CHARLOTTE-ISLANDS,RABBIT ORYCTOLAGUS-CUNICULUS,TUATARA SPHENODON-PUNCTATUS,VIRUS-VECTORED IMMUNOCONTRACEPTION,Wild,alien species,biological invasions,introduced species,island restoration},
month = {aug},
number = {3},
pages = {347--383},
pmid = {14558589},
title = {{Mammal invaders on islands: impact, control and control impact}},
url = {http://www.ncbi.nlm.nih.gov/pubmed/14558589{\%}5Cnhttp://apps.isiknowledge.com/full{\_}record.do?product=UA{\&}colname=WOS{\&}search{\_}mode=CitingArticles{\&}qid=3{\&}SID=N2cfE3Dn1oF18GG75fn{\&}page=4{\&}doc=31},
volume = {78},
year = {2003}
}
@article{Savidge1987,
abstract = {The island of Guam has experienced a precipitous decline of its native forest birds, and several lines of evidence implicate the introduced brown tree snake (Boiga irregularis) as the cause of the range reductions and extinctions. The range expansion of B. irregularis correlated well with the range contraction ofthe forest avifauna. Exceptionally high predation by snakes on bird-baited funnel traps occurred in areas where bird popu? lations had declined. Little or no snake predation occurred in areas with stable bird pop? ulations. Several factors have contributed to Boiga's success in decimating the avifauna on Guam. Although most common in forests, B. irregularis has occupied a variety of habitats on Guam, and few effective barriers to its dispersal exist. Prey refuges are present only in urban areas, on concrete or metal structures, and in the savanna. Boiga's nocturnal and arboreal habits and an apparent keen ability to locate prey, make roosting and nesting birds, eggs, and nestlings vulnerable. Besides birds, B. irregularis feeds on small mammals and lizards. By including the abundant small reptiles as a major component in its diet, Boiga has maintained high densities in forest and second-growth habitats while extermi- nating its more vulnerable prey. This is the first time a snake has been implicated as an agent of extinction.},
author = {Savidge, J. a.},
doi = {10.2307/1938471},
isbn = {0012-9658},
issn = {00129658},
journal = {Ecology},
number = {3},
pages = {660--668},
title = {{Extinction of an island forest avifauna by an introduced snake.}},
url = {http://www.esajournals.org/doi/abs/10.2307/1938471},
volume = {68},
year = {1987}
}
@article{Schoener1996,
abstract = {HISTORICAL ecology contains various examples of how predators introduced onto islands by man have apparently exterminated native prey species(1-6). Conversely, a pioneering experiment(7) showed an increase in number of species with predator presence. Subsequent experiments have shown both increases and decreases in prey diversity(8-10). Here we investigate how predator introduction affects one aspect of prey diversity (number of species or species richness), and prey abundance, We ran a seven-year experiment on an entirely natural system of small islands, using the commonest local lizard as the predator and web spiders as prey. Lizard introduction caused rapid and devastating effects on spider diversity and abundance: within two years, islands onto which lizards had been introduced became almost identical to islands with natural lizard populations. The proportion of species becoming extinct was 12.6 times higher on 'lizard-introduction' islands than on islands without lizards. Locally common and rare species were both reduced by the introduction of lizards, but nearly all of the latter became permanently extinct.},
author = {Schoener, Thomas W. and Spiller, David a.},
doi = {10.1038/381691a0},
isbn = {0028-0836},
issn = {0028-0836},
journal = {Nature},
number = {6584},
pages = {691--694},
pmid = {207},
title = {{Devastation of prey diversity by experimentally introduced predators in the field}},
url = {papers2://publication/uuid/48608078-D3B7-4570-A104-36114746EC4F},
volume = {381},
year = {1996}
}
@article{Hanna2014,
abstract = {ABSTRACT Aim Understanding extinction on islands is critical for biodiversity conservation. Introduced predators are a major cause of island extinctions, but there have been few large-scale studies of the complexity of the effects of predators on island faunas, or how predation interacts with other factors. Using a large database of island mammal populations, we describe and explain patterns of island mammal extinc- tions as a function of introduced predators, life history and geography. Location Three hundred and twenty-three Australian islands. Methods We built a database of 934 island mammal populations, extinct and extant, including life history and ecology, island geography and the presence of introduced predators. To test predictors of extinction probability, we used gener- alized linear mixed models to control partially for phylogenetic non-independence, and decision trees to more fully explore interactive effects. Results The decision trees identified large mammals ({\textgreater} 2.7 kg) as having higher extinction probabilities than small species ({\textless} 2.7 kg). In large species, extinction patterns are consistent with island biogeography theory, with distance from the mainland being the primary predictor of extinction. For small species, the presence of introduced black rats is the primary predictor of extinction. As predicted by mesopredator suppression theory, extinction probabilities are lower on islandswith both black rats and a larger introduced predator (cats, foxes or dingoes), compared with islands with rats but no larger predator. Similarly, extinction probabilities are lower on islands with both a mid-sized (cats or foxes) and a larger (dingoes) predator, compared with islands with cats or foxes only. Main conclusions Island mammal extinctions result from complex interactions of introduced predators, island geography and prey biology. One conservation implication of our results is that eradication of introduced apex predators (cats, foxes or dingoes) from islands could precipitate the expansion of black rat popu- lations, potentially leading to extinction of native mammal species whose remain- ing populations are confined to islands. Keywords},
author = {Hanna, Emily and Cardillo, Marcel},
doi = {10.1111/geb.12103},
isbn = {1466-8238},
issn = {1466822X},
journal = {Global Ecology and Biogeography},
keywords = {Apex predator,Extinction,Introduced species,Island biogeography,Islands,Mesopredator release},
month = {sep},
number = {4},
pages = {395--404},
title = {{Island mammal extinctions are determined by interactive effects of life history, island biogeography and mesopredator suppression}},
url = {http://doi.wiley.com/10.1111/geb.12103},
volume = {23},
year = {2014}
}
@article{Remes2012,
abstract = {Predation is a major factor in ecology, evolution and conservation and thus its understanding is essential for insights into ecological processes and management of endangered populations of prey. Here we conducted a spatially (main island through to offshore islets) and temporally (1938-2005) extensive meta-analysis of published nest predation rates in New Zealand songbirds. We obtained information on nest predation rates from 79 populations (. n=. 4838 nests) of 26 species of songbirds belonging to 17 families. Nest predation rates increased from southwest to northeast and also across the last 60. years (by 15-25{\%} points in both cases). We identified a major impact of exotic mammalian predators. Nest predation was lowest in areas where no exotic predators were present (12.8{\%}), higher in areas with ongoing predator control (33.9{\%}), and highest in areas without control that had the full set of exotic and native nest predators (47.5{\%}). Surprisingly, nest predation rates were higher in introduced as compared to native species. Our analyses demonstrated that human-caused factors (introduced predators and prey) overrode factors such as nest type and habitat identified as important in predicting nest predation in North America and Europe previously. ?? 2012 Elsevier Ltd.},
author = {Reme??, Vladim??r and Matysiokov??, Beata and Cockburn, Andrew},
doi = {10.1016/j.biocon.2012.01.063},
isbn = {0006-3207},
issn = {00063207},
journal = {Biological Conservation},
keywords = {Exotic predators,Extinction,Introductions,Nest predation,New Zealand},
month = {apr},
number = {1},
pages = {54--60},
title = {{Nest predation in New Zealand songbirds: Exotic predators, introduced prey and long-term changes in predation risk}},
url = {http://linkinghub.elsevier.com/retrieve/pii/S0006320712000766},
volume = {148},
year = {2012}
}
@article{Spiller2007a,
abstract = {Major abiotic disturbance can be an important factor influencing food-web dynamics, particularly in areas impacted by the recent increase in hurricane activity. We present a unique set of data on key food-web processes occurring on 10 small islands for three relatively calm years and then four subsequent years during which two hurricanes passed directly over the study site. Herbivory, as measured by leaf damage, was 3.2 times higher in the year after the first hurricane (2000) than in the previous year and was 1.7 times higher in the year after the second hurricane (2002) than in 2001. The effect of a top predator (the lizard, Anolis sagrei) on herbivory strengthened continuously after the first hurricane and overall was 2.4 times stronger during the disturbance period than before. Overall abundance of lizards was 30{\%} lower during the disturbance period than before, and abundances of web spiders and hymenopteran parasitoids were 66{\%} and 59{\%} lower, respectively. We suggest that increased herbivory observed on all islands was caused, at least in part, by the overall reduction in predation by both lizards and arthropods, whereas magnification of the lizard effect on herbivory was caused by reduced compensatory predation by arthropods.},
author = {Spiller, David a. and Schoener, Thomas W.},
doi = {10.1890/0012-9658(2007)88[37:AOIFDF]2.0.CO;2},
isbn = {0012-9658},
issn = {00129658},
journal = {Ecology},
keywords = {Abiotic disturbance,Anolis sagrei,Conocarpus erectus,Food webs,Herbivory,Hurricanes,Leaf damage,Parasitoids,Spiders,Trophic cascade},
month = {jan},
number = {1},
pages = {37--41},
pmid = {17489451},
title = {{Alteration of island food-web dynamics following major disturbance by hurricanes}},
url = {http://www.ncbi.nlm.nih.gov/pubmed/17489451},
volume = {88},
year = {2007}
}
@incollection{Holt2010,
abstract = {Robert H. MacArthur and Edward O. Wilson's The Theory of Island Biogeography, first published by Princeton in 1967, is one of the most influential books on ecology and evolution to appear in the past half century. By developing a general mathematical theory to explain a crucial ecological problem--the regulation of species diversity in island populations--the book transformed the science of biogeography and ecology as a whole. In The Theory of Island Biogeography Revisited, some of today's most prominent biologists assess the continuing impact of MacArthur and Wilson's book four decades after its publication. Following an opening chapter in which Wilson reflects on island biogeography in the 1960s, fifteen chapters evaluate and demonstrate how the field has extended and confirmed--as well as challenged and modified--MacArthur and Wilson's original ideas. Providing a broad picture of the fundamental ways in which the science of island biogeography has been shaped by MacArthur and Wilson's landmark work, The Theory of Island Biogeography Revisited also points the way toward exciting future research.},
address = {Princeton},
author = {Holt, R.D.},
booktitle = {The theory of island biogeography revisited},
editor = {Losos, Jonathan B. and Ricklefs, Robert E.},
isbn = {978-1-4008-3192-0},
pages = {143--185},
publisher = {Princeton University Press},
title = {{Toward a trophic island biogeography.}},
url = {https://books.google.com/books?hl=da{\&}lr={\&}id=slwedU4I4JMC{\&}pgis=1},
year = {2010}
}
@incollection{Schoener2010,
address = {Princeton},
author = {Schoener, Thomas W},
booktitle = {The theory of island biogeography revisited},
editor = {Losos, Jonathan B. and Ricklefs, Robert E.},
pages = {52--87},
publisher = {Princeton University Press},
title = {{The MacArthur-Wilson equilibrium model}},
year = {2010}
}
@book{MacArthur1967,
abstract = {The Theory of Island Biogeography establishes the conditions for the attainment and maintenance of equilibrium species numbers on islands and frag- mented habitats. It employs mathematical models to estimate rates of colonization and turnover, as well as differences in species diversity among islands. (The SCI indicates that this book has been cited in over 1,830 publications.)},
address = {Princeton, NJ},
archivePrefix = {arXiv},
arxivId = {arXiv:1011.1669v3},
author = {MacArthur, Robert H and Wilson, Edward O.},
doi = {10.1126/science.159.3810.71},
editor = {Losos, Jonathan B. and Ricklefs, Robert E.},
eprint = {arXiv:1011.1669v3},
isbn = {9780691088365},
issn = {0036-8075},
keywords = {biogeography,island models,species richness},
pages = {215},
pmid = {250558},
publisher = {Princeton University Press},
title = {{The Theory of Island Biogeography}},
url = {http://www.sciencemag.org/cgi/doi/10.1126/science.159.3810.71},
year = {1967}
}
@article{Mangel2013,
abstract = {We adapt methods from the stochastic theory of invasions ? for which a key question is whether a propagule will grow to an established population or fail ? to show how monitoring early participation in a social collaboration network allows prediction of success. Social collaboration networks have become ubiquitous and can now be found in widely diverse situations. However, there are currently no methods to predict whether a social collaboration network will succeed or not, where success is defined as growing to a specified number of active participants before falling to zero active participants. We illustrate a suitable methodology with Wikipedia. In general, wikis are web-based software that allows collaborative efforts in which all viewers of a page can edit its contents online, thus encouraging cooperative efforts on text and hypertext. The English language Wikipedia is one of the most spectacular successes, but not all wikis succeed and there have been some major failures. Using these new methods, we derive detailed predictions for the English language Wikipedia and in summary for more than 250 other language Wikipedias. We thus show how ideas from population biology can inform aspects of technology in new and insightful ways.$\backslash$nWe adapt methods from the stochastic theory of invasions ? for which a key question is whether a propagule will grow to an established population or fail ? to show how monitoring early participation in a social collaboration network allows prediction of success. Social collaboration networks have become ubiquitous and can now be found in widely diverse situations. However, there are currently no methods to predict whether a social collaboration network will succeed or not, where success is defined as growing to a specified number of active participants before falling to zero active participants. We illustrate a suitable methodology with Wikipedia. In general, wikis are web-based software that allows collaborative efforts in which all viewers of a page can edit its contents online, thus encouraging cooperative efforts on text and hypertext. The English language Wikipedia is one of the most spectacular successes, but not all wikis succeed and there have been some major failures. Using these new methods, we derive detailed predictions for the English language Wikipedia and in summary for more than 250 other language Wikipedias. We thus show how ideas from population biology can inform aspects of technology in new and insightful ways.},
author = {Mangel, M. and Satterthwaite, W.H. and Pirolli, P. and Suh, B. and Zhang, Y.},
doi = {10.1080/15659801.2013.815435},
issn = {1565-9801},
journal = {Israel Journal of Ecology {\&} Evolution},
keywords = {invasion biology,stochastic population theory,soci},
month = {mar},
number = {1},
pages = {17--26},
publisher = {Taylor {\&} Francis},
title = {{Invasion biology and the success of social collaboration networks, with application to Wikipedia}},
url = {http://www.tandfonline.com/doi/abs/10.1080/15659801.2013.815435},
volume = {59},
year = {2013}
}
@article{Poulin2013,
abstract = {1. Parasites affect interactions among species in food webs and should be considered in any analysis of the structure, dynamics or resilience of trophic networks. 2. However, the roles of individual parasite species, such as their importance as connectors within the network, and what factors determine these roles, are yet to be investigated. Here, we test the hypotheses that the species roles of trematode, cestode and nematode parasites in aquatic food webs are influenced by the type of definitive host they use, and also determined by their phylogenetic affiliations. 3. We quantified the network role of 189 helminth species from six highly resolved intertidal food webs. We focused on four measures of centrality (node degree, closeness centrality, betweenness centrality and eigenvalue centrality), which characterize each parasite's position within the web, and on relative connectedness of a parasite species to taxa in its own module vs. other modules of the web (within-module degree and participation coefficient). 4. All six food webs displayed a significant modular structure, that is, they consisted of subsets of species interacting mostly with each other and less with species from other subsets. We demonstrated that the parasites themselves are not generating this modularity, though they contribute to intermodule connectivity. 5. Mixed-effects models revealed only a modest influence of the type of definitive host used (bird or fish) and of the web of origin on the different measures of parasite species roles. In contrast, the taxonomic affiliations of the parasites, included in the models as nested random factors, accounted for 37-93{\%} of the total variance, depending on the measure of species role. 6. Our findings indicate that parasites are important intermodule connectors and thus contribute to web cohesion. We also uncover a very strong phylogenetic signal in parasite species roles, suggesting that the role of any parasite species in a food web, including new invasive species, is to some extent predictable based solely on its taxonomic affiliations.},
author = {Poulin, Robert and Krasnov, Boris R. and Pilosof, Shai and Thieltges, David W.},
doi = {10.1111/1365-2656.12101},
isbn = {0021-8790},
issn = {00218790},
journal = {Journal of Animal Ecology},
keywords = {Connectedness,Modularity,Network centrality,Phylogenetic signal,Trematodes},
month = {jun},
number = {6},
pages = {1265--1275},
pmid = {23800281},
title = {{Phylogeny determines the role of helminth parasites in intertidal food webs}},
url = {http://www.ncbi.nlm.nih.gov/pubmed/23800281},
volume = {82},
year = {2013}
}
@article{Turk-Browne2013,
abstract = {Noninvasive studies of human brain function hold great potential to unlock mysteries of the human mind. The complexity of data generated by such studies, however, has prompted various simplifying assumptions during analysis. Although this has enabled considerable progress, our current understanding is partly contingent upon these assumptions. An emerging approach embraces the complexity, accounting for the fact that neural representations are widely distributed, neural processes involve interactions between regions, interactions vary by cognitive state, and the space of interactions is massive. Because what you see depends on how you look, such unbiased approaches provide the greatest flexibility for discovery.},
archivePrefix = {arXiv},
arxivId = {NIHMS150003},
author = {Turk-Browne, Nicholas B},
doi = {10.1126/science.1238409},
eprint = {NIHMS150003},
isbn = {2122633255},
issn = {1095-9203},
journal = {Science (New York, N.Y.)},
keywords = {Brain,Brain: physiology,Cognition,Cognition: physiology,Humans,Magnetic Resonance Imaging,Multivariate Analysis},
month = {oct},
number = {6158},
pages = {580--4},
pmid = {24179218},
title = {{Functional interactions as big data in the human brain.}},
url = {http://www.sciencemag.org/content/342/6158/580.full},
volume = {342},
year = {2013}
}
@article{Rudolf2011,
abstract = {Abstract Resolving how complexity affects stability of natural communities is of key importance for predicting the consequences of biodiversity loss. Central to previous stability analysis has been the assumption that the resources of a consumer are substitutable. However, during their development, most species change diets; for instance, adults often use different resources than larvae or juveniles. Here, we show that such ontogenetic niche shifts are common in real ecological networks and that consideration of these shifts can alter which species are predicted to be at risk of extinction. Furthermore, niche shifts reduce and can even reverse the otherwise stabilizing effect of complexity. This pattern arises because species with several specialized life stages appear to be generalists at the species level but act as sequential specialists that are hypersensitive to resource loss. These results suggest that natural communities are more vulnerable to biodiversity loss than indicated by previous analyses.},
author = {Rudolf, V. H W and Lafferty, Kevin D.},
doi = {10.1111/j.1461-0248.2010.01558.x},
isbn = {1461-023X},
issn = {14610248},
journal = {Ecology Letters},
keywords = {Body size,Complex life cycle,Diversity-stability,Extinction,Food web,Host-parasite,Ontogenetic niche shift,Predator-prey,Species richness},
month = {jan},
number = {1},
pages = {75--79},
pmid = {21114747},
title = {{Stage structure alters how complexity affects stability of ecological networks}},
url = {http://www.ncbi.nlm.nih.gov/pubmed/21114747},
volume = {14},
year = {2011}
}
@article{Kefi2012,
abstract = {Organisms eating each other are only one of many types of well documented and important interactions among species. Other such types include habitat modification, predator interference and facilitation. However, ecological network research has been typically limited to either pure food webs or to networks of only a few ({\textless}3) interaction types. The great diversity of non-trophic interactions observed in nature has been poorly addressed by ecologists and largely excluded from network theory. Herein, we propose a conceptual framework that organises this diversity into three main functional classes defined by how they modify specific parameters in a dynamic food web model. This approach provides a path forward for incorporating non-trophic interactions in traditional food web models and offers a new perspective on tackling ecological complexity that should stimulate both theoretical and empirical approaches to understanding the patterns and dynamics of diverse species interactions in nature.},
author = {K{\'{e}}fi, Sonia and Berlow, Eric L. and Wieters, Evie A. and Navarrete, Sergio A. and Petchey, Owen L. and Wood, Spencer a. and Boit, Alice and Joppa, Lucas N. and Lafferty, Kevin D. and Williams, Richard J. and Martinez, Neo D. and Menge, Bruce a. and Blanchette, Carol a. and Iles, Alison C. and Brose, Ulrich},
doi = {10.1111/j.1461-0248.2011.01732.x},
isbn = {1461-0248},
issn = {1461023X},
journal = {Ecology Letters},
keywords = {Ecological network,Ecosystem engineering,Facilitation,Food web,Interaction modification,Non-trophic interactions,Trophic interactions},
month = {mar},
number = {4},
pages = {291--300},
pmid = {22313549},
title = {{More than a meal... integrating non-feeding interactions into food webs}},
url = {http://www.ncbi.nlm.nih.gov/pubmed/22313549},
volume = {15},
year = {2012}
}
@article{Guimera2007,
abstract = {Modularity is one of the most prominent properties of real-world complex networks. Here, we address the issue of module identification in two important classes of networks: bipartite networks and directed unipartite networks. Nodes in bipartite networks are divided into two nonoverlapping sets, and the links must have one end node from each set. Directed unipartite networks only have one type of node, but links have an origin and an end. We show that directed unipartite networks can be conveniently represented as bipartite networks for module identification purposes. We report on an approach especially suited for module detection in bipartite networks, and we define a set of random networks that enable us to validate the approach.},
archivePrefix = {arXiv},
arxivId = {physics/0701151},
author = {Guimer??, Roger and Sales-Pardo, Marta and Amaral, Lu??s a Nunes},
doi = {10.1103/PhysRevE.76.036102},
eprint = {0701151},
isbn = {1539-3755},
issn = {15393755},
journal = {Physical Review E - Statistical, Nonlinear, and Soft Matter Physics},
month = {sep},
number = {3},
pages = {036102},
pmid = {17930301},
primaryClass = {physics},
title = {{Module identification in bipartite and directed networks}},
url = {http://link.aps.org/doi/10.1103/PhysRevE.76.036102},
volume = {76},
year = {2007}
}
@article{Olesen2007,
abstract = {In natural communities, species and their interactions are often organized as nonrandom networks, showing distinct and repeated complex patterns. A prevalent, but poorly explored pattern is ecological modularity, with weakly interlinked subsets of species (modules), which, however, internally consist of strongly connected species. The importance of modularity has been discussed for a long time, but no consensus on its prevalence in ecological networks has yet been reached. Progress is hampered by inadequate methods and a lack of large datasets. We analyzed 51 pollination networks including almost 10,000 species and 20,000 links and tested for modularity by using a recently developed simulated annealing algorithm. All networks with {\textgreater}150 plant and pollinator species were modular, whereas networks with {\textless}50 species were never modular. Both module number and size increased with species number. Each module includes one or a few species groups with convergent trait sets that may be considered as coevolutionary units. Species played different roles with respect to modularity. However, only 15{\%} of all species were structurally important to their network. They were either hubs (i.e., highly linked species within their own module), connectors linking different modules, or both. If these key species go extinct, modules and networks may break apart and initiate cascades of extinction. Thus, species serving as hubs and connectors should receive high conservation priorities.},
author = {Olesen, Jens M and Bascompte, Jordi and Dupont, Yoko L and Jordano, Pedro},
doi = {10.1073/pnas.0706375104},
isbn = {0027-8424},
issn = {0027-8424},
journal = {Proceedings of the National Academy of Sciences of the United States of America},
keywords = {Animals,Biological,Ecosystem,Models,Pollination,Pollination: physiology,Population Dynamics},
month = {dec},
number = {50},
pages = {19891--19896},
pmid = {18056808},
title = {{The modularity of pollination networks.}},
url = {http://www.pubmedcentral.nih.gov/articlerender.fcgi?artid=2148393{\&}tool=pmcentrez{\&}rendertype=abstract},
volume = {104},
year = {2007}
}
@article{Allesina2009,
abstract = {The concept of a group is ubiquitous in biology. It underlies classifications in evolution and ecology, including those used to describe phylogenetic levels, the habitat and functional roles of organisms in ecosystems. Surprisingly, this concept is not explicitly included in simple models for the structure of food webs, the ecological networks formed by consumer–resource interactions. We present here the simplest possible model based on groups, and show that it performs substantially better than current models at predicting the structure of large food webs. Our group-based model can be applied to different types of biological and non-biological networks, and for the first time merges in the same framework two important notions in network theory: that of compartments (sets of highly interacting nodes) and that of roles (sets of nodes that have similar interaction patterns). This model provides a basis to examine the significance of groups in biological networks and to develop more accurate models for ecological network structure. It is especially relevant at a time when a new generation of empirical data is providing increasingly large food webs. Few concepts in biology are so pervasive as that of a group. Since the work of Linnaeus more than 250 years ago, biologists have tried to classify organisms into species, species into genera, genera into families, and so on. In ecology, species have been grouped according to their habitat, such as benthic vs. pelagic species in marine environments, below-ground and canopy communities in tropical forests, autotrophs, primary consumers and detri-tivores, based on energy sources, omnivores, specialists, and generalists based on their diet breadth, to name a few examples of grouping species. Surprisingly, however, the concept of group has been largely left out from the construction of simple models for food web structure (Cohen et al. 1990; Williams {\&} Martinez 2000; Cattin et al. 2004; Allesina et al. 2008). Only Cattin et al. (2004) implicitly consider that similar predators act in a similar way. To date no model addressed the presence of groups explicitly. These stochastic models provide a way to construct with a few simple rules {\^{O}}realistic{\~{O}} food webs or networks describing {\^{O}}who eats whom{\~{O}} in an ecosystem. The simplest models are motivated by a few ecological principles and require only two parameters, the total number of species, or species richness, and the total number of connections, or connectance, in the network. The recent comparisons of these models based on likelihoods have therefore consid-ered the same number of parameters (Allesina et al. 2008). However, the use of likelihoods opens the door for more general comparisons among models of varying complexity based on information criteria. We introduce here network models based on the concept of groups and compare for the first time, models with a different number of parameters. We show that dividing species into groups yields critical information for building better models of food web structure, especially for large networks. In all the simple models of food web structure proposed so far, species are ranked into a one-dimensional hierarchy with this ranking providing the basis to establish species{\~{O}} interactions (Cohen et al. 1990; Williams {\&} Martinez 2000; Cattin et al. 2004; Allesina et al. 2008). For example, in the cascade model (Cohen et al. 1990), a species can prey with a given probability upon any species whose position in the ranking is lower, but cannot prey upon those with higher},
author = {Allesina, Stefano and Pascual, Mercedes},
doi = {10.1111/j.1461-0248.2009.01321.x},
isbn = {1461-0248},
issn = {1461023X},
journal = {Ecology Letters},
keywords = {Akaike information criterion,Clustering algorithm,Compartment,Connectance,Food web model,Group,Likelihood,Model selection,Species richness,Trophic role,Trophospecies},
month = {jul},
number = {7},
pages = {652--662},
pmid = {19453619},
title = {{Food web models: a plea for groups}},
url = {http://www.ncbi.nlm.nih.gov/pubmed/19453619},
volume = {12},
year = {2009}
}
@article{Luczkovich2003,
abstract = {We present a graph theoretic model of analysing food web structure called regular equivalence. Regular equivalence is a method for partitioning the species in a food web into "isotrophic classes" that play the same structural roles, even if they are not directly consuming the same prey or if they do not share the same predators. We contrast regular equivalence models, in which two species are members of the same trophic group if they have trophic links to the same set of other trophic groups, with structural equivalence models, in which species are equivalent if they are connected to the exact same other species. Here, the regular equivalence approach is applied to two published food webs: (1) a topological web (Malaysian pitcher plant insect food web) and (2) a carbon-flow web (St. Marks, Florida seagrass ecosystem food web). Regular equivalence produced a more satisfactory set of classes than did the structural approach, grouping basal taxa with other basal taxa and not with top predators. Regular equivalence models provide a way to mathematically formalize trophic position, trophic group and trophic niche. These models are part of a family of models that includes structural models used extensively by ecologists now. Regular equivalence models uncover similarities in trophic roles at a higher level of organization than do the structural models. The approach outlined is useful for measuring the trophic roles of species in food web models, measuring similarity in trophic relations of two or more species, comparing food webs over time and across geographic regions, and aggregating taxa into trophic groups that reduce the complexity of ecosystem feeding relations without obscuring network relationships. In addition, we hope the approach will prove useful in predicting the outcome of predator-prey interactions in experimental studies.},
author = {Luczkovich, Joseph J. and Borgatti, Stephen P. and Johnson, Jeffrey C. and Everett, Martin G.},
doi = {10.1006/jtbi.2003.3147},
file = {:Users/alyssacirtwill/Documents/Papers/Luczkovich et al.{\_}2003{\_}Journal of Theoretical Biology.pdf:pdf},
isbn = {1252328184},
issn = {00225193},
journal = {Journal of Theoretical Biology},
month = {feb},
number = {3},
pages = {303--321},
pmid = {12468282},
title = {{Defining and measuring trophic role similarity in food webs using regular equivalence}},
url = {http://linkinghub.elsevier.com/retrieve/pii/S002251930393147X},
volume = {220},
year = {2003}
}
@article{Aizen2012,
abstract = {The loss of interactions from mutualistic networks could foreshadow both plant and animal species extinctions. Yet, the characteristics of interactions that predispose them to disruption are largely unknown. We analyzed 12 pollination webs from isolated hills ("sierras"), in Argentina, ranging from tens to thousands of hectares. We found evidence of nonrandom loss of interactions with decreasing sierra size. Low interaction frequency and high specialization between interacting partners contributed additively to increase the vulnerability of interactions to disruption. Interactions between generalists in the largest sierras were ubiquitous across sierras, but many of them lost their central structural role in the smallest sierras. Thus, particular configurations of interaction networks, along with unique ecological relations and evolutionary pathways, could be lost forever after habitat reduction.},
archivePrefix = {arXiv},
arxivId = {20},
author = {Aizen, Marcelo A. and Sabatino, Malena and Tylianakis, Jason M.},
doi = {10.1126/science.1215320},
eprint = {20},
isbn = {0036-8075},
issn = {1095-9203},
journal = {Science},
keywords = {Animals,Argentina,Biodiversity,Biological Evolution,Ecosystem,Food Chain,Insects,Plants,Pollination,Population Dynamics,Symbiosis},
month = {mar},
number = {6075},
pages = {1486--1489},
pmid = {22442482},
title = {{Specialization and rarity predict nonrandom loss of interactions from mutualist networks}},
url = {http://www.ncbi.nlm.nih.gov/pubmed/22442482},
volume = {335},
year = {2012}
}
@article{Girvan2002,
abstract = {If better informed voters receive favourable policies, then mass media will affect policy because$\backslash$r$\backslash$nmass media provide most of the information people use in voting. This paper models the incentives of the$\backslash$r$\backslash$nmedia to deliver news to different groups. The increasing-returns-to-scale technology and advertising$\backslash$r$\backslash$nfinancing of media firms induce them to provide more news to large groups, such as taxpayers and$\backslash$r$\backslash$ndispersed consumer interests, and groups that are valuable to advertisers. This news bias alters the tradeoff$\backslash$r$\backslash$nin political competition and therefore introduces a bias in public policy. The paper also discusses the$\backslash$r$\backslash$neffects of broadcast media replacing newspapers as the main information source about politics. The model$\backslash$r$\backslash$npredicts that this change should raise spending on government programmes used by poor and rural voters.},
archivePrefix = {arXiv},
arxivId = {cond-mat/0112110},
author = {Str{\"{o}}mberg, David},
doi = {10.1111/0034-6527.00284},
eprint = {0112110},
isbn = {1467937X},
issn = {00346527},
journal = {Review of Economic Studies},
keywords = {Algorithms,Animals,Community Networks,Computer Simulation,Humans,Models,Nerve Net,Nerve Net: physiology,Neural Networks (Computer),Social Behavior,Theoretical},
month = {jul},
number = {1},
pages = {265--284},
pmid = {12060727},
primaryClass = {cond-mat},
title = {{Mass media competition, political competition, and public policy}},
url = {http://www.pnas.org/content/99/12/7821.short{\%}5Cnhttp://www.ncbi.nlm.nih.gov/pmc/articles/PMC122977/},
volume = {71},
year = {2004}
}
@article{Kirkpatrick1983,
abstract = {There is a deep and useful connection between statistical mechanics (the behavior of systems with many degrees of freedom in thermal equilibrium at a finite temperature) and multivariate or combinatorial optimization (finding the minimum of a given function depending on many parameters). A detailed analogy with annealing in solids provides a framework for optimization of the properties of very large and complex systems. This connection to statistical mechanics exposes new information and provides an unfamiliar perspective on traditional optimization problems and methods.},
author = {Kirkpatrick, S and Gelatt, C D and Vecchi, M P},
journal = {Science},
number = {4598},
pages = {671--680.},
title = {{Optimization by simulated annealing}},
volume = {220},
year = {1982}
}
@article{Milo2002,
abstract = {Complex networks are studied across many fields of science. To uncover their structural design principles, we defined "network motifs," patterns of interconnections occurring in complex networks at numbers that are significantly higher than those in randomized networks. We found such motifs in networks from biochemistry, neurobiology, ecology, and engineering. The motifs shared by ecological food webs were distinct from the motifs shared by the genetic networks of Escherichia coli and Saccharomyces cerevisiae or from those found in the World Wide Web. Similar motifs were found in networks that perform information processing, even though they describe elements as different as biomolecules within a cell and synaptic connections between neurons in Caenorhabditis elegans. Motifs may thus define universal classes of networks. This approach may uncover the basic building blocks of most networks.},
archivePrefix = {arXiv},
arxivId = {0908.1143v1},
author = {Milo, R and Shen-Orr, S and Itzkovitz, S and Kashtan, N and Chklovskii, D and Alon, U},
doi = {10.1126/science.298.5594.824},
eprint = {0908.1143v1},
isbn = {1095-9203 (Electronic)$\backslash$r0036-8075 (Linking)},
issn = {1095-9203},
journal = {Science},
keywords = {*Algorithms,*Electronics,*Food Chain,*Gene Expression Regulation,*Internet,Animals,Caenorhabditis elegans/anatomy {\&} histology/physiol,Escherichia coli/genetics,Nerve Net/*physiology,Neurons/physiology,Saccharomyces cerevisiae/genetics,Synapses/physiology},
month = {oct},
number = {5594},
pages = {824--827},
pmid = {12399590},
title = {{Network motifs: simple building blocks of complex networks}},
url = {http://www.ncbi.nlm.nih.gov/pubmed/12399590},
volume = {298},
year = {2002}
}
@article{Fontaine2011,
abstract = {Interactions among species drive the ecological and evolutionary processes in ecological communities. These interactions are effectively key components of biodiversity. Studies that use a network approach to study the structure and dynamics of communities of interacting species have revealed many patterns and associated processes. Historically these studies were restricted to trophic interactions, although network approaches are now used to study a wide range of interactions, including for example the reproductive mutualisms. However, each interaction type remains studied largely in isolation from others. Merging the various interaction types within a single integrative framework is necessary if we want to further our understanding of the ecological and evolutionary dynamics of communities. Dividing the networks up is a methodological convenience as in the field the networks occur together in space and time and will be linked by shared species. Herein, we outline a conceptual framework for studying networks composed of more than one type of interaction, highlighting key questions and research areas that would benefit from their study.},
author = {Fontaine, Colin and Guimar{\~{a}}es, Paulo R. and K{\'{e}}fi, Sonia and Loeuille, Nicolas and Memmott, Jane and van der Putten, Wim H. and van Veen, Frank J F and Th{\'{e}}bault, Elisa},
doi = {10.1111/j.1461-0248.2011.01688.x},
isbn = {1461-023X},
issn = {1461023X},
journal = {Ecology Letters},
keywords = {Aboveground-belowground,Evolutionary dynamic,Facilitation,Interaction network,Mutualistic,Parasitic,Stability,Traits,Trophic},
month = {nov},
number = {11},
pages = {1170--1181},
pmid = {21951949},
title = {{The ecological and evolutionary implications of merging different types of networks}},
url = {http://www.ncbi.nlm.nih.gov/pubmed/21951949},
volume = {14},
year = {2011}
}
@book{Pascual2006,
abstract = {This book is based on proceedings from a February 2004 Santa Fe Institute workshop. Its contributing chapter authors treat the ecology of predator-prey interactions and food web theory, structure, and dynamics, joining researchers who also work on complex systems and on large nonlinear networks from the points of view of other sub-fields within ecology. Food webs play a central role in the debates on the role of complexity in stability, persistence, and resilience. Better empirical data and the exploding interest in the subject of networks across social, physical, and natural sciences prompted creation of this volume. The book explores the boundaries of what is known of the relationship between structure and dynamics in ecological networks and defines directions for future developments in this field.},
address = {Oxford},
author = {Pascual, Mercedes and Dunne, Jennifer a.},
booktitle = {Ecology},
doi = {10.1890/0012-9658(2007)88[265:ENLSAD]2.0.CO;2},
editor = {Pascual, Mercedes and Dunne, Jennifer A},
isbn = {0195188160},
issn = {0012-9658},
number = {1},
pages = {265--266},
pmid = {6702},
publisher = {Oxford University Press},
title = {{Ecological networks: linking structure and dynamics in food webs}},
url = {https://books.google.es/books?id=1nuajS1tZh0C},
volume = {88},
year = {2007}
}
@article{Huang2013,
abstract = {Most parasites infect multiple hosts, but what factors determine the range of hosts a given parasite can infect? Understanding the broad scale determinants of parasite distributions across host lineages is important for predicting pathogen emergence in new hosts and for estimating pathogen diversity in understudied host species. In this study, we used a new data set on 793 parasite species reported from free-ranging populations of 64 carnivore species to examine the factors that influence parasite sharing between host species. Our results showed that parasites are more commonly shared between phylogenetically related host species pairs. Additionally, host species with higher similarity in biological traits and greater geographic range overlap were also more likely to share parasite species. Of three measures of phylogenetic relatedness considered here, the number divergence events that separated host species pairs most strongly influenced the likelihood of parasite sharing. We also showed that viruses and helminths tend to infect carnivore hosts within more restricted phylogenetic ranges than expected by chance. Overall, our results underscore the importance of host evolutionary history in determining parasite host range, even when simultaneously considering other factors such as host ecology and geographic distribution.},
author = {Huang, Shan and Bininda-Emonds, Olaf R P and Stephens, Patrick R. and Gittleman, John L. and Altizer, Sonia},
doi = {10.1111/1365-2656.12160},
isbn = {1365-2656},
issn = {13652656},
journal = {Journal of Animal Ecology},
keywords = {Biological similarity,Disease sharing,Geographic overlap,Host phylogenetic clustering,Host-parasite interactions,Wild carnivores,Wildlife conservation,Wildlife disease},
number = {3},
pages = {671--680},
pmid = {24289314},
title = {{Phylogenetically related and ecologically similar carnivores harbour similar parasite assemblages}},
url = {http://onlinelibrary.wiley.com/doi/10.1111/1365-2656.12160/abstract},
volume = {83},
year = {2014}
}
@article{Warren2010,
abstract = {Abstract Although parasites represent an important component of ecosystems, few field and theoretical studies have addressed the structure of parasites in food webs. We evaluate the structure of parasitic links in an extensive salt marsh food web, with a new model distinguishing parasitic links from non-parasitic links among free-living species. The proposed model is an extension of the niche model for food web structure, motivated by the potential role of size (and related metabolic rates) in structuring food webs. The proposed extension captures several properties observed in the data, including patterns of clustering and nestedness, better than does a random model. By relaxing specific assumptions, we demonstrate that two essential elements of the proposed model are the similarity of a parasite's hosts and the increasing degree of parasite specialization, along a one-dimensional niche axis. Thus, inverting one of the basic rules of the original model, the one determining consumers' generality appears critical. Our results support the role of size as one of the organizing principles underlying niche space and food web topology. They also strengthen the evidence for the non-random structure of parasitic links in food webs and open the door to addressing questions concerning the consequences and origins of this structure.},
author = {Warren, Christopher P. and Pascual, Mercedes and Lafferty, Kevin D. and Kuris, Armand M.},
doi = {10.1007/s12080-009-0069-x},
isbn = {1874-1738},
issn = {18741738},
journal = {Theoretical Ecology},
keywords = {Food web structure,Food webs with parasites,Niche model},
month = {jan},
number = {4},
pages = {285--294},
pmid = {25540673},
title = {{The inverse niche model for food webs with parasites}},
url = {http://link.springer.com/10.1007/s12080-009-0069-x},
volume = {3},
year = {2010}
}
@article{Lafferty2008,
abstract = {Parasitism is the most common consumer strategy among organisms, yet only recently has there been a call for the inclusion of infectious disease agents in food webs. The value of this effort hinges on whether parasites affect food-web properties. Increasing evidence suggests that parasites have the potential to uniquely alter food-web topology in terms of chain length, connectance and robustness. In addition, parasites might affect food-web stability, interaction strength and energy flow. Food-web structure also affects infectious disease dynamics because parasites depend on the ecological networks in which they live. Empirically, incorporating parasites into food webs is straightforward. We may start with existing food webs and add parasites as nodes, or we may try to build food webs around systems for which we already have a good understanding of infectious processes. In the future, perhaps researchers will add parasites while they construct food webs. Less clear is how food-web theory can accommodate parasites. This is a deep and central problem in theoretical biology and applied mathematics. For instance, is representing parasites with complex life cycles as a single node equivalent to representing other species with ontogenetic niche shifts as a single node? Can parasitism fit into fundamental frameworks such as the niche model? Can we integrate infectious disease models into the emerging field of dynamic food-web modelling? Future progress will benefit from interdisciplinary collaborations between ecologists and infectious disease biologists.},
author = {Lafferty, Kevin D. and Allesina, Stefano and Arim, Matias and Briggs, Cherie J. and {De Leo}, Giulio and Dobson, Andrew P. and Dunne, Jennifer a. and Johnson, Pieter T J and Kuris, Armand M. and Marcogliese, David J. and Martinez, Neo D. and Memmott, Jane and Marquet, Pablo a. and McLaughlin, John P. and Mordecai, Erin a. and Pascual, Mercedes and Poulin, Robert and Thieltges, David W.},
doi = {10.1111/j.1461-0248.2008.01174.x},
isbn = {Thompson, R.M., Mouritsen, K.N. {\&} Poulin, R. ( 2005 ). Importance of parasites and their life cycle characteristics in determining the structure of a large marine food web. J. Anim. Ecol., 74, 77 – 85.},
issn = {1461023X},
journal = {Ecology Letters},
keywords = {Disease,Food web network,Parasite},
month = {jun},
number = {6},
pages = {533--546},
pmid = {18462196},
title = {{Parasites in food webs: The ultimate missing links}},
url = {http://www.pubmedcentral.nih.gov/articlerender.fcgi?artid=2408649{\&}tool=pmcentrez{\&}rendertype=abstract},
volume = {11},
year = {2008}
}
@article{Thieltges2011,
abstract = {This data set presents a food web for Otago Harbour, an intertidal mudflat ecosystem in New Zealand. The harbor consists of a series of mudflats exposed at low tide, each separated from its closest neighbor by 200–400 m. This food web has three noteworthy attributes: (1) high resolution of free-living organisms, (2) inclusion of metazoan parasites and other infectious agents, and (3) inclusion of ontogenetic stages of parasites with complex life cycles. The food web contains 180 nodes, 142 species/assemblages, and 1924 links. Of the 142 species/assemblages, 3 are basal, 123 are free-living, and 19 are infectious. Data on the free- living assemblages and parasitism were gathered during original field sampling and supplemented with information from additional published sources and local expert knowledge. Taxonomic resolution is high, although a few functional or taxonomic groups (e.g., phytoplankton, macroalgae) are lumped into single nodes. Each ontogenetic stage of parasites with complex life cycles is treated separately and coded accordingly. For each node, we have included additional information such as taxonomy, life history, residency, and mobility. Further, for each link, we define a specific interaction type. We present the data and metadata in the system-neutral format standardized by R. F. Hechinger and colleagues, and thus we recognize variables that are not represented in our data set but may be added by further study.},
author = {Mouritsen, Kim N. and Poulin, Robert and McLaughlin, John P. and Thieltges, David W.},
doi = {10.1890/11-0371.1},
isbn = {0012-9658},
issn = {0012-9658},
journal = {Ecology},
number = {10},
pages = {2006--2006},
title = {{Food web including metazoan parasites for an intertidal ecosystem in New Zealand}},
url = {http://www.esajournals.org/doi/abs/10.1890/11-0371.1},
volume = {92},
year = {2011}
}
@article{Mouritsen2011,
abstract = {This data set presents a food web for Otago Harbour, an intertidal mudflat ecosystem in New Zealand. The harbor consists of a series of mudflats exposed at low tide, each separated from its closest neighbor by 200–400 m. This food web has three noteworthy attributes: (1) high resolution of free-living organisms, (2) inclusion of metazoan parasites and other infectious agents, and (3) inclusion of ontogenetic stages of parasites with complex life cycles. The food web contains 180 nodes, 142 species/assemblages, and 1924 links. Of the 142 species/assemblages, 3 are basal, 123 are free-living, and 19 are infectious. Data on the free- living assemblages and parasitism were gathered during original field sampling and supplemented with information from additional published sources and local expert knowledge. Taxonomic resolution is high, although a few functional or taxonomic groups (e.g., phytoplankton, macroalgae) are lumped into single nodes. Each ontogenetic stage of parasites with complex life cycles is treated separately and coded accordingly. For each node, we have included additional information such as taxonomy, life history, residency, and mobility. Further, for each link, we define a specific interaction type. We present the data and metadata in the system-neutral format standardized by R. F. Hechinger and colleagues, and thus we recognize variables that are not represented in our data set but may be added by further study.},
author = {Mouritsen, Kim N. and Poulin, Robert and McLaughlin, John P. and Thieltges, David W.},
doi = {10.1890/11-0371.1},
isbn = {0012-9658},
issn = {0012-9658},
journal = {Ecology},
month = {oct},
number = {10},
pages = {2006--2006},
title = {{Food web including metazoan parasites for an intertidal ecosystem in New Zealand}},
url = {http://www.esajournals.org/doi/abs/10.1890/11-0371.1},
volume = {92},
year = {2011}
}
@article{Zander2011,
abstract = {This data set presents a food web for the Sylt tidal basin, an intertidal ecosystem in Germany and Denmark. The intertidal part of this bight consists of extensive tidal flats with the main habitats being lugworm sandflats, seagrass meadows, and mixed mussel and oyster beds. This food web has three noteworthy attributes: (1) high resolution of free-living organisms, (2) inclusion of metazoan parasites and other infectious agents, and (3) inclusion of ontogenetic stages of parasites with complex life cycles. The food web contains 230 nodes, 161 species/assemblages, and 3338 links. Of the 161 species/assemblages, 6 are basal, 120 are free- living, and 35 are infectious. Data on the free-living assemblages and parasitism were gathered during original field sampling and supplemented with information from additional published sources and local expert knowledge. Taxonomic resolution is high, although a few functional or taxonomic groups (e.g., phytoplankton, macroalgae) are lumped into single nodes. Each ontogenetic stage of parasites with complex life cycles is treated separately and coded accordingly. For each node, we have included additional information such as taxonomy, life history, residency, and vagility. Further, for each link, we define a specific interaction type. We present the data and metadata in the system-neutral format standardized by R. F. Hechinger and colleagues, and thus we recognize variables that are not represented in our data set but may be added by further study.},
author = {Thieltges, David W. and Reise, Karsten and Mouritsen, Kim N. and McLaughlin, John P. and Poulin, Robert},
doi = {10.1890/11-0351.1},
isbn = {0012-9658},
issn = {0012-9658},
journal = {Ecology},
number = {10},
pages = {2005--2005},
title = {{Food web including metazoan parasites for a tidal basin in Germany and Denmark}},
volume = {92},
year = {2011}
}
@article{Kashtan2004,
abstract = {Biological and technological networks contain patterns, termed network motifs, which occur far more often than in randomized networks. Network motifs were suggested to be elementary building blocks that carry out key functions in the network. It is of interest to understand how network motifs combine to form larger structures. To address this, we present a systematic approach to define "motif generalizations": families of motifs of different sizes that share a common architectural theme. To define motif generalizations, we first define "roles" in a subgraph according to structural equivalence. For example, the feedforward loop triad--a motif in transcription, neuronal, and some electronic networks--has three roles: an input node, an output node, and an internal node. The roles are used to define possible generalizations of the motif. The feedforward loop can have three simple generalizations, based on replicating each of the three roles and their connections. We present algorithms for efficiently detecting motif generalizations. We find that the transcription networks of bacteria and yeast display only one of the three generalizations, the multi-output feedforward generalization. In contrast, the neuronal network of C. elegans mainly displays the multi-input generalization. Forward-logic electronic circuits display a multi-input, multi-output hybrid. Thus, networks which share a common motif can have very different generalizations of that motif. Using mathematical modeling, we describe the information processing functions of the different motif generalizations in transcription, neuronal, and electronic networks.},
archivePrefix = {arXiv},
arxivId = {q-bio/0312019},
author = {Kashtan, N. and Itzkovitz, S. and Milo, R. and Alon, U.},
doi = {10.1103/PhysRevE.70.031909},
eprint = {0312019},
isbn = {1539-3755 (Print)$\backslash$r1539-3755 (Linking)},
issn = {15393755},
journal = {Physical Review E - Statistical, Nonlinear, and Soft Matter Physics},
month = {sep},
number = {3 1},
pages = {031909},
pmid = {15524551},
primaryClass = {q-bio},
title = {{Topological generalizations of network motifs}},
url = {http://link.aps.org/doi/10.1103/PhysRevE.70.031909},
volume = {70},
year = {2004}
}
@article{Olesen2008,
abstract = {Despite a strong current interest in ecological networks, the bulk of studies are static descriptions of the structure of networks, and very few analyze their temporal dynamics. Yet, understanding network dynamics is important in order to relate network patterns to ecological processes. We studied the day-to-day dynamics of an arctic pollination interaction network over two consecutive seasons. First, we found that new species entering the network tend to interact with already well-connected species, although there are deviations from this trend due, for example, to morphological mismatching between plant and pollinator traits and nonoverlapping phenophases of plant and pollinator species. Thus, temporal dynamics provides a mechanistic explanation for previously reported network patterns such as the heterogeneous distribution of number of interactions across species. Second, we looked for the ecological properties most likely to be mediating this dynamical process and found that both abundance and phenoph...},
author = {Olesen, Jens M. and Bascompte, Jordi and Elberling, Heidi and Jordano, Pedro},
doi = {10.1890/07-0451.1},
isbn = {0012-9658},
issn = {00129658},
journal = {Ecology},
keywords = {Abundance,Arctic,Constraint,Ecological network,Linkages between species,Mutualistic network,Phenology,Pollination,Preferential attachment},
month = {jun},
number = {6},
pages = {1573--1582},
pmid = {18589522},
title = {{Temporal dynamics in a pollination network}},
url = {http://www.ncbi.nlm.nih.gov/pubmed/18589522},
volume = {89},
year = {2008}
}
@article{Lafferty1996,
abstract = {Parasites that are transmitted from prey to predator are often associated with altered prey behavior. Although many concur that behavior modification is a parasite strategy that facilitates transmission by making parasitized prey easier for predators to capture, there is little evidence from field experiments. We observed that conspicuous behaviors exhibited by killifish (Fundulus parvipinnis) were associated with parasitism by larval trematodes. A field experiment indicated that parasitized fish were substantially more susceptible to predation by final host birds. These results support the behavior-modification hypothesis and emphasize the importance of parasites for predator-prey interactions.},
author = {Lafferty, Kevin D. and {Kimo Morris}, a.},
doi = {10.2307/2265536},
isbn = {00129658},
issn = {00129658},
journal = {Ecology},
keywords = {Behavior-modification hypothesis,Bird predation,Differential predation,Effects of parasites on host behavior,Euhaplorchis californiensis,Field experiment,Fundulus,Killifish,Metacercariae,Parasitestrematode},
number = {5},
pages = {1390--1397},
pmid = {199699132186},
title = {{Altered behavior of parasitized killifish increases susceptibility to predation by bird final hosts}},
url = {http://www.esajournals.org/doi/abs/10.2307/2265536},
volume = {77},
year = {1996}
}
@article{Paine1966,
abstract = {It is suggested that local animal species diversity is related to the number of predators in the system and their efficiency in preventing single species from monopolizing some important, limiting, requisite. In the marine rocky intertidal this requisite usually is space. Where predators capable of preventing monopolies are missing, or are experimentally removed, the systems become less diverse. On a local scale, no relationship between latitude (10⚬ to 49⚬ N.) and diversity was found. On a geographic scale, an increased stability of annual production may lead to an increased capacity for systems to support higher-level carnivores. Hence tropical, or other, ecosystems are more diverse, and are characterized by disproportionately more carnivores.},
archivePrefix = {arXiv},
arxivId = {1011.1669v3},
author = {Paire, Robert T.},
doi = {10.4037/ajcc2016979},
eprint = {1011.1669v3},
isbn = {9788578110796},
issn = {10623264},
journal = {The American Naturalist},
number = {910},
pages = {65--75},
pmid = {25246403},
title = {{Food web complexity and species diversity}},
url = {http://www.jstor.org/stable/2459379{\%}5Cnhttp://www.journals.uchicago.edu/doi/10.1086/282400},
volume = {100},
year = {1966}
}
@article{Sugihara1989,
abstract = {The robustness of five common food web properties is examined by varying the resolution of the data through aggregation of trophic groupings. A surprising constancy in each of these properties is revealed as webs are collapsed down to approximately half their original size. This analysis of 60 invertebrate-dominated community food webs confirms the existence of all but one of these properties in such webs and addresses a common concern held by critics of food web theory that observed food web properties may be sensitive to trophic aggregation. The food web statistics (chain length; predator/prey ratio; fraction of top, intermediate, and bottom species; and rigid circuits) are scaling in the sense that they remain roughly invariant over a wide range of data resolution. As such, within present standards of reporting food web data, these statistics may be used to compare systems whose trophic data are resolved differently within a factor of 2.},
author = {Sugihara, George and Schoenly, Kenneth G. and Trombla, Alan},
doi = {10.1126/science.2740915},
isbn = {0036-8075},
issn = {0036-8075},
journal = {Science, New Series},
keywords = {Animals,Food,Food Supply,Invertebrates,Models, Biological},
month = {jul},
number = {4913},
pages = {48--52},
pmid = {2740915},
title = {{Scale Invariance in Food Web Properties}},
url = {http://www.ncbi.nlm.nih.gov/pubmed/2740915},
volume = {245},
year = {1989}
}
@article{James2012,
abstract = {Complex networks of interactions are ubiquitous and are particularly important in ecological communities, in which large numbers of species exhibit negative (for example, competition or predation) and positive (for example, mutualism) interactions with one another. Nestedness in mutualistic ecological networks is the tendency for ecological specialists to interact with a subset of species that also interact with more generalist species. Recent mathematical and computational analysis has suggested that such nestedness increases species richness. By examining previous results and applying computational approaches to 59 empirical data sets representing mutualistic plant–pollinator networks, we show that this statement is incorrect. A simpler metric—the number of mutualistic partners a species has—is a much better predictor of individual species survival and hence, community persistence. Nestedness is, at best, a secondary covariate rather than a causative factor for biodiversity in mutualistic communities. Analysis of complex networks should be accompanied by analysis of simpler, underpinning mechanisms that drive multiple higher-order network properties.},
author = {James, Alex and Pitchford, Jonathan W and Plank, Michael J},
doi = {10.1038/nature11214},
isbn = {0028-0836},
issn = {1476-4687},
journal = {Nature},
keywords = {Animals,Biodiversity,Ecology,Ecosystem,Models,Theoretical},
month = {jul},
number = {7406},
pages = {227--30},
pmid = {22722863},
publisher = {Nature Publishing Group},
title = {{Disentangling nestedness from models of ecological complexity.}},
url = {http://dx.doi.org/10.1038/nature11214},
volume = {487},
year = {2012}
}
@article{Thieltges2013,
abstract = {While the recent inclusion of parasites into food-web studies has highlighted the role of parasites as consumers, there is accumulating evidence that parasites can also serve as prey for predators. Here we investigated empirical patterns of pre- dation on parasites and their relationships with parasite transmission in eight topological food webs representing marine and freshwater ecosystems. Within each food web, we examined links in the typical predator – prey sub web as well as the predator – parasite sub web, i.e. the quadrant of the food web indicating which predators eat parasites. Most predator – parasite links represented ‘ concomitant predation ' (consumption and death of a parasite along with the prey/host; 58 – 72{\%}), followed by ‘ trophic transmission ' (predator feeds on infected prey and becomes infected; 8 – 32{\%}) and predation on free-living parasite life-cycle stages (4 – 30{\%}). Parasite life-cycle stages had, on average, between 4.2 and 14.2 predators. Among the food webs, as predator richness increased, the number of links exploited by trophically transmitted parasites increased at about the same rate as did the number of links where these stages serve as prey. On the whole, our analyses suggest that predation on parasites has important consequences for both predators and parasites, and food web structure. Because our analysis is solely based on topological webs, determining the strength of these interactions is a promising avenue for future research.},
author = {Thieltges, David W. and Amundsen, Per Arne and Hechinger, Ryan F. and Johnson, Pieter T J and Lafferty, Kevin D. and Mouritsen, Kim N. and Preston, Daniel L. and Reise, Karsten and Zander, C. Dieter and Poulin, Robert},
doi = {10.1111/j.1600-0706.2013.00243.x},
isbn = {1600-0706},
issn = {00301299},
journal = {Oikos},
month = {apr},
number = {10},
pages = {1473--1482},
title = {{Parasites as prey in aquatic food webs: Implications for predator infection and parasite transmission}},
url = {http://doi.wiley.com/10.1111/j.1600-0706.2013.00243.x http://onlinelibrary.wiley.com/doi/10.1111/j.1600-0706.2013.00243.x/full},
volume = {122},
year = {2013}
}
@article{Otto2007,
abstract = {In natural ecosystems, species are linked by feeding interactions that determine energy fluxes and create complex food webs. The stability of these food webs enables many species to coexist and to form diverse ecosystems. Recent theory finds predator-prey body-mass ratios to be critically important for food-web stability. However, the mechanisms responsible for this stability are unclear. Here we use a bioenergetic consumer-resource model to explore how and why only particular predator-prey body-mass ratios promote stability in tri-trophic (three-species) food chains. We find that this 'persistence domain' of ratios is constrained by bottom-up energy availability when predators are much smaller than their prey and by enrichment-driven dynamics when predators are much larger. We also find that 97{\%} of the tri-trophic food chains across five natural food webs exhibit body-mass ratios within the predicted persistence domain. Further analyses of randomly rewired food webs show that body mass and allometric degree distributions in natural food webs mediate this consistency. The allometric degree distributions hold that the diversity of species' predators and prey decreases and increases, respectively, with increasing species' body masses. Our results demonstrate how simple relationships between species' body masses and feeding interactions may promote the stability of complex food webs.},
author = {Otto, Sonja B and Rall, Bj{\"{o}}rn C. and Brose, Ulrich},
doi = {10.1038/nature06359},
isbn = {0028-0836},
issn = {1476-4687},
journal = {Nature},
keywords = {Biodiversity,Biological,Biomass,Body Weight,Feeding Behavior,Feeding Behavior: physiology,Food Chain,Models},
month = {dec},
number = {7173},
pages = {1226--1229},
pmid = {18097408},
title = {{Allometric degree distributions facilitate food-web stability}},
url = {http://www.ncbi.nlm.nih.gov/pubmed/18097408},
volume = {450},
year = {2007}
}
@article{Stouffer2010b,
abstract = {Understanding food-web persistence is an important long-term objective of ecology because of its relevance in maintaining biodiversity. To date, many dynamic studies of food-web behaviour--both empirical and theoretical--have focused on smaller sub-webs, called trophic modules, because these modules are more tractable experimentally and analytically than whole food webs. The question remains to what degree studies of trophic modules are relevant to infer the persistence of entire food webs. Four trophic modules have received particular attention in the literature: tri-trophic food chains, omnivory, exploitative competition, and apparent competition. Here, we integrate analysis of these modules' dynamics in isolation with those of whole food webs to directly assess the appropriateness of scaling from modules to food webs. We find that there is not a direct, one-to-one, relationship between the relative persistence of modules in isolation and their effect on persistence of an entire food web. Nevertheless, we observe that those modules which are most commonly found in empirical food webs are those that confer the greatest community persistence. As a consequence, we demonstrate that there may be significant dynamic justifications for empirically-observed food-web structure.},
author = {Stouffer, Daniel B. and Bascompte, Jordi},
doi = {10.1111/j.1461-0248.2009.01407.x},
isbn = {1461-0248},
issn = {1461023X},
journal = {Ecology Letters},
keywords = {Apparent competition,Dynamics,Ecological networks,Exploitative competition,Food chain,Network motif,Omnivory,Trophic module},
month = {feb},
number = {2},
pages = {154--161},
pmid = {19968697},
title = {{Understanding food-web persistence from local to global scales}},
url = {http://www.ncbi.nlm.nih.gov/pubmed/19968697},
volume = {13},
year = {2010}
}
@article{Lafferty1992,
abstract = {A model that weighs the energetic cost of parasitism for a predator against the energetic value of prey items that transmit the parasite to the predator suggests that there is often no selective pressure to avoid parasitized prey. This offers through modification are moderate and prey capture is facilitated by parasites. Parasite species that benefit of prey are not mutualistic, however. an explanation for why parasites so frequently exploit predators and prey as definitive and intermediate hosts, respectively. Furthermore, predators may actually benefit from their parasites if energetic costs of parasitism predators},
author = {Lafferty, Kevin D.},
doi = {10.1086/285444},
isbn = {0003-0147},
issn = {0003-0147},
journal = {The American Naturalist},
number = {5},
pages = {854},
title = {{Foraging on Prey that are Modified by Parasites}},
url = {http://www.jstor.org/stable/10.2307/2462792},
volume = {140},
year = {1992}
}
@article{Mikheev2010,
abstract = {ABSTRACT:},
author = {Mikheev, V N and Pasternak, a F and Taskinen, J and Valtonen, E T},
doi = {10.1186/1756-3305-3-17},
issn = {1756-3305},
journal = {Parasites {\&} vectors},
month = {jan},
number = {1},
pages = {17},
pmid = {20226098},
title = {{Parasite-induced aggression and impaired contest ability in a fish host.}},
url = {http://www.pubmedcentral.nih.gov/articlerender.fcgi?artid=2845576{\&}tool=pmcentrez{\&}rendertype=abstract},
volume = {3},
year = {2010}
}
@article{Alexander1981,
author = {Alexander, Martin},
doi = {10.1146/annurev.mi.35.100181.000553},
isbn = {0066-4227 (Print)$\backslash$r0066-4227 (Linking)},
issn = {0066-4227},
journal = {Annual review of microbiology},
pages = {113--33},
pmid = {7027898},
title = {{Why microbial predators and parasites do not eliminate their prey and hosts.}},
url = {http://www.ncbi.nlm.nih.gov/pubmed/7027898},
volume = {35},
year = {1981}
}
@article{Beddington1975,
abstract = {n/a},
author = {Beddington, J.R.},
doi = {10.2307/3866},
isbn = {0021-8790},
issn = {00218790},
journal = {The Journal of Animal Ecology},
number = {1},
pages = {331--340},
title = {{Mutual interference between parasites or predators and its effect on searching efficiency}},
url = {http://www.jstor.org/stable/10.2307/3866},
volume = {44},
year = {1975}
}
@article{Freedman1990,
abstract = {A predator-prey population is described in which the prey population may be either a secondary host or a primary host to a parasite, but the predator is always a primary host. Those prey that have been invaded by the parasite have their behavior modified so as to make them more susceptible to predation. The model is described by a system of three autonomous ordinary differential equations. Conditions for persistence of all populations are given in the case that both populations are primary hosts. A brief discussion of the stability of the interior equilibrium is given. {\textcopyright} 1990.},
author = {Freedman, H. I.},
doi = {10.1016/0025-5564(90)90001-F},
issn = {00255564},
journal = {Mathematical Biosciences},
number = {2},
pages = {143--155},
pmid = {2134516},
title = {{A model of predator-prey dynamics as modified by the action of a parasite}},
url = {http://www.sciencedirect.com/science/article/pii/002555649090001F},
volume = {99},
year = {1990}
}
@article{Wells2013,
abstract = {Patterns of host-parasite association are poorly understood in tropical forests. While we typically observe only snapshots of the diverse assemblages and interactions under variable conditions, there is a desire to make inferences about prevalence and host-specificity patterns. We studied the interaction of ticks with non-volant small mammals in forests of Borneo. We inferred the probability of species interactions from individual-level data in a multi-level Bayesian model that incorporated environmental covariates and advanced estimates for rarely observed species through model averaging. We estimated the likelihood of observing particular interaction frequencies under field conditions and a scenario of exhaustive sampling and examined the consequences for inferring host specificity. We recorded a total of 13 different tick species belonging to the five genera Amblyomma, Dermacentor, Haemaphysalis, Ixodes, and Rhipicephalus from a total of 37 different host species (Rodentia, Scandentia, Carnivora, Soricidae) on 237 out of 1,444 host individuals. Infestation probabilities revealed most variation across host species but less variation across tick species with three common rat and two tree shrew species being most heavily infested. Host species identity explained ca. 75 {\%} of the variation in infestation probability and another 8-10 {\%} was explained by local host abundance. Host traits and site-specific attributes had little explanatory power. Host specificity was estimated to be similarly low for all tick species, which were all likely to infest 34-37 host species if exhaustively sampled. By taking into consideration the hierarchical organization of individual interactions that may take place under variable conditions and that shape host-parasite networks, we can discern uncertainty and sampling bias from true interaction frequencies, whereas network attributes derived from observed values may lead to highly misleading results. Multi-level approaches may help to move this field towards inferential approaches for understanding mechanisms that shape the strength and dynamics in ecological networks.},
author = {Wells, Konstans and O'Hara, Robert B. and Pfeiffer, Martin and Lakim, Maklarin B. and Petney, Trevor N. and Durden, Lance a.},
doi = {10.1007/s00442-012-2511-9},
isbn = {1432-1939 (Electronic)$\backslash$r0029-8549 (Linking)},
issn = {00298549},
journal = {Oecologia},
keywords = {Acari,Biotic interaction,Hierarchical model,Host specificity,Multispecies model},
month = {jun},
number = {2},
pages = {307--316},
pmid = {23108423},
title = {{Inferring host specificity and network formation through agent-based models: Tick-mammal interactions in Borneo}},
url = {http://www.ncbi.nlm.nih.gov/pubmed/23108423},
volume = {172},
year = {2013}
}
@article{Knowles2013,
abstract = {Simultaneous infection by multiple parasite species is ubiquitous in nature. Interactions among co-infecting parasites may have important consequences for disease severity, transmission and community-level responses to pertur- bations. However, our current view of parasite interactions in nature comes primarily from observational studies, which may be unreliable at detecting interactions. We performed a perturbation experiment in wild mice, by using an anthelminthic to suppress nematodes, and monitored the con- sequences for other parasite species. Overall, these parasite communities were remarkably stable to perturbation. Only one non-target parasite species responded to deworming, and this response was temporary: we found strong, but short-lived, increases in the abundance of Eimeria protozoa, which share an infection site with the dominant nematode species, suggesting local, dynamic competition. These results, providing a rare and clear exper- imental demonstration of interactions between helminths and co-infecting parasites in wild vertebrates, constitute an important step towards under- standing the wider consequences of similar drug treatments in humans and animals.},
author = {Knowles, S. C. L. and Fenton, a. and Petchey, O. L. and Jones, T. R. and Barber, R. and Pedersen, a. B.},
doi = {10.1098/rspb.2013.0598},
isbn = {0962-8452},
issn = {1471-2954},
journal = {Proceedings of the Royal Society of London B: Biological Sciences},
keywords = {Heligmosomoides polygyrus,co-infection,community ecology,helminth,interaction},
number = {1762},
pages = {20130598},
pmid = {23677343},
title = {{Stability of within-host − parasite communities in a wild mammal system}},
url = {http://dx.doi.org/10.1098/rspb.2013.0598},
volume = {280},
year = {2013}
}
@article{Giannini2013,
abstract = {Biotic interactions have been considered as an important factor to be included in species distribution modelling, but little is known about how different types of interaction or different strategies for modelling affect model performance. This study compares different methods for including interspecific interactions in distribution models for bees, their brood parasites, and the plants they pollinate. Host–parasite interactions among bumble bees (genus Bombus: generalist pollinators and brood parasites) and specialist plant–pollinator interactions between Centris bees and Krameria flowers were used as case studies. We used 7 different modelling algorithms available in the BIOMOD R package. For Bombus, the inclusion of interacting species distributions generally increased model predictive accuracy. The improvement was better when the interacting species was included with its raw distribution rather than with its modeled suitability. However, incorporating the distributions of non-interacting species sometimes resulted in similarly increased model accuracy despite their being no significance of any interaction for the distribution. For the Centris-Krameria system the best strategy for modelling biotic interactions was to include the interacting species model-predicted values. However, the results were less consistent than those for Bombus species, and most models including biotic interactions showed no significant improvement over abiotic models. Our results are consistent with previous studies showing that biotic interactions can be important in structuring species distributions at regional scales. However, correlations between species distributions are not necessarily indicative of interactions. Therefore, choosing the correct biotic information, based on biological and ecological knowledge, is critical to improve the accuracy of species distribution models and forecast distribution change.},
author = {Giannini, Tereza Cristina and Chapman, Daniel S. and Saraiva, Antonio Mauro and Alves-dos-Santos, Isabel and Biesmeijer, Jacobus C.},
doi = {10.1111/j.1600-0587.2012.07191.x},
isbn = {1600-0587},
issn = {09067590},
journal = {Ecography},
month = {jun},
number = {6},
pages = {649--656},
title = {{Improving species distribution models using biotic interactions: A case study of parasites, pollinators and plants}},
url = {http://doi.wiley.com/10.1111/j.1600-0587.2012.07191.x},
volume = {36},
year = {2013}
}
@article{Valery2013,
abstract = {Th e restriction of invasion biology to non-native species has been laid down as one founding principle of the discipline by many researchers. However, this split between native and non-native species is highly controversial. Using a phenomeno- logical approach and a more pragmatic examination of biological invasions, the present paper discusses how this dichotomy has restricted the relevance of the fi eld, both from theoretical and practical viewpoints. We advocate the emergence of a broader disciplinary fi eld. In},
author = {Val{\'{e}}ry, Lo{\"{i}}c and Fritz, Herv{\'{e}} and Lefeuvre, Jean Claude},
doi = {10.1111/j.1600-0706.2013.00445.x},
isbn = {1600-0706},
issn = {00301299},
journal = {Oikos},
month = {aug},
number = {8},
pages = {1143--1146},
title = {{Another call for the end of invasion biology}},
url = {http://doi.wiley.com/10.1111/j.1600-0706.2013.00445.x},
volume = {122},
year = {2013}
}
@article{Orlando2013,
abstract = {The rich fossil record of equids has made them a model for evolutionary processes. Here we present a 1.12-times coverage draft genome from a horse bone recovered from permafrost dated to approximately 560-780 thousand years before present (kyr BP). Our data represent the oldest full genome sequence determined so far by almost an order of magnitude. For comparison, we sequenced the genome of a Late Pleistocene horse (43 kyr BP), and modern genomes of five domestic horse breeds (Equus ferus caballus), a Przewalski's horse (E. f. przewalskii) and a donkey (E. asinus). Our analyses suggest that the Equus lineage giving rise to all contemporary horses, zebras and donkeys originated 4.0-4.5 million years before present (Myr BP), twice the conventionally accepted time to the most recent common ancestor of the genus Equus. We also find that horse population size fluctuated multiple times over the past 2 Myr, particularly during periods of severe climatic changes. We estimate that the Przewalski's and domestic horse populations diverged 38-72 kyr BP, and find no evidence of recent admixture between the domestic horse breeds and the Przewalski's horse investigated. This supports the contention that Przewalski's horses represent the last surviving wild horse population. We find similar levels of genetic variation among Przewalski's and domestic populations, indicating that the former are genetically viable and worthy of conservation efforts. We also find evidence for continuous selection on the immune system and olfaction throughout horse evolution. Finally, we identify 29 genomic regions among horse breeds that deviate from neutrality and show low levels of genetic variation compared to the Przewalski's horse. Such regions could correspond to loci selected early during domestication.},
author = {Orlando, Ludovic and Ginolhac, Aur{\'{e}}lien and Zhang, Guojie and Froese, Duane and Albrechtsen, Anders and Stiller, Mathias and Schubert, Mikkel and Cappellini, Enrico and Petersen, Bent and Moltke, Ida and Johnson, Philip L F and Fumagalli, Matteo and Vilstrup, Julia T and Raghavan, Maanasa and Korneliussen, Thorfinn and Malaspinas, Anna-Sapfo and Vogt, Josef and Szklarczyk, Damian and Kelstrup, Christian D and Vinther, Jakob and Dolocan, Andrei and Stenderup, Jesper and Velazquez, Amhed M V and Cahill, James and Rasmussen, Morten and Wang, Xiaoli and Min, Jiumeng and Zazula, Grant D and Seguin-Orlando, Andaine and Mortensen, Cecilie and Magnussen, Kim and Thompson, John F and Weinstock, Jacobo and Gregersen, Kristian and R{\o}ed, Knut H and Eisenmann, V{\'{e}}ra and Rubin, Carl J and Miller, Donald C and Antczak, Douglas F and Bertelsen, Mads F and Brunak, S{\o}ren and Al-Rasheid, Khaled a S and Ryder, Oliver and Andersson, Leif and Mundy, John and Krogh, Anders and Gilbert, M Thomas P and Kj{\ae}r, Kurt and Sicheritz-Ponten, Thomas and Jensen, Lars Juhl and Olsen, Jesper V and Hofreiter, Michael and Nielsen, Rasmus and Shapiro, Beth and Wang, Jun and Willerslev, Eske},
doi = {10.1038/nature12323},
isbn = {1476-4687 (Electronic)$\backslash$n0028-0836 (Linking)},
issn = {1476-4687},
journal = {Nature},
keywords = {Animals,Conservation of Natural Resources,DNA,DNA: analysis,DNA: genetics,Endangered Species,Equidae,Equidae: classification,Equidae: genetics,Evolution, Molecular,Fossils,Genetic Variation,Genetic Variation: genetics,Genome,Genome: genetics,History, Ancient,Horses,Horses: classification,Horses: genetics,Phylogeny,Proteins,Proteins: analysis,Proteins: chemistry,Proteins: genetics,Yukon Territory},
month = {jul},
number = {7456},
pages = {74--8},
pmid = {23803765},
title = {{Recalibrating Equus evolution using the genome sequence of an early Middle Pleistocene horse.}},
url = {http://www.ncbi.nlm.nih.gov/pubmed/23803765},
volume = {499},
year = {2013}
}
@article{Polishchuk2013,
abstract = {Research on the role of top–down (predation) and bottom–up (food) effects in food webs has led to the understanding that the variability of these effects in space and time is a fundamental feature of natural systems. Consequently, our measurement tools must allow us to evaluate the effects from a dynamical perspective. A population-dynamics approach may be appropriate to the task. More specifically, because food and predators both affect birth rate, birth rate dynamics may be a key to understanding their impact on the population of interest. Based on the Edmondson–Paloheimo model for birth rate, we propose a new population metric to assess the relative strength of top–down vs bottom–up effects. The metric is the ratio of contributions of changes in proportion of adults and fecundity to change in birth rate. Proportion of adults reflects a top–down effect (predators are assumed to be size-selective), fecundity reflects a bottom–up effect, and birth rate appears as a common currency with which to compare the former and the latter. Using microcosm experiments and computer simulations on the cladoceran Daphnia, we calibrate the metric and show that, in both types of tests, the ratio of contributions is typically 0.5–0.7 under a strong bottom–up effect and 2.0–2.2 under a strong top–down effect. This provides experimental evidence that the ratio of contributions may allow one to distinguish a strong top–down effect from a strong bottom–up effect.},
author = {Polishchuk, Leonard V. and Vijverberg, Jacobus and Voronov, Dmitry A. and Mooij, Wolf M.},
doi = {10.1111/j.1600-0706.2012.00046.x},
isbn = {00301299},
issn = {00301299},
journal = {Oikos},
month = {aug},
number = {8},
pages = {1177--1186},
title = {{How to measure top-down vs bottom-up effects: A new population metric and its calibration on Daphnia}},
url = {http://doi.wiley.com/10.1111/j.1600-0706.2012.00046.x},
volume = {122},
year = {2013}
}
@article{Thompson2013,
abstract = {Understanding how diversity interacts with energy supply is of broad ecological interest. Most studies to date have investigated patterns within trophic levels, reflecting a lack of food webs which include information on energy flow. We added parasites to a published marine energy-flow food web, to explore whether parasite diversity is correlated with energy flow to host taxa. Parasite diversity was high with 36 parasite taxa affecting 40 of the 51 animal taxa. Adding parasites increased the number of trophic links per species, trophic link strength, connectance, and food chain lengths. There was evidence of an asymptotic relationship between energy flowing through a food chain and parasite diversity, although there were clear outliers. High parasite diversity was associated with host taxa which were highly connected within the food web. This suggests that energy flow through a taxon may favour parasite diversity, up to a maximal value. The evolutionary and energetic basis for that limitation is of key interest in understanding the basis for parasite diversity in natural food webs and thus their role in food web dynamics},
author = {Thompson, Ross M. and Poulin, Robert and Mouritsen, Kim N. and Thieltges, David W.},
doi = {10.1111/j.1600-0706.2012.00245.x},
issn = {00301299},
journal = {Oikos},
month = {aug},
number = {8},
pages = {1187--1194},
title = {{Resource tracking in marine parasites: going with the flow?}},
url = {http://doi.wiley.com/10.1111/j.1600-0706.2012.00245.x},
volume = {122},
year = {2013}
}
@article{Steiner2013,
abstract = {A general prediction from simple metapopulation models is that spatially synchronized forcing can spatially synchronize population dynamics and destabilize metapopulations. In contrast, spatially asynchronous forcing is predicted to decrease population synchrony and promote temporal stability and population persistence, especially in the presence of dispersal. Only recently have studies begun to experimentally address these predictions. Moreover, few studies have experimentally examined how such processes operate in the context of competition communities. Stabilizing processes may continue to operate when placed within a metacommunity context with multiple competing consumers but only at low to intermediate levels of dispersal. High dispersal rates can reverse these predictions and lead to destabilization. We tested this under controlled conditions using an experimental aquatic system composed of three competing species of zooplankton. Metacommunities experienced different levels of dispersal and environmental forcing in the form of spatially synchronous or asynchronous pH perturbations. We found support that dispersal can have contrasting effects on population stability depending on the degree to which population dynamics were synchronized in space. Dispersal under synchronous forcing or no forcing had either neutral of positive effects on spatial population synchrony of all three zooplankton species. In these treatments, dispersal reduced population stability at the local and metapopulation levels for two of three species. In contrast, asynchronously varying environments reduced population synchrony relative to unforced systems, regardless of dispersal level. In these treatments, dispersal enhanced temporal stability and persistence of populations not by reducing population synchrony but by enhancing population minima and spatial averaging of abundances. High dispersal rates under asynchronous forcing reduced the abundance of one species, consistent with increasing regional competition and general metacommunity theory. However, no effects on its stability or persistence were observed. Our work highlights the context-dependent effects of dispersal on population dynamics in varying environments.},
author = {Steiner, Christopher F. and Stockwell, Richard D. and Kalaimani, Vidhya and Aqel, Zakaria},
doi = {10.1111/j.1600-0706.2012.20936.x},
isbn = {1600-0706},
issn = {00301299},
journal = {Oikos},
month = {aug},
number = {8},
pages = {1195--1206},
title = {{Population synchrony and stability in environmentally forced metacommunities}},
url = {http://doi.wiley.com/10.1111/j.1600-0706.2012.20936.x},
volume = {122},
year = {2013}
}
@article{Garay-Narvaez2013,
abstract = {Human activities have led to massive influxes of pollutants, degrading the habitat of species and simplifying their biodiversity. However, the interaction between food web complexity, pollution and stability is still poorly understood. In this study we evaluate the effect exerted by accumulable pollutants on the relationship between complexity and stability of food webs. We built model food webs with different levels of richness and connectance, and used a bioenergetic model to project the dynamics of species biomasses. Further, we developed appropriate expressions for the dynamics of bioaccumulated and environmental pollutants. We additionally analyzed attributes of organisms' and communities as determinants of species persistence (stability). We found that the positive effect of complexity on stability was enhanced as pollutant stress increased. Additionally we showed that the number of basal species and the maximum trophic level shape the complexity–stability relationship in polluted systems, and that in-degree of consumers determines species extinction in polluted environments. Our study indicates that the form of biodiversity and the complexity of interaction networks are essential to understand and project the effects of pollution and other ecosystem threats.},
author = {Garay-Narvaez, Leslie and Arim, Matias and Flores, Jos{\'{e}} D. and Ramos-Jiliberto, Rodrigo},
doi = {10.1111/j.1600-0706.2012.00218.x},
isbn = {00301299},
issn = {00301299},
journal = {Oikos},
month = {aug},
number = {8},
pages = {1247--1253},
title = {{The more polluted the environment, the more important biodiversity is for food web stability}},
url = {http://doi.wiley.com/10.1111/j.1600-0706.2012.00218.x},
volume = {122},
year = {2013}
}
@article{Fernandez-Gonzalez2013,
abstract = {The diversity of symbionts (commensals, mutualists or parasites) that share the same host species may depend on opportunities and constraints on host exploitation associated with host phenotype or environment. Various host traits may differently influence host accessibility and within-host population growth of each symbiont species, or they may determine the outcome of within-host interactions among coexisting species. In turn, phenotypic diversity of a host species may promote divergent exploitation strategies among its symbiotic organisms. We studied the distribution of two feather mite species, Proctophyllodes sylviae and Trouessartia bifurcata, among blackcaps Sylvia atricapilla wintering in southern Spain during six winters. The host population included migratory and sedentary individuals, which were unequally distributed between two habitat types (forests and shrublands). Visual mite counts showed that both mite species often coexisted on sedentary blackcaps, but were seldom found together on migratory blackcaps. Regardless of host habitat, Proctophyllodes were highly abundant and Trouessartia were scarce on migratory blackcaps, but the abundance of both mite species converged in intermediate levels on sedentary blackcaps. Coexistence may come at a cost for Proctophyllodes, whose load decreased when Trouessartia was present on the host (the opposite was not true). Proctophyllodes load was positively correlated with host wing length (wings were longer in migratory blackcaps), while Trouessartia load was positively correlated to uropygial gland size (sedentary blackcaps had bigger glands), which might render migratory and sedentary blackcaps better hosts for Proctophyllodes and Trouessartia, respectively. Our results draw a complex scenario for mite co-existence in the same host species, where different mite species apparently take advantage of, or are constrained by, divergent host phenotypic traits. This expands our understanding of bird-mite interactions, which are usually viewed as less dynamic in relation to variation in host phenotype, and emphasizes the role of host phenotypic divergence in the diversification of symbiotic organisms. [ABSTRACT FROM AUTHOR]},
author = {Fernandez-Gonzalez, Sofia and {De la Hera}, Iv{\'{a}}n and P{\'{e}}rez-Rodr{\'{i}}guez, Ant{\'{o}}n and P{\'{e}}rez-Tris, Javier},
doi = {10.1111/j.1600-0706.2012.00241.x},
isbn = {1600-0706},
issn = {00301299},
journal = {Oikos},
month = {aug},
number = {8},
pages = {1227--1237},
title = {{Divergent host phenotypes create opportunities and constraints on the distribution of two wing-dwelling feather mites}},
url = {http://doi.wiley.com/10.1111/j.1600-0706.2012.00241.x},
volume = {122},
year = {2013}
}
@article{Streicker2013,
abstract = {Controlling parasites that infect multiple host species often requires targeting single species that dominate transmission. Yet, it is rarely recognised that such 'key hosts' can arise through disparate mechanisms, potentially requiring different approaches for control. We identify three distinct, but not mutually exclusive, processes that underlie host species heterogeneity: infection prevalence, population abundance and infectiousness. We construct a theoretical framework to isolate the role of each process from ecological data and to explore the outcome of different control approaches. Applying this framework to data on 11 gastrointestinal parasites in small mammal communities across the eastern United States reveals variation not only in the magnitude of transmission asymmetries among host species but also in the processes driving heterogeneity. These differences influence the efficiency by which different control strategies reduce transmission. Identifying and tailoring interventions to a specific type of key host may therefore enable more effective management of multihost parasites.},
author = {Streicker, Daniel G. and Fenton, Andy and Pedersen, Amy B.},
doi = {10.1111/ele.12477},
isbn = {1461-0248 (Electronic)$\backslash$n1461-023X (Linking)},
issn = {14610248},
journal = {Ecology Letters},
keywords = {Coccidia,Community epidemiology,Helminth,Management,Parasitism,Species heterogeneity,Super-shedder,Susceptibility},
month = {aug},
number = {10},
pages = {1134--1137},
pmid = {23714379},
title = {{Corrigendum to Differential sources of host species heterogeneity influence the transmission and control of multihost parasites [Ecol. Lett., 16, (2013), 975-984]}},
url = {http://www.ncbi.nlm.nih.gov/pubmed/23714379},
volume = {18},
year = {2015}
}
@article{Legendre2013a,
abstract = {Beta diversity can be measured in different ways. Among these, the total variance of the community data table Y can be used as an estimate of beta diversity. We show how the total variance of Y can be calculated either directly or through a dissimilarity matrix obtained using any dissimilarity index deemed appropriate for pairwise comparisons of community composition data. We addressed the question of which index to use by coding 16 indices using 14 properties that are necessary for beta assessment, comparability among data sets, sampling issues and ordination. Our comparison analysis classified the coefficients under study into five types, three of which are appropriate for beta diversity assessment. Our approach links the concept of beta diversity with the analysis of community data by commonly used methods like ordination and anova. Total beta can be partitioned into Species Contributions (SCBD: degree of variation of individual species across the study area) and Local Contributions (LCBD: comparative indicators of the ecological uniqueness of the sites) to Beta Diversity. Moreover, total beta can be broken up into within- and among-group components by manova, into orthogonal axes by ordination, into spatial scales by eigenfunction analysis or among explanatory data sets by variation partitioning.},
author = {Legendre, Pierre and {De C{\'{a}}ceres}, Miquel},
doi = {10.1111/ele.12141},
isbn = {1461-0248},
issn = {1461023X},
journal = {Ecology Letters},
keywords = {Beta diversity,Community composition data,Community ecology,Dissimilarity coefficients,Local contributions to beta diversity,Properties of dissimilarity coefficients,Species contributions to beta diversity,Variance partitioning},
month = {aug},
number = {8},
pages = {951--963},
pmid = {23809147},
title = {{Beta diversity as the variance of community data: Dissimilarity coefficients and partitioning}},
url = {http://www.ncbi.nlm.nih.gov/pubmed/23809147},
volume = {16},
year = {2013}
}
@article{Sedio2013,
abstract = {The Janzen-Connell hypothesis proposes that plant interactions with host-specific antagonists can impair the fitness of locally abundant species and thereby facilitate coexistence. However, insects and pathogens that associate with multiple hosts may mediate exclusion rather than coexistence. We employ a simulation model to examine the effect of enemy host breadth on plant species richness and defence community structure, and to assess expected diversity maintenance in example systems. Only models in which plant enemy similarity declines rapidly with defence similarity support greater species richness than models of neutral drift. In contrast, a wide range of enemy host breadths result in spatial dispersion of defence traits, at both landscape and local scales, indicating that enemy-mediated competition may increase defence-trait diversity without enhancing species richness. Nevertheless, insect and pathogen host associations in Panama and Papua New Guinea demonstrate a potential to enhance plant species richness and defence-trait diversity comparable to strictly specialised enemies.},
author = {Sedio, Brian E and Ostling, Annette M},
doi = {10.1111/ele.12130},
isbn = {1461-0248 (Electronic)$\backslash$r1461-023X (Linking)},
issn = {1461-0248},
journal = {Ecology Letters},
keywords = {2013,coexistence,community structure,ecology letters,enemy,frequency dependence,host specialisation,interactions,janzen-connell,plant,species richness,tropical forest},
month = {aug},
number = {8},
pages = {995--1003},
pmid = {23773378},
title = {{How specialised must natural enemies be to facilitate coexistence among plants?}},
url = {http://www.ncbi.nlm.nih.gov/pubmed/23773378},
volume = {16},
year = {2013}
}
@article{Calinger2013,
abstract = {Shifting flowering phenology with rising temperatures is occurring worldwide, but the rarity of co-occurring long-term observational and temperature records has hindered the evaluation of phenological responsiveness in many species and across large spatial scales. We used herbarium specimens combined with historic temperature data to examine the impact of climate change on flowering trends in 141 species collected across 116,000 km2 in north-central North America. On average, date of maximum flowering advanced 2.4 days °C−1, although species-specific responses varied from − 13.5 to + 7.3 days °C−1. Plant functional types exhibited distinct patterns of phenological responsiveness with significant differences between native and introduced species, among flowering seasons, and between wind- and biotically pollinated species. This study is the first to assess large-scale patterns of phenological responsiveness with broad species representation and is an important step towards understanding current and future impacts of climate change on species performance and biodiversity.},
author = {Calinger, Kellen M. and Queenborough, Simon and Curtis, Peter S.},
doi = {10.1111/ele.12135},
isbn = {1461-023X},
issn = {14610248},
journal = {Ecology Letters},
keywords = {Climate change,Invasive species,Life history,Phenological responsiveness,Phenology,Pollination syndrome},
month = {aug},
number = {8},
pages = {1037--1044},
pmid = {23786499},
title = {{Herbarium specimens reveal the footprint of climate change on flowering trends across north-central North America}},
url = {http://www.ncbi.nlm.nih.gov/pubmed/23786499},
volume = {16},
year = {2013}
}
@article{Benkman2013,
abstract = {Although the ecological and evolutionary impacts of species interactions have been the foci of much research, the relationship between the strength of species interactions and the intensity of selection has been investigated only rarely. I develop a simple model demonstrating how the opportunity for selection varies with interaction strength, and then use the relationship between the maximum value of the selection differential and the opportunity for selection (Arnold {\&} Wade 1984) to evaluate how selection differentials vary in relation to species interaction strength. This model predicts an initial deceleration and then an accelerating increase in the intensity of selection with increasing strength of antagonistic interactions and with decreasing strength of mutualistic interactions. Empirical data from several studies provide support for this model. These results further support an evolutionary mechanism for some striking patterns of evolutionary diversification including the latitudinal species gradient, and should be relevant to studies of eco-evolutionary dynamics.},
author = {Benkman, Craig W},
doi = {10.1111/ele.12138},
isbn = {1461-023X},
issn = {1461-0248},
journal = {Ecology Letters},
keywords = {Animals,Biodiversity,Biological,Biological Evolution,Biota,Food Chain,Genetic,Insects,Insects: physiology,Models,Plant Physiological Phenomena,Pollination,Seed Dispersal,Seeds,Seeds: physiology,Selection,Symbiosis},
month = {aug},
number = {8},
pages = {1054--60},
pmid = {23763752},
title = {{Biotic interaction strength and the intensity of selection.}},
url = {http://www.ncbi.nlm.nih.gov/pubmed/23763752},
volume = {16},
year = {2013}
}
@article{AlexPerkins2013,
abstract = {Populations on the edge of an expanding range are subject to unique evolutionary pressures acting on their life-history and dispersal traits. Empirical evidence and theory suggest that traits there can evolve rapidly enough to interact with ecological dynamics, potentially giving rise to accelerating spread. Nevertheless, which of several evolutionary mechanisms drive this interaction between evolution and spread remains an open question. We propose an integrated theoretical framework for partitioning the contributions of different evolutionary mechanisms to accelerating spread, and we apply this model to invasive cane toads in northern Australia. In doing so, we identify a previously unrecognised evolutionary process that involves an interaction between life-history and dispersal evolution during range shift. In roughly equal parts, life-history evolution, dispersal evolution and their interaction led to a doubling of distance spread by cane toads in our model, highlighting the potential importance of multiple evolutionary processes in the dynamics of range expansion.},
author = {{Alex Perkins}, T. and Phillips, Benjamin L. and Baskett, Marissa L. and Hastings, Alan},
doi = {10.1111/ele.12136},
isbn = {1461-0248},
issn = {14610248},
journal = {Ecology Letters},
keywords = {Climate shift,Integral projection model,Integrodifference equation,Invasion front,Invasion lag phase,Natural selection,Quantitative genetics,Rhinella marina,Spatial selection,Spatial spread},
month = {aug},
number = {8},
pages = {1079--1087},
pmid = {23809102},
title = {{Evolution of dispersal and life history interact to drive accelerating spread of an invasive species}},
url = {http://www.ncbi.nlm.nih.gov/pubmed/23809102},
volume = {16},
year = {2013}
}
@article{Pearse2013,
abstract = {Humans are altering the global distributional ranges of plants, while their co-evolved herbivores are frequently left behind. Native herbivores often colonise non-native plants, potentially reducing invasion success or causing economic loss to introduced agricultural crops. We developed a predictive model to forecast novel interactions and verified it with a data set containing hundreds of observed novel plant-insect interactions. Using a food network of 900 native European butterfly and moth species and 1944 native plants, we built an herbivore host-use model. By extrapolating host use from the native herbivore-plant food network, we accurately forecasted the observed novel use of 459 non-native plant species by native herbivores. Patterns that governed herbivore host breadth on co-evolved native plants were equally important in determining non-native hosts. Our results make the forecasting of novel herbivore communities feasible in order to better understand the fate and impact of introduced plants.},
author = {Pearse, Ian S. and Altermatt, Florian},
doi = {10.1111/ele.12143},
isbn = {1461-0248},
issn = {14610248},
journal = {Ecology Letters},
keywords = {Herbivory,Host breadth,Invasive species,Novel interaction,Phylogenetic constraint},
month = {aug},
number = {8},
pages = {1088--1094},
pmid = {23800217},
title = {{Predicting novel trophic interactions in a non-native world}},
url = {http://www.ncbi.nlm.nih.gov/pubmed/23800217},
volume = {16},
year = {2013}
}
@article{Quintero2013,
abstract = {A key question in predicting responses to anthropogenic climate change is: how quickly can species adapt to different climatic conditions? Here, we take a phylogenetic approach to this question. We use 17 time-calibrated phylogenies representing the major tetrapod clades (amphibians, birds, crocodilians, mammals, squamates, turtles) and climatic data from distributions of {\textgreater}500 extant species. We estimate rates of change based on differences in climatic variables between sister species and estimated times of their splitting. We compare these rates to predicted rates of climate change from 2000 to 2100. Our results are striking: matching projected changes for 2100 would require rates of niche evolution that are {\textgreater}10000 times faster than rates typically observed among species, for most variables and clades. Despite many caveats, our results suggest that adaptation to projected changes in the next 100years would require rates that are largely unprecedented based on observed rates among vertebrate species.},
author = {Quintero, Ignacio and Wiens, John J.},
doi = {10.1111/ele.12144},
isbn = {1461-023x},
issn = {14610248},
journal = {Ecology Letters},
keywords = {Adaptation,Climate change,Extinction,Niche evolution,Vertebrates},
month = {aug},
number = {8},
pages = {1095--1103},
pmid = {23800223},
title = {{Rates of projected climate change dramatically exceed past rates of climatic niche evolution among vertebrate species}},
url = {http://www.ncbi.nlm.nih.gov/pubmed/23800223},
volume = {16},
year = {2013}
}
@article{Slatyer2013,
abstract = {The range of resources that a species uses (i.e. its niche breadth) might determine the geographical area it can occupy, but consensus on whether a niche breadth–range size relationship generally exists among species has been slow to emerge. The validity of this hypothesis is a key question in ecology in that it proposes a mechanism for commonness and rarity, and if true, may help predict species' vulnerability to extinction. We identified 64 studies that measured niche breadth and range size, and we used a meta-analytic approach to test for the presence of a niche breadth–range size relationship. We found a significant positive relationship between range size and environmental tolerance breadth (z = 0.49), habitat breadth (z = 0.45), and diet breadth (z = 0.28). The overall positive effect persisted even when incorporating sampling effects. Despite significant variability in the strength of the relationship among studies, the general positive relationship suggests that specialist species might be disproportionately vulnerable to habitat loss and climate change due to synergistic effects of a narrow niche and small range size. An understanding of the ecological and evolutionary mechanisms that drive and cause deviations from this niche breadth–range size pattern is an important future research goal.},
author = {Slatyer, Rachel A. and Hirst, Megan and Sexton, Jason P.},
doi = {10.1111/ele.12140},
isbn = {1461-0248},
issn = {1461023X},
journal = {Ecology Letters},
keywords = {Extinction risk,Geographical range,Meta-analysis,Niche breadth,Rarity,Specialisation},
month = {aug},
number = {8},
pages = {1104--1114},
pmid = {23773417},
title = {{Niche breadth predicts geographical range size: A general ecological pattern}},
url = {http://www.ncbi.nlm.nih.gov/pubmed/23773417},
volume = {16},
year = {2013}
}
@article{Laws2013,
abstract = {Because species interactions are often context-dependent, abiotic factors such as temperature and biotic factors such as food quality may alter species interactions with potential consequences to ecosystem structure and function. For example, altered predator–prey interactions may influence the dynamics of trophic cascades, affecting net primary production. In a three-year field experiment, we manipulated a plant–grasshopper–spider food chain in mesic tallgrass prairie to investigate the effects of temperature and food quality on grasshopper performance, and to understand the direct and indirect tritrophic interactions that contribute to trophic cascades. Because spiders are active at cooler temperatures than grasshoppers in our system, we hypothesized that predator effects would be strongest in cooled treatments, and weakest in warmed treatments. Grasshopper spider interactions were highly context-dependent and varied significantly with food quality, temperature treatment and year. Spiders most often reduced grasshopper survival in the cooled and ambient temperature treatments, but had little to no effect on grasshopper survival in the warmed treatments, as hypothesized. In some years, plants compensated for grasshopper herbivory and trophic cascades were not observed despite significant effects of predators on grasshopper survival. However, in the year they were observed, trophic cascades only occurred in cooled treatments where predator effects on grasshoppers were strongest. Predicting ecosystem responses to climate change will require an understanding of how temperature influences species interactions. Our results demonstrate that changes in daily temperature regimes can alter predator–prey interactions among arthropods with consequences for ecosystem processes such as primary production and the relative importance of top–down and bottom–up processes.},
author = {Laws, Angela Nardoni and Joern, Anthony},
doi = {10.1111/j.1600-0706.2012.20419.x},
isbn = {00301299},
issn = {00301299},
journal = {Oikos},
month = {jul},
number = {7},
pages = {977--986},
title = {{Predator-prey interactions in a grassland food chain vary with temperature and food quality}},
url = {http://doi.wiley.com/10.1111/j.1600-0706.2012.20419.x},
volume = {122},
year = {2013}
}
@article{Wang2013,
abstract = {Many animals scatter-hoard seeds to ensure an even supply of food throughout the year and this behavior requires similar foraging decisions. Seed-traits have been shown to affect the final foraging decision but little is known about the decision process itself. Here, we first defined four sequential steps comprising the decision process of scatter-hoarding rodents: 1) upon encountering a seed, should it be ignored or manipulated; 2) if manipulated, should it be eaten in situ or removed elsewhere; 3) upon removal, how far away should it be carried; and finally 4) whether to eat or cache the removed seed. Using experimental seeds with controlled differences in size, tannin and nutrient content, we evaluated how different traits influence each step in this decision process. We found that different traits had distinct effects on each step. Seed size affected all four steps, while nutrient and tannin content primarily affected the first and third steps. By dissecting foraging behavior in relation to experimentally controlled seed-traits, we have created an effective framework within which to understand the unique relationship between scatter-hoarding rodents that both predate and disperse plant seeds.},
author = {Wang, Bo and Ye, Cheng Xi and Cannon, Charles H. and Chen, Jin},
doi = {10.1111/j.1600-0706.2012.20823.x},
isbn = {0030-1299},
issn = {00301299},
journal = {Oikos},
month = {jul},
number = {7},
pages = {1027--1034},
title = {{Dissecting the decision making process of scatter-hoarding rodents}},
url = {http://doi.wiley.com/10.1111/j.1600-0706.2012.20823.x},
volume = {122},
year = {2013}
}
@article{Mckinnon2013,
abstract = {Apparent competition between prey is hypothesized to occur more frequently in environments with low densities of preferred prey, where predators are forced to forage for multiple prey items. In the arctic tundra, numerical and functional responses of predators to preferred prey (lemmings) affect the predation pressure on alternative prey (goose eggs) and predators aggregate in areas of high alternative prey density. Therefore, we hypothesized that predation risk on incidental prey (shorebird eggs) would increase in patches of high goose nest density when lemmings were scarce. To test this hypothesis, we measured predation risk on artificial shorebird nests in quadrats varying in goose nest density on Bylot Island (Nunavut, Canada) across three summers with variable lemming abundance. Predation risk on artificial shorebird nests was positively related to goose nest density, and this relationship was strongest at low lemming abundance when predation risk increased by 600{\%} as goose nest density increased from 0 to 12 nests ha−1. Camera monitoring showed that activity of arctic foxes, the most important predator, increased with goose nest density. Our data support our incidental prey hypothesis; when preferred prey decrease in abundance, predator mediated apparent competition via aggregative response occurs between the alternative and incidental prey items.},
author = {McKinnon, Laura and Berteaux, Dominique and Gauthier, Gilles and B{\^{e}}ty, Jo{\"{e}}l},
doi = {10.1111/j.1600-0706.2012.20708.x},
isbn = {00301299},
issn = {00301299},
journal = {Oikos},
month = {jul},
number = {7},
pages = {1042--1048},
title = {{Predator-mediated interactions between preferred, alternative and incidental prey in the arctic tundra}},
url = {http://doi.wiley.com/10.1111/j.1600-0706.2012.20708.x},
volume = {122},
year = {2013}
}
@article{Evans2013,
abstract = {There have been considerable advances in our understanding of the tolerance of species interaction networks to sequential extinctions of plants and animals. However, communities of species exist in a mosaic of habitats, and the vulnerability of habitats to anthropogenic change varies. Here, we model the cascading effects of habitat loss, driven by plant extinctions, on the robustness of multiple animal groups. Our network is constructed from empirical observations of 11 animal groups in 12 habitats on farmland. We simulated sequential habitat removal scenarios: randomly; according to prior information; and with a genetic algorithm to identify best- and worst-case permutations of habitat loss. We identified two semi-natural habitats (waste ground and hedgerows together comprising {\textless} 5{\%} of the total area of the farm) as disproportionately important to the integrity of the overall network. Our approach provides a new tool for network ecologists and for directing the management and restoration of multiple-habitat sites.},
author = {Evans, Darren M. and Pocock, Michael J. O. and Memmott, Jane},
doi = {10.1111/ele.12117},
isbn = {1461-0248},
issn = {1461023X},
journal = {Ecology Letters},
keywords = {Bio-control,Bio-indicators,Biodiversity,Ecosystem function,Ecosystem services,Extinction,Nature conservation,Plant-animal interaction,Pollination,Restoration},
month = {jul},
number = {7},
pages = {844--852},
pmid = {23692559},
title = {{The robustness of a network of ecological networks to habitat loss}},
url = {http://www.ncbi.nlm.nih.gov/pubmed/23692559},
volume = {16},
year = {2013}
}
@article{Carvalheiro2013,
abstract = {Concern about biodiversity loss has led to increased public investment in conservation. Whereas there is a widespread perception that such initiatives have been unsuccessful, there are few quantitative tests of this perception. Here, we evaluate whether rates of biodiversity change have altered in recent decades in three European countries (Great Britain, Netherlands and Belgium) for plants and flower visiting insects. We compared four 20-year periods, comparing periods of rapid land-use intensification and natural habitat loss (1930–1990) with a period of increased conservation investment (post-1990). We found that extensive species richness loss and biotic homogenisation occurred before 1990, whereas these negative trends became substantially less accentuated during recent decades, being partially reversed for certain taxa (e.g. bees in Great Britain and Netherlands). These results highlight the potential to maintain or even restore current species assemblages (which despite past extinctions are still of great conservation value), at least in regions where large-scale land-use intensification and natural habitat loss has ceased.},
author = {Carvalheiro, Lu{\'{i}}sa Gigante and Kunin, William E. and Keil, Petr and Aguirre-Guti{\'{e}}rrez, Jesus and Ellis, Willem Nicolaas and Fox, Richard and Groom, Quentin and Hennekens, Stephan and {Van Landuyt}, Wouter and Maes, Dirk and {Van de Meutter}, Frank and Michez, Denis and Rasmont, Pierre and Ode, Baudewijn and Potts, Simon Geoffrey and Reemer, Menno and Roberts, Stuart Paul Masson and Schamin{\'{e}}e, Joop and Wallisdevries, Michiel F. and Biesmeijer, Jacobus Christiaan},
doi = {10.1111/ele.12121},
isbn = {1461-0248},
issn = {14610248},
journal = {Ecology Letters},
keywords = {Accumulation curves,Biodiversity loss,Community ecology,Plant-flower visitor communities,Pollination,Similarity,Spatial homogenisation,Species richness estimations,Temporal and spatial patterns},
month = {jul},
number = {7},
pages = {870--878},
pmid = {23692632},
title = {{Species richness declines and biotic homogenisation have slowed down for NW-European pollinators and plants}},
url = {http://www.ncbi.nlm.nih.gov/pubmed/23692632},
volume = {16},
year = {2013}
}
@article{Poisot2013,
abstract = {The biodiversity-ecosystem functioning (BEF) relationship is central in community ecology. Its drivers in competitive systems (sampling effect and functional complementarity) are intuitive and elegant, but we lack an integrative understanding of these drivers in complex ecosystems. Because networks encompass two key components of the BEF relationship (species richness and biomass flow), they provide a key to identify these drivers, assuming that we have a meaningful measure of functional complementarity. In a network, diversity can be defined by species richness, the number of trophic levels, but perhaps more importantly, the diversity of interactions. In this paper, we define the concept of trophic complementarity (TC), which emerges through exploitative and apparent competition processes, and study its contribution to ecosystem functioning. Using a model of trophic community dynamics, we show that TC predicts various measures of ecosystem functioning, and generate a range of testable predictions. We find that, in addition to the number of species, the structure of their interactions needs to be accounted for to predict ecosystem productivity.},
author = {Poisot, Timoth{\'{e}}e and Mouquet, Nicolas and Gravel, Dominique},
doi = {10.1111/ele.12118},
isbn = {1461-0248},
issn = {1461023X},
journal = {Ecology Letters},
keywords = {Biodiversity-ecosystem functioning,Food webs,Trophic interactions},
month = {jul},
number = {7},
pages = {853--861},
pmid = {23692591},
title = {{Trophic complementarity drives the biodiversity-ecosystem functioning relationship in food webs}},
url = {http://www.ncbi.nlm.nih.gov/pubmed/23692591},
volume = {16},
year = {2013}
}
@article{Wilder2013,
abstract = {Understanding why food chains are relatively short in length has been an area of research and debate for decades. We tested if progressive changes in the nutritional content of arthropods with trophic position limit the availability of a key nutrient, lipid, for carnivores. Arthropods at higher trophic levels had progres-sively less lipid and more protein in their bodies compared with arthropods at lower trophic levels. The nutrients present in arthropod bodies were directly related to the nutrients that predators extracted when feeding on those arthropods. As a consequence, nutrient assimilation shifted from lipid-biased to protein-biased as arthropods ascended trophic levels from herbivores to secondary carnivores. Successive changes in the nutritional consequences of predation may ultimately set an upper limit on the number of trophic levels in arthropod communities. Further work is needed to examine the influence of lipid and other nutri-ents on food web traits in a range of ecosystems.},
author = {Wilder, Shawn M. and Norris, Michael and Lee, Raymond W. and Raubenheimer, David and Simpson, Stephen J.},
doi = {10.1111/ele.12116},
isbn = {1461-0248},
issn = {14610248},
journal = {Ecology Letters},
keywords = {Arthropods,Food chain length,Food web,Nutrition,Predation},
month = {jul},
number = {7},
pages = {895--902},
pmid = {23701046},
title = {{Arthropod food webs become increasingly lipid-limited at higher trophic levels}},
url = {http://www.ncbi.nlm.nih.gov/pubmed/23701046},
volume = {16},
year = {2013}
}
@article{Guimera2009,
abstract = {Network analysis is currently used in a myriad of contexts, from identifying potential drug targets to predicting the spread of epidemics and designing vaccination strategies and from finding friends to uncovering criminal activity. Despite the promise of the network approach, the reliability of network data is a source of great concern in all fields where complex networks are studied. Here, we present a general mathematical and computational framework to deal with the problem of data reliability in complex networks. In particular, we are able to reliably identify both missing and spurious interactions in noisy network observations. Remarkably, our approach also enables us to obtain, from those noisy observations, network reconstructions that yield estimates of the true network properties that are more accurate than those provided by the observations themselves. Our approach has the potential to guide experiments, to better characterize network data sets, and to drive new discoveries.},
archivePrefix = {arXiv},
arxivId = {1004.4791},
author = {Guimer{\`{a}}, Roger and Sales-Pardo, Marta},
doi = {10.1073/pnas.0908366106},
eprint = {1004.4791},
isbn = {0027-8424},
issn = {0027-8424},
journal = {Proceedings of the National Academy of Sciences of the United States of America},
number = {52},
pages = {22073--22078},
pmid = {20018705},
title = {{Missing and spurious interactions and the reconstruction of complex networks.}},
url = {http://www.pnas.org/content/106/52/22073.short},
volume = {106},
year = {2009}
}
@book{R,
abstract = {applicability for this approach.},
address = {Vienna, Austria},
archivePrefix = {arXiv},
arxivId = {arXiv:1011.1669v3},
author = {{R Core Team}},
doi = {10.1017/CBO9781107415324.004},
eprint = {arXiv:1011.1669v3},
isbn = {9788578110796},
issn = {08628408},
pmid = {25246403},
publisher = {R Foundation for Statistical Computing},
title = {{R: a language and environment for statistical computing}},
url = {http://www.r-project.org},
year = {2016}
}
@article{Dowle2013,
abstract = {Species density is higher in the tropics (low latitude) than in temperate regions (high latitude) resulting in a latitudinal biodiversity gradient (LBG). The LBG must be generated by differential rates of speciation and/or extinction and/or immigration among regions, but the role of each of these processes is still unclear. Recent studies examining differences in rates of molecular evolution have inferred a direct link between rate of molecular evolution and rate of speciation, and postulated these as important drivers of the LBG. Here we review the molecular genetic evidence and examine the factors that might be responsible for differences in rates of molecular evolution. Critical to this is the directionality of the relationship between speciation rates and rates of molecular evolution.},
author = {Dowle, E J and Morgan-Richards, M and Trewick, S a},
doi = {10.1038/hdy.2013.4},
isbn = {1365-2540 (Electronic)$\backslash$r0018-067X (Linking)},
issn = {1365-2540},
journal = {Heredity},
keywords = {Biodiversity,Climate,Evolution, Molecular,Extinction, Biological,Genetic Speciation,Mutation Rate,Phylogeny,Population Density},
month = {jun},
number = {6},
pages = {501--10},
pmid = {23486082},
publisher = {Nature Publishing Group},
title = {{Molecular evolution and the latitudinal biodiversity gradient.}},
url = {http://www.pubmedcentral.nih.gov/articlerender.fcgi?artid=3656639{\&}tool=pmcentrez{\&}rendertype=abstract},
volume = {110},
year = {2013}
}
@article{Thompson2006,
abstract = {Studies seeking to explain local patterns of diversity have typically relied on niche explanations, reflected in correlations with local environmental conditions, or neutral theory, invoking dispersal processes and speciation. We used macroinvertebrate community data from 10 streams that varied independently in local ecological conditions and spatial proximity. Neutral theory predicts that similarity in communities will be negatively associated with distance between sites, while niche theory suggests that community similarity will be positively associated with similarity in local ecological conditions. Similarity in total invertebrate, grazer and predator assemblages showed negative relationships with distance and, for grazers and predators, positive relationships with local ecological conditions. However, the best model predicting community similarity in all three cases included aspects of both local ecological conditions and distance between sites. When assemblages were analysed according to dispersal ability, high-dispersal species were shown to be freely accessing all sites and community similarity was not well predicted by either local ecology or spatial separation. Assemblages of species with low and moderate dispersal ability were best predicted by combined models, including distance between sites and local ecological factors. The results suggest that the perceived dichotomy between neutral and local environmental processes in determining local patterns of diversity may not be useful. Neutral and niche processes structured these communities differentially depending on trophic level and species traits. We emphasize the potential for both dispersal processes and local environmental conditions to explain local patterns of diversity.},
author = {Thompson, Ross and Townsend, Colin},
doi = {10.1111/j.1365-2656.2006.01068.x},
isbn = {0021-8790},
issn = {00218790},
journal = {Journal of Animal Ecology},
keywords = {Dispersal,Local determinism,Neutral theory,Niche,stream macroinvertebrate},
month = {mar},
number = {2},
pages = {476--484},
pmid = {16638000},
title = {{A truce with neutral theory: Local deterministic factors, species traits and dispersal limitation together determine patterns of diversity in stream invertebrates}},
url = {http://www.ncbi.nlm.nih.gov/pubmed/16638000},
volume = {75},
year = {2006}
}
@article{Farjalla2012,
abstract = {After much debate, there is an emerging consensus that the composition of many ecological communities is determined both by species traits, as proposed by niche theory, as well as by chance events. A critical question for ecology is, therefore, which attributes of species predict the dominance of deterministic or stochastic processes. We outline two hypotheses by which organism size could determine which processes structure ecological communities, and we test these hypotheses by comparing the community structure in bromeliad phytotelmata of three groups of organisms (bacteria, zooplankton, and macroinvertebrates) that encompass a 10 000-fold gradient in body size, but live in the same habitat. Bacteria had no habitat associations, as would be expected from trait-neutral stochastic processes, but still showed exclusion among species pairs, as would be expected from niche-based processes. Macroinvertebrates had strong habitat and species associations, indicating niche-based processes. Zooplankton, with body size between bacteria and macroinvertebrates, showed intermediate habitat associations. We concluded that a key niche process, habitat filtering, strengthened with organism size, possibly because larger organisms are both less plastic in their fundamental niches and more able to be selective in dispersal. These results suggest that the relative importance of deterministic and stochastic processes may be predictable from organism size.},
author = {Farjalla, Vinicius F. and Srivastava, Diane S. and Marino, Nicholas a C and Azevedo, Fernanda D. and Dib, Viviane and Lopes, Paloma M. and Rosado, Alexandre S. and Bozelli, Reinaldo L. and Esteves, Francisco a.},
doi = {10.1890/11-1144.1},
isbn = {10.1890/11-1144.1},
issn = {00129658},
journal = {Ecology},
keywords = {Bacteria,Brazil,Bromeliad,Competitive exclusion,Food web,Habitat filtering,Macroinvertebrates,Niche theory,Restinga de jurubatiba national park,Species co-occurrence,Stochasticity,Variance partitioning,Zooplankton},
month = {jul},
number = {7},
pages = {1752--1759},
pmid = {22919920},
title = {{Ecological determinism increases with organism size}},
url = {http://www.ncbi.nlm.nih.gov/pubmed/22919920},
volume = {93},
year = {2012}
}
@article{Galetti2013,
abstract = {Local extinctions have cascading effects on ecosystem functions, yet little is known about the potential for the rapid evolutionary change of species in human-modified scenarios. We show that the functional extinction of large-gape seed dispersers in the Brazilian Atlantic forest is associated with the consistent reduction of the seed size of a keystone palm species. Among 22 palm populations, areas deprived of large avian frugivores for several decades present smaller seeds than nondefaunated forests, with negative consequences for palm regeneration. Coalescence and phenotypic selection models indicate that seed size reduction most likely occurred within the past 100 years, associated with human-driven fragmentation. The fast-paced defaunation of large vertebrates is most likely causing unprecedented changes in the evolutionary trajectories and community composition of tropical forests.},
archivePrefix = {arXiv},
arxivId = {arXiv:1201.1430v1},
author = {Galetti, Mauro and Guevara, Roger and C{\^{o}}rtes, Marina C. and Fadini, Rodrigo and {Sandro Von Matter}, 4 and {Abra{\~{a}}o B. Leite, 1 F{\'{a}}bio Labecca, 1 Thiago Ribeiro, 1 Carolina S. Carvalho}, 5 and {Rosane G. Collevatti, 5 Mathias M. Pires, 6 Paulo R. Guimar{\~{a}}es Jr., 6 Pedro H. Brancalion}, 7 and {Milton C. Ribeiro}, 1 Pedro Jordano8},
doi = {10.1126/science.1233774},
eprint = {arXiv:1201.1430v1},
isbn = {0036-8075},
issn = {0036-8075},
journal = {Science},
month = {may},
number = {May},
pages = {1086--1091},
pmid = {23723235},
title = {{Functional Extinction of Birds Drives Rapid Evolutionary Changes in Seed Size}},
url = {http://www.sciencemag.org/content/340/6136/1086.abstract?sid=f1b7891d-8a9a-4a5a-801c-905c687a9f36},
volume = {340},
year = {2013}
}
@article{Thompson2012,
abstract = {The global biodiversity crisis concerns not only unprecedented loss of species within communities, but also related consequences for ecosystem function. Community ecology focuses on patterns of species richness and community composition, whereas ecosystem ecology focuses on fluxes of energy and materials. Food webs provide a quantitative framework to combine these approaches and unify the study of biodiversity and ecosystem function. We summarise the progression of food-web ecology and the challenges in using the food-web approach. We identify five areas of research where these advances can continue, and be applied to global challenges. Finally, we describe what data are needed in the next generation of food-web studies to reconcile the structure and function of biodiversity. ?? 2012 Elsevier Ltd.},
author = {Thompson, Ross M. and Brose, Ulrich and Dunne, Jennifer A. and Hall, Robert O. and Hladyz, Sally and Kitching, Roger L. and Martinez, Neo D. and Rantala, Heidi and Romanuk, Tamara N. and Stouffer, Daniel B. and Tylianakis, Jason M.},
doi = {10.1016/j.tree.2012.08.005},
isbn = {0169-5347},
issn = {01695347},
journal = {Trends in Ecology and Evolution},
month = {sep},
number = {12},
pages = {689--697},
pmid = {22959162},
publisher = {Elsevier Ltd},
title = {{Food webs: reconciling the structure and function of biodiversity}},
url = {http://www.ncbi.nlm.nih.gov/pubmed/22959162},
volume = {27},
year = {2012}
}
@article{Wiley2013,
abstract = {Human exploitation of marine ecosystems is more recent in oceanic than near shore regions, yet our understanding of human impacts on oceanic food webs is comparatively poor. Few records of species that live beyond the continental shelves date back more than 60 y, and the sheer size of oceanic regions makes their food webs difficult to study, even in modern times. Here, we use stable carbon and nitrogen isotopes to study the foraging history of a generalist, oceanic predator, the Hawaiian petrel (Pterodroma sandwichensis), which ranges broadly in the Pacific from the equator to near the Aleutian Islands. Our isotope records from modern and ancient, radiocarbon-dated bones provide evidence of over 3,000 y of dietary stasis followed by a decline of ca. 1.8‰ in $\delta$(15)N over the past 100 y. Fishery-induced trophic decline is the most likely explanation for this sudden shift, which occurs in genetically distinct populations with disparate foraging locations. Our isotope records also show that coincident with the apparent decline in trophic level, foraging segregation among petrel populations decreased markedly. Because variation in the diet of generalist predators can reflect changing availability of their prey, a foraging shift in wide-ranging Hawaiian petrel populations suggests a relatively rapid change in the composition of oceanic food webs in the Northeast Pacific. Understanding and mitigating widespread shifts in prey availability may be a critical step in the conservation of endangered marine predators such as the Hawaiian petrel.},
author = {Wiley, Anne E and Ostrom, Peggy H and Welch, Andreanna J and Fleischer, Robert C and Gandhi, Hasand and Southon, John R and Stafford, Thomas W and Penniman, Jay F and Hu, Darcy and Duvall, Fern P and James, Helen F},
doi = {10.1073/pnas.1300213110},
isbn = {1091-6490 (Electronic)$\backslash$n0027-8424 (Linking)},
issn = {1091-6490},
journal = {Proceedings of the National Academy of Sciences of the United States of America},
keywords = {Age Factors,Analysis of Variance,Animals,Birds,Birds: metabolism,Birds: physiology,Bone and Bones,Bone and Bones: chemistry,Carbon Isotopes,Carbon Isotopes: analysis,Carbon Radioisotopes,Carbon Radioisotopes: analysis,Climate,Diet,Feathers,Feathers: chemistry,Food Chain,Hawaii,Human Activities,Humans,Mass Spectrometry,Nitrogen Isotopes,Nitrogen Isotopes: analysis,Pacific Ocean,Polystyrenes},
month = {may},
number = {22},
pages = {8972--8977},
pmid = {23671094},
title = {{Millennial-scale isotope records from a wide-ranging predator show evidence of recent human impact to oceanic food webs}},
url = {http://www.pubmedcentral.nih.gov/articlerender.fcgi?artid=3670381{\&}tool=pmcentrez{\&}rendertype=abstract},
volume = {110},
year = {2013}
}
@article{Chesson1997,
abstract = {Abstract.—Harsh conditions (e.g., mortality and stress) reduce population growth rates directly; secondarily, they may reduce the intensity of interactions between organisms. Near-exclusive fo- cus on the secondary effect of these forms of harshness has led ecologists to believe that they reduce the importance of ecological interactions, such as competition, and favor coexistence of even ecologically very similar species. By examining both the costs and the benefits, we show that harshness alone does not lessen the importance of species interactions or limit their role in community structure. Species coexistence requires niche differences, and harshness does not in itself make coexistence more likely. Fluctuations in environmental conditions (e.g., disturbance, seasonal change, and weather variation) also have been regarded as decreasing species interac- tions and favoring coexistence, but we argue that coexistence can only be favored when fluctua- tions create spatial or temporal niche opportunities. We argue that important diversity-promoting roles for harsh and fluctuating conditions depend on deviations from the assumptions of additive effects and linear dependencies most commonly found in ecological models. Such considerations imply strong roles for species interactions in the diversity of a community.},
author = {Chesson, Peter and Huntly, Nancy},
doi = {10.1086/286080},
isbn = {0003014719715},
issn = {0003-0147},
journal = {The American Naturalist},
number = {5},
pages = {519--553},
pmid = {18811299},
title = {{The Roles of Harsh and Fluctuating Conditions in the Dynamics of Ecological Communities}},
url = {http://www.jstor.org/stable/10.1086/286080},
volume = {150},
year = {1997}
}
@misc{Dunne2013a,
author = {Dunne, Jennifer a.},
number = {1992},
pages = {595--608},
title = {{Dunne - 2013 - Supplemental info}},
year = {2013}
}
@article{Anderson2006,
abstract = {The traditional likelihood-based test for differences in multivariate dispersions is known to be sensitive to nonnormality. It is also impossible to use when the number of variables exceeds the number of observations. Many biological and ecological data sets have many variables, are highly skewed, and are zero-inflated. The traditional test and even some more robust alternatives are also unreasonable in many contexts where measures of dispersion based on a non-Euclidean dissimilarity would be more appropriate. Distance-based tests of homogeneity of multivariate dispersions, which can be based on any dissimilarity measure of choice, are proposed here. They rely on the rotational invariance of either the multivariate centroid or the spatial median to obtain measures of spread using principal coordinate axes. The tests are straightforward multivariate extensions of Levene's test, with P-values obtained either using the traditional F-distribution or using permutation of either least-squares or LAD residuals. Examples illustrate the utility of the approach, including the analysis of stabilizing selection in sparrows, biodiversity of New Zealand fish assemblages, and the response of Indonesian reef corals to an El Ni{\~{n}}o. Monte Carlo simulations from the real data sets show that the distance-based tests are robust and powerful for relevant alternative hypotheses of real differences in spread.},
author = {Anderson, Marti J.},
doi = {10.1111/j.1541-0420.2005.00440.x},
isbn = {1541-0420},
issn = {0006341X},
journal = {Biometrics},
keywords = {Dissimilarity,Heterogeneity,Levene's test,Multivariate analysis,Permutation tests,Principal coordinates,Robust methods},
month = {mar},
number = {1},
pages = {245--253},
pmid = {16542252},
title = {{Distance-based tests for homogeneity of multivariate dispersions}},
url = {http://www.ncbi.nlm.nih.gov/pubmed/16542252},
volume = {62},
year = {2006}
}
@incollection{Hubbell2009,
address = {Princeton},
author = {Hubbell, Stephen P},
booktitle = {The theory of island biogeography revisited},
editor = {Losos, Jonathan B and Ricklefs, Robert E},
isbn = {9781400831920},
keywords = {colonization,density dependence,island biogeography,neutral theory,null model},
mendeley-tags = {colonization,density dependence,island biogeography,neutral theory,null model},
pages = {264--292},
publisher = {Princeton University Press},
title = {{Neutral theory and the theory of island biogeography}},
url = {http://books.google.com/books?hl=en{\&}lr={\&}id=slwedU4I4JMC{\&}oi=fnd{\&}pg=PA264{\&}dq=Neutral+Theory+and+the+Theory+of+Island+Biogeography{\&}ots=CK9MFcFvhr{\&}sig=CF7Nzj2MFMupfHimuZEEZ0FmDdQ},
year = {2009}
}
@article{Sales-Pardo2007,
abstract = {Extracting understanding from the growing "sea" of biological and socioeconomic data is one of the most pressing scientific challenges facing us. Here, we introduce and validate an unsupervised method for extracting the hierarchical organization of complex biological, social, and technological networks. We define an ensemble of hierarchically nested random graphs, which we use to validate the method. We then apply our method to real-world networks, including the air-transportation network, an electronic circuit, an e-mail exchange network, and metabolic networks. Our analysis of model and real networks demonstrates that our method extracts an accurate multiscale representation of a complex system.},
archivePrefix = {arXiv},
arxivId = {0705.1679},
author = {Sales-Pardo, Marta and Guimer{\`{a}}, Roger and Moreira, Andr{\'{e}} a and Amaral, Lu{\'{i}}s a Nunes},
doi = {10.1073/pnas.0703740104},
eprint = {0705.1679},
isbn = {0709460104},
issn = {0027-8424},
journal = {Proceedings of the National Academy of Sciences of the United States of America},
keywords = {Algorithms,Cluster Analysis,Escherichia coli,Escherichia coli: metabolism,Metabolic Networks and Pathways,Metabolism,Models, Economic,Models, Statistical,Models, Theoretical,Probability,Social Class,Systems Biology,Systems Theory,Transportation},
month = {sep},
number = {39},
pages = {15224--9},
pmid = {17881571},
title = {{Extracting the hierarchical organization of complex systems.}},
url = {http://www.pnas.org/content/104/39/15224.abstract},
volume = {104},
year = {2007}
}
@article{Webster2007,
abstract = {A convincing body of evidence now exists to indicate that the ubiquitous protozoan Toxoplasma gondii can cause permanent behavioral changes in its host, even as a consequence of adult-acquired latent infection. Such behavioral alterations appear to be the product of strong selective pressures for the parasite to enhance transmission from its intermediate host reservoir, primarily rodent, to its feline definitive host, wherein sexual reproduction can occur and the life cycle completed. This article reviews evidence of behavioral alterations in animal hosts and considers what these may elucidate about the potential mechanisms involved and what implications such alterations could have on animal and human health.},
author = {Webster, Joanne P.},
doi = {10.1093/schbul/sbl073},
isbn = {0586-7614 (Print)$\backslash$r0586-7614 (Linking)},
issn = {05867614},
journal = {Schizophrenia Bulletin},
keywords = {Behavior,Parasite,Rodents,Schizophrenia},
month = {may},
number = {3},
pages = {752--756},
pmid = {17218613},
title = {{The effect of Toxoplasma gondii on animal behavior: Playing cat and mouse}},
url = {http://www.pubmedcentral.nih.gov/articlerender.fcgi?artid=2526137{\&}tool=pmcentrez{\&}rendertype=abstract},
volume = {33},
year = {2007}
}
@article{House2011,
abstract = {Cat odors induce rapid, innate and stereotyped defensive behaviors in rats at first exposure, a presumed response to the evolutionary pressures of predation. Bizarrely, rats infected with the brain parasite Toxoplasma gondii approach the cat odors they typically avoid. Since the protozoan Toxoplasma requires the cat to sexually reproduce, this change in host behavior is thought to be a remarkable example of a parasite manipulating a mammalian host for its own benefit. Toxoplasma does not influence host response to non-feline predator odor nor does it alter behavior on olfactory, social, fear or anxiety tests, arguing for specific manipulation in the processing of cat odor. We report that Toxoplasma infection alters neural activity in limbic brain areas necessary for innate defensive behavior in response to cat odor. Moreover, Toxoplasma increases activity in nearby limbic regions of sexual attraction when the rat is exposed to cat urine, compelling evidence that Toxoplasma overwhelms the innate fear response by causing, in its stead, a type of sexual attraction to the normally aversive cat odor.},
author = {House, Patrick K. and Vyas, Ajai and Sapolsky, Robert},
doi = {10.1371/journal.pone.0023277},
isbn = {1932-6203 (Electronic)$\backslash$r1932-6203 (Linking)},
issn = {19326203},
journal = {PLoS ONE},
keywords = {Animal,Animal: parasitology,Animal: physiology,Animal: physiopathology,Animal: psychology,Animals,Brain,Brain: metabolism,Brain: physiology,Cats,Fear,Fear: physiology,Fear: psychology,Female,Host-Parasite Interactions,Immunohistochemistry,Long-Evans,Male,Odors,Predatory Behavior,Predatory Behavior: physiology,Proto-Oncogene Proteins c-fos,Proto-Oncogene Proteins c-fos: metabolism,Rats,Sexual Behavior,Signal Transduction,Signal Transduction: physiology,Smell,Smell: physiology,Toxoplasma,Toxoplasma: physiology,Toxoplasmosis,Urine,Urine: chemistry},
month = {jan},
number = {8},
pages = {e23277},
pmid = {21858053},
title = {{Predator cat odors activate sexual arousal pathways in brains of toxoplasma gondii infected rats}},
url = {http://www.pubmedcentral.nih.gov/articlerender.fcgi?artid=3157360{\&}tool=pmcentrez{\&}rendertype=abstract},
volume = {6},
year = {2011}
}
@incollection{Cadle1993,
abstract = {Many managers are at a loss concerning the strategy they should to adopt to deal with global warming and the requirements enforced by the Kyoto Protocol. This article proposes a global approach to anticipate the possible impacts of global warming on organisations and to explore policies and measures that managers can implement to cope with this issue. The frame analysis proposed sheds light on the relevance of proactive or more wait-and-see responses to global warming while stressing the importance of promoting environmental intelligence and other preliminary measures before deciding what strategy to adopt. The article also calls into question monolithic and static views of climate change strategies and illustrates, through examples, the actions that managers can take to put the Kyoto Protocol on their agendas. ?? 2006 Elsevier Ltd. All rights reserved.},
address = {Chicago},
archivePrefix = {arXiv},
arxivId = {z0034},
author = {Boiral, Olivier},
booktitle = {Long Range Planning},
chapter = {25},
doi = {10.1016/j.lrp.2006.07.002},
editor = {Ricklefs, R.E. and Schluter, D},
eprint = {z0034},
isbn = {0024-6301},
issn = {00246301},
number = {3},
pages = {315--330},
pmid = {667},
publisher = {University of Chicago Press},
title = {{Global Warming: Should Companies Adopt a Proactive Strategy?}},
volume = {39},
year = {2006}
}
@article{Strong1980,
abstract = {Null hypotheses entertain the possibility that nothing has happened, that a process has not occurred, or that change has not been produced by a cause of interest. Null hypotheses are reference points against which alternatives should be contrasted. They are used not only in statistics but in all sciences. “This hypothesis{\ldots}is{\ldots} characteristic of all experimentation” (Fisher 1935). In physics for example, an important null hypothesis of the post-Newtonian era was that time is a variable which is independent of all other factors. Modern physics is based upon the alternative hypothesis that time can be a function of space and relative velocities. Another famous null hypothesis, that the speed of light is independent of its direction, inspired the Michelson-Morley experiments, which failed to disprove it. An example in chemistry is that there is no molecular property unique to life, that any synthesis by protoplasm can be repeated in the test tube. Modern biochemistry has failed to disprove this null hypothesis. But the term null hypothesis sounds odd in reference to much of physics and chemistry. It is not found in textbooks nor is it used frequently in conversation about these disciplines. Though all sciences use null hypotheses in principle, the ‘atomistic'1 sciences of physics and chemistry often use them implicitly. In atomistic sciences, fundamental units are simple and quite similar to one another, and effects of phenomena are commonly so distinct that the null state of no effect does not need special recognition.},
author = {Strong, Donald R.},
doi = {10.1007/BF00413928},
isbn = {0039-7857},
issn = {00397857},
journal = {Synthese},
number = {2},
pages = {271--285},
pmid = {1064},
title = {{Null hypotheses in ecology}},
url = {http://www.springerlink.com/index/RN171P6521515641.pdf},
volume = {43},
year = {1980}
}
@article{Warton2012,
abstract = {A critical summary of count data is its mean-variance relationship, yet this is rarely considered in multivariate analysis in ecology. This study considers what is being implicitly assumed about the mean-variance relationship in distance-based analyses - multivariate analyses based on a matrix of pairwise distances - and what the effect is of any misspecification of the mean-variance relationship. It is known that distance-based analyses make implicit assumptions that are typically out-of-step with what is observed in real data, which has major consequences. Potential consequences of this mean-variance misspecifiation are: confounding location and dispersion effects in ordinations; misleading results when trying to identify taxa in which an effect is expressed; failure to detect a multivariate effect unless it is expressed in high-variance taxa. Data transformation does not solve the problem. A solution is to use generalized linear models and their recent multivariate generalisations, which is shown here to have desirable properties.},
author = {Warton, David I. and Wright, Stephen T. and Wang, Yi},
doi = {10.1111/j.2041-210X.2011.00127.x},
isbn = {2041-210X},
issn = {2041210X},
journal = {Methods in Ecology and Evolution},
keywords = {Bray-Curtis distance,Community composition,Generalised linear models,Mean-variance relationship,Multi-dimensional scaling,Multivariate analysis,PERMANOVA,SIMPER,Species-by-site data},
month = {feb},
number = {1},
pages = {89--101},
title = {{Distance-based multivariate analyses confound location and dispersion effects}},
url = {http://doi.wiley.com/10.1111/j.2041-210X.2011.00127.x},
volume = {3},
year = {2012}
}
@article{Manual,
abstract = {This sheet is intended to be used to quickly look up attribute and method names . For a complete reference , including descriptions of},
author = {Manual, Pygrace Reference},
title = {{PyGrace Cheatsheet}}
}
@article{Anderson2001,
abstract = {Abstract Hypothesis-testing methods for multivariate data are needed to make rigorous probability statements about the effects of factors and their interactions in experiments. Analysis of variance is particularly powerful for the analysis of univariate data. The traditional multivariate analogues, however, are too stringent in their assumptions for most ecological multivariate data sets. Non-parametric methods, based on permutation tests, are preferable. This paper describes a new non-parametric method for multivariate analysis of variance, after McArdle and Anderson (in press). It is given here, with several applications in ecology, to provide an alternative and perhaps more intuitive formulation for ANOVA (based on sums of squared distances) to complement the description provided by McArdle and Anderson (in press) for the analysis of any linear model. It is an improvement on previous non-parametric methods because it allows a direct additive partitioning of variation for complex models. It does this while maintaining the flexibility and lack of formal assumptions of other non-parametric methods. The test-statistic is a multivariate analogue to Fisher's F-ratio and is calculated directly from any symmetric distance or dissimilarity matrix. P-values are then obtained using permutations. Some examples of the method are given for tests involving several factors, including factorial and hierarchical (nested) designs and tests of interactions. [ABSTRACT FROM AUTHOR]},
author = {Anderson, Marti J},
doi = {10.1111/j.1442-9993.2001.tb00081.x},
file = {::},
isbn = {1442-9993},
issn = {14429985},
journal = {Austral Ecology},
keywords = {ANALYSIS of variance,ECOLOGY,distance measure,experimental design,linear model,multifactorial,multivariate dissimilarity,partitioning,permutation tests,statistics},
number = {2001},
pages = {32--46},
pmid = {18385208},
title = {{A new method for non-parametric multivariate analysis of variance.}},
url = {http://libproxy.udayton.edu/login?url=http://search.ebscohost.com/login.aspx?direct=true{\&}db=a9h{\&}AN=5472169{\&}site=eds-live},
volume = {26},
year = {2001}
}
@article{Strona2013,
abstract = {Our data set includes 38 008 fish parasite records (for Acanthocephala, Cestoda, Monogenea, Nematoda, Trematoda) compiled from the scientific literature, Internet databases, and museum collections paired to the corresponding host ecological, biogeograph- ical, and phylogenetic traits (maximum length, growth rate, life span, age at maturity, trophic level, habitat preference, geographical range size, taxonomy). The data focus on host features, because specific parasite traits are not consistently available across records. For this reason, the data set is intended as a flexible resource able to extend the principles of ecological niche modeling to the host–parasite system, providing researchers with the data to model parasite niches based on their distribution in host species and the associated host features. In this sense, the database offers a framework for testing general ecological, biogeographical, and phylogenetic hypotheses based on the identification of hosts as parasite habitat. Potential applications of the data set are, for example, the investigation of species–area relationships or the taxonomic distribution of host-specificity. The provided host–parasite list is that currently used by Fish Parasite Ecology Software Tool (FishPEST, http://purl.oclc.org/fishpest), which is a website that allows researchers to model several aspects of the relationships between fish parasites and their hosts. The database is intended for researchers who wish to have more freedom to analyze the database than currently possible with FishPEST. However, for readers who have not seen FishPEST, we recommend using this as a starting point for interacting with the database.},
author = {Strona, Giovanni and Palomares, Maria Lourdes D and Bailly, Nicolas and Galli, Paolo and Lafferty, Kevin D},
doi = {10.1890/12-1419.1},
isbn = {0012-9658},
issn = {0012-9658},
journal = {Ecology},
keywords = {FishPEST,host range,host specificity,parasite species richness},
number = {2},
pages = {544},
title = {{Host range, host ecology, and distribution of more than 11 800 fish parasite species}},
url = {http://dx.doi.org/10.1890/12-1419.1},
volume = {94},
year = {2013}
}
@article{Allan2013,
abstract = {The importance of competition between similar species in driving community assembly is much debated. Recently, phylogenetic patterns in species composition have been investigated to help resolve this question: phylogenetic clustering is taken to imply environmental filtering, and phylogenetic overdispersion to indicate limiting similarity between species. We used experimental plant communities with random species compositions and initially even abundance distributions to examine the development of phylogenetic pattern in species abundance distributions. Where composition was held constant by weeding, abundance distributions became overdispersed through time, but only in communities that contained distantly related clades, some with several species (i.e., a mix of closely and distantly related species). Phylogenetic pattern in composition therefore constrained the development of overdispersed abundance distributions, and this might indicate limiting similarity between close relatives and facilitation/complementarity between distant relatives. Comparing the phylogenetic patterns in these communities with those expected from the monoculture abundances of the constituent species revealed that interspecific competition caused the phylogenetic patterns. Opening experimental communities to colonization by all species in the species pool led to convergence in phylogenetic diversity. At convergence, communities were composed of several distantly related but species-rich clades and had overdispersed abundance distributions. This suggests that limiting similarity processes determine which species dominate a community but not which species occur in a community. Crucially, as our study was carried out in experimental communities, we could rule out local evolutionary or dispersal explanations for the patterns and identify ecological processes as the driving force, underlining the advantages of studying these processes in experimental communities. Our results show that phylogenetic relations between species provide a good guide to understanding community structure and add a new perspective to the evidence that niche complementarity is critical in driving community assembly.},
author = {Allan, Eric and Jenkins, Tania and Fergus, Alexander J F and Roscher, Christiane and Fischer, Markus and Petermann, Jana and Weisser, Wolfgang W. and Schmid, Bernhard},
doi = {10.1890/11-2279.1},
isbn = {3058},
issn = {00129658},
journal = {Ecology},
keywords = {Biodiversity,Community assembly,Convergence,Jena Experiment,Limiting similarity,Phylogenetic dispersion},
number = {2},
pages = {465--477},
pmid = {23691665},
title = {{Experimental plant communities develop phylogenetically overdispersed abundance distributions during assembly}},
url = {http://www.esajournals.org/doi/pdf/10.1890/11-2279.1},
volume = {94},
year = {2013}
}
@article{Gillespie2013,
abstract = {Researchers increasingly recognize the important role of mutualisms in structuring communities and view positive interactions in a community context rather than as simple pairwise interactions. Indirect effects, such as those that predators have on lower trophic levels, are a key process in community ecology. However, such top-down indirect effects have rarely been extended to mutualisms. Antagonists of one mutualist have the potential to negatively affect the second mutualist through negative effects on their partner, and the magnitude of such effects should vary with mutualism strength. Bumble bees are ecologically and economically important pollinators that are an ideal system to determine if such indirect effects play an important role in mutualisms. Bumble bees are attacked by an array of parasites and predators, and they interact with a range of plants that vary in their dependence on bumble bees for reproduction. We tested whether variation in parasitism rates by Nosema bombi, Crithidia bombi, and conopid flies correlated with reproduction of greenhouse-raised plants placed in the field. At multiple sites over two years, we studied four plant species that varied in reliance on bumble bees as pollinators. We found a consistent negative relationship between Nosema parasitism and measures of pollination for Trifolium pratense and Solanum carolinense, plant species with high bumble bee visitation, whereas Rudbeckia hirta and Daucus carota, plant species with generalized pollination, experienced no impacts of Nosema. However, both Crithidia and conopids showed inconsistent relationships with pollination service. Although these patterns are correlational, they provide evidence that parasites of bumble bees may have negative indirect effects on plants, and that mutualism strength can moderate the magnitude of such effects.},
author = {Gillespie, S. D. and Adler, L. S.},
doi = {10.1890/12-0406.1},
isbn = {0012-9658 (Print)$\backslash$r0012-9658 (Linking)},
issn = {00129658},
journal = {Ecology},
keywords = {Bombus spp,Bumble bee,Conopid,Crithidia bombi,Indirect effects,Mutualism,Nosema bombi,Parasitism,Pollination service,USA,Western Massachusetts},
number = {2},
pages = {454--464},
pmid = {23691664},
title = {{Indirect effects on mutualisms: Parasitism of bumble bees and pollination service to plants}},
url = {http://www.esajournals.org/doi/abs/10.1890/12-0406.1},
volume = {94},
year = {2013}
}
@article{Axelsen2013,
abstract = {Extinction from habitat loss is the signature conservation problem of the twenty-first century. Despite its importance, estimating extinction rates is still highly uncertain because no proven direct methods or reliable data exist for verifying extinctions. The most widely used indirect method is to estimate extinction rates by reversing the species-area accumulation curve, extrapolating backwards to smaller areas to calculate expected species loss. Estimates of extinction rates based on this method are almost always much higher than those actually observed. This discrepancy gave rise to the concept of an 'extinction debt', referring to species 'committed to extinction' owing to habitat loss and reduced population size but not yet extinct during a non-equilibrium period. Here we show that the extinction debt as currently defined is largely a sampling artefact due to an unrecognized difference between the underlying sampling problems when constructing a species-area relationship (SAR) and when extrapolating species extinction from habitat loss. The key mathematical result is that the area required to remove the last individual of a species (extinction) is larger, almost always much larger, than the sample area needed to encounter the first individual of a species, irrespective of species distribution and spatial scale. We illustrate these results with data from a global network of large, mapped forest plots and ranges of passerine bird species in the continental USA; and we show that overestimation can be greater than 160{\%}. Although we conclude that extinctions caused by habitat loss require greater loss of habitat than previously thought, our results must not lead to complacency about extinction due to habitat loss, which is a real and growing threat.},
author = {He, Fangliang and Hubbell, Stephen P},
doi = {10.1038/nature09985},
isbn = {00280836 (ISSN)},
issn = {1476-4687},
journal = {Nature},
keywords = {Animals,Biodiversity,Biological,Ecosystem,Extinction,Models,Passeriformes,Passeriformes: classification,Population Density,Statistical,Trees,Trees: growth {\&} development,United States},
number = {7347},
pages = {368--371},
pmid = {21593870},
title = {{Species-area relationships always overestimate extinction rates from habitat loss. Supplementary information.}},
url = {http://www.ncbi.nlm.nih.gov/pubmed/21593870},
volume = {473},
year = {2011}
}
@article{Young2013,
abstract = {Many different drivers, including productivity, ecosystem size, and disturbance, have been considered to explain natural variation in the length of food chains. Much remains unknown about the role of these various drivers in determining food chain length, and particularly about the mechanisms by which they may operate in terrestrial ecosystems, which have quite different ecological constraints than aquatic environments, where most food chain length studies have been thus far conducted. In this study, we tested the relative importance of ecosystem size and productivity in influencing food chain length in a terrestrial setting. We determined that (1) there is no effect of ecosystem size or productive space on food chain length; (2) rather, food chain length increases strongly and linearly with productivity; and (3) the observed changes in food chain length are likely achieved through a combination of changes in predator size, predator behavior, and consumer diversity along gradients in productivity. These results lend new insight into the mechanisms by which productivity can drive changes in food chain length, point to potential for systematic differences in the drivers of food web structure between terrestrial and aquatic systems, and challenge us to consider how ecological context may control the drivers that shape food chain length.},
author = {Young, Hillary S. and Mccauley, Douglas J. and Dunbar, Robert B. and Hutson, Michael S. and Ter-Kuile, Ana Miller and Dirzo, Rodolfo},
doi = {10.1890/12-0729.1},
isbn = {0012-9658},
issn = {00129658},
journal = {Ecology},
keywords = {Ecosystem size,Food chain length,Food web structure,Islands,Palmyra atoll,Productivity},
number = {3},
pages = {692--701},
pmid = {23687895},
title = {{The roles of productivity and ecosystem size in determining food chain length in tropical terrestrial ecosystems}},
url = {http://www.esajournals.org/doi/abs/10.1890/12-0729.1},
volume = {94},
year = {2013}
}
@article{Lefcheck2013,
abstract = {The degree of dietary generalism among consumers has important consequences for population, community, and ecosystem processes, yet the effects on consumer fitness of mixing food types have not been examined comprehensively. We conducted a meta-analysis of 161 peer-reviewed studies reporting 493 experimental manipulations of prey diversity to test whether dietmixing enhances consumer fitness based on the intrinsic nutritional quality of foods and consumer physiology. Averaged across studies, mixed diets conferred significantly higher fitness than the average of single-species diets, but not the best single prey species. More than half of individual experiments, however, showed maximal growth and reproduction on mixed diets, consistent with the predicted benefits of a balanced diet.Mixed diets including chemically defended prey were no better than the average prey type, opposing the prediction that a diverse diet dilutes toxins. Finally, mixed-model analysis showed that the effect of diet mixing was stronger for herbivores than for higher trophic levels. The generally weak evidence for the nutritional benefits of diet mixing in these primarily laboratory experiments suggests that diet generalism is not strongly favored by the inherent physiological benefits of mixing food types, but is more likely driven by ecological and environmental influences on consumer foraging},
author = {Lefcheck, Jonathan S. and Whalen, Matthew a. and Davenport, Theresa M. and Stone, Joshua P. and Duffy, J. Emmett},
doi = {10.1890/12-0192.1},
isbn = {0012-9658 (Print)$\backslash$n0012-9658 (Linking)},
issn = {00129658},
journal = {Ecology},
keywords = {Balanced diet,Biodiversity,Diet mixing,Dietary specialism vs generalism,Meta-analysis,Nutritional ecology,Toxin dilution,Trophic transfer},
number = {3},
pages = {565--572},
pmid = {23687882},
title = {{Physiological effects of diet mixing on consumer fitness: A meta-analysis}},
url = {http://www.esajournals.org/doi/pdf/10.1890/12-0192.1},
volume = {94},
year = {2013}
}
@article{Shipley2013,
abstract = {Classical path analysis is a statistical technique used to test causal hypotheses involving multiple variables without latent variables, assuming linearity, multivariate normality, and a sufficient sample size. The d-separation (d-sep) test is a generalization of path analysis that relaxes these assumptions. Although model selection using Akaike's information criterion (AIC) is well established for classical path analysis, this model selection technique has not yet been developed for d-sep tests. In this paper, I explain how to use the AIC statistic for d-sep tests, give a worked example, and include instructions (supplemental},
author = {Shipley, Bill},
doi = {10.1890/12-0976.1},
isbn = {0012-9658},
issn = {00129658},
journal = {Ecology},
keywords = {AIC statistic,D-separation (d-sep) test,Path analysis,Structural equation modeling(SEM)},
number = {3},
pages = {560--564},
pmid = {23687881},
title = {{The AIC model selection method applied to path analytic models compared using a d-separation test}},
url = {http://www.esajournals.org/doi/pdf/10.1890/12-0976.1},
volume = {94},
year = {2013}
}
@article{Laliberte2010,
abstract = {Human activities drive biotic homogenization (loss of regional diversity) of many taxa. However, whether species interaction networks (e.g., food webs) can also become homogenized remains largely unexplored. Using 48 quantitative parasitoid-host networks replicated through space and time across five tropical habitats, we show that deforestation greatly homogenized network structure at a regional level, such that interaction composition became more similar across rice and pasture sites compared with forested habitats. This was not simply caused by altered consumer and resource community composition, but was associated with altered consumer foraging success, such that parasitoids were more likely to locate their hosts in deforested habitats. Furthermore, deforestation indirectly homogenized networks in time through altered mean consumer and prey body size, which decreased in deforested habitats. Similar patterns were obtained with binary networks, suggesting that interaction (link) presence-absence data may be sufficient to detect network homogenization effects. Our results show that tropical agroforestry systems can support regionally diverse parasitoid-host networks, but that removal of canopy cover greatly homogenizes the structure of these networks in space, and to a lesser degree in time. Spatiotemporal homogenization of interaction networks may alter coevolutionary outcomes and reduce ecological resilience at regional scales, but may not necessarily be predictable from community changes observed within individual trophic levels.},
author = {Lalibert{\'{e}}, Etienne and Tylianakis, Jason M.},
doi = {10.1890/09-1328.1},
isbn = {0012-9658},
issn = {00129658},
journal = {Ecology},
keywords = {Biodiversity,Body size,Ecuador,Food web,Insect,Interaction network,Parasitoid,Predator,Prey},
number = {6},
pages = {1740--1747},
pmid = {20583715},
title = {{Deforestation homogenizes tropical parasitoid-host networks}},
url = {http://www.esajournals.org/doi/abs/10.1890/09-1328.1},
volume = {91},
year = {2010}
}
@article{Laliberte2012,
abstract = {There is much concern that the functioning of ecosystems will be affected by human-induced changes in biodiversity, of which land-use change is the most important driver. However, changes in biodiversity may be only one of many pathways through which land use alters ecosystem functioning, and its importance relative to other pathways remains unclear. In particular, although biodiversity-ecosystem function research has focused primarily on grasslands, the increases in agricultural inputs (e.g., fertilization, irrigation) and grazing pressure that drive change in grasslands worldwide have been largely ignored. Here we show that long-term (27-year) manipulations of soil resource availability and sheep grazing intensity caused marked, consistent shifts in grassland plant functional composition and diversity, with cascading (i.e., causal chains of) direct, indirect, and interactive effects on multiple ecosystem functions. Resource availability exerted dominant control over above-ground net primary production (ANPP), both directly and indirectly via shifts in plant functional composition. Importantly, the effects of plant functional diversity and grazing intensity on ANPP shifted from negative to positive as agricultural inputs increased, providing strong evidence that soil resource availability modulates the impacts of plant diversity and herbivory on primary production. These changes in turn altered litter decomposition and, ultimately, soil carbon sequestration, highlighting the relevance of ANPP as a key integrator of ecosystem functioning. Our study reveals how human alterations of bottom-up (resources) and top-down (herbivory) forces together interact to control the functioning of grazing systems, the most extensive land use on Earth.},
author = {Lalibert{\'{e}}, Etienne and Tylianakis, Jason M.},
doi = {10.1890/11-0338.1},
isbn = {0012-9658},
issn = {00129658},
journal = {Ecology},
keywords = {Biodiversity,Carbon sequestration,Ecosystem services,Functional diversity,Grasslands,Grazing intensity,Litter decomposition,Long-term experiment,Mackenzie Basin South Island New Zealand,Primary production,Resource availability},
month = {jan},
number = {1},
pages = {145--155},
pmid = {22486095},
title = {{Cascading effects of long-term land-use changes on plant traits and ecosystem functioning}},
url = {http://www.ncbi.nlm.nih.gov/pubmed/22486095},
volume = {93},
year = {2012}
}
@misc{Anderson2005,
abstract = {PERMANOVA is a computer program for testing the simultaneous response of one or more variables to one or more factors in an ANOVA experimental design on the basis of any distance measure, using permutation methods. These notes for users assume knowledge of multi-factorial ANOVA, which has the same basic logic in multivariate as in univariate analysis, and an understanding of what it means to test a multivariate hypothesis. A more complete description of the method is given in Anderson (2001a) and McArdle {\&} Anderson (2001). The program includes: choice of appropriate transformation and/or standardization of the data; choice of 19 distance (or dissimilarity) measures to use as the basis of the analysis; option to rank the distances in the distance matrix before the analysis; analysis and partitioning of the total sum of squares according to the full model, including appropriate treatment of factors that are fixed or random, crossed (orthogonal) or nested (hierarchical), and all interaction terms; correct calculation of an appropriate distance-based pseudo F-statistic for each term in the model, based on expected mean squares as in univariate ANOVA (Winer et al. 1991, Searle et al.1992); correct permutation procedures to obtain P-values for each term in the model, using the correct permutable units (Anderson {\&} ter Braak 2003); choice of permutation method: raw data units or residuals under either a reduced or a full model (Anderson 2001b, Anderson {\&} Legendre 1999, Anderson {\&} Robinson 2001); correct P-values also obtained through Monte Carlo random draws from the asymptotic permutation distribution (Anderson {\&} Robinson 2003); option to include one or more covariables (i.e., to perform ANCOVA or MANCOVA); pair-wise a posteriori comparisons of levels for single factors, including within individual levels of other factors in the case of significant interactions and the use of correct permutable units in each case.},
author = {Anderson, Marti J},
booktitle = {Austral Ecology},
doi = {10.1139/cjfas-58-3-626},
file = {:Users/alyssacirtwill/Downloads/anderson2001.pdf:pdf},
isbn = {0706-652X},
issn = {12057533},
keywords = {PERMANOVA,bioestad{\'{i}}stica},
number = {3},
pages = {1--24},
pmid = {5684003},
title = {{PERMANOVA Permutational multivariate analysis of variance}},
url = {http://www.stat.auckland.ac.nz/{~}mja{\%}5Cnhttp://www.stat.auckland.ac.nz/{~}mja/prog/PERMANOVA{\_}UserNotes.pdf},
volume = {58},
year = {2005}
}
@article{Oksanen2013,
abstract = {This tutorial demostrates the use of ordination methods in R package vegan. The tutorial assumes familiarity both with R and with community ordination. Package vegan supports all basic ordination methods, including non-metric multidimensional scaling. The constrained ordination methods include constrained analysis of proximities, redundancy analysis and constrained correspondence analysis. Package vegan also has support functions for fitting environmental variables and for ordination graphics.},
author = {Oksanen, Jari},
doi = {10.1016/0169-5347(88)90124-3},
isbn = {0171-8630},
issn = {01695347},
journal = {Trends in Ecology {\&} Evolution},
number = {5},
pages = {121},
title = {{Multivariate analysis of ecological communities in R: vegan tutorial}},
volume = {3},
year = {2013}
}
@article{Woodward2009,
abstract = {P{\textgreater}1. Dramatic advances have been made recently in the study of biodiversity-ecosystem functioning (B-EF) relations and food web ecology. These fields are now starting to converge, and this fusion has the potential to improve our understanding of how environmental stressors modulate ecosystem processes and the supply of 'goods and services'. 2. Food web structure and dynamics can exert particularly strong influences on B-EF relations in fresh waters, as consumer-resource interactions (e.g. trophic cascades) are often more important than horizontal interactions within trophic levels. For instance, many freshwater food webs are size structured, with large organisms tending to occupy the higher trophic levels and often exerting powerful effects on ecosystem processes. However, because they are also vulnerable to perturbations, non-random losses of these large taxa can alter both food web structure and ecosystem functioning profoundly. 3. Recently, the focus of food web research has shifted away from exploring patterns, towards developing an understanding of processes (e.g. quantifying fluxes of individuals, biomass, energy, nutrients) and how the two interact. Many of the best-characterized food webs are from fresh waters, and these ecosystems are now being used to address some of the shortcomings of earlier B-EF studies. I have identified several key gaps in our current knowledge and highlighted potentially fruitful avenues of future B-EF and food web research. 4. A major challenge for this newly emerging research is to place it within a unified theoretical framework. The application of metabolic theory and ecological stoichiometry may help to achieve this goal by considering biological systems within the constraints imposed upon them by physical and chemical laws.},
author = {Woodward, Guy},
doi = {10.1111/j.1365-2427.2008.02081.x},
isbn = {0046-5070},
issn = {00465070},
journal = {Freshwater Biology},
keywords = {Community ecology,Ecological stoichiometry,Ecosystem services,Foraging theory,Metabolic theory of ecology},
month = {oct},
number = {10},
pages = {2171--2187},
pmid = {3005},
title = {{Biodiversity, ecosystem functioning and food webs in fresh waters: Assembling the jigsaw puzzle}},
url = {http://doi.wiley.com/10.1111/j.1365-2427.2008.02081.x},
volume = {54},
year = {2009}
}
@article{Setala2002,
abstract = {The objective of the present paper, using decomposer food webs as a tool, is to explore the levels of the ecological hierarchy (trophic groups, feeding guilds, species populations) at which reduction in complexity brings about significant changes in ecosystem performance. A review is given of various mini-ecosystem studies that have recently been conducted at the University of Jyvaskyla. It is hypothesized that the typical features of soils as a habitat, and the peculiarities of belowground food webs, such as the commonness of indirect interactions (mediated through abiotic resources) among the biota, together with the high frequency of polyphagy/omnivory among soil organisms, produce a diversity-ecosystem functioning relationship that is likely to differ from those of aquatic and aboveground food webs. Experiments showed that alterations in 'trophic levels' were reflected in Significant changes in decomposition processes, which, in turn, had substantial impact on primary production. Similarly, heterogeneity within 'trophic levels' was shown to be associated with increased growth of birch and pine seedlings. In contrast, species diversity within a feeding guild had little or no influence on ecosystem-level processes. However, the species-specific properties of individual taxa were shown to be more influential in affecting plant growth than species number per se. For example, the presence of an omnivorous enchytraeid species in the mini-ecosystems was observed to consistently be associated with high biomass production of tree seedlings. It is concluded that the so-called trophic dynamic models based on direct feeding interactions are of limited value in predicting the outcomes of interactions taking place belowground.},
author = {Set{\"{a}}l{\"{a}}, Heikki},
doi = {10.1046/j.1440-1703.2002.00480.x},
isbn = {0912-3814},
issn = {09123814},
journal = {Ecological Research},
keywords = {Decomposer food webs,Ecosystem function,Nutrient dynamics,Omnivory},
number = {2},
pages = {207--215},
title = {{Sensitivity of ecosystem functioning to changes in trophic structure, functional group composition and species diversity in belowground food webs}},
volume = {17},
year = {2002}
}
@article{Sempere2013,
abstract = {A fish farm in Southeastern Spain was described using an Ecopath mass-balanced model, aimed at characterising its structure, the interactions among ecological groups and the impact of fish farms and fisheries. The model comprised 41 functional groups (including the artificial food input). Comparing consumption and respiration to total system throughput suggests lower energy use in the fish farm, resulting in an accumulation of detritus. The production to total system throughput ratio was low due to the low efficiency of the modelled ecosystem. The connectance and system omnivory indexes were low, typical of a simple or immature food web in terms of structure and dynamics. Artificial food pellets provided energy and nutrients to sustain system function and generate a considerable reserve from which it can draw to meet unexpected perturbations. The study shows the substantial effect the artificial food pellets have on the wild aggregated fishes, which could act to buffer the ecosystem and hence prevent environmental degradation. ?? 2012 Elsevier B.V.},
author = {Bayle-Sempere, Just T. and Arregu{\'{i}}n-S{\'{a}}nchez, Francisco and Sanchez-Jerez, Pablo and Salcido-Guevara, Luis a. and Fernandez-Jover, Dami{\'{a}}n and Zetina-Rej{\'{o}}n, Manuel J.},
doi = {10.1016/j.ecolmodel.2012.08.028},
isbn = {03043800},
issn = {03043800},
journal = {Ecological Modelling},
keywords = {Aquaculture,Impact,Management,Mass-balanced models,Mediterranean,Wild aggregated fishes},
month = {jan},
pages = {135--147},
publisher = {Elsevier B.V.},
title = {{Trophic structure and energy fluxes around a Mediterranean fish farm}},
url = {http://linkinghub.elsevier.com/retrieve/pii/S0304380012004395},
volume = {248},
year = {2013}
}
@article{Tylianakis2013,
abstract = {Three-quarters of global food crops depend at least partly on pollination by animals, usually insects ( 1 ). These crops form an increasing fraction of global food demand ( 2 ). Given this importance, widespread declines in pollinator diversity ( 3 ) have led to concern about a global “pollination crisis” ( 4 ). However, others have argued that this concern is premature and that conservation action cannot yet be justified on the basis of deteriorating pollination ( 5 ). Are concerns of a pollinator crisis exaggerated, and can we make do with better management of honeybee colonies? Two articles in this issue provide compelling answers to these questions. On page 1611, Burkle et al. demonstrate that native wild pollinators are declining ( 6 ). On page 1608, Garibaldi et al. show that managed honeybees cannot compensate for this loss ( 7 ).},
author = {Tylianakis, J. M.},
doi = {10.1126/science.1235464},
isbn = {0036-8075},
issn = {0036-8075},
journal = {Science},
keywords = {Agricultural,Agricultural: growth {\&} development,Animals,Bees,Bees: physiology,Biological,Crops,Extinction,Fruit,Fruit: growth {\&} development,Insects,Insects: physiology,Poaceae,Poaceae: growth {\&} development,Pollination,Trees,Trees: growth {\&} development},
month = {feb},
number = {6127},
pages = {1532--1533},
pmid = {23449995},
title = {{The Global Plight of Pollinators}},
url = {http://www.ncbi.nlm.nih.gov/pubmed/23449995{\%}5Cnhttp://www.sciencemag.org/cgi/doi/10.1126/science.1235464},
volume = {339},
year = {2013}
}
@article{Burkle2013,
abstract = {Using historic data sets, we quantified the degree to which global change over 120 years disrupted plant-pollinator interactions in a temperate forest understory community in Illinois, USA. We found degradation of interaction network structure and function and extirpation of 50{\%} of bee species. Network changes can be attributed to shifts in forb and bee phenologies resulting in temporal mismatches, nonrandom species extinctions, and loss of spatial co-occurrences between extant species in modified landscapes. Quantity and quality of pollination services have declined through time. The historic network showed flexibility in response to disturbance; however, our data suggest that networks will be less resilient to future changes.},
author = {Burkle, L. a. and Marlin, J. C. and Knight, T. M.},
doi = {10.1126/science.1232728},
file = {:Users/alyssacirtwill/Documents/Papers/Burkle, Marlin, Knight{\_}2013{\_}Science.pdf:pdf},
isbn = {0036-8075},
issn = {0036-8075},
journal = {Science},
keywords = {Animals,Bees,Bees: physiology,Biological,Extinction,Flowers,Flowers: growth {\&} development,Illinois,Poaceae,Poaceae: growth {\&} development,Pollination,Portulacaceae,Portulacaceae: growth {\&} development,Trees,Trees: growth {\&} development},
month = {feb},
number = {6127},
pages = {1611--1615},
pmid = {23449999},
title = {{Plant-Pollinator Interactions over 120 Years: Loss of Species, Co-Occurrence, and Function}},
url = {http://www.sciencemag.org/cgi/doi/10.1126/science.1232728},
volume = {339},
year = {2013}
}
@article{Gotelli2001,
abstract = {Species richness is a fundamental measurement of community and regional diversity, and it underlies many ecological models and conservation strategies. In spite of its importance, ecologists have not always appreciated the effects of abundance and sampling effort on richness measures and comparisons. We survey a series of common pitfalls in quantifying and comparing taxon richness. These pitfalls can be largely avoided by using accumulation and rarefaction curves, which may be based on either individuals or samples. These taxon sampling curves contain the basic information for valid richness comparisons, including category-subcategory ratios (species-to-genus and species-to-individual ratios). Rarefaction methods - both sample-based and individual-based - allow for meaningful standardization and comparison of datasets. Standardizing data sets by area or sampling effort may produce very different results compared to standardizing by number of individuals collected, and it is not always clear which measure of diversity is more appropriate. Asymptotic richness estimators provide lower-bound estimates for taxon-rich groups such as tropical arthropods, in which observed richness rarely reaches an asymptote, despite intensive sampling. Recent examples of diversity studies of tropical trees, stream invertebrates, and herbaceous plants emphasize the importance of carefully quantifying species richness using taxon sampling curves.},
author = {Gotelli, Nicholas J. and Colwell, Robert K.},
doi = {10.1046/j.1461-0248.2001.00230.x},
isbn = {1461-023X},
issn = {1461023X},
journal = {Ecology Letters},
keywords = {Accumulation curves,Asymptotic richness,Biodiversity,Category-subcategory ratios,Rarefaction,Richness estimation,Species density,Species richness,Taxon sampling,Taxonomic ratios},
number = {4},
pages = {379--391},
pmid = {223},
title = {{Quantifying biodiversity: Procedures and pitfalls in the measurement and comparison of species richness}},
url = {http://onlinelibrary.wiley.com/doi/10.1046/j.1461-0248.2001.00230.x/full},
volume = {4},
year = {2001}
}
@article{Houle1998,
abstract = {This paper evaluates whether the floating island model is a plausible transoceanic mode of dispersal among small to medium-sized land vertebrates. The actual Atlantic Ocean served as a model of winds and currents velocity, and data were sampled from modern marine pilot charts in two areas of the ocean. The main objective was to quantify the number of days required for a floating island to cross three Paleogene water barriers, 50, 40 and 30 Mya, namely the Atlantic Ocean, the Caribbean Sea as well as the Southeast Indian Ocean between Sundaland and the northern Australian Plate. Paleodistances and ancient circulation patterns of winds and currents are known in each region. It is proposed that paleowinds (not paleocurrents), and its effect on floating objects, were the key accelerating force of floating islands' velocity. In the most conservative scenario, the Paleogene Atlantic oceanic barrier could have been crossed westerly by a floating island in 7.7 days at 50 Mya, 10.8 days at 40 Mya and 14.7 days at 30 Mya; the Paleogene Caribbean Sea from South to North America could have been surmounted northerly in 18.2 days at 50 Mya, 16.6 days at 40 Mya and 15.1 days at 30 Mya; finally, the Southeast Indian Ocean from Australia to Sundaland could have been crossed northerly in 25.6 days at 50 Mya, 19.5 days at 40 Mya and 12.2 days at 30 Mya. There are many physical, physiological and behavioral constraints to transoceanic migrations. Such migrations can be considered as potentially successful however if it is shown that the journey did not exceed the survival limit of a given genetically viable group of animals. Metabolic studies involving survival limits to water deprivation suggest that proposed scenarios are plausible for small to medium size mammals and most land reptiles. These studies also suggest that migrating groups were probably preadapted in their original habitats (before the oceanic journey) to some degree of temporary dehydration and therefore to strong or moderate seasonal variations in water and food availability in order to survive the transoceanic event. Considering for example, the origin of the South American platyrrhine monkeys and caviomorph rodents, respective ancestors of these mammal groups probably lived in habitats with definite dry and rainy seasons. This conclusion implies that the survival limit of both platyrrhines and rodents ancestors ranged between 10 and 15 days, an assumption related to the transoceanic scenarios presented here but nevertheless compatible with dehydration studies. From the tectonic perspective, when plates are moving rapidly (27-55 mm/yr), the number of days a floating island needs to cross an oceanic water barrier is doubled every 20 My when these plates are distancing each other (e.g. South America-Africa) or fractioned by two every 20 My when the plates are approaching each other (e.g. Australia towards Southeast Asia). An additional interesting point of the model is its applicability to other epochs and other water barriers.},
author = {Houle, Alain},
isbn = {13669516},
issn = {1366-9516},
journal = {Diversity and Distributions},
keywords = {Floating Island Model, Rafting, Transoceanic Migra},
number = {5/6},
pages = {201--216},
title = {{Floating Islands: A Mode of Long-Distance Dispersal for Small and Medium- Sized Terrestrial Vertebrates}},
url = {http://links.jstor.org/sici?sici=1366-9516(199809/11)4:5/6{\%}3C201:FIAMOL{\%}3E2.0.CO;2-P},
volume = {4},
year = {1998}
}
@article{Miyamoto2004,
abstract = {Intertidal mussel species often provide a secondary substrate for competitively inferior species, while excluding them from the primary substrate. To evaluate the net effect, we conducted field experiments that specifically focused on interactions between mussels Septifer virgatus (Mieg- mann) and algae species. Mussels affected the abundance of 7 algal species differentially, with effects being positive, neutral, or negative. The red alga Porphyra yezoensis grew more abundantly on mussel shells than on rock surfaces. Mussels facilitated recruitment intensity of this species, resulting in increased adult cover on the shells. In contrast, the green alga Monostroma angicava grew less abundantly on mussel shells than on rock surfaces. Mussel shells did not modify recruit- ment intensity of this alga, but did inhibit its frond growth, and would thus seem to reduce adult cover. Modifications of grazer density by the mussels did not affect either of these algae species. The results indicate that the net effect of mussels on competitively inferior species is not grazer-mediated, and varies from species to species.},
author = {Miyamoto, Yasushi and Noda, Takashi},
doi = {10.3354/meps276293},
isbn = {0171-8630},
issn = {01718630},
journal = {Marine Ecology Progress Series},
keywords = {Algae,Competitive exclusion,Competitively dominant species,Ecosystem engineer,Facilitation,Mussel,Net mussel effect},
number = {1},
pages = {293--298},
title = {{Effects of mussels on competitively inferior species: Competitive exclusion to facilitation}},
url = {http://www.int-res.com/articles/meps2004/276/m276p293.pdf},
volume = {276},
year = {2004}
}
@article{Gingras2013,
abstract = {An inverse relationship between body size and advertisement call frequency has been found in several frog species. However, the generalizability of this relation- ship across different clades and across a large distribution of species remains underexplored. We investigated this relationship in a large sample of 136 species belonging to four clades of anurans (Bufo, Hylinae, Leptodactylus and Rana) using semi-automatic, high-throughput analysis software. We employed two measures of call frequency: fundamental frequency (F0) and dominant frequency (DF). The slope of the relationship between male snout-vent length (SVL) and frequency did not differ significantly among the four clades. However, Rana call at a significantly lower frequency relative to size than the other clades, and Bufo call at a signifi- cantly higher frequency relative to size than Leptodactylus. Because the relation- ship between F0 and body size may be more straightforwardly explained by biomechanical constraints, we confirmed that a similar inverse relationship was observed between F0 and SVL. Finally, spectral flatness, an indicator of the tonality of the vocalizations, was found to be inversely correlated with SVL, contradicting an oft-cited prediction that larger animals should have rougher voices. Our results confirm a tight and widespread link between body size and call frequency in anurans, and suggest that laryngeal allometry and vocal fold dimen- sions in particular are responsible},
author = {Gingras, B. and Boeckle, M. and Herbst, C. T. and Fitch, W. T.},
doi = {10.1111/j.1469-7998.2012.00973.x},
isbn = {0952-8369},
issn = {09528369},
journal = {Journal of Zoology},
keywords = {Allometry,Anurans,Bioacoustics},
month = {feb},
number = {2},
pages = {143--150},
title = {{Call acoustics reflect body size across four clades of anurans}},
url = {http://doi.wiley.com/10.1111/j.1469-7998.2012.00973.x},
volume = {289},
year = {2013}
}
@article{Feldman2013,
abstract = {Body size and body shape are tightly related to an animal's physiology, ecology and life history, and, as such, play a major role in understanding ecological and evolutionary phenomena. Because organisms have different shapes, only a uniform proxy of size, such as mass, may be suitable for comparisons between taxa. Unfortunately, snake masses are rarely reported in the literature. On the basis of 423 species of snakes in 10 families, we developed clade-specific equations for the estimation of snake masses from snout–vent lengths and total lengths. We found that snout–vent lengths predict masses better than total lengths. By examining the effects of phylogeny, as well as ecological and life history traits on the relationship between mass and length, we found that viviparous species are heavier than oviparous species, and diurnal species are heavier than nocturnal species. Furthermore, microhabitat preferences profoundly influence body shape: arboreal snakes are lighter than terrestrial snakes, whereas aquatic snakes are heavier than terrestrial snakes of a similar length},
author = {Feldman, Anat and Meiri, Shai},
doi = {10.1111/j.1095-8312.2012.02001.x},
isbn = {1095-8312},
issn = {00244066},
journal = {Biological Journal of the Linnean Society},
keywords = {Body mass,Body size,Microhabitat,Mode of reproduction,Shape,Snout,Total length,Venomousness,Vent length},
number = {1},
pages = {161--172},
title = {{Length-mass allometry in snakes}},
url = {http://onlinelibrary.wiley.com/doi/10.1111/j.1095-8312.2012.02001.x/full},
volume = {108},
year = {2013}
}
@article{Poulin2011,
abstract = {Interspecifi c variation in parasite species richness among host species has generated much empirical research. As in compar- isons among geographical areas, controlling for variation in host body size is crucial because host size determines resource availability. Recent developments in the use of species – area relationships (SARs) to detect hotspots of biodiversity provide a powerful way to control for host body size, and to identify ‘ hot ' and ‘ cold hosts ' of parasite diversity, i.e. hosts with more or fewer parasites than expected from their size. Applying SAR modelling to six large datasets on parasite species richness in vertebrates, we search for hot and cold hosts and assess the eff ect of other ecological variables on the probability that a host species is hot/cold taking body size (and sampling eff ort) into account. Five non-sigmoid SAR models were fi tted to the data by optimisation; their relative likelihood was evaluated using the Bayesian information criterion, before deriving an averaged SAR function. Overall, the fi t between the fi ve SAR models and the actual data was poor; there was substantial uncertainty surrounding the fi tted models, and the best model diff ered among the six datasets. T ese results show that host body size is not a strong or consistent determinant of parasite species richness across taxa. Hotspots were defi ned as host species lying above the upper limit of the 80{\%} confi dence interval of the averaged SAR, and coldspots as species lying below its lower limit. Our analyses revealed (1) no apparent eff ect of specifi c ecological factors (i.e. water temperature, mean depth range, latitude or population density) on the likelihood of a host species being a hot or coldspot; (2) evidence of phylogenetic clustering, i.e. hosts from certain families are more likely to be hotspots (or coldspots) than other species, independently of body size. T ese fi ndings suggest that host phylogeny may sometimes outweigh specifi c host ecological traits as a predictor of whether or not a host species harbours more (or fewer) parasite species than expected for its size.},
author = {Poulin, Robert and Guilhaumon, Fran{\c{c}}ois and Randhawa, Haseeb S. and Luque, Jos{\'{e}} L. and Mouillot, David},
doi = {10.1111/j.1600-0706.2010.19036.x},
isbn = {0030-1299},
issn = {00301299},
journal = {Oikos},
month = {may},
number = {5},
pages = {740--747},
title = {{Identifying hotspots of parasite diversity from species-area relationships: Host phylogeny versus host ecology}},
url = {http://doi.wiley.com/10.1111/j.1600-0706.2010.19036.x},
volume = {120},
year = {2011}
}
@article{Freckleton2003,
abstract = {Aim: To describe the geographical pattern of mean body size of the non-volant mammals of the Nearctic and Neotropics and evaluate the influence of five environmental variables that are likely to affect body size gradients. $\backslash$nLocation: The Western Hemisphere.$\backslash$nMethods: We calculated mean body size (average log mass) values in 110 × 110 km cells covering the continental Nearctic and Neotropics. We also generated cell averages for mean annual temperature, range in elevation, their interaction, actual evapotranspiration, and the global vegetation index and its coefficient of variation. Associations between mean body size and environmental variables were tested with simple correlations and ordinary least squares multiple regression, complemented with spatial autocorrelation analyses and split-line regression. We evaluated the relative support for each multiple-regression model using AIC. $\backslash$nResults: Mean body size increases to the north in the Nearctic and is negatively correlated with temperature. In contrast, across the Neotropics mammals are largest in the tropical and subtropical lowlands and smaller in the Andes, generating a positive correlation with temperature. Finally, body size and temperature are nonlinearly related in both regions, and split-line linear regression found temperature thresholds marking clear shifts in these relationships (Nearctic 10.9 °C; Neotropics 12.6 °C). The increase in body sizes with decreasing temperature is strongest in the northern Nearctic, whereas a decrease in body size in mountains dominates the body size gradients in the warmer parts of both regions. $\backslash$nMain conclusions: We confirm previous work finding strong broad-scale Bergmann trends in cold macroclimates but not in warmer areas. For the latter regions (i.e. the southern Nearctic and the Neotropics), our analyses also suggest that both local and broad-scale patterns of mammal body size variation are influenced in part by the strong mesoscale climatic gradients existing in mountainous areas. A likely explanation is that reduced habitat sizes in mountains limit the presence of larger-sized mammals.},
author = {Freckleton, Robert P and Harvey, Paul H and Pagel, Mark},
doi = {10.1086/374346},
isbn = {0003-0147},
issn = {0003-0147},
journal = {The American naturalist},
number = {5},
pages = {821--825},
pmid = {12858287},
title = {{Bergmann's rule and body size in mammals.}},
url = {http://www.jstor.org/stable/10.1086/374346},
volume = {161},
year = {2003}
}
@article{Eastman2013,
abstract = {* Approaches for efficient statistical estimation of large phylogenies are now available (Bioinformatics, 2006, 22, 2688), and yet we lack adequate tools for synthesizing information from previous analyses into large timetrees. Here, we present a cross-platform r tool that integrates with tree of life efforts by mapping divergence times from an existing timetree (a ‘reference') to another uncalibrated phylogeny (a ‘target') that samples from the same lineage. Leveraging existing methods for rate-smoothing phylograms, this tool enables the rapid generation of very large timetrees where direct estimation of the timing of lineage diversification is either impracticable or impossible. * The primary output of the tool is to return divergence times for nodes resolved as concordant between the reference and target. Given the computed set of secondary calibrations, post hoc tree transformation can be accomplished using existing resources that assume either a strict or relaxed evolutionary clock. * Our software is provided open source in the geiger package (http://cran.r-project.org/package=geiger) and is thoroughly demonstrated in the Supporting Information.},
author = {Eastman, Jonathan M. and Harmon, Luke J. and Tank, David C.},
doi = {10.1111/2041-210X.12051},
isbn = {2041-210X},
issn = {2041210X},
journal = {Methods in Ecology and Evolution},
keywords = {Divergence time,Phylogenetics,Time scaling,Tree of life,geiger},
month = {mar},
number = {7},
pages = {688--691},
title = {{Congruification: Support for time scaling large phylogenetic trees}},
url = {http://doi.wiley.com/10.1111/2041-210X.12051},
volume = {4},
year = {2013}
}
@article{Poulin1995,
author = {Poulin, Robert},
journal = {Ecological Monographs},
keywords = {branch lengths,component community,ectoparasites,gastrointestinal helminths,habitat,host body size,host diet,independent contrasts,latitude},
number = {3},
pages = {283--302},
title = {{Phylogeny , Ecology , and the Richness of Parasite Communities in Vertebrates Author ( s ): Robert Poulin PHYLOGENY , ECOLOGY , AND THE RICHNESS OF PARASITE COMMUNITIES IN VERTEBRATES '}},
url = {http://www.jstor.org/stable/10.2307/2937061},
volume = {65},
year = {1995}
}
@article{Swenson2007,
abstract = {The relative importance of biotic, abiotic, and stochastic processes in structuring ecological communities continues to be a central focus in community ecology. In order to assess the role of phylogenetic relatedness on the nature of biodiversity we first quantified the degree of phylogenetic niche conservatism of several plant traits linked to plant form and function. Next we quantified the degree of phylogenetic relatedness across two fundamental scaling dimensions: plant size and neighborhood size. The results show that phylogenetic niche conservatism is likely widespread, indicating that closely related species are more functionally similar than distantly related species. Utilizing this information we show that three of five tropical forest dynamics plots (FDPs) exhibit similar scale-dependent patterns of phylogenetic structuring using only a spatial scaling axis. When spatial- and size-scaling axes were analyzed in concert, phylogenetic overdispersion of co-occurring species was most important at small spatial scales and in four of five FDPs for the largest size class. These results suggest that phylogenetic relatedness is increasingly important: (1) at small spatial scales, where phylogenetic overdispersion is more common, and (2) in large size classes, where phylogenetic overdispersion becomes more common throughout ontogeny. Collectively, our results highlight the critical spatial and size scales at which the degree of phylogenetic relatedness between constituent species influences the structuring of tropical forest diversity.},
author = {Swenson, Nathan G. and Enquist, Brian J. and Thompson, Jill and Zimmerman, Jess K.},
doi = {10.1890/06-1499.1},
isbn = {0012-9658 (Print)$\backslash$r0012-9658 (Linking)},
issn = {00129658},
journal = {Ecology},
keywords = {Body size,Community ecology,Phylogenetic trait conservatism,Phylogeny,Scaling,Species pool,Specific leaf area,Stoichiometry,Tropical forest dynamics plot,Wood density},
month = {jul},
number = {7},
pages = {1770--1780},
pmid = {17645023},
title = {{The influence of spatial and size scale on phylogenetic relatedness in tropical forest communities}},
url = {http://www.ncbi.nlm.nih.gov/pubmed/17645023},
volume = {88},
year = {2007}
}
@article{Brooks1979,
abstract = {Testing the context and extent of host-parasite coevolution. Syst. Zool. 28:299- 307.--Coevolution is defined as a combination of two processes: co-accommodation between host and parasite with no implication of host or parasite speciation and co-speciation, indicating concomitant host and parasite speciation. Parasite speciation in general is viewed as primarily the result of allopatric speciation processes regardless of host speciation or changes in host type. The observation that co-speciation of hosts and parasites forms a predominant pattern relates to a more general principle of biotic allopatric speciation explicit in the vicariance biogeography model, more than to an assumption that host speciation somehow causes parasite speciation. The close ecological relationship between hosts and parasites, which may be depicted using a variation of the MacArthur-Wilson island biogeography model, explains their spatial proximity at any time during which an isolating event occurs and thus may be necessary in some cases, but is not sufficient to explain parasite speciation, coevolution, or parasite phylogeny.},
author = {Brooks, D. R.},
doi = {10.1093/sysbio/28.3.299},
isbn = {0039-7989},
issn = {1063-5157},
journal = {Systematic Biology},
number = {3},
pages = {299--307},
title = {{Testing the Context and Extent of Host-Parasite Coevolution}},
url = {http://sysbio.oxfordjournals.org/content/28/3/299.short},
volume = {28},
year = {1979}
}
@book{Cohen1978,
abstract = {What is the minimum dimension of a niche space necessary to represent the overlaps among observed niches? This book presents a new technique for obtaining a partial answer to this elementary question about niche space. The author bases his technique on a relation between the combinatorial structure of food webs and the mathematical theory of interval graphs.Professor Cohen collects more than thirty food webs from the ecological literature and analyzes their statistical and combinatorial properties in detail. As a result, he is able to generalize: within habitats of a certain limited physical and temporal heterogeneity, the overlaps among niches, along their trophic (feeding) dimensions, can be represented in a one-dimensional niche space far more often than would be expected by chance alone and perhaps always. This compatibility has not previously been noticed. It indicates that real food webs fall in a small subset of the mathematically possible food webs.Professor Cohen discusses other apparently new features of real food webs, including the constant ratio of the number of kinds of prey to the number of kinds of predators in food webs that describe a community. In conclusion he discusses possible extensions and limitations of his results and suggests directions for future research.},
address = {Princeton},
author = {Cohen, J E},
booktitle = {Princeton University Press},
doi = {10.1016/0025-5564(79)90090-7},
isbn = {0691082022},
issn = {00255564},
number = {3-4},
pages = {189},
pmid = {683203},
publisher = {Princeton University Press},
title = {{Food webs and niche space.}},
url = {http://linkinghub.elsevier.com/retrieve/pii/0025556479900907},
volume = {44},
year = {1978}
}
@incollection{Dunne2006,
abstract = {Descriptions of food-web relationships first appeared more than a cen- tury ago, and the quantitative analysis of the network structure of food webs dates back several decades. Recent improvements in food-web data collection and analysis methods, coupled with a resurgence of interdis- ciplinary research on the topology of many kinds of “real-world”net- works, have resulted in renewed interest in food-web structure. This chapter reviews the history of the search for generalities in the struc- ture of complex food webs, and discusses current and future research trends. Analysis of food-web structure has used empirical and model- ing approaches, and has been inspired both by questions from ecology such as “What factors promote stability of complex ecosystems given internal dynamics and external perturbations?”and questions from net- work research such as “Do food webs display universal structure similar to other types of networks?”Recent research has suggested that once variable diversity and connectance are taken into account, there are uni- versal coarse-grained characteristics of how trophic links and species defined according to trophic function are distributed within food webs. In addition, aspects of food-web network structure have been shown to strongly influence the robust functioning and dynamical persistence of ecosystems.},
address = {New York},
archivePrefix = {arXiv},
arxivId = {9780195188165},
author = {Dunne, Jennifer A.},
booktitle = {Ecological networks: linking structure to dynamics in food webs},
chapter = {2},
editor = {Pascual, Mercedes and Dunne, Jennifer A.},
eprint = {9780195188165},
isbn = {9780195188165},
keywords = {ICTP Trieste Italy (print shop{\_}F)},
pages = {27--86},
pmid = {9780195188165},
publisher = {Oxford University Press},
title = {{The network structure of food webs}},
year = {2006}
}
@article{Kuchaiev2010,
abstract = {Sequence comparison and alignment has had an enormous impact on our understanding of evolution, biology and disease. Comparison and alignment of biological networks will probably have a similar impact. Existing network alignments use information external to the networks, such as sequence, because no good algorithm for purely topological alignment has yet been devised. In this paper, we present a novel algorithm based solely on network topology, that can be used to align any two networks. We apply it to biological networks to produce by far the most complete topological alignments of biological networks to date. We demonstrate that both species phylogeny and detailed biological function of individual proteins can be extracted from our alignments. Topology-based alignments have the potential to provide a completely new, independent source of phylogenetic information. Our alignment of the protein-protein interaction networks of two very different species-yeast and human-indicate that even distant species share a surprising amount of network topology, suggesting broad similarities in internal cellular wiring across all life on Earth.},
archivePrefix = {arXiv},
arxivId = {0810.3280},
author = {Kuchaiev, Oleksii and Milenkovic, Tijana and Memisevic, Vesna and Hayes, Wayne and Przulj, Natasa},
doi = {10.1098/rsif.2010.0063},
eprint = {0810.3280},
isbn = {1742-5662},
issn = {1742-5662},
journal = {Journal of the Royal Society, Interface / the Royal Society},
keywords = {Algorithms,Biological,Computational Biology,Humans,Models,Phylogeny,Proteins,Proteins: chemistry,Proteins: genetics,Proteins: metabolism,Saccharomyces cerevisiae,Saccharomyces cerevisiae Proteins,Saccharomyces cerevisiae Proteins: chemistry,Saccharomyces cerevisiae Proteins: genetics,Saccharomyces cerevisiae Proteins: metabolism,Saccharomyces cerevisiae: genetics,Saccharomyces cerevisiae: metabolism},
month = {sep},
number = {50},
pages = {1341--1354},
pmid = {20236959},
title = {{Topological network alignment uncovers biological function and phylogeny.}},
url = {http://rsif.royalsocietypublishing.org/content/early/2010/03/24/rsif.2010.0063{\%}5Cnhttp://rsif.royalsocietypublishing.org/cgi/doi/10.1098/rsif.2010.0063},
volume = {7},
year = {2010}
}
@article{Mouquet2012,
abstract = {Ecophylogenetics can be viewed as an emerging fusion of ecology, biogeography and macroevolution. This new and fast-growing field is promoting the incorporation of evolution and historical contingencies into the ecological research agenda through the widespread use of phylogenetic data. Including phylogeny into ecological thinking represents an opportunity for biologists from different fields to collaborate and has provided promising avenues of research in both theoretical and empirical ecology, towards a better understanding of the assembly of communities, the functioning of ecosystems and their responses to environmental changes. The time is ripe to assess critically the extent to which the integration of phylogeny into these different fields of ecology has delivered on its promise. Here we review how phylogenetic information has been used to identify better the key components of species interactions with their biotic and abiotic environments, to determine the relationships between diversity and ecosystem functioning and ultimately to establish good management practices to protect overall biodiversity in the face of global change. We evaluate the relevance of information provided by phylogenies to ecologists, highlighting current potential weaknesses and needs for future developments. We suggest that despite the strong progress that has been made, a consistent unified framework is still missing to link local ecological dynamics to macroevolution. This is a necessary step in order to interpret observed phylogenetic patterns in a wider ecological context. Beyond the fundamental question of how evolutionary history contributes to shape communities, ecophylogenetics will help ecology to become a better integrative and predictive science.},
author = {Mouquet, Nicolas and Devictor, Vincent and Meynard, Christine N. and Munoz, Francois and Bersier, Louis F{\'{e}}lix and Chave, J{\'{e}}r{\^{o}}me and Couteron, Pierre and Dalecky, Ambroise and Fontaine, Colin and Gravel, Dominique and Hardy, Olivier J. and Jabot, Franck and Lavergne, S{\'{e}}bastien and Leibold, Mathew and Mouillot, David and M{\"{u}}nkem{\"{u}}ller, Tamara and Pavoine, Sandrine and Prinzing, Andreas and Rodrigues, Ana S L and Rohr, Rudolf P. and Th{\'{e}}bault, Elisa and Thuiller, Wilfried},
doi = {10.1111/j.1469-185X.2012.00224.x},
isbn = {1464-7931},
issn = {14647931},
journal = {Biological Reviews},
keywords = {Community ecology,Conservation biology,Ecological networks,Ecophylogenetics,Ecosystem functioning,Evolution,Phylogenetics},
month = {nov},
number = {4},
pages = {769--785},
pmid = {22432924},
title = {{Ecophylogenetics: Advances and perspectives}},
url = {http://www.ncbi.nlm.nih.gov/pubmed/22432924},
volume = {87},
year = {2012}
}
@article{Macarthur2011,
abstract = {See full-text article at JSTOR},
author = {MacArthur, Robert},
doi = {10.2307/1929601},
isbn = {0012-9658},
issn = {00129658},
journal = {Ecology},
number = {3},
pages = {533},
pmid = {566},
title = {{Fluctuations of Animal Populations and a Measure of Community Stability}},
url = {http://www.esajournals.org/doi/abs/10.2307/1929601},
volume = {36},
year = {1955}
}
@article{Lancho-Barrantes2012,
abstract = {La colaboraci{\'{o}}n internacional mejora impacto de las citas. Colaborar con un pa{\'{i}}s incrementa las citas que recibe de ella. Sin embargo, algunos pa{\'{i}}ses colaboradores ofrecen mayores incrementos en este sentido que otros, y as{\'{i}} mismo algunos pa{\'{i}}ses reciben mayores incrementos de sus pa{\'{i}}ses socios que otros. Se ha observado una cierta tendencia a que estos incrementos sean m{\'{a}}s bajos en los pa{\'{i}}ses con mayores impactos. Adem{\'{a}}s, todos los pa{\'{i}}ses estudiados tuvieron un impacto mayor en el hogar como resultado de colaborar, a pesar de este incremento fue menor que la obtenida de otros pa{\'{i}}ses. Por {\'{u}}ltimo, hab{\'{i}}a diferencias en el comportamiento de los pa{\'{i}}ses entre las diversas disciplinas cient{\'{i}}ficas, y los efectos son mayores en Ciencias Sociales, seguido de Ingenier{\'{i}}a},
author = {Lancho-Barrantes, B{\'{a}}rbara S. and Guerrero-Bote, Vicente P. and de Moya-Aneg{\'{o}}n, F{\'{e}}lix},
doi = {10.1007/s11192-012-0797-3},
isbn = {10.1007/s11192-012-0797-3},
issn = {01389130},
journal = {Scientometrics},
keywords = {Citation analysis,Citation increment,Scientific collaboration in subject areas,Scientometrics},
month = {jul},
number = {3},
pages = {817--831},
title = {{Citation increments between collaborating countries}},
url = {http://link.springer.com/10.1007/s11192-012-0797-3},
volume = {94},
year = {2013}
}
@article{McCann1998,
abstract = {Abstract Ecological models show that complexity usually destabilizes food webs 1, 2, predicting that food webs should not amass the large numbers of interacting species that are in fact found in nature 3, 4, 5. Here, using nonlinear models, we study the influence of ... $\backslash$n},
author = {McCann, Kevin and Hastings, Alan and Huxel, Gary R},
doi = {10.1038/27427},
isbn = {0028-0836},
issn = {0028-0836},
journal = {Nature},
number = {6704},
pages = {794--798},
pmid = {17434536},
title = {{Weak trophic interactions and the balance of nature}},
url = {http://www.nature.com/doifinder/10.1038/27427{\%}5Cnpapers3://publication/doi/10.1038/27427},
volume = {395},
year = {1998}
}
@article{Costa2008,
abstract = {The niche expansion and niche variation hypotheses predict that release from interspecific competition will promote niche expansion in depauperate assemblages. Niche expansion can occur by different mechanisms, including an increase in within-individual, among-individual, or bimodal variation (sexual dimorphism). Here we explore whether populations with larger niche breadth have a higher degree of diet variation. We also test whether populations from depauperate lizard assemblages differ in dietary resource use with respect to variation within and/or among individuals and sexual dimorphism. We found support for the niche expansion and niche variation hypotheses. Populations in assemblages with low phylogenetic diversity had a higher degree of individual variation, suggesting a tendency for niche expansion. We also found evidence suggesting that the mechanism causing niche expansion is an increase in variation among individuals rather than an increase in within-individual variation or an increase in bimodal variation due to sexual dimorphism.},
author = {Costa, G.C. Gabriel C and Mesquita, D.O. Daniel O and Colli, Guarino R G.R. Guarino R and Vitt, L.J. Laurie J},
doi = {10.1086/592998},
isbn = {0003-0147},
issn = {1537-5323},
journal = {The American naturalist},
keywords = {Adaptation,Animals,Biological,Biological: physiology,Brazil,Diet,Ecosystem,Lizards,Lizards: physiology,Models,Regression Analysis,Sex Characteristics,Species Specificity,Theoretical,cerrado,corresponding author,costagc,e-mail,edu,individual specialization,islands,lizards,niche,niche width,ou,variation hypothesis},
month = {dec},
number = {6},
pages = {868--77},
pmid = {18950275},
title = {{Niche expansion and the niche variation hypothesis: does the degree of individual variation increase in depauperate assemblages?}},
url = {http://www.ncbi.nlm.nih.gov/pubmed/18950275{\%}5Cnhttp://www.jstor.org/stable/20491468},
volume = {172},
year = {2008}
}
@article{Dunne2004,
abstract = {Previous studies suggest that food-web theory has yet to account for major differencesin food-web properties of marine versus other types of ecosystems. We examined this issue by analyzingthe network structure of food webs for the Northeast US Shelf, a Caribbean reef, andBenguela, off South Africa. The values of connectance (links per species2), link density (links per species),mean chain length, and fractions of intermediate, omnivorous, and cannibalistic taxa of thesemarine webs are somewhat high but still within the ranges observed in other webs. We further comparedthe marine webs by using the empirically corroborated `niche model' that accounts forobserved variation in diversity (taxon number) and complexity (connectance). Our results substantiatepreviously reported results for estuarine, fresh-water, and terrestrial datasets, which suggeststhat food webs from different types of ecosystems with variable diversity and complexity share fundamentalstructural and ordering characteristics. Analyses of potential secondary extinctions resultingfrom species loss show that the structural robustness of marine food webs is also consistent withtrends from other food webs. As expected, given their relatively high connectance, marine food websappear fairly robust to loss of most-connected taxa as well as random taxa. Still, the short averagepath length between marine taxa (1.6 links) suggests that effects from perturbations, such as overfishing,can be transmitted more widely throughout marine ecosystems than previously appreciated.},
author = {Dunne, Jennifer A. and Williams, Richard J. and Martinez, Neo D.},
doi = {10.3354/meps273291},
isbn = {0171-8630},
issn = {01718630},
journal = {Marine Ecology Progress Series},
keywords = {Biodiversity loss,Connectance,Food webs,Marine ecosystems,Network structure,Niche model,Robustness},
pages = {291--302},
pmid = {222691100025},
title = {{Network structure and robustness of marine food webs}},
volume = {273},
year = {2004}
}
@article{May1972,
abstract = {May (1972, 1973) and Hastings (1982a, b, 1983a, b) announced criteria for the probable stability or instability, as n ??? ???, of systems of n linear ordinary differential equations or difference equations with random coefficients fixed in time. However, simple, explicit counter-examples show that, without some additional conditions, the claims of May and Hastings can be false. ?? 1985 Academic Press Inc. (London) Ltd All rights reserved.},
archivePrefix = {arXiv},
arxivId = {10.1038/238413a0},
author = {May, Robert M},
doi = {10.1016/S0022-5193(85)80081-3},
eprint = {238413a0},
isbn = {0028-0836},
issn = {10958541},
journal = {Nature},
keywords = {connectance,simulation,stability},
pages = {413--414},
pmid = {4559589},
primaryClass = {10.1038},
title = {{Will a large complex system be stable?}},
url = {http://adsabs.harvard.edu/abs/1972Natur.238..413M},
volume = {238},
year = {1972}
}
@article{Roman2006,
abstract = {Most invasion histories include an estimated arrival time, followed by range expansion. Yet, such linear progression may not tell the entire story. The European green crab (Carcinus maenas) was first recorded in the US in 1817, followed by an episodic expansion of range to the north. Its population has recently exploded in the Canadian Maritimes. Although it has been suggested that this northern expansion is the result of warming sea temperatures or cold-water adaptation, Canadian populations have higher genetic diversity than southern populations, indicating that multiple introductions have occurred in the Maritimes since the 1980s. These new genetic lineages, probably from the northern end of the green crab's native range in Europe, persist in areas that were once thought to be too cold for the original southern invasion front. It is well established that ballast water can contain a wide array of nonindigenous species. Ballast discharge can also deliver genetic variation on a level comparable to that of native populations. Such gene flow not only increases the likelihood of persistence of invasive species, but it can also rapidly expand the range of long-established nonindigenous species.},
author = {Roman, Joe},
doi = {10.1098/rspb.2006.3597},
isbn = {0962-8452 (Print)$\backslash$n0962-8452 (Linking)},
issn = {0962-8452},
journal = {Proceedings of the Royal Society B: Biological Sciences},
keywords = {carcinus maenas,cline,cryptic invasion,mitochondrial dna,planktonic dispersal},
month = {oct},
number = {1600},
pages = {2453--2459},
pmid = {16959635},
title = {{Diluting the founder effect: cryptic invasions expand a marine invader's range.}},
url = {http://www.pubmedcentral.nih.gov/articlerender.fcgi?artid=1634897{\&}tool=pmcentrez{\&}rendertype=abstract},
volume = {273},
year = {2006}
}
@article{Slatkin1974,
abstract = {A model of the competition between two species is developed which is based on the work of Cohen (1970) and Levins and Culver (1971) and which considers the effect of competition on the colonization and extinction rates of the two species. The results are that in some cases it is possible for one species to exclude another species from a geographic region, but there is no possibility of a @'priority effect@' where the first species in the region can always exclude the other. Thus the equilibrium level of each species is determined by the parameters of the system and not by the initial conditions. Also, it is possible for two similar species to coexist in a region. A predator can increase the extinction rate of each species and, in some cases, permit coexistence where it would otherwise not be possible.},
author = {Slatkin, Montgomery},
doi = {10.2307/1934625},
isbn = {00129658},
issn = {00129658},
journal = {Ecology},
keywords = {colonization,competition,extinction,geographic strutctutre},
number = {1},
pages = {128--134},
title = {{Competition and Regional Coexistence}},
url = {http://www.jstor.org/stable/1934625},
volume = {55},
year = {1974}
}
@article{Hedrick2001,
abstract = {The Tiburon Island population of desert bighorn sheep has increased in size from 20 founders in 1975 to approximately 650 in 1999. This population is now the only population being used as the source stock for transplantations throughout northern Mexico. To evaluate the genetic variation in this population, we examined 10 microsatellite loci and a major histocompatibility complex (MHC) locus. The genetic variation was significantly less than found in other populations of the same subspecies in Arizona. Using a model that takes into account the effects of genetic drift on genetic distance, most of the genetic distance observed between the Tiburon population and Arizona samples could be explained. Because of the low genetic variation found in the Tiburon population, it is suggested that the Tiburon population should be supplemented with additional unrelated animals or that the transplant populations should be supplemented with unrelated animals.},
author = {Hedrick, P. W. and Gutierrez-Espeleta, G. a. and Lee, R. N.},
doi = {10.1046/j.1365-294X.2001.01243.x},
isbn = {0962-1083},
issn = {09621083},
journal = {Molecular Ecology},
keywords = {Founder effect,Genetic distance,Genetic drift,Major histocompatibiliy complex,Microsatellite loci},
month = {apr},
number = {4},
pages = {851--857},
pmid = {11348494},
title = {{Founder effect in an island population of bighorn sheep}},
url = {http://www.ncbi.nlm.nih.gov/pubmed/11348494},
volume = {10},
year = {2001}
}
@article{Volkov,
abstract = {Biomineralization in the marine phytoplankton Emiliania huxleyi is a stringently controlled intracellular process. The molecular basis of coccolith production is still relatively unknown although its importance in global biogeochemical cycles and varying sensitivity to increased pCO2 levels has been well documented. This study looks into the role of several candidate Ca2+, H+ and inorganic carbon transport genes in E. huxleyi, using quantitative reverse transcriptase PCR. Differential gene expression analysis was investigated in two isogenic pairs of calcifying and non-calcifying strains of E. huxleyi and cultures grown at various Ca2+ concentrations to alter calcite production. We show that calcification correlated to the consistent upregulation of a putative HCO3- transporter belonging to the solute carrier 4 (SLC4) family, a Ca2+/H+ exchanger belonging to the CAX family of exchangers and a vacuolar H+-ATPase. We also show that the coccolith-associated protein, GPA is downregulated in calcifying cells. The data provide strong evidence that these genes play key roles in E. huxleyi biomineralization. Based on the gene expression data and the current literature a working model for biomineralization-related ion transport in coccolithophores is presented.},
archivePrefix = {arXiv},
arxivId = {q-bio/0504018},
author = {MacKinder, Luke and Wheeler, Glen and Schroeder, Declan and von Dassow, Peter and Riebesell, Ulf and Brownlee, Colin},
doi = {10.1111/j.1462-2920.2011.02561.x},
eprint = {0504018},
isbn = {1462-2920},
issn = {14622912},
journal = {Environmental Microbiology},
number = {12},
pages = {3250--3265},
pmid = {21902794},
primaryClass = {q-bio},
title = {{Expression of biomineralization-related ion transport genes in Emiliania huxleyi}},
url = {http://www.nature.com/nature/journal/v424/n6952/abs/nature01883.html},
volume = {13},
year = {2011}
}
@article{Condit2002,
abstract = {The high alpha-diversity of tropical forests has been amply documented, but beta-diversity—how species composition changes with distance—has seldom been studied. We present quantitative estimates of beta-diversity for tropical trees by comparing species composition of plots in lowland terra firme forest in Panama, Ecuador, and Peru. We compare observations with predictions derived from a neutral model in which habitat is uniform and only dispersal and speciation influence species turnover. We find that beta-diversity is higher in Panama than in western Amazonia and that patterns in both areas are inconsistent with the neutral model. In Panama, habitat variation appears to increase species turnover relative to Amazonia, where unexpectedly low turnover over great distances suggests that population densities of some species are bounded by as yet unidentified processes. At intermediate scales in both regions, observations can be matched by theory, suggesting that dispersal limitation, with speciation, influences species turnover},
author = {Condit, Richard},
doi = {10.1126/science.1066854},
isbn = {0036-8075},
issn = {00368075},
journal = {Science},
keywords = {DENSITIES,DENSITY,DISTANCE,LIMITATION,LOWLAND,PATTERN,PATTERNS,POPULATION,POPULATION-DENSITY,SCALE,TREE,TREES,TROPICAL FOREST,TROPICAL FORESTS,TROPICAL TREE,dispersal,forest,population density,species composition},
month = {jan},
number = {5555},
pages = {666--669},
pmid = {11809969},
title = {{Beta-Diversity in Tropical Forest Trees}},
url = {http://www.sciencemag.org/content/295/5555/666.abstract{\%}5Cnhttp://www.sciencemag.org/cgi/doi/10.1126/science.1066854},
volume = {295},
year = {2002}
}
@article{Hubbell2005,
abstract = {Probably no ecologist in the world with even a modicum of field experience would seriously question the exist- ence of niche differences among competing species on the same trophic level. The real question, however, is how did these niche differences evolve, how are they maintained ecologically, and what niche differences, if any, matter to the assembly of ecological communities? By ecological community I refer to co-occurring assem- blages of trophically similar species. By assembly I mean which species, having which niche traits, and how many species, co-occur in a given community. In my judgement, despite a long and rich tradition of research on these questions in community ecology (Chase {\{}{\&}{\}} Leibold 2003), we are still far from having answers.},
author = {Hubbell, Stephen P.},
doi = {10.1111/j.0269-8463.2005.00965.x},
isbn = {1365-2435},
issn = {02698463},
journal = {Functional Ecology},
number = {1},
pages = {166--172},
pmid = {227672000022},
title = {{Neutral theory in community ecology and the hypothesis of functional equivalence}},
url = {http://onlinelibrary.wiley.com/doi/10.1111/j.0269-8463.2005.00965.x/full},
volume = {19},
year = {2005}
}
@article{Hubbell1979,
abstract = {Patterns of tree abundance and dispersion in a tropical deciduous (dry) forest are summarized. The generalization that tropical trees have spaced adults did not hold. All species were either clumped or randomly dispersed, with rare species more clumped than common species. Breeding system was unrelated to species abundance or dispersion, but clumping was related to mode of seed dispersal. Juvenile densities decreased approximately exponentially away from adults. Rare species gave evidence of poor reproductive performance compared with their performance when common in nearby forests. Patterns of relative species abundance in the dry forest are compared with patterns in other forests, and are explained by a simple stochastic model based on random-walk immigration and extinction set in motion by periodic community disturbance.},
author = {Hubbell, Sp},
doi = {10.1126/science.203.4387.1299},
isbn = {8028675581},
issn = {0036-8075},
journal = {Science},
month = {mar},
number = {4387},
pages = {1299--1309},
pmid = {17780463},
title = {{Tree dispersion, abundance, and diversity in a tropical dry forest}},
url = {http://www.planta.cn/forum/files{\_}planta/tree{\_}dispersion{\_}abundance{\_}and{\_}diversity{\_}in{\_}a{\_}tropical{\_}dry{\_}forest{\_}121.pdf},
volume = {203},
year = {1979}
}
@article{Lawton1999,
abstract = {The dictionary definition of a law is: "Generalized formulation based on a series of events or processes observed to recur regularly under certain conditions; a widely observable tendency". I argue that ecology has numerous laws in this sense of the word, in the form of widespread, repeatable patterns in nature, but hardly any laws that are universally true. Typically, in other words, ecological patterns and the laws, rules and mechanisms that underpin them are contingent on the organisms involved, and their environment. This contingency is manageable at a relatively simple level of ecological organisation (for example the population dynamics of single and small numbers of species), and re-emerges also in a manageable form in large sets of species, over large spatial scales, or over long time periods, in the form of detail-free statistical patterns - recently called 'macroecology'. The contingency becomes over- whelmingly complicated at intermediate scales, characteristic of community ecology, where there are a large number of case histories, and very little other than weak, fuzzy generalizations. These arguments are illustrated by focusing on examples of typical studies in community ecology, and by way of contrast, on the macroecological relationship that emerges between local species richness and the size of the regional pool of species. The emergent pattern illustrated by local vs regional richness plots is extremely simple, despite the vast number of contingent processes and interactions involved in its generation. To discover general patterns, laws and rules in nature, ecology may need to pay less attention to the 'middle ground' of community ecology, relying less on reductionism and experimental manipulation, but increasing research efforts into macroecolo},
author = {Lawton, John H.},
doi = {10.1017/S0031182006002150},
isbn = {0031-1820},
issn = {0031-1820},
journal = {Oikos},
keywords = {aggregation,biomass,contingency,interaction networks,macroecology,metabolism,scale,species},
number = {2},
pages = {177--192},
pmid = {17234043},
title = {{Are there general laws in parasite ecology?}},
url = {http://www.jstor.org/stable/10.2307/3546712},
volume = {84},
year = {1999}
}
@article{Simberloff2004,
abstract = {Because of the contingency and complexity of its subject matter, community ecology has few general laws. Laws and models in community ecology are highly contingent, and their domain is usually very local. This fact does not mean that community ecology is a weak science; in fact, it is the locus of exciting advances, with growing mechanistic understanding of causes, patterns, and processes. Further, traditional community ecological research, often local, experimental, and reductionist, is crucial in understanding and responding to many environmental problems, including those posed by global changes. For both scientific and societal reasons, it is not time to abandon community ecology.},
author = {Simberloff, D},
doi = {10.1086/420777},
isbn = {0003-0147},
issn = {1537-5323},
journal = {American Naturalist},
keywords = {community ecology,general,general laws,introduced species,kelp,longleaf pine,of general laws,red-cockaded woodpecker,science and the importance,the first issue of,whether community ecology has},
month = {jun},
number = {6},
pages = {787--799},
pmid = {15266378},
title = {{Community ecology: Is it time to move on?}},
volume = {163},
year = {2004}
}
@article{Poulsen2002,
abstract = {Arboreal frugivores, such as primates and hornbills, are important seed dispersers for many tropical plant species, yet the degree to which they use the same resources is unknown. If primates and hornbills consume the same fruit species, they may be redundant in their roles as seed dispersers, and the loss of one of these taxa may be compensated for by the other. To examine resource use by tropical frugivores, we quantified the feeding habits of two hornbill species, Ceratogymna atrata and C. cylindricus, and five primate species, Colobus guereza, Lophocebus albigena, Cercopithecus pogonias, C. cephus, and C. nictitans, in the lowland rainforest of south-central Cameroon. Based on over 2200 feeding observations recorded between January and December 1998, we characterized the diets and estimated dietary overlap among frugivore species. Previous studies have calculated dietary overlap by counting the number of diet species that two animals share, often leading to inflated estimates of overlap. Our method incorporated the proportional use of diet species and fruit availability into randomization procedures, allowing a clearer assessment of the actual degree of overlap. This added complexity of analysis revealed that, although the diets of a hornbill and a primate species may have as many as 36 plant species in common, actual dietary overlap is low. These results suggested that there are distinct hornbill and primate feeding assemblages, with primates consuming a greater diversity of plant species and higDifferential resources use by primatesher levels of nonfruit items like leaves and seeds. Using Correspondence Analysis, we also identified two primate assemblages, separated largely by degree of frugivory and folivory. In addition, we found that hornbills feed at significantly higher strata in the forest canopy and eat fruits of different colors than primates. Averaged across the year, overlap between groups (hornbill–primate) was significantly lower than combined within-group overlap (primate–primate and hornbill–hornbill), showing that primates and hornbills have dissimilar diets and are not redundant as seed dispersers. In equatorial Africa, primate populations face greater declines than hornbill populations because of hunting. It is unlikely that seed dispersal by hornbills will compensate for the loss of primates in maintaining},
author = {Poulsen, John R. and Clark, Connie J. and Connor, Edward F. and Smith, Thomas B.},
doi = {10.1890/0012-9658(2002)083[0228:DRUBPA]2.0.CO;2},
isbn = {0012-9658},
issn = {00129658},
journal = {Ecology},
keywords = {Cameroon,Ceratogymna,Cercopithecus,Colobus,Dietary overlap,Frugivory,Hornbill,Lophocebus,Monte Carlo methods,Primate,Seed dispersal},
number = {1},
pages = {228--240},
pmid = {173117700020},
title = {{Differential resource use by primates and hornbills: Implications for seed dispersal}},
url = {http://www.esajournals.org/doi/full/10.1890/0012-9658(2002)083{\%}5B0228:DRUBPA{\%}5D2.0.CO;2},
volume = {83},
year = {2002}
}
@article{Tylianakis2010,
abstract = {Recent work has shown that antagonist (e.g. predator-prey food web) and mutualist (e.g. pollinator-plant) network structure can be altered by global environmental change drivers, and that these alterations may have important ecosystem-level consequences. This has prompted calls for the conservation of network structure, but precisely which attributes of webs should be conserved remains unclear. Further, the extent to which network metrics characterise the spatiotemporally-variable dynamic structure of interacting communities is unknown. Here, we summarise the attributes of web structure that are predicted to confer stability or increased function to a system, as these may be of greatest interest to conservation biologists. However, empirical evaluation of these effects is lacking in most cases, and we discuss whether stability is even desirable in all contexts. The incorporation of web attributes into conservation monitoring requires that changes in these attributes can be recorded (sampled) with relative ease. We contrast the sensitivity of metrics to sampling effort, and highlight those (such as nestedness and connectance) that could easily be incorporated into conservation monitoring. Despite our growing understanding of the characteristics of food webs that confer stability and function, numerous practical challenges need to be overcome before the goal of conserving species interaction networks can be achieved. {\textcopyright} 2009 Elsevier Ltd.},
author = {Tylianakis, Jason M. and Lalibert{\'{e}}, Etienne and Nielsen, Anders and Bascompte, Jordi},
doi = {10.1016/j.biocon.2009.12.004},
file = {:Users/alyssacirtwill/Documents/Papers/Tylianakis et al.{\_}2010{\_}Biological Conservation.pdf:pdf},
isbn = {0006-3207},
issn = {00063207},
journal = {Biological Conservation},
keywords = {Climate change,Food web,Global change,Global warming,Invasion,Land use change,Mutualism,Parasitoid,Pollination},
month = {oct},
number = {10},
pages = {2270--2279},
pmid = {19860667},
publisher = {Elsevier Ltd},
title = {{Conservation of species interaction networks}},
url = {http://linkinghub.elsevier.com/retrieve/pii/S0006320709005126},
volume = {143},
year = {2010}
}
@article{Tylianakis2008a,
abstract = {Interactions in food webs indicate the structure and stability of ecosystems. Now, new research uses these interactions to illustrate the vulnerability of pollination webs to invasive plants and pollinators.},
author = {Tylianakis, Jason M.},
doi = {10.1371/journal.pbio.0060047},
isbn = {1544-9173},
issn = {15449173},
journal = {PLoS Biology},
keywords = {Animal,Animals,Ecology,Ecosystem,Food Chain,Sexual Behavior,Species Specificity},
month = {feb},
number = {2},
pages = {0224--0228},
pmid = {18303951},
title = {{Understanding the web of life: the birds, the bees, and sex with aliens}},
url = {http://www.pubmedcentral.nih.gov/articlerender.fcgi?artid=2253639{\&}tool=pmcentrez{\&}rendertype=abstract},
volume = {6},
year = {2008}
}
@article{Tylianakis2008,
abstract = {The main drivers of global environmental change (CO 2 enrichment, nitrogen deposition, climate, biotic invasions and land use) cause extinctions and alter species distributions, and recent evidence shows that they exert pervasive impacts on various antagonistic and mutualistic interactions among species. In this review, we synthesize data from 688 published studies to show that these drivers often alter competitive interactions among plants and animals, exert multitrophic effects on the decomposer food web, increase intensity of pathogen infection, weaken mutualisms involving plants, and enhance herbivory while having variable effects on predation. A recurrent finding is that there is substantial variability among studies in both the magnitude and direction of effects of any given GEC driver on any given type of biotic interaction. Further, we show that higher order effects among multiple drivers acting simultaneously create challenges in predicting future responses to global environmental change, and that extrapolating these complex impacts across entire networks of species interactions yields unanticipated effects on ecosystems. Finally, we conclude that in order to reliably predict the effects of GEC on community and ecosystem processes, the greatest single challenge will be to determine how biotic and abiotic context alters the direction and magnitude of GEC effects on biotic interactions.},
author = {Tylianakis, Jason M. and Didham, Raphael K. and Bascompte, Jordi and Wardle, David a.},
doi = {10.1111/j.1461-0248.2008.01250.x},
isbn = {1461-023X},
issn = {1461023X},
journal = {Ecology Letters},
keywords = {CO2,Climate change,Competition,Disease,Food web,Global warming,Interaction effect,Land-use change,Mycorrhiza,Nitrogen deposition,Parasite,Pollination,Seed dispersal},
month = {dec},
number = {12},
pages = {1351--1363},
pmid = {19062363},
title = {{Global change and species interactions in terrestrial ecosystems}},
url = {http://doi.wiley.com/10.1111/j.1461-0248.2008.01250.x},
volume = {11},
year = {2008}
}
@article{Cain1938,
author = {Cain, SA},
journal = {American Midland Naturalist},
number = {3},
pages = {573--581},
title = {{The species-area curve}},
url = {http://www.jstor.org/stable/10.2307/2420468},
volume = {19},
year = {1938}
}
@article{Vargas2012,
abstract = {Since nobody has witnessed the arrival of early plant colonists on isolated islands, the actual long-distance dispersal (hereafter LDD) has historically been a matter of speculation. In the present study, we offer a new approach that evaluates whether particular syndromes for LDD (i.e. the set of traits related to diaspore dispersal by animals, wind and sea currents) have been favourable in the natural colonization of the Gal{\'{a}}pagos Islands by plants. Dispersal syndromes of the 251 native genera (509 angiosperm species) presently acknowledged as native were carefully studied, combining data from floristic lists of the Gal{\'{a}}pagos Islands, diaspore traits, characteristics of continental relatives and our own observations. We used these genera (and occasionally infrageneric groups) as the working units to infer the number of introductions and colonists. A final number of native plants was inferred and analysed after correcting by pollen records of six species from six genera previously considered exotic (palaeobotanical correction). The number of early colonists was also corrected by incorporating information from the few (n= 12) phylogenetic studies of genera from both the Gal{\'{a}}pagos Islands and the Americas (phylogenetic correction). A total of 372 colonization events were inferred for the native flora using the latest check-list. The proportions of native colonists grouped into five categories were: endozoochory 16.4{\%}, epizoochory 15.7{\%}, hydrochory 18.6{\%}, anemochory 13.3{\%}, and unassisted diaspores 36.0{\%}. These figures did not vary significantly on analysing only the 99 genera that include endemic species in order to rule out any human-mediated introductions. Irrespective of the roles of the different agents involved in LDD, diaspores with no special syndrome for LDD (unassisted diapores), such as many dry fruits, have been successful in reaching and colonizing the Gal{\'{a}}pagos archipelago. This finding leads us to suggest that unpredictable and so far unknown LDD mechanisms should be further considered in the theory of island biogeography.},
author = {Vargas, Pablo and Heleno, R. and Traveset, a. and Nogales, M.},
doi = {10.1111/j.1600-0587.2011.06980.x},
isbn = {1600-0587},
issn = {09067590},
journal = {Ecography},
month = {jan},
number = {1},
pages = {33--43},
title = {{Colonization of the Gal??pagos Islands by plants with no specific syndromes for long-distance dispersal: A new perspective}},
url = {http://doi.wiley.com/10.1111/j.1600-0587.2011.06980.x},
volume = {35},
year = {2012}
}
@article{Edgar1990,
abstract = {Seagrass plants with associated macrofauna were confined within transparent Perspex tubes in the field at two Western Australian sites, Cliff Head and Seven Mile Beach, and faunal population changes in the microcosms monitored and compared with changes in the field. Three microcosm treatments were set up at Cliff Head: (i) "dark" microcosms, in which primary production was reduced by wrapping the tubes in black plastic, (ii) "faunal reduction" microcosms, in which the food resources available to each animal were enhanced by removing 80{\%} of the fauna, and (iii) "control" microcosms. Faunal abundances in control microcosms remained relatively constant, while the population numbers of nearly half of the common epifaunal species placed in dark microcosms declined relative to numbers in the control treatments. The abundances of almost all species in the faunal reduction microcosms rapidly increased, with proportionately more crustaceans carrying eggs in faunal reduction microcosms than in the other treatments. These results are consistent with the hypothesis that epifaunal populations are food limited. Other studies have shown that the diets of different epifaunal species are broadly overlapping, hence diffuse exploitative competition is probably a major structuring agent amongst the seagrass epifauna. The populations of almost all epifaunal species fluctuated greatly with season in the field. These annual population cycles did not, however, generally correspond closely between the Cliff Head and Seven Mile Beach sites. Therefore, no single factor, including those implicated to be important in other studies (such as temperature, photoperiod, seagrass biomass and epiphytic biomass), had a controlling influence on the population dynamics of the majority of epifaunal species. The annual population cycles of almost all species at Cliff Head nevertheless coincided with highest faunal densities occurring in late summer or early autumn, followed by a rapid decline to very low densities in winter. Animal populations inside microcosms did not change greatly at the time of rapid declines in field populations. A large-scale autumn emigration of mobile crustaceans from the Cliff Head site in response to low levels of microalgal food and/or dissolved oxygen is postulated to have occurred. ?? 1990.},
author = {Edgar, Graham J.},
doi = {10.1016/0022-0981(90)90029-C},
isbn = {0022-0981},
issn = {00220981},
journal = {Journal of Experimental Marine Biology and Ecology},
keywords = {Competition,Epifauna,Macrofauna,Population dynamics,Population regulation,Seagrass},
month = {dec},
number = {2-3},
pages = {205--234},
pmid = {23},
title = {{Population regulation, population dynamics and competition amongst mobile epifauna associated with seagrass}},
url = {http://linkinghub.elsevier.com/retrieve/pii/002209819090029C},
volume = {144},
year = {1990}
}
@article{Smith2012,
abstract = {The Tropical Niche Conservatism hypothesis is a leading explanation for why biodiversity increases towards the equator. The model suggests that most lineages have tropical origins, with few dispersing into temperate regions. However, biotas are comprised of lineages with differing geographical origins, thus it is unclear whether lineages that originated on different continents will exhibit similar patterns of niche conservatism. Here, we summarised biogeographical patterns of New World vertebrates and compared species diversity patterns between families that originated in North and South America. Overall, families with southern origins exhibit niche conservatism with many lineages restricted to the Neotropics, whereas many northern-origin families are distributed across the Neotropics and the Nearctic. Consequently, northern lineages have contributed to high tropical biodiversity, but southern lineages have contributed relatively little to temperate biodiversity in North America. The asymmetry in niche conservatism between northern and southern lineages is an important contributor to the biodiversity gradient},
author = {Smith, Brian Tilston and Bryson, Robert W. and Houston, Derek D. and Klicka, John},
doi = {10.1111/j.1461-0248.2012.01855.x},
isbn = {1461-0248},
issn = {1461023X},
journal = {Ecology Letters},
keywords = {Biodiversity,Biogeography,Nearctic region,Neotropical region,Niche evolution,Tropical niche conservatism},
month = {nov},
number = {11},
pages = {1318--1325},
pmid = {22909289},
title = {{An asymmetry in niche conservatism contributes to the latitudinal species diversity gradient in New World vertebrates}},
url = {http://www.ncbi.nlm.nih.gov/pubmed/22909289},
volume = {15},
year = {2012}
}
@article{Heatwole1991,
abstract = {richness of ferns and flowering plant species were related to 7 geographic variables - the most important of which was insular size (area was also important) but distance was not (perhaps because of chain effect or not great enough distance).},
author = {Heatwole, H},
isbn = {03050270},
issn = {03050270},
journal = {Journal of Biogeography},
keywords = {species richness, endemism, biodiversity},
number = {2},
pages = {213--221},
title = {{Factors affecting the number of species of plants on islands of the Great Barrier Reef, Australia}},
url = {http://www.jstor.org/stable/10.2307/2845294},
volume = {18},
year = {1991}
}
@article{Kurle2011,
abstract = {Trophic cascades are extensively documented in nature, but they are also known to vary widely in strength and frequency across ecosystems. Therefore, much effort has gone into understanding which ecological factors generate variation in cascade strength. To identify which factors covary with the strength of cascades in streams, we performed a concurrent experiment across 17 streams throughout the Sierra Nevada Mountains. We eliminated top consumers from experimental substrates using electrical exclusions and compared the strength of indirect effects of consumers on the biomass of primary producers relative to control patches. In each stream we 1) classified the dominant invertebrate herbivores according to life-history traits that influence their susceptibility to predators, 2) determined the abundance and diversity of algae and herbivores, and 3) measured production-to-biomass ratios (P:B) of the stream biofilm. This allowed us to assess three common predictions about factors thought to influence the strength of trophic cascades: cascade strength 1) is weaker in systems dominated by herbivores with greater ability to evade or defend against predators, 2) is stronger in systems characterized by low species diversity, and 3) increases with increasing producer P:B. When averaged across all streams, the indirect effect of predators increased the biomass of periphyton by a mean 60{\%}. However, impacts of predators on algae varied widely, ranging from effects that exacerbated algal loss to herbivores, to strong cascades that increased algal biomass by 4.35 times. Cascade strength was not related to herbivore traits or species diversity, but decreased significantly with increasing algal diversity and biofilm P:B in a stream. Partial regression analyses suggested that the relationship between cascade strength and algal diversity was spurious, and that the only significant covariate after statistically controlling for cross-correlations was algal P: B. Our study contributes to the ongoing debate about why trophic cascade strength varies in nature and is useful because it eliminates factors that have no potential to explain variation in cascades within these stream ecosystems.},
archivePrefix = {arXiv},
arxivId = {arXiv:1011.1669v3},
author = {Kurle, Carolyn M. and Cardinale, Bradley J.},
doi = {10.1111/j.1600-0706.2011.19465.x},
eprint = {arXiv:1011.1669v3},
isbn = {0030-1299},
issn = {00301299},
journal = {Oikos},
month = {dec},
number = {12},
pages = {1897--1908},
pmid = {19137956},
title = {{Ecological factors associated with the strength of trophic cascades in streams}},
url = {http://doi.wiley.com/10.1111/j.1600-0706.2011.19465.x},
volume = {120},
year = {2011}
}
@article{Magnuson1998,
abstract = {To evaluate the roles of extinction and isolation in predicting richness and composition of fish assemblages in small forest lakes of Finland and Wisconsin, we analyzed data from 114 Finnish and 55 Wisconsin lakes 0.2-86.9 ha in area. Six isolation variables characterized properties of stream corridors, land barriers, and source pools of invading species; four extinction variables were related to habitat severity, lake area, and productivity. Two types of multivariate analyses were used: the nonparametric classification and regression trees (CART) and the parametric linear discriminant analysis (LDA). Both types of analyses showed that extinction variables were collectively more important than isolation variables in predicting richness and composition both in Finland and Wisconsin. We interpret that the greater importance of extinction vs. isolation results, not because isolation is unimportant, but because the probability of an arrival of a new species is much less than that of an extinction. Thus, the time after an extinction event before a subsequent invasion is long relative to the time after an invasion event before a subsequent extinction; consequently, fish assemblages sampled at a given point in time more likely represent the stamp of the extinctions than of the invasions. This conclusion was robust to the differences in the geomorphic settings and fish faunas of Finland and Wisconsin. However, the importance of individual isolation and extinction variables in determining richness and composition differed between the two regions, apparently more from differences in geomorphic settings than from differences in fish faunas. Influences of horizontal rather than vertical barriers over land and water were more apparent in Wisconsin, with its lower relief and higher incidence of lakes without stream connections; influences of the area of the nearest lake (representing the size of the available species pool) and stream gradient were more important in Finland, with its higher relief and higher incidence of lakes with stream connections. The importance of individual extinction variables also differed between the two regions, again reflecting differences in the geomorphic settings of the two lake districts and the strong influence that lake position in the landscape has in determining limnological features of the lake.},
author = {Magnuson, John J. and Tonn, William M. and Banerjee, Asit and Toivonen, Jorma and Sanchez, Oliva and Rask, Martti},
doi = {10.1890/0012-9658(1998)079[2941:iveita]2.0.co;2},
isbn = {0012-9658},
issn = {00129658},
journal = {Ecology},
keywords = {Assemblages, extinction, Finland, fish, insular en,Species composition,Species richness,Wisconsin},
number = {8},
pages = {2941--2956},
title = {{Isolation vs. extinction in the assembly of fishes in small northern lakes}},
url = {http://www.esajournals.org/doi/abs/10.1890/0012-9658(1998)079{\%}5B2941:IVEITA{\%}5D2.0.CO{\%}3B2},
volume = {79},
year = {1998}
}
@article{Kingsland2004,
abstract = {The roots of ecology are historically extremely diverse, with contributions from many fields of science. A sampling of ways of thinking ecologically, ranging from the early 19th to the early 20th century, reveals the richness of ecological science. By examining historical examples from biogeography, natural history, the science of energy, and biomedical sciences, we can appreciate the many different contexts in which ecological thinking has evolved, whether as part of larger projects to systematize and unify knowledge of the world, or in response to particular problems that were solved by taking a fresh approach. It is important, when educating students and the public, to convey this diversity of ecological thought and the nature of ecology as an integrative discipline.},
author = {Kingsland, Sharon},
doi = {10.1890/1540-9295(2004)002[0367:CTICOE]2.0.CO;2},
isbn = {1540-9295},
issn = {15409309},
journal = {Frontiers in Ecology and the Environment},
month = {sep},
number = {7},
pages = {367--374},
title = {{Conveying the intellectual challenge of ecology: An historical perspective}},
url = {http://www.jstor.org/stable/3868362?origin=crossref},
volume = {2},
year = {2004}
}
@article{Lynch2003,
author = {Lynch, S},
journal = {Manuscript Available At: Http://Www. Princeton. Edu/{\~{}} {\ldots}},
number = {March},
pages = {1--18},
title = {{Expanding the Model Capabilities : Dummy Variables , Interactions , and Nonlinear Transformations}},
url = {http://scholar.google.com/scholar?hl=en{\&}btnG=Search{\&}q=intitle:Expanding+the+Model+Capabilities+:+Dummy+Variables+,+Interactions+,+and+Nonlinear+Transformations{\#}0},
year = {2003}
}
@article{Sciences2002,
abstract = {Carl Linnaeus (1707-1778) was a leading naturalistof the 1700s (Lindroth 1973, 1983, Morton1981:259-276, 281-285, Goerke 1993, Broberg 2000,Spary 2002). All ecologists know he founded modernnomenclature and systematics (Larson 1971, Stafleu1971, Mayr 1982:171-180, Eriksson 1983), but he isless well known for inventing an ecological sciencehe called the economy of nature. He explained it in1749, but the overly broad science of natural history,which he had pursued since childhood, was alreadyecological in outlook and content. In 1749 he generalizedand formalized what he had been recordingspecifically and informally. A series of 186 essays,largely by Linnaeus, were defended by his studentsas dissertations for their doctoral degrees (Jackson1913, Ramsbottom 1959:151-153, Smit 1989:118-119, Kiger et al.1999:231), and one of these wasSpecimen academicum de oeconomia naturae (1749),defended by Isaac J. Biberg. Linnaeus republishedthese dissertations in 10 volumes entitled AmoenitatesAcademica (Academic Pleasures, 1749-1790),though the last two volumes appeared posthumously.The Amoenitates Academica has been reprinted severaltimes, and 19 dissertations are translated intoEnglish (Linnaeus 1775, 1781, 1977a, b). There is ahelpful Index to Scientific Names of Organisms citedin Linnaean Dissertations (Kiger et al. 1999), with aguide to collected editions. Linnaeus' earlier naturalhistory observations are recorded in travel books andother writings. All of his travel books and the dissertationsare listed in B. H. Soulsby's catalogue of Linnaeus'works (1933:23-26, 99-151). Florence Caddy(1886-1887) provides two good maps on Linnaeus'travels, though the caption to the one at the end ofvolume I is misdated 1735-1738 (read 1732-1738).},
author = {Egerton, Frank N.},
doi = {10.1890/0012-9623-91.1.21},
isbn = {0012-9623},
issn = {0012-9623},
journal = {Bulletin of the Ecological Society of America},
number = {1},
pages = {72--88},
title = {{A History of the ecological sciences, Part 23: Linnaeus and the economy of nature}},
volume = {88},
year = {2007}
}
@article{Kaufman1995,
abstract = {Thirty years of study to demonstrate, quantify, and explain the latitudinal gradient of species richness in mammals of the New World have produced two results; a surety that such a gradient exists and a lack of consensus as to what causes the inverse relationship between species richness and latitude. If the effects of continental area are removed, the latitudinal gradient remains strong and proves to be universal for the New World. This gradient in richness occurs not only at the species level, but also at a series of macro taxonomic levels (generic, familial, and ordinal), which represents a distribution of ecological types, or bau- plans. What causes these phenomena? I assert that none of the specific explanations that have been proposed (e.g., competition or spatial heterogeneity) alone can account for the latitudinal pattern. A universal explanation is likely to be more general; I propose that there is a shift in the impact of abiotic and biotic factors that limit species along the gradient from the poles to the equator. This shift, in tum, produces a change in factors that influence species richness from one that limits the number of species in polar regions to one that limits the physical or niche space of species in the tropics. Ultimately, these phenomena produce the latitudinal gradient (as well as the Rapoport's rule phenomena). Effects of abiotic and biotic factors are different for species and bauplans; species are impacted heavi- ly by both, whereas bauplans are influenced much more strongly by abiotic factors.},
author = {Kaufman, Dawn M.},
doi = {10.2307/1382344},
isbn = {1590; TY  - JOUR},
issn = {00222372},
journal = {Journal of Mammalogy},
keywords = {Mammal,New World,a question of great,abiotic factors,and those concerned with,bats,bauplan,biologi-,biotic interactions,diversity,interest to biogeog-,latitudinal gradient,macrotaxa,mammals,north america,raphers,south america,species},
number = {2},
pages = {322--334},
title = {{Diversity of New World mammals: universality of the latitudinal gradients of species and bauplans}},
url = {http://www.jstor.org/stable/10.2307/1382344},
volume = {76},
year = {1995}
}
@article{Dorp1987,
abstract = {The aim of this study was to assess the impact of isolation on forest bird communities in agricultural land- scapes in The Netherlands. We studied the avifauna of 235 small (0.1-39 ha) woodlots composed of mature deciduous trees in 1984- 1985. These woodlots were selected in the eastern and central/southern part of the country within 22 regions showing great differences in landscape structure, i.e., degree of isolation. Multiple regression analysis indicated that woodlot size was the best single predictor of species number and probability of occurrence of most species. It turned out that the isolation variables, area of wood, number of woods, interpatch distance, and proximity and density of connecting elements, explained small but significant parts of the residual variances in species number. No single species was significantly affected by the density of con- necting elements. Biogeographical differences between two groups of regions were emphasized. Evidence of four woodland- species suggested that regional abundance affected the probability of occurrence in small isolates.},
author = {van Dorp, D. and Opdam, P. F M},
doi = {10.1007/BF02275266},
isbn = {0921-2973},
issn = {09212973},
journal = {Landscape Ecology},
keywords = {birds,connectivity,forest fragmentation,patch,rural landscape},
month = {jul},
number = {1},
pages = {59--73},
pmid = {2197},
title = {{Effects of patch size, isolation and regional abundance on forest bird communities}},
url = {http://link.springer.com/10.1007/BF02275266},
volume = {1},
year = {1987}
}
@article{Hawkins2001,
abstract = {There is no abstract for this article. The text below is the first paragraph of text within the article.},
author = {Hawkins, Bradford a.},
doi = {10.1016/S0160-9327(00)01369-7},
isbn = {0169-5347},
issn = {01609327},
journal = {Endeavour},
month = {sep},
number = {3},
pages = {133--134},
pmid = {11725309},
title = {{Ecology's oldest pattern?}},
url = {http://linkinghub.elsevier.com/retrieve/pii/S0160932700013697},
volume = {25},
year = {2001}
}
@article{Donohue2013,
abstract = {Ecological stability is touted as a complex and multifaceted concept, including components such as variability, resistance, resilience, persistence and robustness. Even though a complete appreciation of the effects of perturbations on ecosystems requires the simultaneous measurement of these multiple components of stability, most ecological research has focused on one or a few of those components analysed in isolation. Here, we present a new view of ecological stability that recognises explicitly the non-independence of components of stability. This provides an approach for simplifying the concept of stability. We illustrate the concept and approach using results from a field experiment, and show that the effective dimensionality of ecological stability is considerably lower than if the various components of stability were unrelated. However, strong perturbations can modify, and even decouple, relationships among individual components of stability. Thus, perturbations not only increase the dimensionality of stability but they can also alter the relationships among components of stability in different ways. Studies that focus on single forms of stability in isolation therefore risk underestimating significantly the potential of perturbations to destabilise ecosystems. In contrast, application of the multidimensional stability framework that we propose gives a far richer understanding of how communities respond to perturbations.},
author = {Donohue, Ian and Petchey, Owen L. and Montoya, Jos{\'{e}} M. and Jackson, Andrew L. and McNally, Luke and Viana, Mafalda and Healy, Kevin and Lurgi, Miguel and O'Connor, Nessa E. and Emmerson, Mark C.},
doi = {10.1111/ele.12086},
isbn = {1461-0248},
issn = {1461023X},
journal = {Ecology Letters},
keywords = {Ecosystem function,Ellipsoid,Extinction,Invasion,Multidimensional stability,Persistence,Resilience,Resistance,Robustness,Variability},
number = {4},
pages = {421--429},
pmid = {23419041},
title = {{On the dimensionality of ecological stability}},
volume = {16},
year = {2013}
}
@article{Baselga2010,
abstract = {m{\'{e}}todo legal - integra an{\'{a}}lise de turnover (perda e ganho de spp) e de rede (dispers{\~{a}}o entre as metacomunidades) - compara os dois e tenta inferir o processos mais influentes na varia{\c{c}}{\~{a}}o da composi{\c{c}}{\~{a}}o},
author = {Baselga, Andr{\'{e}}s},
doi = {10.1111/j.1466-8238.2009.00490.x},
isbn = {1466-822X},
issn = {1466822X},
journal = {Global Ecology and Biogeography},
keywords = {Beta diversity,Cerambycidae,Europe,Nestedness,Similarity measures,Spatial turnover},
month = {jan},
number = {1},
pages = {134--143},
title = {{Partitioning the turnover and nestedness components of beta diversity}},
url = {http://doi.wiley.com/10.1111/j.1466-8238.2009.00490.x},
volume = {19},
year = {2010}
}
@article{Pires2011,
abstract = {BACKGROUND: Simple models inspired by processes shaping consumer-resource interactions have helped to establish the primary processes underlying the organization of food webs, networks of trophic interactions among species. Because other ecological interactions such as mutualisms between plants and their pollinators and seed dispersers are inherently based in consumer-resource relationships we hypothesize that processes shaping food webs should organize mutualistic relationships as well.$\backslash$n$\backslash$nMETHODOLOGY/PRINCIPAL FINDINGS: We used a likelihood-based model selection approach to compare the performance of food web models and that of a model designed for mutualisms, in reproducing the structure of networks depicting mutualistic relationships. Our results show that these food web models are able to reproduce the structure of most of the mutualistic networks and even the simplest among the food web models, the cascade model, often reproduce overall structural properties of real mutualistic networks.$\backslash$n$\backslash$nCONCLUSIONS/SIGNIFICANCE: Based on our results we hypothesize that processes leading to feeding hierarchy, which is a characteristic shared by all food web models, might be a fundamental aspect in the assembly of mutualisms. These findings suggest that similar underlying ecological processes might be important in organizing different types of interactions.},
author = {Pires, Mathias M. and Prado, Paulo I. and Guimar{\~{a}}es, Paulo R.},
doi = {10.1371/journal.pone.0027280},
editor = {Traveset, Anna},
isbn = {1932-6203},
issn = {19326203},
journal = {PLoS ONE},
number = {11},
pages = {e27280},
pmid = {22073303},
title = {{Do food web models reproduce the structure of mutualistic networks?}},
url = {http://www.pubmedcentral.nih.gov/articlerender.fcgi?artid=3206955{\&}tool=pmcentrez{\&}rendertype=abstract},
volume = {6},
year = {2011}
}
@article{AngelaD.LuisDavidT.S.HaymanThomasJ.O'SheaPaulM.CryanAmyT.GilbertJulietR.C.PulliamJamesN.MillsMaryE.TimoninCraigK.R.WillisAndrewA.CunninghamAnthonyR.FooksCharlesE.Rupprecht2013,
abstract = {Bats are the natural reservoirs of a number of high-impact viral zoonoses. We present a quantitative analysis to address the hypothesis that bats are unique in their propensity to host zoonotic viruses based on a comparison with rodents, another important host order. We found that bats indeed host more zoonotic viruses per species than rodents, and we identified life-history and ecological factors that promote zoonotic viral richness. More zoonotic viruses are hosted by species whose distributions overlap with a greater number of other species in the same taxonomic order (sympatry). Specifically in bats, there was evidence for increased zoonotic viral richness in species with smaller litters (one young), greater longevity and more litters per year. Furthermore, our results point to a new hypothesis to explain in part why bats host more zoonotic viruses per species: the stronger effect of sympatry in bats and more viruses shared between bat species suggests that interspecific transmission is more prevalent among bats than among rodents. Although bats host more zoonotic viruses per species, the total number of zoonotic viruses identified in bats (61) was lower than in rodents (68), a result of there being approximately twice the number of rodent species as bat species. Therefore, rodents should still be a serious concern as reservoirs of emerging viruses. These findings shed light on disease emergence and perpetuation mechanisms and may help lead to a predictive framework for identifying future emerging infectious virus reservoirs.},
author = {Luis, Angela D. and Hayman, David T S and O'Shea, Thomas J and Cryan, Paul M and Gilbert, Amy T and Pulliam, Juliet R C and Mills, James N and Timonin, Mary E and Willis, Craig K R and Cunningham, Andrew a and Fooks, Anthony R and Rupprecht, Charles E and Wood, James L N and Webb, Colleen T},
doi = {10.1098/rspb.2012.2753},
isbn = {0962-8452$\backslash$r1471-2954},
issn = {1471-2954},
journal = {Proceedings of the Royal Society B: Biological Sciences},
keywords = {Animals,Chiroptera,Chiroptera: virology,Disease Reservoirs,Disease Reservoirs: virology,Genome,Host-Pathogen Interactions,Rodentia,Rodentia: virology,Sympatry,Viral,Virus Diseases,Virus Diseases: transmission,Zoonoses,Zoonoses: transmission,Zoonoses: virology,gls},
mendeley-tags = {gls},
number = {1756},
pages = {20122753},
pmid = {23378666},
title = {{A comparison of bats and rodents as reservoirs of zoonotic viruses: are bats special?}},
url = {http://www.pubmedcentral.nih.gov/articlerender.fcgi?artid=3574368{\&}tool=pmcentrez{\&}rendertype=abstract{\%}5Cnhttp://rspb.royalsocietypublishing.org/content/280/1756/20122753.short{\%}5Cnhttp://rspb.royalsocietypublishing.org/cgi/doi/10.1098/rspb.2012.2753{\%}5Cnhttp://www},
volume = {280},
year = {2013}
}
@article{RodriguezCabal2012,
abstract = {The tens rules states that 10 {\%} of all introduced species establish and about 10 {\%} of those species become invasive. Several studies have failed to support the tens rule. However, these studies are beset by a general weakness: many unsuccessful invasions are never reported, and without these data tests of the tens rules are inadequate. Here, using data on the establishment success of non-native birds in Hawaii and Britain and comparing these data with those from a previous study, we show that lack of information about failed species introductions, and the tendency to report species that have become invasive more than those that have not, result in an overestimate of the establishment success and invasion rates of non-native species.},
author = {Rodriguez-Cabal, Mariano a. and Williamson, Mark and Simberloff, Daniel},
doi = {10.1007/s10530-012-0285-y},
issn = {13873547},
journal = {Biological Invasions},
keywords = {Birds,Exotic species,Invasion,Tens rule},
month = {jul},
number = {2},
pages = {249--252},
title = {{Overestimation of establishment success of non-native birds in Hawaii and Britain}},
url = {http://www.springerlink.com/index/10.1007/s10530-012-0285-y},
volume = {15},
year = {2013}
}
@article{Roll2007,
abstract = {We investigated characteristics of established non-indigenous (ENI) terrestrial vertebrates in Israel and adjacent areas, as well as attributes of areas they occupy. Eighteen non-indigenous birds have established populations in this region since 1850. A database of their attributes was compiled, analyzed, and compared to works from elsewhere. Most ENI bird species are established locally; a few are spreading or widespread. There has been a recent large increase in establishment. All ENI birds are of tropical origin, mostly from the Ethiopian and Oriental regions; the main families are Sturnidae, Psittacidae, Anatidae, and Columbidae. Most species have been deliberately brought to Israel in captivity and subsequently released or escaped. Most of these birds are commensal with humans to some degree, are not typically migratory, and have mean body mass larger than that of the entire order. ENI birds are not distributed randomly. There are centers in the Tel-Aviv area and along the Rift Valley, which is also a corridor of spread. Positive correlations were found between ENI bird richness and mean annual temperature and urbanization. Mediterranean forests and desert regions have fewer ENI species than expected. Apart from birds we report on non-indigenous species of reptiles (2) and mammals (2) in this region},
author = {Roll, Uri and Dayan, Tamar and Simberloff, Daniel},
doi = {10.1007/s10530-007-9160-7},
isbn = {1387-3547},
issn = {13873547},
journal = {Biological Invasions},
keywords = {Birds,Introduced species,Israel,Land vertebrates,Non-indigenous species},
month = {sep},
number = {5},
pages = {659--672},
title = {{Non-indigenous terrestrial vertebrates in Israel and adjacent areas}},
url = {http://www.springerlink.com/index/10.1007/s10530-007-9160-7},
volume = {10},
year = {2008}
}
@article{Lafferty2009,
abstract = {A robust food web is one in which few secondary extinctions occur after removing species. We investigated how parasites affected the robustness of the Carpinteria Salt Marsh food web by conducting random species removals and a hypothetical, but plausible, species invasion. Parasites were much more likely than free-living species to suffer secondary extinctions following the removal of a free-living species from the food web. For this reason, the food web was less robust with the inclusion of parasites. Removal of the horn snail, Cerithidea californica, resulted in a disproportionate number of secondary parasite extinctions. The exotic Japanese mud snail, Batillaria attramentaria, is the ecological analogue of the native California horn snail and can completely replace it following invasion. Owing to the similarities between the two snail species, the invasion had no effect on predator-prey interactions. However, because the native snail is host for 17 host-specific parasites, and the invader is host to only one, comparison of a food web that includes parasites showed significant effects of invasion on the native community. The hypothetical invasion also significantly reduced the connectance of the web because the loss of 17 native trematode species eliminated many links.},
author = {Lafferty, Kevin D and Kuris, Armand M},
doi = {10.1098/rstb.2008.0220},
isbn = {1471-2970 (Electronic)$\backslash$r0962-8436 (Linking)},
issn = {0962-8436},
journal = {Philosophical transactions of the Royal Society of London. Series B, Biological sciences},
keywords = {connectance,ecological analogue,food web,stability,subwebs,trematode},
month = {jun},
number = {1524},
pages = {1659--1663},
pmid = {19451117},
title = {{Parasites reduce food web robustness because they are sensitive to secondary extinction as illustrated by an invasive estuarine snail.}},
url = {http://www.pubmedcentral.nih.gov/articlerender.fcgi?artid=2685421{\&}tool=pmcentrez{\&}rendertype=abstract},
volume = {364},
year = {2009}
}
@article{Bascompte2010,
abstract = {Understanding the architecture of species relationships may help predict how ecosystems respond to change.},
author = {Bascompte, Jordi},
doi = {10.1126/science.1194255},
isbn = {0036-8075},
issn = {1095-9203},
journal = {Science (New York, N.Y.)},
keywords = {Animals,Biological,Ecosystem,Food Chain,Insects,Insects: physiology,Models,Plant Physiological Phenomena,Pollination,Symbiosis,ecological networks,ucture and dynamics of},
month = {aug},
number = {5993},
pages = {765--6},
pmid = {20705836},
title = {{Structure and dynamics of ecological networks.}},
url = {http://www.ncbi.nlm.nih.gov/pubmed/20705836},
volume = {329},
year = {2010}
}
@article{Combes1996,
abstract = {The influence of parasites in ecosystems, especially on biodiversity, is discussed. Various examples illustrate the role that parasites play in the outcome of interspecific competition, in the success of invading species, and in the separation of emerging species. Parasites can be stabilizers or destabilizers, depending on factors such as susceptibility of hosts and size of the ecosystem. Parasites play a major role each time 'something' disturbs living beings at the populational and/or specific level, as they do at the individual level},
author = {Combes, Claude},
doi = {10.1007/BF00054413},
isbn = {0960-3115},
issn = {0960-3115},
journal = {Biodiversity and Conservation},
keywords = {biodiversity,hybrid zones,outcome of competition,parasites,stability},
number = {8},
pages = {953--962},
title = {{Parasites, biodiversity and ecosystem stability}},
url = {http://www.springerlink.com/index/K2G202631056141X.pdf},
volume = {5},
year = {1996}
}
@article{Brandt2009,
author = {Brandt, Angela J and Seabloom, Eric W and Hosseini, Parviez R and Brandt, J and Eric, W and Hosseini, R},
journal = {Ecology},
keywords = {California,California grasslands,Darwin's naturalization,Santa Ynez,Sedgwick Reserve,USA.,alternate stable states,coexistence,exotic species,hypothesis,invasion,phylogenetic relatedness,plant-soil feedback},
number = {4},
pages = {1063--1072},
title = {{Phylogeny and Provenance Affect Plant-Soil Feedbacks in Invaded California Grasslands Published by : Ecological Society of America content in a trusted digital archive . We use information technology and tools to increase productivity and facilitate new f}},
url = {http://www.esajournals.org/doi/pdf/10.1890/08-0054.1},
volume = {90},
year = {2013}
}
@article{Dobson2008,
abstract = {Estimates of the total number of species that inhabit the Earth have increased significantly since Linnaeus's initial catalog of 20,000 species. The best recent estimates suggest that there are approximately 6 million species. More emphasis has been placed on counts of free-living species than on parasitic species. We rectify this by quantifying the numbers and proportion of parasitic species. We estimate that there are between 75,000 and 300,000 helminth species parasitizing the vertebrates. We have no credible way of estimating how many parasitic protozoa, fungi, bacteria, and viruses exist. We estimate that between 3{\%} and 5{\%} of parasitic helminths are threatened with extinction in the next 50 to 100 years. Because patterns of parasite diversity do not clearly map onto patterns of host diversity, we can make very little prediction about geographical patterns of threat to parasites. If the threats reflect those experienced by avian hosts, then we expect climate change to be a major threat to the relatively small proportion of parasite diversity that lives in the polar and temperate regions, whereas habitat destruction will be the major threat to tropical parasite diversity. Recent studies of food webs suggest that approximately 75{\%} of the links in food webs involve a parasitic species; these links are vital for regulation of host abundance and potentially for reducing the impact of toxic pollutants. This implies that parasite extinctions may have unforeseen costs that impact the health and abundance of a large number of free-living species.},
author = {Dobson, Andy and Lafferty, Kevin D and Kuris, Armand M and Hechinger, Ryan F and Jetz, Walter},
doi = {10.1073/pnas.0803232105},
isbn = {1148211489},
issn = {0027-8424},
journal = {Proceedings of the National Academy of Sciences of the United States of America},
keywords = {climate change  habitat loss  parasite biodivers},
number = {Supplement 1},
pages = {11482--11489},
pmid = {18695218},
title = {{Colloquium paper: homage to Linnaeus: how many parasites? How many hosts?}},
url = {http://www.pnas.org/content/105/suppl.1/11482.short},
volume = {105},
year = {2008}
}
@article{Williams2004,
abstract = {While trophic levels have found broad application throughout ecology, they are also in much contention on analytical and empirical grounds. Here, we use a new generation of data and theory to examine long-standing questions about trophic-level limits and degrees of omnivory. The data include food webs of the Chesapeake Bay, U.S.A., the island of Saint Martin, a U.K. grassland, and a Florida seagrass community, which appear to be the most trophically complete food webs available in the primary literature due to their inclusion of autotrophs and empirically derived estimates of the relative energetic contributions of each trophic link. We show that most (54{\%}) of the 212 species in the four food webs can be unambiguously assigned to a discrete trophic level. Omnivory among the remaining species appears to be quite limited, as judged by the standard deviation of omnivores' energy-weighted food-chain lengths. This allows simple algorithms based on binary food webs without energetic details to yield surprisingly accurate estimates of species' trophic and omnivory levels. While maximum trophic levels may plausibly exceed historically asserted limits, our analyses contradict both recent empirical claims that these limits are exceeded and recent theoretical claims that rampant omnivory eliminates the scientific utility of the trophic-level concept.},
author = {Williams, Richard J and Martinez, Neo D},
doi = {10.1086/381964},
isbn = {00030147},
issn = {1537-5323},
journal = {The American Naturalist},
keywords = {Algorithms,Animals,Biological,Diet,Food Chain,Models,Poaceae},
month = {mar},
number = {3},
pages = {458--68},
pmid = {15026980},
title = {{Limits to trophic levels and omnivory in complex food webs: theory and data.}},
url = {http://www.ncbi.nlm.nih.gov/pubmed/15026980},
volume = {163},
year = {2004}
}
@article{Magurran2012,
abstract = {How do species divide resources to produce the characteristic species abundance distributions seen in nature? One way to resolve this problem is to examine how the biomass (or capacity) of the spatial guilds that combine to produce an abundance distribution is allocated among species. Here we argue that selection on body size varies across guilds occupying spatially distinct habitats. Using an exceptionally well-characterized estuarine fish community, we show that biomass is concentrated in large bodied species in guilds where habitat structure provides protection from predators, but not in those guilds associated with open habitats and where safety in numbers is a mechanism for reducing predation risk. We further demonstrate that while there is temporal turnover in the abundances and identities of species that comprise these guilds, guild rank order is conserved across our 30-year time series. These results demonstrate that ecological communities are not randomly assembled but can be decomposed into guilds where capacity is predictably allocated among species.},
author = {Magurran, Anne E. and Henderson, P. a.},
doi = {10.1098/rspb.2012.1379},
isbn = {0962-8452},
issn = {0962-8452},
journal = {Proceedings of the Royal Society B: Biological Sciences},
keywords = {biodiversity,biomass,body size,estuarine fish,predation},
month = {oct},
number = {1743},
pages = {3722--3726},
pmid = {22787020},
title = {{How selection structures species abundance distributions}},
url = {http://www.pubmedcentral.nih.gov/articlerender.fcgi?artid=3415926{\&}tool=pmcentrez{\&}rendertype=abstract},
volume = {279},
year = {2012}
}
@article{Dunne2002,
abstract = {Abstract Food-web structure mediates dramatic effects of biodiversity loss including secondary and 'cascading' extinctions. We studied these effects by simulating primary species loss in 16 food webs from terrestrial and aquatic ecosystems and measuring robustness in terms of ...},
author = {Dunne, Jennifer A. and Williams, Richard J. and Martinez, Neo D.},
doi = {10.1046/j.1461-0248.2002.00354.x},
isbn = {1461-023X},
issn = {1461023X},
journal = {Ecology Letters},
keywords = {Biodiversity,Connectance,Ecosystem function,Food web,Network structure,Robustness,Secondary extinctions,Species loss,Species richness,Topology},
number = {4},
pages = {558--567},
pmid = {176644900015},
title = {{Network structure and biodiversity loss in food webs: robustness increases with connectance}},
volume = {5},
year = {2002}
}
@article{Mooney2007,
abstract = {Explaining the coexistence of closely related species sharing a single resource has been a long-standing challenge in ecology. Here we report on studies comparing the aphids Aphis nerii and A. asclepiadis that feed sympatrically on the milkweed Asclepias syriaca in northeastern North America. We sought to identify tradeoffs among species' attributes that might promote coexistence, but in most instances A. nerii was superior to A. asclepiadis. Aphis nerii was 84{\%} more fecund, fed upon 880{\%} more phloem sap, and was affected 70{\%} less by intraspecific competition as compared to A. asclepiadis. In interspecific competition, A. nerii reduced A. asclepiadis abundance by 77{\%}, whereas A. asclepiadis did not affect A. nerii. In dispersal trials, 10{\%} of winged A. nerii but only 1{\%} of A. asclepiadis successfully moved from non-host plants to A. syriaca. We also investigated whether there were differences in aphid interactions with milkweed cardenolides. Jasmonic acid increased milkweed cardenolides by 33{\%}, a realistic amount similar to that induced by several leaf-chewing herbivores. Nevertheless, jasmonate-induced cardenolides failed to affect aphid performance and aphid feeding had no effect on milkweed cardenolide concentration. Yet cardenolides were important for aphid resistance to predators; A. nerii sequestered 25{\%} more cardenolides and was preyed upon 50{\%} less than A. asclepiadis. Interactions with cardenolides thus again favored A. nerii over A. asclepiadis. Given that A. nerii and A. asclepiadis are decidedly not equivalent in their demography, competitive ability, defense and dispersal, our results strongly refute the notion that neutral processes can explain coexistence of these aphids. Based on field observations, we propose two tradeoffs - timing of milkweed colonization and relationships with ants - as putative mechanisms for the coexistence of these congeners.},
author = {Mooney, Kailen a. and Jones, Patricia and Agrawal, Anurag a.},
doi = {10.1111/j.2007.0030-1299.16284.x},
isbn = {0030-1299},
issn = {00301299},
journal = {Oikos},
number = {3},
pages = {450--458},
pmid = {4571},
title = {{Coexisting congeners: Demography, competition, and interactions with cardenolides for two milkweed-feeding aphids}},
url = {http://onlinelibrary.wiley.com/doi/10.1111/j.2007.0030-1299.16284.x/full},
volume = {117},
year = {2008}
}
@article{Kuris2008,
abstract = {Parasites can have strong impacts but are thought to contribute little biomass to ecosystems. We quantified the biomass of free-living and parasitic species in three estuaries on the Pacific coast of California and Baja California. Here we show that parasites have substantial biomass in these ecosystems. We found that parasite biomass exceeded that of top predators. The biomass of trematodes was particularly high, being comparable to that of the abundant birds, fishes, burrowing shrimps and polychaetes. Trophically transmitted parasites and parasitic castrators subsumed more biomass than did other parasitic functional groups. The extended phenotype biomass controlled by parasitic castrators sometimes exceeded that of their uninfected hosts. The annual production of free-swimming trematode transmission stages was greater than the combined biomass of all quantified parasites and was also greater than bird biomass. This biomass and productivity of parasites implies a profound role for infectious processes in these estuaries.},
author = {Kuris, Armand M and Hechinger, Ryan F and Shaw, Jenny C and Whitney, Kathleen L and Aguirre-Macedo, Leopoldina and Boch, Charlie a and Dobson, Andrew P and Dunham, Eleca J and Fredensborg, Brian L and Huspeni, Todd C and Lorda, Julio and Mababa, Luzviminda and Mancini, Frank T and Mora, Adrienne B and Pickering, Maria and Talhouk, Nadia L and Torchin, Mark E and Lafferty, Kevin D},
doi = {10.1038/nature06970},
isbn = {0028-0836},
issn = {0028-0836},
journal = {Nature},
keywords = {Animals,Biomass,California,Ecosystem,Host-Parasite Interactions,Pacific Ocean,Parasites,Parasites: isolation {\&} purification,Parasites: physiology,Snails,Snails: parasitology,Trematoda,Trematoda: isolation {\&} purification,Trematoda: physiology,Trematode Infections,Trematode Infections: parasitology,Wetlands},
month = {jul},
number = {7203},
pages = {515--518},
pmid = {18650923},
title = {{Ecosystem energetic implications of parasite and free-living biomass in three estuaries.}},
url = {http://www.ncbi.nlm.nih.gov/pubmed/18650923},
volume = {454},
year = {2008}
}
@article{Thompson2004,
abstract = {1. Despite their documented effects on trophic interactions and community structure, parasites are rarely included in food web analyses. The transmission routes of most parasitic helminths follow closely the trophic relationships among their successive hosts and are thus embedded in food webs, in a way that may influence energy flow and the structure of the web. 2. We investigated the impact of parasitism on the food web structure of a New Zealand intertidal mudflat community. Different versions of the food web were analysed, one with no parasites, one with all parasite species and several other versions, each including a single parasite species. We measured key food web metrics such as food chain length, linkage density and proportions of top, intermediate and basal species. 3. The inclusion of all parasite species in the food web resulted in greatly increased mean and maximum food chain length, but had little impact on linkage density and realized connectance. The main change caused by introduction of parasites was the relegation of a number of species from top predators to intermediate status, although the addition of parasites as top predators left the actual ratio of predators to prey relatively unchanged. 4. When individual parasites were added to the food web, their effect on food web properties was generally minimal. However, one trematode species that affected several host species, because of its complex life cycle and low host specificity, produced food web properties similar to those in the web version including all parasite species. 5. The respective effect of individual parasite species was roughly proportional to the number of host species they affected, and thus the life cycle characteristics of parasites determine to a large extent their impact on food web structure. The next step would be to quantify how they affect energy flow through the web.},
author = {Thompson, Ross M. and Mouritsen, Kim N. and Poulin, Robert},
doi = {10.1111/j.1365-2656.2004.00899.x},
isbn = {0021-8790},
issn = {00218790},
journal = {Journal of Animal Ecology},
keywords = {Energy,Helminths,Host specificity,Intertidal mudflat},
month = {dec},
number = {1},
pages = {77--85},
title = {{Importance of parasites and their life cycle characteristics in determining the structure of a large marine food web}},
url = {http://doi.wiley.com/10.1111/j.1365-2656.2004.00899.x},
volume = {74},
year = {2005}
}
@article{Martinez1995,
abstract = {trophic food webs. We show that recently tent with independent and basal trophic levels and "scale" or, in more specific terms, advanced scale-dependence connecting data and assumptions - from local to A central question in food-web research focuses on the relation between species at different the number hypotheses properties of species in are consis- of the smallest food webs with those of the largest. Consistent with scale dependence, these data and assumptions suggest that mean chain length and the fractions of species at top, intermediate, (i.e., small-scaled) webs with 2 to 102 species. Our analysis also suggests that the fractions of species in different trophic levels become asymptotically ally to globally scaled food webs with 104 to 107 species, respectively. A new hypothesis advanced. Our analyses specify for the first time specific functional effects of scale on food-web structure scale dependencies at the local scale as well as estimates},
author = {Martinez, Nd and Lawton, Jh},
issn = {0030-1299},
journal = {Oikos},
number = {2},
pages = {148--154},
title = {{Scale and food-web structure: from local to global}},
url = {http://www.jstor.org/stable/10.2307/3545903},
volume = {73},
year = {1995}
}
@article{Gaston2002,
abstract = {1. A number of generalizations have been made as to the effects of the area of occupancy, population size, dispersal ability and body size of species on their relative rates of local colonization and extinction. 2. Here, data on the breeding bird assemblage of Britain are used to test these generalizations. The complete geographical ranges of British birds have been censused twice, in the periods 1968-72 and 1988-91, allowing rates of colonization and extinction between these periods to be estimated. 3. The local colonization dynamics of species are influenced independently by their range sizes and the dispersal abilities of adult birds: species with smaller range sizes and larger dispersal distances were more likely to have colonized new areas between the two census periods. 4. The local extinction dynamics of species are influenced independently by their population sizes and body masses: species with smaller population sizes and body sizes were more likely to have gone extinct from areas inhabited in the first census period. 5. These results remain when controlling for the effects of phylogenetic relatedness. 6. These analyses uphold many commonly held generalizations about the correlates of local colonization and extinction, and suggest that the long-term evolutionary history of these bird species has influenced their potential to respond to current ecological conditions.},
author = {Gaston, Kevin J. and Blackburn, Tim M.},
doi = {10.1046/j.1365-2656.2002.00607.x},
isbn = {1365-2656},
issn = {00218790},
journal = {Journal of Animal Ecology},
keywords = {Body mass,British birds,Dispersal distance,Population size,Range size},
number = {3},
pages = {390--399},
pmid = {294},
title = {{Large-scale dynamics in colonization and extinction for breeding birds in Britain}},
url = {http://onlinelibrary.wiley.com/doi/10.1046/j.1365-2656.2002.00607.x/full},
volume = {71},
year = {2002}
}
@article{Marcogliese1997,
abstract = {Parasites have the capacity to regulate host populations and may be important determinants of community structure, yet they are usually neglected in studies of food webs. Parasites can provide much of the information on host biology, such as diet and migration, that is necessary to construct accurate webs. Because many parasites have complex life cycles that involve several different hosts, and often depend on trophic interactions for transmission, parasites provide complementary views of web structure and dynamics. Incorporation of parasites in food webs can substantially after baste web properties, Including connectance, chain length and proportions of top and basal species, and can allow the testing of specific hypotheses related to food-web dynamics.},
author = {Marcogliese, D J and Cone, D K},
doi = {10.1016/S0169-5347(97)01080-X},
isbn = {0169-5347},
issn = {0169-5347},
journal = {Trends in ecology {\&} evolution},
month = {aug},
number = {8},
pages = {320--5},
pmid = {21238094},
title = {{Food webs: a plea for parasites.}},
url = {http://www.sciencedirect.com/science/article/pii/S016953479701080X},
volume = {12},
year = {1997}
}
@article{Warren1994,
abstract = {Patterns in food web structure have provided an important, though contentious, testing ground for ideas about the population dynamics and energetics of multispecies systems. One of the most debated of these patterns is the apparent decrease in food web connectance as the number of species in a web Increases. Several contrasting mechanisms that might determine food web connectance have been suggested. These mechanisms, in combination with new, food web data, suggest that the conventional pattern, and explanations for it, may well be open to dispute. The true nature of the relationship between connectance and species number has implications for the explanation of other web patterns and for theories of food web structure, but a general explanation remains elusive.},
author = {Warren, Philip H.},
doi = {10.1016/0169-5347(94)90178-3},
issn = {01695347},
journal = {Trends in Ecology {\&} Evolution},
number = {4},
pages = {136--141},
pmid = {2474},
title = {{Making connections in food webs}},
url = {http://www.sciencedirect.com/science/article/pii/0169534794901783},
volume = {9},
year = {1994}
}
@article{Calcagno2010,
abstract = {We introduce glmulti , an R package for automated model selection and multi-model inference with glm and related functions. From a list of explanatory variables, the pro- vided function glmulti builds all possible unique models involving these variables and, optionally, their pairwise interactions. Restrictions can be speci ed for candidate models, by excluding speci c terms, enforcing marginality, or controlling model complexity. Mod- els are tted with standard R functions like glm . The n best models and their support (e.g., (Q)AIC, (Q)AICc, or BIC) are returned, allowing model selection and multi-model inference through standard R functions. The package is optimized for large candidate sets by avoiding memory limitation, facilitating parallelization and providing, in addition to exhaustive screening, a compiled genetic algorithm method. This article brie y presents the statistical framework and introduces the package, with applications to simulated and real data.},
author = {Calcagno, Vincent and Mazancourt, Claire De},
doi = {10.18637/jss.v034.i12},
isbn = {1548-7660},
issn = {15487660},
journal = {Journal of statistical software},
keywords = {aic,bic,genetic algorithm,glm,marginality,rjava,step,variable selection},
number = {12},
pages = {1--29},
title = {{glmulti : An R Package for Easy Automated Model Selection with ( Generalized ) Linear Models}},
volume = {34},
year = {2010}
}
@article{Sutherland2013,
abstract = {1. Fundamental ecological research is both intrinsically interesting and provides the basic knowledge required to answer applied questions of importance to the management of the natural world. The 100th anniversary of the British Ecological Society in 2013 is an opportune moment to reflect on the current status of ecology as a science and look forward to high-light priorities for future work.$\backslash$r$\backslash$n2. To do this, we identified 100 important questions of fundamental importance in pure ecology. We elicited questions from ecologists working across a wide range of systems and disciplines. The 754 questions submitted (listed in the online appendix) from 388 participants were narrowed down to the final 100 through a process of discussion, rewording and repeated rounds of voting. This was done during a two-day workshop and thereafter.$\backslash$r$\backslash$n3. The questions reflect many of the important current conceptual and technical pre-occupations of ecology. For example, many questions concerned the dynamics of environmental change and complex ecosystem interactions, as well as the interaction between ecology and evolution.$\backslash$r$\backslash$n4. The questions reveal a dynamic science with novel subfields emerging. For example, a group of questions was dedicated to disease and micro-organisms and another on human impacts and global change reflecting the emergence of new subdisciplines that would not have been foreseen a few decades ago.$\backslash$r$\backslash$n5. The list also contained a number of questions that have perplexed ecologists for decades and are still seen as crucial to answer, such as the link between population dynamics and life-history evolution.$\backslash$r$\backslash$n6. Synthesis. These 100 questions identified reflect the state of ecology today. Using them as an agenda for further research would lead to a substantial enhancement in understanding of the discipline, with practical relevance for the conservation of biodiversity and ecosystem function.},
author = {Sutherland, William J. and Freckleton, Robert P. and Godfray, H. Charles J and Beissinger, Steven R. and Benton, Tim and Cameron, Duncan D. and Carmel, Yohay and Coomes, David a. and Coulson, Tim and Emmerson, Mark C. and Hails, Rosemary S. and Hays, Graeme C. and Hodgson, Dave J. and Hutchings, Michael J. and Johnson, David and Jones, Julia P G and Keeling, Matt J. and Kokko, Hanna and Kunin, William E. and Lambin, Xavier and Lewis, Owen T. and Malhi, Yadvinder and Mieszkowska, Nova and Milner-Gulland, E. J. and Norris, Ken and Phillimore, Albert B. and Purves, Drew W. and Reid, Jane M. and Reuman, Daniel C. and Thompson, Ken and Travis, Justin M J and Turnbull, Lindsay a. and Wardle, David a. and Wiegand, Thorsten},
doi = {10.1111/1365-2745.12025},
isbn = {0022-0477},
issn = {00220477},
journal = {Journal of Ecology},
keywords = {Community ecology,Ecology,Ecosystems,Evolutionary ecology,Population ecology,Research priorities},
number = {1},
pages = {58--67},
pmid = {23874189},
title = {{Identification of 100 fundamental ecological questions}},
volume = {101},
year = {2013}
}
@article{Sample2012,
abstract = {Length-weight regression models were generated for 10 orders and 35 families of adult and larval insects using a power model. Additional models were generated that incorporated width as an independent variable, to account for varying body morphology within insect taxa. Inclusion of width improved the generalized insect model and models at the order level, but was of less value in improving family level models. The predictive value of all models was high; estimates were generally within +/-2 mg of the actual values. The parameter values for our models were similar to those produced by other researchers.},
author = {Sample, Bradley E and Cooper, Robert J and Greer, Richard D and Whitmore, Robert C},
doi = {10.2307/2426503},
isbn = {00030031},
issn = {0003-0031},
journal = {American Midland Naturalist},
number = {2},
pages = {234--240},
title = {{Estimation of insect biomass by length and width}},
url = {http://www.jstor.org/stable/2426503},
volume = {129},
year = {1993}
}
@article{Rosindell2012,
abstract = {Ecological neutral theory has elicited strong opinions in recent years. Here, we review these opinions and strip away some unfortunate problems with semantics to reveal three major underlying questions. Only one of these relates to neutral theory and the importance of ecological drift, whereas the others involve the link between pattern and process, the tradeoff between simplicity and complexity in modeling, and the role of stochasticity and drift in ecology. We explain how neutral theory cannot be simultaneously used both as a null hypothesis and as an approximation. However, we also show how neutral theory always has a valuable use in one of these two roles, even though the real world is not neutral. ?? 2012 Elsevier Ltd.},
author = {Rosindell, James and Hubbell, Stephen P. and He, Fangliang and Harmon, Luke J. and Etienne, Rampal S.},
doi = {10.1016/j.tree.2012.01.004},
isbn = {0169-5347},
issn = {01695347},
journal = {Trends in Ecology and Evolution},
keywords = {Biological Evolution,Ecology,Models, Theoretical,Stochastic Processes},
month = {may},
number = {4},
pages = {203--208},
pmid = {22341498},
publisher = {Elsevier Ltd},
title = {{The case for ecological neutral theory}},
url = {http://www.ncbi.nlm.nih.gov/pubmed/22341498},
volume = {27},
year = {2012}
}
@article{Petchey2008,
abstract = {Understanding what structures ecological communities is vital to answering questions about extinctions, environmental change, trophic cascades, and ecosystem functioning. Optimal foraging theory was conceived to increase such understanding by providing a framework with which to predict species interactions and resulting community structure. Here, we use an optimal foraging model and allometries of foraging variables to predict the structure of real food webs. The qualitative structure of the resulting model provides a more mechanistic basis for the phenomenological rules of previous models. Quantitative analyses show that the model predicts up to 65{\%} of the links in real food webs. The deterministic nature of the model allows analysis of the model's successes and failures in predicting particular interactions. Predacious and herbivorous feeding interactions are better predicted than pathogenic, parasitoid, and parasitic interactions. Results also indicate that accurate prediction and modeling of some food webs will require incorporating traits other than body size and diet choice models specific to different types of feeding interaction. The model results support the hypothesis that individual behavior, subject to natural selection, determines individual diets and that food web structure is the sum of these individual decisions.},
author = {Petchey, Owen L and Beckerman, Andrew P and Riede, Jens O and Warren, Philip H},
doi = {10.1073/pnas.0710672105},
file = {:Users/alyssacirtwill/Documents/Papers/Petchey et al.{\_}2008{\_}Proceedings of the National Academy of Sciences of the United States of America.pdf:pdf},
isbn = {0027-8424},
issn = {1091-6490},
journal = {Proceedings of the National Academy of Sciences of the United States of America},
keywords = {Animals,Biological,Food Chain,Models},
month = {mar},
number = {11},
pages = {4191--4196},
pmid = {18337512},
title = {{Size, foraging, and food web structure}},
url = {http://www.pubmedcentral.nih.gov/articlerender.fcgi?artid=2393804{\&}tool=pmcentrez{\&}rendertype=abstract},
volume = {105},
year = {2008}
}
@article{Alon2007,
abstract = {Transcription regulation networks control the expression of genes. The transcription networks of well-studied microorganisms appear to be made up of a small set of recurring regulation patterns, called network motifs. The same network motifs have recently been found in diverse organisms from bacteria to humans, suggesting that they serve as basic building blocks of transcription networks. Here I review network motifs and their functions, with an emphasis on experimental studies. Network motifs in other biological networks are also mentioned, including signalling and neuronal networks.},
author = {Alon, U},
doi = {nrg2102 [pii]\r10.1038/nrg2102},
file = {:Users/alyssacirtwill/Documents/Papers/Alon{\_}2007{\_}Nature Reviews Genetics.pdf:pdf},
isbn = {1471-0056 (Print)$\backslash$r1471-0056 (Linking)},
issn = {1471-0056},
journal = {Nature Reviews Genetics},
keywords = {*Gene Expression Regulation,*Models,Animals,Bacteria/genetics/metabolism,Evolution,Fungi/genetics/metabolism,Genetic,Homeostasis,Humans,Regulon/genetics,Transcription,Transcription Factors/genetics/metabolism},
month = {jun},
number = {6},
pages = {450--461},
pmid = {17510665},
title = {{Network motifs: theory and experimental approaches}},
url = {http://www.ncbi.nlm.nih.gov/entrez/query.fcgi?cmd=Retrieve{\&}db=PubMed{\&}dopt=Citation{\&}list{\_}uids=17510665},
volume = {8},
year = {2007}
}
@article{Stouffer2010,
abstract = {1. Food webs, the set of predator–prey interactions in an ecosystem, are a prototypical complex system. Much research to date has concentrated on the use of models to identify and explain the key structural features which characterize food webs. 2. These models often fall into two general categories: (i) phenomenological models which are built upon a set of heuristic rules in order to explain some empirical observation and (ii) population- level models in which interactions between individuals result in emergent properties for the food web. Both types of models have helped to uncover how food-web structure is a product of factors such as foraging behaviour, prey selection and species' body sizes. 3. Historically, the two types of models have followed rather different approaches to the problem. Despite the apparent differences, the overlap between the two styles of models is substantial. Examples are highlighted here. 4. By paying greater attention to both the similarities and differences between the two, we will be better able to demonstrate the ecological insights offered by phenomenological models. This will help us, for example, design experiments which could validate or refute underlying assumptions of the models. By linking models to data, scaling from individuals to networks, we will be closer to understanding the true origins of food-web structure.},
author = {Stouffer, Daniel B.},
doi = {10.1111/j.1365-2435.2009.01644.x},
isbn = {0269-8463},
issn = {02698463},
journal = {Functional Ecology},
keywords = {Food webs,Network theory,Phenomenological models,Population-level models,Theoretical ecology},
month = {feb},
number = {1},
pages = {44--51},
title = {{Scaling from individuals to networks in food webs}},
url = {http://blackwell-synergy.com/doi/abs/10.1111/j.1365-2435.2009.01644.x},
volume = {24},
year = {2010}
}
@article{Housworth2004,
abstract = {The phylogenetic mixed model is an application of the quantitative-genetic mixed model to interspecific data. Although this statistical framework provides a potentially unifying approach to quantitative-genetic and phylogenetic analysis, the model has been applied infrequently because of technical difficulties with parameter estimation. We recommend a reparameterization of the model that eliminates some of these difficulties, and we develop a new estimation algorithm for both the original maximum likelihood and new restricted maximum likelihood estimators. The phylogenetic mixed model is particularly rich in terms of the evolutionary insight that might be drawn from model parameters, so we also illustrate and discuss the interpretation of the model parameters in a specific comparative analysis.},
author = {Housworth, Elizabeth A. and Martins, Em{\'{i}}lia P. and Lynch, Michael},
doi = {10.1086/380570},
isbn = {ISSN 0003-0147},
issn = {0003-0147},
journal = {The American Naturalist},
keywords = {appreciated that interspecific analyses,be compromised if they,can,comparative method,fail to account for,it is now well,mixed model,phenotypic evolution,phylogenetic analysis,phylogenetic heritability,quantitative genetics,the statistical},
month = {jan},
number = {1},
pages = {84--96},
pmid = {14767838},
title = {{The Phylogenetic Mixed Model}},
url = {http://www.journals.uchicago.edu/doi/10.1086/380570},
volume = {163},
year = {2004}
}
@article{Howard2012,
author = {Howard, F W},
pages = {1--9},
title = {{American Palm Cixiid , Myndus crudus Van Duzee ( Insecta : Hemiptera : Auchenorrhyncha : Fulgoroidea :}},
year = {2012}
}
@article{Holmberg2011,
abstract = {The statistical analysis of mixed effects models for binary and count data is investigated. In the statistical computing environment R, there are a few packages that estimate models of this kind. The package lme4 is a de facto standard for mixed effects models. The package glmmML allows non-normal distributions in the specification of random intercepts. It also allows for the estimation of a fixed effects model, assuming that all cluster intercepts are distinct fixed parameters; moreover, a bootstrapping technique is implemented to replace asymptotic analysis. The random intercepts model is fitted using a maximum likelihood estimator with adaptive GaussHermite and Laplace quadrature approximations of the likelihood function. The fixed effects model is fitted through a profiling approach, which is necessary when the number of clusters is large. In a simulation study, the two approaches are compared. The fixed effects model has severe bias when the mixed effects variance is positive and the number of clusters is large. ?? 2011 Elsevier B.V. All rights reserved.},
author = {Brostr{\"{o}}m, G{\"{o}}ran and Holmberg, Henrik},
doi = {10.1016/j.csda.2011.06.011},
isbn = {0167-9473},
issn = {01679473},
journal = {Computational Statistics and Data Analysis},
keywords = {Bernoulli distribution,GaussHermite quadrature,Implicit derivation,Laplace approximation,Poisson distribution,Profiling},
number = {12},
pages = {3123--3134},
title = {{Generalized linear models with clustered data: Fixed and random effects models}},
volume = {55},
year = {2011}
}
@article{Hofer2009,
abstract = {We sampled 505 specimens of 7 arachnid orders (313 Araneae, 65 Opiliones, 111 Pseudoscorpiones, 10 Ricinulei, 3 Schizomida, 1 Thelyphonida, 2 Scorpiones) in natural forest and agroforestry sites in central Amazonia to analyze fresh and dry mass to body length relations. The low number of schizomids, scorpions, and thelyphonids did not allow statistical analyses, but the raw data are given, because these represent the first data published for these groups from Amazonia. For all other orders general mass-length relationships for ecological studies were determined. Non-linear regressions with a power model proved to describe the relations very well and are highly significant for all taxa and groups analyzed. The resulting equations can thus be used to estimate biomass of large samples of arachnids from Amazonia based on individual body length measurements. Linear regressions of mass to length with log-transformed data also described the relation adequately, but using the resulting equations to estimate biomass of the whole spider sample caused a higher bias. This is because small biases of mass-length relation of the largest spider individuals are exponentiated. However, linear regressions behaved better for spiders smaller than 8 mm. The ratio of dry to fresh mass was around 0.3 for spiders; 0.4 for pseudoscorpions, schizomids, and thelyphonids; 0.44 for opilionids; and 0.53 for Ricinulei. A second sample of 99 spiders from a South Brazilian Atlantic Forest revealed similar mass-length relations, but a different dry to fresh mass ratio. For spiders, the usefulness of general equations to determine the biomass of bulk samples from ecological studies with certain precision requirements was further explored by using the equations from the two datasets crosswise, regarding the resulting bias and by applying equations to a further dataset from an ecological investigation. In conclusion and accordance to former studies, general equations derived from mass-length regressions of bulk samples including many specimens of different families and guilds are appropriate for an estimation of the biomass of bulk samples from ecological studies. Equations from mass-length regressions from the literature, resulting from spider samples in temperate regions, should not be used to estimate biomass of samples from neotropical spider assemblages, especially when absolute biomass is of interest and when precision is required. They underestimate biomass of tropical assemblages due to a strong bias in mass-length relation of tropical spiders larger than 10 mm. Depending on the distribution of large spiders in samples, considerable biases in single samples could affect ecological analyses.},
author = {H{\"{o}}fer, Hubert and Ott, Ricardo},
doi = {10.1636/T08-21.1},
isbn = {0161-8202},
issn = {0161-8202},
journal = {Journal of Arachnology},
keywords = {Arachnida,Brazil,mass-length relationship},
number = {2},
pages = {160--169},
title = {{Estimating biomass of Neotropical spiders and other arachnids (Araneae, Opiliones, Pseudoscorpiones, Ricinulei) by mass-length regressions}},
url = {http://dx.doi.org/10.1636/T08-21.1},
volume = {37},
year = {2009}
}
@article{Nakahara1995,
abstract = {Ten new species of Tetraleurodes (bireflexa, caulicola, chivela, confusa, dorsirugosa, mexicana, perseae, pseudacaciae, quercicola, tuberculosa) are described and four previously known species are redescribed. The acaciae group consisting of seven species is proposed, and a key to the acaciae group and 12 North American species is provided. In addition, herberti Penny is synonymized with acaciae (Quaintance), nudus Sampson and Drews is synonymized with fici Quaintance and Baker, and stanfordi (Bemis) is synonymized with perileuca (Cockerell) Aleurotrachelus cacaorum Bondar is reassigned to Tetraleurodes, and T. papilliferus Sampson and Drews is reassigned to Aleurotrachelus.},
author = {Nakahara, Sueo},
issn = {0749-6737},
journal = {Insecta Mundi},
keywords = {1978,aleyrodidae,i n a catalog,included 50 species,k e y,mound and halsey,new combinations,new species,north america,of the world,species,synonyms,tetraleurodes,tetraleurodes is one of,the,the larger genera in,whitefly,whitefly family aleyrodidae},
number = {1-2},
pages = {105--150},
title = {{Taxonomic Studies of the Genus Tetraleurodes (Homoptera: Aleyrodidae)}},
url = {http://journals.fcla.edu/mundi/article/view/24778{\%}5Cnhttp://journals.fcla.edu/mundi/article/view/24778/24109},
volume = {9},
year = {1995}
}
@article{Freund1995,
abstract = {The issid planthopper genus Acanalonia is reviewed and a key to the 18 species provided. Detailed descriptions and illustrations of the complete external morphology or A. conica (Say), and descriptions and illustrations of the male and female external genitalia ofthe species of United States Acanalonia are given. The principal genitalic features used to separate species included: male-shape and length of the aedeagal caudal and lateral processes, and presence of caudal extensions; female-shape of the 8th abdominal segment and the number of teeth on the gonapophysis of the 8th segment.},
author = {Freund, Rebecca and Wilson, Stephen W},
isbn = {*},
issn = {0749-6737},
journal = {Insecta Mundi},
keywords = {acanalonia,fulgoroidea,homoptera,issidae,north america},
pages = {195--216},
title = {{The planthopper genus Acanalonia in the United States (Homoptera: Issidae): male and female genitalic morphology}},
url = {http://journals.fcla.edu/mundi/article/view/24789{\%}5Cnhttp://journals.fcla.edu/mundi/article/view/24789/24120},
volume = {9},
year = {1995}
}
@article{Lomolino2009,
abstract = {Biogeographers study all patterns in the geographic variation of life, from the spatial variation in genetic and physiological characteristics of cells and individuals, to the diversity and dynamics of biological communities among continental biotas or across oceanic archipelagoes. The field of island biogeography, in particular, has provided some genuinely transformative insights for the biological sciences, especially ecology and evolutionary biology. Our purpose here is to review the historical development of island biogeography theory during the 20th century by identifying the common threads that run through four sets of contributions made during this period, including those by Eugene Gordon Munroe (1948, 1953), Edward O. Wilson (1959, 1961), Frank W. Preston (1962a,b), and the seminal collaborations between Wilson and Robert H. MacArthur (1963, 1967), which revolutionized the field and served as its paradigm for nearly four decades. This epistemological account not only reviews the intriguing history of island theory, but it also includes fundamental lessons for advancing science through transformative integrations. Indeed, as is likely the case with many disciplines, island theory advanced not as a simple accumulation of facts and an orderly succession of theories and paradigms, but rather in fits and starts through a reticulating phylogeny of ideas and alternating periods of specialization and reintegration. We conclude this review with a summary of the salient features of this scientific revolution in the contest of Kuhn's structure, which strongly influenced theoretical advances during this period, and we then describe some of the fundamental assumptions and tenets of an emerging reintegration of island biogeography theory.},
author = {Lomolino, Mark V. and {H. Brown}, James},
doi = {10.1086/648123},
isbn = {0033-5770},
issn = {0033-5770},
journal = {The Quarterly Review of Biology},
keywords = {Animals,Biodiversity,Ecosystem,Medical,Phylogeny,Population Dynamics,Topography},
month = {dec},
number = {4},
pages = {357--390},
pmid = {20039528},
title = {{The Reticulating Phylogeny of Island Biogeography Theory}},
url = {http://www.journals.uchicago.edu/doi/10.1086/648123},
volume = {84},
year = {2009}
}
@article{Bolker2009a,
abstract = {How should ecologists and evolutionary biologists analyze nonnormal data that involve random effects? Nonnormal data such as counts or proportions often defy classical statistical procedures. Generalized linear mixed models (GLMMs) provide a more flexible approach for analyzing nonnormal data when random effects are present. The explosion of research on GLMMs in the last decade has generated considerable uncertainty for practitioners in ecology and evolution. Despite the availability of accurate techniques for estimating GLMM parameters in simple cases, complex GLMMs are challenging to fit and statistical inference such as hypothesis testing remains difficult. We review the use (and misuse) of GLMMs in ecology and evolution, discuss estimation and inference and summarize 'best-practice' data analysis procedures for scientists facing this challenge. {\textcopyright} 2008 Elsevier Ltd. All rights reserved.},
archivePrefix = {arXiv},
arxivId = {1003.3921v1},
author = {Bolker, Benjamin M. and Brooks, Mollie E. and Clark, Connie J. and Geange, Shane W. and Poulsen, John R. and Stevens, M. Henry H and White, Jada Simone S},
doi = {10.1016/j.tree.2008.10.008},
eprint = {1003.3921v1},
isbn = {0169-5347},
issn = {01695347},
journal = {Trends in Ecology and Evolution},
keywords = {Bayes Theorem,Biological Evolution,Data Interpretation, Statistical,Ecology,Likelihood Functions,Linear Models,Software},
month = {mar},
number = {3},
pages = {127--135},
pmid = {19185386},
title = {{Generalized linear mixed models: a practical guide for ecology and evolution}},
url = {http://www.ncbi.nlm.nih.gov/pubmed/19185386},
volume = {24},
year = {2009}
}
@article{Costa2009,
abstract = {Based on geographical and home range sizes, physiology, and gape limitation, a positive relationship between predator size and diet breadth is expected. Alternatively, larger predators might avoid smaller prey; in this case no relationship would be found. Here, I used a large data set on the diets of marine predators to describe and identify mechanisms responsible for the relationships among predator body size, diet breadth, and the mean, minimum, maximum, and variance of prey size. I found no relationship between predator size and diet breadth. Mean, minimum, maximum, and variance of prey size were all positively associated with predator size. I found that larger predators increase their minimum and maximum prey size with similar slopes, which explains the lack of relationship between predator size and diet breadth. The results support predictions of the hypothesis that optimal foraging is the main factor constraining the shape of the relationships among predator size, prey size, and diet breadth. Future research should focus on examining the relationship between body size and the breadth of different niche axis across different groups of organisms to assess whether a positive relationship between body size and niche breadth is a general rule in macroecology.},
author = {Costa, Gabriel C.},
doi = {10.1890/08-1150.1},
isbn = {0012-9658},
issn = {00129658},
journal = {Ecology},
keywords = {Body size,Diet,Macroecology,Marine predators,Niche breadth,Optimal foraging,Phylogenetic contrast,Predator,Prey size},
number = {7},
pages = {2014--2019},
pmid = {19694148},
title = {{Predator size, prey size, and dietary niche breadth relationships in marine predators}},
url = {http://www.esajournals.org/doi/abs/10.1890/08-1150.1},
volume = {90},
year = {2009}
}
@article{Wang2012,
abstract = {There is a recognized need to anticipate tipping points, or critical transitions, in social-ecological systems. Studies of mathematical and experimental systems have shown that systems may 'wobble' before a critical transition. Such early warning signals may be due to the phenomenon of critical slowing down, which causes a system to recover slowly from small impacts, or to a flickering phenomenon, which causes a system to switch back and forth between alternative states in response to relatively large impacts. Such signals for transitions in social-ecological systems have rarely been observed, not the least because high-resolution time series are normally required. Here we combine empirical data from a lake-catchment system with a mathematical model and show that flickering can be detected from sparse data. We show how rising variance coupled to decreasing autocorrelation and skewness started 10-30 years before the transition to eutrophic lake conditions in both the empirical records and the model output, a finding that is consistent with flickering rather than critical slowing down. Our results suggest that if environmental regimes are sufficiently affected by large external impacts that flickering is induced, then early warning signals of transitions in modern social-ecological systems may be stronger, and hence easier to identify, than previously thought.},
author = {Wang, Rong and Dearing, John A. and Langdon, Peter G. and Zhang, Enlou and Yang, Xiangdong and Dakos, Vasilis and Scheffer, Marten},
doi = {10.1038/nature11655},
isbn = {0028-0836},
issn = {1476-4687},
journal = {Nature},
keywords = {19th Century,20th Century,21st Century,China,Diatoms,Diatoms: isolation {\&} purification,Eutrophication,Forecasting,Forecasting: methods,Geologic Sediments,Geologic Sediments: analysis,Geologic Sediments: chemistry,History,Lakes,Lakes: chemistry,Phosphorus,Phosphorus: analysis,Time Factors},
month = {dec},
number = {7429},
pages = {419--422},
pmid = {23160492},
publisher = {Nature Publishing Group},
title = {{Flickering gives early warning signals of a critical transition to a eutrophic lake state}},
url = {http://www.nature.com/doifinder/10.1038/nature11655{\%}5Cnhttp://www.ncbi.nlm.nih.gov/pubmed/23160492{\%}5Cn{\%}3CGo to ISI{\%}3E://WOS:000312488200054{\%}5Cnhttp://www.nature.com/nature/journal/v492/n7429/pdf/nature11655.pdf},
volume = {492},
year = {2012}
}
@article{Laurance2012,
abstract = {Abstract. Strong global demand for tropical timber and agricultural products has driven large-scale logging and subsequent conversion of tropical forests. Given that the majority of tropical landscapes have been or will likely be logged, the protection ofbiodiversity within tropical forests thus depends on whether species can persist in these economically exploited lands, and if species cannot persist, whether we can protect enough primary forest from logging and conversion. However, our knowledge of the impact of logging and conversion on biodiversity is limited to a few taxa, often sampled in different locations with complex land-use histories, hampering attempts to plan cost-effective conservation strategies and to draw conclusions across taxa. Spanning a land-use gradient of primary forest, once- and twice-logged forests, and oil palm plantations, we used traditional sampling andDNAmetabarcoding to compile an extensive data set inSabah,MalaysianBorneofornine vertebrate andinvertebrate taxa toquantify thebiological impacts of logging and oil palm, develop cost-effective methods of protecting biodiversity, and examinewhether there is congruence inresponse among taxa.Logged forests retained high species richness, including, on average, 70{\%}of species found in primary forest. In contrast, conversion to oil palmdramatically reduces species richness, with significantly fewer primary-forest species than foundonloggedforest transects for seventaxa.Usinga systematic conservationplanning analysis, we show that efficient protection of primary-forest species is achieved with land portfolios that include a large proportion of logged-forest plots. Protecting logged forests is thus a cost-effective method of protecting an ecologically and taxonomically diverse range of species, particularly when conservation budgets are limited. Six indicator groups (birds, leaf-litter ants, beetles, aerial hymenopterans, flies, and true bugs) proved to be consistently good predictors of the response of the other taxa to logging and oil palm. Our results confidently establish the high conservation value of logged forests and the low value of oil palm. Cross-taxon congruence in responses to disturbance also suggests that the practice of focusing on key indicator taxa yields important information of general biodiversity in studies of logging and oil palm.},
author = {Edwards, David P. and Magrach, Ainhoa and Woodcock, Paul and Ji, Yinqiu and Lim, Norman T L and Edwards, Felicity a. and Larsen, Trond H. and Hsu, Wayne W. and Benedick, Suzan and Khen, Chey Vun and Chung, Arthur Y C and Reynolds, Glen and Fisher, Brendan and Laurance, William F. and Wilcove, David S. and Hamer, Keith C. and Yu, Douglas W.},
doi = {10.1038/nature11318},
isbn = {1476-4687 (Electronic)$\backslash$r0028-0836 (Linking)},
issn = {10510761},
journal = {Ecological Applications},
keywords = {Cost-effective conservation,Indicator taxa,Oil palm plantation agriculture,Sabah, Malaysian Borneo,Selective logging,Southeast Asia,Timber concessions,Tropical rain forest},
month = {sep},
number = {8},
pages = {2029--2049},
pmid = {22832582},
publisher = {Nature Publishing Group},
title = {{Selective-logging and oil palm: Multitaxon impacts, biodiversity indicators, and trade-offs for conservation planning}},
url = {http://www.ncbi.nlm.nih.gov/pubmed/22832582},
volume = {24},
year = {2014}
}
@article{MacArthur1963a,
abstract = {A graphical equilibrium model, balancing immigration and extinction rates of species, has been developed which appears fully consistent with the fauna-area curves and the distance effect seen in land and freshwater bird faunas of the Indo-Australian islands. The establishment of the equilibrium condition allows the development of a more precise zoogeographic theory than hitherto possible. One new and non-obvious prediction can be made from the model which is immediately verifiable from existing data, that the number of species increases with area more rapidly on far islands than on near ones. Similarly, the number of species on large islands decreases with distance faster than does the number of species on small islands. As groups of islands pass from the unsaturated to saturated conditions, the variance-to-mean ratio should change from unity to about one-half. When the faunal buildup reaches 90{\%} of the equilibrium number, the extinction rate in species/year should equal 2.303 times the variance divided by the time (in years) required to reach the 90{\%} level. The implications of this relation are discussed with reference to the Krakatau faunas, where the buildup rate is known. A 'radiation zone,' in which the rate of intra-archipelagic exchange of autochthonous species approaches or exceeds extraarchipelagic immigration toward the outer limits of the taxon's range, is predicted as still another consequence of the equilibrium condition. This condition seems to be fulfilled by conventional information but cannot be rigorously tested with the existing data. Where faunas are at or near equilibrium, it should be possible to devise indirect estimates of the actual immigration and extinction rates, as well as of the times required to reach equilibrium. It should also be possible to estimate the mean dispersal distance of propagules overseas from the zoogeographic data. Mathematical models have been constructed to these ends and certain applications suggested. The main purpose of the paper is to express the criteria and implications of the equilibrium condition, without extending them for the present beyond the IndoAustralian bird faunas. CR - Copyright {\&}{\#}169; 1963 Society for the Study of Evolution},
archivePrefix = {arXiv},
arxivId = {arXiv:1011.1669v3},
author = {Macarthur, Robert H and Wilson, Edward},
doi = {10.2307/2407089},
eprint = {arXiv:1011.1669v3},
isbn = {00143820},
issn = {00143820},
journal = {Evolution},
number = {4},
pages = {373--387},
pmid = {1536},
title = {{An Equilibrium Theory of Insular Zoogeography Robert H. MacArthur;}},
url = {http://www.jstor.org/stable/2407089},
volume = {17},
year = {1963}
}
@article{Simberloff1976,
abstract = {The application of island biogeography theory to conservation practice is premature. Theoretically and empirically, a major conclusion of such applications-that refuges should always consist of the largest possible single area-can be incorrect under a variety of biologically feasible conditions. The cost and irreversibility of large-scale conservation programs demand a prudent approach to the application of an insufficiently validated theory.},
author = {Simberloff, Daniel S. and Abele, Lawrence G.},
doi = {10.1126/science.191.4224.285},
isbn = {1630130044},
issn = {00368075},
journal = {Science},
number = {4224},
pages = {285--286},
pmid = {17832147},
title = {{Island Biogeography Theory and Conservation Practice}},
url = {http://www.sciencemag.org/content/191/4224/285.abstract},
volume = {New Series},
year = {1976}
}
@article{Garland2005,
abstract = {Over the past two decades, comparative biological analyses have undergone profound changes with the incorporation of rigorous evolutionary perspectives and phylogenetic information. This change followed in large part from the realization that traditional methods of statistical analysis tacitly assumed independence of all observations, when in fact biological groups such as species are differentially related to each other according to their evolutionary history. New phylogenetically based analytical methods were then rapidly developed, incorporated into ;the comparative method', and applied to many physiological, biochemical, morphological and behavioral investigations. We now review the rationale for including phylogenetic information in comparative studies and briefly discuss three methods for doing this (independent contrasts, generalized least-squares models, and Monte Carlo computer simulations). We discuss when and how to use phylogenetic information in comparative studies and provide several examples in which it has been helpful, or even crucial, to a comparative analysis. We also consider some difficulties with phylogenetically based statistical methods, and of comparative approaches in general, both practical and theoretical. It is our personal opinion that the incorporation of phylogeny information into comparative studies has been highly beneficial, not only because it can improve the reliability of statistical inferences, but also because it continually emphasizes the potential importance of past evolutionary history in determining current form and function.},
author = {Garland, Theodore and Bennett, Albert F and Rezende, Enrico L},
doi = {10.1242/jeb.01745},
isbn = {0022-0949},
issn = {0022-0949},
journal = {The Journal of experimental biology},
keywords = {allometry,comparative method,evolutionary physiology,generalized least-squares models,independent contrasts,model of evolution,phylogeny,statistical analysis},
month = {aug},
number = {Pt 16},
pages = {3015--3035},
pmid = {16081601},
title = {{Phylogenetic approaches in comparative physiology.}},
url = {http://www.ncbi.nlm.nih.gov/pubmed/16081601},
volume = {208},
year = {2005}
}
@article{Naisbit2011,
abstract = {Body mass is a fundamental characteristic that affects metabolism, life history, and population abundance and frequently sets bounds on who eats whom in food webs. Based on a collection of topological food webs, Ulrich Brose and colleagues presented a general relationship between the body mass of predators and their prey and analyzed how mean predator-prey body-mass ratios differed among habitats and predator metabolic categories. Here we show that the general body-mass relationship conceals significant variation associated with both predator and prey phylogeny. Major-axis regressions between the log body mass of predators and prey differed among taxonomic groups. The global pattern for Kingdom Animalia had slope {\textgreater} 1, but phyla and classes varied, and several had slopes significantly {\textless} 1. The predator-prey body-mass ratio can therefore decrease or increase with increasing body mass, depending on the taxon considered. We also found a significant phylogenetic signal in analyses of prey body-mass range for predators and predator body-mass range for prey, with stronger signal in the former. Besides providing insights into how characteristics of trophic interactions evolve, our results emphasize the need to integrate phylogeny to improve models of community structure and dynamics or to achieve a metabolic theory of food-web ecology.},
author = {Naisbit, Russell E. and Kehrli, Patrik and Rohr, Rudolf P. and Bersier, Louis F{\'{e}}lix},
doi = {10.1890/10-2234.1},
isbn = {10.1890/10-2234.1},
issn = {00129658},
journal = {Ecology},
keywords = {Allometry,Body mass,Body-size ratio,Food webs,Grafen's q,Metabolic theory of food-web ecology,Pagel's k,Predation,Trophic interactions},
month = {dec},
number = {12},
pages = {2183--2189},
pmid = {22352156},
title = {{Phylogenetic signal in predator-prey body-size relationships}},
url = {http://www.ncbi.nlm.nih.gov/pubmed/22352156},
volume = {92},
year = {2011}
}
@article{Plank2011,
abstract = {A stability analysis of the steady state of marine ecosystems is described. The study was motivated by the approximate invariance of biomass in logarithmic size intervals, which is widely observed in marine ecosystems. This invariance is recovered as the steady state of dynamic models of size spectra, which, unlike traditional species-based models of food webs, explicitly account for the mass gained by an individual organism when it eats a prey item. Little is known about the ecological conditions affecting the stability of the steady state, and a new method is developed to examine this. The results show that stability is enhanced by: (a) decreasing the mean predator-to-prey mass ratio (PPMR), (b) increasing the diet breadth of predators, (c) increasing the strength of intrinsic mortality relative to predation mortality, (d) increasing the biomass conversion efficiency. When perturbed from steady state, size spectra develop a wave-like shape, with an average wavelength especially sensitive to the mean PPMR. These waves move from small to large body size at an average speed which depends on the rate of growth of organisms. In contrast to traditional food web models, stability is enhanced as connectance (diet breadth) increases and as food chain length is increased by reducing the PPMR.},
author = {Plank, Michael John and Law, Richard},
doi = {10.1007/s12080-011-0137-x},
isbn = {1874-1738$\backslash$r1874-1746},
issn = {18741738},
journal = {Theoretical Ecology},
keywords = {Food web,Jump-growth equation,Power law,Predator-to-prey mass ratio,Size spectrum,Size-dependent predation},
month = {sep},
number = {4},
pages = {465--480},
title = {{Ecological drivers of stability and instability in marine ecosystems}},
url = {http://www.springerlink.com/index/10.1007/s12080-011-0137-x},
volume = {5},
year = {2011}
}
@article{Mouillot2008,
abstract = {The finding of invariant structures in species interaction webs is of central importance for ecology, with the greatest challenge remaining the elucidation of the processes governing these universal web patterns. Here we quantify the degree of intervality of seven fish-metazoan and 33 mammal-flea webs, i.e., the number of irreducible gaps in parasite diets along the host spectrum, and then challenge the idea that some invariant structures may emerge in host-parasite webs. Using a null model of random links between parasite and host species we find that empirical host-parasite webs exhibit a strong bias toward contiguity of parasite diet, i.e., toward intervality. Going one step further, we demonstrate that a null model with phylogenetic constraints on host-parasite links produced webs very similar to empirical ones, particularly when phylogenetic constraints occur at the family level, that is, when two hosts from the same family are more likely to be infected than two random hosts. In addition, we propose a new standardized measure of intervality which describes a novel "facet" of natural networks as it is independent of connectance or web size. We suggest using this measure as a surrogate of web maturity or saturation as phylogenetic constraints can drive webs toward intervality.},
author = {Mouillot, David and Krasnov, Boris R. and Poulin, Robert},
doi = {10.1890/07-1241.1},
isbn = {0012-9658},
issn = {00129658},
journal = {Ecology},
keywords = {Bipartite networks,Diet contiguity,Fish-metazoan webs,Mammal-flea webs,Null model,Web saturation},
month = {jul},
number = {7},
pages = {2043--2051},
pmid = {18705389},
title = {{High intervality explained by phylogenetic constraints in host parasite webs}},
url = {http://www.ncbi.nlm.nih.gov/pubmed/18705389},
volume = {89},
year = {2008}
}
@article{Nature2005,
abstract = {Nature guide to authors: Summary paragraph for Letters},
author = {Nature},
doi = {10.1107/S0365110X59001529},
isbn = {0365-110X},
issn = {0365110X},
journal = {Nature},
number = {May},
pages = {2005},
title = {{How to construct a Nature summary paragraph}},
volume = {118},
year = {2005}
}
@article{Legendre2013,
abstract = {Large-scale regional marine ecosystems can be compared for various processes that include their structure and biodiversity, functioning, services, and effects on biogeochemical processes. The comparisons can proceed from data up, or from conceptual models down, or from a combination of models and data. This study proposes a typology of methods and approaches that are currently used, or could possibly be used for making large-scale ecosystem comparisons. The various methods and approaches are illustrated with examples drawn from the literature. ?? 2011 Elsevier B.V.},
author = {Legendre, Louis and Niquil, Nathalie},
doi = {10.1016/j.jmarsys.2011.11.021},
isbn = {0924-7963},
issn = {09247963},
journal = {Journal of Marine Systems},
keywords = {Comparative studies,Ecosystems,Large-scale,Methodology,Regions,Typology},
month = {jan},
pages = {4--21},
publisher = {Elsevier B.V.},
title = {{Large-scale regional comparisons of ecosystem processes: Methods and approaches}},
url = {http://linkinghub.elsevier.com/retrieve/pii/S0924796311002880},
volume = {109-110},
year = {2013}
}
@article{Ives2010,
abstract = {We develop statistical methods for phylogenetic logistic regression in which the dependent variable is binary (0 or 1) and values are nonindependent among species, with phylogenetically related species tending to have the same value of the dependent variable. The methods are based on an evolutionary model of binary traits in which trait values switch between 0 and 1 as species evolve up a phylogenetic tree. The more frequently the trait values switch (i.e., the higher the rate of evolution), the more rapidly correlations between trait values for phylogenetically related species break down. Therefore, the statistical methods also give a way to estimate the phylogenetic signal of binary traits. More generally, the methods can be applied with continuous- and/or discrete-valued independent variables. Using simulations, we assess the statistical properties of the methods, including bias in the estimates of the logistic regression coefficients and the parameter that estimates the strength of phylogenetic signal in the dependent variable. These analyses show that, as with the case for continuous-valued dependent variables, phylogenetic logistic regression should be used rather than standard logistic regression when there is the possibility of phylogenetic correlations among species. Standard logistic regression does not properly account for the loss of information caused by resemblance of relatives and as a result is likely to give inflated type I error rates, incorrectly identifying regression parameters as statistically significantly different from zero when they are not.},
author = {Ives, Anthony R. and Garland, Theodore},
doi = {10.1093/sysbio/syp074},
isbn = {1063-5157},
issn = {10635157},
journal = {Systematic Biology},
keywords = {Analysis of covariance,Ancestor reconstruction,Comparative methods,Generalized least squares,Independent contrasts,Morphometrics,Phylogeny,Regression for binary outcomes},
month = {jan},
number = {1},
pages = {9--26},
pmid = {20525617},
title = {{Phylogenetic logistic regression for binary dependent variables}},
url = {http://www.ncbi.nlm.nih.gov/pubmed/20525617},
volume = {59},
year = {2010}
}
@article{Stouffer2012,
abstract = {Studies of ecological networks (the web of interactions between species in a community) demonstrate an intricate link between a community's structure and its long-term viability. It remains unclear, however, how much a community's persistence depends on the identities of the species present, or how much the role played by each species varies as a function of the community in which it is found. We measured species' roles by studying how species are embedded within the overall network and the subsequent dynamic implications. Using data from 32 empirical food webs, we find that species' roles and dynamic importance are inherent species attributes and can be extrapolated across communities on the basis of taxonomic classification alone. Our results illustrate the variability of roles across species and communities and the relative importance of distinct species groups when attempting to conserve ecological communities.},
author = {Stouffer, D. B. and Sales-Pardo, M. and Sirer, M. I. and Bascompte, J.},
doi = {10.1126/science.1216556},
isbn = {0036-8075},
issn = {0036-8075},
journal = {Science},
number = {6075},
pages = {1489--1492},
pmid = {22442483},
title = {{Evolutionary conservation of species' roles in food webs}},
volume = {335},
year = {2012}
}
@article{Yodzis2001,
abstract = {If humankind occupies the ecological role of top predator, then within individual ecosystems it must compete with other top predators for valuable food resources. This notion presents a fascinating ecological problem, with tremendous social and economic ramifications. Because of the socioeconomic dimension, the scientific debate has at times been controversial. The recent attention and data gathering resources focused on the problem present a unique opportunity to test and refine ecological theory in the arena of complex, large-scale systems.},
author = {Yodzis, Peter},
doi = {10.1016/S0169-5347(00)02062-0},
isbn = {0169-5347},
issn = {01695347},
journal = {Trends in Ecology and Evolution},
month = {feb},
number = {2},
pages = {78--84},
pmid = {11165705},
title = {{Must top predators be culled for the sake of fisheries?}},
url = {http://www.ncbi.nlm.nih.gov/pubmed/11165705},
volume = {16},
year = {2001}
}
@article{Lafferty2006,
abstract = {Parasitism is the most common animal lifestyle, yet food webs rarely include parasites. The few earlier studies have indicated that including parasites leads to obvious increases in species richness, number of links, and food chain length. A less obvious result was that adding parasites slightly reduced connectance, a key metric considered to affect food web stability. However, reported reductions in connectance after the addition of parasites resulted from an inappropriate calculation. Two alternative corrective approaches applied to four published studies yield an opposite result: parasites increase connectance, sometimes dramatically. In addition, we find that parasites can greatly affect other food web statistics, such as nestedness (asymmetry of interactions), chain length, and linkage density. Furthermore, whereas most food webs find that top trophic levels are least vulnerable to natural enemies, the inclusion of parasites revealed that mid-trophic levels, not low trophic levels, suffered the highest vulnerability to natural enemies. These results show that food webs are very incomplete without parasites. Most notably, recognition of parasite links may have important consequences for ecosystem stability because they can increase connectance and nestedness.},
author = {Lafferty, Kevin D. and Dobson, Andrew P and Kuris, Armand M},
doi = {10.1073/pnas.0604755103},
isbn = {0027-8424 1091-6490},
issn = {0027-8424},
journal = {Proceedings of the National Academy of Sciences},
keywords = {Animals,Biodiversity,Biological,Biomass,Ecology,Ecology: methods,Ecosystem,Evolution,Food Chain,Models,Parasites,Parasites: physiology,Predatory Behavior},
month = {jul},
number = {30},
pages = {11211--6},
pmid = {16844774},
title = {{Parasites dominate food web links}},
url = {http://www.ncbi.nlm.nih.gov/pubmed/16844774},
volume = {103},
year = {2006}
}
@article{Bascompte2005,
abstract = {There are two common approaches to food webs. On the one hand, empirical studies have described aggregate statistical measures of many-species food webs. On the other hand, theoretical studies have explored the dynamic properties of simple tri-trophic food chains (i.e., trophic modules). The question remains to what extent results based on simple modules are relevant for whole food webs. Here we bridge between these two independent research agendas by exploring the relative frequency of different trophic mod- ules in the five most resolved food webs. While apparent competition and intraguild pre- dation are overrepresented when compared to a suite of null models, the frequency of omnivory highly varies across communities. Inferences about the representation of modules may also depend on the null model used for statistical significance. Key},
author = {Bascompte, Jordi and Meli{\'{a}}n, Carlos J.},
doi = {10.1890/05-0101},
isbn = {0012-9658},
issn = {00129658},
journal = {Ecology},
keywords = {Apparent competition,Complex networks,Food webs,Food-web models,Intraguild predation,Network motifs,Omnivory},
number = {11},
pages = {2868--2873},
pmid = {711},
title = {{Simple trophic modules for complex food webs}},
url = {http://www.esajournals.org/doi/pdf/10.1890/05-0101},
volume = {86},
year = {2005}
}
@article{Sanchez2012,
abstract = {We constructed the food webs of six Mediterranean streams in order to determine ecological generalities derived from analysis of their structure and to explore stabilizing forces within these ecosystems. Fish, macroinvertebrates, primary producers and detritus are the components of the studied food webs. Analysis focused on a suite of food web properties that describe species' trophic habits, linkage complexity and food chains. A great structural similarity was found in analyzed food webs; we therefore suggest average values for the structural properties of Mediterranean stream food webs. Percentage of omnivorous species was positively correlated with connectance, and there was a predominance of intermediate trophic level species that had established simple links with detritus. In short, our results suggest that omnivory and the weak interactions of detritivores have a stabilizing role in these food webs.},
author = {S{\'{a}}nchez-Carmona, R. and Encina, L. and Rodr{\'{i}}guez-Ruiz, a. and Rodr{\'{i}}guez-S{\'{a}}nchez, M. V. and Granado-Lorencio, C.},
doi = {10.1007/s10452-012-9400-5},
isbn = {1045201294},
issn = {13862588},
journal = {Aquatic Ecology},
keywords = {Food webs,Mediterranean stream,Structural properties,Trophic interactions},
month = {may},
number = {3},
pages = {311--324},
title = {{Food web structure in Mediterranean streams: Exploring stabilizing forces in these ecosystems}},
url = {http://www.springerlink.com/index/10.1007/s10452-012-9400-5},
volume = {46},
year = {2012}
}
@article{Jiang2010,
abstract = {One of the oldest ideas in invasion biology, known as Darwin's naturalization hypothesis, suggests that introduced species are more successful in communities in which their close relatives are absent. We conducted the first experimental test of this hypothesis in laboratory bacterial communities varying in phylogenetic relatedness between resident and invading species with and without a protist bacterivore. As predicted, invasion success increased with phylogenetic distance between the invading and the resident bacterial species in both the presence and the absence of protistan bacterivory. The frequency of successful invader establishment was best explained by average phylogenetic distance between the invader and all resident species, possibly indicating limitation by the availability of the unexploited niche (i.e., organic substances in the medium capable of supporting the invader growth); invader abundance was best explained by phylogenetic distance between the invader and its nearest resident relative, possibly indicating limitation by the availability of the unexploited optimal niche (i.e., the subset of organic substances supporting the best invader growth). These results were largely driven by one resident bacterium (a subspecies of Serratia marcescens) posting the strongest resistance to the alien bacterium (another subspecies of S. marcescens). Overall, our findings support phylogenetic relatedness as a useful predictor of species invasion success.},
author = {Jiang, Lin and Tan, Jiaqi and Pu, Zhichao},
doi = {10.1086/650720},
isbn = {0003-0147},
issn = {1537-5323},
journal = {The American naturalist},
keywords = {Ecosystem,Gram-Positive Bacteria,Phylogeny,Population Dynamics,Serratia marcescens,Tetrahymena pyriformis},
month = {apr},
number = {4},
pages = {415--23},
pmid = {20170339},
title = {{An experimental test of Darwin's naturalization hypothesis.}},
url = {http://www.ncbi.nlm.nih.gov/pubmed/20170339},
volume = {175},
year = {2010}
}
@article{MacArthur1963,
abstract = {A graphical equilibrium model, balancing immigration and extinction rates of species, has been developed which appears fully consistent with the fauna-area curves and the distance effect seen in land and freshwater bird faunas of the Indo-Australian islands. The establishment of the equilibrium condition allows the development of a more precise zoogeographic theory than hitherto possible. One new and non-obvious prediction can be made from the model which is immediately verifiable from existing data, that the number of species increases with area more rapidly on far islands than on near ones. Similarly, the number of species on large islands decreases with distance faster than does the number of species on small islands. As groups of islands pass from the unsaturated to saturated conditions, the variance-to-mean ratio should change from unity to about one-half. When the faunal buildup reaches 90{\%} of the equilibrium number, the extinction rate in species/year should equal 2.303 times the variance divided by the time (in years) required to reach the 90{\%} level. The implications of this relation are discussed with reference to the Krakatau faunas, where the buildup rate is known. A 'radiation zone,' in which the rate of intra-archipelagic exchange of autochthonous species approaches or exceeds extraarchipelagic immigration toward the outer limits of the taxon's range, is predicted as still another consequence of the equilibrium condition. This condition seems to be fulfilled by conventional information but cannot be rigorously tested with the existing data. Where faunas are at or near equilibrium, it should be possible to devise indirect estimates of the actual immigration and extinction rates, as well as of the times required to reach equilibrium. It should also be possible to estimate the mean dispersal distance of propagules overseas from the zoogeographic data. Mathematical models have been constructed to these ends and certain applications suggested. The main purpose of the paper is to express the criteria and implications of the equilibrium condition, without extending them for the present beyond the IndoAustralian bird faunas. CR - Copyright {\&}{\#}169; 1963 Society for the Study of Evolution},
archivePrefix = {arXiv},
arxivId = {1011.1669v3},
author = {MacArthur, Robert H. and Wilson, Edward O.},
doi = {10.4037/ajcc2016979},
eprint = {1011.1669v3},
isbn = {9788578110796},
issn = {10623264},
journal = {The American Naturalist},
number = {4},
pages = {373--387},
pmid = {25246403},
title = {{The equilibrium theory of insular zoogeography}},
url = {http://www.jstor.org/stable/2407089},
volume = {14},
year = {1963}
}
@article{Stouffer2007,
abstract = {Food webs aim to provide a thorough representation of the trophic interactions found in an ecosystem. The complexity of empirical food webs, however, is leading many ecologists to focus dynamic ecosystem studies on smaller microcosm or mesocosm studies based upon community modules, which comprise three to five species and the interactions likely to have ecological relevance. We provide here a structural counterpart to community modules. We investigate food-web 'motifs' which are n-species connected subgraphs found within the food web. Remarkably, we find that the over- and under-representation of three-species motifs in empirical food webs can be understood through comparison to a static food-web model, the niche model. Our result conclusively demonstrates that predation upon species with some 'characteristic' niche value is the prey selection mechanism consistent with the structural properties of empirical food webs.},
author = {Stouffer, Daniel B and Camacho, Juan and Jiang, Wenxin and Amaral, Lu{\'{i}}s A. Nunes},
doi = {10.1098/rspb.2007.0571},
file = {:Users/alyssacirtwill/Documents/Papers/Stouffer et al.{\_}2007{\_}Proceedings of the Royal Society B Biological Sciences.pdf:pdf},
isbn = {0962-8452},
issn = {0962-8452},
journal = {Proceedings of the Royal Society B: Biological Sciences},
keywords = {Animals,Biological,Ecosystem,Food Chain,Models,Predatory Behavior},
month = {aug},
number = {1621},
pages = {1931--1940},
pmid = {17567558},
title = {{Evidence for the existence of a robust pattern of prey selection in food webs.}},
url = {http://www.ncbi.nlm.nih.gov/pubmed/17567558{\%}5Cnhttp://www.pubmedcentral.nih.gov/articlerender.fcgi?artid=PMC2275185},
volume = {274},
year = {2007}
}
@article{Twomey1945,
abstract = {See full-text article at JSTOR},
author = {Twomey, Arthur C},
issn = {00129615},
journal = {Ecological Monographs},
number = {2},
pages = {173--205},
title = {{The bird population of an Elm-Maple forest with special reference to aspection, territorialism, and coactions}},
volume = {15},
year = {1945}
}
@article{Cromar1996,
author = {Monographs, Source Ecological and Feb, No and Tavares-cromar, Annette F and Williams, D Dudley},
journal = {Ecological Monographs},
keywords = {aquatic macroinvertebrates},
number = {1},
pages = {91--113},
title = {{The Importance of Temporal Resolution in Food Web Analysis : Evidence from a Detritus-Based Stream THE IMPORTANCE OF TEMPORAL RESOLUTION IN FOOD WEB ANALYSIS : EVIDENCE FROM A DETRITUS-BASED STREAM1}},
url = {http://www.jstor.org/stable/10.2307/2963482},
volume = {66},
year = {2012}
}
@article{Thompson2005a,
abstract = {Thompson, R. M. and Townsend, C. R. 2005. Energy availability, spatial heterogeneity and ecosystem size predict food-web structure in streams. {\'{A}}/ Oikos 108: 137 {\'{A}}/148. We used standardized techniques to assemble eighteen food webs in streams. Our aim was to identify the determinants of food-web structure with particular reference to energy availability (related to land use), spatial heterogeneity and ecosystem size (both independent of land use). Forested streams displayed lower algal productivity and higher standing crops of organic matter than the grassland streams. The organic matter in the pine streams was probably of lower quality than that elsewhere. Measures of energy availability and spatial heterogeneity predicted species richness and connectance. A combination of energy availability, spatial heterogeneity and ecosystem size accounted for the representation of particular invertebrate feeding groups in the streams. Algal production and organic matter standing crop were important determinants of invertebrate biomass and overall food-web structure. Grassland sites showed a positive relationship between algal productivity and food chain length whereas forest sites displayed a positive relationship between ecosystem size and food chain length. Therefore, these results provide support for both Pimm's productivity hypothesis and Cohen and Newman's ecosystem size hypothesis.},
author = {Thompson, R. M. and Townsend, C. R.},
doi = {10.1111/j.0030-1299.2005.11600.x},
isbn = {0030-1299},
issn = {00301299},
journal = {Oikos},
number = {1},
pages = {137--148},
pmid = {6705},
title = {{Energy availability, spatial heterogeneity and ecosystem size predict food-web structure in streams}},
volume = {108},
year = {2005}
}
@article{Thompson2003,
abstract = {Native pine-forest streams from Maine and North Carolina, USA, and exotic pine-forest streams from New Zealand were compared to assess the effects of geographic location on three aspects of community structure: (1) taxonomic composition, (2) trophic structure (summarized in terms of functional feeding groups), and (3) food web structure (as connectivity food webs). In addition, pine-forest assemblages in New Zealand were compared to assemblages from New Zealand native forest and grassland streams. Taxonomic similarity was, as expected, low for invertebrates, but there were strong similarities in the algal assemblages in different geographic locations. Trophic structure analysis was unable to distinguish either geographic or land-use effects. Food web analysis revealed structural similarities between the pine-forest streams, regardless of location, but there were clear differences among land uses in New Zealand. Pine-forest streams were typified by food webs with few algal species, low internal connectance and a relatively square shape. Grass- land food webs were more triangular in shape and exhibited high internal connectance, while native forest food webs had intermediate characteristics. The results show that the native stream biota, despite a distinct species composition, can adapt to a novel riparian vegetation type and produce trophic and food web structures that are difficult to distinguish from those in the country of origin.},
author = {Thompson, Author R M and Townsend, C R},
journal = {Ecology},
keywords = {biogeography,exotic vs,food web,invasions,maine,native forests,new,north carolina,streams,usa,zealand},
number = {1},
pages = {145--161},
title = {{Impacts on Stream Food Webs of Native and Exotic Forest : An Intercontinental Comparison Published by : Ecological Society of America IMPACTS ON STREAM FOOD WEBS OF NATIVE AND EXOTIC FOREST : AN INTERCONTINENTAL COMPARISON}},
url = {http://www.esajournals.org/doi/pdf/10.1890/0012-9658(2003)084{\%}5B0145:IOSFWO{\%}5D2.0.CO{\%}3B2},
volume = {84},
year = {2010}
}
@article{Society2011c,
abstract = {applicability for this approach.},
archivePrefix = {arXiv},
arxivId = {arXiv:1011.1669v3},
author = {Hess, N. C L and Carlson, D. J. and Inder, J. D. and Jesulola, E. and Mcfarlane, J. R. and Smart, N. a.},
doi = {10.1017/CBO9781107415324.004},
eprint = {arXiv:1011.1669v3},
isbn = {9788578110796},
issn = {08628408},
journal = {Physiological Research},
keywords = {Blood pressure,Hypertension,Isometric exercise},
number = {3},
pages = {461--468},
pmid = {25246403},
title = {{Clinically meaningful blood pressure reductions with low intensity isometric handgrip exercise. A randomized trial}},
volume = {65},
year = {2016}
}
@article{Polis1991,
abstract = {Food webs in the real world are much more complex than food-web literature would have us believe. This is illustrated by the web of the sand community in the Coachella Valley desert. The biota include 174 species of vascular plants, 138 species of vertebrates, more than 55 species of arachnids, and an unknown (but great) number of microorganisms, insects (2,000-3,000 estimated species), acari, and nematodes. Trophic relations are presented in a series of nested subwebs and delineations of the community. Complexity arises from the large number of interactive species, the frequency of omnivory, age structure, looping, the lack of compartmentalization, and the complexity of the arthropod and soil faunas. Web features found in the Coachella also characterize other communities and should produce equivalently complex webs. If anything, diversity and complexity in most nondesert habitats are greater than those in deserts. Patterns from the Coachella web are compared with theoretical predictions and "empirical generalizations" derived from catalogs of published webs. The Coachella web differs greatly: chains are longer, omnivory and loops are not rare, connectivity is greater (species interact with many more predators and prey), top predators are rare or nonexistent, and prey-to-predator ratios are greater than 1.0. The evidence argues that actual community food webs are extraordinarily more complex than those webs cataloged by theorists. I argue that most cataloged webs are oversimplified caricatures of actual communities. That cataloged webs depict so few species, absurdly low ratios of predators on prey and prey eaten by predators, so few links, so little omnivory, a veritable absence of looping, and such a high proportion of top predators argues strongly that they poorly represent real biological communities. Consequently, the practice of abstracting empirical regularities from such catalogs yields an inaccurate and artifactual view of trophic interactions within communities. Contrary to strong assertions by many theorists, patterns from food webs of real communities generally do not support predictions arising from dynamic and graphic models of food-web structure.},
author = {{Polis Gary A.}},
doi = {10.1086/285208},
isbn = {0003-0147},
issn = {0003-0147},
journal = {The American Naturalist},
number = {1},
pages = {123},
pmid = {2445},
title = {{Complex trophic interactions in deserts : an empirical critique of food-web theory}},
url = {http://www.journals.uchicago.edu/doi/10.1086/285208},
volume = {138},
year = {1991}
}
@article{Paviour-Smith1956,
abstract = {The results of seasonal studies of the fauna of a salt meadow on the Otago Peninsula are given. The plant community consisted of 4 species of plants; their average height is about 1 cm, and they form a dense peaty turf about 20 cm deep on the top of pure sand. Faunal samples taken from sea to dry land showed considerable variation. The mesofauna was at its maximum in late summer (February) and at its minimum in winter (July to September). The microfauna changed in the opposite sense. A food web was constructed to indicate the trophic levels in the community. The maximum number of animals found in the salt meadow was approximately 7½ million per square metre, when the microfauna was at its maximum. Zoomasses tended to show an inverse relationship to the pyramid of numbers and the maximum zoomass per square metre, approximately 32 g occurred when the microfauna was at its maximum.},
author = {Paviour-Smith, K},
journal = {Trans. R. Soc. NZ},
number = {3},
pages = {525--554},
title = {{The biotic community of a salt meadow in New Zealand}},
url = {http://rsnz.natlib.govt.nz/volume/rsnz{\_}83/rsnz{\_}83{\_}03{\_}008510.pdf},
volume = {83},
year = {1956}
}
@book{Opitz1996,
address = {Manila},
author = {Opitz, Silvia},
isbn = {9718709606},
pages = {356},
publisher = {International Center for Living Aquatic Resources Management},
title = {{Trophic Interactions in Caribbean Coral Reefs}},
year = {1996}
}
@article{Percival2011,
author = {Percival, E and Whitehead, H},
doi = {10.2307/2256044},
isbn = {00220477},
issn = {00220477},
journal = {Journal of Ecology},
number = {2},
pages = {282--314},
title = {{A Quantitative Study of the Fauna of Some Types of Stream-Bed}},
volume = {17},
year = {1929}
}
@article{Memmott2000,
author = {Memmott, J and Martinez, N D and Cohen, J E},
journal = {Animal Ecology},
keywords = {food-webs},
number = {1},
pages = {1--15},
title = {{Predators, parasites and pathogens: species richness, trophic generality, and body sizes in a natural food web}},
url = {http://onlinelibrary.wiley.com/doi/10.1046/j.1365-2656.2000.00367.x/full},
volume = {69},
year = {2000}
}
@article{Koslucher1973,
author = {Koslucher, Dale G. and Minshall, G. Wayne},
doi = {10.2307/3225248},
issn = {00030023},
journal = {Transactions of the American Microsopial Socienty},
month = {jul},
number = {3},
pages = {441--452},
title = {{Food Habits of Some Benthic Invertebrates in a Northern Cool-Desert Stream (Deep Creek, Curlew Valley, Idaho-Utah)}},
url = {http://www.jstor.org/stable/3225248?origin=crossref},
volume = {92},
year = {1973}
}
@incollection{Jonsson2005,
abstract = {This chapter demonstrates that methods to describe ecological communities can be better understood, and can reveal new patterns, by labeling each species that appears in a community's food web with the numerical abundance and average body size of individuals of that species. We illustrate our new approach, and relate it to previous approaches, by analyzing data from the pelagic community of a small lake, Tuesday Lake, in Michigan. Although many of the relationships we describe have been well studied individually, we are not aware of any single community for which all of these relationships have been analyzed simultaneously. An overview of some of the results of the present study, with further theoretical extensions, has been published elsewhere (Cohen et al., 2003). Our new approach yields four major results. Though many patterns in the structure of an ecological community have been traditionally treated as independent, they are in fact connected. In at least one real ecosystem, many of these patterns are relatively robust after a major perturbation. Some of these patterns may be predictably consistent from one community to another. Locally, however, some community characteristics need not necessarily coincide with previously reported patterns for guilds or larger geographical scales. We describe our major findings under these headings: trivariate relationships (that is, relationships combining the food web, body size, and species abundance); bivariate relationships; univariate relationships; and the effects of food web perturbation. Trivariate Relationships: Species with small body mass occur low in the food web of Tuesday Lake and are numerically abundant. Larger-bodied species occur higher in the food web and are less numerically abundant. Body size explains more of the variation in numerical abundance than does trophic height. Body mass varies almost 12 orders of magnitude and numerical abundance varies by almost 10 orders of magnitude, but biomass abundance (the product of body mass times numerical abundance) varies by far less, about 5 orders of magnitude. The nearly inverse relationship between body mass and numerical abundance, and the relative constancy of biomass, are illustrated by a new food web graph (Fig. 3), which shows the food web in the plane with axes corresponding to body mass and numerical abundance. Bivariate Relationships: The pelagic community of Tuesday Lake shows a pyramid of numbers but not a pyramid of biomass. The biomass of species increases very slowly with increasing body size, by only 2 orders of magnitude as body mass increases by 12 orders of magnitude. The biomass-body size spectrum is roughly flat, as in other studies at larger spatial scales. Prey body mass is positively correlated to predator body mass. Prey abundance and predator abundance are positively correlated for numerical abundance but not for biomass abundance. Body size and trophic height are positively correlated. Body size and numerical abundance are negatively correlated. The slope of the linear regression of log numerical abundance as a function of log body size in Tuesday Lake is not significantly different from -3{\{}plus 45 degree rule{\}}4 across all species but is significantly greater than -1 at the 5{\%} significance level. This -3{\{}plus 45 degree rule{\}}4 slope is similar to that found in studies at larger, regional scales, but different from that sometimes observed at local scales. The slope within the phytoplankton and zooplankton (each group considered separately) is much less steep than -3{\{}plus 45 degree rule{\}}4, which is in agreement with an earlier observation that the slope tends to be more negative as the range of body masses of the organisms included in a study increases. A novel combination of the food web with data on body size and numerical abundance, together with an argument based on energetic mechanisms, refines and tightens the relationship between numerical abundance and body size. The regression of log body mass as a linear function of log numerical abundance across all species has a slope not significantly different from -1, but significantly less than -3{\{}plus 45 degree rule{\}}4. The estimated slope is significantly different from the reciprocal of the estimated slope of log numerical abundance as a function of log body mass. Thus, if log body mass is viewed as an independent variable and log numerical abundance is viewed as a dependent variable, the slope of the linear relationship could be -3{\{}plus 45 degree rule{\}}4 but could not be -1 at the 5{\%} significance level. Conversely, if log numerical abundance is viewed as an independent variable and log body mass as a dependent variable, the slope of the linear relationship could be -1 but could not be -4{\{}plus 45 degree rule{\}}3 (which is the reciprocal of -3{\{}plus 45 degree rule{\}}4) at the 5{\%} significance level. While a linear relationship is a good approximation in both cases, Cohen and Carpenter (in press) showed that only the model with log body mass as the independent variable meets the assumptions of linear regression analysis for these data. Univariate Relationships: The food web of Tuesday Lake has a pyramidal trophic structure. The number of trophic links between species in nearby trophic levels is higher than would be expected if trophic links were distributed randomly among the species. Food chains are shorter than would be expected if links were distributed randomly. Species low in the food web tend to have more predators and fewer prey than species high in the web. The distribution of body size is right-log skewed. The rank-numerical abundance relationship is approximately broken-stick within phytoplankton and zooplankton while the rank-biomass abundance relationship is approximately log-normal across all species. The slope of the right tail of the body mass distribution is much less steep than has been suggested for regional scales and not log-uniform as found at local scales for restricted taxonomic groups. Effect of Food Web Perturbation: The data analyzed here were collected in 1984 and 1986. In 1985, three species of planktivorous fishes were removed and one species of piscivorous fish was introduced. The data reveal some differences between 1984 and 1986 in the community's species composition and food web. Most other community characteristics seem insensitive to this major manipulation. Different fields of ecology have focused on different subsets of the bivariate relationships illustrated here. Integration of the relationships as suggested in this chapter could bring these fields closer. The new descriptive data structure (food web plus numerical abundance and body size of each species) can promote the integration of food web studies with, for example, population biology and biogeochemistry. {\textcopyright} 2005 Elsevier Inc. All rights reserved.},
author = {Jonsson, Tomas and Cohen, Joel E. and Carpenter, Stephen R.},
booktitle = {Advances in Ecological Research},
chapter = {1},
doi = {10.1016/S0065-2504(05)36001-6},
editor = {Caswell, H},
isbn = {0120139367},
issn = {00652504},
pages = {1--84},
pmid = {917},
publisher = {Elsevier Ltd},
title = {{Food Webs, Body Size, and Species Abundance in Ecological Community Description}},
volume = {36},
year = {2005}
}
@article{Jaarsma1998,
abstract = {The food-web communities of two stony tributaries of the Taieri River in New Zealand were documented. We placed heavy emphasis on algal and macroinvertebrate taxonomy, identifying most taxa to species or morpho-species level. Food-web attributes were derived from symmetrical matrices using an Excel macro. The values of the food-web attributes are generally consistent with generalities that have previously been reported in food-web studies, although some hypothesisied relationships between connectance and food-web size did not hold. The patterns detected were robust to the inclusion or exclusion of species that were identified in gut contents but not in field samples. The attributes strongly reflected the trophic status of the streams. Dempsters Creek was classified as autotrophic (based on high primary production and a high ratio of production to respiration) and displayed a larger food-web size and longer food-chains than those seen in Healy Creek, which was classified as hetero-trophic (lower primary production and production to respiration ratio). The importance of dynamic environmental attributes such as production and disturbance in structuring food-webs is emphasised in this study.},
author = {Jaarsma, N G and {De Boer}, S M and Townsend, C R and Thompson, R M and Edwards, E D and Conservancy, Otago},
doi = {10.1080/00288330.1998.9516825},
issn = {0028-8330},
journal = {New Zealand Journal of Marine and Freshwater Research},
keywords = {community patterns,disturbance,food-webs,primary production,stream algae,stream invertebrates},
number = {32},
pages = {271--286},
pmid = {159},
title = {{Characterising food-webs in two New Zealand streams}},
url = {http://www.tandfonline.com/action/journalInformation?journalCode=tnzm20{\%}5Cnhttp://dx.doi.org/10.1080/00288330.1998.9516825},
volume = {32},
year = {1998}
}
@article{Huxham1996,
abstract = {JSTOR is a not-for-profit service that helps scholars, researchers, and students discover, use, and build upon a wide range of content in a trusted digital archive. We use information technology and tools to increase productivity and facilitate new forms of scholarship. For more information about JSTOR, please contact support@jstor.org. This content downloaded from 90.218.33.234 on Wed, 09 Dec 2015 19:42:02 UTC All use subject to JSTOR Terms and Conditions},
author = {Huxham, M and Beaney, S and Raffaelli, D},
doi = {10.2307/3546201},
isbn = {0030-1299},
issn = {00301299},
journal = {Source: Oikos},
number = {2},
pages = {284--300},
title = {{Nordic Society Oikos Do Parasites Reduce the Chances of Triangulation in a Real Food Web?}},
url = {http://www.jstor.org/stable/3546201{\%}5Cnhttp://www.jstor.org/stable/3546201?seq=1{\&}cid=pdf-reference{\#}references{\_}tab{\_}contents{\%}5Cnhttp://www.jstor.org/page/},
volume = {76},
year = {1996}
}
@article{Hechinger2011a,
abstract = {This data set presents food webs for three North American Pacific coast estuaries and a ‘‘Metaweb'' composed of the species/stages compiled from all three estuaries. The webs have four noteworthy attributes: (1) parasites (infectious agents), (2) body-size information, (3) biomass information, and (4) ontogenetic stages of many animals with complex life cycles. The estuaries are Carpinteria Salt Marsh, California (CSM); Estero de Punta Banda, Baja California (EPB); and Bahı´a Falsa in Bahı´a San Quintı´ n, Baja California (BSQ). Most data on species assemblages and parasitism were gathered via consistent sampling that acquired body size and biomass information for plants and animals larger than ;1 mm, and for many infectious agents (mostly metazoan parasites, but also some microbes). We augmented this with information from additional published sources and by sampling unrepresented groups (e.g., plankton). We estimated free-living consumer–resource links primarily by extending a previously published version of the CSM web (which the current CSM web supplants) and determined most parasite consumer–resource links from direct observation. We recognize 21 possible link types including four general interactions: predators consuming prey, parasites consuming hosts, predators consuming parasites, and parasites consuming parasites. While generally resolved to the species level, we report stage-specific nodes for many animals with complex life cycles. We include additional biological information for each node, such as taxonomy, lifestyle (free-living, infectious, commensal, mutualist), mobility, and residency. The Metaweb includes 500 nodes, 314 species, and 11 270 links projected to be present given appropriate species' co-occurrences. Of these, 9247 links were present in one or more of the estuarine webs. The remaining 2023 links were not present in the estuaries but are included here because they may occur in other places or times. Initial analyses have examined and are examining the interrelationships among consumer strategy, body size, abundance, biomass, trophic level, life stages, and food-web structure and dynamics. Further use of these data may enable a more general exploration how infectious processes and parasites impact communities and ecosystems. Additionally, we present the data and metadata in a standardized format, attempting to provide a system-neutral template for future food-web assembly and publication.},
author = {Hechinger, Ryan F. and Lafferty, Kevin D. and McLaughlin, John P. and Fredensborg, Brian L. and Huspeni, Todd C. and Lorda, Julio and Sandhu, Parwant K. and Shaw, Jenny C. and Torchin, Mark E. and Whitney, Kathleen L. and Kuris, Armand M.},
doi = {10.1890/10-1383.1},
isbn = {0012-9658},
issn = {0012-9658},
journal = {Ecology},
number = {3},
pages = {791--791},
title = {{Food webs including parasites, biomass, body sizes, and life stages for three California/Baja California estuaries}},
volume = {92},
year = {2011}
}
@article{Havens1992,
abstract = {The degree to which widely accepted generalizations about food web structure apply to natural communities was determined through examination of 50 pelagic webs sampled consistently with even taxonomic resolution of all trophic levels. The fraction of species in various trophic categories showed no significant overall trends as the number of species varied from 10 to 74. In contrast, the number of links per species increased fourfold over the range of species number, suggesting that the link-species scaling law, defined on the basis of aggregated webs, does not reflect a real ecological trend.},
author = {Havens, K},
doi = {10.1126/science.257.5073.1107},
isbn = {1095-9203 (Electronic)$\backslash$r0036-8075 (Linking)},
issn = {0036-8075},
journal = {Science},
number = {5073},
pages = {1107--9},
pmid = {17840281},
title = {{Scale and structure in natural food webs.}},
url = {http://www.ncbi.nlm.nih.gov/pubmed/17840281},
volume = {257},
year = {1992}
}
@article{Goldwasser1993,
abstract = {JSTOR is a not-for-profit service that helps scholars, researchers, and students discover, use, and build upon a wide range of content in a trusted digital archive. We use information technology and tools to increase productivity and facilitate new forms of scholarship. Abstract. We document the construction of a relatively large food web (44 species) from the island of St. Martin in the northern Lesser Antilles, and compare it with patterns observed in other, generally smaller food webs. In constructing this web, we integrate data from a variety of studies, many of which focussed on Anolis lizards and their vertebrate predators. In addition to determining the links between predators and prey, we estimate the frequencies of predation (the link strengths), and find an approximately bell-shaped distribution with a majority of links of intermediate frequencies. Some of the properties of this web contrast strongly with those of webs in the ECOWeB compilation. In particular, our analysis shows this web to possess an unusual richness of intermediate species (relative to top predators or basal species) and of links between those intermediate species. The number and lengths of chains are also unusually high, as is the degree of omnivory. Nor does this web match the predictions of the cascade model, which predicts even higher proportions of intermediate species and links between them, and even more numerous chains. It appears that these and other differences are not due simply to the large number of species involved here, but it is not yet clear whether they should be ascribed to the completeness with which some of the diets are known, to differences between the ways this and other webs were constructed, or to unique ecological conditions on the island of St. Martin.},
author = {Goldwasser, Lloyd and Roughgarden, Joan},
doi = {10.2307/1940492},
issn = {00129658},
journal = {Ecology},
keywords = {ECOWeb,ecology,food web,interaction,predation,strength,terrestrial},
number = {4},
pages = {1216----1233.},
title = {{Construction and analysis of a large caribbean food web}},
url = {http://www.jstor.org/stable/1940492{\%}5Cnhttp://www.jstor.org/page/info/about/policies/terms.jsp{\%}5Cnhttp://www.jstor.org{\%}5Cnhttp://www.jstor.org/discover/10.2307/1940492?uid=3738296{\&}uid=2134{\&}uid=2{\&}uid=70{\&}uid=4{\&}sid=21102482841821},
volume = {74},
year = {1993}
}
@article{Fryer1959,
author = {Fryer, Geoffrey and Sc, B and Ph, D},
journal = {Proceedings of the Zoological Society of London},
number = {2},
pages = {153--281},
title = {{the Trophic Interrelationships and Ecology O F Some Littoral Communities O F Lake Nyasa With Especial Reference To the Fishes , and a Discussion O F the Evolution O F a Group O F Rock-Frequenting Cichlidae}},
volume = {132},
year = {1957}
}
@article{Christian1999,
abstract = {Trophic structure of ecosystems is a unifying concept in ecology; however, the quantification of trophic level of individual components has not received the attention one might expect. Ecosystem network analysis provides a format to make several assessments of trophic structure of communities, including the effective trophic level (i.e. non-integer) of these components. We applied network analysis to a Halodule wrightii community in Goose Creek Bay, St. Marks National Wildlife Refuge, Florida, USA, during January and February 1994 where we sampled a wide variety of taxa. Unlike most applications of network analysis, the field sampling design was specific for network construction. From these data and literature values, we constructed and analyzed one of the most complex, highly articulated and site specific foodweb networks to be done. Care was taken to structure the network to reflect best the field data and ecology of populations within the requirements of analysis software. This involved establishing internally consistent rules of data manipulation and compartment aggregation. Special attention was paid to the microbial components of the food web. Consumer compartments comprised effective trophic levels from 2.0 (herbivore/detritivore) to 4.32 (where a level 4.0 represents 'secondary carnivory'), and these values were used to organize data interpretation. The effective trophic levels of consumers tended to aggregate near integer values, but the spread from integer values increased with increasing level. Detritus and benthic microalgae acted as important sources of food in the extended diets of many consumers. 'Bottom-up' control appeared important through mixed trophic impact analysis, and the extent of positive impacts decreased with increasing trophic level. 'Top-down' control was limited to a few consumers with relatively large production or biomass relative to their trophic position. Overall, ordering results from various network analysis algorithms by effective trophic level proved useful in highlighting the potential influence of different taxa to trophodynamics. Although the calculation of effective trophic level has been available for some time, its application to the evaluation of other analyses has previously not received due consideration.},
author = {Christian, Robert R. and Luczkovich, Joseph J.},
doi = {10.1016/S0304-3800(99)00022-8},
isbn = {0304-3800},
issn = {03043800},
journal = {Ecological Modelling},
keywords = {Carbon flow,Effective trophic level,Network analysis,Seagrass community},
month = {apr},
number = {1},
pages = {99--124},
pmid = {227},
title = {{Organizing and understanding a winter's seagrass foodweb network through effective trophic levels}},
url = {http://linkinghub.elsevier.com/retrieve/pii/S0304380099000228},
volume = {117},
year = {1999}
}
@article{Fish1978,
abstract = {Three hundred and thirty-two fish of 8 species (5 Salmo salar, 68 Salvelinus fontinalis, 32 S.namaycush, 102 Coregonus clupeaformis, 1 Prosopium cylindraceum, 107 Esox lucius, 14 Catostomus catostomus, 3 Couesius plumbeus) from the Smallwood Reservoir, Labrador, Canada, were examined for metazoan parasites, using conventional parasitological techniques. 15 genera of parasites were recovered (2 of Monogenea, 2 of Digenea, 4 of Cestoda, 3 of Nematoda, 2 of Acanthocephala and 2 of Copepoda), these including 7 new host records. Significant differences were noted in the prevalence of the various parasites which were common to the different species of fish examined. No differences were recorded in the parasite burden of male and female fish. There was no correlation between the number of parasite species per infected fish and host age except in the case of Salvelinus namaycush. Food items of the fish examined were also noted},
author = {Chinniah, V C and Threlfall, W},
doi = {10.1111/j.1095-8649.1978.tb03427.x},
issn = {10958649},
journal = {Journal of Fish Biology},
keywords = {AGE,Acanthocephala,BURDEN,COMMON,Canada,Canada,Labrador,Smallwood Reservoir,Catostomus,Copepoda,Coregonus,Coregonus clupeaformis,DIFFERENCE,DIGENEA,Diseases,Esox lucius,FEMALE,FOOD,GENERA,HOST,HOST AGE,HOST RECORD,Interactions,LABRADOR,Male,Metazoan parasites,Monogenea,NEWFOUNDLAND,NO,NUMBER,Nematoda,New host,New host records,PLATYHELMINTHES,PREVALENCE,Pisces,Prosopium,RECORD,RECORDS,RESERVOIR,SALAR,SALMO-SALAR,SALVELINUS-FONTINALIS,SALVELINUS-NAMAYCUSH,SPECIES INTERACTIONS,Salmo,Salmo salar,Salvelinus,Salvelinus fontinalis,Salvelinus namaycush,Species interaction,article,biological,cestoda,conventional,correlation,disease,fish,freshwater,host records,interaction,metazoan,metazoan parasite,new host record,parasite,parasite burden,parasites,species,technique,�},
pages = {213},
title = {{Metazoan parasites of fish from the Smallwood Reservoir, Labrador, Canada}},
year = {1978}
}
@article{Bondavalli2011,
abstract = {Indirect trophic effects play important roles in eco- system dynamics and can at times oppose and dominate the action of direct feeding linkages. Each predator directly exerts a negative effect upon its prey, but predators may also provide indirect ben- efits to their prey. In ecosystems, such benefits are effected via indirect trophic pathways that can provide a more than compensating positive influ- ence. The ecosystem of the Big Cypress National Preserve (southwest Florida) appears to contain an unusually high number of such predators—most notably, the American alligator, Alligator mississippi- ensis. The trophic exchanges of carbon among the 68 principal taxa comprising the cypress wetland eco- system have been quantified during both wet and dry seasons. The network analysis program IMPACTS identified predators that potentially have a positive influence on some of their prey. A total of 64 of these instances were recorded for the wet season and 44 for the dry. Taxa that, on balance, have positive effects upon their prey include fishes, turtles, snakes, birds, and, most significantly, alliga- tors. The feeding habits of alligators benefit a con- spicuous number (11) of their prey (invertebrates, frogs, mice, and rats). Further trophic analysis re- veals that the predation by alligators on snakes and turtles accounts for most of the trophic benefits bestowed. The actions of alligators in modifying their physical environment has been cited else- where as contributing to the maintenance of biotic diversity. It appears that the trophic influence of this species adds further evidence to the important role it plays in the functional ecology of the cypress wetland.},
author = {Bondavalli, Cristina and Ulanowicz, Robert E.},
doi = {10.1007/s100219900057},
isbn = {1432-9840},
issn = {14329840},
journal = {Ecosystems},
keywords = {Alligator mississippiensis,Cypress swamps,Ecosystem,Indirect interactions,Network analysis,Predator-prey interaction},
number = {1},
pages = {49--63},
title = {{Unexpected effects of predators upon their prey: The case of the american alligator}},
volume = {2},
year = {1999}
}
@article{White2011,
abstract = {Aquatic endotherms living in polar regions are faced with a multitude of challenges, including low air and water temperatures and low illumination, especially in winter. Like other endotherms from cold environments, Great Cormorants (Phalacrocorax carbo) living in Arctic waters were hypothesized to respond to these challenges through a combination of high daily rate of energy expenditure (DEE) and high food requirements, which are met by a high rate of catch per unit effort (CPUE). CPUE has previously been shown in Great Cormorants to be the highest of any diving bird. In the present study, we tested this hypothesis by making the first measurements of DEE and foraging activity of Arctic-dwelling Great Cormorants throughout the annual cycle. We demonstrate that, in fact, Great Cormorants have surprisingly low rates of DEE. This low DEE is attributed primarily to very low levels of foraging activity, particularly during winter, when the cormorants spent only 2{\%} of their day submerged. Such a low level of foraging activity can only be sustained through consistently high foraging performance. We demonstrate that Great Cormorants have one of the highest recorded CPUEs for a diving predator; 18.6 g per minute submerged (95{\%} prediction interval 13.0-24.2 g/min) during winter. Temporal variation in CPUE was investigated, and highest CPUE was associated with long days and shallow diving depths. The effect of day length is attributed to seasonal variation in prey abundance. Shallow diving leads to high CPUE because less time is spent swimming between the surface and the benthic zone where foraging occurs. Our study demonstrates the importance of obtaining accurate measurements of physiology and behavior from free-living animals when attempting to understand their ecology.},
author = {White, Craig R. and Gr{\'{e}}millet, David and Green, Jonathan a. and Martin, Graham R. and Butler, Patrick J.},
doi = {10.1890/09-1951.1},
isbn = {0012-9658},
issn = {00129658},
journal = {Ecology},
keywords = {Arctic,Basal metabolic rate,CPUE,Catch per unit effort,Daily energy expenditure,Day length,Diving depths,Field metabolic rate,Foraging efficiency,Great Cormorant,Greenland,Phalacrocorax carbo,Seasonal variation},
month = {feb},
number = {2},
pages = {475--486},
pmid = {21618926},
title = {{Metabolic rate throughout the annual cycle reveals the demands of an Arctic existence in Great Cormorants}},
url = {http://www.ncbi.nlm.nih.gov/pubmed/21618926},
volume = {92},
year = {2011}
}
@article{Huneman2010,
abstract = {This paper argues that besides mechanistic explanations, there is a kind of explanation that relies upon "topological" properties of systems in order to derive the explanandum as a consequence, and which does not consider mechanisms or causal processes. I first investigate topological explanations in the case of ecological research on the stability of ecosystems. Then I contrast them with mechanistic explanations, thereby distinguishing the kind of realization they involve from the realization relations entailed by mechanistic explanations, and explain how both kinds of explanations may be articulated in practice. The second section, expanding on the case of ecological stability, considers the phenomenon of robustness at all levels of the biological hierarchy in order to show that topological explanations are indeed pervasive there. Reasons are suggested for this, in which "neutral network" explanations are singled out as a form of topological explanation that spans across many levels. Finally, I appeal to the distinction of explanatory regimes to cast light on a controversy in philosophy of biology, the issue of contingence in evolution, which is shown to essentially involve issues about realization.},
author = {Huneman, Philippe},
doi = {10.1007/s11229-010-9842-z},
isbn = {0039-7857},
issn = {00397857},
journal = {Synthese},
keywords = {Evolutionary contingency,Explanation,Mechanisms,Realization,Robustness,Topology},
month = {dec},
number = {2},
pages = {213--245},
title = {{Topological explanations and robustness in biological sciences}},
url = {http://www.springerlink.com/index/10.1007/s11229-010-9842-z},
volume = {177},
year = {2010}
}
@article{Prinzing2003,
abstract = {Optimal-foraging theory, specifically the ‘‘Time-Limited Disperser Model,'' predicts that animals that can search for resources for a long period will be specialists, whereas animals that have limited search times will be generalists. So far, this model has only been tested within individual species, i.e., among animals of similar physiology and life history. I tested the model across multiple species, using a taxonomically diverse community of arthropods found on exposed tree trunks in northern Germany. I sampled 14 arthropod species from different microhabitat types (various cryptogam species and crevice types) and quantified the microhabitat-niche breadths of species by Simpson index. Then, I used multiple regression analysis across phylogenetically independent contrasts to examine the relationship between microhabitat-niche breadth and traits that can control search time (degree of residence on the trunks, tolerance of abiotic stress, speed of move- ment, generation time). I found that microhabitat specialists had longer generation times, spent more of their life on the trunks and could move faster than generalists. This allowed specialists to search the trunk for a longer period of time than generalists and to require less time to traverse a given search distance. These three findings supported the Time- Limited Disperser Model. However, specialists were no more tolerant of abiotic stress (i.e., desiccation) than were generalists. That is, specialists could not search the trunk during more adverse climatic conditions than could generalists, contradicting the Time-Limited Disperser Model. Overall, the results supported three out of four predictions of the Time- Limited Disperser Model at an interspecific level. Specialists can search for a long time; generalists are pressed for time},
author = {Prinzing, Andreas},
doi = {10.1890/0012-9658(2003)084[1744:AGPFTA]2.0.CO;2},
isbn = {0012-9658},
issn = {00129658},
journal = {Ecology},
keywords = {Arachnida,Behavioral ecology,Canopy,Constraint,Insecta,Life history,Microhabitat use,Niche breadth,Optimal foraging,Search time,Specialization,Time-Limited Disperser Model},
number = {7},
pages = {1744--1755},
title = {{Are generalists pressed for time? An interspecific test of the Time-Limited Disperser Model}},
url = {http://www.esajournals.org/doi/abs/10.1890/0012-9658(2003)084{\%}5B1744:AGPFTA{\%}5D2.0.CO{\%}3B2},
volume = {84},
year = {2003}
}
@article{Stouffer2011c,
abstract = {It has recently been noted that empirical food webs are significantly compartmentalized; that is, subsets of species exist that interact more frequently among themselves than with other species in the community. Although the dynamic implications of compartmentalization have been debated for at least four decades, a general answer has remained elusive. Here, we unambiguously demonstrate that compartmentalization acts to increase the persistence of multitrophic food webs. We then identify the mechanisms behind this result. Compartments in food webs act directly to buffer the propagation of extinctions throughout the community and augment the long-term persistence of its constituent species. This contribution to persistence is greater the more complex the food web, which helps to reconcile the simultaneous complexity and stability of natural communities.},
author = {Stouffer, Daniel B and Bascompte, Jordi},
doi = {10.1073/pnas.1014353108},
institution = {Integrative Ecology Group, Estaci{\'{o}}n Biol{\'{o}}gica de Do{\~{n}}ana, Consejo Superior de Investigaciones Cient{\'{i}}ficas, 41092 Seville, Spain. stouffer@ebd.csic.es},
isbn = {0027-8424},
issn = {1091-6490},
journal = {Proceedings of the National Academy of Sciences of the United States of America},
keywords = {Extinction, Biological,Food Chain,Models, Biological,Species Specificity},
number = {9},
pages = {3648--52},
pmid = {21307311},
publisher = {National Academy of Sciences},
title = {{Compartmentalization increases food-web persistence.}},
url = {http://www.pnas.org/content/108/9/3648.full},
volume = {108},
year = {2011}
}
@article{Shurin2007,
abstract = {Analyses of temporal patterns of diversity across a wide range of taxa have found that more diverse communities often show smaller compositional changes over time. This generality indicates that high diversity is associated with greater temporal stability in species composition. We examined patterns of diversity and community stability in zooplankton time series data from 36 lakes sampled over a combined 483 years. The species-time relationship was flatter in more species-rich lakes in the temperate zone. However, high-latitude lakes had both low richness and low turnover. These patterns were consistent for turnover both within and among years. Daily, annual and long-term richness were all higher in large lakes while turnover was unaffected by the surface area. Richness on all time scales, as well as turnover within and among years, all declined at high latitude. Species-area relations and latitudinal gradients in richness therefore reflect different temporal components of diversity. Our results suggest that diversity shows strong associations with compositional stability that vary qualitatively across biogeographical provinces. Community stability increases with diversity among lakes in the temperate zone; however, the two are negatively correlated across latitudinal gradients. These patterns indicate that either the direct effects of diversity on stability or their covariance with environmental fluctuations vary with latitude.},
author = {Shurin, Jonathan B. and Arnott, Shelley E. and Hillebrand, Helmut and Longmuir, Allyson and Pinel-Alloul, Bernadette and Winder, Monika and Yan, Norman D.},
doi = {10.1111/j.1461-0248.2006.01009.x},
isbn = {1461-0248 (Electronic)$\backslash$r1461-023X (Linking)},
issn = {1461023X},
journal = {Ecology Letters},
keywords = {Extinction,Invasion,Latitudinal gradients,Species-time relationship},
month = {feb},
number = {2},
pages = {127--134},
pmid = {17257100},
title = {{Diversity-stability relationship varies with latitude in zooplankton}},
url = {http://www.ncbi.nlm.nih.gov/pubmed/17257100},
volume = {10},
year = {2007}
}
@article{Samelius2011,
abstract = {We examined how large seasonal influxes of migratory prey influenced population dynamics of arctic foxes and how this varied with fluctuations in small mammal (lemming and vole) abundance—the main prey of arctic foxes throughout most of their range. Specifically, we compared how arctic fox abundance, breeding density and litter size varied inside and outside a large goose colony and in relation to annual variation in small mammal abundance. Information-theoretic model selection showed that (1) breeding density and fox abundance were 2–3 times higher inside the colony than they were outside the colony and (2) litter size, breeding density and annual variation in fox abundance in the colony tracked fluctuations in lemming abundance. The influence of lemming abundance on reproduction and abundance of arctic foxes outside the colony was inconclusive, largely because fox densities outside the colony were low, which made it difficult to detect such relationships. Lemming abundance was, thus, the main factor governing reproduction and abundance of arctic foxes in the colony, whereas seasonal influxes of geese and their eggs provided foxes with external subsidies that elevated breeding density and fox abundance above that which lemmings could support. This study highlights (1) the relative importance of migratory prey and other foods on the abundance and reproduction by local consumers and (2) how migratory animals function as vectors of nutrient transfer between distant ecosystems such as Arctic environments and wintering areas by geese thousands of kilometres to the south.},
author = {Samelius, Gustaf and Alisauskas, Ray T. and Larivi{\`{e}}re, Serge},
doi = {10.1007/s00300-011-1005-2},
isbn = {0030001110},
issn = {07224060},
journal = {Polar Biology},
keywords = {External subsidies,Food hoarding,Migratory prey,Population dynamics,Pulsed resources},
month = {apr},
number = {10},
pages = {1475--1484},
title = {{Seasonal pulses of migratory prey and annual variation in small mammal abundance affect abundance and reproduction by arctic foxes}},
url = {http://www.springerlink.com/index/10.1007/s00300-011-1005-2},
volume = {34},
year = {2011}
}
@article{Bluthgen2007,
abstract = {The topology of ecological interaction webs holds important information for theories of coevolution, biodiversity, and ecosystem stability [1-6]. However, most previous network analyses solely counted the number of links and ignored variation in link strength. Because of this crude resolution, results vary with scale and sampling intensity, thus hampering a comparison of network patterns at different levels [7-9]. We applied a recently developed [10] quantitative and scale-independent analysis based on information theory to 51 mutualistic plant-animal networks, with interaction frequency as measure of link strength. Most networks were highly structured, deviating significantly from random associations. The degree of specialization was independent of network size. Pollination webs were significantly more specialized than seed-dispersal webs, and obligate symbiotic ant-plant mutualisms were more specialized than nectar-mediated facultative ones. Across networks, the average specialization of animal and plants was correlated, but is constrained by the ratio of plant to animal species involved. In pollination webs, rarely visited plants were on average more specialized than frequently attended ones, whereas specialization of pollinators was positively correlated with their interaction frequency. We conclude that quantitative specialization in ecological communities mirrors evolutionary trade-offs and constraints of web architecture. This approach can be easily expanded to other types of biological interactions. {\textcopyright} 2007 Elsevier Ltd. All rights reserved.},
author = {Bl{\"{u}}thgen, Nico and Menzel, Florian and Hovestadt, Thomas and Fiala, Brigitte and Bl{\"{u}}thgen, Nils},
doi = {10.1016/j.cub.2006.12.039},
isbn = {0960-9822},
issn = {09609822},
journal = {Current Biology},
keywords = {EVO{\_}ECOL},
month = {feb},
number = {4},
pages = {341--346},
pmid = {17275300},
title = {{Specialization, constraints, and conflicting interests in mutualistic networks}},
url = {http://www.ncbi.nlm.nih.gov/pubmed/17275300},
volume = {17},
year = {2007}
}
@article{KaiserBunbury2010,
abstract = {Species extinctions pose serious threats to the functioning of ecological communities worldwide. We used two qualitative and quantitative pollination networks to simulate extinction patterns following three removal scenarios: random removal and systematic removal of the strongest and weakest interactors. We accounted for pollinator behaviour by including potential links into temporal snapshots (12 consecutive 2-week networks) to reflect mutualists' ability to 'switch' interaction partners (re-wiring). Qualitative data suggested a linear or slower than linear secondary extinction while quantitative data showed sigmoidal decline of plant interaction strength upon removal of the strongest interactor. Temporal snapshots indicated greater stability of re-wired networks over static systems. Tolerance of generalized networks to species extinctions was high in the random removal scenario, with an increase in network stability if species formed new interactions. Anthropogenic disturbance, however, that promote the extinction of the strongest interactors might induce a sudden collapse of pollination networks.},
author = {Kaiser-Bunbury, Christopher N. and Muff, Stefanie and Memmott, Jane and M{\"{u}}ller, Christine B. and Caflisch, Amedeo},
doi = {10.1111/j.1461-0248.2009.01437.x},
isbn = {1461-023X},
issn = {1461023X},
journal = {Ecology Letters},
keywords = {Behaviour,Complex networks,Extinction,Habitat restoration,Mauritius,Mutualism,Network re-wiring,Pollination},
month = {apr},
number = {4},
pages = {442--452},
pmid = {20100244},
title = {{The robustness of pollination networks to the loss of species and interactions: a quantitative approach incorporating pollinator behaviour}},
url = {http://www.ncbi.nlm.nih.gov/pubmed/20100244},
volume = {13},
year = {2010}
}
@article{Rezende2009,
abstract = {A long-standing question in community ecology is whether food webs are organized in compartments, where species within the same compartment interact frequently among themselves, but show fewer interactions with species from other compartments. Finding evidence for this community organization is important since compartmentalization may strongly affect food web robustness to perturbation. However, few studies have found unequivocal evidence of compartments, and none has quantified the suite of mechanisms generating such a structure. Here, we combine computational tools from the physics of complex networks with phylogenetic statistical methods to show that a large marine food web is organized in compartments, and that body size, phylogeny, and spatial structure are jointly associated with such a compartmentalized structure. Sharks account for the majority of predatory interactions within their compartments. Phylogenetically closely related shark species tend to occupy different compartments and have divergent trophic levels, suggesting that competition may play an important role structuring some of these compartments. Current overfishing of sharks has the potential to change the structural properties, which might eventually affect the stability of the food web.},
author = {Rezende, Enrico L. and Albert, Eva M. and Fortuna, Miguel A. and Bascompte, Jordi},
doi = {10.1111/j.1461-0248.2009.01327.x},
isbn = {1461-023X},
issn = {1461023X},
journal = {Ecology Letters},
keywords = {Compartments,Complex networks,Food web,Phylogenetic analyses,Sharks,Trophic level},
month = {aug},
number = {8},
pages = {779--788},
pmid = {19490028},
title = {{Compartments in a marine food web associated with phylogeny, body mass, and habitat structure}},
url = {http://www.ncbi.nlm.nih.gov/pubmed/19490028},
volume = {12},
year = {2009}
}
@article{Post2002,
abstract = {Food-chain length is a central characteristic of ecological communities that has attracted considerable attention for over 75 years because it strongly affects community structure, ecosystem processes and contaminant concentrations. Conventional wisdom holds that either resource availability or dynamical stability limit food-chain length; however, new studies and new techniques challenge the conventional wisdom and broaden the discourse on food-chain length. Recent results suggest that resource availability limits food-chain length only in systems with very low resource availability, and call into question the theoretical basis for dynamical stability as a determinant of food-chain length. Evidence currently points towards a complex and contingent framework of interacting constraints that includes the history of community organization, resource availability, the type of predator-prey interactions, disturbance and ecosystem size. Within this framework, the debate has shifted from a search for singular explanations to a search for when and where different constraints operate to determine food-chain length.},
author = {Post, David M.},
doi = {10.1016/S0169-5347(02)02455-2},
isbn = {9781424497270},
issn = {01695347},
journal = {Trends in Ecology and Evolution},
month = {jun},
number = {6},
pages = {269--277},
pmid = {175678500011},
title = {{The long and short of food-chain length}},
url = {http://linkinghub.elsevier.com/retrieve/pii/S0169534702024552},
volume = {17},
year = {2002}
}
@article{Dickman2008,
abstract = {The efficiency of energy transfer through food chains [food chain efficiency (FCE)] is an important ecosystem function. It has been hypothesized that FCE across multiple trophic levels is constrained by the efficiency at which herbivores use plant energy, which depends on plant nutritional quality. Furthermore, the number of trophic levels may also constrain FCE, because herbivores are less efficient in using plant production when they are constrained by carnivores. These hypotheses have not been tested experimentally in food chains with 3 or more trophic levels. In a field experiment manipulating light, nutrients, and food-chain length, we show that FCE is constrained by algal food quality and food-chain length. FCE across 3 trophic levels (phytoplankton to carnivorous fish) was highest under low light and high nutrients, where algal quality was best as indicated by taxonomic composition and nutrient stoichiometry. In 3-level systems, FCE was constrained by the efficiency at which both herbivores and carnivores converted food into production; a strong nutrient effect on carnivore efficiency suggests a carryover effect of algal quality across 3 trophic levels. Energy transfer efficiency from algae to herbivores was also higher in 2-level systems (without carnivores) than in 3-level systems. Our results support the hypothesis that FCE is strongly constrained by light, nutrients, and food-chain length and suggest that carryover effects across multiple trophic levels are important. Because many environmental perturbations affect light, nutrients, and food-chain length, and many ecological services are mediated by FCE, it will be important to apply these findings to various ecosystem types.},
author = {Dickman, Elizabeth M and Newell, Jennifer M and Gonz{\'{a}}lez, Mar{\'{i}}a J and Vanni, Michael J},
doi = {10.1073/pnas.0805566105},
isbn = {1091-6490 (Electronic)$\backslash$n0027-8424 (Linking)},
issn = {0027-8424},
journal = {Proceedings of the National Academy of Sciences of the United States of America},
keywords = {Animals,Energy Transfer,Fishes,Food Chain,Light,Plankton,Plankton: metabolism},
month = {nov},
number = {47},
pages = {18408--18412},
pmid = {19011082},
title = {{Light, nutrients, and food-chain length constrain planktonic energy transfer efficiency across multiple trophic levels.}},
url = {http://www.pubmedcentral.nih.gov/articlerender.fcgi?artid=2587603{\&}tool=pmcentrez{\&}rendertype=abstract},
volume = {105},
year = {2008}
}
@article{Sahli2006,
abstract = {JSTOR is a not-for-profit service that helps scholars, researchers, and students discover, use, and build upon a wide range of content in a trusted digital archive. We use information technology and tools to increase productivity and facilitate new forms of scholarship. For more information about JSTOR, please contact support@jstor.org. Abstract Despite the development of diversity indices in community ecology that incorporate both richness and evenness, pollination biologists commonly use only pollinator richness to estimate generalization. Similarly, while pollination biologists have stressed the utility of pollinator importance, incorporating both pollinator abundance and effectiveness, importance values have not been included in estimates of generalization in pol lination systems. In this study, we estimated pollinator generalization for 17 plant species using Simpson's diversity index, which includes richness and evenness. We compared these estimates with estimates based on only pollinator richness, and compared diversity esti mates calculated using importance data with those using only visitation data. We found that pollinator richness explains only 57-65{\%} of the variation in diversity, and that, for most plant species, pollinator importance was determined primarily by differences in visitation rather than by differences in effectiveness. While simple rich ness may suffice for broad comparisons of pollinator generalization, measures that incorporate evenness will provide a much more accurate understanding of gener alization. Although incorporating labor-intensive mea surements of pollinator effectiveness are less necessary for broad surveys, effectiveness estimates will be impor tant for detailed studies of some plant species. Unfor tunately, at this point it is impossible to predict a priori which species these are.},
author = {Sahli, Heather F. and Conner, Jeffrey K.},
doi = {10.1007/s00442-006-0396-1},
isbn = {0029-8549},
issn = {00298549},
journal = {Oecologia},
keywords = {Diversity,Effectiveness,Evenness,Pollinator importance,Specialization},
month = {jun},
number = {3},
pages = {365--372},
pmid = {16514533},
title = {{Characterizing ecological generalization in plant-pollination systems}},
url = {http://www.ncbi.nlm.nih.gov/pubmed/16514533},
volume = {148},
year = {2006}
}
@article{Schaik2012,
abstract = {Most tropical woody plants produce new leaves and flowers in bursts rather than continuously, and most tropical forest communities display seasonal variation in the presence of new leaves, flowers, and fruits. This patterning suggests that phenological changes represent adaptations to either biotic or abiotic factors. Biotic factors may select for either a staggering or a clustering of the phenological activity of individual plant species. We review the evidence for several hypotheses. The idea that plant species can reduce predation by synchronizing their phenological activity has the best support. However, because biotic factors are often arbitrary with respect to the timing of these peaks, it is essential also to consider abiotic influences. A review of published studies demonstrates a major role for climate. Peaks in irradiance are accompanied by peaks in flushing and flowering except where water stress makes this impossible. Thus, in seasonally dry forests, many plants concentrate leafing and flowering around the start of the rainy season; they also tend to fruit at the same time, probably to minimize seedling mortality during the subsequent dry season. Phenological variation at the level of the forest community affects primary consumers who respond by dietary switching, seasonal breeding, changes in range use, or migration. During periods of scarcity, certain plant products, keystone resources, act as mainstays of the primary consumer community.},
author = {Vanschaik, C P and Terborgh, J W and Wright, S J},
doi = {10.1146/annurev.es.24.110193.002033},
isbn = {0066-4162},
issn = {0066-4162},
journal = {Annual Review of Ecology and Systematics},
keywords = {flowering,fruiting,leafing,phenology,tropical forests},
number = {1993},
pages = {353--377},
pmid = {881},
title = {{the Phenology of Tropical Forests - Adaptive Significance and Consequences for Primary Consumers}},
volume = {24},
year = {1993}
}
@article{Olesen2011,
abstract = {Ecological networks are complexes of interacting species, but not all potential links among species are realized. Unobserved links are either missing or forbidden. Missing links exist, but require more sampling or alternative ways of detection to be verified. Forbidden links remain unobservable, irrespective of sampling effort. They are caused by linkage constraints. We studied one Arctic pollination network and two Mediterranean seed-dispersal networks. In the first, for example, we recorded flower-visit links for one full season, arranged data in an interaction matrix and got a connectance C of 15 per cent. Interaction accumulation curves documented our sampling of interactions through observation of visits to be robust. Then, we included data on pollen from the body surface of flower visitors as an additional link 'currency'. This resulted in 98 new links, missing from the visitation data. Thus, the combined visit-pollen matrix got an increased C of 20 per cent. For the three networks, C ranged from 20 to 52 per cent, and thus the percentage of unobserved links (100 - C) was 48 to 80 per cent; these were assumed forbidden because of linkage constraints and not missing because of under-sampling. Phenological uncoupling (i.e. non-overlapping phenophases between interacting mutualists) is one kind of constraint, and it explained 22 to 28 per cent of all possible, but unobserved links. Increasing phenophase overlap between species increased link probability, but extensive overlaps were required to achieve a high probability. Other kinds of constraint, such as size mismatch and accessibility limitations, are briefly addressed.},
author = {Olesen, Jens M and Bascompte, Jordi and Dupont, Yoko L and Elberling, Heidi and Rasmussen, Claus and Jordano, Pedro},
doi = {10.1098/rspb.2010.1371},
isbn = {0962-8452},
issn = {0962-8452},
journal = {Proceedings of the Royal Society B: Biological Sciences},
keywords = {Animals,Arctic Regions,Birds,Demography,Ecosystem,Insects,Mammals,Mediterranean Region,Pollination,Pollination: physiology,Seeds,Seeds: physiology,Symbiosis,Symbiosis: physiology,arctic,climate,mediterranean shrubland,mismatch,pollination},
month = {mar},
number = {1706},
pages = {725--732},
pmid = {20843845},
title = {{Missing and forbidden links in mutualistic networks.}},
url = {http://apps.webofknowledge.com.libproxy.nau.edu/full{\_}record.do?product=CABI{\&}search{\_}mode=GeneralSearch{\&}qid=27{\&}SID=2CmmrNTYpwVeEfIWbVM{\&}page=1{\&}doc=10{\&}cacheurlFromRightClick=no},
volume = {278},
year = {2011}
}
@article{Kato2010,
abstract = {1. To examine spatial heterogeneity of trophic pathways on a small scale ({\textless}5 m diameter), we conducted dual stable isotope (d13C and d15N) analyses of invertebrate communities and their potential food sources in three patchy habitats [sphagnum lawn (SL), vascular- plant carpet (VC) and sphagnum carpet] within a temperate bog (Mizorogaike Pond, Kyoto, Japan). 2. In total, 19 invertebrate taxa were collected from the three habitats, most of which were stenotopic, i.e. collected from a single habitat. Amongst the habitats, significant variation was observed in the isotopic signatures of dominant plant tissues and their detrital matter [benthic particulate organic matter (BPOM)], both of which were potential organic food sources for invertebrates. Site-specific isotopic variation amongst detritivores was found in d13C but not in d15N, reflecting site-specificity in the isotopic signatures of basal foods. The eurytopic hydrophilid beetle Helochares striatus was found in all habitats, but showed clear site variation in its isotopic signatures, suggesting that it strongly relies on foods within its own habitat. 3. The most promising potential foods for detritivores were the dead leaf stalks of a dominant plant in the VC and BPOM in the SL and carpet. An isotopic mixing model (IsoSource version 1.3.1) estimated that aquatic predators rely on unknown trophic sources with higher d13C than detritus, whereas terrestrial predators forage on allochthonous as well as autochthonous prey, suggesting that the latter predators might play key roles in coupling between habitats. 4. Our stable isotope approach revealed that immobile detritivores are confined to their small patchy habitats but that heterogeneous trophic pathways can be coupled by mobile predators, stressing the importance of habitat heterogeneity and predator coupling in characterising food webs in bog ecosystems.},
author = {Kato, Yoshikazu and Hori, Michio and Okuda, Noboru and Tayasu, Ichiro and Takemon, Yasuhiro},
doi = {10.1111/j.1365-2427.2009.02295.x},
isbn = {00465070$\backslash$n13652427},
issn = {00465070},
journal = {Freshwater Biology},
keywords = {Compartmentalisation,Coupling,Food web,Peatland,Stable isotopes},
month = {feb},
number = {2},
pages = {450--462},
title = {{Spatial heterogeneity of trophic pathways in the invertebrate community of a temperate bog}},
url = {http://doi.wiley.com/10.1111/j.1365-2427.2009.02295.x},
volume = {55},
year = {2010}
}
@article{Guimera2010a,
abstract = {The response of an ecosystem to perturbations is mediated by both antagonistic and facilitative interactions between species. It is thought that a community's resilience depends crucially on the food web-the network of trophic interactions-and on the food web's degree of compartmentalization. Despite its ecological importance, compartmentalization and the mechanisms that give rise to it remain poorly understood. Here we investigate several definitions of compartments, propose ways to understand the ecological meaning of these definitions, and quantify the degree of compartmentalization of empirical food webs. We find that the compartmentalization observed in empirical food webs can be accounted for solely by the niche organization of species and their diets. By uncovering connections between compartmentalization and species' diet contiguity, our findings help us understand which perturbations can result in fragmentation of the food web and which can lead to catastrophic effects. Additionally, we show that the composition of compartments can be used to address the long-standing question of what determines the ecological niche of a species.},
author = {Guimer{\`{a}}, R. and Stouffer, D. B. and Sales-Pardo, M. and Leicht, E. a. and Newman, M. E J and Amaral, L. a N},
doi = {10.1890/09-1175.1},
isbn = {0012-9658},
issn = {00129658},
journal = {Ecology},
keywords = {Compartmentalization,Compartments,Ecological networks,Food web patterns,Food web structure,Food webs,Modularity,Niche},
month = {oct},
number = {10},
pages = {2941--2951},
pmid = {21058554},
title = {{Origin of compartmentalization in food webs}},
url = {http://www.ncbi.nlm.nih.gov/pubmed/21058554},
volume = {91},
year = {2010}
}
@article{Waser1996,
abstract = {applicability for this approach.},
archivePrefix = {arXiv},
arxivId = {arXiv:1011.1669v3},
author = {Waser, Nickolas M. and Chittka, Lars and Price, Mary V. and Williams, Neal M. and Ollerton, Geff},
doi = {10.1017/CBO9781107415324.004},
eprint = {arXiv:1011.1669v3},
isbn = {9788578110796},
issn = {08628408},
journal = {Ecology},
keywords = {Blood pressure,Hypertension,Isometric exercise},
number = {4},
pages = {1043--1060},
pmid = {25246403},
title = {{Generalization in pollination systems, and why it matters}},
url = {https://esajournals.onlinelibrary.wiley.com/doi/abs/10.2307/2265575},
volume = {77},
year = {1996}
}
@article{Estrada2007,
abstract = {We analyse the robustness of food webs against species loss by considering the influence of several structural factors of the networks, such as connectance, degree distribution and expansibility. The last concept refers to the absence of structural bottlenecks in the food web, whose removal separate the network into large isolate clusters. In theory networks with identical connectance can display different expansibility characteristics. Using the spectral scaling method we studied 17 food networks and classified them as good expansion (GE) and not-GE networks. The combination of GE properties and degree distribution of species permitted the classification of food webs into six different classes. These classes characterize the differences in robustness of food webs to species loss. While the webs having uniform degree distributions and displaying GE properties are the most robust to species loss, the presence of bottlenecks and skewed distribution of the number of links per species make food webs very vulnerable to primary removal of species. ?? 2006 Elsevier Ltd. All rights reserved.},
author = {Estrada, Ernesto},
doi = {10.1016/j.jtbi.2006.08.002},
isbn = {0022-5193},
issn = {00225193},
journal = {Journal of Theoretical Biology},
keywords = {Food web,Good expansion,Graph theory,Network structure,Secondary extinctions,Species removal,Spectral scaling},
month = {jan},
number = {2},
pages = {296--307},
pmid = {16987531},
title = {{Food webs robustness to biodiversity loss: the roles of connectance, expansibility and degree distribution}},
url = {http://www.ncbi.nlm.nih.gov/pubmed/16987531},
volume = {244},
year = {2007}
}
@article{Gilbert2009,
abstract = {Many ecologists are concerned that biodiversity loss from human impact on natural ecosystems could compromise ecosystem stability. A relationship between diversity and stability was proposed by MacArthur [MacArthur, R.H., 1955. Fluctuation of animal populations and a measure of community stability. Ecology 36, 533-536.]. Current thinking (for example, McCann, K., 2000. The diversity-stability debate. Nature 405, 228-233.) acknowledges that interaction pattern among species, rather than species richness per se, is one element of this relationship. Dunne et al. [Dunne, J.A., Williams, R.J., Martinez, N.D., 2002a. Network structure and biodiversity loss in food webs: robustness increases with connectance. Ecol. Lett. 5, 558-567.] showed that the robustness of 16 food webs is correlated with their connectance. Connectance is one measure of interaction pattern. Robustness relates to the maintenance of network integrity and so has consequences for stability; the loss of integrity must have ecosystem-wide implications. This paper tests the hypothesis that changes in a food web's connectance indicate changes in its robustness. It concludes that any change in connectance with species loss, but especially large, negative changes, constitutes a decrease in robustness. Estimation of the change in connectance could support interpretation of monitoring data on species composition, acting as an indicator of food web robustness and, indirectly, of ecosystem stability. It could assist managers to understand the implications of biodiversity loss caused by human intervention in ecosystems, and could assist either choice of intervention or amelioration of impacts. ?? 2008 Elsevier Ltd. All rights reserved.},
author = {Gilbert, Alison J.},
doi = {10.1016/j.ecolind.2008.01.010},
isbn = {1470-160X},
issn = {1470160X},
journal = {Ecological Indicators},
keywords = {Biodiversity loss,Connectance,Degree distribution,Ecological networks,Robustness,Species richness},
month = {jan},
number = {1},
pages = {72--80},
title = {{Connectance indicates the robustness of food webs when subjected to species loss}},
url = {http://linkinghub.elsevier.com/retrieve/pii/S1470160X08000125},
volume = {9},
year = {2009}
}
@article{Gonzalez2009,
abstract = {We studied the effect of climate on the plant-pollinator communities in the West Indies. We constructed plots of 200 m × 5 m in two distinct habitats on the islands of Dominica, Grenada and Puerto Rico (total of six plots) and recorded visitors to all plant species in flower. In total we recorded 447 interactions among 144 plants and 226 pollinator species. Specifically we describe how rainfall and temperature affect proportional richness and importance of the different pollinator functional groups. We used three measures of pollinator importance: number of interactions, number of plant species visited and betweenness centrality. Overall rainfall explained most of the variation in pollinator richness and relative importance. Bird pollination tended to increase with rainfall, although not significantly, whereas insects were significantly negatively affected by rainfall. However, the response among insect groups was more complex; bees were strongly negatively affected by rainfall, whereas dipterans showed similar trends to birds. Bird, bee and dipteran variation along the climate gradient can be largely explained by their physiological capabilities to respond to rainfall and temperature, but the effect of climate on other insect pollinator groups was more obscure. This study contributes to the understanding of how climate may affect neotropical plant-pollinator communities},
author = {Gonzalez, A. and Dalsgaard, B. and Ollerton, J. and Timmermann, A. and Olesen, J. and Andersen, L. and Tossas, A.},
doi = {10.1017/S0266467409990034},
isbn = {0266-4674},
issn = {0266-4674},
journal = {Journal of Tropical Ecology},
keywords = {QC980 Climatology and weather,QK900 Plant ecology,QK926 Pollination},
month = {jul},
number = {05},
pages = {493},
title = {{Effects of climate on pollination networks in the West Indies}},
url = {http://journals.cambridge.org/action/displayJournal?jid=TRO},
volume = {25},
year = {2009}
}
@article{Williams2009,
abstract = {Degree distributions are widely used to characterize networks, including food webs, and play a vital role in models of food web structure. To date, there have been no mechanistic or statistical explanations for the form of food web degree distributions. Here, I introduce models for food web degree distributions based on the principle of maximum entropy (MaxEnt) and show that the distributions of the number of consumers and resources in 23 (45{\%}) and 35 (69{\%}) of 51 food webs are not significantly different at a 95{\%} confidence level from the MaxEnt distribution. These findings offer a new null model for the most probable degree distributions in food webs and other networks. They suggest that there is relatively little pressure favoring generalist or specialist consumption strategies but that biological drivers or methodological bias may force the consumer distribution away from the MaxEnt form.},
archivePrefix = {arXiv},
arxivId = {0901.0976},
author = {Williams, Richard J.},
doi = {10.1007/s12080-009-0052-6},
eprint = {0901.0976},
isbn = {1874-1738},
issn = {18741738},
journal = {Theoretical Ecology},
keywords = {Consumer distribution,Network,Null model,Resource distribution},
month = {may},
number = {1},
pages = {45--52},
title = {{Simple MaxEnt models explain food web degree distributions}},
url = {http://www.springerlink.com/index/10.1007/s12080-009-0052-6},
volume = {3},
year = {2010}
}
@article{Lang2012,
abstract = {1. Model analyses show that the stability of population dynamics and food web persistence increase with the strength of interference competition. Despite this critical importance for community stability, little is known about how external factors such as the environmental temperature affect intraspecific interference competition. 2. We aimed to fill this void by studying the functional responses of two ground beetle species of different body size, Pterostichus melanarius and Poecilus versicolor. These functional response experiments were replicated across four predator densities and two temperatures to address the impact of temperature on intraspecific interference competition. 3. We generally expected that warming should increase the speed of movement, encounter rates and in consequence interference among predator individuals. In our experiment, this expectation was supported by the results obtained for the larger predator, P. melanarius, whereas the opposite pattern characterized the interference behaviour of the smaller predator P. versicolor. 4. These results suggest potentially nontrivial implications for the effects of environmental temperature on intraspecific interference competition, for which we propose an explanation based on the different sensitivity to warming of metabolic rates of both species. As expected, increasing temperature led to stronger interference competition of the larger species, P. melanarius, which exhibited a weaker increase in metabolic rate with increasing temperature. The stronger increase in the metabolic rate of the smaller predator, P. versicolor, had to be compensated by increasing searching activity for prey, which did not leave time for increasing interference. 5. Together, these results suggest that any generalization how interference competition responds to warming should also take the species' metabolic response to temperature increases into account.},
author = {Lang, Birgit and Rall, Bj{\"{o}}rn C. and Brose, Ulrich},
doi = {10.1111/j.1365-2656.2011.01931.x},
isbn = {0021-8790},
issn = {00218790},
journal = {Journal of Animal Ecology},
keywords = {Food webs,Functional responses,Global warming,Interaction strength,Metabolic rates},
month = {may},
number = {3},
pages = {516--523},
pmid = {22112157},
title = {{Warming effects on consumption and intraspecific interference competition depend on predator metabolism}},
url = {http://www.ncbi.nlm.nih.gov/pubmed/22112157},
volume = {81},
year = {2012}
}
@article{Walters2008,
abstract = {Streams experience frequent natural disturbance and are undergoing considerable anthropogenic disturbance due to dam construction and water diversion. Disturbance is known to impact community structure, but its effect on food chain length is still a matter of considerable debate. Theoretical models show that longer food chains are less resilient to disturbance, so food chain length is predicted to be shorter following a disturbance event. Here we experimentally test the effect of disturbance on food chain length in streams by diverting stream flow. We found that our experimental low-flow disturbance did not alter food chain length. We did see an effect on body-size structure in our food webs suggesting that food chain length may be an insensitive indicator of disturbance. We suggest that habitat heterogeneity and food web complexity buffer the effect of disturbance on food chain length. The theoretical predictions of disturbance on food chain length are only likely to be seen in homogeneous systems that closely approximate the linear food chains the models are based upon.},
author = {Walters, Annika W. and Post, David M.},
doi = {10.1890/08-0273.1},
isbn = {0012-9658 (Print)$\backslash$r0012-9658 (Linking)},
issn = {00129658},
journal = {Ecology},
keywords = {Disturbance,Food chain length,Food web,Stream community,Water diversion},
number = {12},
pages = {3261--3267},
pmid = {19137932},
title = {{An experimental disturbance alters fish size structure but not food chain length in streams}},
volume = {89},
year = {2008}
}
@article{Dupont2009,
abstract = {Pollination networks are representations of all interactions between co-existing plants and their flower visiting animals at a given site. Although the study of networks has become a distinct sub-discipline in pollination biology, few studies have attempted to quantify spatio-temporal variation in species composition and structure of networks. We here investigate patterns of year-to-year change in pollination networks from six different sites spanning a large latitudinal gradient. We quantified level of species persistence and interactions among years, and examined year-to-year variation of network structural parameters in relation to latitude and sampling effort. In addition, we tested for correlations between annual variation in network parameters and short and long-term climate change variables. Numbers of plant and animal species and interactions were roughly constant from one year to another at all sites. However, composition of species and interactions changed from one year to another. Turnover was particularly high for flower visitors and interactions. On the other hand, network structural parameters (connectance, nestedness, modularity and centralization) remained remarkably constant between years, regardless of network size and latitude. Inter-annual variation of network parameters was not related to short or long term variation in climate variables (mean annual temperature and annual precipitation). We thus conclude that pollination networks are highly dynamic and variable in composition of species and interactions among years. However, general patterns of network structure remain constant, indicating that species may be replaced by topologically similar species. These results suggest that pollination networks are to some extent robust against factors affecting species occurrences},
author = {Dupont, Yoko L. and Padr{\'{o}}n, Benigno and Olesen, Jens M. and Petanidou, Theodora},
doi = {10.1111/j.1600-0706.2009.17594.x},
isbn = {0030-1299},
issn = {00301299},
journal = {Oikos},
month = {aug},
number = {8},
pages = {1261--1269},
pmid = {1220},
title = {{Spatio-temporal variation in the structure of pollination networks}},
url = {http://doi.wiley.com/10.1111/j.1600-0706.2009.17594.x},
volume = {118},
year = {2009}
}
@article{Zhou2011,
abstract = {Aim Studies comparing feeding habits across a genus in different geographical regions or habitats can identify factors associated with adaptive feeding behaviour, linking key ecological traits between consumers and their environment. We investigated biogeographical patterns in dietary composition and trophic diversity across the genus Martes in relation to geographical range and environmental variables. We hypothesized that widely distributed opportunistic Martes species should demonstrate adaptive variations in dietary composition and trophic diversity relative to regional geographical location (e.g. latitude, elevation), environmental variation (e.g. temperature, rainfall, snow cover and primary productivity) and concomitant variation in food supply. Location Europe, Asia and North America. Methods We examined the dietary habits of martens (Martes spp.) using original data expressed as relative frequency of occurrence, and using principal components analysis to extract the main gradients in diet composition. These were then used as response variables in regression analyses, predicted from latitude or elevation. Multiple regression analyses were performed to assess the influence of food types and environmental variables on the trophic diversity index. Results A clear latitudinal gradient in dietary composition was observed. Small mammals were the primary food type, but were less abundant in the diet of martens at lower latitude and elevation. Vegetable matter and insects were consumed more frequently in southerly and/or lower-elevation localities. Trophic diversity was lower at higher elevation, and increased with a decline in consumption of the dominant food types, i.e. rodents, fruits and insects. Trophic diversity also increased with increasing mean temperature. Main conclusions Biogeographical variations in feeding habits across the genus Martes proved to be associated with latitude, local climate (especially temperature regime) and the availability of alternative potential foods. On an extensive geographical scale, martens respond to varying food availability by adjusting their foraging strategy and thus should be considered facultative generalists. At the species level, however, different climatic variables emerged as differentially important, indicative of adaptations to local conditions. Martes species are opportunistic and flexible feeders, and thus their conservation requires informed management, mindful of how changes in environmental conditions might influence their varied food supply.},
author = {Zhou, You Bing and Newman, Chris and Xu, Wen Ting and Buesching, Christina D. and Zalewski, Andrzej and Kaneko, Yayoi and Macdonald, David W. and Xie, Zong Qiang},
doi = {10.1111/j.1365-2699.2010.02396.x},
isbn = {03050270},
issn = {03050270},
journal = {Journal of Biogeography},
keywords = {Adaptive foraging,Climate,Fruit availability,Generalist,Holarctic region,Latitudinal patterns,Martens,Martes,Opportunistic feeder,Trophic diversity},
month = {jan},
number = {1},
pages = {137--147},
pmid = {3516},
title = {{Biogeographical variation in the diet of Holarctic martens (genus Martes, Mammalia: Carnivora: Mustelidae): Adaptive foraging in generalists}},
url = {http://doi.wiley.com/10.1111/j.1365-2699.2010.02396.x},
volume = {38},
year = {2011}
}
@article{Turchin1997,
abstract = {Vole dynamics in northern Europe exhibit a well-defined geographical gradient, with oscillatory populations being confined to high latitudes. It has been proposed that oscillations in northern vole populations are driven by their interaction with specialist predators (weasels), while the more southern rodent populations are relatively stable because of regulation by generalist predators. We tested this generalist/specialist predation hypothesis by constructing an empirically based model for vole population dynamics, estimating its parameters, and making predictions about the quantitative pattern of the latitudinal shift in vole dynamics. Our results indicated that the model accurately predicted the latitudinal shift in the amplitude and periodicity of population fluctuations. Moreover, the model predicted that vole dynamics should shift from stable to chaotic as latitude is increased, a result in agreement with nonlinear time-series analysis of the data. The striking success of the model at predicting the shifts in amplitude and stability along the geographical gradient in northern Europe provides strong support for the key role of specialist and generalist predators in vole population dynamics.},
author = {Turchin, P and Hanski, I},
doi = {10.1086/286027},
isbn = {0003-0147 (Print)$\backslash$r0003-0147 (Linking)},
issn = {0003-0147},
journal = {American Naturalist Nat},
keywords = {AVIAN,CYCLES,DENSITY VARIATIONS,FUNCTIONAL-RESPONSES,MICROTUS-AGRESTIS,NORTHERN FENNOSCANDIA,POPULATIONS,PREDATORS,RODENT,TERM POPULATION,TIME-SERIES ANALYSIS,WEASEL MUSTELA-NIVALIS},
number = {5},
pages = {842--874},
pmid = {18811252},
title = {{An empirically based model for latitudinal gradient in vole population dynamics}},
url = {http://www.ncbi.nlm.nih.gov/pubmed/18811252},
volume = {149},
year = {1997}
}
@article{Romanuk2005,
abstract = {The success of species invasions depends on both the characteristics of the invaded habitat and the traits of the invasive species. At local scales biodiversity may act as a barrier to invasion; however, the mechanism by which biodiversity confers invasion resistance to a community has been the subject of considerable debate. The purpose of this study was to test the hypothesis that productivity and diversity affected the ability of a regionally available species to colonize communities from which it is absent. We hypothesized that the invasibility of rock pool invertebrate communities would increase with increasing nutrients and decrease with increasing diversity. We tested this possibility using naturally invaded outdoor aquatic microcosms. We demonstrated that the invasibility of an experimental multi-trophic aquatic community by a competitive native midge species (Ceratopogonidae: Dasyhelea sp.) was determined by an interaction between resource availability, diversity, and the densities of two competitive ostracods species. Nutrient enrichment increased invasion success; however, within nutrient-enriched microcosms, invasion success was highest in the low-diversity treatments. Our results suggest that resource availability may in fact be the principal mechanism determining invasibility at local scales in multi-trophic rock pool communities; however resource availability can be determined by both nutrient input as well as by the diversity of the biotic community.},
author = {Romanuk, Tamara N. and Kolasa, Jurek},
doi = {10.1007/s10530-004-0997-8},
isbn = {1387-3547},
issn = {13873547},
journal = {Biological Invasions},
keywords = {Aquatic,Dasyhelea,Invasion,Nutrient limitation,Rock pools,Species richness,Zooplankton},
month = {jul},
number = {4},
pages = {711--722},
title = {{Resource limitation, biodiversity, and competitive effects interact to determine the invasibility of rock pool microcosms}},
url = {http://www.springerlink.com/index/10.1007/s10530-004-0997-8},
volume = {7},
year = {2005}
}
@article{Beisner2006,
abstract = {Productivity influences the availability of resources for colonizing species. Biodiversity may also influence invasibility of communities because of more complete use of resource types with increasing species richness. We hypothesized that communities with higher environmental productivity and lower species richness should be more invasible by a competitor than those where productivity is low or where richness is high.We experimentally examined the invasion resistance of herbivorous meiofauna of Jamaican rock pools by a competitor crustacean (Ostracoda: Potamocypris sp. (Brady)) by contrasting three levels of nutrient input and four levels of species richness. Although relative abundance (dominance) of the invasive was largely unaffected by resource availability, increasing resources did increase the success rate of establishment. Effects of species richness on dominance were more pronounced with a trend towards the lowest species richness treatment of 2 resident species being more invasible than those with 4, 6, or 7 species. These results can be attributed to a ?sampling effect' associated with the introduction of Alona davidii (Richard) into the higher biodiversity treatments. Alona dominated the communities where it established and precluded dominance by the introduced ostracod. Our experimental study supports the idea that niche availability and community interactions define community invasibility and does not support the application of a neutral community model for local food web management where predictions of exotic species impacts are needed.},
author = {Beisner, Beatrix E. and Hovius, Jonathan and Hayward, April and Kolasa, Jurek and Romanuk, Tamara N.},
doi = {10.1007/s10530-005-2061-8},
isbn = {5149874647},
issn = {13873547},
journal = {Biological Invasions},
keywords = {Biodiversity-invasion relationships,Meiofauna,Niche,Resource availability,Sampling effect},
month = {jan},
number = {4},
pages = {655--664},
title = {{Environmental productivity and biodiversity effects on invertebrate community invasibility}},
url = {http://www.springerlink.com/index/10.1007/s10530-005-2061-8},
volume = {8},
year = {2006}
}
@article{Romanuk2006,
abstract = {With global freshwater biodiversity declining at an even faster rate than in the most disturbed terrestrial ecosystems, understanding the effects of changing environmental conditions on relationships between biodiversity and the variability of community and population processes in aquatic ecosystems is of significant interest. Evidence is accumulating that biodiversity loss results in more variable communities; however, the mechanisms underlying this effect have been the subject of considerable debate. We manipulated species richness and nutrients in outdoor aquatic microcosms composed of naturally occurring assemblages of zooplankton and benthic invertebrates to determine how the relationship between species richness and variability might change under different nutrient conditions. Temporal variability of populations and communities decreased with increasing species richness in low nutrient microcosms. In contrast, we found no relationship between species richness and either population or community variability in nutrient enriched microcosms. Of the different mechanisms we investigated (e.g. overyielding, statistical averaging, insurance effects, and the stabilizing effect of species richness on populations) the only one that was consistent with our results was that increases in species richness led to more stable community abundances through the stabilizing effect of species richness on the component populations. While we cannot conclusively determine the mechanism(s) by which species richness stabilized populations, our results suggest that more complete resource-use in the more species-rich low nutrient communities may have dampened population fluctuations.},
author = {Romanuk, Tamara N. and Vogt, Richard J. and Kolasa, Jurek},
doi = {10.1111/j.2006.0030-1299.14739.x},
isbn = {0030-1299},
issn = {00301299},
journal = {Oikos},
month = {aug},
number = {2},
pages = {291--302},
title = {{Nutrient enrichment weakens the stabilizing effect of species richness}},
url = {http://doi.wiley.com/10.1111/j.2006.0030-1299.14739.x},
volume = {114},
year = {2006}
}
@article{Romanuk2009,
abstract = {A central and perhaps insurmountable challenge of invasion ecology is to predict which combinations of species and habitats most effectively promote and prevent biological invasions. Here, we integrate models of network structure and nonlinear population dynamics to search for potential generalities among trophic factors that may drive invasion success and failure. We simulate invasions where 100 different species attempt to invade 150 different food webs with 15-26 species and a wide range (0.06-0.32) of connectance. These simulations yield 11 438 invasion attempts by non-basal species, 47 per cent of which are successful. At the time of introduction, whether or not the invader is a generalist best predicts final invasion success; however, once the invader establishes itself, it is best distinguished from unsuccessful invaders by occupying a lower trophic position and being relatively invulnerable to predation. In general, variables that reflect the interaction between an invading species and its new community, such as generality and trophic position, best predict invasion success; however, for some trophic categories of invaders, fundamental species traits, such as having the centre of the feeding range low on the theoretical niche axis (for non-omnivorous and omnivorous herbivores), or the topology of the food web (for tertiary carnivores), best predict invasion success. Across all invasion scenarios, a discriminant analysis model predicted successful and failed invasions with 76.5 per cent accuracy for properties at the time of introduction or 100 per cent accuracy for properties at the time of establishment. More generally, our results suggest that tackling the challenge of predicting the properties of species and habitats that promote or inhibit invasions from food web perspective may aid ecologists in identifying rules that govern invasions in natural ecosystems.},
author = {Romanuk, Tamara N and Zhou, Yun and Brose, Ulrich and Berlow, Eric L and Williams, Richard J and Martinez, Neo D},
doi = {10.1098/rstb.2008.0286},
isbn = {0962-8436},
issn = {1471-2970},
journal = {Philosophical transactions of the Royal Society of London. Series B, Biological sciences},
keywords = {Animals,Biodiversity,Ecosystem,Energy Metabolism,Food Chain,Models, Biological,Nonlinear Dynamics,Population Dynamics},
month = {jun},
number = {1524},
pages = {1743--54},
pmid = {19451125},
title = {{Predicting invasion success in complex ecological networks.}},
url = {http://rstb.royalsocietypublishing.org/content/364/1524/1743.short},
volume = {364},
year = {2009}
}
@article{Srivastava2004,
abstract = {Several recent, high-impact ecological studies feature natural microcosms as tools for testing effects of fragmentation, metacommunity theory or links between biodiversity and ecosystem processes. These studies combine the microcosm advantages of small size, short generation times, contained structure and hierarchical spatial arrangement with advantages of field studies: natural environmental variance, 'openness' and realistic species combinations with shared evolutionary histories. This enables tests of theory pertaining to spatial and temporal dynamics, for example, the effects of neighboring communities on local diversity, or the effects of biodiversity on ecosystem function. Using examples, we comment on the position of natural microcosms in the roster of ecological research strategies and tools. We conclude that natural microcosms are as versatile as artificial microcosms, but as complex and biologically realistic as other natural systems. Research to date combined with inherent attributes of natural microcosms make them strong candidate model systems for ecology.},
author = {Srivastava, Diane S. and Kolasa, Jurek and Bengtsson, Jan and Gonzalez, Andrew and Lawler, Sharon P. and Miller, Thomas E. and Munguia, Pablo and Romanuk, Tamara and Schneider, David C. and Trzcinski, M. Kurtis},
doi = {10.1016/j.tree.2004.04.010},
isbn = {0169-5347},
issn = {01695347},
journal = {Trends in Ecology and Evolution},
month = {jul},
number = {7},
pages = {379--384},
pmid = {16701289},
title = {{Are natural microcosms useful model systems for ecology?}},
url = {http://www.ncbi.nlm.nih.gov/pubmed/16701289},
volume = {19},
year = {2004}
}
@article{Allesina2012,
abstract = {Forty years ago, Robert May questioned a central belief in ecology by proving that sufficiently large or complex ecological networks have probability of persisting close to zero. To prove this point, he analyzed large networks in which species interact at random. However, in natural systems pairs of species have well-defined interactions (e.g., predator-prey, mutualistic or competitive). Here we extend May's results to these relationships and find remarkable differences between predator-prey interactions, which increase stability, and mutualistic and competitive, which are destabilizing. We provide analytic stability criteria for all cases. These results have broad applicability in ecology. For example, we show that, surprisingly, the probability of stability for predator-prey networks is decreased when we impose realistic food web structure or we introduce a large preponderance of weak interactions. Similarly, stability is negatively impacted by nestedness in bipartite mutualistic networks.},
archivePrefix = {arXiv},
arxivId = {1105.2071},
author = {Allesina, Stefano and Tang, Si},
doi = {10.1038/nature10832},
eprint = {1105.2071},
file = {:Users/alyssacirtwill/Documents/Papers/Allesina, Tang{\_}2012{\_}Nature.pdf:pdf},
isbn = {0028-0836},
issn = {0028-0836},
journal = {Nature},
keywords = {Animals,Biological,Competitive Behavior,Competitive Behavior: physiology,Ecosystem,Food Chain,Models,Plant Physiological Phenomena,Predatory Behavior,Symbiosis},
month = {mar},
number = {7388},
pages = {205--208},
pmid = {22343894},
publisher = {Nature Publishing Group},
title = {{Stability criteria for complex ecosystems}},
url = {http://arxiv.org/abs/1105.2071{\%}5Cnhttp://www.nature.com/doifinder/10.1038/nature10832},
volume = {483},
year = {2012}
}
@article{Chalcraft2011,
author = {Chalcraft, David R and Williams, John W and Smith, Melinda D and Willig, Michael R},
keywords = {biodiversity,environmental,grasslands,heterogeneity,lter,productivity,scale,species composition,species richness,species turnover,stability,terrestrial plants},
number = {10},
pages = {2701--2708},
title = {{Scale Dependence in the Species-Richness-Productivity Relationship : The Role of Species Turnover SCALE DEPENDENCE IN THE SPECIES-RICHNESS-PRODUCTIVITY RELATIONSHIP : THE ROLE OF SPECIES TURNOVER}},
volume = {85},
year = {2011}
}
@article{Dalsgaard2011,
abstract = {Large-scale geographical patterns of biotic specialization and the underlying drivers are poorly understood, but it is widely believed that climate plays an important role in determining specialization. As climate-driven range dynamics should diminish local adaptations and favor generalization, one hypothesis is that contemporary biotic specialization is determined by the degree of past climatic instability, primarily Quaternary climate-change velocity. Other prominent hypotheses predict that either contemporary climate or species richness affect biotic specialization. To gain insight into geographical patterns of contemporary biotic specialization and its drivers, we use network analysis to determine the degree of specialization in plant-hummingbird mutualistic networks sampled at 31 localities, spanning a wide range of climate regimes across the Americas. We found greater biotic specialization at lower latitudes, with latitude explaining 20-22{\%} of the spatial variation in plant-hummingbird specialization. Potential drivers of specialization--contemporary climate, Quaternary climate-change velocity, and species richness--had superior explanatory power, together explaining 53-64{\%} of the variation in specialization. Notably, our data provides empirical evidence for the hypothesized roles of species richness, contemporary precipitation and Quaternary climate-change velocity as key predictors of biotic specialization, whereas contemporary temperature and seasonality seem unimportant in determining specialization. These results suggest that both ecological and evolutionary processes at Quaternary time scales can be important in driving large-scale geographical patterns of contemporary biotic specialization, at least for co-evolved systems such as plant-hummingbird networks.},
author = {Dalsgaard, Bo and Mag{\aa}rd, Else and Fjelds{\aa}, Jon and Gonz{\'{a}}lez, Ana M Mart{\'{i}}n and Rahbek, Carsten and Olesen, Jens M. and Ollerton, Jeff and Alarc{\'{o}}n, Ruben and Araujo, Andrea Cardoso and Cotton, Peter a. and Lara, Carlos and Machado, Caio Graco and Sazima, Ivan and Sazima, Marlies and Timmermann, Allan and Watts, Stella and Sandel, Brody and Sutherland, William J. and Svenning, Jens Christian},
doi = {10.1371/journal.pone.0025891},
isbn = {1932-6203},
issn = {19326203},
journal = {PLoS ONE},
keywords = {Animals,Biodiversity,Birds,Climate Change,Geography,Plants,Rain},
month = {jan},
number = {10},
pages = {e25891},
pmid = {21998716},
title = {{Specialization in plant-hummingbird networks is associated with species richness, contemporary precipitation and quaternary climate-change velocity}},
url = {http://www.pubmedcentral.nih.gov/articlerender.fcgi?artid=3187835{\&}tool=pmcentrez{\&}rendertype=abstract},
volume = {6},
year = {2011}
}
@article{Schleuning2011,
abstract = {The degree of interdependence and potential for shared coevolutionary history of frugivorous animals and fleshy-fruited plants are contentious topics. Recently, network analyses revealed that mutualistic relationships between fleshy-fruited plants and frugivores are mostly built upon generalized associations. However, little is known about the determinants of network structure, especially from tropical forests where plants' dependence on animal seed dispersal is particularly high. Here, we present an in-depth analysis of specialization and interaction strength in a plant-frugivore network from a Kenyan rain forest. We recorded fruit removal from 33 plant species in different forest strata (canopy, midstory, understory) and habitats (primary and secondary forest) with a standardized sampling design (3447 interactions in 924 observation hours). We classified the 88 frugivore species into guilds according to dietary specialization (14 obligate, 28 partial, 46 opportunistic frugivores) and forest dependence (50 forest species, 38 visitors). Overall, complementary specialization was similar to that in other plant-frugivore networks. However, the plant-frugivore interactions in the canopy stratum were less specialized than in the mid- and understory, whereas primary and secondary forest did not differ. Plant specialization on frugivores decreased with plant height, and obligate and partial frugivores were less specialized than opportunistic frugivores. The overall impact of a frugivore increased with the number of visits and the specialization on specific plants. Moreover, interaction strength of frugivores differed among forest strata. Obligate frugivores foraged in the canopy where fruit resources were abundant, whereas partial and opportunistic frugivores were more common on mid- and understory plants, respectively. We conclude that the vertical stratification of the frugivore community into obligate and opportunistic feeding guilds structures this plant-frugivore network. The canopy stratum comprises stronger links and generalized associations, whereas the lower strata are composed of weaker links and more specialized interactions. Our results suggest that seed-dispersal relationships of plants in lower forest strata are more prone to disruption than those of canopy trees.},
author = {Schleuning, Matthias and Bl{\"{u}}thgen, Nico and Fl{\"{o}}Rchinger, Martina and Braun, Julius and Schaefer, H. Martin and B{\"{O}}hing-Gaese, Katrin},
doi = {10.1890/09-1842.1},
isbn = {0012-9658},
issn = {00129658},
journal = {Ecology},
keywords = {Africa,Birds,Coevolution,Diet specialization,Frugivory,Jakamega Forest,Kenya,Monkeys,Mutualistic network,Plant-animal interactions,Rain forest,Seed dispersal},
month = {jan},
number = {1},
pages = {26--36},
pmid = {21560673},
title = {{Specialization and interaction strength in a tropical plant-frugivore network differ among forest strata}},
url = {http://www.ncbi.nlm.nih.gov/pubmed/21560673},
volume = {92},
year = {2011}
}
@article{Jennings2003,
abstract = {Maximum food-chain length has been correlated with resource availability, ecosystem size, environmental stability and colonization history. Some of these correlations may result from environmental effects on predator-prey body size ratios. We investigate relationships between maximum food-chain length, predator-prey mass ratios, primary production and environmental stability in marine food webs with a natural history of community assembly. Our analyses provide empirical evidence that smaller mean predator-prey body size ratios are characteristic of more stable environments and that food chains are longer when mean predator-prey body size ratios are small. We conclude that environmental effects on predator-prey body size ratios contribute to observed differences in maximum food-chain length.},
author = {Jennings, Simon and Warr, Karema J},
doi = {10.1098/rspb.2003.2392},
isbn = {0962-8452},
issn = {0962-8452},
journal = {Proceedings of the Royal Society of London B},
keywords = {food chain length,food web,predator,prey interactions},
month = {jul},
number = {1522},
pages = {1413--1417},
pmid = {12965034},
title = {{Smaller predator-prey body size ratios in longer food chains.}},
url = {http://www.pubmedcentral.nih.gov/articlerender.fcgi?artid=1691388{\&}tool=pmcentrez{\&}rendertype=abstract},
volume = {270},
year = {2003}
}
@article{Lee,
abstract = {Nestedness characterizes the linkage pattern of networked systems, indicating the likelihood that a node is linked to the neighbors of the nodes with larger degrees than it. Networks of mutualistic relationship between distinct groups of species in ecological communities exhibit such nestedness, which is known to support the network's robustness. Despite such importance, the quantitative characteristics of nestedness are little understood. Here, we take a graph-theoretic approach to derive the scaling properties of nestedness in various model networks. Our results show how the heterogeneous connectivity patterns enhance nestedness. Also, we find that the nestedness of bipar- tite networks depends sensitively on the fraction of different types of nodes, causing nestedness to scale differently for nodes of different types.},
archivePrefix = {arXiv},
arxivId = {arXiv:1110.2825v2},
author = {Lee, Deok-Sun and Maeng, Seong Eun and Lee, Jae Woo},
doi = {10.3938/jkps.60.648},
eprint = {arXiv:1110.2825v2},
isbn = {0374-4884},
issn = {0374-4884},
journal = {Journal of the Korean Physical Society},
keywords = {10,3938,60,648,complex network,doi,jkps,nestedness,scaling},
number = {4},
pages = {648--656},
title = {{Scaling of nestedness in complex networks}},
volume = {60},
year = {2012}
}
@article{Karr1976,
author = {Karr, J. R.},
journal = {The American Naturalist},
number = {976},
pages = {973--994},
title = {{Seasonality, Resource Availability, and Community Diversity in Tropical Bird Communities}},
url = {http://www.jstor.org/stable/10.2307/2460025},
volume = {110},
year = {1976}
}
@article{Martinez1994,
abstract = {A demographic study on the social spider Anelosimus eximius (Araneae: Theridiidae) demonstrates no differential mortality of the sexes during the age of reproduction and no large difference in their maturation times to explain the highly female-biased sex ratios in adults. Moreover, sex ratios within the range of 0.04 to 0.40 males per female are already present at the earliest stage at which sexes can be distinguished in the field. Fisher's theory predicts a 1:1 sex ratio as evolutionarily stable. How, then, are the observed ratios attained and maintained? It is suggested that the unique population structure and dynamics of this social spider resulted in a change of balance between the opposing forces of group and individual selection, making evolutionarily stable a sex ratio that increases colony survival and proliferation.},
author = {Avil{\'{e}}s, Leticia},
doi = {10.2307/2678832},
isbn = {1630130044},
issn = {10773711},
journal = {The American Naturalist},
keywords = {coexistence,competition,dispersal,erogeneity,kernels,on biological,regional-scale ecological dynamics depend,spatial het-,spatial scale},
number = {1},
pages = {1--12},
pmid = {17891731},
title = {{Sex-Ratio Bias and Possible Group Selection in the Social Spider Anelosimus eximius}},
url = {http://www.jstor.org/stable/2461281},
volume = {128},
year = {1986}
}
@article{Srinivasan2007,
abstract = {Although an ecosystem's response to biodiversity loss depends on the order in which species are lost, the extinction sequences generally used to explore such responses in food webs have been ecologically unrealistic. We investigate how several extinction orders affect the minimum number of secondary extinctions expected within pelagic food webs from 34 temperate freshwater lakes. An ecologically plausible extinction order is derived from the geographically nested pattern of species composition among the lakes and is corroborated by species' pH tolerances. Simulations suggest that lake communities are remarkably robust to this realistic extinction order and highly sensitive to the reverse sequence of species loss. This sensitivity is not well explained by the known sensitivity of networks to the loss of highly connected species but appears to be better explained by our observation that trophic specialists preferentially consume widely distributed species at low risk of extinction. Our results highlight an important aspect of community organization that may help to maintain biodiversity amidst changing environments.},
author = {Srinivasan, U. Thara and Dunne, Jennifer A. and Harte, John and Martinez, Neo D.},
doi = {10.1890/06-0971},
isbn = {0012-9658},
issn = {00129658},
journal = {Ecology},
keywords = {Acidification,Adirondack lakes,Biodiversity,Complex networks,Extinction order,Food web,Lake communities,Nestedness,Pelagic,Random extinctions,Robustness,Species loss},
number = {3},
pages = {671--682},
pmid = {17503595},
title = {{Response of complex food webs to realistic extinction sequences}},
url = {http://www.esajournals.org/doi/abs/10.1890/06-0971},
volume = {88},
year = {2007}
}
@article{Vazquez2005,
abstract = {1. Recent studies have evaluated the distribution of specialization in species interaction networks. Species abundance patterns have been hypothesized to determine observed topological patterns. We evaluate this hypothesis in the context of host-parasite interaction networks.2. We used two independent series of data sets, one consisting of data for seven sites describing interactions between freshwater fish and their metazoan parasites and another consisting of data for 25 localities describing interactions between fleas and their mammalian hosts. We evaluated the influence of species abundance patterns on the distribution of specialization in these host-parasite interaction networks with the aid of null models.3. In parallel with recent studies of plant-animal mutualistic networks, our analyses suggest that host-parasite interactions in these systems are highly asymmetric: specialist parasites tend to interact with hosts with high parasite richness, whereas hosts with low parasite richness tend to interact mainly with generalist parasites.4. The observed distribution of specialization was predicted by a null model that assumed that species-specific probabilities of being assigned a link during the randomization process were roughly proportional to their relative abundance. Thus, abundant hosts tend to harbour richer parasite faunas, with a high proportion of rare specialists.},
author = {V{\'{a}}zquez, Diego P. and Poulin, Robert and Krasnov, Boris R. and Shenbrot, Georgy I.},
doi = {10.1111/j.1365-2656.2005.00992.x},
isbn = {0021-8790},
issn = {00218790},
journal = {Journal of Animal Ecology},
keywords = {Abundance,Asymmetric specialization,Host-parasite interactions,Network structure,Null models},
number = {5},
pages = {946--955},
pmid = {231612100015},
title = {{Species abundance and the distribution of specialization in host-parasite interaction networks}},
volume = {74},
year = {2005}
}
@article{Vazquez2004,
abstract = {We examine Robert MacArthur's hypothesis that niche breadth is positively associated with latitude (the latitude-niche breadth hypothesis). This idea has been influential and long standing, yet no studies have evaluated its generality or the validity of its assumptions. We review the theoretical arguments suggesting a positive relationship between niche breadth and latitude. We also use available evidence to evaluate the assumptions and predictions of MacArthur's latitude-niche breadth hypothesis. We find that neither the assumptions nor the predictions of the hypothesis are supported by data. We propose an alternative hypothesis linking latitude with niche breadth. Unlike previous ideas, our conceptual framework does not require equilibrial assumptions and is based on recently uncovered patterns of species interactions.},
author = {V{\'{a}}zquez, Diego P and Stevens, Richard D},
doi = {10.1086/421445},
isbn = {0003-0147},
issn = {0003-0147},
journal = {The American naturalist},
keywords = {and above all a,breadth,economics of consumer choice,environmental variability,latitudinal gradients,niche,population variability,specialization,species richness,steady or,suggest,that a productive,the},
month = {jul},
number = {1},
pages = {E1--E19},
pmid = {15266376},
title = {{The latitudinal gradient in niche breadth: concepts and evidence.}},
url = {http://www.ncbi.nlm.nih.gov/pubmed/15266376},
volume = {164},
year = {2004}
}
@article{Vazquez2002,
abstract = {Niche breadth of species has been hypothesized to be associated with species' responses to disturbance. Disturbance is usually believed to affect specialists negatively, while generalists are believed to benefit from disturbance; we call this the "specialization-disturbance" hypothesis. We also propose an associated hypothesis (the "specialization-asymmetry-disturbance" hypothesis) under which both specialization and asymmetry of interactions would explain species' responses to disturbance. We test these hypotheses using data from a plant-pollinator system that has been grazed by cattle (i.e., a biological disturbance) in southern Argentina. We quantified specialization in species interactions, specialization of interaction partners, and species' responses to disturbance. We found no relationship between degree of specialization and a species' response to disturbance. We also found that plant-pollinator interactions tend to be asymmetric in this system; there was no relationship between the degree of specialization of a given species and the degree of specialization of its interaction partners. However, asymmetry of interactions did not explain the variability in species' responses to disturbance. Thus, both hypotheses are rejected by our data. Possible reasons include failure to assess crucial resources, substantial direct effects of disturbance, inaccurate measures of specialization, difficulty detecting highly nonlinear relationships, and limitations of a nonexperimental approach. Or, in fact, there may be no relationship between specialization and response to disturbance.},
author = {V{\'{a}}zquez, Diego P and Simberloff, Daniel},
doi = {10.1086/339991},
isbn = {0003-0147},
issn = {0003-0147},
journal = {The American naturalist},
keywords = {asymmetric interactions,cattle,disturbance,generali-,grazing,in fact,ization,most fundamental con-,mutualism,plant-pollinator interactions,southern andes,special-,specialization is arguably the,temperate forests of the,zation},
number = {6},
pages = {606--623},
pmid = {18707385},
title = {{Ecological specialization and susceptibility to disturbance: conjectures and refutations.}},
url = {http://www.jstor.org/stable/10.1086/339991},
volume = {159},
year = {2002}
}
@article{VanderZanden2007,
abstract = {Food chain length is a fundamental ecosystem property, and plays a central role in determining ecosystem functioning. Recent advances in the field of stable isotope ecology allow the estimation of food chain length (FCL) from stable nitrogen isotope (515N) data. We conducted a global literature synthesis and estimated FCL for 219 lake, stream, and marine ecosystems. Streams had shorter food chains ({\~{}}3.5 trophic levels) than marine and lake ecosystems ({\~{}}4.0 trophic levels). In marine systems, inclusion of marine mammals increased FCL by 2/3 of a trophic level. For each ecosystem type, estimates of FCL were normally distributed and spanned two full trophic levels. Comparison with published connectance food webs revealed similar mean FCL values, though stable isotope-derived FCL estimates were less variable. At the global scale, FCL showed weak or no relationships with ecosystem size, mean annual air temperature, or latitude. Our study highlights the utility of stable isotopes for quantifying among-system food web variability, and the application of this approach for assessing global- scale},
author = {{Jake Vander Zanden}, M. and {W. Fetzer}, William},
doi = {10.1111/j.2007.0030-1299.16036.x},
isbn = {0030-1299},
issn = {00301299},
journal = {Oikos},
month = {aug},
number = {8},
pages = {1378--1388},
title = {{Global patterns of aquatic food chain length}},
url = {http://doi.wiley.com/10.1111/j.2007.0030-1299.16036.x},
volume = {116},
year = {2007}
}
@article{Thompson2007b,
abstract = {The concept of trophic levels is one of the oldest in ecology and informs our understanding of energy flow and top-down control within food webs, but it has been criticized for ignoring omnivory. We tested whether trophic levels were apparent in 58 real food webs in four habitat types by examining patterns of trophic position. A large proportion of taxa (64.4{\%}) occupied integer trophic positions, suggesting that discrete trophic levels do exist. Importantly however, the majority of those trophic positions were aggregated around integer values of 0 and 1, representing plants and herbivores. For the majority of the real food webs considered here, secondary consumers were no more likely to occupy an integer trophic position than in randomized food webs. This means that, above the herbivore trophic level, food webs are better characterized as a tangled web of omnivores. Omnivory was most common in marine systems, rarest in streams, and intermediate in lakes and terrestrial food webs. Trophic-level-based concepts such as trophic cascades may apply to systems with short food chains, but they become less valid as food chains lengthen.},
author = {Thompson, Ross M. and Hemberg, Martin and Starzomski, Brian M. and Shurin, Jonathan B.},
doi = {10.1890/05-1454},
isbn = {0012-9658},
issn = {00129658},
journal = {Ecology},
keywords = {Food webs,Meta-analysis,Null models,Omnivory,Trophic levels,Trophic position},
number = {3},
pages = {612--617},
pmid = {17503589},
title = {{Trophic levels and trophic tangles: the prevalence of omnivory in real food webs}},
url = {http://www.esajournals.org/doi/abs/10.1890/05-1454},
volume = {88},
year = {2007}
}
@article{Thompson2005,
abstract = {The majority of food-web studies currently used to test ecological theory have integrated information over large spatial and temporal scales. We aimed to assess the degree to which food webs display patch-scale variation, and the consequences for emergent properties at the larger scale of the stream reach. Spatial heterogeneity in ecological conditions (habitat structure and food resources) and food-web structure were measured in three streams. All food webs were constructed using equivalent effort at a patch scale (0.06 m(2)) and a reach scale (30-m stream length). A mosaic of habitat structure and food resources was reflected in considerable variability in food-web structure among patches, but there was less variation within than among streams. The variability in food-web attributes among patches could not always be related to ecological conditions, but food resource availability affected connectedness of food webs and trophic structure (measured as functional feeding groups). Of particular note was the result that reach-summary food webs were consistently different from patch-specific food webs in each stream. Reach-scale food webs underestimated connectance but overestimated prey: predator ratios and the number of trophic links. Summary webs sometimes placed species together that, in fact, did not coexist in the field. Such within-site food-web heterogeneity needs to be taken into account in future multiple-site comparisons of food-web structure.},
author = {Thompson, R M and Townsend, C R},
isbn = {0012-9658},
journal = {Ecology},
keywords = {2004,and reanalysis of existing,brose et al,connectance,data,ecosystem size,food web,have strongly indicated both,holt 2002,new zealand,productivit,productivity,reach,south island,spatial scale,species richness,stream patch vs,streams,the pervasiveness},
number = {7},
pages = {1916--1925},
title = {{Food-web topology varies with spatial scale in a patchy environment}},
url = {http://www.esajournals.org/doi/abs/10.1890/04-1352},
volume = {86},
year = {2005}
}
@article{Srivastava2006,
abstract = {Although previous studies have shown that ecosystem functions are affected by either trophic structure or habitat structure, there has been little consideration of their combined effects. Such interactions may be particularly important in systems where habitat and trophic structure covary. I use the aquatic insects in bromeliads to examine the combined effects of trophic structure and habitat structure on a key ecosystem function: detrital processing. In Costa Rican bromeliads, trophic structure naturally covaries with both habitat complexity and habitat size, precluding any observational analysis of interactions between factors. I therefore designed mesocosms that allowed each factor to be manipulated separately. Increases in mesocosm complexity reduced predator (damselfly larva) efficiency, resulting in high detritivore abundances, indirectly increasing detrital processing rates. However, increased complexity also directly reduced the per capita foraging efficiency of the detritivores. Over short time periods, these trends effectively cancelled each other out in terms of detrital processing. Over longer time periods, more complex patterns emerged. Increases in mesocosm size also reduced both predator efficiency and detritivore efficiency, leading to no net effect on detrital processing. In many systems, ecosystem functions may be impacted by strong interactions between trophic structure and habitat structure, cautioning against examining either effect in isolation.},
author = {Srivastava, Diane S.},
doi = {10.1007/s00442-006-0467-3},
isbn = {0029-8549},
issn = {00298549},
journal = {Oecologia},
keywords = {Aquatic insects,Decomposition,Habitat complexity,Habitat size,Predation},
month = {sep},
number = {3},
pages = {493--504},
pmid = {16896779},
title = {{Habitat structure, trophic structure and ecosystem function: Interactive effects in a bromeliad-insect community}},
url = {http://www.ncbi.nlm.nih.gov/pubmed/16896779},
volume = {149},
year = {2006}
}
@article{Spiller1994,
author = {Spiller, David A. and Schoener, Thomas W},
journal = {Ecology},
keywords = {compensatory predation,food web,lizards,predators,species interactions,spiders},
number = {1},
pages = {182--196},
title = {{Effects of top and intermediate predators in a terrestrial food web}},
volume = {75},
year = {1994}
}
@article{Schemske2009,
abstract = {Biotic interactions are believed to play a role in the origin and maintenance of species diversity, and multiple hypotheses link the latitudinal diversity gradient to a presumed gradient in the importance of biotic interactions. Here we address whether biotic interactions are more important at low latitudes, finding support for this hypothesis from a wide range of interactions. Some of the best-supported examples are higher herbivory and insect predation in the tropics, and predominantly tropical mutualisms such as cleaning symbioses and ant-plant interactions. For studies that included tropical regions, biotic interactions were never more important at high latitudes. Although our results support the hypothesis that biotic interactions are more important in the tropics, additional research is needed, including latitudinal comparison of rates of molecular evolution for genes involved in biotic interacti ons, estimates of gradients in interaction strength, and phylogenetic comparisons of the traits that mediate biotic interactions.},
author = {Schemske, D W and Mittelbach, G G and Cornell, H V and Sobel, J M and Roy, K},
doi = {10.1146/annurev.ecolsys.39.110707.173430},
isbn = {1543-592x},
issn = {1543-592X},
journal = {Annual Review of Ecology Evolution and Systematics},
keywords = {biotic factors,community diversity,diversity gradient,drilling predation,fungal endophytes,latitudinal gradient,life-history traits,nest predation,parasite species richness,plant-herbivore interactions,salt-marsh plants,sexual size dimorphism,species diversity,tropical forests,tropics},
month = {dec},
number = {2009},
pages = {245--269},
pmid = {272455700012},
title = {{Is There a Latitudinal Gradient in the Importance of Biotic Interactions?}},
volume = {40},
year = {2009}
}
@article{Roth2007,
abstract = {Geographical gradients in the stability of cyclic populations of herbivores and their predators may relate to the degree of specialization of predators. However, such changes are usually associated with transition from specialist to generalist predator species, rather than from geographical variation in dietary breadth of specialist predators. Canada lynx (Lynx canadensis) and snowshoe hare (Lepus americanus) populations undergo cyclic fluctuations in northern parts of their range, but cycles are either greatly attenuated or lost altogether in the southern boreal forest where prey diversity is higher. We tested the influence of prey specialization on population cycles by measuring the stable carbon and nitrogen isotope ratios in lynx and their prey, estimating the contribution of hares to lynx diet across their range, and correlating this degree of specialization to the strength of their population cycles. Hares dominated the lynx diet across their range, but specialization on hares decreased in southern and western populations. The degree of specialization correlated with cyclic signal strength indicated by spectral analysis of lynx harvest data, but overall variability of lynx harvest (the standard deviation of natural-log-transformed harvest numbers) did not change significantly with dietary specialization. Thus, as alternative prey became more important in the lynx diet, the fluctuations became decoupled from a regular cycle but did not become less variable. Our results support the hypothesis that alternative prey decrease population cycle regularity but emphasize that such changes may be driven by dietary shifts among dominant specialist predators rather than exclusively through changes in the predator community.},
author = {Roth, James D. and Marshall, John D. and Murray, Dennis L. and Nickerson, David M. and Steury, Todd D.},
doi = {10.1890/07-0147.1},
isbn = {0012-9658},
issn = {00129658},
journal = {Ecology},
keywords = {Alternative prey,Canada lynx,Lepus americanus,Lynx canadensis,Population cycles,Snowshoe hare,Specialist/generalist predation hypothesis,Stable isotope ratios},
month = {nov},
number = {11},
pages = {2736--2743},
pmid = {18051641},
title = {{Geographical gradients in diet affect population dynamics of Canada lynx}},
url = {http://www.ncbi.nlm.nih.gov/pubmed/18051641},
volume = {88},
year = {2007}
}
@article{Rohde1992,
author = {Rohde, Klaus},
journal = {Oikos},
number = {3},
pages = {514--527},
title = {{Latitudinal gradients in species diversity : the search for the primary cause}},
volume = {65},
year = {1992}
}
@incollection{Riede2010,
address = {Burlington},
author = {Riede, Jens O. and Rall, Bj{\"{o}}rn C. and Banasek-Richter, Carolin and Navarrete, Sergio A. and Wieters, Evie A. and Emmerson, Mark C. and Jacob, Ute and Brose, Ulrich},
booktitle = {Advances in Ecological Research},
doi = {10.1016/S0065-2504(10)42003-6},
editor = {Woodward, Guy},
file = {:Users/alyssacirtwill/Documents/Papers/Riede et al.{\_}2010{\_}Advances in Ecological Research.pdf:pdf},
isbn = {9780123813633},
pages = {139--170},
publisher = {Elsevier Ltd.},
title = {{Scaling of food-web properties with diversity and complexity across ecosystems}},
volume = {42},
year = {2010}
}
@article{Pennings2011,
abstract = {  Let V {\&}{\#}8834;R{\textless}sup{\textgreater}n{\textless}/sup{\textgreater}be a real algebraic set described by finitely many polynomials equations g{\textless}inf{\textgreater}j{\textless}/inf{\textgreater}(x)=0, j{\&}{\#}8712;J, and let f be a real polynomial, nonnegative on V. We show that for every {\&}{\#}8712;{\&}gt;0, there exist nonnegative scalars {\{}{\&}{\#}955;{\textless}inf{\textgreater}j{\textless}/inf{\textgreater}{\}}{\textless}inf{\textgreater}j{\&}{\#}8712;J{\textless}/inf{\textgreater}such that, for all r sufficiently large, f{\textless}inf{\textgreater}{\&}{\#}8712;r{\textless}/inf{\textgreater}+{\&}{\#}8721;{\textless}inf{\textgreater}j{\&}{\#}8712;J{\textless}/inf{\textgreater}{\&}{\#}955;{\textless}inf{\textgreater}j{\textless}/inf{\textgreater}g{\textless}sup{\textgreater}2{\textless}/sup{\textgreater}{\textless}inf{\textgreater}j{\textless}/inf{\textgreater}, is a sum of squares, for some polynomial f{\textless}inf{\textgreater}{\&}{\#}8712;r{\textless}/inf{\textgreater}with a simple and explicit form in terms of f and the parameters {\&}{\#}8712;{\&}gt;0, r{\&}{\#}8712;N, and such that ||f-f{\textless}inf{\textgreater}{\&}{\#}8712;r{\textless}/inf{\textgreater}||{\textless}inf{\textgreater}1{\textless}/inf{\textgreater}{\&}{\#}8594;0 as {\&}{\#}8712;{\&}{\#}8594;0. This representation is an obvious certificate of nonnegativity of f{\textless}inf{\textgreater}{\&}{\#}8712;r{\textless}/inf{\textgreater}on V, and valid with no assumption on V. In addition, this representation is also useful from a computational point of view, as we can define semidefinite programming relaxations to approximate the global minimum of f on a real algebraic set V, or a basic closed semi-algebraic set K, and again, with no assumption on V or K.},
archivePrefix = {arXiv},
arxivId = {math/0412400},
author = {Lasserre, Jean B.},
doi = {10.1109/CDC.2005.1583094},
eprint = {0412400},
isbn = {0780395689},
issn = {1052-6234},
journal = {Proceedings of the 44th IEEE Conference on Decision and Control, and the European Control Conference, CDC-ECC '05},
keywords = {gastropods,grasshoppers,herbivore interac-,interaction strength,latitude,plant,salt marsh,spartina,tions,top-down effects},
number = {9},
pages = {5837--5841},
pmid = {3541},
primaryClass = {math},
title = {{SOS approximation of polynomials nonnegative on an algebraic set}},
url = {http://www.esajournals.org/doi/abs/10.1890/04-1022},
volume = {2005},
year = {2005}
}
@article{Macpherson2002,
abstract = {The increase in species richness from the poles to the Equator has been observed in numerous terrestrial and aquatic taxa. A number of different hypotheses have been put forward as explanations for this trend, e.g. area and energy availability. However, whether these hypotheses apply to large spatial scales in marine environments remains unclear. The present study shows a clear latitudinal gradient from high to low latitude (from 80 degrees N to 70 degrees S) in marine species richness for 6643 species (fishes and invertebrates) in 10 different taxa dwelling in benthic and pelagic habitats on both sides of the Atlantic. The patterns in benthic taxa are strongly influenced by coastal hydrographic processes, with marked peaks and troughs, and consequently the gradients are not symmetric along both Atlantic sides. Pelagic taxa show a plateau-shaped distribution and the influence from coastal events on gradients could not be demonstrated. The relationships between species richness and different environmental factors indicate that area size does not explain the latitudinal pattern in benthic species richness on a large spatial scale. Sea-surface temperature (positive relationship) is the best predictor of this pattern for benthic species, and nitrate concentration (negative relationship) is the best predictor for pelagic species. The results call into question the existence of a single primary cause that would explain the pattern in marine species richness on a large spatial scale.},
author = {Macpherson, E},
doi = {10.1098/rspb.2002.2091},
isbn = {0962-8452},
issn = {0962-8452},
journal = {Proceedings. Biological sciences / The Royal Society},
keywords = {Animals,Atlantic Ocean,Chlorophyll,Chlorophyll: metabolism,Ecosystem,Fishes,Fishes: physiology,Invertebrates,Invertebrates: physiology,Logistic Models,Nitrates,Nitrates: analysis,Population Density,Seawater,Seawater: chemistry,Temperature},
month = {aug},
number = {1501},
pages = {1715--20},
pmid = {12204133},
title = {{Large-scale species-richness gradients in the Atlantic Ocean.}},
url = {http://www.pubmedcentral.nih.gov/articlerender.fcgi?artid=1691087{\&}tool=pmcentrez{\&}rendertype=abstract},
volume = {269},
year = {2002}
}
@article{Loeuille2005,
abstract = {Explaining the structure of terrestrial and aquatic food webs remains one of the most important challenges of ecological theory. Most existing models use emergent properties of food webs, such as diversity and connectance as parameters, to determine other food-web descriptors. Lower-level processes, in particular adaptation (whether by behavioral, developmental, or evolutionary mechanisms), are usually not considered. Here, we show that complex, realistic food webs may emerge by evolution from a single ancestor based on very simple ecological and evolutionary rules. In our model, adaptation acts on body size, whose impact on the metabolism and interactions of organisms is well established. Based on parameters defined at the organism scale, the model predicts emergent properties at the food-web scale. Variations of two key parameters (width of consumption niche and competition intensity) allow very different food-web structures and functionings to emerge, which are similar to those observed in some of the best-documented food webs.},
author = {Loeuille, Nicolas and Loreau, Michel},
doi = {10.1073/pnas.0408424102},
isbn = {0027-8424},
issn = {0027-8424},
journal = {Proceedings of the National Academy of Sciences of the United States of America},
keywords = {Adaptation, Biological,Animals,Biological Evolution,Body Size,Ecology,Feeding Behavior,Food Chain,Mathematics,Models, Biological},
month = {apr},
number = {16},
pages = {5761--5766},
pmid = {15824324},
title = {{Evolutionary emergence of size-structured food webs.}},
url = {http://www.pubmedcentral.nih.gov/articlerender.fcgi?artid=556288{\&}tool=pmcentrez{\&}rendertype=abstract},
volume = {102},
year = {2005}
}
@article{Jennings2005,
abstract = {No definitive explanation for the form of the relationship between species diversity and ecosystem productivity exists nor is there agreement on the mechanisms linking diversity and productivity across scales. Here, we examine changes in the form of the diversity-productivity relationship within and across the plant communities at three observational scales: plots, alliances, and physiognomic vegetation types (PVTs). Vascular plant richness data are from 4,760 20 m2 vegetation field plots. Productivity estimates in grams carbon per square meter are from annual net primary productivity (ANPP) models. Analyses with generalized linear models confirm scale dependence in the species diversity-productivity relationship. At the plot focus, the observed diversity-productivity relationship was weak. When plot data were aggregated to a focus of vegetation alliances, a hump-shaped relationship was observed. Species turnover among plots cannot explain the observed hump-shaped relationship at the alliance focus because we used mean plot richness across plots as our index of species richness for alliances and PVTs. The sorting of alliances along the productivity gradient appears to follow regional patterns of moisture availability, with alliances that occupy dry environments occurring within the increasing phase of the hump-shaped pattern, alliances that occupy mesic to hydric environments occurring near the top or in the decreasing phase of the curve, and alliances that occupy the wettest environments having the fewest species and the highest ANPP. This pattern is consistent with the intermediate productivity theory but appears to be inconsistent with the predictions of water-energy theory.},
author = {Jennings, Michael D. and Williams, John W. and Stromberg, Mark R.},
doi = {10.1007/s00442-005-0011-x},
isbn = {0029-8549},
issn = {00298549},
journal = {Oecologia},
keywords = {Biodiversity,Community ecology,GLM,Primary productivity},
month = {may},
number = {4},
pages = {607--618},
pmid = {15909130},
title = {{Diversity and productivity of plant communities across the Inland Northwest, USA}},
url = {http://www.ncbi.nlm.nih.gov/pubmed/15909130},
volume = {143},
year = {2005}
}
@article{Hillebrand2010,
abstract = {The decline of biodiversity with latitude has received great attention, but both the concise pattern and the causes of the gradient are under strong debate. Most studies of the latitudinal gradient comprise only one or few organism types and are often restricted to certain region or habitat types. To test for significant variation in the gradient between organisms, habitats, or regions, a meta-analysis was conducted on nearly 600 latitudinal gradients assembled from the literature. Each gradient was characterized by two effect sizes, strength (correlation coefficient) and slope, and additionally by 14 variables describing organisms, habitats, and regions. The analysis corroborated the high generality of the latitudinal diversity decline. Gradients on regional scales were significantly stronger and steeper than on local scales, and slopes also varied with sampling grain. Both strength and slope increased with organism body mass, and strength increased with trophic level. The body mass-effect size relation varied for ecto- versus homeotherm organisms and for different dispersal types, suggesting allometric effects on energy use and dispersal ability as possible mechanisms for the body mass effect. Latitudinal gradients were weaker and less steep in freshwater than in marine or terrestrial environments and differed significantly between continents and habitat types. The gradient parameters were not affected by hemisphere or the latitudinal range covered. This analysis is the first to describe these general and significant patterns, which have important consequences for models aiming to explain the latitudinal gradient.},
archivePrefix = {arXiv},
arxivId = {1011.1669v3},
author = {Hillebrand, Helmut},
doi = {10.4037/ajcc2016979},
eprint = {1011.1669v3},
isbn = {9788578110796},
issn = {10623264},
journal = {The American Naturalist},
keywords = {body mass,macroecology,species richness,trophic level},
number = {2},
pages = {192--211},
pmid = {25246403},
title = {{On the generality of the latitudinal diversity gradient}},
volume = {163},
year = {2004}
}
@article{Guimaraes2006,
abstract = {Mutualistic networks involving plants and their pollinators or frugivores have been shown recently to exhibit a particular asymmetrical organization of interactions among species called nestedness: a core of reciprocal generalists accompanied by specialist species that interact almost exclusively with generalists. This structure contrasts with compartmentalized assemblage structures that have been verified in antagonistic food webs. Here we evaluated whether nestedness is a property of another type of mutualism-the interactions between ants and extrafloral nectary-bearing plants--and whether species richness may lead to differences in degree of nestedness among biological communities. We investigated network structure in four communities in Mexico. Nested patterns in ant-plant networks were very similar to those previously reported for pollination and frugivore systems, indicating that this form of asymmetry in specialization is a common feature of mutualisms between free-living species, but not always present in species-poor systems. Other ecological factors also appeared to contribute to the nested asymmetry in specialization, because some assemblages showed more extreme asymmetry than others even when species richness was held constant. Our results support a promising approach for the development of multispecies coevolutionary theory, leading to the idea that specialization may coevolve in different but simple ways in antagonistic and mutualistic assemblages.},
author = {Guimar{\~{a}}es, Paulo R and Rico-Gray, Victor and dos Reis, S{\'{e}}rgio Furtado and Thompson, John N},
doi = {10.1098/rspb.2006.3548},
isbn = {0962-8452 (Print)$\backslash$n0962-8452 (Linking)},
issn = {0962-8452},
journal = {Proceedings. Biological sciences / The Royal Society},
keywords = {Adaptation, Physiological,Animals,Ants,Ants: physiology,Ecosystem,Food Chain,Models, Biological,Plant Physiological Phenomena,Population Dynamics},
month = {aug},
number = {1597},
pages = {2041--7},
pmid = {16846911},
title = {{Asymmetries in specialization in ant-plant mutualistic networks.}},
url = {http://www.pubmedcentral.nih.gov/articlerender.fcgi?artid=1635486{\&}tool=pmcentrez{\&}rendertype=abstract},
volume = {273},
year = {2006}
}
@article{Gillooly2001,
abstract = {We derive a general model, based on principles of biochemical kinetics and allometry, that characterizes the effects of temperature and body mass on metabolic rate. The model fits metabolic rates of microbes, ectotherms, endotherms (including those in hibernation), and plants in temperatures ranging from 0 degrees to 40 degrees C. Mass- and temperature-compensated resting metabolic rates of all organisms are similar: The lowest (for unicellular organisms and plants) is separated from the highest (for endothermic vertebrates) by a factor of about 20. Temperature and body size are primary determinants of biological time and ecological roles.},
author = {Gillooly, James F. and Brown, James H. and West, Geoffrey B. and Savage, Van M. and Charnov, Eric L.},
doi = {10.1126/science.1061967},
isbn = {0036-8075},
issn = {1095-9203},
journal = {Science (New York, N.Y.)},
keywords = {Amphibians,Amphibians: metabolism,Animals,Basal Metabolism,Body Constitution,Body Temperature,Body Weight,Carbon Dioxide,Carbon Dioxide: metabolism,Fishes,Fishes: metabolism,Fractals,Longevity,Mammals,Mammals: metabolism,Mathematics,Models, Biological,Oxygen Consumption,Plants,Plants: metabolism,Reptiles,Reptiles: metabolism,Species Specificity,Temperature},
month = {sep},
number = {5538},
pages = {2248--2251},
pmid = {11567137},
title = {{Effects of size and temperature on metabolic rate}},
url = {Gilooly{\_}etal{\_}science{\_}01.pdf{\%}5Cn10.1126/science.1061967{\%}5Cn11567137},
volume = {293},
year = {2001}
}
@article{Garvey2011,
abstract = {Abstract Most community-based models in ecology assume that all individuals within a species respond similarly to environmental conditions and thereby exert identical effects as consumers or prey. Rather, individuals differ among systems, with important implications for population demographics and community interactions. For widely distributed assemblages made up of poikilotherms with high first-year mortality, species-specific differences in growth reaction norms as affected by both temperature and genotype will influence biotic interactions. For a broadly distributed fish assemblage, first-year growth does not vary with latitude for a planktivorous prey species, but declines with increasing latitude for a terminal piscivore. Size-based competitive interactions between these species are likely to be more intense at high latitudes, as they spend an extended time sharing resources during early life. Such patterns probably are pervasive and must be considered when seeking to understand species interactions. Improving our knowledge of how temperature and local adaptations affect size-based interactions should enhance our ability to manage and conserve widespread assemblages.},
author = {Garvey, James E and Devries, Dennis R and Wright, Russell a and Miner, Jeffrey G},
doi = {10.1641/0006-3568(2003)053[0141:EAAABL]2.0.CO;2},
isbn = {0006-3568},
issn = {0006-3568},
journal = {BioScience},
keywords = {biotic interaction,community growth,ectotherm,latitude},
number = {2},
pages = {141--150},
pmid = {5735},
title = {{Energetic Adaptations along a Broad Latitudinal Gradient: Implications for Widely Distributed Assemblages}},
url = {http://dx.doi.org/10.1641/0006-3568(2003)053[0141:EAAABL]2.0.CO{\%}5Cn2},
volume = {53},
year = {2003}
}
@article{Caribbean2005,
abstract = {We compared the community-structure of reef-fish over different spatial scales, levels of exposure, and physical complexity in 12 study zones of Bocas del Toro, Panama. Two hundred and eightyeight visual censuses were conducted on 48 benthic transects from April to September 2002. Substrate coverage and surface complexity was also recorded. We found 128 fish species in 38 families with increasing species richness from sheltered to expose and from low-complexity to intermediate and high-complexity zones. Only 7{\%} of the species occurred in all zones. Gobies and pomacentrids were most abundant in sheltered areas and labrids at exposed zones. Eleven species showed significant size-segregations between zones, suggesting ontogenic movements, with smaller sizes in low-complexity zones, and larger-sizes in intermediate to high complexity areas. Species-richness and diversity are high in three of the four exposed zones and in the main areas of massive-coral reefs and significantly correlate with certain types of complex substrates. Highly mobile fish were more abundant in exposed rocky zones while sedentary fish were more abundant in sheltered massive and foliaceous corals zones. Towards the most exposed areas, the number of mobile invertebrate-feeding fish species greatly increased, while territorial herbivores increased in sheltered zones. Roving herbivores (scarids and acanthurids) showed lower frequency than territorial herbivores in all zones. Demersal zooplankton feeders were common in sheltered areas and oceanic planktivores in exposed areas. Omnivores were more abundant in zones of rubble and sand. Carnivores were less frequent, but contribute to the majority of species. We concluded that the species richness in Bocas del Toro relates to the structural complexity of the substrate rather than substrate type. While some species change their preferred habitat during ontogeny, general species diversity increased with habitat complexity. This increase was more pronounced in exposed zones. It seem that water current strength and waves, which select for swimming capacity, play an important but still little understood role in the organization of fish assemblages in rocky and coral reefs.},
author = {Dominici-Arosemena, Arturo and Wolff, Matthias},
isbn = {0008-6452},
issn = {00086452},
journal = {Caribbean Journal of Science},
keywords = {Distribution,Exposure level,Fish diversity,Fish mobility,Lagoonal system,Mesoamerican Caribbean,Trophic groups},
number = {3},
pages = {613--637},
title = {{Reef fish community structure in Bocas del Toro (Caribbean, Panama): Gradients in habitat complexity and exposure}},
volume = {41},
year = {2005}
}
@article{Gardner2007,
abstract = {A major goal of ecology is to determine the causes of the latitudinal gradient in global distribution of species richness. Current evidence points to either energy availability or habitat heterogeneity as the most likely environmental drivers in terrestrial systems, but their relative importance is controversial in the absence of analyses of global (rather than continental or regional) extent. Here we use data on the global distribution of extant continental and continental island bird species to test the explanatory power of energy availability and habitat heterogeneity while simultaneously addressing issues of spatial resolution, spatial autocorrelation, geometric constraints upon species' range dynamics, and the impact of human populations and historical glacial ice-cover. At the finest resolution (18), topographical variability and temperature are identified as the most important global predictors of avian species richness in multi- predictor models. Topographical variability is most important in single-predictor models, followed by productive energy. Adjusting for null expectations based on geometric constraints on species richness improves overall model fit but has negligible impact on tests of environmental predictors. Conclusions concerning the relative importance of environmental predictors of species richness cannot be extrapolated from one biogeographic realm to others or the globe. Rather a global perspective confirms the primary importance of mountain ranges in high-energy areas. Keywords:},
author = {Davies, R. G and Orme, C. D. L and Storch, D. and Olson, V. a and Thomas, G. H and Ross, S. G and Ding, T.-S. and Rasmussen, P. C and Bennett, P. M and Owens, I. P.F and Blackburn, T. M and Gaston, K. J},
doi = {10.1098/rspb.2006.0061},
isbn = {0962-8452},
issn = {0962-8452},
journal = {Proceedings of the Royal Society B: Biological Sciences},
keywords = {geometric constraints,global biodiversity,habitat heterogeneity,species richness,species-energy theory,topography},
month = {jan},
number = {1614},
pages = {1189--1197},
pmid = {17035169},
title = {{Topography, energy and the global distribution of bird species richness}},
url = {http://rspb.royalsocietypublishing.org/cgi/doi/10.1098/rspb.2006.0061},
volume = {274},
year = {2007}
}
@article{Buckley2010,
abstract = {Aim We investigated patterns of species richness and composition of the aquatic food web found in the liquid-filled leaves of the North American purple pitcher plant, Sarracenia purpurea (Sarraceniaceae), from local to continental scales. Location We sampled 20 pitcher-plant communities at each of 39 sites spanning the geographic range of S. purpurea– from northern Florida to Newfoundland and westward to eastern British Columbia. Methods Environmental predictors of variation in species composition and species richness were measured at two different spatial scales: among pitchers within sites and among sites. Hierarchical Bayesian models were used to examine correlates and similarities of species richness and abundance within and among sites. Results Ninety-two taxa of arthropods, protozoa and bacteria were identified in the 780 pitcher samples. The variation in the species composition of this multi-trophic level community across the broad geographic range of the host plant was lower than the variation among pitchers within host-plant populations. Variation among food webs in richness and composition was related to climate, pore-water chemistry, pitcher-plant morphology and leaf age. Variation in the abundance of the five most common invertebrates was also strongly related to pitcher morphology and site-specific climatic and other environmental variables. Main conclusions The surprising result that these communities are more variable within their host-plant populations than across North America suggests that the food web in S. purpurea leaves consists of two groups of species: (1) a core group of mostly obligate pitcher-plant residents that have evolved strong requirements for the host plant and that co-occur consistently across North America, and (2) a larger set of relatively uncommon, generalist taxa that co-occur patchily. [ABSTRACT FROM AUTHOR] Copyright of Global Ecology {\&} Biogeography is the property of Wiley-Blackwell and its content may not be copied or emailed to multiple sites or posted to a listserv without the copyright holder's express written permission. However, users may print, download, or email articles for individual use. This abstract may be abridged. No warranty is given about the accuracy of the copy. Users should refer to the original published version of the material for the full abstract. (Copyright applies to all Abstracts.)},
author = {Buckley, Hannah L. and Miller, Thomas E. and Ellison, Aaron M. and Gotelli, Nicholas J.},
doi = {10.1111/j.1466-8238.2010.00554.x},
isbn = {1466822X},
issn = {1466822X},
journal = {Global Ecology and Biogeography},
keywords = {Food web,Hierarchical Bayesian modelling,Latitudinal gradients,North America,Sarracenia purpurea,Species composition,Species richness},
month = {jun},
number = {5},
pages = {711--723},
pmid = {52670327},
title = {{Local- to continental-scale variation in the richness and composition of an aquatic food web}},
url = {http://doi.wiley.com/10.1111/j.1466-8238.2010.00554.x},
volume = {19},
year = {2010}
}
@article{Briand1983,
abstract = {environmental control of 40 real food webs},
author = {Briand, F.},
doi = {10.2307/1937073},
isbn = {00129658},
issn = {00129658},
journal = {Ecology},
keywords = {connectance,environmental variability,food web structure,habitat type},
number = {2},
pages = {253--263},
pmid = {155},
title = {{Environmental control of food web structure.}},
url = {http://www.jstor.org/stable/10.2307/1937073},
volume = {64},
year = {1983}
}
@article{Beckerman2006,
abstract = {Food webs, the networks of feeding links between species, are central to our understanding of ecosystem structure, stability, and function. One of the key aspects of food web structure is complexity, or connectance, the number of links expressed as a proportion of the total possible number of links. Connectance (complexity) is linked to the stability of webs and is a key parameter in recent models of other aspects of web structure. However, there is still no fundamental biological explanation for connectance in food webs. Here, we propose that constraints on diet breadth, driven by optimal foraging, provide such an explanation. We show that a simple diet breadth model predicts highly constrained values of connectance as an emergent consequence of individual foraging behavior. When combined with features of real food web data, such as taxonomic and trophic aggregation and cumulative sampling of diets, the model predicts well the levels of connectance and scaling of connectance with species richness, seen in real food webs. This result is a previously undescribed synthesis of foraging theory and food web theory, in which network properties emerge from the behavior of individuals and, as such, provides a mechanistic explanation of connectance currently lacking in food web models.},
author = {Beckerman, a P and Petchey, O L and Warren, P H},
doi = {10.1073/pnas.0603039103},
isbn = {0027-8424},
issn = {0027-8424},
journal = {Proceedings of the National Academy of Sciences of the United States of America},
keywords = {adaptation,adaptive behavior,communities,connectance,contingency model,diet breadth,models,network structure,networks,robustness,stability},
month = {sep},
number = {37},
pages = {13745--13749},
pmid = {16954193},
title = {{Foraging biology predicts food web complexity}},
volume = {103},
year = {2006}
}
@article{August2010,
abstract = {The relationship between mammal community structure and vertical variation in hab- itat physiognomy (complexity) and horizontal variation in habitat form (heterogeneity) was examined on five study areas in the llanos of Venezuela. Data on the small mammals ({\textless} 1 kg) of the study sites were obtained through a mark-recapture trapping program of {\textgreater}38 000 trap nights from 1976-1978. Data on the distribution of large, nonvolant mammals were obtained during 24 mo of field observation. Measures of habitat complexity and habitat heterogeneity were derived using principal components analysis. There was little association between habitat structure and the richness, diversity, abundance, and biomass of small mammals. Abiotic factors, such as the degree of wet-season flooding, probably play an important role in patterns of small mammal distribution and abundance. The total number of mammal species was positively correlated with habitat complexity but not correlated with habitat heterogeneity. Increasing species richness across the complexity gradient was probably accommo- dated by increasing potential food resources. New species were added to complex communities primarily through guild expansion rather than guild addition.},
author = {August, P. V.},
journal = {Ecology},
keywords = {community structure,diversity,habitat complexity,habitat heterogeneity,richness,species,tropical mammals},
number = {6},
pages = {1495--1507},
title = {{The role of habitat complexity and heterogeneity in structuring tropicalmammal communities}},
url = {http://www.jstor.org/stable/10.2307/1937504},
volume = {64},
year = {1983}
}
@article{Gotceitas2011,
abstract = {Numerous studies have demonstrated a negative relationship between increasing habitat complexity and predator foraging success. Results from many of these studies suggest a non-linear relationship, and it has been hypothesised that some "threshold level" of complexity is required before foraging success is reduced significantly. We examined this hypothesis using largemouth bass (Micropterus salmoides) foraging on juvenile bluegill sunfish (Lepomis macrochirus) in various densities of artificial vegetation. Largemouth foraging success differed significantly among the densities of vegetation tested. Regression analysis revealed a non-linear relationship between increasing plant stem density and predator foraging success. Logistic analysis demonstrated a significant fit of our data to a logistic model, from which was calculated the threshold level of plant stem density necessary to reduce predator foraging success. Studies with various prey species have shown selection by prey for more complex habitats as a refuge from predation. In this study, we also examined the effects of increasing habitat complexity (i.e. plant stem density) on choice of habitat by juvenile bluegills while avoiding predation. Plant stem density significantly effected choice of habitat as a refuge. The relationship between increasing habitat complexity and prey choice of habitat was found to be positive and non-linear. As with predator foraging success, logistic analysis demonstrated a significant fit of our data to a logistic model. Using this model we calculated the "threshold" level of habitat complexity required before prey select a habitat as a refuge. This density of vegetation proved to be considerably higher than that necessary to significantly reduce predator foraging success, indicating that bluegill select habitats safe from predation. Implications of these results and various factors which may affect the relationships described are discussed.},
author = {Gotceitas, Vytenis and Colgan, Patrick},
doi = {10.1007/BF00380145},
isbn = {00298549},
issn = {00298549},
journal = {Oecologia},
keywords = {Bluegill,Foraging,Habitat Complexity,Predation,Thresholds},
number = {2},
pages = {158--166},
title = {{Predator foraging success and habitat complexity: quantitative test of the threshold hypothesis}},
volume = {80},
year = {1989}
}
@article{Sankamethawee2011,
abstract = {Fleshy-fruited plants in tropical forests largely rely on vertebrate frugivores to disperse their seeds. Although this plant-animal interaction is typically considered a diffuse mutualism, it is fundamental as it provides the template on which tropical forest communities are structured. We applied a mutualistic network approach to investigate the relationship between small-fruited fleshy plant species and the fruit-eating bird community in an intact evergreen forest in northeast Thailand. A minimum of 53 bird species consumed fruits of 136 plant species. Plant-avian frugivore networks were highly asymmetrical, with observed networks filling 30{\%} of all potential links. Whereas some of the missing links in the present study might be due to undersampling, forbidden links can be attributed to size constraints, accessibility and phenological uncoupling, and although the majority of missing links were unknown (58.2{\%}), many were probably due to a given bird species being either rare or only a very occasional fruit eater. The most common frugivores were bulbuls, barbets and fairy-bluebirds, which were responsible for the majority of fruit removal from small fleshy fruited species in our system. Migratory birds seemed to be a minor component of the plant-frugivore networks, accounting for only 3{\%} of feeding visits to fruiting trees; they filled 2{\%} of the overall potential networks. The majority of interactions were generalized unspecific; however, Saurauia roxburghii Wall. appeared to be dependent on flowerpeckers for dispersal, while Thick-billed Pigeons were only seen to eat figs.},
author = {Sankamethawee, Wangworn and Pierce, Andrew J. and Gale, George a. and Hardesty, Britta Denise},
doi = {10.1111/j.1749-4877.2011.00244.x},
isbn = {1749-4877},
issn = {17494877},
journal = {Integrative Zoology},
keywords = {Avian frugivore,Nestedness,Plant-frugivore networks,Seed dispersal,Tropical forest},
month = {sep},
number = {3},
pages = {195--212},
pmid = {21910839},
title = {{Plant-frugivore interactions in an intact tropical forest in north-east Thailand}},
url = {http://www.ncbi.nlm.nih.gov/pubmed/21910839},
volume = {6},
year = {2011}
}
@article{Mello2011,
abstract = {In networks of plant-animal mutualisms, different animal groups interact preferentially with different plants, thus forming distinct modules responsible for different parts of the service. However, what we currently know about seed dispersal networks is based only on birds. Therefore, we wished to fill this gap by studying bat-fruit networks and testing how they differ from bird-fruit networks. As dietary overlap of Neotropical bats and birds is low, they should form distinct mutualistic modules within local networks. Furthermore, since frugivory evolved only once among Neotropical bats, but several times independently among Neotropical birds, greater dietary overlap is expected among bats, and thus connectance and nestedness should be higher in bat-fruit networks. If bat-fruit networks have higher nestedness and connectance, they should be more robust to extinctions. We analyzed 1 mixed network of both bats and birds and 20 networks that consisted exclusively of either bats (11) or birds (9). As expected, the structure of the mixed network was both modular (M = 0.45) and nested (NODF = 0.31); one module contained only birds and two only bats. In 20 datasets with only one disperser group, bat-fruit networks (NODF = 0.53 ± 0.09, C = 0.30 ± 0.11) were more nested and had a higher connectance than bird-fruit networks (NODF = 0.42 ± 0.07, C = 0.22 ± 0.09). Unexpectedly, robustness to extinction of animal species was higher in bird-fruit networks (R = 0.60 ± 0.13) than in bat-fruit networks (R = 0.54 ± 0.09), and differences were explained mainly by species richness. These findings suggest that a modular structure also occurs in seed dispersal networks, similar to pollination networks. The higher nestedness and connectance observed in bat-fruit networks compared with bird-fruit networks may be explained by the monophyletic evolution of frugivory in Neotropical bats, among which the diets of specialists seem to have evolved from the pool of fruits consumed by generalists.},
author = {Mello, Marco Aurelio Ribeiro and Marquitti, Fl{\'{a}}via Maria Darcie and Guimar{\~{a}}es, Paulo R. and Kalko, Elisabeth Klara Viktoria and Jordano, Pedro and de Aguiar, Marcus Aloizio Martinez},
doi = {10.1007/s00442-011-1984-2},
isbn = {0029-8549},
issn = {00298549},
journal = {Oecologia},
keywords = {Complex networks,Ecosystem services,Food webs,Guilds,Mutualisms},
month = {sep},
number = {1},
pages = {131--140},
pmid = {21479592},
title = {{The modularity of seed dispersal: differences in structure and robustness between bat- and bird-fruit networks}},
url = {http://www.ncbi.nlm.nih.gov/pubmed/21479592},
volume = {167},
year = {2011}
}
@article{Leibold2010,
abstract = {Trophic structure, the partitioning of biomass among trophic levels, is a major characteristic of ecosystems. Most studies of the forces that shape trophic struc-ture emphasize either " bottom-up " or " top-down " regulation of populations and communities. Recent work has shown that these two forces are not mutually exclusive alternatives, but efforts to model their interaction still often yield unre-alistic predictions. We focus on the problems involved with modeling situations in which community composition, including both the number of trophic levels and the species composition within a trophic level, can change. We review the devel-opment of these ideas, emphasizing in particular how compositional change can alter theoretical expectations about the regulation of trophic structure. A compar-ison of studies on the effects of predators and resource productivity in limnetic ecosystems reveals an intriguing disparity between the results of manipulative experiments and those of correlational studies. We suggest that this contrast is a result of the difference in the temporal scales operating in the two types of studies. Ecosystem-level variables may appear to approach an equilibrium in short-term press experiments; however, processes such as invasion and extinction of species will not have time to play out in most such experiments. We found that the responses of ecosystems to short-term experimental treatments involve less change in species composition than is found in natural communities that have diverged in response to local conditions over longer periods. We argue that the results of short-term experiments support the predictions of models in which},
author = {Leibold, Mathew A and Chase, Jonathan M and Shurin, Jonathan B and Downing, Amy L},
doi = {10.1146/annurev.ecolsys.28.1.467},
isbn = {00664162},
issn = {0066-4162},
journal = {Annu. Rev. Ecol. Syst},
keywords = {biomanipulation,compositional change,productivity,top-down vs bottom-up,trophic cascade},
number = {1},
pages = {467--94},
pmid = {678},
title = {{Species Turnover and the Regulation of Trophic Structure}},
volume = {28},
year = {1997}
}
@article{Sole2002,
abstract = {Biotic recoveries following mass extinctions are characterized by a process in which whole ecologies are reconstructed from low-diversity systems, often characterized by opportunistic groups. The recovery process provides an unexpected window to ecosystem dynamics. In many aspects, recovery is very similar to ecological succession, but important differences are also apparently linked to the innovative patterns of niche construction observed in the fossil record. In this paper, we analyse the similarities and differences between ecological succession and evolutionary recovery to provide a preliminary ecological theory of recoveries. A simple evolutionary model with three trophic levels is presented, and its properties (closely resembling those observed in the fossil record) are compared with characteristic patterns of ecological response to disturbances in continuous models of three-level ecosystems.},
author = {Sol{\'{e}}, Ricard V and Montoya, Jos{\'{e}} M and Erwin, Douglas H},
doi = {10.1098/rstb.2001.0987},
isbn = {0962-8436},
issn = {0962-8436},
journal = {Philosophical transactions of the Royal Society of London. Series B, Biological sciences},
keywords = {assembly dynamics,biodiversity recoveries,food web structure,macroevolutionary model,mass extinctions},
month = {may},
number = {1421},
pages = {697--707},
pmid = {12079530},
title = {{Recovery after mass extinction: evolutionary assembly in large-scale biosphere dynamics.}},
url = {http://www.pubmedcentral.nih.gov/articlerender.fcgi?artid=1692978{\&}tool=pmcentrez{\&}rendertype=abstract},
volume = {357},
year = {2002}
}
@article{Pimm2010,
author = {Pimm, Stuart L and Lawton, John H},
journal = {Journal of Animal Ecology},
number = {3},
pages = {879--898},
title = {{Are food webs divided into compartments?}},
url = {http://www.jstor.org/stable/4233},
volume = {49},
year = {1980}
}
@article{Duffy2010,
abstract = {Proposed links between biodiversity and ecosystem processes have generated intense interest and controversy in recent years. With few exceptions, however, empirical studies have focused on grassland plants and laboratory aquatic microbial systems, whereas there has been little attention to how changing animal diversity may influence ecosystem processes. Meanwhile, a separate research tradition has demonstrated strong top-down forcing in many systems, but has considered the role of diversity in these processes only tangentially. Integration of these research directions is necessary for more complete understanding in both areas. Several considerations suggest that changing diversity in multi-level food webs can have important ecosystem effects that can be qualitatively different than those mediated by plants. First, extinctions tend to be biased by trophic level: higher-level consumers are less diverse, less abundant, and under stronger anthropogenic pressure on average than wild plants, and thus face greater risk of extinction. Second, unlike plants, consumers often have impacts on ecosystems disproportionate to their abundance. Thus, an early consequence of declining diversity will often be skewed trophic structure, potentially reducing top-down influence. Third, where predators remain abundant, declining diversity at lower trophic levels may change effectiveness of predation and penetrance of trophic cascades by reducing trait diversity and the potential for compensation among species within a level. The mostly indirect evidence available provides some support for this prediction. Yet effects of changing animal diversity on functional processes have rarely been tested experimentally. Evaluating impacts of biodiversity loss on ecosystem function requires expanding the scope of current experimental research to multi-level food webs. A central challenge to doing so, and to evaluating the importance of trophic cascades specifically, is understanding the distribution of interaction strengths within natural communities and how they change with community composition. Although topology of most real food webs is extremely complex, it is not at all clear how much of this complexity translates to strong dynamic linkages that influence aggregate biomass and community composition. Finally, there is a need for more detailed data on patterns of species loss from real ecosystems (community “disassembly” rules).},
author = {Duffy, J Emmett},
doi = {10.1034/j.1600-0706.2002.990201.x},
isbn = {1600-0706},
issn = {0030-1299},
journal = {Oikos},
number = {2},
pages = {201--219},
pmid = {21512726},
title = {{Biodiversity and ecosystem function: the consumer connection}},
url = {http://dx.doi.org/10.1034/j.1600-0706.2002.990201.x},
volume = {99},
year = {2002}
}
@article{Joppa2010,
abstract = {Questions: Are interaction patterns in species interaction networks different from what one expects by chance alone? In particular, are these networks nested - a pattern where resources taken by more specialized consumers form a proper subset of those taken by more generalized consumers? Organisms: Fifty-nine and 42 networks of mutualistic and host-parasitoid interactions, respectively. Analytical methods: For each network, the observed degree of nestedness is compared with the distribution of nestedness values derived from a collection of 1000 random networks. Those networks with nestedness values lower than 95{\%} of all random values are considered 'unusually nested'. The analysis considers two different metrics of nestedness and five different network randomization algorithms, each of which differs in the ecological assumptions imposed. Results: Most ecological networks are unusually nested when compared with loosely constrained random networks. Comparisons with highly constrained networks temper these findings, but we still report a significant preponderance of nested networks (typically those with the most species). Conclusions: Bascompte et al. (2003) previously showed most observed mutualistic networks to be unusually nested. Later work using more ecologically realistic randomization algorithms cast doubt on those results. Across the largest set of species interactions considered to date, we conclude that an unexpectedly large number of interaction networks are patterned in a non-random manner. ¬{\textcopyright} 2010 Stuart L. Pimm.},
author = {Joppa, Lucas N. and Montoya, Jos{\'{e}} M. and Sol{\'{e}}, Richard and Sanderson, Jim and Pimm, Stuart L.},
isbn = {1522-0613},
issn = {15220613},
journal = {Evolutionary Ecology Research},
keywords = {Ecological network,Food web,Host-parasitoid,Mutualism,Nestedness,Null model},
number = {1},
pages = {35--46},
title = {{On nestedness in ecological networks}},
volume = {12},
year = {2010}
}
@article{Olesen2002,
abstract = {ecent reviews of plant-pollinator mutualistic networks showed that gen- eralization is a common pattern in this type of interaction. Here we examine the ecological correlates of generalization patterns in plant-pollinator networks, especially how interaction patterns covary with latitude, elevation, and insularity. We review the few published anal- yses of whole networks and include unpublished material, analyzing 29 complete plant- pollinator networks that encompass arctic, alpine, temperate, Mediterranean, and subtrop- ical-tropical areas. The number of interactions observed (I) was a linear function of network size (M) the maximum number of interactions: lnI = 0.575 + 0.61 lnM; R2 = 0.946. The connectance (C), the fraction of observed interactions relative to the total possible, decreased exponentially with species richness, the sum of animal and plant species in each community (A + P): C = 13.83 exp[-0.003(A + P)]. After controlling for species richness, the residual connectance was significantly lower in highland ({\textgreater}1500 m elevation) than in lowland networks and differed marginally among biogeographic regions, with both alpine and trop- ical networks showing a trend for lower residual connectance. The two Mediterranean networks showed the highest residual connectance. After correcting for variation in network size, plant species were shown to be more generalized at higher latitude and lowland habitats, but showed increased specialization on islands. Oceanic island networks showed an im- poverishment of potential animal pollinators (lower ratio of animal to plant species, A: P. compared to mainland networks) associated with this trend of increased specialization. Plants, but not their flower-visiting animals, supported the often-repeated statements about higher specificity in the tropics than at higher latitudes. The pattern of interaction build- up as diversity increases in pollination networks does not differ appreciably from other mutualisms, such as plant-seed disperser networks or more complex food webs.},
archivePrefix = {arXiv},
arxivId = {arXiv:1011.1669v3},
author = {Olesen, Jens M. and Jordano, Pedro},
doi = {10.1890/0012-9658(2002)083[2416:GPIPPM]2.0.CO;2},
eprint = {arXiv:1011.1669v3},
isbn = {0012-9658},
issn = {00129658},
journal = {Ecology},
keywords = {Food web,Geographic variation,Insects,Interaction-web connectance,Mutualism,Networks,Plant-animal interaction,Pollination,Specialization},
number = {9},
pages = {2416--2424},
pmid = {178153900008},
title = {{Geographic patterns in plant-pollinator mutualistic networks}},
url = {http://www.esajournals.org/doi/pdf/10.1890/0012-9658(2002)083{\%}5B2416:GPIPPM{\%}5D2.0.CO{\%}3B2},
volume = {83},
year = {2002}
}
@article{Fortuna2010,
abstract = {1. Understanding the structure of ecological networks is a crucial task for interpreting community and ecosystem responses to global change. 2. Despite the recent interest in this subject, almost all studies have focused exclusively on one specific network property. The question remains as to what extent different network properties are related and how understanding this relationship can advance our comprehension of the mechanisms behind these patterns. 3. Here, we analysed the relationship between nestedness and modularity, two frequently studied network properties, for a large data set of 95 ecological communities including both plant-animal mutualistic and host-parasite networks. 4. We found that the correlation between nestedness and modularity for a population of random matrices generated from the real communities decreases significantly in magnitude and sign with increasing connectance independent of the network type. At low connectivities, networks that are highly nested also tend to be highly modular; the reverse happens at high connectivities. 5. The above result is qualitatively robust when different null models are used to infer network structure, but, at a finer scale, quantitative differences exist. We observed an important interaction between the network structure pattern and the null model used to detect it. 6. A better understanding of the relationship between nestedness and modularity is important given their potential implications on the dynamics and stability of ecological communities.},
author = {Fortuna, Miguel A. and Stouffer, Daniel B. and Olesen, Jens M. and Jordano, Pedro and Mouillot, David and Krasnov, Boris R. and Poulin, Robert and Bascompte, Jordi},
doi = {10.1111/j.1365-2656.2010.01688.x},
isbn = {0021-8790},
issn = {00218790},
journal = {Journal of Animal Ecology},
keywords = {Complex networks,Food webs,Host-parasite,Mutualistic networks,Plant-pollinator,Plant-seed disperser},
month = {jul},
number = {4},
pages = {811--817},
pmid = {20374411},
title = {{Nestedness versus modularity in ecological networks: two sides of the same coin?}},
url = {http://www.ncbi.nlm.nih.gov/pubmed/20374411},
volume = {79},
year = {2010}
}
@article{Nielsen2007,
abstract = {1. Ecological networks have been shown to display a nested structure. To be nested, a network must consist of a core group of generalists all interacting with each other, and with extreme specialists interacting only with generalist species. 2. Studies on ecological networks are especially prone to sampling effects, as they involve entire species assemblages. However, we know of no study addressing to what extent nestedness depends on sampling effort, despite the numerous studies discussing the ecological and evolutionary implications of nested networks. 3. Here we manipulate sampling effort in time and space and show that nestedness is less sensitive to sampling effort than number of species and links within the network. 4. That a structural property of an ecological network appears less prone to sampling bias is encouraging for other studies of ecological networks. This is because it indicates that the sensitivity of ecological networks properties to effects of sampling effort might be smaller than previously expected.},
author = {Nielsen, Anders and Bascompte, Jordi},
doi = {10.1111/j.1365-2745.2007.01271.x},
isbn = {0022-0477},
issn = {00220477},
journal = {Journal of Ecology},
keywords = {Ecological networks,Food webs,Mutualistic interactions,Nestedness,Network structure,Pollination,Sampling effort},
month = {sep},
number = {5},
pages = {1134--1141},
pmid = {249166700022},
title = {{Ecological networks, nestedness and sampling effort}},
url = {http://www.blackwell-synergy.com/doi/abs/10.1111/j.1365-2745.2007.01271.x},
volume = {95},
year = {2007}
}
@article{Pires2011a,
abstract = {1. Much of the current understanding of ecological systems is based on theory that does not explicitly take into account individual variation within natural populations. However, individuals may show substantial variation in resource use. This variation in turn may be translated into topological properties of networks that depict interactions among individuals and the food resources they consume (individual-resource networks). 2. Different models derived from optimal diet theory (ODT) predict highly distinct patterns of trophic interactions at the individual level that should translate into distinct network topologies. As a consequence, individual-resource networks can be useful tools in revealing the incidence of different patterns of resource use by individuals and suggesting their mechanistic basis. 3. In the present study, using data from several dietary studies, we assembled individual-resource networks of 10 vertebrate species, previously reported to show interindividual diet variation, and used a network-based approach to investigate their structure. 4. We found significant nestedness, but no modularity, in all empirical networks, indicating that (i) these populations are composed of both opportunistic and selective individuals and (ii) the diets of the latter are ordered as predictable subsets of the diets of the more opportunistic individuals. 5. Nested patterns are a common feature of species networks, and our results extend its generality to trophic interactions at the individual level. This pattern is consistent with a recently proposed ODT model, in which individuals show similar rank preferences but differ in their acceptance rate for alternative resources. Our findings therefore suggest a common mechanism underlying interindividual variation in resource use in disparate taxa.},
author = {Pires, M. M. and Guimar{\~{a}}es, P. R. and Ara{\'{u}}jo, M. S. and Giaretta, a. a. and Costa, J. C L and dos Reis, S. F.},
doi = {10.1111/j.1365-2656.2011.01818.x},
file = {:Users/alyssacirtwill/Documents/Papers/Pires et al.{\_}2011{\_}Journal of Animal Ecology.pdf:pdf},
isbn = {1365-2656},
issn = {00218790},
journal = {Journal of Animal Ecology},
keywords = {Complex networks,Interindividual variation,Modularity,Nestedness,Optimal diet theory},
month = {jul},
number = {4},
pages = {896--903},
pmid = {21644976},
title = {{The nested assembly of individual-resource networks}},
url = {http://www.ncbi.nlm.nih.gov/pubmed/21644976},
volume = {80},
year = {2011}
}
@article{Mcnab2012,
author = {Mcnab, K},
number = {5},
pages = {845--854},
title = {{On the Ecological Significance of Bergmann ' s Rule Author ( s ): Brian K . McNab Reviewed work ( s ): Published by : Ecological Society of America Stable URL : http://www.jstor.org/stable/1936032 . ON T ' HE ECOLOGICAL SIGNIFICANCE OF BERGMANN ' S RULE '}},
volume = {52},
year = {2012}
}
@article{Martinez1991,
abstract = {A detailed and relatively evenly resolved food web of Little Rock Lake, Wisconsin, was constructed to evaluate the sensitivity of food-web patterns to the level of detail (degree of resolution) in food-web data. This study presents definitions (e.g., ecosystem food webs) and methods for constructing and reducing the resolution of food webs to provide relatively pragmatic and rigorous touchstones for consistency in future food-web studies. This analysis suggests that food-web patterns such as the scale-invariant links-per-species ratio, short chain lengths, and limited number of trophic levels are constrained by the resolution of food-web data rather than by ecological factors. Patterns less sensitive to changes in resolution such as directed connectance (the proportion of observed directed links to all possible directed links) may be robust food-web attributes. The food web of Little Rock Lake appears to be the first highly and evenly resolved food web of a large natural ecosystem originally documented for the purpose of examining quantitative food-web patterns. This ecosystem food web contains roughly twice as many species as the largest web to date. It also may provide the most credible portrait available of the detailed trophic structure of a whole ecosystem. The 93-trophic-species web of Little Rock Lake differs from previously published trophic-species webs by having more links per species (L/S = 11), longer chain lengths (average: greater-than-or-equal-to 10, maximum: greater-than-to-equal-to 16), species at higher trophic levels (maximum: = 12), higher fractions of intermediate species, and smaller fractions of top species and links to top species. The sensitivity of quantitative food-web patterns to changes in resolution was examined in several series of trophically aggregated Little Rock Lake webs. Each of the series starts with a highly and relatively evenly resolved web with 182 consumer, producer, and decomposer taxa and ends with low-resolution webs with 9 aggregates of taxa. Taxa were aggregated based on the proportion of predators and prey shared by the taxa. Different series of webs were generated using different criteria for linking aggregates to evaluate the sensitivity of food-web patterns to linkage criteria. The sensitivity analysis revealed that several, but not all, quantitative food-web patterns are very sensitive to systematic aggregation of the web. Sensitive patterns include number of links per species, linkage complexity, the distributions of chain lengths and species among trophic levels, and the proportions of top species and links to top species. Less-sensitive patterns include connectance, the ratio of predators to prey, the proportions of intermediate and basal species, and the proportions of links that are between intermediate and basal species. Directed connectance is the only pattern examined that is both very robust to trophic aggregation and generally comparable to other community webs. Quantitative food-web patterns in published community webs are generally similar to highly aggregated Little Rock Lake webs (versions with 9-40 aggregates). These findings suggest that previously described community food webs are severely aggregated versions of more elaborate webs similar to that of Little Rock Lake.},
author = {Martinez, Neo D.},
doi = {10.2307/2937047},
isbn = {0012-9615},
issn = {00129615},
journal = {Ecological Monographs},
keywords = {Aggregation,Community food web,Connectance,Ecosystem food web,Food chains,Food webs,Linkage criteria,Little Rock Lake,Quantitative food-web patterns,Resolution,Trophic levels,Wisconsin},
number = {4},
pages = {367--392},
pmid = {2671497},
title = {{Artifacts or attributes? Effects of resolution on the Little Rock Lake food web}},
url = {http://www.jstor.org/stable/10.2307/2937047},
volume = {61},
year = {1991}
}
@article{V2012,
abstract = {The structure of ecological interaction networks is often interpreted as a product of meaningful ecological and evolutionary mechanisms that shape the degree of specialization in community associations. However, here we show that both unweighted network metrics (connectance, nestedness, and degree distribution) and weighted network metrics (interaction evenness, interaction strength asymmetry) are strongly constrained and biased by the number of observations. Rarely observed species are inevitably regarded as "specialists," irrespective of their actual associations, leading to biased estimates of specialization. Consequently, a skewed distribution of species observation records (such as the lognormal), combined with a relatively low sampling density typical for ecological data, already generates a "nested" and poorly "connected" network with "asymmetric interaction strengths" when interactions are neutral. This is confirmed by null model simulations of bipartite networks, assuming that partners associate randomly in the absence of any specialization and any variation in the correspondence of biological traits between associated species (trait matching). Variation in the skewness of the frequency distribution fundamentally changes the outcome of network metrics. Therefore, interpretation of network metrics in terms of fundamental specialization and trait matching requires an appropriate control for such severe constraints imposed by information deficits. When using an alternative approach that controls for these effects, most natural networks of mutualistic or antagonistic systems show a significantly higher degree of reciprocal specialization (exclusiveness) than expected under neutral conditions. A higher exclusiveness is coherent with a tighter coevolution and suggests a lower ecological redundancy than implied by nested networks.},
author = {Bl{\"{u}}thgen, Nico and Fr{\"{u}}nd, Jochen and Vazquez, Diego P. and Menzel, Florian},
doi = {10.1890/07-2121.1},
isbn = {0012-9658},
issn = {00129658},
journal = {Ecology},
keywords = {Abundance distribution,Biological traits,Connectance,Degree distribution,Ecological networks,Interaction diversity,Interaction strength,Nestedness,Null models,Specialization},
number = {12},
pages = {3387--3399},
pmid = {19137945},
title = {{What do interaction network metrics tell us about specialization and biological traits?}},
url = {http://www.esajournals.org/doi/abs/10.1890/07-2121.1},
volume = {89},
year = {2008}
}
@article{Oe-cho2010,
abstract = {Nested structure, in which specialists interact with subsets of species with which generalists interact, has been repeatedly found in networks of mutualistic interactions and thus is considered a general feature of mutualistic communities. However, it is uncertain how exclusive nested structure is for mutualistic communities since few studies have evaluated nestedness in other types of networks. Here, we show that 31 published food webs consist of bipartite subwebs that are as highly nested as mutualistic networks, contradicting the hypothesis that antagonistic interactions disfavor nested structure. Our findings suggest that nested networks may be a common pattern of communities that include resource-consumer interactions. In contrast to the hypothesis that nested structure enhances biodiversity in mutualistic communities, we also suggest that nested food webs increase niche overlap among consumers and thus prevent their coexistence. We discuss potential mechanisms for the emergence of nested structure in food webs and other types of ecological networks.},
author = {Kondoh, Michio and Kato, Satoshi and Sakato, Yoshikuni},
doi = {10.1890/09-2219.1},
isbn = {0012-9658},
issn = {00129658},
journal = {Ecology},
keywords = {Antagonistic network,Bipartite food web,Complementarity hypothesis,Complex network,Food web,Mutualism,Mutualistic network,Nestedness,Null model analysis,Trophic interaction},
number = {11},
pages = {3123--3130},
pmid = {21141173},
title = {{Food webs are built up with nested subwebs}},
url = {http://www.esajournals.org/doi/abs/10.1890/09-2219.1},
volume = {91},
year = {2010}
}
@article{Petchey2010,
abstract = {Few models concern how environmental variables such as temperature affect community structure. Here, we develop a model of how temperature affects food web connectance, a powerful driver of population dynamics and community structure. We use the Arrhenius equation to add temperature dependence of foraging traits to an existing model of food web structure. The model predicts potentially large temperature effects on connectance. Temperature-sensitive food webs exhibit slopes of up to 0.01 units of connectance per 1 degrees C change in temperature. This corresponds to changes in diet breadth of one resource item per 2 degrees C (assuming a food web containing 50 species). Less sensitive food webs exhibit slopes down to 0.0005, which corresponds to about one resource item per 40 degrees C. Relative sizes of the activation energies of attack rate and handling time determine whether warming increases or decreases connectance. Differences in temperature sensitivity are explained by differences between empirical food webs in the body size distributions of organisms. We conclude that models of temperature effects on community structure and dynamics urgently require considerable development, and also more and better empirical data to parameterize and test them.},
author = {Petchey, Owen and Brose, Ulrich and Rall, Bj{\"{o}}rn},
doi = {10.1098/rstb.2010.0011},
isbn = {0962-8436},
issn = {0962-8436 VN  - readcube.com},
journal = {Philosophical transactions of the Royal Society of London. Series B, Biological sciences},
keywords = {activation energy,allometric diet breadth model,body size,foraging,functional response},
month = {jul},
number = {1549},
pages = {2081--2091},
pmid = {20513716},
title = {{Predicting the effects of temperature on food web connectance.}},
url = {http://dx.doi.org/10.1098/rstb.2010.0011{\%}5Cn/Users/erinlarson/Documents/ReadCube Media/Philos Trans R Soc Lond B Biol Sci 2010 Petchey OL.pdf},
volume = {365},
year = {2010}
}
@article{Martinez1993,
abstract = {This analysis of 11 large community food webs shows that food web structure systematically varies as the trophic resolution of food web data is reduced. Mean chain length, the ratio of links per species, and the fractions of intermediate species and links between intermediate species decrease as the trophic-species webs are aggregated to half their size. Concurrently, the fractions of top species, basal species, and links between top and basal species increase. Directed connectance and the predator/prey ratio appear to be relatively robust to reducing the trophic resolution of food web data. Significant effects of reducing the taxonomic resolution of food webs are also demonstrated. These findings are inconsistent with prominent claims that most properties are robust to varying the resolution of food web data. Problems responsible for this inconsistency are identified.},
author = {Martinez, N D},
doi = {10.1126/science.260.5105.242},
isbn = {0036-8075},
issn = {0036-8075},
journal = {Science},
month = {apr},
number = {5105},
pages = {242--243},
pmid = {17807184},
title = {{Effect of scale on food web structure.}},
url = {http://www.ncbi.nlm.nih.gov/pubmed/17807184},
volume = {260},
year = {1993}
}
@article{Field2009,
abstract = {Aim We surveyed the empirical literature to determine how well six diversity hypotheses account for spatial patterns in species richness across varying scales of grain and extent. Location Worldwide. Methods We identified 393 analyses (‘cases') in 297 publications meeting our criteria. These criteria included the requirement that more than one diversity hypothesis was tested for its relationship with species richness. We grouped variables representing the hypotheses into the following ‘correlate types': climate/productivity, environmental heterogeneity, edaphics/nutrients, area, biotic interactions and dispersal/history (colonization limitation or other historical or evolutionary effect). For each case we determined the ‘primary' variable: the one most strongly correlated with taxon richness. We defined ‘primacy' as the proportion of cases in which each correlate type was represented by the primary variable, relative to the number of times it was studied. We tested for differences in both primacy and mean coefficient of determination of the primary variable between the hypotheses and between categories of five grouping variables: grain, extent, taxon (animal vs. plant), habitat medium (land vs. water) and insularity (insular vs. connected). Results Climate/productivity had the highest overall primacy, and environmental heterogeneity and dispersal/history had the lowest. Primacy of climate/productivity was much higher in large-grain and large-extent studies than at smaller scales. It was also higher on land than in water, and much higher in connected systems than in insular ones. For other hypotheses, differences were less pronounced. Throughout, studies on plants and animals showed similar patterns. Coefficients of determination of the primary variables differed little between hypotheses and across the grouping variables, the strongest effects being low means in the smallest grain class and for edaphics/nutrients variables, and a higher mean for water than for land in connected systems but vice versa in insular systems. We highlight areas of data deficiency. Main conclusions Our results support the notion that climate and productivity play an important role in determining species richness at large scales, particularly for non-insular, terrestrial habitats. At smaller extents and grain sizes, the primacy of the different types of correlates appears to differ little from null expectation. In our analysis, dispersal/history is rarely the best correlate of species richness, but this may reflect the difficulty of incorporating historical factors into regression models, and the collinearity between past and current climates. Our findings are consistent with the view that climate determines the capacity for species richness. However, its influence is less evident at smaller spatial scales, probably because (1) studies small in extent tend to sample little climatic range, and (2) at large grains some other influences on richness tend to vary mainly within the sampling unit.},
author = {Field, Richard and Hawkins, Bradford A. and Cornell, Howard V. and Currie, David J. and Diniz-Filho, J. Alexandre F and Gu{\'{e}}gan, Jean Fran{\c{c}}ois and Kaufman, Dawn M. and Kerr, Jeremy T. and Mittelbach, Gary G. and Oberdorff, Thierry and O'Brien, Eileen M. and Turner, John R G},
doi = {10.1111/j.1365-2699.2008.01963.x},
isbn = {1365-2699},
issn = {03050270},
journal = {Journal of Biogeography},
keywords = {Area,Climatic gradient,Dispersal,Diversity gradient,Extent,Grain,History,Islands,Latitudinal gradient,Productivity},
month = {jan},
number = {1},
pages = {132--147},
pmid = {19323025},
title = {{Spatial species-richness gradients across scales: a meta-analysis}},
url = {http://doi.wiley.com/10.1111/j.1365-2699.2008.01963.x},
volume = {36},
year = {2008}
}
@article{Ulanowicz1979,
author = {Ulanowicz, Robert E. and Kemp, W M.},
isbn = {0003-0147},
journal = {The American Naturalist},
number = {6},
pages = {871--883},
title = {{Toward Canonical Trophic Aggregations}},
volume = {114},
year = {1979}
}
@article{Chen2010,
abstract = {Knowing how an increase in the resource base of a food web produces effects that propagate through the web is central to developing a clearer understanding of food- web structure and dynamics. In a detritus-based terrestrial food web, we measured the responses of predaceous arthropods to increases in prey arthropods that occurred in response to experimentally enhancing the web's resource base. Open 2 X 5 m plots on the floor of a deciduous forest were randomly assigned to either a Food Enhancement or Control treat- ment. We supplemented the resource base of the arthropod community of the leaf litter layer for 3.5 mo by periodically adding chopped mushrooms, potatoes, and instant fruit fly medium to the Food Enhancement plots. Major taxa of detritivores and fungivores increased in response to added food. Densities of springtails (Collembola) were on average 3 X higher in the Food Enhancement than Control plots. Numbers of adult fungus gnats (Diptera: Sciaridae and Mycetophilidae) did not differ significantly between treatments after 6 wk but were {\textgreater}2X higher in Food Enhancement plots at the end of the experiment. Total Diptera were twice as abundant in Food Enhancement plots on both census dates. Arthropod groups that include a range of feeding strategies also increased. Mites (Ac- arina), which include detritivores, fungivores, and predators, were twice as abundant in the experimental treatment. Staphylinid and carabid beetles (Coleoptera), which are primarily predaceous but include omnivorous species, were several times more numerous in the Food Enhancement plots. Effects of increasing the resource base propagated through the food web, leading to higher densities of the major strictly predaceous arthropod taxa. Centipedes (Chilopoda), pseudoscorpions (Pseudoscorpionida), and spiders (Araneae) were -2X as abundant in the Food Enhancement treatment. Thus, our experiment uncovered substantial bottom-up lim- itation in this detritus-based food web, expressed as responses by predaceous arthropods at least two trophic links removed from the experimentally elevated resource. Key},
author = {Chen, Benrong and Wise, David H.},
doi = {10.1890/0012-9658(1999)080[0761:BULOPA]2.0.CO;2},
isbn = {0012-9658},
issn = {00129658},
journal = {Ecology},
keywords = {Bottom-up limitation,Collembola,Detrital food web,Detritivores,Forest floor,Fungivores,Fungus gnats,Predaceous arthropods,Schizocosa,Spiders,Terrestrial arthropod community},
number = {3},
pages = {761--772},
pmid = {4524394},
title = {{Bottom-up limitation of predaceous arthropods in a detritus-based terrestrial food web}},
volume = {80},
year = {1999}
}
@article{Rico-Gray2011,
abstract = {Despite recognition of key biotic processes in shaping the structure of biological communities, few empirical studies have explored the influences of abiotic factors on the structural properties of mutualistic networks. We tested whether temperature and precipitation contribute to temporal variation in the nestedness of mutualistic ant-plant networks. While maintaining their nested structure, nestedness increased with mean monthly precipitation and, particularly, with monthly temperature. Moreover, some species changed their role in network structure, shifting from peripheral to core species within the nested network. We could summarize that abiotic factors affect plant species in the vegetation (e.g., phenology), meaning presence/absence of food sources, consequently an increase/decrease of associations with ants, and finally, these variations to fluctuations in nestedness. While biotic factors are certainly important, greater attention needs to be given to abiotic factors as underlying determinants of the structures of ecological networks.},
author = {Rico-Gray, Victor and D{\'{i}}az-Castelazo, Cecilia and Ram{\'{i}}rez-Hern{\'{a}}ndez, Alfredo and Guimar{\~{a}}es, Paulo R. and Holland, J. Nathaniel},
doi = {10.1007/s11829-011-9170-3},
isbn = {1872-8855},
issn = {18728855},
journal = {Arthropod-Plant Interactions},
keywords = {Effect of abiotic factors,Mutualistic networks,M{\'{e}}xico,Veracruz},
month = {dec},
number = {2},
pages = {289--295},
title = {{Abiotic factors shape temporal variation in the structure of an ant-plant network}},
url = {http://www.springerlink.com/index/10.1007/s11829-011-9170-3},
volume = {6},
year = {2012}
}
@article{Sugihara1997,
abstract = {Historically, ecologists have been more inter- ested in organisms feeding at the tops of food chains than in organisms feeding at or near the bottom. The problem of taxonomic and trophic inconsistency within and among described food webs is central to criticisms of contemporary food web research. To study the e.ects of taxonomic and trophic aggregation on food web properties, 38 published food webs, each containing a large fraction of investigator-de{\textcopyright}ned biological species, were systematically aggregated by taxonomy and trophic (functional) group similarity. During each step of taxo- nomic and trophic aggregation, eight food web proper- ties (MIN, MAX, mean chain lengths; the fractions of basal, intermediate and top species; the ratio of all links by the total number of species, L/S; and rigid circuits) were calculated and their departures from the original, unaggregated version were recorded. We found only two properties showing wide systematic departure from initial values after both taxonomic and trophic group aggregation: the fraction of basal species and L/S. One reason for the relative `constancy' of the six other properties was due in part to large numbers of trophi- cally equivalent species (species with identical sets of prey and predators) found in these and other published webs. In the 38 webs, the average number of trophically equivalent species was 45{\%} and ranged from a low of 13{\%} in aquatic webs to a high of 71{\%} in certain ter- restrial systems (i.e., carrion webs). Six of the eight properties (MIN, MAX and mean chain lengths, the fractions of top and basal species, and the L/S ratio) were found to be more sensitive to taxonomic than to trophic aggregation. The relatively smaller variations observed in trophically lumped versions suggest that food web properties more aptly re{\^{i}}ect functional, rather than taxonomic, attributes of real food webs. These {\textcopyright}ndings parallel earlier trophic-based results, and bolster the conclusion that uneven lumping of taxonomic and trophic groups in published food web reports do not modify markedly the scaling behaviour of most of their descriptive properties.},
author = {Sugihara, George and Bersier, Louis F{\'{e}}lix and Schoenly, Kenneth},
doi = {10.1007/s004420050310},
isbn = {0029-8549},
issn = {00298549},
journal = {Oecologia},
keywords = {Community structure,Data resolution,Food web scaling laws,Scale-invariance},
number = {2},
pages = {272--284},
pmid = {105},
title = {{Effects of taxonomic and trophic aggregation on food web properties}},
url = {http://link.springer.com/article/10.1007/s004420050310},
volume = {112},
year = {1997}
}
@article{Link2002,
abstract = {More recent and extensive food web studies have questioned some of the prevailing paradigms of food web theory. Yet with few exceptions, most food webs and associated metrics are reported for freshwater or terrestrial systems. I analyzed the food web of the Northeast US Shelf ecosystem across a large spatial and temporal extent. This speciose food web exhibits a predator:prey ratio (0.95) and percentage of intermediate species (89{\%}) similar to most other food webs. Other statistics, such as the percentage of omnivory (62{\%}), percentage of cannibalistic species (31{\%}), num- ber of cycles (5{\%}), and the total number of links (L; 1562) and species (S; 81) are similar to more recent and extensively studied food webs. Finally, this food web exhibits a linkage density (L/S; 19.3), connectivity (C; 48.2{\%}), and Lyapunov stability proxy (S × C; 39.1) that are an order of magnitude higher than other webs or are disproportionate to the number of species observed in this system. Although the exact S and C relationship is contentious, the connectivity of food webs with more than 40 species is approximately 10{\%}, which is very different from the near 50{\%} observed for this ecosys- tem. The openness of marine ecosystems, lack of specialists, long lifespans, and large size changes across the life histories of many marine species can collectively make marine food webs more highly connected than their terrestrial and freshwater counterparts, contrary to food web theory. Changes in connectivity also have ramifications for ecosystem functioning and Lyapunov stability. The high con- nectivity of this food web and the mathematical determinants for stability are consistent with the weak nature of species interactions that have been observed and that are required for system persis- tence. Yet the historically high exploitation rates of marine organisms obfuscate our understanding of marine food web stability. It is possible that marine food webs are inherently very different from their terrestrial or freshwater counterparts, implying the need for modified paradigms of food web theory.},
author = {Link, Jason},
doi = {10.3354/meps230001},
isbn = {0171-8630},
issn = {01718630},
journal = {Marine Ecology Progress Series},
keywords = {Connectivity,Continental shelf,Food web dynamics,Predator,Prey,Species interactions,Stability},
pages = {1--9},
pmid = {17015021},
title = {{Does food web theory work for marine ecosystems?}},
url = {http://www.vliz.be/imis/imis.php?refid=36644},
volume = {230},
year = {2002}
}
@article{Montoya2003,
abstract = {highest-quality empirical food webs to date, introducing a new topological property called the link distribution frequency (i.e. degree distribution), defined as the frequency of species  SL  with  L  links. Non-trivial differences are shown in link distribution frequencies between species-rich and species-poor communities, which might have important consequences for the responses of ecosystems to disturbances. Coarse-grained topological properties observed, as species richness-connectance and number of links-species richness relationships, provide no support for the theory of links-species scaling law or constant connectance across empirical food webs investigated. We further explore these observations by means of simulated food webs resulting from multitrophic assembly models using different functional responses between species. Species richness-connectance and links-species richness relationships of empirical food webs are reproduced by our models, but degree distributions are not properly predicted, suggesting the need of new theoretical approximations to food web assembly. The best agreement between empirical and simulated webs occurs for low values of interaction strength between species, corroborating previous empirical and theoretical findings where weak interactions govern food web dynamics.},
author = {Montoya, Jose M. and Sol{\'{e}}, R V and Sole, Ricard V.},
doi = {10.1034/j.1600-0706.2003.12031.x},
isbn = {0030-1299},
issn = {0030-1299},
journal = {Oikos},
keywords = {Link distribution,assembly,food web,interactions,model},
month = {sep},
number = {3},
pages = {614--622},
pmid = {123},
title = {{Topological properties of food webs: from real data to community assembly models}},
url = {http://www.blackwell-synergy.com/links/doi/10.1034/j.1600-0706.2003.12031.x},
volume = {102},
year = {2003}
}
@article{Dunne2002a,
abstract = {This paper examines the interpretation of the World Heritage city Luang Prabang (the former royal capital of Laos), investigating the relationships between the goals and strategies of international organizations such as UNESCO and the priorities of the Lao state. Refuting the idea that the World Heritage system represents a form of cultural globalization, the authors instead suggest that there is a marked convergence of the interests of international heritage bodies managing World Heritage and the Lao authorities anxious to portray a particular vision of national identity through selective recognition of cultural heritage locations.},
author = {Dunne, Jennifer A. and Williams, Richard J. and Martinez, Neo D.},
doi = {10.1073/pnas.},
file = {:Users/alyssacirtwill/Desktop/dunne2002.pdf:pdf},
isbn = {0891243208},
issn = {0967828X},
journal = {Proceedings of the National Academy of Sciences of the United States of America},
keywords = {Cultural heritage,Globalization,Laos,Luang Prabang,Nationalism},
number = {20},
pages = {12917--12922},
pmid = {59748879},
title = {{Food-web structure and network theory: the role of connectance and size}},
volume = {99},
year = {2002}
}
@article{Everitt2006,
abstract = {Book},
archivePrefix = {arXiv},
arxivId = {arXiv:1011.1669v3},
author = {Everitt, Brian S and Hothorn, Torsten},
doi = {10.1198/tas.2003.s221},
eprint = {arXiv:1011.1669v3},
isbn = {9781420079333},
issn = {0003-1305},
journal = {The American Statistician},
keywords = {Principal Components Analysis},
month = {feb},
number = {2},
pages = {359},
pmid = {14183309},
publisher = {Chapman and Hall/CRC},
title = {{A Handbook of Statistical Analyses Using}},
url = {papers2://publication/uuid/107899C5-7674-48E2-8062-70941327F784},
volume = {57},
year = {2010}
}
@article{Ho2008,
abstract = {A Floresta Ombr{\'{o}}fila Mista Alto-Montana {\'{e}} uma forma{\c{c}}{\~{a}}o pouco estudada que ocorre em altitudes acima de 1.000 m. Os objetivos deste estudo foram conhecer os padr{\~{o}}es flor{\'{i}}sticos e estruturais do componente arb{\'{o}}reo de um fragmento desta floresta na regi{\~{a}}o do Planalto Sul Catarinense e determinar as vari{\'{a}}veis ambientais que influenciam esses padr{\~{o}}es. O levantamento da composi{\c{c}}{\~{a}}o flor{\'{i}}stica e estrutural e a coleta das vari{\'{a}}veis ambientais foram conduzidos em 50 parcelas de 200 m2 . Nelas, todos os indiv{\'{i}}duos arb{\'{o}}reos com CAP (circunfer{\^{e}}ncia medida a altura do peito) ≥ 15,7 cm foram medidos (CAP e altura) e identificados. Foram coletadas, em cada parcela, vari{\'{a}}veis ambientais relacionadas {\`{a}}s caracter{\'{i}}sticas qu{\'{i}}micas e f{\'{i}}sicas dos solos, topogr{\'{a}}ficas e de cobertura do dossel. Foram calculados os par{\^{a}}metros fitossociol{\'{o}}gicos e a estrutura diam{\'{e}}trica da comunidade e das popula{\c{c}}{\~{o}}es com valor de import{\^{a}}ncia (VI) acima de 5 {\%}. A similaridade flor{\'{i}}stico-estrutural entre as parcelas foi analisada pela NMDS (Nonmetric Multidimensional Scaling) e os vetores das vari{\'{a}}veis ambientais significativas (p {\textless} 0,05) foram plotados a posteriori. Foram identificadas 50 esp{\'{e}}cies arb{\'{o}}reas distribu{\'{i}}das em 33 g{\^{e}}neros e 20 fam{\'{i}}lias bot{\^{a}}nicas. As esp{\'{e}}cies com maior VI foram: Araucaria angustifolia (Bertol.) Kuntze (17,32 {\%}), Myrceugenia euosma (O.Berg) D.Legrand (15,24 {\%}) e Acca sellowiana (O.Berg) Burret (7,84 {\%}). A estrutura diam{\'{e}}trica de toda a comunidade e das popula{\c{c}}{\~{o}}es estudadas (exceto Dicksonia sellowiana Hook.) teve distribui{\c{c}}{\~{a}}o pr{\'{o}}xima ao “J invertido”. A an{\'{a}}lise NMDS demonstrou maior porcentagem de argila nas parcelas com maior densidade de Araucaria angustifolia e menor porcentagem, nas parcelas com maior densidade de Dicksonia sellowiana, Inga lentiscifolia Benth. e Ocotea pulchella Mart. As parcelas de maior declividade tiveram maior densidade de Drimys brasiliensis Miers e aquelas de menor declividade, maior cota e maior cobertura do dossel, tiveram maior ocorr{\^{e}}ncia de Drimys angustifolia Miers, Prunus myrtifolia (L.) Urb., Calyptranthes concinna DC. e Myrceugenia oxysepala (Burret) D.Legrand {\&} Kausel},
author = {Higuchi, Pedro and da Silva, Ana Carolina and de Almeida, Jaime Antonio and Bortoluzzi, Roseli Lopes Da Costa and Mantovani, Adelar and Ferreira, Tiago De Souza and de Souza, Sheila Trierveiler and Gomes, Juliano Pereira and da Silva, Karina Montibeller},
doi = {10.1177/0272989X07302131},
isbn = {3318},
issn = {01039954},
journal = {Ciencia Florestal},
keywords = {Araucaria forest,NMDS,Nebular forest},
number = {1},
pages = {153--164},
title = {{Flor{\'{i}}stica e estrutura do componente arb{\'{o}}reo e an{\'{a}}lise ambiental de um fragmento de floresta ombr{\'{o}}fila mista alto-montana no Munic{\'{i}}pio de Painel, SC}},
volume = {23},
year = {2013}
}
@article{Apellaniz2012,
abstract = {Productivity, habitat heterogeneity and environmental similarity are of the most widely accepted hypotheses to explain spatial patterns of species richness and species composition similarity. Environmental factors may exhibit seasonal changes affecting species distributions. We explored possible changes in spatial patterns of bird species richness and species composition similarity. Feeding habits are likely to have a major influence in bird-environment associations and, given that food availability shows seasonal changes in temperate climates, we expect those associations to differ by trophic group (insectivores or granivores). We surveyed birds and estimated environmental variables along line-transects covering an E-W gradient of annual precipitation in the Pampas of Argentina during the autumn and the spring. We examined responses of bird species richness to spatial changes in habitat productivity and heterogeneity using regression analyses, and explored potential differences between seasons of those responses. Furthermore, we used Mantel tests to examine the relationship between species composition similarity and both the environmental similarity between sites and the geographic distance between sites, also assessing differences between seasons in those relationships. Richness of insectivorous birds was directly related to primary productivity in both seasons, whereas richness of seed-eaters showed a positive association with habitat heterogeneity during the spring. Species composition similarity between assemblages was correlated with both productivity similarity and geographic proximity during the autumn and the spring, except for insectivore assemblages. Diversity within main trophic groups seemed to reflect differences in their spatial patterns as a response to changes between seasons in the spatial patterns of food resources. Our findings suggest that considering different seasons and functional groups in the analyses of diversity spatial pattern could contribute to better understand the determinants of biological diversity in temperate climates. {\textcopyright} 2011 The Authors. Austral Ecology {\textcopyright} 2011 Ecological Society of Australia.},
author = {Apellaniz, Melisa and Bellocq, M. Isabel and Filloy, Julieta},
doi = {10.1111/j.1442-9993.2011.02311.x},
issn = {14429985},
journal = {Austral Ecology},
keywords = {Environmental similarity,Habitat heterogeneity,Primary productivity,Species richness,Species turnover},
month = {aug},
number = {5},
pages = {547--555},
title = {{Bird diversity patterns in Neotropical temperate farmlands: The role of environmental factors and trophic groups in the spring and autumn}},
url = {http://doi.wiley.com/10.1111/j.1442-9993.2011.02311.x},
volume = {37},
year = {2012}
}
@article{Williams,
author = {Williams, Richard J},
pages = {1--6},
title = {{Supplementary Information for “ Simple MaxEnt Models for Food Web Degree Distributions ” by}}
}
@article{Vermaat2009,
abstract = {The covariance among a range of 20 network structural properties of food webs plus net primary productivity was assessed for 14 published food webs using principal components analysis. Three primary components explained 84{\%} of the variability in the data sets, suggesting substantial covariance among the properties employed in the literature. The first dimension explained 48{\%} of the variance and could be ascribed to connectance, covarying significantly with the proportion of intermediate species and characteristic path length. The second dimension explained 19{\%} and was related to trophic species richness. The third axis explained 17{\%} and was related to ecosystem net primary productivity. A distinct opposite clustering of connectance, the proportion of intermediate species, and mean trophic level vs. the proportion of top and basal species and path length suggests a dichotomy in food-web structure. Food webs appear either clustered and highly interconnected or elongated with fewer links.},
author = {Vermaat, Jan E. and Dunne, Jennifer A. and Gilbert, Alison J.},
doi = {10.1890/07-0978.1},
isbn = {0012-9658},
issn = {00129658},
journal = {Ecology},
keywords = {Connectance,Food-web properties,Network topology,Principal components,Trophic species},
month = {jan},
number = {1},
pages = {278--282},
pmid = {19294932},
title = {{Major dimensions in food-web structure properties}},
url = {http://www.ncbi.nlm.nih.gov/pubmed/19294932 http://www.esajournals.org/doi/pdf/10.1890/07-0978.1},
volume = {90},
year = {2009}
}
@article{Almany2004,
abstract = {Priority effects occur when established residents influence the colonization of individuals entering the community and thus provide insight into mechanisms underlying spatial differences and temporal changes in community composition. Using 20 spatially isolated patch reefs, I factorially manipulated the presence and absence of resident predators (groupers and dottybacks) and potential competitors (damselfishes) to determine whether and how they affect subsequent recruitment and mortality of newly settled fishes. During the 50-day experiment at Lizard Island (Great Barrier Reef, western Pacific), prior residency by predators dramatically reduced recruitment of damselfish, surgeonfish, butterflyfish, and rabbitfish and increased damselfish recruit mortality. In contrast, prior residency by potential competitors only reduced recruitment of damselfish and rabbitfish and did not affect recruit mortality. Effects of competitors were likely due to aggressive interactions between competitors and recruits that increased susceptibility of recruits to predators. Effects of residents were strongest within 48 hours of settlement, resulting in rapid establishment of patterns that persisted to the conclusion of the experiment. These results are similar to those from a comparable experiment in the Bahamas (western Atlantic), suggesting that priority effects may be a generally important cause of temporal and spatial variability in the composition of reef fish communities.},
author = {Almany, Glenn R.},
doi = {10.1890/03-3166},
isbn = {0012-9658},
issn = {00129658},
journal = {Ecology},
keywords = {Community dynamics,Competition,Predation,Recruitment,Settlement},
number = {10},
pages = {2872--2880},
title = {{Priority effects in coral reef fish communities of the great barrier reef}},
url = {papers://a0f45058-72ca-4dc9-be03-e459899c7713/Paper/p423},
volume = {85},
year = {2004}
}
@article{Breslow1993,
author = {Breslow, Norman E. and Clayton, David G},
doi = {10.2307/2290687},
isbn = {0162-1459},
issn = {01621459},
journal = {Journal of the American Statistical Association},
keywords = {penalized quasi-likelihood,spatial aggregation,variance components},
number = {421},
pages = {9},
pmid = {15238340},
title = {{Approximate Inference in Generalized Linear Mixed Models}},
url = {http://www.jstor.org/stable/2290687?origin=crossref},
volume = {88},
year = {1993}
}
@article{Stouffer2006b,
abstract = {Intervality of a food web is related to the number of trophic dimensions characterizing the niches in a community. We introduce here a mathematically robust measure for food web intervality. It has previously been noted that empirical food webs are not strictly interval; however, upon comparison to suitable null hypotheses, we conclude that empirical food webs actually do exhibit a strong bias toward contiguity of prey, that is, toward intervality. Further, our results strongly suggest that empirically observed species and their diets can be mapped onto a single dimension. This finding validates a critical assumption in the recently proposed static niche model and provides guidance for ongoing efforts to develop dynamic models of ecosystems.},
author = {Stouffer, Daniel B and Camacho, Juan and Amaral, Lu{\'{i}}s Antonio Nunes},
doi = {10.1073/pnas.0603844103},
isbn = {0603844103},
issn = {0027-8424},
journal = {Proceedings of the National Academy of Sciences of the United States of America},
keywords = {Biological,Computer Simulation,Empirical Research,Food Chain,Models},
month = {dec},
number = {50},
pages = {19015--19020},
pmid = {17146055},
title = {{A robust measure of food web intervality.}},
url = {http://www.pubmedcentral.nih.gov/articlerender.fcgi?artid=1748169{\&}tool=pmcentrez{\&}rendertype=abstract},
volume = {103},
year = {2006}
}
@article{Shurin2006,
abstract = {Ecologists have greatly advanced our understanding of the processes that regulate trophic structure and dynamics in ecosystems. However, the causes of systematic variation among ecosystems remain controversial and poorly elucidated. Contrasts between aquatic and terrestrial ecosystems in particular have inspired much speculation, but only recent empirical quantification. Here, we review evidence for systematic differences in energy flow and biomass partitioning between producers and herbivores, detritus and decomposers, and higher trophic levels. The magnitudes of different trophic pathways vary considerably, with less herbivory, more decomposers and more detrital accumulation on land. Aquatic-terrestrial differences are consistent across the global range of primary productivity, indicating that structural contrasts between the two systems are preserved despite large variation in energy input. We argue that variable selective forces drive differences in plant allocation patterns in aquatic and terrestrial environments that propagate upward to shape food webs. The small size and lack of structural tissues in phytoplankton mean that aquatic primary producers achieve faster growth rates and are more nutritious to heterotrophs than their terrestrial counterparts. Plankton food webs are also strongly size-structured, while size and trophic position are less strongly correlated in most terrestrial (and many benthic) habitats. The available data indicate that contrasts between aquatic and terrestrial food webs are driven primarily by the growth rate, size and nutritional quality of autotrophs. Differences in food-web architecture (food chain length, the prevalence of omnivory, specialization or anti-predator defences) may arise as a consequence of systematic variation in the character of the producer community.},
author = {Shurin, Jb and Gruner, Daniel S. and Hillebrand, Helmut},
doi = {10.1098/rspb.2005.3377},
isbn = {0962-8452},
issn = {0962-8452},
journal = {Proceedings of the Royal Society B},
keywords = {Animals,Biomass,Feeding Behavior,Food Chain,Models,Theoretical,Water},
month = {jan},
number = {1582},
pages = {1--9},
pmid = {16519227},
title = {{All wet or dried up? Real differences between aquatic and terrestrial food webs}},
url = {http://rspb.royalsocietypublishing.org/content/273/1582/1.short{\%}5Cnhttp://www.pubmedcentral.nih.gov/articlerender.fcgi?artid=1560001{\&}tool=pmcentrez{\&}rendertype=abstract},
volume = {273},
year = {2006}
}
@article{GarlandJr.2000,
abstract = {Two phylogenetic comparative methods, independent contrasts and generalized least squares models, can be used to determine the statistical relationship between two or more traits. We show that the two approaches are functionally identical and that either can be used to make statistical inferences about values at internal nodes of a phylogenetic tree (hypothetical ancestors), to estimate relationships between characters, and to predict values for unmeasured species. Regression equations derived from independent contrasts can be placed back onto the original data space, including computation of both confidence intervals and prediction intervals for new observations. Predictions for unmeasured species (including extinct forms) can be made increasingly accurate and precise as the specificity of their placement on a phylogenetic tree increases, which can greatly increase statistical power to detect, for example, deviation of a single species from an allometric prediction. We reexamine published data for basal metabolic rates (BMR) of birds and show that conventional and phylogenetic allometric equations differ significantly. In new results, we show that, as compared with nonpasserines, passerines exhibit a lower rate of evolution in both body mass and mass-corrected BMR; passerines also have significantly smaller body masses than their sister clade. These differences may justify separate, clade-specific allometric equations for prediction of avian basal metabolic rates.},
author = {{Garland, Jr.,}, Theodore and Ives, Anthony R.},
doi = {10.1086/303327},
isbn = {0003-0147},
issn = {0003-0147},
journal = {The American Naturalist},
keywords = {a renaissance,allometry,ancestor reconstruction,and comparative data sets,are now,comparative method,in the last decade,interspecific comparisons have undergone,metabolic rate,phylogeny,regression},
month = {mar},
number = {3},
pages = {346--364},
pmid = {10718731},
title = {{Using the Past to Predict the Present: Confidence Intervals for Regression Equations in Phylogenetic Comparative Methods}},
volume = {155},
year = {2000}
}
@article{Gravel2011,
abstract = {MacArthur and Wilson's Theory of Island Biogeography (TIB) is among the most well-known process-based explanations for the distribution of species richness. It helps understand the species-area relationship, a fundamental pattern in ecology and an essential tool for conservation. The classic TIB does not, however, account for the complex structure of ecological systems. We extend the TIB to take into account trophic interactions and derive a species-specific model for occurrence probability. We find that the properties of the regional food web influence the species-area relationship, and that, in return, immigration and extinction dynamics affect local food web properties. We compare the accuracy of the classic TIB to our trophic TIB to predict community composition of real food webs and find strong support for our trophic extension of the TIB. Our approach provides a parsimonious explanation to species distributions and open new perspectives to integrate the complexity of ecological interactions into simple species distribution models.},
author = {Gravel, Dominique and Massol, Fran{\c{c}}ois and Canard, Elsa and Mouillot, David and Mouquet, Nicolas},
doi = {10.1111/j.1461-0248.2011.01667.x},
isbn = {1461-023X},
issn = {1461023X},
journal = {Ecology Letters},
keywords = {Complexity,Ecological network,Food web,Island biogeography,Metacommunity,Species-area relationship},
month = {oct},
number = {10},
pages = {1010--1016},
pmid = {21806744},
title = {{Trophic theory of island biogeography}},
url = {http://www.ncbi.nlm.nih.gov/pubmed/21806744},
volume = {14},
year = {2011}
}
@article{Piechnik2008c,
abstract = {Ecologists have found many patterns in food-web structure. Some, like the constant connectance hypothesis, lack definitive explanatory mechanisms. In response, we investigated whether community assembly mechanisms could explain why trophic complexity consistently scales with species richness among ecosystems. We analyzed how food-web structure developed during the community assembly recorded in Simberloff and Wilson's classic biogeography experiment. Using their arthropod surveys, we constructed six time series of food-webs from pre- and post-defaunation censuses of six experimental islands, and synthesized trophic information for 250 species from the literature and expert sources. We found that the fraction of specialist species increased and the fraction of generalists decreased during food-web assembly. Directed connectance initially declined over time, despite an increase in species richness, but eventually leveled off as predicted by the constant connectance hypothesis of diversity-complexity scaling. The initial decline was explained by later colonization by trophic specialists, probably due to limited resource availability during early colonization. Late- colonizing super-generalists maintained constant connectance at later dates. This relationship between colonization success and trophic breadth helps explain food-web patterns and corroborates assertions that community assembly is systematically influenced by species' trophic breadths},
author = {Piechnik, DA and Lawler, SP and Martinez, ND},
doi = {10.1111/j.2007.0030-1299.15915.x},
file = {:Users/alyssacirtwill/Documents/Papers/Piechnik, Lawler, Martinez{\_}2008{\_}Oikos.pdf:pdf},
journal = {Oikos},
pages = {665--674},
title = {{Food{\^{a}}€web assembly during a classic biogeographic study: species'“trophic breadth” corresponds to colonization order}},
url = {http://onlinelibrary.wiley.com/doi/10.1111/j.0030-1299.2008.15915.x/full},
volume = {117},
year = {2008}
}
@article{Neubert1997,
abstract = {Abstract Resilience is a component of ecological stability; it is assessed as the rate at which perturbations to a stable ecological system decay. The most frequently used estimate of resilience is based on the eigenvalues of the system at its equilibrium. In most cases, this ...},
author = {Neubert, Michaei G. and Caswell, Hal},
doi = {10.1890/0012-9658(1997)078[0653:ATRFMT]2.0.CO;2},
isbn = {00129658},
issn = {00129658},
journal = {Ecology},
keywords = {Compartment models,Eigenvalues,Pulse perturbations,Reactivity,Relative stability,Resilience,Return time,Transient vs. asymptotic dynamics},
number = {3},
pages = {653--665},
pmid = {1011},
title = {{Alternatives to resilience for measuring the responses of ecological systems to perturbations}},
url = {http://doi.wiley.com/10.1890/0012-9658(1997)078[0653:ATRFMT]2.0.CO;2},
volume = {78},
year = {1997}
}
@article{Eubert2009,
author = {Eubert, M Ichael G N and Aswell, H a L C},
doi = {Doi 10.1890/08-2014.1},
isbn = {0012-9658},
journal = {Ecology},
keywords = {multivariate time series,resilience,stability,transient dynamics},
number = {10},
pages = {2683--2688},
title = {{Detecting reactivity R eports R eports}},
volume = {90},
year = {2009}
}
@article{Berg2011,
abstract = {Human-induced alterations in the birth and mortality rates of species and in the strength of interactions within and between species can lead to changes in the structure and resilience of ecological communities. Recent research points to the importance of considering the distribution of body sizes of species when exploring the response of communities to such perturbations. Here, we present a new size-based approach for assessing the sensitivity and elasticity of community structure (species equilibrium abundances) and resilience (rate of return to equilibrium) to changes in the intrinsic growth rate of species and in the strengths of species interactions. We apply this approach on two natural systems, the pelagic communities of the Baltic Sea and Lake V{\"{a}}ttern, to illustrate how it can be used to identify potential keystone species and keystone links. We find that the keystone status of a species is closely linked to its body size. The analysis also suggests that communities are structurally and dynamically more sensitive to changes in the effects of prey on their consumers than in the effects of consumers on their prey. Moreover, we discuss how community sensitivity analysis can be used to study and compare the fragility of communities with different body size distributions by measuring the mean sensitivity or elasticity over all species or all interaction links in a community. We believe that the community sensitivity analysis developed here holds some promise for identifying species and links that are critical for the structural and dynamic robustness of ecological communities},
author = {Berg, Sofia and Christianou, Maria and Jonsson, Tomas and Ebenman, Bo},
doi = {10.1111/j.1600-0706.2010.18864.x},
isbn = {1600-0706},
issn = {00301299},
journal = {Oikos},
month = {apr},
number = {4},
pages = {510--519},
title = {{Using sensitivity analysis to identify keystone species and keystone links in size-based food webs}},
url = {http://doi.wiley.com/10.1111/j.1600-0706.2010.18864.x},
volume = {120},
year = {2011}
}
@article{Jiang2011,
abstract = {Abstract The relationship between resident species diversity and invasion is generally negative in experimental studies but takes various forms in observational studies of natural communities. We hypothesized that stochastic species colonization, which applies to natural communities but not to experimental communities generally assembled through simultaneous species introduction, may lead to nonnegative diversity-invasion relationships via incurring priority effects. To test this hypothesis, we manipulated both resident species diversity and colonization history in sequentially assembled communities of bacterivorous protist species. We found that, despite a significant effect of assembly history on invader abundance, invader abundance decreased with diversity. This result was largely driven by positive selection effects associated with the dominant influence of an invasion-resistant species, which shared the most similar resource use pattern with the invader, and by the overall weak priority effects observed for the resident communities. Increasing species diversity, however, significantly strengthened priority effects, providing the first experimental support for the idea that larger species pools promote alternative community states. We suggest that elucidating mechanisms regulating the strength of priority effects may help in understanding variation in diversity-invasion relationships among natural communities.},
author = {Jiang, Lin and Brady, Lauren and Tan, Jiaqi},
doi = {10.1086/661242},
isbn = {0003-0147},
issn = {00030147},
journal = {The American naturalist},
keywords = {alternative stable states,assembly,biological invasions,community,community invasibility,priority effects,species diversity},
month = {sep},
number = {3},
pages = {411--418},
pmid = {21828996},
title = {{Species diversity, invasion, and alternative community States in sequentially assembled communities.}},
url = {http://www.ncbi.nlm.nih.gov/pubmed/21828996},
volume = {178},
year = {2011}
}
@article{Peay2012,
abstract = {Priority effects, in which the outcome of species interactions depends on the order of their arrival, are a key component of many models of community assembly. Yet, much remains unknown about how priority effects vary in strength among species in a community and what factors explain this variation. We experimented with a model natural community in laboratory microcosms that allowed us to quantify the strength of priority effects for most of the yeast species found in the floral nectar of a hummingbird-pollinated shrub at a biological preserve in northern California. We found that priority effects were widespread, with late-arriving species experiencing strong negative effects from early-arriving species. However, the magnitude of priority effects varied across species pairs. This variation was phylogenetically non-random, with priority effects stronger between closer relatives. Analysis of carbon and amino acid consumption profiles indicated that competition between closer relatives was more intense owing to higher ecological similarity, consistent with Darwin's naturalization hypothesis. These results suggest that phylogenetic relatedness between potential colonists may explain the strength of priority effects and, as a consequence, the degree to which community assembly is historically contingent.},
author = {Peay, Kabir G and Belisle, Melinda and Fukami, Tadashi},
doi = {10.1098/rspb.2011.1230},
isbn = {1471-2954 (Electronic)$\backslash$r0962-8452 (Linking)},
issn = {1471-2954},
journal = {Proceedings. Biological sciences / The Royal Society},
keywords = {Biological,California,Models,Phylogeny,Plant Nectar,Population Dynamics,Yeasts,Yeasts: isolation {\&} purification,Yeasts: physiology},
month = {feb},
number = {1729},
pages = {749--58},
pmid = {21775330},
title = {{Phylogenetic relatedness predicts priority effects in nectar yeast communities.}},
url = {http://www.ncbi.nlm.nih.gov/pubmed/21775330{\%}5Cnhttp://www.pubmedcentral.nih.gov/articlerender.fcgi?artid=3248732{\&}tool=pmcentrez{\&}rendertype=abstract},
volume = {279},
year = {2012}
}
@article{Fukami2005,
abstract = {Despite decades of research, it remains controversial whether ecological communities converge towards a common structure determined by environmental conditions irrespective of assembly history. Here, we show experimentally that the answer depends on the level of community organization considered. In a 9-year grassland experiment, we manipulated initial plant composition on abandoned arable land and subsequently allowed natural colonization. Initial compositional variation caused plant communities to remain divergent in species identities, even though these same communities converged strongly in species traits. This contrast between species divergence and trait convergence could not be explained by dispersal limitation or community neutrality alone. Our results show that the simultaneous operation of trait-based assembly rules and species-level priority effects drives community assembly, making it both deterministic and historically contingent, but at different levels of community organization.},
author = {Fukami, Tadashi and Bezemer, T. Martijn and Mortimer, Simon R. and {Van Der Putten}, Wim H.},
doi = {10.1111/j.1461-0248.2005.00829.x},
isbn = {1461-023X},
issn = {1461023X},
journal = {Ecology Letters},
keywords = {Alternative states,Assembly history,Assembly rules,Community convergence,Dispersal limitation,Ecological restoration,Historical contingency,Neutral theory,Priority effects,Succession},
month = {dec},
number = {12},
pages = {1283--1290},
pmid = {233313500005},
title = {{Species divergence and trait convergence in experimental plant community assembly}},
url = {http://doi.wiley.com/10.1111/j.1461-0248.2005.00829.x},
volume = {8},
year = {2005}
}
@article{Montoya2006,
abstract = {Darwin used the metaphor of a 'tangled bank' to describe the complex interactions between species. Those interactions are varied: they can be antagonistic ones involving predation, herbivory and parasitism, or mutualistic ones, such as those involving the pollination of flowers by insects. Moreover, the metaphor hints that the interactions may be complex to the point of being impossible to understand. All interactions can be visualized as ecological networks, in which species are linked together, either directly or indirectly through intermediate species. Ecological networks, although complex, have well defined patterns that both illuminate the ecological mechanisms underlying them and promise a better understanding of the relationship between complexity and ecological stability.},
author = {Montoya, J M and Pimm, S L and Sole, R V},
doi = {Doi 10.1038/Nature04927},
isbn = {0028-0836},
issn = {1476-4687},
journal = {Nature},
keywords = {body-size,coevolutionary networks,community description,compartments,complex networks,food-web structure,patterns,species abundance,stability,trophic interactions},
month = {jul},
number = {7100},
pages = {259--264},
pmid = {16855581},
title = {{Ecological networks and their fragility}},
volume = {442},
year = {2006}
}
@article{Cumming2007,
abstract = {Error bars commonly appear in figures in publications, but experimental biologists are often unsure how they should be used and interpreted. In this article we illustrate some basic features of error bars and explain how they can help communicate data and assist correct interpretation. Error bars may show confidence intervals, standard errors, standard deviations, or other quantities. Different types of error bars give quite different information, and so figure legends must make clear what error bars represent. We suggest eight simple rules to assist with effective use and interpretation of error bars.},
archivePrefix = {arXiv},
arxivId = {Error bars in experimental biology},
author = {Cumming, Geoff and Fidler, Fiona and Vaux, David L.},
chapter = {7},
doi = {10.1083/jcb.200611141},
editor = {Hart, Paul J B and Reynolds, John D},
eprint = {Error bars in experimental biology},
institution = {School of Psychological Science and 2Department of Biochemistry, La Trobe University, Melbourne, Victoria, Australia 3086. g.cumming@latrobe.edu.au},
isbn = {0021-9525},
issn = {00219525},
journal = {Journal of Cell Biology},
number = {1},
pages = {7--11},
pmid = {17420288},
publisher = {The Rockefeller University Press},
title = {{Error bars in experimental biology}},
url = {http://www.ncbi.nlm.nih.gov/pubmed/17420288},
volume = {177},
year = {2007}
}
@article{Bolker2009,
abstract = {How should ecologists and evolutionary biologists analyze nonnormal data that involve random effects? Nonnormal data such as counts or proportions often defy classical statistical procedures. Generalized linear mixed models (GLMMs) provide a more flexible approach for analyzing nonnormal data when random effects are present. The explosion of research on GLMMs in the last decade has generated considerable uncertainty for practitioners in ecology and evolution. Despite the availability of accurate techniques for estimating GLMM parameters in simple cases, complex GLMMs are challenging to fit and statistical inference such as hypothesis testing remains difficult. We review the use (and misuse) of GLMMs in ecology and evolution, discuss estimation and inference and summarize 'best-practice' data analysis procedures for scientists facing this challenge. {\textcopyright} 2008 Elsevier Ltd. All rights reserved.},
archivePrefix = {arXiv},
arxivId = {1003.3921v1},
author = {Bolker, Benjamin M. and Brooks, Mollie E. and Clark, Connie J. and Geange, Shane W. and Poulsen, John R. and Stevens, M. Henry H and White, Jada Simone S},
chapter = {127},
doi = {10.1016/j.tree.2008.10.008},
editor = {Seidlhofer, Barbara},
eprint = {1003.3921v1},
institution = {Department of Botany and Zoology, University of Florida, Gainesville, FL 32611-8525, USA. bolker@ufl.edu},
isbn = {0169-5347},
issn = {01695347},
journal = {Trends in Ecology and Evolution},
number = {3},
pages = {127--135},
pmid = {19185386},
publisher = {Elsevier Ltd},
title = {{Generalized linear mixed models: a practical guide for ecology and evolution}},
url = {http://www.ncbi.nlm.nih.gov/pubmed/19185386},
volume = {24},
year = {2009}
}
@article{Ves2011,
abstract = {There is growing appreciation that ecological communities are phylogenetically structured, with phylogenetically closely related species either more or less likely to co-occur at the same site. Here, we present phylogenetic generalized linear mixed models (PGLMMs) that can statistically test a wide variety of phylogenetic patterns in community structure. In contrast to most current statistical approaches that rely on community metrics and randomization tests, PGLMMs are model-based statistics that fit observed presence/absence data to underlying hypotheses about the distributions of species among communities. We built four PGLMMs to address (1) phylogenetic patterns in community composition, (2) phylogenetic variation in species sensitivities to environmental gradients among communities, (3) phylogenetic repulsion in which closely related species are less likely to co-occur, and (4) trait-based variation in species sensitivities to environmental gradients. We also built a fifth PGLMM to test a key underly...},
author = {Ives, Anthony R. and Helmus, Matthew R.},
doi = {10.1890/10-1264.1},
isbn = {0012-9615},
issn = {00129615},
journal = {Ecological Monographs},
keywords = {Ecophylogenetics,Environmental gradient,Generalized linear models, GLMM,Null model,Phylogenetic community structure,Phylogenetic diversity,Phylogenetic signal,Trait variation,Trait-based community assembly},
number = {3},
pages = {511--525},
title = {{Generalized linear mixed models for phylogenetic analyses of community structure}},
volume = {81},
year = {2011}
}
@article{Wilson1969,
author = {Wilson, Edward O. and Simberloff, Daniel S.},
journal = {Ecology},
number = {2},
pages = {267--278},
title = {{Experimental zoogeography of islandas: Defaunation and monitoring techniques}},
url = {http://www.jstor.org/stable/10.2307/1934855},
volume = {50},
year = {1969}
}
@article{Simberloff1969,
abstract = {We report here the first evidence of faunistic equilibrium obtained through controlled, replicated experiments, together with an analysis of the immigration and extinction processes of animal species based on direct observations. The colonization of six small mangrove islands in Florida Bay by terrestrial arthropods was monitored at frequent intervals for 1 year after removal of the original fauna by methyl bromide fumigation. Both the observed data and climatic considerations imply that seasonality had little effect upon the basic shape of the colonization curves of species present vs. time. By 250 days after defaunation, the faunas of all the islands except the most distant one ("El") had regained species numbers and composition similar to those of untreated islands even though population densities were still abnormally low. Although early colonists included both weak and strong fliers, the former, particularly psocopterans, were usually the first to produce large populations. Among these same early invaders were the taxa displaying both the highest extinction rates and the greatest variability in species composition on the different islands. Ants, the ecological dominants of mangrove islands, were among the last to colonize, but they did so with the highest degree of predictability. The colonization curves plus static observations on untreated islands indicate strongly that a dynamic equilibrium number of species exists for any island. We believe the curves are produced by colonization involving little if any interaction, then a gradual decline as inter? action becomes important, and finally, a lasting dynamic equilibrium. Equations are given for the early immigration, extinction, and colonization curves. Dispersal to these islands is predominantly through aerial transport, both active and pas? sive. Extinction of the earliest colonists is probably caused chiefly by such physical factors as drowning or lack of suitable breeding sites and less commonly by competition and predation. !"-s population sizes increase it is expected that competition and predation will become more important. Observed turnover rates showed wide variance, with most values between 0.05 and 0.50 species/day. True turnover rates are probably much higher; with 0.67 species/day the extreme lower limit on any island. This very high value is at least roughly consistent with the turnover equation derived from the MacArthur-Wilson equilibrium model, which predicts turnover rates on the order of 0.1-1.0 species/day on the experimental islands.},
archivePrefix = {arXiv},
arxivId = {arXiv:1011.1669v3},
author = {Simberloff, Daniel S and Wilson, Edward O},
doi = {10.2307/1934856},
eprint = {arXiv:1011.1669v3},
isbn = {00129658},
issn = {00129658},
journal = {Ecology},
number = {2},
pages = {278--296},
pmid = {7622},
title = {{Experimental zoogeography of islands: the colonization of empty islands}},
url = {http://www.jstor.org/stable/1934856},
volume = {50},
year = {1968}
}
@article{Piechnik2008a,
author = {Piechnik, Da Et Al},
journal = {Oikos},
pages = {1--13},
title = {{Piechnik et al 2008 Appendix}},
year = {2008}
}
@article{Piechnik2008,
abstract = {Ecologists have found many patterns in food-web structure. Some, like the constant connectance hypothesis, lack definitive explanatory mechanisms. In response, we investigated whether community assembly mechanisms could explain why trophic complexity consistently scales with species richness among ecosystems. We analyzed how food-web structure developed during the community assembly recorded in Simberloff and Wilson's classic biogeography experiment. Using their arthropod surveys, we constructed six time series of food-webs from pre- and post-defaunation censuses of six experimental islands, and synthesized trophic information for 250 species from the literature and expert sources. We found that the fraction of specialist species increased and the fraction of generalists decreased during food-web assembly. Directed connectance initially declined over time, despite an increase in species richness, but eventually leveled off as predicted by the constant connectance hypothesis of diversity-complexity scaling. The initial decline was explained by later colonization by trophic specialists, probably due to limited resource availability during early colonization. Late- colonizing super-generalists maintained constant connectance at later dates. This relationship between colonization success and trophic breadth helps explain food-web patterns and corroborates assertions that community assembly is systematically influenced by species' trophic breadths.},
author = {Piechnik, Denise A. and Lawler, Sharon P. and Martinez, Neo D.},
doi = {10.1111/j.0030-1299.2008.15915.x},
file = {:Users/alyssacirtwill/Documents/Papers/Piechnik, Lawler, Martinez{\_}2008{\_}Oikos.pdf:pdf},
isbn = {0030-1299},
issn = {00301299},
journal = {Oikos},
number = {5},
pages = {665--674},
pmid = {6075},
title = {{Food-web assembly during a classic biogeographic study: Species' "trophic breadth" corresponds to colonization order}},
url = {http://onlinelibrary.wiley.com/doi/10.1111/j.0030-1299.2008.15915.x/full},
volume = {117},
year = {2008}
}
@article{Simberloff1978,
abstract = {A number of models for the process of island colonization have been proposed in the literature, ranging from completely stochastic and noninteractive through deterministic and heavily influenced by interspecific competition. Biogeographic distributional data are cited in support of these models, but statistical tests have often been omitted or, where performed, are so weak that one is not completely inclined to accept the hypotheses in question even where rejection is not yet indicated. Broadly speaking, plant and invertebrate distributional data have been interpreted as not heavily influenced by interactions, while vertebrate data have often been cast in a context of diffuse competition. Analysis to date has not been sufficiently profound that one must discard Occam's razor and seek different explanations for the distributions of different taxa.},
author = {Simberloff, Daniel},
doi = {10.2307/2460046},
isbn = {00030147},
issn = {0003-0147},
journal = {American Naturalist},
number = {986},
pages = {713--726},
pmid = {17891731},
title = {{Using island biogeographic distributions to determine if colonization is stochastic}},
url = {http://www.jstor.org/stable/2460046},
volume = {112},
year = {1978}
}
@article{Simberloff1970,
abstract = {1) For any taxon higher than genus, the expected mean number of species per genus (E(S/G)) for a randomly drawn subset is a non-decreasing function (approximately linear for many taxa) of the size of the subset, and is therefore lower than the mean number of species per genus for the entire taxon. 2) The relationship of actual S/G for an island biota to S/G expected for a random subset of identical size drawn from the species pool of the presumed source area is relatively insensitive to modifications of taxonomy and to use of a source area slightly larger than the real one. This relationship is subject to more drastic change upon use of incomplete species lists for island or source or a presumed source area smaller than or different from the real source. 3) Pseudo-random drawings of biotas of identical size to those of a number of islands show that, in general, the mean number of species per genus, though lower than in the source biota, is higher than would be expected on a hypothesis of random colonization. 4) The deviation of actual from expected S/G is not strongly correlated with island area, maximum elevation, and distance from source. 5) The claimed positive correlation between species range and genus size, even if it should be shown to exist, is probably insufficient to account for the magnitude of these deviations. 6) The main causes of the excess of insular S/G's over those predicted by chance are probably two simultaneous tendencies: a) similarity of congeneric species in ecological requirements. b) similarity of congeneric species in dispersal capabilities. CR - Copyright {\&}{\#}169; 1970 Society for the Study of Evolution},
author = {Simberloff, D.S},
doi = {10.2307/2406712},
isbn = {00143820},
issn = {00143820},
journal = {Evolution},
number = {1},
pages = {pp. 23--47},
title = {{Taxonomic Diversity of Island Biotas}},
url = {http://www.jstor.org/stable/2406712},
volume = {24},
year = {1970}
}

