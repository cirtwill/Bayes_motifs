\documentclass[12pt]{article} 
\usepackage{amsmath} 
\usepackage[dvips]{graphicx}
\usepackage{multirow} 
\usepackage{geometry} 
\usepackage{pdflscape}
\usepackage{amsmath}
\usepackage[labelfont=bf]{caption} 
\usepackage{setspace}
\usepackage[running]{lineno} 
% \usepackage[numbers,sort]{natbib}
\usepackage[numbers]{natbib} 
\usepackage{array}
\usepackage[table]{xcolor}
\usepackage{xr}

\newcommand{\us}{\rm \setlength{\leftskip}{0.3cm} \setlength{\rightskip}{0.3cm}}
\newcommand{\them}{\it \setlength{\leftskip}{0cm} \setlength{\rightskip}{0cm}}

\topmargin -1.5cm % 0.0cm 
\oddsidemargin 0.0cm % 0.2cm 
\textwidth 6.5in
\textheight 9.0in % 21cm
\footskip 1.0cm % 1.0cm

\usepackage{authblk}

\setlength{\parskip}{0.8em}
\setlength{\parindent}{0em}
\bibpunct{(}{)}{,}{a}{}{,}
\clubpenalty = 10000
\widowpenalty = 10000

\begin{document} 

\section*{Response to reviewer comments}

\begin{flushleft}
\textbf{Manuscript number: } \texttt{RSPB-2021-2331} \\
``Species motif participation provides unique information about species risk of extinction''\\
Alyssa R. Cirtwill, Anna {\AA}kesson, Kate L. Wootton \& Anna Ekl\"of
\end{flushleft}




\subsection*{Associate Editor}
    \them
    I enjoyed reading this paper, as did the two reviewers, and I would argue that there is the potential to make a large contribution to the literature on this new way of looking at trophic webs. I found the argument for using motifs much more compelling than previous generations of methods in this area. The major point for improvement however is to make this manuscript more approachable by and useful to empiricists.  
    
    \us
    We thank the Associate Editor for their kind words and are especially pleased to have made a stronger case for motifs than some previous efforts have managed. We have taken the Associate Editor and Reviewers' comments to heart and hope that the revised manuscript is more approachable.
    
    \them 
    The following highlights key areas for improvement: first, as reviewer 1 argues, the text can be clarified and made more direct towards the specific goals of the paper, as well as making some of the terminology more transparent.  
    
    \us
    We have thoroughly revised the introduction for clarity and directness, added a glossary in case of any remaining confusion about our terminology, and added explanations of the relevance of each statistical test we conduct. We hope that this makes our goals clearer and results easier to interpret. 
    
    \them
    Second, as reviewer 2 argues there is scope for including a figure about the motifs in the context of their food web to facilitate the understanding about their structure and ecological meaning. The top of Figure 1 helps, but there is room to expand.  
    
    \us
    We have added a conceptual figure showing the four motifs within a food web, with particular attention to how species with the same degree and trophic level can have different motif participation. We hope that this makes the text as a whole more approachable.

    \them
    Last, the authors generate what they argue are ``a range of plausible network architectures''.  But there is never any comparison of these architecture to empirical networks.  Nor is there any calculation of the key motif parameters on empirical data (for example the x-axis of figure 1). It will expand the audience for this manuscript greatly to include some empirical data to help empiricists understand for example in well-studied systems, what is the proportion of these different motifs.  This comparison does not need to be exhaustive, but it will serve two main roles in the manuscript: (1) provide support for the assertion that the authors have generated ``a range of plausible network architectures'' and (2) allow empiricists to understand how these simulations relate to real ecosystems.  
    
    \us
    While the niche model has been shown in previous work to simulate empirical networks reasonably well~\citep{Stouffer2005a,Stouffer2006}, we are happy to provide further confirmation in the supplemental information.
    Specifically, we compare the motif profiles and consumers' motif participation of our simulated webs to the  unipartite webs in the same size range (excluding those where predators and prey were differently resolved) available from the globalweb database.
    Not all of the empirical webs have motif profiles within the simulated range; this could be because of flaws in our simulation or because of flaws in the empirical webs (e.g., poorly-resolved basal resources, a common problem in even high-quality empirical webs). 
    While we cannot say conclusively which is the case, we are encouraged by the fact that the consumer motif participations in our simulated webs span the motif participations in the empirical web.
    We hope that this comparison will improve empiricists' confidence in our simulations.

\clearpage

\subsection*{Referee 1}

    \them
    Thank you for the opportunity to review this paper. I found it very interesting and appreciated the comparison between motifs and more ``classic'' network metrics. I have a couple major concerns which I do not anticipate being that difficult to address, and some smaller comments.
    
    \us
    We thank the reviewer for the thorough engagement with our work, the encouraging words, and the constructive criticisms. 

    \subsubsection*{Major comments}
    
        \them
        First, I think the introduction could be clearer. I think it spent too much time discussing bottom up effects, which I appreciate because that is one of the major limitations to this approach, however, I think some of the text could be better spent setting up some of the additional questions of the paper and clarifying various terminology (e.g. local, meso, and global scale gets thrown about, but is a bit of a surprise).
        
        \us
        The introduction has been thoroughly re-written, and is hopefully clearer now in terms of both terminology and our questions of interest. Additionally, we have now included a glossary in \emph{Appendix S11}
        
        \them
        Second, I found the analysis, particularly the divvying up into 5 parts rather circular. We know that connectance and species richness has a large impact on both motif expression and on persistence. Instead of multiple separate analyses, I think these can all be included in one large model. 
        
        \us One large model will not serve the purpose of our manuscript. We agree that connectance and species richness have a large impact on persistence, but re-proving these well-known effects is not our main focus. Instead, our main goal is to show whether, as is frequently proposed, information about motif participation or profiles can be used to comment on extinction risk. Secondarily, we hope to place these relationships in the context of known relationships between extinction risk and other network properties. We believe that separate, simpler analyses are more effective for this purpose than constructing one large model that hides the relationships between network properties.
        
        We appreciate the Reviewer's general point that the purpose of these separate analyses was not clear. To remedy this, we have restated our questions of interest at the beginning of the `statistical analyses' section and now preface our description of each test with a statement of how it fits into the framework of our manuscript. We hope that the question addressed by each test is now clearer.
        
        \them
        Moreover, I would be interested in an analysis which pits trophic level and in-degree against the motifs. In particular, collecting data on trophic level is much easier than collecting entire food web data for motif roles.
        
        \us
        We are not convinced that collecting data on trophic level is easier than collecting entire food web data. If the Reviewer is referring to stable-isotope trophic levels, then we must point out that calculating trophic levels requires a good background understanding of the food sources for the focal species and rates of incorporation of different isotopes across trophic levels (and how these may differ between taxa). For a large community (e.g., 100 species), it could well be simpler to at least approximate the entire food web (giving trophic level, degree, and motifs simultaneously) than to calculate stable isotope-based trophic levels for each species. Indeed, the larger set of empirical food webs than comprehensive stable-isotope data suggests that food webs are easier to estimate at scale than trophic levels alone.
        
        
        To address this concern, we have added a brief statement in the introduction to make it clear to readers that researchers who have enough information to calculate in-degree and trophic level can also calculate motifs. Motif roles, therefore, provide additional information than degree and trophic level alone without requiring extra sampling.
        
        
        In the same vein, since we cannot imagine a scenario in which researchers have either degree and trophic level or motif participation information but not both (as calculating degrees of all species requires the same diet information used to calculate motifs, even if we assume trophic levels are derived from stable isotopes), we do not believe that it is productive to pit motifs against trophic level. Our aim is to add motif participation to the toolbox of researchers wishing to predict extinction risk, not attempt to supplant trophic level, which we agree provides a large amount of useful information and should be taken into account.

        
        \them
        Third, I realize that this technique is already published on, however, some of the limitations of the technique may impact the utility of the results. I think the study would be greatly enhanced by a smaller comparison of a more traditional dynamic food web model approach and this approach – even just placed in the appendix. I note that the authors rely quite heavily on a paper on BioRxivs which has done something similar, so anticipate this wouldn't be that difficult, but would be very convincing.
        
        \us We agree that a comparison between Bayesian and dynamical simulations could be interesting, however performing such a comparison is beyond the scope of this manuscript. As the Reviewer notes, a subset of the authors of this manuscript are working on an analysis of the relationships between motif participation and persistence in a dynamical framework (currently under review). We do not believe that including the material from this second paper in an appendix is justified when there are already a published comparisons of extinctions simulated by Bayesian and dynamical approaches (\citealp[]{Eklof2013a}). That paper focused on comparing the number and identity of secondary extinctions between the two approaches, and additionally evaluated which functional response of consumers to loss of prey that most accurately predicted the secondary extinctions from a fully-fledged dynamical model. We use this information in the current manuscript and make our simulations with the functional form that was shown to most accurately predict the same secondary extinctions as the dynamical model. We have now clarified this in the manuscript (line 126-127). However, we would also like to point out that there are no guarantees that advanced (as in parameter intensive) dynamical models describe the ecological reality better, as comparisons with extensive, highly-resolved empirical extinction scenarios are still lacking.     
        
        
        In addition, we believe that the Reviewer has over-estimated our reliance on the BioRxiv pre-print currently under review. Our previous draft had only three references to this pre-print: when describing the generation of simulated networks, when noting that degree and motif participation are not independent (which we also demonstrate \emph{de novo} in the present manuscript), and when briefly comparing the dynamical and Bayesian results in the discussion.
        We have now removed two of these references.
        We present the full network-generation methods in \emph{Appendix 1} and, to avoid confusion by referring to results presented in the preprint, have softened our language to say that degree and motif participation \emph{may} be correlated. 
        The latter is not a major point in our manuscript; we mention it only to emphasize that controlling for the total number of motifs in which a species appears does not necessarily control for degree. 
        Thus, we see some overlap in the associations of persistence with degree and with normalised motifs.
        
        
        We have retained the reference to the pre-print in the discussion because this brief comparison of results using dynamical and Bayesian simulations appears to be exactly what the Reviewer has asked for.
        As the Reviewer points out, different approaches to simulating disturbance can have different results, and we explicitly mention the main differences in the Discussion. 
        As the pre-print is freely available to any who wish to critically examine it, we believe that this is a good compromise between maintaining conciseness of the present manuscript and comparing different simulation approaches.
        If, all going well, the preprint is accepted for publication by the time the present manuscript is accepted, we will of course update the reference to provide the peer-reviewed version of the text.
        
        
        \them
        Finally, I appreciate that the authors have used similar methods in a paper on BioRxiv – however, the current study relies quite heavily on it in the methodology. To my knowledge this paper is not peer-reviewed (or at least not yet) and thus I am less inclined to accept these citations as appropriate rationale for various decisions.
        
        \us See above. The only methodological tie between the two manuscripts was when simulating the networks. The underlying software used is peer-reviewed~\citep{Delmas2017} and has been used in several other published studies, as well as the pre-print originally referenced (currently under review). To avoid the mis-perception that we rely on unproven methods, we now present the full network-generation methods in an Appendix.

    \subsubsection*{Minor comments}
        
        \them
        Line 65 – 70: I agree that in-degree and trophic level may be coarse description of a species' role within its food-web. However, the examples provided do not convince me that species' roles based on motifs are a more nuanced way of describing a species' role within its food web since the two examples provided refer to the influence disturbance may have on a basal resource – an influence that I'm not convinced a species' role based on motifs will impact.  
        
        
        \us
        We have revised the introduction throughout, and have taken care to clearly link the extra information given by motifs to our intuition about extinction risk. In particular, motifs give information about a consumer's prey that may reflect the prey's risk of extinction and thereby the predators' risk (e.g., a predator consuming generalist prey will participate in many three-species chain motifs). We have also added a conceptual figure illustrating the four motifs and explaining how different motifs imply different pathways for bottom-up disturbances.
        
        
        \them
        I do think that species' roles based on motifs are a more nuanced way of describing species' roles, and I am interested in the central question of this paper – however, this rationale would not convince me if I was unfamiliar with motifs/species roles/finer scale network descriptors.  As this is a central tenet of the paper I think the argument needs to be constructed better. Furthermore, just because a metric is coarse does not mean it is not an effective predictor – and when applying theoretical/computational results to empirical systems sometimes coarse is better. It is easier to measure the trophic position of a species using stable isotopes and approximate extinction probabilities than it is to construct an entire food web.
        
        \us In our analysis, in-degree was a much stronger predictor of extinction risk than trophic level. We thus do not escape the need for full food-web information to predict extinction risk using coarse measures, even if we accept that stable isotope analyses are much less work than constructing a full food web (and we are not fully convinced of this, see response above). 
        Regardless, we are not suggesting that motifs replace degree or trophic level as all four metrics can be calculated using the same data. We are interested in showing that motifs can provide information about extinction risk that is not given by degree and trophic level.
        
        
        To more clearly illustrate how motifs can provide more information than degree and trophic level, we have added a conceptual figure. This figure illustrates the four motifs and shows how two species with the same degree and trophic level can have different motif participation. We hope that this figure and the revised introduction are more persuasive.
        
        
        \them
        Line 82-84: This sentence doesn't quite make sense to me.
        
        \us
        This sentence was removed when re-writing the introduction.
        
        \them
        Line 86: is this ``focal'' species the consumer mentioned in line 85?
        
        \us
        Rephrased, yes it is the same species, i.e. do the extinction risk of a consumer depend upon the motifs in which it participates in?
        
        \them
        Line 86: the impact of the motif profile of the full food web on the consumer's extinction risk has come as a bit of a surprise to me. I think this could have been introduced better in the introduction – not extensively because I think it is probably a secondary question, but at this point I have no hypotheses for why I would even expect the motif profile of the full food web to out perform the motifs in which a focal species participates.
        
        \us
        We have now revised the introduction to introduce the motif profile of the whole network earlier and make its potential relevance to species-level extinction risk clearer.
        Note that we do not expect the frequencies of motifs at the network to out-perform species-level motif participation. In fact, we do not compare the predictive abilities of motif profiles and motif participation as these models relate to different things. We connect network-level motif profiles to the network-level average extinction risk of all consumers and species-level motif participation to species-level extinction risk of one consumer at a time. We regret the confusion about the goals of our different analyses in the previous draft and hope that it is now clear how our analyses fit together.
        
        
        
        \them
        Line 87 – 89: similar to comment above – I think the idea of global and local food web properties has not been introduced at all, nor the classic food web properties (which ones are being measured? Only the four specific motifs thought to contribute to stability are mentioned).
        
        \us We have revised the introduction to more clearly introduce these concepts. 
        
        \them
        Line 92 – 95: this seems to be a flaw to the Bayesian network simulations – we know that mutual predation happens in food webs – especially if the food web isn't explicitly size structured, and when nested within the broader food web context these feedback loops may have stabilizing, or destabilizing effects.
        \smallskip
        Line 149 – 152: these seem like large limitations to me.
        
        
        \us In revising the introduction, we now introduce the Bayesian network approach and its limitations in the methods section. We want to stress that the Reviewer is absolutely correct that these are important limitations and we have further clarified that in the text. At the same time we have clarified the advantages that the Bayesian network approach provides in order to better make the case that the Bayesian approach is useful in this case.
        First, we clearly state that the lack of top-down effects in the Bayesian networks is a limitation. However, we also point out that in prior studies Bayesian simulations have captured most secondary extinctions identified using dynamical simulations. This suggests that, in many cases, bottom-up effects are the main drivers of secondary extinction. We also emphasize the major gain in computational speed when using the Bayesian approach, which allows us to investigate different levels of disturbance rather than simple species removals that needs to be repeated numerous times. Finally, we have also added text to clarify how the process of making the food webs acyclic can affect the results, but that food web robustness rarely is affected.  
        
        \them
        Line 164 – 166: Does this normalization control for differences in connectance? It seems like the more connected the networks are the higher degree species will have, and then connectance (definitively known to impact species persistence) will be a major driver
        
        \us To the extent that species in higher-connectance networks tend to participate in more motifs, this normalization does address connectance, however that is not its main purpose and it has been tested whether normalizing motif roles in this way fully controls for connectance.
        Since this normalization does not remove correlations between degree and motif participation (as discussed in~\citealp[]{Cirtwill2021_inprep}), we are not confident in making such a claim.
        Instead, to control for connectance explicitly, we include it (and species richness) in a random effect in all regressions where it is not a main effect. 
        We take this approach because re-evaluating the known effects of connectance and species richness is not a major theme in the manuscript.
        We are more interested in comparing extinction risk for species within the same food web (where connectance is constant) than between food webs (where connectance, species richness, and --in empirical systems-- local conditions will have large effects).
        We have lightly revised the methods section (quoted below) to make this focus clearer and to state that, while our motif role normalization does not control for degree or connectance, we do explicitly consider how degree and connectance (as well as species richness and trophic level) may be related. 
        
        
        
            ``Note, however, that this normalization does not necessarily control for differences due to degree or connectance as high- and low-degree species (or species in more- and less-connected networks) may also tend to participate in different types of motifs.
            We address the potential associations between motif participation, degree, and connectance explicitly below.''
        
        
        \them
        Line 182: what is network mean persistence? Species persistence is the probability of species persistence referred to in the methods? Be explicit. Also, what are the ``other network properties''?
        
        \us
        We have revised the methods text to clarify that network mean persistence is the mean probability of persistence across all consumers in the network. We also now specify that ``other network properties'' refer to species richness, connectance, in-degree, and trophic level.
        
        \them
        Line 192: the idea of disturbance level came as a surprise – a careful reading of the methods pinpointed where you discuss this, however it is not discussed in the statistical analysis part, or in the main objectives of the paper at the end of the introduction.
        
        \us This concept is now introduced in the introduction. We give examples from other theoretical studies and empirical studies briefly indicating how intensity of disturbance can affect extinction risk. We also explicitly mention disturbance level in our objectives paragraph and at the start of our statistical analysis section. However, we do note that disturbance level was included as a predictor in the equations in our previous version of the manuscript. We hope that the current revision makes this parameter a more obvious part of our approach.
        
        \them
        Line 193: Does it make sense to include Sn and Cn as random intercepts if it is known that both of these values impact 1) the number of motifs (which I don't think has been rigorously controlled for – see above) and 2) network persistence? (same with eq'n S3)
        
        \us
        We explicitly control for differences in the number of motifs through the frequency normalisation discussed above. This is a more direct approach than controlling for effects of S and C on the total number of motifs, which are likely but have not been thoroughly explored to date. We include the random intercepts to account for potential effects of S and C on network persistence that are \emph{not} due to effects on the total number of motifs. We agree with the Reviewer that the effects of S and C are well-known and important to account for, but they are not the main focus of our manuscript. We therefore choose to keep them as random effects in these equations.
        
        
        \them
        Line 197 – 199: This sounds more statistically rigorous here than it does in the appendix where these are just visually compared.
        
        \us We have added the word ``visually'' to the main text.
        
        \them
        Line 201: I have no idea what local and meso-scale refer to. Likewise global.
        
        \us These terms are explained at their first use in the introduction or methods. To avoid confusion, we also add a glossary in \emph{Appendix S11} so that readers can easily look up these terms. We have also revised the paragraph starting on line 201 to include parenthetical definitions and clarify the scale which we are considering.
        
        \them
        Line 200-206: The statistical analyses of the various components of this study seem very circular to me. Various local metrics depend on global metrics but treated differently in each step. There must be a more appropriate statistical technique for considering all of these moving parts in a comprehensive manner – even using model selection to compare.
        
        \us See response to the Reviewer's second major point, above. We reiterate that our goal is not to prove that motifs should supplant other network metrics for predicting extinction risk. Rather, we want to test the proposal in earlier motif-based papers that motifs are related to extinction risk \emph{as well as} other network properties. Model selection is therefore not of particular interest for our purposes.
        
        \them
        Line 214: Why mean, and why not just that network persistence?
        
        \us We use the term ``network mean persistence'' because this is the mean persistence for all species within the network. ``Network persistence'' could easily be misunderstood as referring to R50 or some other network-level measure of stability. To avoid confusion, we have added the term to the glossary in \emph{Appendix S11}
        
        
        \them
        Line 247-254: This would all be greatly enhanced if there were some error bars on the figures.
        
        \us We have added error bars representing 95\% confidence intervals to this figure.
        
        \them
        Line 259-262: This result seems odd to me. Is this just because all of the potential apparent competition interactions are filled first, and really is more of an artefact of the constraints about the types of motifs that are possible within these networks? And perhaps the connectances being explored are unrealistic given these constraints?
        
        \us While we do not use a generative model that fills some motifs/interactions before others, we do not see any particular reason why apparent competition motifs should be filled first. More likely, the high frequencies of apparent competition reflects the tendency for resources to outnumber consumers, creating the classic ``pyramid shape'' food web. We can confirm, however, that the connectances we explore are realistic. We compare the structure of our simulated networks to the largest set of high-quality, similarly-sized empirical networks we could find. These networks had 50-77 species and connectances of 0.095-0.227, covering the upper half of connectances we simulate. We also, per the Associate Editor's request, compare the motif profiles and motif participation of these empirical webs to our simulated webs. We hope that this comparison will demonstrate the realism of our simulated networks.
        
        \them
        Line 289: Odd use of direct competition when most other papers use exploitative competition
        
        \us We are not aware of any other motif-focused papers calling this motif ``exploitative competition''. Our terminology follows the nomenclature from the introduction of motifs to food webs~\citep{Stouffer2007}.
        Note that ``direct competition'' refers simply to two species which consume the same prey and is agnostic as to whether or not competition involves interference or is purely exploitative. 
        As exploitative and interference competition are not the same ecologically, we have maintained the conventional motif terminology so as not to imply an overly specific interpretation.
        
        
        \them
        Line 334: in-degree and trophic level are not sufficient to explain the trends, yet there is a strong negative correlation between apparent competition and in-degree. This seems contradictory.
        
        \us The Reviewer is correct that there is a strong  correlation between apparent competition and in-degree, as we state in the text. It is also true that apparent competition and in-degree show opposite relationships with persistence, such that the apparent competition-persistence relationship could be described by the in-degree-persistence relationship. However, as detailed later in the paragraph, this ignores the strong correlation between apparent competition and trophic level which would suggest that persistence should show different relationships to apparent competition. This means that either trophic level is unrelated to persistence (which we know is not true) or some other factor is over-riding the ``effect'' (there is no clear causality from trophic level to motifs or vice versa) of trophic level on apparent competition.
        We have revised this paragraph with particular focus on these relationships and hope that clarifies our reasoning .
        
        
        \begin{quotation}
            For apparent competition, there were strong and opposing correlations with in-degree and trophic level but the relationship between apparent competition and persistence only matches the trends for in-degree.
            This suggests that some other factor is over-riding the relationship between apparent competition and trophic level.        
        \end{quotation}
        

        
        \them
        SI references incomplete (search (?))
        
        \us
        Done.




\subsection*{Referee 2}

    \them
    The authors present an interesting study about how motif based measures can improve our understanding of species persistence in food webs after bottom up disturbances. The idea of this study is original and the authors made a great effort to explain a novel methodology, such as motifs, in a clear way. In addition, I consider the exploration of relationships among level of disturbances, network structural properties, species level indices and species persistence very important and necessary to understand food webs stability and to help future authors to select the most appropriate variables according to their goals.

    \us
    We thank the Reviewer for the thorough reading of our manuscript and the helpful comments. 
    
    \them
    I found the introduction very easy to follow with an appropriate background supporting the goals of the study allowing to link concepts such as secondary extinction, network structure and motifs. Perhaps authors might reduce the number of paragraphs of the introduction by simply joining some of them. For example, I would suggest to join the second and third paragraph since both refer to the approaches for studying secondary extinctions. Similarly, I would suggest to join the fourth and fifth paragraph because both explain how the structure of different scales (whole-network and species level) affect the response to disturbances.
    
    \us We have thoroughly revised the introduction in response to Reviewer 1's comments and, in the process, have condensed the sections referring to secondary extinctions and reworked the sections referring to different scales. We have also somewhat expanded our descriptions of different scales and how they can affect response to description. We hope that local-scale and global-scale properties now more obviously justify their own paragraphs.
    
    \them
    Furthermore, the methods and statistical analysis were consistent with the proposed goals and very well written. I would like to highlight the correct use of motifs that authors made in the ms, which allows them to deduce new ecological information in addition to common network indices. Perhaps authors might include a figure with the used motifs in the appendix to facilitate the understanding about their structure and ecological meaning. 
    
    \us  We thank the Reviewer for following our methods and analysis and are glad that they match our goals. We have added a conceptual figure showing the four motifs within a food web, with particular attention to how species with the same degree and trophic level can have different motif participation. We hope that this makes the text as a whole more approachable.
    
    \them
    Regarding the results and discussion section, I found them very well written and fluent. However, because of the great number of results, I would suggest keeping consistency between the subtitles in the section methods and results referring to the same goal.
    
    \us We have revised the subheadings to ensure consistency and thank the Reviewer for the reminder.

    \subsubsection*{Minor revisions}

        \them
        Line 16: I suggest changing the word ``vulnerability'' to ``species persistence'' to keep consistency along the manuscript.
        
        \us
        Changed.
        
        \them
        Line 23-26: this sentence could be clearer. Perhaps authors could rephrase to: ``The structure of the plant community affects stability across all levels of ecosystem organization, which is crucial for sustaining a healthy energy flow in the ecosystem as a whole (Proulx et al., 2010; Scherber et al., 2010, Rosenblatt \& Schmitz, 2016). For example, the loss of diversity…''
        
        \us Changed as suggested.
        
        \them
        Line 57-59: This sentence seems counterintuitive to me. How could more connected networks be more resistant to secondary extinctions than less connected ones? A higher connectance indicates more pathways available where the energy flows, promoting the propagation of a disturbance (Vieira \& Almeida-Neto, 2015). Please justify your statement.
        Vieira, M. C., \& Almeida‐Neto, M. (2015). A simple stochastic model for complex coextinctions in mutualistic networks: robustness decreases with connectance. Ecology Letters, 18(2), 144-152.
        
        \us Connectance can be a double-edged sword depending on the type of disturbance and structure of the rest of the web. Although higher connectance does allow more pathways for a disturbance to propagate, on average a higher connectance also means that each species has more prey. If species can switch between prey, this can allow consumers to persist until the effects of the disturbance has abated. This is easiest to understand when the focal disturbance is species removal. In a low-connectance network, the consumer(s) of a removed species are less likely to have other resources. If a consumer's only prey is removed, it must necessarily go secondarily extinct. In a high-connectance network where most species have many prey, the consumer's fate is much less certain following a single-species removal. We appreciate the prompt to clarify these opposing potential effects of connectance, and have added a reference to the Vieira \& Almeida-Neto paper suggested. These lines now read:
        
            \begin{quotation}
                 Highly-connected networks may be more resistant to secondary extinctions \citep{Dunne2002, Eklof2006} because consumers, on average, have access to a larger number of prey species and can switch from a declining prey to another resource. However, a well-connected network also includes more pathways through which a disturbance can spread, affecting more species and potentially causing more secondary extinctions~\citep{Vieira2015}.
                    The net effect of connectance on extinction therefore likely depends on the severity of disturbance and the ability of consumers to subsist on a reduced set of prey.    
            \end{quotation}
        
        
        \them
        Line 61: Please add new cites to sustain the statement. For example:
        Gilarranz, L. J., Rayfield, B., Liñán-Cembrano, G., Bascompte, J., \& Gonzalez, A. (2017). Effects of network modularity on the spread of perturbation impact in experimental metapopulations. Science, 357(6347), 199-201.
        
        \us
        Added, thanks for mentioning!
        
        \them
        Line 112: Please add ``, respectively'' after ``network''.
        
        \us
        Done.
        
        \them
        Line 161: Perhaps authors might add a figure of the four three-species motifs in the appendix explaining the ecological meaning of each structure. I think this could help readers to understand motifs.
        
        \us We have added these motifs to the conceptual figure described above and hope that this helps readers understand.
        
        \them
        Line 187: Please add ``Species'' before ``persistence'' in the title to improve clarity.
        
        \us
        Done.
        
        \them
        Line 200: Please add ``Motif'' before ``participation'' in the sentence.
        
        \us
        Done.
        
        \them
        Line 207: Add ``Network'' before ``persistence'' in the title.
        
        \us
        Done.
        
        \them
        Line 240: Please keep consistency between the subtitles in the section methods and results referring to the same goal.
        
        \us We have revised the subtitles to be consistent. Thank you for the suggestion.
        
        \them
        Line 270-272: Interesting. This could be related to the connectance of the species. Consumers with higher number of prey might have lower persistence at higher levels of disturbances because the disturbance has more pathways to propagate and reach the consumer.
        
        \us We are not sure what the distinction between a species' degree and its connectance might be - we are used to connectance referring to network-level link density. 
        However, we agree that this is an interesting result and had similar thoughts to the Reviewer that the interaction between degree and disturbance might indicate some kind of trade-off between the benefits of having many prey and the drawbacks of being exposed to disturbance through many pathways.
        We had minimized this theme in the Discussion in order to meet word count limits, but have added a brief example in the final paragraph.
        
        % Original line: In-degree shows an even stronger interaction with disturbance; having more prey species is associated with higher persistence without  disturbance, but lower persistence at high levels of disturbance.
        
        \them
        Line 330-331: Unclear the expression ``, and interactions among,''. Please rephrase.
        
        \us We have revised this line to ``Motifs can thus act as a tool for synthesizing the information provided by other measures of a species' place in its community and for understanding the interactions among these measures.'' and hope that this is now clear.
        
        \them
        Line 357-358: Unclear the expression ``than directly''. Please explain.
        
        \us
        Rephrased. Here, we meant that although we did not see a strong effect on network mean persistence of global-scale properties (such as connectance and size), we found that, in particular, connectance altered the distribution of motifs which in turn affected the network mean persistence.  
        
        \them
        Line 376: Please remove the italic formatting of the word ``and''.
        
        \us
        Changed.
        
        \them
        Figure 5: Please rephrase the first sentence to: ``The proportions of the four motifs in a network's motif profile were related to network size, connectance, and their interaction.''
        
        \us
        Changed, thanks for catching!
        
        \them
        Appendix:
        
        Page 3: Please remove the expression ``(?)'' from the sentence in the third paragraph.
        
        Page 7: Please remove the expression ``(?)'' from the last sentence.
        
        Page 10: Please remove the expression ``(?)'' from the first paragraph of the section S3.3.
        
        \us
        Done. These were incorrect references and have now been corrected.

\clearpage
\bibliographystyle{vancouver}
\bibliography{anna_bib_new}
\end{document}