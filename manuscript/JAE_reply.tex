\documentclass[12pt]{article} 
\usepackage{amsmath} 
\usepackage[dvips]{graphicx}
\usepackage{multirow} 
\usepackage{geometry} 
\usepackage{pdflscape}
\usepackage[labelfont=bf]{caption} 
\usepackage{setspace}
\usepackage[running]{lineno} 
% \usepackage[numbers,sort]{natbib}
\usepackage[round]{natbib} 
\usepackage{array}
\usepackage[table]{xcolor}

\topmargin -1.5cm % 0.0cm 
\oddsidemargin 0.0cm % 0.2cm 
\textwidth 6.5in
\textheight 9.0in % 21cmhttps://www.overleaf.com/project/6048001a3ac22e3aff2ced4a
\footskip 1.0cm % 1.0cm

\newenvironment{refquote}{\bigskip \begin{it}}{\end{it}\medskip}

\usepackage{authblk}

\begin{document}

% Due June 18.

\section*{Reply to Editor}

    \begin{refquote}
        Decision by the Editor (Prof. Darren Evans):

        Thank you for submitting your manuscript to Journal of Animal Ecology.

        I have now received reviewers' reports and the Associate Editor's comments on your manuscript and looked at it myself. As you can see from their comment below, they liked aspects of your work but raised a number of issues around context and clarity. As a result, I cannot accept this current version of the manuscript but I am giving you the option to submit a revised version of the manuscript for consideration.

        Although we are giving you the option to complete a major revision on this manuscript, please note that acceptance and publication is not guaranteed. Your revised manuscript is due on 18-Jun-2023. If you foresee any problems with meeting this deadline, please contact the editorial office at: admin@journalofanimalecology.org

    \end{refquote}
    

    \textbf{R:} We thank the Editor for the option to revise and resubmit and for the helpful feedback from all concerned. We have taken careful note of the Associate Editor's and Reviewers' comments, particularly with regard to clarity and our statistical methods, and have extensively revised the manuscript. As a result, we are confident that our aims and results are now clearer and that that introduction and discussion more clearly place our work within the context of the literature. We hope that the Editor will agree that our work now meets the high standards of the Journal of Animal Ecology.

    [[Some nice closing?]]


\clearpage

\section*{Reply to Associate Editor}

    \begin{refquote}

        Associate Editor Comments for Authors:
        Dear Alyssa Cirtwill,

        Thank you for submitting the above manuscript to Journal of Animal Ecology. The manuscript has now been evaluated by two independent referees. Both of them agree that this study is interesting. However, they have raised some issues that should be addressed in a revision. Among these, lack of clarity in the multitude of concepts used throughout the manuscript has been highlighted by both referees. Also, certain results need more contextualisation. The writing structure, mainly in the introduction and discussion sections, are not easy to follow, and some figures are difficult to interpret (e.g. Figure 4). In sum, the manuscript would benefit from increasing clarity, context and making the study goal more focused. One of the referees also points to certain aspects of the statistical analysis that need to be clarified. These issues should be addressed in a revised manuscript.

        I hope that these comments and those of the reviewers assist the authors in preparing a revision.

        Best wishes,

        Daniel Montoya
        Associate Editor of Journal of Animal Ecology

    \end{refquote}


    \textbf{R:} We thank the Associate Editor for his feedback and appreciate the points raised by the Reviewers. We have extensively revised our manuscript with regard to clarity and believe that the current version makes our main aim and conclusions clearer. We have also redone our analyses in line with the concern raised by Reviewer 1. These changes have, to our reading, greatly improved the manuscript and we are pleased to present this revision for new consideration.


\clearpage

\section*{Reply to Reviewer \#1} [[Done?]]

    \begin{refquote}

        In this paper the authors examine how a species' participation in different three-species motifs within a food web impacts their probability of persistence. They apply a Bayesian network approach to determine whether a species goes extinct following a disturbance, focusing on disturbance to the basal trophic levels.

        Overall I felt this was a very good paper. The introduction laid out the context in a clear and concise manner, and I thought the methods were well described, both in terms of what was done as well as potential limitations of the approach. I also liked the different links that were tested between motif profiles and both global and local network properties. 

    \end{refquote}

    \textbf{R:} We thank the Reviewer for taking the time to read the manuscript and for their thoughtful feedback. We have redone our analyses with binomial error distribution as suggested (though with glm rather than glmer as the random effects proved singular). Our main results did not change much with this improved approach, though some of our supplemental results (e.g., those relating to overall network structure) were simplified. We hope that the revised manuscript is now clearer and easier to read and that our statistical approach is more reasonable. Below, we respond to each of the Reviewer's comments (preceded by \textbf{R:}) in turn.

    \subsection*{Comments}

        \begin{enumerate}

            \item One concern I had is in regards to the statistical analysis. Given the use of lmer, I assume that the error distribution was assumed to be Gaussian, though the response variable is constrained to be between 0 and 1. It may be more appropriate to use a binomial or quasi-binomial error distribution (using glmer).

            \textbf{R:} Yes, a binomial error distribution would be more appropriate. We have corrected the error and updated our results accordingly and thank the Reviewer for pointing this out. Our results were not strongly affected by this change in statistical methodology, except that the relationship between network size and average consumer persistence was no longer significant.

            \item The only other comment I had was that Fig. 4 was a bit difficult to interpret. I was wondering if it might not be better to facet across motif and connectance, with each panel showing that motif's slope distribution at different disturbance levels. I think this would make it easier to compare the distributions for each motif, which is what the text of the results seemed to focus on.

            \textbf{R:} We have reorganised the figure as suggested and, ourselves, now find it much clearer. We thank the Reviewer for the suggestion and hope that they also now find it easier to interpret. As an additional aid to understanding these results, we have added additional figures in the SI showing trends extracted from a model including proportion of motif, disturbance strength, network size, and connectance (across all networks within a size and connectance class). These can be compared with the peaks of the distributions of the within-network slopes to more easily understand how to these values relate to persistence.

        \end{enumerate}

\clearpage

\section*{Reply to Reviewer \#2}

    \begin{refquote}

        The theme of this study certainly captured my attention and interest, and the overall concepts explored in the article speaks to a broad audience of researchers. Although the topics introduced in the article are related, I found that the authors incorporated a few too many high-level ideas
        and did not unpack all of them, causing the goal of the study to be somewhat difficult to determine. For example, in the introduction, the authors begin their first paragraph by talking about disturbance in plant communities. Although disturbance is clearly relevant to consumer persistence, it would be clearer if the authors began by providing insight and definitions of what a motif is, why it is important, and how it relates to consumer persistence. With this jumping off point, the authors could mention that disturbance in plant communities influences consumers differentially depending on their structural configuration (i.e., motif). As it is in its current form, it feels like the authors are jumping from one idea to the next without clarifying the details of any one of them.

        I feel that this study includes a lot of work and analysis that is meaningful. For example, at line 195-196, the authors state “This suggests that network motif profiles may be more useful than global network properties when predicting mean extinction risk of species in a network.”. To me, this sentence highlights how their results fit into an up-and-coming area of ecological research: monitoring the likelihood of consumer persistence in efficient ways. Nonetheless, I found the structure of the writing in the introduction and discussion difficult to follow and some of the key points from the results were not easy to interpret because the authors have so many different conceptual ideas incorporated throughout the article.

    \end{refquote}


    \textbf{R:} We thank the Reviewer for taking the time to read our manuscript and for the thoughtful feedback. We particularly appreciate the pointers where the structure of our manuscript was unclear an where it was difficult to extract our main aims and results. In the present revision we have substantially reworked the introduction and discussion and reordered the methods and results in order to present our work in a more streamlined and, hopefully, understandable manner. We believe that the manuscript has been greatly improved as a result. Note that, in response to a comment from the other Reviewer, we have also redone our analyses and some results have slightly changed from the previous submission. Below, we respond to each of the Reviewer's comments (preceded by \textbf{R:}) in turn.


    \subsection*{Specific comments}

        \begin{enumerate}

            \item Summary: I found that the ideas explored in the paper were somewhat unclear from the summary. Is the overall goal of the study to understand participation in a motif (what I understood from the abstract and title) or is it to understand how the presence of particular sub-webs (i.e., motifs) influences persistence? It seems like it is both from the results presented, but the authors highlight the former extensively and leave the latter as more of an after-thought. This may be addressed by simply taking out words like “additionally” from the start of the last few summary points.

                \textbf{R:} We have thoroughly revised the summary to make it clearer that our focus is on species participation in motifs rather than the presence of motifs within a network. We hope that it now more-closely parallels the rest of the manuscript and is easier to follow.

    
            \item Introduction Line 13-16: The wording here makes it seem like species loss is both the stressor/disturbance and the response variable. This would be clearer if the authors clarify whether they are studying the effects of disturbance at the primary producer level on consumer persistence right away. 

                \textbf{R:} The sentence the reviewer refers to is ``How disturbances such as population decline or extinction lead to the loss of species via  direct and indirect effects (i.e., secondary extinctions) has long been a vibrant area of research (Curtsdotter et al., 2011a; Dunne & Williams, 2009; Eklöf \& Ebenman, 2006 Santos et al., 2021).'' We can understand the reviewer´s confusion, but in fact species extinctions are both the disturbance and the response.
                An initial extinction of a species (i.e. the primary extinction) can, since species are connected to each other via  consumer--resource interactions, cause other species to go extinct (i.e. secondary extinctions). We have now clarified this on line XX: 

                \begin{quotation}
                The loss of a species in a food web can, due to the mutual dependencies among species, trigger a cascade of additional extinctions, so called secondary extinctions. A species' risk of going secondarily extinct is known to depend on its position within a food web~\citep{Santos2021,curtsdotter2011robustness, dunne2009cascading, Eklof2006}.
                \end{quotation}
            
            \item It would also be helpful to define exactly what definition of disturbance they are referring to in the introduction.
            
             \textbf{R:}
             We have now clarified how we in this manuscript define a disturbance. This is done on line XX. 
             
             \begin{quotation}
             Here, we focus on persistence after the loss of plants or other basal resources, i.e., the effects of bottom-up disturbances.
            \end{quotation}

            \item Line 42-44: I am uncertain of what this sentence is trying to convey. In the sentence prior to this one, the authors suggest that trophic level and in-degree are too coarse of descriptors. Then, instead of providing insight into why or how more detailed metrics would be desirable, they mention that the consumer's risk of extinction is dependent on its prey. It is unclear how prey-dependent extinction of consumers is related to the coarseness of the metrics used to analyze food webs. This sentence might be highlighting a key point in the article, but as it is currently written it is very difficult to understand what this sentence intends to describe.

                \textbf{R:} We agree with the reviewer that this was not so clearly expressed. We have now re-formulated this section and clarified with an example, beginning at line XX. 
                
                \begin{quotation}
                However, unlike in-degree and trophic level, which focus only on direct prey of the focal species or on the lengths of paths between the focal species and basal resources, motif participation also incorporates information on the focal species' interaction partners. Two species might both have three prey and a trophic level of 3, but they might have quite different motif participation if one species's prey are strict specialists and the others' prey are generalists. Importantly, since specialists are generally more vulnerable to extinction than generalists, the predator of specialists will be more likely to lose prey and go extinct itself than a predator of generalists.
                \end{quotation}

            \item Line 48: This paragraph contains important information to understanding the concept of motifs; however, I found that the description of motif participation is vague and difficult to interpret in the context of this study. For example, the “second-step indirect interactions the species are affected by, thereby capturing important structures that are missed by global and local properties” does not provide any details that would help the reader understand why motif participation is a relevant metric. Specifically, what important structures are the authors referring to here, and how are they captured by the second-step indirect interactions? Further, is there a better way to describe what a second-step indirect interaction is? I am unsure what the authors mean by this.

                \textbf{R:} To avoid introducing extra terminology, we have replaced all instances of `global' and `local' with `large-scale' and `small-scale' respectively. We hope it is now easier to understand motifs as a meso-scale descriptor of network structure. See line XX, quoted below, for a short description. We have also, as described above, explained how motif participation captures information about a focal species' interaction partners, and thereby the potential for indirect interactions. We do not refer to second-step indirect interactions in order to avoid confusion with that term.


                \begin{quotation}
                Therefore, a species' motif participation profile captures the indirect interactions the species are affected by, thereby capturing important structures that are missed by global (network level) and local (species level) properties.  As such, motif participation may complement degree and trophic level to more fully predict species' extinction risk.
                \end{quotation}
            
            \item Line 87-88: Local and global food web properties has been mentioned more than once but it is unclear what these properties are specifically. Perhaps defining what this is referring to earlier on in the introduction would make it more clear to the reader.

                \textbf{R:} See comment above. 

            
            \item Line 92-95: The authors bring up many conceptual topics in their introduction, but perhaps the most relevant topic: network motif structures, has been almost entirely overlooked in terms of providing background information to the reader about which motifs are most common, how they influence energy transfer, and what encourages the establishment of the different motif structures across food webs. The authors mention that they will be investigating food chain, omnivory, apparent and direct competition motifs only briefly at the end of the introduction. From my perspective, the authors are missing key details about what is already known about how these different structural configurations of sub-webs can influence stability and consumer persistence. For example, networks with a high degree of omnivory may be more stable than simple food chains, but this depends on the strength of the omnivorous interactions among trophic guilds. [[Did they take this from our discussion, the point they found contradictory?]] It would be helpful to provide more insight into the gaps in knowledge surrounding the prevalence of these motifs in food webs more generally, and why this would help food web researchers understand consumer persistence and how it is related to resilience/stability since this is where I see the greatest amount of novelty from this article.

                \textbf{R:}
                TO DO!


            \item Methods: Line 98: The authors have not specified what species/trophic guilds are being considered as secondary extinctions in this study. Is a primary extinction at the primary producer level? Is a secondary extinction referring to an extinction of a species that relies on a species that has faced disturbance-induced extinction? This should be clarified.

                \textbf{R:} A primary extinction is the extinction of a species that is directly disturbed (or removed, in the case of simulated species removal) while a secondary extinction is the extinction of any species not directly disturbed (as a result of the loss of its resources to primary or earlier secondary extinctions, as the Reviewer says). As we disturb primary producers and do not simulate top-down effects, only consumers can go secondarily extinction in our framework. We now define secondary extinction parenthetically in the introduction and at the start of the methods (quoted below) and hope that the meaning of secondary extinction is now clear.

                \begin{quotation} 
                Traditionally, there are two main approaches for studying secondary extinctions (i.e., extinctions of species that were not directly disturbed). 
                \end{quotation}

            \item Line 104-107: I am not convinced that this is always the case. Many dynamical models are simplified and allow researchers to use key parameters to model food webs and understand how major flows of energy transfer influence consumer persistence.

                \textbf{R:} If the Reviewer is thinking of EcoPath with EcoSim, that type of model only works with quite a small food web with very well-understood interaction strengths and initial population sizes. We are not aware of any dynamical model that is fast over a large food web - certainly this was not the case when some of the authors conducted dynamical simulations of webs with 50-100 species. [[Anyone have a better answer? From the focus on major energy flows I think it's EcoSim they're thinking of...]]


            \item Line 108: I am not well-versed in Bayesian statistics and will therefore refrain from evaluating the methodology used here. I am hopeful that the other reviewers will be able to provide feedback on this section of the article.

                \textbf{R:} Please note that we are using a Bayesian network modelling framework and not Bayesian statistics. All statistical models are quite standard linear models or general linear models. The Bayesian network method has been used and validated in previous papers (see references in the main text) and has not been substantially modified for this paper. The novelty in our work lies in the relationships between motifs and persistence, not in the network modelling \emph{per se}.


            \item Line 185: The authors should provide information about how they have determined what networks would be more biologically likely than others.

                \textbf{R:} The niche model has been previously demonstrated to generate plausible network architectures, including similar motif profiles (see citations in main text). We chose size and connectance in order to meet computational limitations (very large networks are difficult to model) and to overlap with the ranges of size and connectance in empirical food webs used to develop much of the current network theory. 


                We have lightly rephrased the start of this paragraph (quoted below) and added some references to papers describing empirical networks with similar size and connectance to those we simulate.
                Also note that we compare our simulated results with results for a set of more recent, larger, and especially well-resolved empirical networks in the SI. Although some of the whole-network structural properties differed, the range of species motif participations in our simulated networks spanned the whole range of observed motif participation in these empirical networks.


                \begin{quotation}
                We generated simulated networks using the niche model, which has been shown to recreate the structure of empirical networks well~\citep{Williams2000,Stouffer2007}.
                To capture a range of network architectures similar to those in well-studied empirical networks~\citep{Dunne2002,Dunne2002a}, we simulated networks with sizes (S) ranging from 50 to 100 species (in steps of 10) and connectances (C) ranging from 0.02 to 0.18 (in steps of 0.04). 
                All networks were generated using the function ``nichemodel'' within the Julia~\citep{Bezanson2017julia} package \emph{BioEnergeticFoodWebs}~\citep{bioenergfw,Delmas2017}.   
                For full details and a comparison to empirical networks, see \emph{Appendix S2}.
                \end{quotation}


            \item Line 192: It is not obvious how the authors have decided to define a focal species in their study. It should be made clear what trophic guild or species they are specifically looking at in each of these networks as the wording now is quite vague.

                \textbf{R:} A focal species is simply that for which the motif role is currently being calculated ... all consumers are taken as the focal species in turn. Note that as these networks are simulated the species (nodes) in them do not correspond to any particular focal species or trophic guild. The vagueness is intentional as there are no characteristics of, say, invertebrates vs. vertebrates or mammals vs. reptiles incorporated in the networks. 

                We have revised this line (quoted below) and hope it is now clearer that a species' motif participation role refers to that same species' appearance in different motifs, and that this applies to any species one selects.
                \begin{quotation}
                A species' ``motif participation role'' is the frequency with which that species appears in each of the motifs present in a network~\citep{Stouffer2012}.
                \end{quotation}


            \item Line 194-195: It is unclear whether this is a general phenomenon or specific to this study. I am also uncertain of whether the vectors that the authors are referring to related to the number of species in the motifs or the number of motifs in a network.

                \textbf{R:} The size of a motif participation vector is always related to the number of motifs considered, not the number of species (there is one vector per species). We only considered three-species motifs in this study, as stated in lines XX, XX, and XX. Since there are only four \emph{unique} three-species motifs in these acyclic networks (a general feature of all acyclic networks), motif participation vectors are four-dimensional for all species. This will be true for any motif participation roles calculated using three-species motifs and acyclic networks, unless one included placeholders of 0 for the 9 three-species motifs containing cycles (see~\citep{Stouffer2007} for a full list of these motifs). In that case all motif participation vectors would be 13-dimensional, but still only 4 dimensions would be informative.


                We have rephrased this line (quoted below) to remove the phrase `in our case' and make it clearer that four-dimensional motif participation vectors are a general feature of acyclic networks.
                \begin{quotation}
                    As only four unique three-species motifs can appear in acyclic networks, these are four-dimensional vectors. 
                \end{quotation}

                Note that each motif can appear within the network many times with a different set of three interacting species (e.g., a network might contain the three-species-chains `hawk-mouse-oat',  `hawk-mouse-grasshopper', and `mouse-grasshopper-oat') so that the \emph{total occurrences} of all three-species motifs in a network will be much greater than four.


            \item Results: Overall, I found the results difficult to interpret, mainly because of the number of results presented. I had issue understanding what figure 2 was showing, since the x-axis ``proportion of role'' was not alluded to described as an explanatory variable anywhere in the text. I wasn't able to determine how the authors expected the probability of persistence to change across this gradient for each of the motifs, and how the results may have deviated from that expectation. This problem was mirrored in most of the figures presented.

                \textbf{R:} Proportion of role is the proportion of the species' motif role made up by the particular motif in each panel. This misunderstanding seems linked to a general difficulty in understanding precisely how we define motif roles. To help remedy this, we have now added motif role vectors to Figure 1 to make it explicit how counts of motifs are translated into vectors of normalised motif frequencies. These frequencies are equivalent to the proportion of a species' total motif participation made up by each motif. We also now use the perminology ``proportion of motif in role'' and ``proportion of motif in network profile'' in the axis labels of Figs. 2, 3, and 5 for consistency. We have also thoroughly revised the results section to emphasize our main findings and place supplemental results in the proper context.


            \item Network mean persistence and motif profiles: The ways in which persistence is related to motif profiles is interesting and valuable to explore; however, I am concerned with some of the interpretation of these results. Specifically, higher proportions of omnivory may lead to reduced average persistence if the average interaction strengths among organisms are strong; however, many food webs are composed of weakly interacting species where omnivory increases the likelihood of community-level persistence. Similarly, apparent competition is known to be destabilizing to food webs when energy is drastically diverted away from one species. The results presented here counter those examples, so it is important that the authors provide well-founded context for these findings.

                \textbf{R:} Previous research has shown that omnivory can either be stabilising or destabilising, depending on the precise disturbance, strengths of interactions, etc. Our results, which deal specifically with persistence after bottom-up disturbance, add to that picture of complexity without contradicting earlier findings. Note that we do not say that omnivory is \emph{never} beneficial - only that in our networks, in response to our focal disturbance, is is generally detrimental. We also note that our results show apparent competition as associated with less persistence when disturbance is weak, which is compatible with the possibility that the Reviewer mentions above (though we do not specifically model prey switching, which seems to be what the Reviewer refers to).  In the revision we have expanded our discussion on how the strength and direction of motif participation-persistence relationships vary network-to-network (especially in networks with different connectance). We hope that this addition provides some of the nuance the Reviewer missed from the previous revision.


            \item Discussion: I found the discussion section relatively well-written, and the authors pointed out how their analyses may be relevant in understanding how motif participation may be an important metric to consider when aiming to conserve species and prevent extinction.

                \textbf{R:} We thank the Reviewer for their compliments.


            \item However, I am concerned that the authors do not provide any analysis for top-down effects on persistence. Although they are forthcoming about this (line 428-430), the interpretation of their analysis may be less relevant in the context of managing real food webs since predation does of course influence persistence.

                \textbf{R:} Using the Bayesian network framework, we simply cannot include top-down effects on persistence. However, in a parallel study (started around the same time but by chance published earlier) some of the authors of the present study did simulate both top-down and bottom-up effects in a dynamical model. The dynamical model has its own caveats (in general, dynamical models are sensative to parameter choice and, due to the time and computational resources required, a full sensitivity analysis could not be performed). However, [[something comparing the results between papers - somewhat similar?]]


                It may also reassure the Reviewer to know that in the initial studies introducing the Bayesian network framework~\citep{}, Bayesian networks captured XX\% of the extinctions identified by dynamical models run on the same set of networks. As we use the same modelling framework as in those studies, we can reasonably expect similar results.


                Finally, consider the situation from an ecological perspective. We simulate disturbance to primary producers only (and appreciate that this may not have been clear in the previous revision). While predation certainly does influence persistence, the disturbance we introduce will first act to remove food sources from some herbivores. Some fraction of these herbivores will then go extinct, reducing resources for their predators. Some fraction of these predators will also go extinct due to the reduction in their food. It is only those predators which are affected by declining resources, do not go extinct, and are able to increase predation on their remaining prey which can cause top-down extinctions. While this obviously does occur in nature, logically this is a smaller number of extinctions than the set of herbivore and predator extinctions caused by bottom-up effects. This is also reflected in the similar results from Bayesian networks and dynammical models in the studies cited above - if top-down effects were responsible for the majority of secondary extinctions we would see quite different performance.


                None of this is to discount the reality of top-down extinctions or their interest from a scientific standpoint. However, since the Bayesian network framework does not include top-down effects any comment we might make about how top-down effects might affect our results would necessarily be pure speculation. Instead, we have added a reference to another paper exploring the relationship between motif roles and persistence in a dynamical context. That paper, due to the computational limits inherent in the dynamical approach, uses only a single-species disturbance. It does, however, include top-down as well as bottom-up effects and may provide the information the Reviewer is seeking.
 


            \item I feel that the discussion would be greatly improved if the authors focus in on one idea rather than three at once (i.e., disturbance strength; mean network persistence; motif participation). I am left trying to discern how motif participation affects persistence and have not been able to fully understand this, and I initially was under the impression that this was the ultimate goal of the study.

                \textbf{R:} We have extensively revised the discussion to focus more clearly on the relationships between motif participation and persistence, although we note that some of these relationships depend on the strength of disturbance. As such, disturbance strength remains a strong secondary theme in the present version. We hope that this revision is easier to follow.


        \end{enumerate}

\clearpage

    \bibliographystyle{ecollett} 
    \bibliography{manual} % Do not abbreviate journal titles, papers with >

\end{document}